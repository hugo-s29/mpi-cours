\usepackage[utf8]{inputenc}
\newcommand\hmmax{0}
\newcommand\bmmax{0}
\usepackage[OT2,OT1,T1]{fontenc}
\usepackage{textcomp}
\usepackage[bookmarks]{hyperref}
\usepackage[french]{babel}
\usepackage[fontsize=8pt]{fontsize}
\usepackage{amssymb}
\usepackage{amsmath}
\usepackage{amsthm}
\usepackage{tikz}
\usepackage{mathtools}
\usepackage{tkz-tab}
\usepackage[inline]{asymptote}
\usepackage{frcursive}
\usepackage{verbatim}
\usepackage{moresize}
\usepackage{pifont}
\usepackage{xfrac}
\usepackage{thmtools}
\usepackage{diagbox}
\usepackage{centernot}
\usepackage{multicol}
\usepackage{nicematrix}
\usepackage{accents}
\usepackage{stmaryrd}
\usepackage{setspace}
\usepackage{cancel}
\usepackage[f]{esvect}
\usepackage{wrapfig}
\usepackage{floatflt}
\usepackage{dsfont}
\usepackage{subcaption}
\usepackage{pdflscape}
\usepackage{soulutf8}
\usepackage[Rejne]{fncychap}
\usepackage{listings}
\usepackage[framemethod=TikZ]{mdframed}
\usepackage{graphicx}
\usepackage{qtree}
\usepackage{adjustbox}
\usepackage[thinlines]{easytable}
\usepackage{hep-math-font}
\usepackage{algorithm}
\usepackage[rightComments=false,beginComment=$\qquad\triangleright$~]{algpseudocodex}
\usepackage{yhmath}
\usepackage{calligra}
\usepackage{enumitem}
\usepackage{tgschola}
\usepackage{dutchcal}
\usepackage{BOONDOX-calo}

\frenchspacing
\reversemarginpar
\newif\ifsimple

% better underline
\setuldepth{a}

\usetikzlibrary{babel}
\theoremstyle{definition}

\definecolor{green}{HTML}{60A917}
\def\asydir{asy}
\newcommand{\cwd}{.}

% figure support
\usepackage{import}
\usepackage{xifthen}
%\pdfminorversion=7
%\usepackage{pdfpages}
%\usepackage{transparent}
%\newcommand{\incfig}[1]{%
%	\def\svgwidth{\columnwidth}
%	\import{\cwd/figures/}{#1.pdf_tex}
%}

\usepackage{float}
\usepackage{tikz-cd}
\usepackage{thmtools}
\usepackage{thm-restate}
\usepackage{etoolbox}
\usepackage{footnote}

\let\oldemptyset\emptyset
\let\emptyset\varnothing

\let\ge\geqslant
\let\le\leqslant
\let\preceq\preccurlyeq
\let\succeq\succcurlyeq

\renewcommand{\C}{\mathds{C}}
\newcommand{\R}{\mathds{R}}
\newcommand{\Z}{\mathds{Z}}
\newcommand{\N}{\mathds{N}}
\newcommand{\Q}{\mathds{Q}}
\renewcommand{\O}{\emptyset}

\renewcommand\Re{\mathop{\expandafter\mathfrak{Re}}}
\renewcommand\Im{\mathop{\expandafter\mathfrak{Im}}}

\renewcommand{\thepart}{\Roman{part}}
\newcommand{\centered}[1]{\begin{center}#1\end{center}}

\let\th\relax
\let\det\relax
\DeclareMathOperator{\Arccos}{Arccos}
\DeclareMathOperator{\Arcsin}{Arcsin}
\DeclareMathOperator{\Arctan}{Arctan}
\DeclareMathOperator{\Argsh}{Argsh}
\DeclareMathOperator{\ch}{ch}
\DeclareMathOperator{\sh}{sh}
\DeclareMathOperator{\th}{th}
\DeclareMathOperator{\Card}{Card}
\DeclareMathOperator{\com}{com}
\DeclareMathOperator{\Ker}{Ker}
\DeclareMathOperator{\Aut}{Aut}
\DeclareMathOperator{\id}{id}
\DeclareMathOperator{\rg}{rg}
\DeclareMathOperator{\PPCM}{PPCM}
\DeclareMathOperator{\PGCD}{PGCD}
\DeclareMathOperator{\argmax}{argmax}
\DeclareMathOperator{\argmin}{argmin}
\DeclareMathOperator{\Vect}{Vect}
\DeclareMathOperator{\cotan}{cotan}
\DeclareMathOperator{\Mat}{Mat}
\DeclareMathOperator{\det}{det}
\DeclareMathOperator{\tr}{tr}
\DeclareMathOperator{\Cov}{Cov}
\DeclareMathOperator{\Supp}{Supp}
\DeclareMathOperator{\Cl}{\mathcal{C}\!\ell}
\DeclareMathOperator*{\po}{\text{\cursive o}}
\DeclareMathOperator*{\gO}{\mathcal{O}}
\DeclareMathOperator*{\dom}{dom}
\DeclareMathOperator*{\sgn}{sgn}
\DeclareMathOperator*{\codim}{codim}
\DeclareMathOperator*{\simi}{\sim}
\DeclareMathOperator{\Reg}{Reg}
\DeclareMathOperator{\Rec}{Rec}
\DeclareMathOperator{\lit}{lit}

%\pdfsuppresswarningpagegroup=1

\newcommand{\emptyenv}[2][{}] {
	\newenvironment{#2}[1][{}] {
		\vspace{-16pt}
		#1
		\vspace{16pt}
		\expandafter\noindent\comment
	}{
		\expandafter\noindent\endcomment
	}
}

\mdfsetup{skipabove=1em,skipbelow=1em, innertopmargin=1.1em, innerbottommargin=6pt,}

\newmdenv[frametitlerule=true,roundcorner=5pt,subtitlebelowline=true,subtitleaboveline=true]{recap-box}

\declaretheoremstyle[
	mdframed={ },
	headpunct={:},
	numbered=no,
	headfont=\normalfont\bfseries,
	bodyfont=\normalfont,
	postheadspace=1em]{defnstyle}

\declaretheoremstyle[
	mdframed={
		rightline=false, topline=false, bottomline=false,
		innerlinewidth=0.4pt,outerlinewidth=0.4pt,
		middlelinewidth=2pt,
		linecolor=black,middlelinecolor=white,
	},
	headpunct={:},
	numbered=no,
	headfont=\normalfont\bfseries,
	bodyfont=\normalfont,
	postheadspace=1em]{thmstyle}

\declaretheoremstyle[
	headpunct={:},
	postheadspace=\newline,
	numbered=no,
	headfont=\normalfont\scshape]{rmkstyle}

\declaretheoremstyle[
	headfont=\normalfont\itshape,
	numbered=no,
	postheadspace=\newline,
	mdframed={ rightline=false, topline=false, bottomline=false },
	headpunct={:},
	qed=\qedsymbol]{prvstyle}

\declaretheorem[style=defnstyle, name=Définition]{defn}
\declaretheorem[style=defnstyle, name=Algorithme]{algo}
\declaretheorem[style=defnstyle, name=Propriété -- Définition]{prop-defn}

\declaretheorem[style=thmstyle, name=Théorème]{thm}
\declaretheorem[style=thmstyle, name=Axiome]{axm}
\declaretheorem[style=thmstyle, name=Propriété]{prop}
\declaretheorem[style=thmstyle, name=Corollaire]{crlr}
\declaretheorem[style=thmstyle, name=Lemme]{lem}

\declaretheorem[style=rmkstyle, name=Remarque]{rmk}
\declaretheorem[style=rmkstyle, name=Rappel]{rap}


\AtBeginDocument{
	\ifsimple
		\emptyenv{exm}
		\emptyenv{cexm}
		\emptyenv{exo}
		\emptyenv[\hfill$\blacksquare$]{prv}
	\else
		\declaretheorem[style=rmkstyle, name=Exemple]{exm}
		\declaretheorem[style=rmkstyle, name=Contre-exemple]{cexm}
		\declaretheorem[style=rmkstyle, name=Exercice]{exo}
		\declaretheorem[style=prvstyle, name=Preuve]{prv}
	\fi
}

\makeatother
\usepackage{fancyhdr}
\pagestyle{fancy}

\fancyhead[R]{}
\fancyhead[L]{\thepart}
\fancyhead[C]{\parttitle}

\fancyfoot[C]{\thepage}
\fancyfoot[L]{}
\fancyfoot[R]{}

\newcommand*\parttitle{}
\let\origpart\part
\renewcommand*{\part}[2][]{%
   \ifx\\#1\\% optional argument not present?
      \origpart{#2}%
      \renewcommand*\parttitle{#2}%
   \else
      \origpart[#1]{#2}%
      \renewcommand*\parttitle{#1}%
   \fi
}

\makeatletter

\newcommand{\tendsto}[1]{\xrightarrow[#1]{}}
\newcommand{\danger}{{\large\fontencoding{U}\fontfamily{futs}\selectfont\char 66\relax}\;}
\newcommand{\ex}{\fbox{ex}\;}
\renewcommand{\mod}[1]{~\left[ #1 \right]}
\newcommand{\todo}[1]{{\color{blue} À faire : #1}}
\newcommand{\vrt}[1]{\rotatebox{-90}{$#1$}}
\newcommand{\avrt}[1]{\rotatebox{90}{$#1$}}
\newcommand{\tsup}[1]{\textsuperscript{\underline{#1}}}
\newcommand{\revs}[1]{\rotatebox{180}{$#1$}}

\DeclareMathOperator{\ou}{\text{ ou }}
\DeclareMathOperator{\et}{\text{ et }}
\DeclareMathOperator{\si}{\text{ si }}
\DeclareMathOperator{\non}{\text{ non }}

\newcommand{\iffdef}{\mathop{\iff}^{\text{def.}}}

\renewcommand{\title}[2]{
	\AtBeginDocument{
		\begin{titlepage}
			\begin{center}
				\vspace{10cm}
				{\Large \sc Chapitre #1}\\
				\vspace{1cm}
				{\HUGE \calligra #2}\\
				\vfill
				Hugo {\sc Salou} MPI${}^{\star}$\\
				{\ssmall Dernière mise à jour le \@date }
			\end{center}
		\end{titlepage}
	}
}

\newcommand{\titletd}[2]{
	\AtBeginDocument{
		\begin{titlepage}
			\begin{center}
				\vspace{10cm}
				{\Large \sc Td \rm n${}^{\mathrm{o}}$ #1}\\
				\vspace{1cm}
				{\HUGE \calligra #2}\\
				\vfill
				Hugo {\sc Salou} MPI${}^{\star}$\\
				{\ssmall Dernière mise à jour le \@date }
			\end{center}
		\end{titlepage}
	}
}

\let\bx\boxed
\newcommand{\s}{\text{\cursive s}}
\renewcommand{\t}{{}^t\!}
\newcommand{\T}{{\!\!\,\top}}
\newcommand{\eme}{\ensuremath{{}^{\text{ème}}}~}
%\let\oldfract\fract
%\renewcommand{\fract}[2]{\oldfract{\displaystyle #1}{\displaystyle #2}}
% \let\textstyle\displaystyle
% \let\scriptstyle\displaystyle
% \let\scriptscriptstyle\displaystyle
%\everymath{\displaystyle}


\makeatletter
\def\moverlay{\mathpalette\mov@rlay}
\def\mov@rlay#1#2{\leavevmode\vtop{%
   \baselineskip\z@skip \lineskiplimit-\maxdimen
   \ialign{\hfil$\m@th#1##$\hfil\cr#2\crcr}}}
\newcommand{\charfusion}[3][\mathord]{
    #1{\ifx#1\mathop\vphantom{#2}\fi
        \mathpalette\mov@rlay{#2\cr#3}
      }
    \ifx#1\mathop\expandafter\displaylimits\fi}
\makeatother

\newcommand{\cupdot}{\charfusion[\mathbin]{\cup}{\cdot}}
%\newcommand{\bigcupdot}{\charfusion[\mathop]{\bigcup}{\cdot}}
\newcommand{\plusbar}{\charfusion[\mathbin]{+}{\color{blue}/}}
\let\olddiamond\diamond
\renewcommand{\diamond}{\ensuremath{\,\olddiamond\,}}
\newcommand{\missingpart}{\centered{\color{red} \Large Il manque une partie du cours ici}}

\newcommand{\cbox}[2]{{\color{#1}{\fbox{\color{black} #2}}}}
\newcommand{\clrhl}[2]{{\large\sc\color{#1} #2}}
\newcommand{\mat}[1]{\begin{pmatrix} #1 \end{pmatrix}}
\newcommand{\mathunderline}[1]{\text{\underline{$#1$}}}
\let\oldboxtimes\boxtimes
\renewcommand{\boxtimes}{\raisebox{-0.2mm}{$\;\oldboxtimes\;$}}
\newcommand{\addrecap}{\ifsimple\else\startrecap\begin{multicols}{2}
	\vfill
	\recapsep{Ordre}
	\vfill
	\begin{recap-box}[frametitle={Ordre bien fondé}]
		Un ordre est \textit{bien fondé} s'il n'existe pas de suite infiniment strictement décroissante, \textit{i.e.}\ toute partie non vide de $E$\/ admet un élément minimal.
	\end{recap-box}
	\vfill
	\begin{recap-box}[frametitle={Ordre produit}]
		On définit l'\textit{ordre produit} $\preceq_\times$\/ de $(A, \preceq_A)$\/ et $(B, \preceq_B)$\/ comme \[
			(a, b) \preceq_\times (a', b')\!\!\iffdef\!\!a \preceq_A a' \text{ et } b \preceq_B b'
		.\] 
		Cet ordre préserve le caractère bien fondé de $\preceq_A$\/ et $\preceq_B$, mais pas le caractère total de la relation.
		On étend cet ordre à l'ensemble $(A^n, \preceq_\times)$, en itérant l'ordre produit.
	\end{recap-box}
	\vfill
	\begin{recap-box}[frametitle={Ordre lexicographique}]
		On définit l'\textit{ordre lexicographique} $\preceq_\ell$\/ de $(A, \preceq_A)$\/ et $(B, \preceq_B)$\/ comme \[
			(a, b) \preceq_\ell (a', b') \iffdef
			\begin{array}{|l}
				\!\!(a \prec_A a')\\
				\!\!\text{ou }\\
				\!\!(a = a' \text{ et } b \preceq_B b').
			\end{array}
		\]
		Cet ordre conserve le caractère bien fondé de $\preceq_A$\/ et $\preceq_B$, ainsi que le caractère total. On peut également étendre cet ordre à un ensemble $(A^n, \preceq_\ell)$.
		On étend également cet ordre à l'ensemble $(A^*, \preceq_\ell)$\/ comme \[
			\underset{\mathclap{\substack{\vrt\in\\A^n}}} u
			\prec_\ell
			\underset{\mathclap{\substack{\vrt\in\\A^{n+m}}}} v
			\iffdef 
			\begin{array}{|l}
				\!\! \exists i \in \llbracket 1, n + 1 \rrbracket,\\
				\!\! (\forall j < i,\: u_j = u_i) \text{ et}\\
				\!\! (i = n + 1 \text{ ou } u_i \prec_A v_i).
			\end{array}
		\]
	\end{recap-box}
	\vfill
	\begin{recap-box}
		Le théorème de l'induction bien fondée nous autorise à faire une récurrence forte sur un ensemble ayant un ordre bien fondé.
	\end{recap-box}
	\vfill
	\recapsep{Induction nommée}
	\vfill
	\begin{recap-box}
		Une \textit{règle de construction nommée} est de la forme \[
			\mathbf{S}\Big|_C^r \quad\text{ ou }\quad \mathbf{S}(y, \underbrace{\square, \ldots, \square}_r)
		\] où $\mathbf{S}$\/ est un symbole, $r \in \N$\/ est l'arité de $\mathbf{S}$, et $C$\/ est un ensemble non vide.
		On omettra parfois l'ensemble $C$\/ s'il est trivial (\textit{i.e.} de taille 1).
		Une \textit{règle de base} est de la forme $\mathbf{S}\big|^0_C$\/ ; une \textit{règle d'induction} est de la forme $\mathbf{S}\big|^n_C$.
	\end{recap-box}
	\vfill
	\begin{recap-box}
		On définit la \textit{hauteur} $h$\/ comme \begin{align*}
			h: \bigcup_{n \in \N} X_n &\longrightarrow \N \\
			x &\longmapsto \min \{n \in \N  \mid  x \in X_n\}.
		\end{align*}
		On définit la relation $\preceq$\/ bien fondée telle que $x \preceq y$\/ si $x$\/ est défini à partir de $y$. Si $x \preceq y$, alors $h(x) \le h(y)$.
	\end{recap-box}
	\vfill
\end{multicols}
\begin{recap-box}
	Un ensemble $\bigcup_{n \in \N} X_n$\/ est dit \textit{défini par induction nommée} à partir des règles $R$, si $X_0 = \{\mathbf{S}(a) \mid  \mathbf{S}^0_C \in R,\:a \in C\}$, et \[
		X_{n+1} = X_n \cup \{\mathbf{S}(a, t_1, \ldots, t_r) \mid \mathbf{S}^r_C \in R,\:a \in C, (t_1, \ldots, t_r) \in (X_n)^r\} 
	.\]
\end{recap-box}
\fi}
\newcommand{\indep}{\perp\!\!\!\!\!\:\perp}
\newcommand{\startrecap}{
	\part{Bilan}
	\setlength{\abovedisplayskip}{0pt}
	\setlength{\belowdisplayskip}{0pt}
	\setlength{\abovedisplayshortskip}{0pt}
	\setlength{\belowdisplayshortskip}{0pt}
}


\definecolor{white}{HTML}{faf4ed}
\definecolor{black}{HTML}{575279}
\definecolor{mauve}{HTML}{907aa9}
\definecolor{blue}{HTML}{286983}
\definecolor{red}{HTML}{d7827e}
\definecolor{yellow}{HTML}{ea9d34}
\definecolor{gray}{HTML}{9893a5}
\definecolor{grey}{HTML}{9893a5}

\setlength{\parskip}{7pt}

\newcommand{\red}[1]{{\color{red}#1}}
\newcommand{\green}[1]{{\color{green}#1}}
\newcommand{\blue}[1]{{\color{cyan}#1}}

\pagecolor{white}
\color{black}

% code syntax highlighting
\lstset{ %
	basicstyle=\ttfamily\footnotesize,
	backgroundcolor=\color{white},    % choose the background color
	commentstyle=\color{blue},        % comment style
	keywordstyle=\bfseries\color{mauve},       % keyword style
	stringstyle=\color{pink},         % string literal style
	captionpos=b,                    % sets the caption-position to bottom
	escapeinside={\%*}{*)},          % if you want to add LaTeX within your code
	breaklines=true,                 % automatic line breaking only at whitespace
	numbers=left,                    
	numbersep=5pt,
	tabsize=2,
	xleftmargin=.2\textwidth, xrightmargin=.2\textwidth
}

\renewcommand{\lstlistingname}{{\sc Code}}
\renewcommand{\lstlistlistingname}{List of \lstlistingname s}
\newcommand\ubar[1]{\underaccent{\bar}{#1}}

\begin{asydef}
	settings.prc = false;
	settings.render=0;

	white = rgb("faf4ed");
	black = rgb("575279");
	blue = rgb("286983");
	red = rgb("d7827e");
	yellow = rgb("f6c177");
	orange = rgb("ea9d34");
	gray = rgb("9893a5");
	grey = rgb("9893a5");
	deepcyan = rgb("56949f");
	pink = rgb("b4637a");
	magenta = rgb("eb6f92");
	green = rgb("a0d971");
	purple = rgb("907aa9");

	defaultpen(black + fontsize(8pt));

	import three;
	currentlight = nolight;
\end{asydef}

%\let\sc\scshape
%\let\bf\bfseries
%\let\it\itshape

\makeatletter
\g@addto@macro\bfseries{\boldmath}
\makeatother
\let\ds\displaystyle
\let\ts\textstyle
\def\bfm#1{\mathbf{#1}}

\newcommand\ZdZ{\ensuremath{\sfrac\Z{2\Z}}}

\def\algorithmicrequire{\textbf{Entrée}}
\def\algorithmicensure{\textbf{Sortie}}
\def\algorithmicend{\textbf{fin}}
\def\algorithmicif{\textbf{si}}
\def\algorithmicthen{\textbf{alors}}
\def\algorithmicelse{\textbf{sinon}}
\def\algorithmicelsif{\algorithmicelse\ \algorithmicif}
\def\algorithmicendif{\algorithmicend\ \algorithmicif}
\def\algorithmicfor{\textbf{pour}}
\def\algorithmicforall{\textbf{pour tout}}
\def\algorithmicdo{\textbf{faire}}
\def\algorithmicendfor{\algorithmicend\ \algorithmicfor}
\def\algorithmicwhile{\textbf{tant que}}
\def\algorithmicendwhile{\algorithmicend\ \algorithmicwhile}
\def\algorithmicloop{\textbf{répéter indéfiniment}}
\def\algorithmicendloop{\algorithmicend\ \algorithmicloop}
\def\algorithmicrepeat{\textbf{répéter}}
\def\algorithmicuntil{\textbf{jusqu'à que}}
\def\algorithmicprint{\textbf{afficher}}
\def\algorithmicreturn{\textbf{retourner}}
\def\algorithmictrue{\textsc{vrai}}
\def\algorithmicfalse{\textsc{faux}}

\let\Sortie\Ensure
\let\Entree\Require
\floatname{algorithm}{Algorithme}

\let\mathscr\mathcal

\newcommand{\snput}[1]{
	\begin{center}
		\Large \texttt{#1}
	\end{center}
	\input{#1}
}

\newbox\tmpbox

%red cross, green tick
\def\rcs{\ensuremath{{\color{red}\times}}}
\def\gtk{\ensuremath{{\color{green}\vee}}}
