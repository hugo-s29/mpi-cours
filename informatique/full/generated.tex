%==============================================================
% Fichier généré automatiquement par `run.py,' ne pas modifier
%==============================================================

\part{Cours}

{
	\chap[-1]{Ordres et induction}
	\minitoc
	\renewcommand{\cwd}{../cours/chap-01/}
	\addmacros{
		\section{(Ne pas) être diagonalisable}

\begin{defn}
	Soit une matrice carrée $A$. On dit que $A$\/ est {\it diagonalisable}\/ s'il existe une matrice inversible~$P \in \mathrm{GL}_n(\mathds{K})$\/ telle que $P^{-1}\cdot A\cdot P$\/ est diagonale.
\end{defn}

\begin{exo}
	\begin{enumerate}
		\item Montrons que la matrice $B = {7\: 1\choose 0\:7}$\/ n'est pas diagonalisable.
			Par l'absurde : on suppose qu'il existe $P \in \mathrm{GL}_2(\R)$\/ et $(\lambda_1, \lambda_2) \in \R^2$\/ tels que \[
				P^{-1} \cdot B \cdot P = \begin{bmatrix}
					\lambda_1 & 0\\
					0&\lambda_2
				\end{bmatrix}
			.\] On applique la trace $\tr$\/ et le déterminant $\det$\/ :
			\begin{gather*}
				\tr(B) = \tr{\lambda_1\:0\choose 0\:\lambda_2} \quad\text{d'où}\quad \lambda_1 + \lambda_2 = 7 + 7 = 14 = \s\\
				\det(B) = \det{\lambda_1\:0\choose 0\:\lambda_2} \quad\text{d'où}\quad \lambda_1 \times \lambda_2 = 7 \times 7 = 49 = p
			\end{gather*}
			D'où $\lambda_1$\/ et $\lambda_2$\/ sont des solutions de l'équation $X^2 - \s X + p = 0$. Or
			\begin{align*}
				X^2 - \s X + p = 0 \iff& X^2 - 14X + 49 = 0\\
				\iff& (X-7)^2 = 0\\
				\iff& X = 7.
			\end{align*}
			D'où 
			\begin{align*}
				B = P P^{-1} B P P^{-1} = P \begin{pmatrix}
					7&0\\
					0&7
				\end{pmatrix} P^{-1} = P \cdot 7I_2\cdot P^{-1} = 7I_2.
			\end{align*}
			La matrice $B$\/ n'est donc pas diagonalisable.

			De même, montrons que la matrice $A$\/ n'est pas diagonalisable. On remarque que \[
				A \cdot \mat{1\\1\\1} = \begin{pmatrix}
					0&1&2\\
					1&0&2\\
					0&0&3
				\end{pmatrix} \begin{pmatrix}
					1\\1\\1
				\end{pmatrix} = \begin{pmatrix}
					3\\3\\3
				\end{pmatrix} = 3\begin{pmatrix}
					1\\1\\1
				\end{pmatrix} 
			.\] Ainsi, \[
				P^{-1}\cdot A\cdot P = \begin{pmatrix}
					3&0&0\\
					0&?&0\\
					0&0&?
				\end{pmatrix}\qquad\text{où}\qquad P = \begin{pmatrix}
					1&?&?\\
					1&?&?\\
					1&?&?
				\end{pmatrix}
			.\] De même, $A\left( \substack{1\\1\\0} \right) = 1 \times \left( \substack{1\\1\\0} \right)$. D'où \[
				P^{-1}\cdot A\cdot P = \begin{pmatrix}
					3&0&0\\
					0&1&0\\
					0&0&?
				\end{pmatrix}\qquad\text{où}\qquad P = \begin{pmatrix}
					1&1&?\\
					1&1&?\\
					1&0&?
				\end{pmatrix}
			.\] Finalement, on en conclut que \[
				P = \begin{pmatrix}
					3&0&0\\
					0&1&0\\
					0&0&-1
				\end{pmatrix} \qquad \text{et}\qquad P^{-1}\cdot A\cdot P = \begin{pmatrix}
					1&1&1\\
					1&1&-1\\
					1&0&0
				\end{pmatrix} = D
			.\]
			De plus, la matrice $P$\/ est inversible car $\det P \neq 0$.
		\item Pour calculer $A^n$, on pourrait chercher un polynôme annulateur $Q$\/ de $A$, et on exprime $X^n = Q \times T_n + R_n$, et donc $A^n = R_n(A)$.
			Mais, on peut également diagonaliser $A$\/ (si elle est diagonalisable).
			Ainsi,  \[
				D^n = (P^{-1}\cdot A\cdot P)^n = P^{-1}\cdot A\cdot \cancel P\cdot \cancel{P^{-1}} \cdot \ldots\cdot \cancel{P^{-1}} \cdot A \cdot P = P^{-1}\cdot  A^n\cdot P
			.\] D'où $A^n = P \cdot D^n \cdot P^{-1}$. Or, \[
				D^n = \begin{pmatrix}
					3&0&0\\
					0&1&0\\
					0&0&-1
				\end{pmatrix}^n = \begin{pmatrix}
					3^n&0&0\\
					0&1^n&0\\
					0&0&(-1)^n
				\end{pmatrix}
			.\]
			On calcule donc $A^{n}$\/ en calculant l'inverse de $P$\/ : \[
				A^n = \begin{pmatrix}
					1&1&1\\
					1&1&-1\\
					1&0&0
				\end{pmatrix} \begin{pmatrix}
					3^n&0&0\\
					0&1^n&0\\
					0&0&(-1)^n
				\end{pmatrix} \cdot P^{-1}
			.\]
		\item
			\begin{align*}
				\begin{rcases*}
					\hfill u_{n+1} = v_n + 2w_n\\
					\hfill v_{n+1} = u_n + 2w_n\\
					\hfill w_{n+1} = 3w_n
				\end{rcases*} \iff& \begin{pmatrix}
					u_{n+1}\\v_{n+1}\\w_{n+1}
				\end{pmatrix} = \begin{pmatrix}
					0&1&2\\
					1&0&2\\
					0&0&3
				\end{pmatrix} \begin{pmatrix}
					u_n\\ v_n\\ w_n
				\end{pmatrix}\\
				\iff& U_{n+1} = A\cdot U_n\\
				\iff& U'_{n+1} = D \cdot U'_{n}
			\end{align*}
			où $D = P^{-1} \cdot A \cdot P$, $U'_{n+1} = P\cdot U_{n+1}$\/ et $U'_n = P\cdot U_n$.
			\begin{align*}
				\phantom{\begin{rcases*}
					\hfill mm_{n+1} = v_n + 2w_n\\
					\hfill v_{n+1} = u_n + 2w_n\\
					\hfill w_{n+1} = 3w_n
				\end{rcases*}} \iff&
				\begin{pmatrix}
					u'_{n+1}\\v'_{n+1}\\w'_{n+1}
				\end{pmatrix} = \begin{pmatrix}
					3&0&0\\
					0&1&0\\
					0&0&-1
				\end{pmatrix} \cdot \begin{pmatrix}
					u'_n\\
					v'_n\\
					w'_n
				\end{pmatrix}\\
				\iff& \begin{cases}
					u'_{n+1} = 3u'_n\\
					v'_{n+1} = v'_n\\
					w'_{n+1} = -w'_n
				\end{cases}\\
				\iff& \begin{cases}
					u'_n = K\times  3^n\\
					v'_n = L\\
					w'_n = M \times (-1)^n
				\end{cases}
			\end{align*}
			Ainsi, \[
				\begin{pmatrix}
					u_n\\v_n\\w_n
				\end{pmatrix} = \underbrace{\begin{pmatrix}
					1&1&1\\
					1&1&-1\\
					1&0&0
				\end{pmatrix}}_P \cdot \begin{pmatrix}
					K\times 3^n\\
					L\\
					M\times (-1)^n
				\end{pmatrix}
			.\] D'où $u_n = K\cdot 3^n + L + M \cdot (-1)^n$, $v_n = K\times 3^n + L - M \cdot (-1)^n$\/ et $w_n = K\cdot 3^n$, où les constantes $K$, $L$\/ et $M$\/ sont des constantes fixées par les conditions initiales.
		\item
			\begin{align*}
				\begin{rcases*}
					\hfill x'(t) = y(t) + 2z(t)\\
					\hfill y'(t) = x(t) + 2z(t)\\
					\hfill z'(t) = 3z(t)
				\end{rcases*} \iff& \begin{pmatrix}
					x'(t)\\
					y'(t)\\
					z'(t)
				\end{pmatrix} = \begin{pmatrix}
					0&1&2\\
					1&0&2\\
					0&0&3
				\end{pmatrix} \cdot \begin{pmatrix}
					x(t)\\
					y(t)\\
					z(t)
				\end{pmatrix}\\
				\iff& X'(t) = A\cdot X(t)\\
				\iff& U'(t) = D \cdot U(t) \text{ avec } D = P^{-1} \cdot A\cdot P \text{ et } X(t) = P\cdot U(t)\\
				\iff& \begin{pmatrix}
					u'(t)\\
					v'(t)\\
					w'(t)
				\end{pmatrix} = \begin{pmatrix}
					3&0&0\\
					0&1&0\\
					0&0&-1
				\end{pmatrix} \cdot \begin{pmatrix}
					u(t)\\
					v(t)\\
					w(t)
				\end{pmatrix}\\
				\iff& \begin{cases}
					u'(t) = 3u(t)\\
					v'(t) = v(t)\\
					w'(t) = -w(t)
				\end{cases}\\
				\iff& \begin{cases}
					u(t) = K \cdot \mathrm{e}^{3t}\\
					v(t) = L \cdot \mathrm{e}^{t}\\
					w(t) = M \cdot \mathrm{e}^{-t}
				\end{cases}
			\end{align*}
			Ainsi \[
				\begin{pmatrix}
					x(t)\\
					y(t)\\
					z(t)
				\end{pmatrix} = \underbrace{\begin{pmatrix}
					1&1&1\\
					1&1&-1\\
					1&0&0
				\end{pmatrix}}_P \cdot \begin{pmatrix}
					K \times \mathrm{e}^{3t}\\
					L \cdot \mathrm{e}^{t}\\
					M \cdot \mathrm{e}^{-t}
				\end{pmatrix}
			.\] 
			D'où $x(t) = K\cdot \mathrm{e}^{3t} + L \cdot \mathrm{e}^{t} + M \cdot \mathrm{e}^{-t}$, $y(t) = K \cdot \mathrm{e}^{3t} + L \cdot \mathrm{e}^{t} - M \cdot \mathrm{e}^{-t}$\/ et $z(t) = K\cdot \mathrm{e}^{3t}$. Les constantes $K$, $L$\/ et $M$\/ peuvent être déterminées à partir des conditions initiales.
	\end{enumerate}
\end{exo}

\begin{rmkn}[équations différentielles]
	On considère l'équation différentielle $(*)$ : $x'(t) = \lambda \cdot x(t)$.
	Les fonctions $x : t \mapsto K\cdot \mathrm{e}^{\lambda t}$\/ sont des solutions de cette équation. On peut utiliser la méthode de {\sc Lagrange}\/ : la méthode de la~\guillemotleft~variation de la constante.~\guillemotright\@ On cherche des solutions sous la forme $x(t) = k(t) \cdot \mathrm{e}^{\lambda t}$ (vision du~physicien). D'où $k(t) = x(t) / \mathrm{e}^{\lambda t}$\/ (vision du mathématicien). De plus, $x'(t) = k'(t) \mathrm{e}^{\lambda t} + k(t) \lambda \mathrm{e}^{\lambda t}$.
	Ainsi, on injecte ce $k(t)$\/ dans l'équation différentielle :
	\begin{align*}
		(*) \iff& k'(t) \mathrm{e}^{\lambda t} + k(t) \lambda \mathrm{e}^{\lambda t} = \lambda k(t)\mathrm{e}^{\lambda t}\\
		\iff& k'(t) \mathrm{e}^{\lambda t} = 0\\
		\iff& k'(t) = 0\\
		\iff& \exists K \in \R\,\:k(t) = K.
	\end{align*}
	Les solutions trouvées dans l'exercice précédent sont donc les uniques solutions du système d'équations différentielles.

	De même, pour résoudre une équation différentielle avec 2\tsup{nd} membre de la forme \[
		(**) : \qquad x'(t) - \lambda \cdot x(t) = b(t)
	.\]
	La fonction $t \mapsto x(t)$\/ est une solution de l'équation {\sc sans}\/ 2\tsup{nd} membre si et seulement si \[
		\exists K \in \R,\:\forall t \in \R,\quad x(t) = K \cdot \mathrm{e}^{\lambda t}
	.\]
	\begin{center}
		\slshape Comment résoudre l'équation différentielle {\scshape avec}\/ 2\tsup{nd} membre si on connaît la solution générale de l'équation {\scshape sans}\/ 2\tsup{nd} membre ?
	\end{center}
	On utilise la méthode le la variation de la constante.
	Soit $x(t) = k(t) \cdot \mathrm{e}^{\lambda t}$. Ainsi, en injectant cette expression de $x$\/ dans l'équation $(**)$, on trouve
	\begin{align*}
		(**) \iff& k'(t) \mathrm{e}^{\lambda t} + k(t) \cdot \lambda \mathrm{e}^{\lambda t} = \lambda k(t) \mathrm{e}^{\lambda t} + b(t)\\
		\iff& k'(t) \mathrm{e}^{\lambda t} = b(t)\\
		\iff& k'(t) = b(t) \cdot \mathrm{e}^{-\lambda t}\\
		\iff& k(t) = \int_{0}^{t} b(u)\cdot \mathrm{e}^{-\lambda u}~\mathrm{d}u + K\\
		\iff& x(t) = \left( \int_{0}^{t} b(u) \cdot \mathrm{e}^{-\lambda u}~\mathrm{d}u + K \right) \mathrm{e}^{\lambda t}\\
		\iff& x(t) = \underbrace{\int_{0}^{t} b(u) \cdot \mathrm{e}^{\lambda (t-u)}~\mathrm{d}u}_{\text{solution particulière}} + \underbrace{K \cdot \mathrm{e}^{\lambda t}}_{\substack{\text{solution}\\\text{générale}\\\text{de $(*)$}}}.
	\end{align*}
\end{rmkn}

		\begin{exm}
	On pose $f$, le sinus cardinal :  \begin{align*}
		f: \R^* &\longrightarrow \R \\
		t &\longmapsto \frac{\sin t}{t}.
	\end{align*}
	\begin{figure}[H]
		\centering
		\begin{asy}
			import graph;
			size(10cm);
			draw((-10, 0) -- (10, 0), Arrow(TeXHead));
			draw((0, -3) -- (0, 5), Arrow(TeXHead));
			real f(real x) {
				if(x == 0) { return 3; }
				else {return 3*sin(x) / x;}
			}
			draw(graph(f, -10, 10), magenta);
		\end{asy}
		\caption{Sinus cardinal}
	\end{figure}

	La fonction $f$\/ est continue sur ${]0,8]}$\/ mais $\lim_{t\to 0} \frac{\sin t}{t} = 1$. D'où $\int_{0}^{8} \frac{\sin t}{t}~\mathrm{d}t$\/ est faussement impropre en $0$\/ et donc convergente.


	Mais attention ! On ne dit pas \guillemotleft~{\color{red}soit $f : t \mapsto \frac{1}{t}$. L'intégrale $\int_{8}^{+\infty} \frac{1}{t}~\mathrm{d}t$\/ est faussement impropre en $+\infty$\/ car $\lim_{t\to +\infty}\frac{1}{t} = 0$}.~\guillemotright
\end{exm}

\section{Intégrer les $\mathbf{\sim}$, $\po$, et \textit{O}}

\begin{thm}
	\hfill$\O$\hfill\null
\end{thm}

\begin{thm}
	Le 2.\ n'est pas la réciproque du 1.\ mais la contraposée.
\end{thm}

\begin{prop}
	\hfill$\O$\hfill\null
\end{prop}

\begin{exm}
	On considère l'intégrale $\int_{2}^{+\infty} \frac{1}{t^2+ \cos t}~\mathrm{d}t$, c'est une intégrale impropre en $+\infty$.
	On recherche un équivalent de $\frac{1}{t^2 + \cos t}$\/ en $+\infty$ : \[
		\frac{1}{t^2 + \cos t} \simi_{t\to +\infty} \frac{1}{t^2}
	\] qui ne change pas de signe. Or, $\int_{2}^{+\infty} \frac{1}{t^2}~\mathrm{d}t$\/ converge car c'est une intégrale de {\sc Riemann}\/ avec $\alpha = 2 > 1$.
	On en déduit que l'intégrale $I$\/ converge.

	On procède autrement : \[
		0 \le \frac{1}{t^2 + \cos t} \le \frac{1}{t^2 - 1}
	.\] Or, $\int_{2}^{+\infty} \frac{1}{t^2 - 1}~\mathrm{d}t$\/ converge car
	\begin{align*}
		\int_{2}^{x} \frac{1}{t^2 - 1}~\mathrm{d}t &= \int_{2}^{x} \left( \frac{\sfrac12}{t-1} - \frac{\sfrac12}{t+1} \right) ~\mathrm{d}t \\
		&= \frac{1}{2} \int_{2}^{x} \frac{1}{t-1}~\mathrm{d}t - \frac{1}{2}\int_{2}^{x} \frac{1}{t+1}~\mathrm{d}t \\
		&= \frac{1}{2} \Big[\ln|t-1|\Big]_2^x - \frac{1}{2}\Big[\ln |t+1|\Big]_2^x \\
	\end{align*}
	D'où \[
		\int_{2}^{x} \frac{1}{t^2 - 1}~\mathrm{d}t = \frac{1}{2} \left[ \ln\left| \frac{t-1}{t+1} \right| \right]_2^x = \frac{1}{2}\ln \left| \frac{x-1}{x+1} \right| + \frac{1}{2} \ln 3 \tendsto{x\to +\infty} \frac{1}{2} \ln 3
	.\] donc l'intégrale $I$\/ converge et $I \le \frac{1}{2} \ln_3$.
\end{exm}

\begin{exo}
	\begin{enumerate}
		\item L'intégrale $I = \int_{0}^{1} \frac{\sin t}{t^2}~\mathrm{d}t$\/ est impropre en 0. On utilise un équivalent : $\sin t \simi_{t\to 0} t$\/ qui ne change pas de signe. Or, $\int_{0}^{t} \frac{1}{t}~\mathrm{d}t$\/ diverge (par critère de {\sc Riemann}). Donc $I$\/ diverge.
			
			L'intégrale $J = \int_{1}^{+\infty} \sin \frac{1}{t}~\mathrm{d}t$\/ est généralisée en $+\infty$. On cherche un équivalent en $+\infty$\/ : \[
				\sin \frac{1}{t} \simi_{t\to +\infty} \frac{1}{t}
			\] qui ne change pas de signe. Or, $\int_{1}^{+\infty} \frac{1}{t}~\mathrm{d}t$\/ diverge par critère de {\sc Riemann}. On en déduit que $J$\/ diverge également.
		\item L'intégrale $\int_{0}^{+\infty} \frac{1}{t^2}~\mathrm{d}t$\/ est impropre, {\bf et}\/ en 0, {\bf et}\/ en $+\infty$. Le théorème ne marche donc pas.
			En effet $t\mapsto \frac{1}{t^2}$\/ n'est pas continue par morceaux en 0, ce qui était le cas pour $t\mapsto \frac{1}{1+t^2}$.
	\end{enumerate}
\end{exo}

\begin{rmkn}[Retour sur la {\sc remarque}\/ 5]
	L'intégrale $\int_{0}^{+\infty} \frac{1}{\ln(1+t)}~\mathrm{d}t$\/ est impropre en 0 {\bf et}\/ en $+\infty$. $\int_{0}^{+\infty} \frac{1}{\ln(1+t)}~\mathrm{d}t$\/ converge si et seulement si $\int_{0}^{7} \frac{1}{\ln(1+t)}~\mathrm{d}t$\/ {\bf et}\/ $\int_{7}^{+\infty} \frac{1}{\ln(1+t)}~\mathrm{d}t$\/ convergent.
	Et si elles convergent \[
		\int_{0}^{+\infty} \frac{1}{\ln(1+t)}~\mathrm{d}t = \int_{0}^{7} \frac{1}{\ln(1+t)}~\mathrm{d}t + \int_{7}^{+\infty} \frac{1}{\ln(1+t)}~\mathrm{d}t
	.\]
	On n'utilise pas deux barrières en même temps. Sinon, les intégrales doublement impropres peuvent, et converger, et diverger.
\end{rmkn}

\begin{prop}[avec $\sim$]
	Si $f(t) \simi_{t\to b} g(t)$\/ qui ne change pas de signe. Alors,
	\begin{itemize}
		\item ou bien $\ds\int_{a}^{b} f(t)~\mathrm{d}t$\/ et $\ds\int_{a}^{b} g(t)~\mathrm{d}t$\/ convergent et $\ds \int_{x}^{b} f(t)~\mathrm{d}t \simi_{x\to b} \int_{x}^{b} g(t)~\mathrm{d}t$.
		\item ou bien $\ds\int_{a}^{b} f(t)~\mathrm{d}t$\/ et $\ds\int_{a}^{b} g(t)~\mathrm{d}t$\/ divergent et $\ds\int_{a}^{x} f(t)~\mathrm{d}t \simi_{x\to b} \int_{a}^{x} g(t)~\mathrm{d}t$.
	\end{itemize}
	Cette proposition est équivalente à le {\sc lemme}\/ 13 sur les séries.
\end{prop}



		\addrecap
	}
	\def\addmacros#1{#1}
}

{
	\chap[0]{Logique}
	\minitoc
	\renewcommand{\cwd}{../cours/chap00/}
	\addmacros{
		\section{(Ne pas) être diagonalisable}

\begin{defn}
	Soit une matrice carrée $A$. On dit que $A$\/ est {\it diagonalisable}\/ s'il existe une matrice inversible~$P \in \mathrm{GL}_n(\mathds{K})$\/ telle que $P^{-1}\cdot A\cdot P$\/ est diagonale.
\end{defn}

\begin{exo}
	\begin{enumerate}
		\item Montrons que la matrice $B = {7\: 1\choose 0\:7}$\/ n'est pas diagonalisable.
			Par l'absurde : on suppose qu'il existe $P \in \mathrm{GL}_2(\R)$\/ et $(\lambda_1, \lambda_2) \in \R^2$\/ tels que \[
				P^{-1} \cdot B \cdot P = \begin{bmatrix}
					\lambda_1 & 0\\
					0&\lambda_2
				\end{bmatrix}
			.\] On applique la trace $\tr$\/ et le déterminant $\det$\/ :
			\begin{gather*}
				\tr(B) = \tr{\lambda_1\:0\choose 0\:\lambda_2} \quad\text{d'où}\quad \lambda_1 + \lambda_2 = 7 + 7 = 14 = \s\\
				\det(B) = \det{\lambda_1\:0\choose 0\:\lambda_2} \quad\text{d'où}\quad \lambda_1 \times \lambda_2 = 7 \times 7 = 49 = p
			\end{gather*}
			D'où $\lambda_1$\/ et $\lambda_2$\/ sont des solutions de l'équation $X^2 - \s X + p = 0$. Or
			\begin{align*}
				X^2 - \s X + p = 0 \iff& X^2 - 14X + 49 = 0\\
				\iff& (X-7)^2 = 0\\
				\iff& X = 7.
			\end{align*}
			D'où 
			\begin{align*}
				B = P P^{-1} B P P^{-1} = P \begin{pmatrix}
					7&0\\
					0&7
				\end{pmatrix} P^{-1} = P \cdot 7I_2\cdot P^{-1} = 7I_2.
			\end{align*}
			La matrice $B$\/ n'est donc pas diagonalisable.

			De même, montrons que la matrice $A$\/ n'est pas diagonalisable. On remarque que \[
				A \cdot \mat{1\\1\\1} = \begin{pmatrix}
					0&1&2\\
					1&0&2\\
					0&0&3
				\end{pmatrix} \begin{pmatrix}
					1\\1\\1
				\end{pmatrix} = \begin{pmatrix}
					3\\3\\3
				\end{pmatrix} = 3\begin{pmatrix}
					1\\1\\1
				\end{pmatrix} 
			.\] Ainsi, \[
				P^{-1}\cdot A\cdot P = \begin{pmatrix}
					3&0&0\\
					0&?&0\\
					0&0&?
				\end{pmatrix}\qquad\text{où}\qquad P = \begin{pmatrix}
					1&?&?\\
					1&?&?\\
					1&?&?
				\end{pmatrix}
			.\] De même, $A\left( \substack{1\\1\\0} \right) = 1 \times \left( \substack{1\\1\\0} \right)$. D'où \[
				P^{-1}\cdot A\cdot P = \begin{pmatrix}
					3&0&0\\
					0&1&0\\
					0&0&?
				\end{pmatrix}\qquad\text{où}\qquad P = \begin{pmatrix}
					1&1&?\\
					1&1&?\\
					1&0&?
				\end{pmatrix}
			.\] Finalement, on en conclut que \[
				P = \begin{pmatrix}
					3&0&0\\
					0&1&0\\
					0&0&-1
				\end{pmatrix} \qquad \text{et}\qquad P^{-1}\cdot A\cdot P = \begin{pmatrix}
					1&1&1\\
					1&1&-1\\
					1&0&0
				\end{pmatrix} = D
			.\]
			De plus, la matrice $P$\/ est inversible car $\det P \neq 0$.
		\item Pour calculer $A^n$, on pourrait chercher un polynôme annulateur $Q$\/ de $A$, et on exprime $X^n = Q \times T_n + R_n$, et donc $A^n = R_n(A)$.
			Mais, on peut également diagonaliser $A$\/ (si elle est diagonalisable).
			Ainsi,  \[
				D^n = (P^{-1}\cdot A\cdot P)^n = P^{-1}\cdot A\cdot \cancel P\cdot \cancel{P^{-1}} \cdot \ldots\cdot \cancel{P^{-1}} \cdot A \cdot P = P^{-1}\cdot  A^n\cdot P
			.\] D'où $A^n = P \cdot D^n \cdot P^{-1}$. Or, \[
				D^n = \begin{pmatrix}
					3&0&0\\
					0&1&0\\
					0&0&-1
				\end{pmatrix}^n = \begin{pmatrix}
					3^n&0&0\\
					0&1^n&0\\
					0&0&(-1)^n
				\end{pmatrix}
			.\]
			On calcule donc $A^{n}$\/ en calculant l'inverse de $P$\/ : \[
				A^n = \begin{pmatrix}
					1&1&1\\
					1&1&-1\\
					1&0&0
				\end{pmatrix} \begin{pmatrix}
					3^n&0&0\\
					0&1^n&0\\
					0&0&(-1)^n
				\end{pmatrix} \cdot P^{-1}
			.\]
		\item
			\begin{align*}
				\begin{rcases*}
					\hfill u_{n+1} = v_n + 2w_n\\
					\hfill v_{n+1} = u_n + 2w_n\\
					\hfill w_{n+1} = 3w_n
				\end{rcases*} \iff& \begin{pmatrix}
					u_{n+1}\\v_{n+1}\\w_{n+1}
				\end{pmatrix} = \begin{pmatrix}
					0&1&2\\
					1&0&2\\
					0&0&3
				\end{pmatrix} \begin{pmatrix}
					u_n\\ v_n\\ w_n
				\end{pmatrix}\\
				\iff& U_{n+1} = A\cdot U_n\\
				\iff& U'_{n+1} = D \cdot U'_{n}
			\end{align*}
			où $D = P^{-1} \cdot A \cdot P$, $U'_{n+1} = P\cdot U_{n+1}$\/ et $U'_n = P\cdot U_n$.
			\begin{align*}
				\phantom{\begin{rcases*}
					\hfill mm_{n+1} = v_n + 2w_n\\
					\hfill v_{n+1} = u_n + 2w_n\\
					\hfill w_{n+1} = 3w_n
				\end{rcases*}} \iff&
				\begin{pmatrix}
					u'_{n+1}\\v'_{n+1}\\w'_{n+1}
				\end{pmatrix} = \begin{pmatrix}
					3&0&0\\
					0&1&0\\
					0&0&-1
				\end{pmatrix} \cdot \begin{pmatrix}
					u'_n\\
					v'_n\\
					w'_n
				\end{pmatrix}\\
				\iff& \begin{cases}
					u'_{n+1} = 3u'_n\\
					v'_{n+1} = v'_n\\
					w'_{n+1} = -w'_n
				\end{cases}\\
				\iff& \begin{cases}
					u'_n = K\times  3^n\\
					v'_n = L\\
					w'_n = M \times (-1)^n
				\end{cases}
			\end{align*}
			Ainsi, \[
				\begin{pmatrix}
					u_n\\v_n\\w_n
				\end{pmatrix} = \underbrace{\begin{pmatrix}
					1&1&1\\
					1&1&-1\\
					1&0&0
				\end{pmatrix}}_P \cdot \begin{pmatrix}
					K\times 3^n\\
					L\\
					M\times (-1)^n
				\end{pmatrix}
			.\] D'où $u_n = K\cdot 3^n + L + M \cdot (-1)^n$, $v_n = K\times 3^n + L - M \cdot (-1)^n$\/ et $w_n = K\cdot 3^n$, où les constantes $K$, $L$\/ et $M$\/ sont des constantes fixées par les conditions initiales.
		\item
			\begin{align*}
				\begin{rcases*}
					\hfill x'(t) = y(t) + 2z(t)\\
					\hfill y'(t) = x(t) + 2z(t)\\
					\hfill z'(t) = 3z(t)
				\end{rcases*} \iff& \begin{pmatrix}
					x'(t)\\
					y'(t)\\
					z'(t)
				\end{pmatrix} = \begin{pmatrix}
					0&1&2\\
					1&0&2\\
					0&0&3
				\end{pmatrix} \cdot \begin{pmatrix}
					x(t)\\
					y(t)\\
					z(t)
				\end{pmatrix}\\
				\iff& X'(t) = A\cdot X(t)\\
				\iff& U'(t) = D \cdot U(t) \text{ avec } D = P^{-1} \cdot A\cdot P \text{ et } X(t) = P\cdot U(t)\\
				\iff& \begin{pmatrix}
					u'(t)\\
					v'(t)\\
					w'(t)
				\end{pmatrix} = \begin{pmatrix}
					3&0&0\\
					0&1&0\\
					0&0&-1
				\end{pmatrix} \cdot \begin{pmatrix}
					u(t)\\
					v(t)\\
					w(t)
				\end{pmatrix}\\
				\iff& \begin{cases}
					u'(t) = 3u(t)\\
					v'(t) = v(t)\\
					w'(t) = -w(t)
				\end{cases}\\
				\iff& \begin{cases}
					u(t) = K \cdot \mathrm{e}^{3t}\\
					v(t) = L \cdot \mathrm{e}^{t}\\
					w(t) = M \cdot \mathrm{e}^{-t}
				\end{cases}
			\end{align*}
			Ainsi \[
				\begin{pmatrix}
					x(t)\\
					y(t)\\
					z(t)
				\end{pmatrix} = \underbrace{\begin{pmatrix}
					1&1&1\\
					1&1&-1\\
					1&0&0
				\end{pmatrix}}_P \cdot \begin{pmatrix}
					K \times \mathrm{e}^{3t}\\
					L \cdot \mathrm{e}^{t}\\
					M \cdot \mathrm{e}^{-t}
				\end{pmatrix}
			.\] 
			D'où $x(t) = K\cdot \mathrm{e}^{3t} + L \cdot \mathrm{e}^{t} + M \cdot \mathrm{e}^{-t}$, $y(t) = K \cdot \mathrm{e}^{3t} + L \cdot \mathrm{e}^{t} - M \cdot \mathrm{e}^{-t}$\/ et $z(t) = K\cdot \mathrm{e}^{3t}$. Les constantes $K$, $L$\/ et $M$\/ peuvent être déterminées à partir des conditions initiales.
	\end{enumerate}
\end{exo}

\begin{rmkn}[équations différentielles]
	On considère l'équation différentielle $(*)$ : $x'(t) = \lambda \cdot x(t)$.
	Les fonctions $x : t \mapsto K\cdot \mathrm{e}^{\lambda t}$\/ sont des solutions de cette équation. On peut utiliser la méthode de {\sc Lagrange}\/ : la méthode de la~\guillemotleft~variation de la constante.~\guillemotright\@ On cherche des solutions sous la forme $x(t) = k(t) \cdot \mathrm{e}^{\lambda t}$ (vision du~physicien). D'où $k(t) = x(t) / \mathrm{e}^{\lambda t}$\/ (vision du mathématicien). De plus, $x'(t) = k'(t) \mathrm{e}^{\lambda t} + k(t) \lambda \mathrm{e}^{\lambda t}$.
	Ainsi, on injecte ce $k(t)$\/ dans l'équation différentielle :
	\begin{align*}
		(*) \iff& k'(t) \mathrm{e}^{\lambda t} + k(t) \lambda \mathrm{e}^{\lambda t} = \lambda k(t)\mathrm{e}^{\lambda t}\\
		\iff& k'(t) \mathrm{e}^{\lambda t} = 0\\
		\iff& k'(t) = 0\\
		\iff& \exists K \in \R\,\:k(t) = K.
	\end{align*}
	Les solutions trouvées dans l'exercice précédent sont donc les uniques solutions du système d'équations différentielles.

	De même, pour résoudre une équation différentielle avec 2\tsup{nd} membre de la forme \[
		(**) : \qquad x'(t) - \lambda \cdot x(t) = b(t)
	.\]
	La fonction $t \mapsto x(t)$\/ est une solution de l'équation {\sc sans}\/ 2\tsup{nd} membre si et seulement si \[
		\exists K \in \R,\:\forall t \in \R,\quad x(t) = K \cdot \mathrm{e}^{\lambda t}
	.\]
	\begin{center}
		\slshape Comment résoudre l'équation différentielle {\scshape avec}\/ 2\tsup{nd} membre si on connaît la solution générale de l'équation {\scshape sans}\/ 2\tsup{nd} membre ?
	\end{center}
	On utilise la méthode le la variation de la constante.
	Soit $x(t) = k(t) \cdot \mathrm{e}^{\lambda t}$. Ainsi, en injectant cette expression de $x$\/ dans l'équation $(**)$, on trouve
	\begin{align*}
		(**) \iff& k'(t) \mathrm{e}^{\lambda t} + k(t) \cdot \lambda \mathrm{e}^{\lambda t} = \lambda k(t) \mathrm{e}^{\lambda t} + b(t)\\
		\iff& k'(t) \mathrm{e}^{\lambda t} = b(t)\\
		\iff& k'(t) = b(t) \cdot \mathrm{e}^{-\lambda t}\\
		\iff& k(t) = \int_{0}^{t} b(u)\cdot \mathrm{e}^{-\lambda u}~\mathrm{d}u + K\\
		\iff& x(t) = \left( \int_{0}^{t} b(u) \cdot \mathrm{e}^{-\lambda u}~\mathrm{d}u + K \right) \mathrm{e}^{\lambda t}\\
		\iff& x(t) = \underbrace{\int_{0}^{t} b(u) \cdot \mathrm{e}^{\lambda (t-u)}~\mathrm{d}u}_{\text{solution particulière}} + \underbrace{K \cdot \mathrm{e}^{\lambda t}}_{\substack{\text{solution}\\\text{générale}\\\text{de $(*)$}}}.
	\end{align*}
\end{rmkn}

		\begin{exm}
	On pose $f$, le sinus cardinal :  \begin{align*}
		f: \R^* &\longrightarrow \R \\
		t &\longmapsto \frac{\sin t}{t}.
	\end{align*}
	\begin{figure}[H]
		\centering
		\begin{asy}
			import graph;
			size(10cm);
			draw((-10, 0) -- (10, 0), Arrow(TeXHead));
			draw((0, -3) -- (0, 5), Arrow(TeXHead));
			real f(real x) {
				if(x == 0) { return 3; }
				else {return 3*sin(x) / x;}
			}
			draw(graph(f, -10, 10), magenta);
		\end{asy}
		\caption{Sinus cardinal}
	\end{figure}

	La fonction $f$\/ est continue sur ${]0,8]}$\/ mais $\lim_{t\to 0} \frac{\sin t}{t} = 1$. D'où $\int_{0}^{8} \frac{\sin t}{t}~\mathrm{d}t$\/ est faussement impropre en $0$\/ et donc convergente.


	Mais attention ! On ne dit pas \guillemotleft~{\color{red}soit $f : t \mapsto \frac{1}{t}$. L'intégrale $\int_{8}^{+\infty} \frac{1}{t}~\mathrm{d}t$\/ est faussement impropre en $+\infty$\/ car $\lim_{t\to +\infty}\frac{1}{t} = 0$}.~\guillemotright
\end{exm}

\section{Intégrer les $\mathbf{\sim}$, $\po$, et \textit{O}}

\begin{thm}
	\hfill$\O$\hfill\null
\end{thm}

\begin{thm}
	Le 2.\ n'est pas la réciproque du 1.\ mais la contraposée.
\end{thm}

\begin{prop}
	\hfill$\O$\hfill\null
\end{prop}

\begin{exm}
	On considère l'intégrale $\int_{2}^{+\infty} \frac{1}{t^2+ \cos t}~\mathrm{d}t$, c'est une intégrale impropre en $+\infty$.
	On recherche un équivalent de $\frac{1}{t^2 + \cos t}$\/ en $+\infty$ : \[
		\frac{1}{t^2 + \cos t} \simi_{t\to +\infty} \frac{1}{t^2}
	\] qui ne change pas de signe. Or, $\int_{2}^{+\infty} \frac{1}{t^2}~\mathrm{d}t$\/ converge car c'est une intégrale de {\sc Riemann}\/ avec $\alpha = 2 > 1$.
	On en déduit que l'intégrale $I$\/ converge.

	On procède autrement : \[
		0 \le \frac{1}{t^2 + \cos t} \le \frac{1}{t^2 - 1}
	.\] Or, $\int_{2}^{+\infty} \frac{1}{t^2 - 1}~\mathrm{d}t$\/ converge car
	\begin{align*}
		\int_{2}^{x} \frac{1}{t^2 - 1}~\mathrm{d}t &= \int_{2}^{x} \left( \frac{\sfrac12}{t-1} - \frac{\sfrac12}{t+1} \right) ~\mathrm{d}t \\
		&= \frac{1}{2} \int_{2}^{x} \frac{1}{t-1}~\mathrm{d}t - \frac{1}{2}\int_{2}^{x} \frac{1}{t+1}~\mathrm{d}t \\
		&= \frac{1}{2} \Big[\ln|t-1|\Big]_2^x - \frac{1}{2}\Big[\ln |t+1|\Big]_2^x \\
	\end{align*}
	D'où \[
		\int_{2}^{x} \frac{1}{t^2 - 1}~\mathrm{d}t = \frac{1}{2} \left[ \ln\left| \frac{t-1}{t+1} \right| \right]_2^x = \frac{1}{2}\ln \left| \frac{x-1}{x+1} \right| + \frac{1}{2} \ln 3 \tendsto{x\to +\infty} \frac{1}{2} \ln 3
	.\] donc l'intégrale $I$\/ converge et $I \le \frac{1}{2} \ln_3$.
\end{exm}

\begin{exo}
	\begin{enumerate}
		\item L'intégrale $I = \int_{0}^{1} \frac{\sin t}{t^2}~\mathrm{d}t$\/ est impropre en 0. On utilise un équivalent : $\sin t \simi_{t\to 0} t$\/ qui ne change pas de signe. Or, $\int_{0}^{t} \frac{1}{t}~\mathrm{d}t$\/ diverge (par critère de {\sc Riemann}). Donc $I$\/ diverge.
			
			L'intégrale $J = \int_{1}^{+\infty} \sin \frac{1}{t}~\mathrm{d}t$\/ est généralisée en $+\infty$. On cherche un équivalent en $+\infty$\/ : \[
				\sin \frac{1}{t} \simi_{t\to +\infty} \frac{1}{t}
			\] qui ne change pas de signe. Or, $\int_{1}^{+\infty} \frac{1}{t}~\mathrm{d}t$\/ diverge par critère de {\sc Riemann}. On en déduit que $J$\/ diverge également.
		\item L'intégrale $\int_{0}^{+\infty} \frac{1}{t^2}~\mathrm{d}t$\/ est impropre, {\bf et}\/ en 0, {\bf et}\/ en $+\infty$. Le théorème ne marche donc pas.
			En effet $t\mapsto \frac{1}{t^2}$\/ n'est pas continue par morceaux en 0, ce qui était le cas pour $t\mapsto \frac{1}{1+t^2}$.
	\end{enumerate}
\end{exo}

\begin{rmkn}[Retour sur la {\sc remarque}\/ 5]
	L'intégrale $\int_{0}^{+\infty} \frac{1}{\ln(1+t)}~\mathrm{d}t$\/ est impropre en 0 {\bf et}\/ en $+\infty$. $\int_{0}^{+\infty} \frac{1}{\ln(1+t)}~\mathrm{d}t$\/ converge si et seulement si $\int_{0}^{7} \frac{1}{\ln(1+t)}~\mathrm{d}t$\/ {\bf et}\/ $\int_{7}^{+\infty} \frac{1}{\ln(1+t)}~\mathrm{d}t$\/ convergent.
	Et si elles convergent \[
		\int_{0}^{+\infty} \frac{1}{\ln(1+t)}~\mathrm{d}t = \int_{0}^{7} \frac{1}{\ln(1+t)}~\mathrm{d}t + \int_{7}^{+\infty} \frac{1}{\ln(1+t)}~\mathrm{d}t
	.\]
	On n'utilise pas deux barrières en même temps. Sinon, les intégrales doublement impropres peuvent, et converger, et diverger.
\end{rmkn}

\begin{prop}[avec $\sim$]
	Si $f(t) \simi_{t\to b} g(t)$\/ qui ne change pas de signe. Alors,
	\begin{itemize}
		\item ou bien $\ds\int_{a}^{b} f(t)~\mathrm{d}t$\/ et $\ds\int_{a}^{b} g(t)~\mathrm{d}t$\/ convergent et $\ds \int_{x}^{b} f(t)~\mathrm{d}t \simi_{x\to b} \int_{x}^{b} g(t)~\mathrm{d}t$.
		\item ou bien $\ds\int_{a}^{b} f(t)~\mathrm{d}t$\/ et $\ds\int_{a}^{b} g(t)~\mathrm{d}t$\/ divergent et $\ds\int_{a}^{x} f(t)~\mathrm{d}t \simi_{x\to b} \int_{a}^{x} g(t)~\mathrm{d}t$.
	\end{itemize}
	Cette proposition est équivalente à le {\sc lemme}\/ 13 sur les séries.
\end{prop}



		\begin{prop}
	La relation $\preceq$\/ est un \textit{pré-ordre} :
	\begin{itemize}
		\item $\preceq $\/ est réflective ;
		\item $\preceq $\/ est transitive.
	\end{itemize}
\end{prop}

\begin{prv}
	Soit $Q$\/ un problème de décision.
	\begin{itemize}
		\item $Q \preceq Q$\/ par la fonction identité, qui est totale et calculable.
		\item Soient $Q$, $R$\/ et $S$\/ trois problèmes de décision tels que $Q \preceq R$\/ et $R \preceq S$. Soit donc $f_1$\/ la réduction de $Q$\/ à $R$, et $f_2$\/ la réduction de $R$\/ à $S$. Soit $f = f_2 \circ f_1 : \mathcal{E}_Q \to \mathcal{E}_S$. La fonction $f$\/ est totale comme composée de fonctions totales, $f$\/ est calculable comme composée de fonctions calculables. De plus,
			\begin{align*}
				\forall e \in \mathcal{E}_Q,\qquad f(e) \in S^+ \iff& f_2(f_1(e)) \in S^+\\
				\iff& f_1(e) \in R^+\\
				\iff& e \in Q^+
			\end{align*}
	\end{itemize}
\end{prv}

\section{Classe \textbf{P} et \textbf{NP}}

Pour répondre à un problème, on peut le résoudre par des algorithmes plus ou moins rapides. Mais, l'objectif de cette section est de montrer que certains problèmes ne peuvent se résoudre que par des algorithmes lents, et que l'on ne peut pas faire mieux.

\begin{defn}
	Le modèle de calcul impose une représentation des entrées par chaînes de caractères. Cela induit donc une notion de \textit{taille d'entrée}, qui est la longueur de la chaîne de caractères.
	\index{taille d'entrée}
\end{defn}


\subsection{Complexité d'une machine}

\begin{defn}
	Étant donné une machine $\mathcal{M}$ et une entrée $w \in \Sigma^*$, on note $C^\mathcal{M}(w)$\/ le nombre d'opérations élémentaires effectuées lors de l'appel de $\mathcal{M}$\/ sur $w$. Lorsque $\smash{w \xrightarrow[\mathcal{M}]{} {\circlearrowleft}}$, on définit $C^\mathcal{M} = +\infty$.

	Pour $n \in \N$, on définit alors \[
		C^\mathcal{M}_n = \max \{ C^\mathcal{M}(w)  \mid w \in \Sigma^n \}
	.\]
	\index{machine!nombre d'opérations élémentaires!($C^\mathcal{M}(w)$)}
	\index{machine!nombre d'opérations élémentaires!maximal pour un mot de taille $n$\/ ($C^\mathcal{M}_n$)}
\end{defn}

\begin{rmk}
	On a, $\forall n \in \N$, $C_n^\mathcal{M} \in \bar{\N} = \N \cup \{+\infty\}$.
\end{rmk}

\begin{defn}
	Soit $f : \N\to \N$\/ une fonction totale et calculable. On note $\textsc{Time}(f)$\/ l'ensemble des machines $\mathcal{M}$\/ telles que
	\begin{itemize}
		\item $\mathcal{M}$\/ s'arrête sur toute entrée ;
		\item $\big(C_n^\mathcal{M}\big)_{n \in \N} = \mathcal{O}\big(\big(f(n)\big)_{n \in \N}\big)$.
	\end{itemize}
	\index{machine!ensemble $\textsc{Time}(f)$}
\end{defn}

\subsection{Classe \textbf{P}}

\begin{defn}
	On dit d'une machine $\mathcal{M}$\/ qu'elle est de \textit{complexité polynômiale} dès lors qu'il existe $k \in \N$\/ tel que $\mathcal{M} \in \textsc{Time}(n^k)$.
	\index{machine!de complexité polynômiale}
\end{defn}

\begin{defn}
	On dit d'une fonction (partielle ou non), qu'elle est \textit{calculable en temps polynômial} dès lors qu'il existe une machine $\mathcal{M}$\/ de complexité polynômiale la calculant.
	\index{fonction!calculable!en temps polynômial}
\end{defn}

\begin{exm}
	\begin{itemize}
		\item l'identité ($n \mapsto n$)
		\item la fonction successeur ($n\mapsto n+1$)
	\end{itemize}
\end{exm}


		\section{Arbres couvrants de poids minimum}

\begin{exm}
	On considère le graphe ci-dessous.
	\begin{figure}[H]
		\centering
		\tikzfig{ex-graphe-pondere}
		\caption{Arbre pondéré}
	\end{figure}
	On cherche à \guillemotleft~supprimer~\guillemotright\ des arrêtes de ce graphe afin d'avoir un poids total minimum, tout en conservant la connexité du graphe.
	Une structure assurant cette condition est un arbre.

	Pour résoudre ce problème, on part du graphe vide, et on ajoute les arrêtes les moins coûteuses en premier.
\end{exm}

\begin{defn}[Arbre]
	Soit $G = (S,A)$\/ un graphe non-orienté. On dit que $G$\/ est un \textit{arbre} si $G$\/ est connexe et acyclique.
	\index{arbre}
\end{defn}

\begin{defn}[Arbre couvrant]
	Étant donné un graphe non orienté pondéré par poids positifs $G = (S, A, c)$,\footnotemark\ on dit de $G' = (S', A')$\/ que c'est un \textit{arbre couvrant} de $G$\/ si $S' = S$\/ et $A' \subseteq A$, et $G'$\/ est un arbre.
	\index{arbre!couvrant}
\end{defn}
\footnotetext{on dit que $c$\/ est la fonction de pondération de ce graphe}

\begin{defn}[Arbre couvrant de poids minimum]
	Étant donné un graphe non orienté pondéré $G = (S, A, c)$\/ et un arbre couvrant $T = (S', A')$, on appelle \textit{poids} de l'arbre $T$\/ la valeur $\sum_{a \in A'} c(a)$.
	\index{arbre!couvrant!poids}

	Si $G$\/ est connexe, il admet au moins un arbre couvrant, on peut définir l'\textit{arbre couvrant de poids minimum} (\textit{\textsc{acpm}}).
	\index{arbre!couvrant!de poids minimum}
\end{defn}

On définir alors le problème \[
	\textsc{acpm}\text{\footnotemark}
	\begin{cases}
		\text{\textbf{Entrée}}&: G = (S, A, c) \text{ connexe}\\
		\text{\textbf{Sortie}}&: \text{ le poids de l'arbre couvrant de poids minimum}.
	\end{cases}
\]
\footnotetext{Arbre Couvrant de Poids Minimum}

\begin{algorithm}[H]
	\centering
	\begin{algorithmic}[1]
		\Entree $G = (S, A, c)$\/ un graphe connexe
		\Sortie Un arbre couvrant de poids minimum
		\State $B \gets \O$\/ 
		\State $U \gets \O$\/
		\While{il existe $u$\/ et $v$\/ tels que $u \nsim_B v$}
			\State Soit $\{x,y\} \in A \setminus U$\/ de poids minimal
			\If{$x \sim_B y$}
				\State $U \gets \big\{\!\{x,y\}\!\big\} \cup U$
			\Else
				\State $U \gets \big\{\!\{x,y\}\!\big\} \cup U$
				\State $B \gets \big\{\!\{x,y\}\!\big\} \cup B$
			\EndIf
		\EndWhile
		\State\Return $T = (S,B)$\/
	\end{algorithmic}
	\caption{Algorithme de \textsc{Kruskal}}
\end{algorithm}

\begin{prop}
	L'algorithme de \textsc{Kruskal} est correct.
\end{prop}

\begin{prv}
	\begin{enumerate}
		\item Il existe un arbre couvrant de poids minimum utilisant les arrêtes de $B$ ;
		\item $B \subseteq U \subseteq A$\/ ;
		\item $\forall \{u,v\} \in U$, $u \sim_B v$.
	\end{enumerate}
	Ces trois propriétés sont invariantes.
	\begin{description}
		\item[Initialement] $B = \O = U$, donc \textsc{ok}.
		\item[Propagation] Soient $\ubar{B}$\/ et $\ubar{U}$\/ (resp.\ $\bar{B}, \bar{U}$) les valeurs de $B$\/ et $U$\/ avant (resp.\ après) une itération de boucle. Supposons que $\ubar{B}$\/ et $\ubar{U}$\/ satisfont les propriétés 1, 2 et 3. Montrons que $\bar{B}$\/ et $\bar{U}$\/ les satisfont aussi.
			\begin{enumerate}
				\item[2.] On a $\{x,y\} \in A$\/ et $\ubar{B} \subseteq \ubar{U} \subseteq A$, donc \[
						\bar{B} \subseteq \ubar{B} \cup \{\!\{x,y\}\!\} \subseteq \ubar{U} \cup \{\!\{x,y\}\!\} \subseteq A
					.\]
				\item[3.] Soit $\{u,v\} \in \bar{U}$.
					\begin{itemize}
						\item Si $\{u,v\} \in \ubar{U}$, alors de 3, $u \sim_{\ubar{B}} v$. Or, $\ubar{B} \subseteq \bar{B}$\/ et donc $u \sim_{\bar{B}} v$.
						\item Sinon, $\{ u,v\} = \{x,y\}$, alors $x = u$\/ et $v = y$.
							\begin{itemize}
								\item Sous-cas 1 : $\bar{B} = \ubar{B} \cup \{\!\{x,y\}\!\}$, alors $x \sim_{\bar{B}} y$.
								\item Sous-cas 2 : $\bar{B} = \ubar{B}$, alors par condition du \textbf{si}, $x \sim_{\ubar{B}} y$\/ et donc $x \sim_{\bar{B}} y$.
							\end{itemize}
					\end{itemize}
				\item[1.]
					Soit $\mathcal{T}$\/ un \textsc{acpm} contenant $\ubar{B}$.
					\begin{itemize}
						\item Cas 1 : $\bar{B} = \ubar{B}$, \textsc{ok}
						\item Cas 2 : $\bar{B} = \ubar{B} \cup \{\!\{x,y\}\!\}$.
							\begin{itemize}
								\item Sous-cas 1 : $\{x,y\} \in \mathcal{T}$, alors $\mathcal{T}$\/ est un \textsc{acpm} qui contient $\bar{B}$.
								\item Sous-cas 2 : $\{x,y\}  \not\in \mathcal{T}$, $\mathcal{T}$\/ est un arbre couvrant, donc il contient une chaîne de $x$\/ à $y$\/ : \[
											\{\overset{\substack{x\\[-1mm]\vrt=}}{x_0},x_1\},\{x_1,x_2\},\ldots,\{x_{n-1},\underset{\substack{\vrt=\\y}}{x_n}\}
									.\]
									Or, $\forall i \in \llbracket 1,n-1 \rrbracket$, $x_i \sim_{\ubar{B}} x_{i+1}$. Par transitivité, on a donc $x = x_0 \sim_{\ubar{B}} x_n = y$, ce qui n'est pas le cas.
									Il existe donc $i_0 \in \llbracket 0,n-1 \rrbracket$, tel que $x_{i_0} \nsim_{\ubar{B}} x_{i_0 + 1}$\/ et donc $\{x_{i_0}, x_{i_0 + 1}\} \not\in \ubar{U}$. D'où, d'après 3, on a $\{x_{i_0}, x_{i_0 + 1}\}  \not\in \ubar{B}$
									Considérons alors $\mathcal{T}' = \big(\mathcal{T} \setminus \{\!\{x_{i_0},x_{i_0+1}\}\!\}\big)  \cup \{\!\{x,y\}\!\}$. Montrons que $\mathcal{T}'$\/ est un \textsc{acpm} contenant $B$, en commençant par montrer que c'est un arbre couvrant. L'arbre $\mathcal{T}'$\/ a $n-1$\/ arrêtes (autant que $\mathcal{T}$). Montrons que $\mathcal{T}'$\/ est connexe.
									Soit $(a,b) \in S^2$. $\mathcal{T}$\/ est connexe, soit donc une chaîne \[
										C : \quad a = u_0, u_1, \ldots, u_n = b
									\] de $\mathcal{T}$. Si la chaîne $C$\/ n'utilise pas l'arrête $\{x_{i_0},x_{i_0+1}\}$, alors $C$\/ est une chaîne de $\mathcal{T}'$. Sinon, on pose $\mu$\/ et $\tau$\/ tels que \[
									\underbrace{a,\ldots,x_{i_0}}_{\mu},\underbrace{x_{i_0+1},\ldots,b}_{\tau}
									.\]
									Soit alors la chaîne
									\begin{align*}
										\overbrace{a,\ldots,x_{i_0}}^{\mu},x_{i_0-1},x_{i_0-2},\ldots,x_0 = x,&\\
										\underbrace{b,\ldots,x_{i_0 + 1}}_{\tau},x_{i_0+2},\ldots,x_{n-1},x_n=y&
									\end{align*}
									qui est dans $\mathcal{T}'$.
									Montrons que le poids est minimum. Notons $P(\mathcal{T})$\/ le poids de l'arbre. On a donc \[
										P(\mathcal{T}') = P(\mathcal{T}) + c(\{x,y\}) - c(\{x_{i_0},x_{i_0+1}\})
									.\] Par choix glouton, ($\{x_{i_0}, x_{i_0+1} \not\in \ubar{U}\}$), $c(\{x,y\}) \le c(\{x_{i_0},x_{i_0+1}\})$\/ donc $P(\mathcal{T}') \le P(\mathcal{T})$, et $\mathcal{T}$\/ étant de poids min, $P(\mathcal{T}') = P(\mathcal{T})$\/ et $\mathcal{T}'$\/ est un \textsc{acpm} contenant $\bar{B}$.
							\end{itemize}
					\end{itemize}
			\end{enumerate}
	\end{description}

	Les invariants le sont.
\end{prv}

À la fin, $B$\/ induit un graphe connexe et $B$\/ est contenu dans un \textsc{acpm}, c'en est donc un.


		\paragraph{Une structure pour la gestion des partitions : \textsf{UnionFind}.}

\begin{defn}[Type de données abstrait \textsf{UnionFind}]
	On définit le type de données abstrait \textsf{UnionFind} comme contenant
	\begin{itemize}
		\item un type \texttt{t} de partitions ;
		\item un type \texttt{elem} des éléments manipulés par les partitions ;
		\item $\texttt{initialise\_partition} : \texttt{elem list} \to \texttt{t}$\/ retournant le partitionnement dans lequel chaque élément est seul dans sa classe ;
		\item $\texttt{find} : (\texttt{t} \mathbin{\texttt{*}} \texttt{elem}) \to \texttt{elem}$\/ retournant un représentant de la classe de l'élément. Si deux éléments $x$\/ et $y$\/ sont dans la même classe, dans le partitionnement $p$, alors $\texttt{find}(p,x) = \texttt{find}(p,y)$ ;
		\item $\texttt{union} : (\texttt{t} \mathbin{\texttt{*}} \texttt{elem} \mathbin{\texttt{*}} \texttt{elem}) \to \texttt{t}$\/ retourne le partitionnement dans lequel on a fusionné les classes des arguments.
	\end{itemize}
	\index{type \textsf{UnionFind}}
\end{defn}

\begin{exm}
	On réalise le \textit{pseudo-code} ci-dessous.
	\begin{itemize}
		\item $p \gets \texttt{initialise\_partition}([1, 2, 3, 4, 5])$\/ $\leadsto$ $\{\{1\}, \{2\}, \{3\}, \{4\}, \{5\}\}$
		\item $\texttt{find}(p, 1) = 1$\/ 
		\item $\texttt{union}(p, 1, 3)$\/ $\leadsto$ $\{\{1,3\}, \{2\}, \{4\}, \{5\}\}$
		\item $\texttt{find}(p, 1) = \texttt{find}(p, 3)$
	\end{itemize}
\end{exm}

On implémente ce type abstrait en \textsc{OCaml}.

\begin{rmk}[Niveau zéro -- listes de liste]~
	\begin{lstlisting}[language=caml,caption=Implémentation du type \textsf{UnionFind} en \textsc{OCaml}]
type 'a t = 'a list list

let initialise_partition (l: 'a list): 'a t =
	List.map (fun x -> [ x ] ) l

let rec find (p: 'a t) (x: 'a): 'a =
	match p with
	| classe :: classes ->
			if List.mem x classe then List.hd classe
			else find classes x
	| [] -> raise Not_Found

let est_equiv (p: 'a t) (x: 'a) (y: 'a): bool = 
	(find p x) = (find p y)

let rec extrait_liste (x: 'a) (p: 'a t): 'a list * 'a p =
	match p with
	| classe :: classes ->
			if List.mem x classe then (classe, classes)
			else
				let cl, cls' = extrait_liste x classes in
				(cl, classe :: cls')
	| [] -> raise Not_Found

let union (p: 'a t) (x: 'a) (y: 'a): 'a t =
	if est_equiv p x y then p
	else
		let cx, p' = extrait_liste x p in
		let cy, p'' = extrait_liste y p' in
		(cx @ cy) :: p''
	\end{lstlisting}
\end{rmk}

\begin{rmk}[Niveau un -- tableau de classes]
	Dans la case du tableau, on inscrit le numéro de sa classe.
	Pour \texttt{find}, on prend le premier ayant la même classe.
	Pour \texttt{union}, on re-numérote vers un numéro commun.
	Par exemple, \[
		\begin{array}{|c|c|c|c|c|c|}
			\hline
			0 & 1 & 0 & 0 & 1 & 2\\ \hline
			0 & 1 & 2 & 3 & 4 & 5 \\ \hline
		\end{array}\quad\quad\longleftrightarrow\quad\quad\{\{0,2,3\},\{1,4\},\{5\}\}
	.\]
\end{rmk}

\begin{rmk}[Niveau deux -- tableau de représentants]
	Dans les cases du tableau, on écrit le représentant de la classe de $i$.
	Pour \texttt{find}, on lit la case.
	Pour \texttt{union}, on re-numérote vers un numéro commun.
	Par exemple, \[
		\begin{array}{|c|c|c|c|c|c|}
			\hline
			2 & 4 & 2 & 2 & 4 & 5\\ \hline
			0 & 1 & 2 & 3 & 4 & 5 \\ \hline
		\end{array}\quad\quad\longleftrightarrow\quad\quad\{\{0,2,3\},\{1,4\},\{5\}\}
	.\]
\end{rmk}

\begin{rmk}[Niveau trois -- arbres]
	Pour $\texttt{union}(0, 1)$, on cherche le représentant de 0 (2) puis celui de 1 (4). On fait pointer 4 vers 2.
	Pour la suite de l'implémentation, \textit{c.f.}\ \textsc{dm}$_3$.

	\begin{figure}[H]
		\centering
		\tikzfig{ex-unionfind-arbres}
		\caption{Représentation par des arbres}
	\end{figure}
\end{rmk}

Avec cette nouvelle structure, on peut maintenant revenir sur l'algorithme de \textsc{Kruskal}.

\begin{algorithm}[H]
	\centering
	\begin{algorithmic}[1]
		\Entree Un graphe $G = (S, A, c)$\/ un graphe non orienté, pondéré
		\Sortie Un \textsc{acpm}
		\State Soit $(e_i)_{i\in\llbracket 1,m \rrbracket}$\/ un tri des arrêtes par coût croissant
		\State $f \gets 0$\/ \Comment{Nombre d'\textsl{\texttt{union}} effectuées}
		\State $p \gets \texttt{initialise\_partition}(S)$\/ 
		\State $I \gets 0$\/ 
		\State $B \gets \O$\/ 
		\While{$f < n - 1$}
			\State $\{x,y\} \gets e_I$\/ 
			\If{$\texttt{find}(p, x) \neq \texttt{find}(p, y)$}
				\State $p \gets \texttt{union}(p, x, y)$\/ 
				\State $B \gets B \cup \{\!\{x,y\}\!\}$\/ 
				\State $f \gets f + 1$\/ 
			\EndIf
			\State $I \gets I + 1$\/
		\EndWhile
		\State\Return $(S,B)$
	\end{algorithmic}
	\caption{Algorithme de \textsc{Kruskal} -- version 2}
\end{algorithm}

\paragraph{Étude de complexité.}
Notons $C_{\texttt{find}}^n$\/ un majorant du coût de \texttt{find} sur une structure contenant $n$\/ éléments, notons $C_{\texttt{union}}^n$\/ un majorant du coût de \texttt{union} sur une structure contenant $n$\/ éléments, et notons $C_{\texttt{init}}^n$\/ un majorant du coût de \texttt{init} sur une structure contenant $n$\/ éléments.
La complexité de cet algorithme est de \[
	\mathcal{O}\big(C_{\texttt{init}}^n + 2m\:C_{\texttt{find}}^n + n\: C_{\texttt{union}}^n + m \log_2 m\big)
.\]

\section{Couplage dans un graphe biparti}

\begin{defn}[Couplage]
	On appelle \textit{couplage} d'un graphe non orienté $G = (S, A)$, la donnée d'un sous-ensemble $C \subseteq A$\/ tel que \[
		\forall \{x,y\}, \{x',y'\} \in C,\:
		\quad\quad \{x,y\} \cap \{x',y'\} \neq \O
		\implies
		\{x,y\}  = \{x', y'\}
	.\]

	\index{graphe!couplage}
\end{defn}

\begin{figure}[H]
	\centering
	\tikzfig{ex-couplage}
	\caption{Exemple de couplage}
\end{figure}

\begin{exm}
	On réutilise l'exemple ci-dessous dans toute la section.
	L'ensemble $C = \{\{a,2\}, \{b,3\}\}$\/ est un couplage.
	Mais, l'ensemble $C' = \{\{a,1\}, \{a,2\}\}$\/ n'en est pas un.
\end{exm}

\begin{defn}
	Un couplage est dit \textit{maximal} s'il est maximal pour l'inclusion ($\subseteq$).
	Un couplage est dit \textit{maximum} si son cardinal est maximal.
	\index{graphe!couplage!maximal}
	\index{graphe!couplage!maximum}
\end{defn}

\begin{exm}
	Dans l'exemple précédent, 
	\begin{itemize}
		\item le couplage $C = \{\{a,2\}, \{b,3\}\}$\/ n'est ni maximal, ni maximum ;
		\item le couplage $C' = \{\{a,2\}, \{b, 3\}, \{d, 4\}\}$\/ est maximal mais pas maximum ;
		\item le couplage $C'' = \{\{a,1\}, \{b,3\}, \{c,2\}, \{d,4\}\}$\/ est maximum.
	\end{itemize}
\end{exm}

\begin{rmk}
	Dans toute la suite, on ne considère que des graphes bipartis.
\end{rmk}

\begin{defn}
	Étant donné un graphe biparti $G = (S, A)$\/ et un couplage $C$, un sommet $x$\/ est dit \textit{libre} dès lors que \[
		\forall \{y,z\} \in C,\: x \not\in \{y,z\}
	.\]
	Une chaîne élémentaire\footnotemark $(c_0, c_1, \ldots, c_{2p+1})$\/ est dit \textit{augmentante} si
	\begin{itemize}
		\item $c_0$\/ et $c_{2n+1}$\/ sont libres ;
		\item $\forall i \in \llbracket 0,p \rrbracket$, $\{c_{2i}, c_{2i+1}\}\in A \setminus C$ ;
		\item $\forall i \in \llbracket 0,p-1 \rrbracket$, $\{c_{2i+1}, c_{2i+2}\} \in C$.
	\end{itemize}
\end{defn}
\footnotetext{\textit{i.e.}\ une chaîne sans boucles.}

\begin{exm}~
	\begin{figure}[H]
		\centering
		\tikzfig{ex-chaine-augmentante}
		\caption{Chaîne augmentante}
	\end{figure}
\end{exm}

\begin{exm}
	Dans l'exemple de cette section, $(d, 4)$\/ et $(c, 2, a, 1)$\/ sont deux chaînes augmentantes.
\end{exm}

\begin{prop}
	Étant donné un graphe biparti $G = (S, A)$\/ avec $S = S_1 \cupdot S_2$ (partitionnement du graphe biparti), un couplage $C$\/ est maximum si, et seulement s'il n'admet pas de chaînes augmentantes.
\end{prop}

\begin{prv}
	\begin{itemize}
		\item[``$\implies$'']
			Soit $C$\/ un couplage admettant une chaîne augmentante. Montrons que $C$\/ n'est pas maximum.
			Soit la chaîne augmentante\footnotemark \[
				c_0 \to c_1 \Rightarrow c_2 \to c_3 \Rightarrow c_4 \to \cdots \to c_{2p - 1} \Rightarrow c_{2p} \to c_{2p+1}
			.\]
			On considère alors le couplage \[
				C' = \Big(C \setminus \big\{\{c_{2i+1},c_{2i+2} \} \mid i \in \llbracket 0,p-1 \rrbracket\big\}\Big) \cup \big\{\{c_{2i},c_{2i+1}\}  \mid i \in \llbracket 0,p \rrbracket\big\}
			.\]
			On transforme donc la chaîne en \[
				c_0 \Rightarrow c_1 \to c_2 \Rightarrow c_3 \to \cdots \to c_{2p-1} \to c_{2p} \Rightarrow c_{2p+1}
			.\]
			C'est bien un couplage, et $\Card(C') = \Card C + 1$. $C$\/ n'est donc pas un couplage maximum.
		\item[``$\impliedby$'']
			Soit $C$\/ un couplage non maximum. Montrons que $C$\/ admet une chaîne augmentante. Soit $M$\/ un couplage maximum, et $D = C \mathrel{\triangle} M = (C \setminus M) \cupdot (M \setminus C)$.
			On a $\Card C < \Card M$\/ et $\Card(C \setminus M) < \Card(M \setminus C)$.
			On remarque que, si $c_0 \to c_1 \to c_2 \to \cdots \to c_{p-1}\to c_p$\/ est une chaîne de $D$, (si $c_0 \to c_1 \in C \setminus M$\/ et $c_1 \to c_2 \in C \setminus M$\/ donc $c_1$\/ est dans deux arrêtes distinctes d'un couplage $C$, ce qui est absurde ; de même pour les autres arrêtes). Ainsi, 2 arrêtes consécutives ne sont pas dans la même composante de l'union $(C \setminus M) \cupdot (M \setminus C)$.
			Considérons la relation d'équivalence $\sim$\/ sur $D$\/ définie par $\{x,y\} \sim \{z,t\} \iffdef$ il existe une chaîne de $D$\/ utilisant l'arrête $\{x,y\}$\/ et l'arrête $\{z,t\}$.
			Soit le partitionnement $D_1, \ldots, D_q$\/ de $D$\/ par $\sim$.
			Par inégalité de cardinal, il existe un $D_i$\/ tel que \[
				\Card \{e \in D_i  \mid e \in C\} < \Card \{e \in D_i  \mid e \in M\}
			.\] L'ensemble $D_i$\/ contient alors une chaîne augmentante.
	\end{itemize}
\end{prv}
\footnotetext{On représente $\Rightarrow$ pour les arrêtes dans le couplage $C$.}

		\addrecap
	}
	\def\addmacros#1{#1}
}

{
	\chap[1]{Langages réguliers et Automates}
	\minitoc
	\renewcommand{\cwd}{../cours/chap01/}
	\addmacros{
		\section{(Ne pas) être diagonalisable}

\begin{defn}
	Soit une matrice carrée $A$. On dit que $A$\/ est {\it diagonalisable}\/ s'il existe une matrice inversible~$P \in \mathrm{GL}_n(\mathds{K})$\/ telle que $P^{-1}\cdot A\cdot P$\/ est diagonale.
\end{defn}

\begin{exo}
	\begin{enumerate}
		\item Montrons que la matrice $B = {7\: 1\choose 0\:7}$\/ n'est pas diagonalisable.
			Par l'absurde : on suppose qu'il existe $P \in \mathrm{GL}_2(\R)$\/ et $(\lambda_1, \lambda_2) \in \R^2$\/ tels que \[
				P^{-1} \cdot B \cdot P = \begin{bmatrix}
					\lambda_1 & 0\\
					0&\lambda_2
				\end{bmatrix}
			.\] On applique la trace $\tr$\/ et le déterminant $\det$\/ :
			\begin{gather*}
				\tr(B) = \tr{\lambda_1\:0\choose 0\:\lambda_2} \quad\text{d'où}\quad \lambda_1 + \lambda_2 = 7 + 7 = 14 = \s\\
				\det(B) = \det{\lambda_1\:0\choose 0\:\lambda_2} \quad\text{d'où}\quad \lambda_1 \times \lambda_2 = 7 \times 7 = 49 = p
			\end{gather*}
			D'où $\lambda_1$\/ et $\lambda_2$\/ sont des solutions de l'équation $X^2 - \s X + p = 0$. Or
			\begin{align*}
				X^2 - \s X + p = 0 \iff& X^2 - 14X + 49 = 0\\
				\iff& (X-7)^2 = 0\\
				\iff& X = 7.
			\end{align*}
			D'où 
			\begin{align*}
				B = P P^{-1} B P P^{-1} = P \begin{pmatrix}
					7&0\\
					0&7
				\end{pmatrix} P^{-1} = P \cdot 7I_2\cdot P^{-1} = 7I_2.
			\end{align*}
			La matrice $B$\/ n'est donc pas diagonalisable.

			De même, montrons que la matrice $A$\/ n'est pas diagonalisable. On remarque que \[
				A \cdot \mat{1\\1\\1} = \begin{pmatrix}
					0&1&2\\
					1&0&2\\
					0&0&3
				\end{pmatrix} \begin{pmatrix}
					1\\1\\1
				\end{pmatrix} = \begin{pmatrix}
					3\\3\\3
				\end{pmatrix} = 3\begin{pmatrix}
					1\\1\\1
				\end{pmatrix} 
			.\] Ainsi, \[
				P^{-1}\cdot A\cdot P = \begin{pmatrix}
					3&0&0\\
					0&?&0\\
					0&0&?
				\end{pmatrix}\qquad\text{où}\qquad P = \begin{pmatrix}
					1&?&?\\
					1&?&?\\
					1&?&?
				\end{pmatrix}
			.\] De même, $A\left( \substack{1\\1\\0} \right) = 1 \times \left( \substack{1\\1\\0} \right)$. D'où \[
				P^{-1}\cdot A\cdot P = \begin{pmatrix}
					3&0&0\\
					0&1&0\\
					0&0&?
				\end{pmatrix}\qquad\text{où}\qquad P = \begin{pmatrix}
					1&1&?\\
					1&1&?\\
					1&0&?
				\end{pmatrix}
			.\] Finalement, on en conclut que \[
				P = \begin{pmatrix}
					3&0&0\\
					0&1&0\\
					0&0&-1
				\end{pmatrix} \qquad \text{et}\qquad P^{-1}\cdot A\cdot P = \begin{pmatrix}
					1&1&1\\
					1&1&-1\\
					1&0&0
				\end{pmatrix} = D
			.\]
			De plus, la matrice $P$\/ est inversible car $\det P \neq 0$.
		\item Pour calculer $A^n$, on pourrait chercher un polynôme annulateur $Q$\/ de $A$, et on exprime $X^n = Q \times T_n + R_n$, et donc $A^n = R_n(A)$.
			Mais, on peut également diagonaliser $A$\/ (si elle est diagonalisable).
			Ainsi,  \[
				D^n = (P^{-1}\cdot A\cdot P)^n = P^{-1}\cdot A\cdot \cancel P\cdot \cancel{P^{-1}} \cdot \ldots\cdot \cancel{P^{-1}} \cdot A \cdot P = P^{-1}\cdot  A^n\cdot P
			.\] D'où $A^n = P \cdot D^n \cdot P^{-1}$. Or, \[
				D^n = \begin{pmatrix}
					3&0&0\\
					0&1&0\\
					0&0&-1
				\end{pmatrix}^n = \begin{pmatrix}
					3^n&0&0\\
					0&1^n&0\\
					0&0&(-1)^n
				\end{pmatrix}
			.\]
			On calcule donc $A^{n}$\/ en calculant l'inverse de $P$\/ : \[
				A^n = \begin{pmatrix}
					1&1&1\\
					1&1&-1\\
					1&0&0
				\end{pmatrix} \begin{pmatrix}
					3^n&0&0\\
					0&1^n&0\\
					0&0&(-1)^n
				\end{pmatrix} \cdot P^{-1}
			.\]
		\item
			\begin{align*}
				\begin{rcases*}
					\hfill u_{n+1} = v_n + 2w_n\\
					\hfill v_{n+1} = u_n + 2w_n\\
					\hfill w_{n+1} = 3w_n
				\end{rcases*} \iff& \begin{pmatrix}
					u_{n+1}\\v_{n+1}\\w_{n+1}
				\end{pmatrix} = \begin{pmatrix}
					0&1&2\\
					1&0&2\\
					0&0&3
				\end{pmatrix} \begin{pmatrix}
					u_n\\ v_n\\ w_n
				\end{pmatrix}\\
				\iff& U_{n+1} = A\cdot U_n\\
				\iff& U'_{n+1} = D \cdot U'_{n}
			\end{align*}
			où $D = P^{-1} \cdot A \cdot P$, $U'_{n+1} = P\cdot U_{n+1}$\/ et $U'_n = P\cdot U_n$.
			\begin{align*}
				\phantom{\begin{rcases*}
					\hfill mm_{n+1} = v_n + 2w_n\\
					\hfill v_{n+1} = u_n + 2w_n\\
					\hfill w_{n+1} = 3w_n
				\end{rcases*}} \iff&
				\begin{pmatrix}
					u'_{n+1}\\v'_{n+1}\\w'_{n+1}
				\end{pmatrix} = \begin{pmatrix}
					3&0&0\\
					0&1&0\\
					0&0&-1
				\end{pmatrix} \cdot \begin{pmatrix}
					u'_n\\
					v'_n\\
					w'_n
				\end{pmatrix}\\
				\iff& \begin{cases}
					u'_{n+1} = 3u'_n\\
					v'_{n+1} = v'_n\\
					w'_{n+1} = -w'_n
				\end{cases}\\
				\iff& \begin{cases}
					u'_n = K\times  3^n\\
					v'_n = L\\
					w'_n = M \times (-1)^n
				\end{cases}
			\end{align*}
			Ainsi, \[
				\begin{pmatrix}
					u_n\\v_n\\w_n
				\end{pmatrix} = \underbrace{\begin{pmatrix}
					1&1&1\\
					1&1&-1\\
					1&0&0
				\end{pmatrix}}_P \cdot \begin{pmatrix}
					K\times 3^n\\
					L\\
					M\times (-1)^n
				\end{pmatrix}
			.\] D'où $u_n = K\cdot 3^n + L + M \cdot (-1)^n$, $v_n = K\times 3^n + L - M \cdot (-1)^n$\/ et $w_n = K\cdot 3^n$, où les constantes $K$, $L$\/ et $M$\/ sont des constantes fixées par les conditions initiales.
		\item
			\begin{align*}
				\begin{rcases*}
					\hfill x'(t) = y(t) + 2z(t)\\
					\hfill y'(t) = x(t) + 2z(t)\\
					\hfill z'(t) = 3z(t)
				\end{rcases*} \iff& \begin{pmatrix}
					x'(t)\\
					y'(t)\\
					z'(t)
				\end{pmatrix} = \begin{pmatrix}
					0&1&2\\
					1&0&2\\
					0&0&3
				\end{pmatrix} \cdot \begin{pmatrix}
					x(t)\\
					y(t)\\
					z(t)
				\end{pmatrix}\\
				\iff& X'(t) = A\cdot X(t)\\
				\iff& U'(t) = D \cdot U(t) \text{ avec } D = P^{-1} \cdot A\cdot P \text{ et } X(t) = P\cdot U(t)\\
				\iff& \begin{pmatrix}
					u'(t)\\
					v'(t)\\
					w'(t)
				\end{pmatrix} = \begin{pmatrix}
					3&0&0\\
					0&1&0\\
					0&0&-1
				\end{pmatrix} \cdot \begin{pmatrix}
					u(t)\\
					v(t)\\
					w(t)
				\end{pmatrix}\\
				\iff& \begin{cases}
					u'(t) = 3u(t)\\
					v'(t) = v(t)\\
					w'(t) = -w(t)
				\end{cases}\\
				\iff& \begin{cases}
					u(t) = K \cdot \mathrm{e}^{3t}\\
					v(t) = L \cdot \mathrm{e}^{t}\\
					w(t) = M \cdot \mathrm{e}^{-t}
				\end{cases}
			\end{align*}
			Ainsi \[
				\begin{pmatrix}
					x(t)\\
					y(t)\\
					z(t)
				\end{pmatrix} = \underbrace{\begin{pmatrix}
					1&1&1\\
					1&1&-1\\
					1&0&0
				\end{pmatrix}}_P \cdot \begin{pmatrix}
					K \times \mathrm{e}^{3t}\\
					L \cdot \mathrm{e}^{t}\\
					M \cdot \mathrm{e}^{-t}
				\end{pmatrix}
			.\] 
			D'où $x(t) = K\cdot \mathrm{e}^{3t} + L \cdot \mathrm{e}^{t} + M \cdot \mathrm{e}^{-t}$, $y(t) = K \cdot \mathrm{e}^{3t} + L \cdot \mathrm{e}^{t} - M \cdot \mathrm{e}^{-t}$\/ et $z(t) = K\cdot \mathrm{e}^{3t}$. Les constantes $K$, $L$\/ et $M$\/ peuvent être déterminées à partir des conditions initiales.
	\end{enumerate}
\end{exo}

\begin{rmkn}[équations différentielles]
	On considère l'équation différentielle $(*)$ : $x'(t) = \lambda \cdot x(t)$.
	Les fonctions $x : t \mapsto K\cdot \mathrm{e}^{\lambda t}$\/ sont des solutions de cette équation. On peut utiliser la méthode de {\sc Lagrange}\/ : la méthode de la~\guillemotleft~variation de la constante.~\guillemotright\@ On cherche des solutions sous la forme $x(t) = k(t) \cdot \mathrm{e}^{\lambda t}$ (vision du~physicien). D'où $k(t) = x(t) / \mathrm{e}^{\lambda t}$\/ (vision du mathématicien). De plus, $x'(t) = k'(t) \mathrm{e}^{\lambda t} + k(t) \lambda \mathrm{e}^{\lambda t}$.
	Ainsi, on injecte ce $k(t)$\/ dans l'équation différentielle :
	\begin{align*}
		(*) \iff& k'(t) \mathrm{e}^{\lambda t} + k(t) \lambda \mathrm{e}^{\lambda t} = \lambda k(t)\mathrm{e}^{\lambda t}\\
		\iff& k'(t) \mathrm{e}^{\lambda t} = 0\\
		\iff& k'(t) = 0\\
		\iff& \exists K \in \R\,\:k(t) = K.
	\end{align*}
	Les solutions trouvées dans l'exercice précédent sont donc les uniques solutions du système d'équations différentielles.

	De même, pour résoudre une équation différentielle avec 2\tsup{nd} membre de la forme \[
		(**) : \qquad x'(t) - \lambda \cdot x(t) = b(t)
	.\]
	La fonction $t \mapsto x(t)$\/ est une solution de l'équation {\sc sans}\/ 2\tsup{nd} membre si et seulement si \[
		\exists K \in \R,\:\forall t \in \R,\quad x(t) = K \cdot \mathrm{e}^{\lambda t}
	.\]
	\begin{center}
		\slshape Comment résoudre l'équation différentielle {\scshape avec}\/ 2\tsup{nd} membre si on connaît la solution générale de l'équation {\scshape sans}\/ 2\tsup{nd} membre ?
	\end{center}
	On utilise la méthode le la variation de la constante.
	Soit $x(t) = k(t) \cdot \mathrm{e}^{\lambda t}$. Ainsi, en injectant cette expression de $x$\/ dans l'équation $(**)$, on trouve
	\begin{align*}
		(**) \iff& k'(t) \mathrm{e}^{\lambda t} + k(t) \cdot \lambda \mathrm{e}^{\lambda t} = \lambda k(t) \mathrm{e}^{\lambda t} + b(t)\\
		\iff& k'(t) \mathrm{e}^{\lambda t} = b(t)\\
		\iff& k'(t) = b(t) \cdot \mathrm{e}^{-\lambda t}\\
		\iff& k(t) = \int_{0}^{t} b(u)\cdot \mathrm{e}^{-\lambda u}~\mathrm{d}u + K\\
		\iff& x(t) = \left( \int_{0}^{t} b(u) \cdot \mathrm{e}^{-\lambda u}~\mathrm{d}u + K \right) \mathrm{e}^{\lambda t}\\
		\iff& x(t) = \underbrace{\int_{0}^{t} b(u) \cdot \mathrm{e}^{\lambda (t-u)}~\mathrm{d}u}_{\text{solution particulière}} + \underbrace{K \cdot \mathrm{e}^{\lambda t}}_{\substack{\text{solution}\\\text{générale}\\\text{de $(*)$}}}.
	\end{align*}
\end{rmkn}

		\begin{exm}
	On pose $f$, le sinus cardinal :  \begin{align*}
		f: \R^* &\longrightarrow \R \\
		t &\longmapsto \frac{\sin t}{t}.
	\end{align*}
	\begin{figure}[H]
		\centering
		\begin{asy}
			import graph;
			size(10cm);
			draw((-10, 0) -- (10, 0), Arrow(TeXHead));
			draw((0, -3) -- (0, 5), Arrow(TeXHead));
			real f(real x) {
				if(x == 0) { return 3; }
				else {return 3*sin(x) / x;}
			}
			draw(graph(f, -10, 10), magenta);
		\end{asy}
		\caption{Sinus cardinal}
	\end{figure}

	La fonction $f$\/ est continue sur ${]0,8]}$\/ mais $\lim_{t\to 0} \frac{\sin t}{t} = 1$. D'où $\int_{0}^{8} \frac{\sin t}{t}~\mathrm{d}t$\/ est faussement impropre en $0$\/ et donc convergente.


	Mais attention ! On ne dit pas \guillemotleft~{\color{red}soit $f : t \mapsto \frac{1}{t}$. L'intégrale $\int_{8}^{+\infty} \frac{1}{t}~\mathrm{d}t$\/ est faussement impropre en $+\infty$\/ car $\lim_{t\to +\infty}\frac{1}{t} = 0$}.~\guillemotright
\end{exm}

\section{Intégrer les $\mathbf{\sim}$, $\po$, et \textit{O}}

\begin{thm}
	\hfill$\O$\hfill\null
\end{thm}

\begin{thm}
	Le 2.\ n'est pas la réciproque du 1.\ mais la contraposée.
\end{thm}

\begin{prop}
	\hfill$\O$\hfill\null
\end{prop}

\begin{exm}
	On considère l'intégrale $\int_{2}^{+\infty} \frac{1}{t^2+ \cos t}~\mathrm{d}t$, c'est une intégrale impropre en $+\infty$.
	On recherche un équivalent de $\frac{1}{t^2 + \cos t}$\/ en $+\infty$ : \[
		\frac{1}{t^2 + \cos t} \simi_{t\to +\infty} \frac{1}{t^2}
	\] qui ne change pas de signe. Or, $\int_{2}^{+\infty} \frac{1}{t^2}~\mathrm{d}t$\/ converge car c'est une intégrale de {\sc Riemann}\/ avec $\alpha = 2 > 1$.
	On en déduit que l'intégrale $I$\/ converge.

	On procède autrement : \[
		0 \le \frac{1}{t^2 + \cos t} \le \frac{1}{t^2 - 1}
	.\] Or, $\int_{2}^{+\infty} \frac{1}{t^2 - 1}~\mathrm{d}t$\/ converge car
	\begin{align*}
		\int_{2}^{x} \frac{1}{t^2 - 1}~\mathrm{d}t &= \int_{2}^{x} \left( \frac{\sfrac12}{t-1} - \frac{\sfrac12}{t+1} \right) ~\mathrm{d}t \\
		&= \frac{1}{2} \int_{2}^{x} \frac{1}{t-1}~\mathrm{d}t - \frac{1}{2}\int_{2}^{x} \frac{1}{t+1}~\mathrm{d}t \\
		&= \frac{1}{2} \Big[\ln|t-1|\Big]_2^x - \frac{1}{2}\Big[\ln |t+1|\Big]_2^x \\
	\end{align*}
	D'où \[
		\int_{2}^{x} \frac{1}{t^2 - 1}~\mathrm{d}t = \frac{1}{2} \left[ \ln\left| \frac{t-1}{t+1} \right| \right]_2^x = \frac{1}{2}\ln \left| \frac{x-1}{x+1} \right| + \frac{1}{2} \ln 3 \tendsto{x\to +\infty} \frac{1}{2} \ln 3
	.\] donc l'intégrale $I$\/ converge et $I \le \frac{1}{2} \ln_3$.
\end{exm}

\begin{exo}
	\begin{enumerate}
		\item L'intégrale $I = \int_{0}^{1} \frac{\sin t}{t^2}~\mathrm{d}t$\/ est impropre en 0. On utilise un équivalent : $\sin t \simi_{t\to 0} t$\/ qui ne change pas de signe. Or, $\int_{0}^{t} \frac{1}{t}~\mathrm{d}t$\/ diverge (par critère de {\sc Riemann}). Donc $I$\/ diverge.
			
			L'intégrale $J = \int_{1}^{+\infty} \sin \frac{1}{t}~\mathrm{d}t$\/ est généralisée en $+\infty$. On cherche un équivalent en $+\infty$\/ : \[
				\sin \frac{1}{t} \simi_{t\to +\infty} \frac{1}{t}
			\] qui ne change pas de signe. Or, $\int_{1}^{+\infty} \frac{1}{t}~\mathrm{d}t$\/ diverge par critère de {\sc Riemann}. On en déduit que $J$\/ diverge également.
		\item L'intégrale $\int_{0}^{+\infty} \frac{1}{t^2}~\mathrm{d}t$\/ est impropre, {\bf et}\/ en 0, {\bf et}\/ en $+\infty$. Le théorème ne marche donc pas.
			En effet $t\mapsto \frac{1}{t^2}$\/ n'est pas continue par morceaux en 0, ce qui était le cas pour $t\mapsto \frac{1}{1+t^2}$.
	\end{enumerate}
\end{exo}

\begin{rmkn}[Retour sur la {\sc remarque}\/ 5]
	L'intégrale $\int_{0}^{+\infty} \frac{1}{\ln(1+t)}~\mathrm{d}t$\/ est impropre en 0 {\bf et}\/ en $+\infty$. $\int_{0}^{+\infty} \frac{1}{\ln(1+t)}~\mathrm{d}t$\/ converge si et seulement si $\int_{0}^{7} \frac{1}{\ln(1+t)}~\mathrm{d}t$\/ {\bf et}\/ $\int_{7}^{+\infty} \frac{1}{\ln(1+t)}~\mathrm{d}t$\/ convergent.
	Et si elles convergent \[
		\int_{0}^{+\infty} \frac{1}{\ln(1+t)}~\mathrm{d}t = \int_{0}^{7} \frac{1}{\ln(1+t)}~\mathrm{d}t + \int_{7}^{+\infty} \frac{1}{\ln(1+t)}~\mathrm{d}t
	.\]
	On n'utilise pas deux barrières en même temps. Sinon, les intégrales doublement impropres peuvent, et converger, et diverger.
\end{rmkn}

\begin{prop}[avec $\sim$]
	Si $f(t) \simi_{t\to b} g(t)$\/ qui ne change pas de signe. Alors,
	\begin{itemize}
		\item ou bien $\ds\int_{a}^{b} f(t)~\mathrm{d}t$\/ et $\ds\int_{a}^{b} g(t)~\mathrm{d}t$\/ convergent et $\ds \int_{x}^{b} f(t)~\mathrm{d}t \simi_{x\to b} \int_{x}^{b} g(t)~\mathrm{d}t$.
		\item ou bien $\ds\int_{a}^{b} f(t)~\mathrm{d}t$\/ et $\ds\int_{a}^{b} g(t)~\mathrm{d}t$\/ divergent et $\ds\int_{a}^{x} f(t)~\mathrm{d}t \simi_{x\to b} \int_{a}^{x} g(t)~\mathrm{d}t$.
	\end{itemize}
	Cette proposition est équivalente à le {\sc lemme}\/ 13 sur les séries.
\end{prop}



		\begin{prop}
	La relation $\preceq$\/ est un \textit{pré-ordre} :
	\begin{itemize}
		\item $\preceq $\/ est réflective ;
		\item $\preceq $\/ est transitive.
	\end{itemize}
\end{prop}

\begin{prv}
	Soit $Q$\/ un problème de décision.
	\begin{itemize}
		\item $Q \preceq Q$\/ par la fonction identité, qui est totale et calculable.
		\item Soient $Q$, $R$\/ et $S$\/ trois problèmes de décision tels que $Q \preceq R$\/ et $R \preceq S$. Soit donc $f_1$\/ la réduction de $Q$\/ à $R$, et $f_2$\/ la réduction de $R$\/ à $S$. Soit $f = f_2 \circ f_1 : \mathcal{E}_Q \to \mathcal{E}_S$. La fonction $f$\/ est totale comme composée de fonctions totales, $f$\/ est calculable comme composée de fonctions calculables. De plus,
			\begin{align*}
				\forall e \in \mathcal{E}_Q,\qquad f(e) \in S^+ \iff& f_2(f_1(e)) \in S^+\\
				\iff& f_1(e) \in R^+\\
				\iff& e \in Q^+
			\end{align*}
	\end{itemize}
\end{prv}

\section{Classe \textbf{P} et \textbf{NP}}

Pour répondre à un problème, on peut le résoudre par des algorithmes plus ou moins rapides. Mais, l'objectif de cette section est de montrer que certains problèmes ne peuvent se résoudre que par des algorithmes lents, et que l'on ne peut pas faire mieux.

\begin{defn}
	Le modèle de calcul impose une représentation des entrées par chaînes de caractères. Cela induit donc une notion de \textit{taille d'entrée}, qui est la longueur de la chaîne de caractères.
	\index{taille d'entrée}
\end{defn}


\subsection{Complexité d'une machine}

\begin{defn}
	Étant donné une machine $\mathcal{M}$ et une entrée $w \in \Sigma^*$, on note $C^\mathcal{M}(w)$\/ le nombre d'opérations élémentaires effectuées lors de l'appel de $\mathcal{M}$\/ sur $w$. Lorsque $\smash{w \xrightarrow[\mathcal{M}]{} {\circlearrowleft}}$, on définit $C^\mathcal{M} = +\infty$.

	Pour $n \in \N$, on définit alors \[
		C^\mathcal{M}_n = \max \{ C^\mathcal{M}(w)  \mid w \in \Sigma^n \}
	.\]
	\index{machine!nombre d'opérations élémentaires!($C^\mathcal{M}(w)$)}
	\index{machine!nombre d'opérations élémentaires!maximal pour un mot de taille $n$\/ ($C^\mathcal{M}_n$)}
\end{defn}

\begin{rmk}
	On a, $\forall n \in \N$, $C_n^\mathcal{M} \in \bar{\N} = \N \cup \{+\infty\}$.
\end{rmk}

\begin{defn}
	Soit $f : \N\to \N$\/ une fonction totale et calculable. On note $\textsc{Time}(f)$\/ l'ensemble des machines $\mathcal{M}$\/ telles que
	\begin{itemize}
		\item $\mathcal{M}$\/ s'arrête sur toute entrée ;
		\item $\big(C_n^\mathcal{M}\big)_{n \in \N} = \mathcal{O}\big(\big(f(n)\big)_{n \in \N}\big)$.
	\end{itemize}
	\index{machine!ensemble $\textsc{Time}(f)$}
\end{defn}

\subsection{Classe \textbf{P}}

\begin{defn}
	On dit d'une machine $\mathcal{M}$\/ qu'elle est de \textit{complexité polynômiale} dès lors qu'il existe $k \in \N$\/ tel que $\mathcal{M} \in \textsc{Time}(n^k)$.
	\index{machine!de complexité polynômiale}
\end{defn}

\begin{defn}
	On dit d'une fonction (partielle ou non), qu'elle est \textit{calculable en temps polynômial} dès lors qu'il existe une machine $\mathcal{M}$\/ de complexité polynômiale la calculant.
	\index{fonction!calculable!en temps polynômial}
\end{defn}

\begin{exm}
	\begin{itemize}
		\item l'identité ($n \mapsto n$)
		\item la fonction successeur ($n\mapsto n+1$)
	\end{itemize}
\end{exm}


		\section{Arbres couvrants de poids minimum}

\begin{exm}
	On considère le graphe ci-dessous.
	\begin{figure}[H]
		\centering
		\tikzfig{ex-graphe-pondere}
		\caption{Arbre pondéré}
	\end{figure}
	On cherche à \guillemotleft~supprimer~\guillemotright\ des arrêtes de ce graphe afin d'avoir un poids total minimum, tout en conservant la connexité du graphe.
	Une structure assurant cette condition est un arbre.

	Pour résoudre ce problème, on part du graphe vide, et on ajoute les arrêtes les moins coûteuses en premier.
\end{exm}

\begin{defn}[Arbre]
	Soit $G = (S,A)$\/ un graphe non-orienté. On dit que $G$\/ est un \textit{arbre} si $G$\/ est connexe et acyclique.
	\index{arbre}
\end{defn}

\begin{defn}[Arbre couvrant]
	Étant donné un graphe non orienté pondéré par poids positifs $G = (S, A, c)$,\footnotemark\ on dit de $G' = (S', A')$\/ que c'est un \textit{arbre couvrant} de $G$\/ si $S' = S$\/ et $A' \subseteq A$, et $G'$\/ est un arbre.
	\index{arbre!couvrant}
\end{defn}
\footnotetext{on dit que $c$\/ est la fonction de pondération de ce graphe}

\begin{defn}[Arbre couvrant de poids minimum]
	Étant donné un graphe non orienté pondéré $G = (S, A, c)$\/ et un arbre couvrant $T = (S', A')$, on appelle \textit{poids} de l'arbre $T$\/ la valeur $\sum_{a \in A'} c(a)$.
	\index{arbre!couvrant!poids}

	Si $G$\/ est connexe, il admet au moins un arbre couvrant, on peut définir l'\textit{arbre couvrant de poids minimum} (\textit{\textsc{acpm}}).
	\index{arbre!couvrant!de poids minimum}
\end{defn}

On définir alors le problème \[
	\textsc{acpm}\text{\footnotemark}
	\begin{cases}
		\text{\textbf{Entrée}}&: G = (S, A, c) \text{ connexe}\\
		\text{\textbf{Sortie}}&: \text{ le poids de l'arbre couvrant de poids minimum}.
	\end{cases}
\]
\footnotetext{Arbre Couvrant de Poids Minimum}

\begin{algorithm}[H]
	\centering
	\begin{algorithmic}[1]
		\Entree $G = (S, A, c)$\/ un graphe connexe
		\Sortie Un arbre couvrant de poids minimum
		\State $B \gets \O$\/ 
		\State $U \gets \O$\/
		\While{il existe $u$\/ et $v$\/ tels que $u \nsim_B v$}
			\State Soit $\{x,y\} \in A \setminus U$\/ de poids minimal
			\If{$x \sim_B y$}
				\State $U \gets \big\{\!\{x,y\}\!\big\} \cup U$
			\Else
				\State $U \gets \big\{\!\{x,y\}\!\big\} \cup U$
				\State $B \gets \big\{\!\{x,y\}\!\big\} \cup B$
			\EndIf
		\EndWhile
		\State\Return $T = (S,B)$\/
	\end{algorithmic}
	\caption{Algorithme de \textsc{Kruskal}}
\end{algorithm}

\begin{prop}
	L'algorithme de \textsc{Kruskal} est correct.
\end{prop}

\begin{prv}
	\begin{enumerate}
		\item Il existe un arbre couvrant de poids minimum utilisant les arrêtes de $B$ ;
		\item $B \subseteq U \subseteq A$\/ ;
		\item $\forall \{u,v\} \in U$, $u \sim_B v$.
	\end{enumerate}
	Ces trois propriétés sont invariantes.
	\begin{description}
		\item[Initialement] $B = \O = U$, donc \textsc{ok}.
		\item[Propagation] Soient $\ubar{B}$\/ et $\ubar{U}$\/ (resp.\ $\bar{B}, \bar{U}$) les valeurs de $B$\/ et $U$\/ avant (resp.\ après) une itération de boucle. Supposons que $\ubar{B}$\/ et $\ubar{U}$\/ satisfont les propriétés 1, 2 et 3. Montrons que $\bar{B}$\/ et $\bar{U}$\/ les satisfont aussi.
			\begin{enumerate}
				\item[2.] On a $\{x,y\} \in A$\/ et $\ubar{B} \subseteq \ubar{U} \subseteq A$, donc \[
						\bar{B} \subseteq \ubar{B} \cup \{\!\{x,y\}\!\} \subseteq \ubar{U} \cup \{\!\{x,y\}\!\} \subseteq A
					.\]
				\item[3.] Soit $\{u,v\} \in \bar{U}$.
					\begin{itemize}
						\item Si $\{u,v\} \in \ubar{U}$, alors de 3, $u \sim_{\ubar{B}} v$. Or, $\ubar{B} \subseteq \bar{B}$\/ et donc $u \sim_{\bar{B}} v$.
						\item Sinon, $\{ u,v\} = \{x,y\}$, alors $x = u$\/ et $v = y$.
							\begin{itemize}
								\item Sous-cas 1 : $\bar{B} = \ubar{B} \cup \{\!\{x,y\}\!\}$, alors $x \sim_{\bar{B}} y$.
								\item Sous-cas 2 : $\bar{B} = \ubar{B}$, alors par condition du \textbf{si}, $x \sim_{\ubar{B}} y$\/ et donc $x \sim_{\bar{B}} y$.
							\end{itemize}
					\end{itemize}
				\item[1.]
					Soit $\mathcal{T}$\/ un \textsc{acpm} contenant $\ubar{B}$.
					\begin{itemize}
						\item Cas 1 : $\bar{B} = \ubar{B}$, \textsc{ok}
						\item Cas 2 : $\bar{B} = \ubar{B} \cup \{\!\{x,y\}\!\}$.
							\begin{itemize}
								\item Sous-cas 1 : $\{x,y\} \in \mathcal{T}$, alors $\mathcal{T}$\/ est un \textsc{acpm} qui contient $\bar{B}$.
								\item Sous-cas 2 : $\{x,y\}  \not\in \mathcal{T}$, $\mathcal{T}$\/ est un arbre couvrant, donc il contient une chaîne de $x$\/ à $y$\/ : \[
											\{\overset{\substack{x\\[-1mm]\vrt=}}{x_0},x_1\},\{x_1,x_2\},\ldots,\{x_{n-1},\underset{\substack{\vrt=\\y}}{x_n}\}
									.\]
									Or, $\forall i \in \llbracket 1,n-1 \rrbracket$, $x_i \sim_{\ubar{B}} x_{i+1}$. Par transitivité, on a donc $x = x_0 \sim_{\ubar{B}} x_n = y$, ce qui n'est pas le cas.
									Il existe donc $i_0 \in \llbracket 0,n-1 \rrbracket$, tel que $x_{i_0} \nsim_{\ubar{B}} x_{i_0 + 1}$\/ et donc $\{x_{i_0}, x_{i_0 + 1}\} \not\in \ubar{U}$. D'où, d'après 3, on a $\{x_{i_0}, x_{i_0 + 1}\}  \not\in \ubar{B}$
									Considérons alors $\mathcal{T}' = \big(\mathcal{T} \setminus \{\!\{x_{i_0},x_{i_0+1}\}\!\}\big)  \cup \{\!\{x,y\}\!\}$. Montrons que $\mathcal{T}'$\/ est un \textsc{acpm} contenant $B$, en commençant par montrer que c'est un arbre couvrant. L'arbre $\mathcal{T}'$\/ a $n-1$\/ arrêtes (autant que $\mathcal{T}$). Montrons que $\mathcal{T}'$\/ est connexe.
									Soit $(a,b) \in S^2$. $\mathcal{T}$\/ est connexe, soit donc une chaîne \[
										C : \quad a = u_0, u_1, \ldots, u_n = b
									\] de $\mathcal{T}$. Si la chaîne $C$\/ n'utilise pas l'arrête $\{x_{i_0},x_{i_0+1}\}$, alors $C$\/ est une chaîne de $\mathcal{T}'$. Sinon, on pose $\mu$\/ et $\tau$\/ tels que \[
									\underbrace{a,\ldots,x_{i_0}}_{\mu},\underbrace{x_{i_0+1},\ldots,b}_{\tau}
									.\]
									Soit alors la chaîne
									\begin{align*}
										\overbrace{a,\ldots,x_{i_0}}^{\mu},x_{i_0-1},x_{i_0-2},\ldots,x_0 = x,&\\
										\underbrace{b,\ldots,x_{i_0 + 1}}_{\tau},x_{i_0+2},\ldots,x_{n-1},x_n=y&
									\end{align*}
									qui est dans $\mathcal{T}'$.
									Montrons que le poids est minimum. Notons $P(\mathcal{T})$\/ le poids de l'arbre. On a donc \[
										P(\mathcal{T}') = P(\mathcal{T}) + c(\{x,y\}) - c(\{x_{i_0},x_{i_0+1}\})
									.\] Par choix glouton, ($\{x_{i_0}, x_{i_0+1} \not\in \ubar{U}\}$), $c(\{x,y\}) \le c(\{x_{i_0},x_{i_0+1}\})$\/ donc $P(\mathcal{T}') \le P(\mathcal{T})$, et $\mathcal{T}$\/ étant de poids min, $P(\mathcal{T}') = P(\mathcal{T})$\/ et $\mathcal{T}'$\/ est un \textsc{acpm} contenant $\bar{B}$.
							\end{itemize}
					\end{itemize}
			\end{enumerate}
	\end{description}

	Les invariants le sont.
\end{prv}

À la fin, $B$\/ induit un graphe connexe et $B$\/ est contenu dans un \textsc{acpm}, c'en est donc un.


		\paragraph{Une structure pour la gestion des partitions : \textsf{UnionFind}.}

\begin{defn}[Type de données abstrait \textsf{UnionFind}]
	On définit le type de données abstrait \textsf{UnionFind} comme contenant
	\begin{itemize}
		\item un type \texttt{t} de partitions ;
		\item un type \texttt{elem} des éléments manipulés par les partitions ;
		\item $\texttt{initialise\_partition} : \texttt{elem list} \to \texttt{t}$\/ retournant le partitionnement dans lequel chaque élément est seul dans sa classe ;
		\item $\texttt{find} : (\texttt{t} \mathbin{\texttt{*}} \texttt{elem}) \to \texttt{elem}$\/ retournant un représentant de la classe de l'élément. Si deux éléments $x$\/ et $y$\/ sont dans la même classe, dans le partitionnement $p$, alors $\texttt{find}(p,x) = \texttt{find}(p,y)$ ;
		\item $\texttt{union} : (\texttt{t} \mathbin{\texttt{*}} \texttt{elem} \mathbin{\texttt{*}} \texttt{elem}) \to \texttt{t}$\/ retourne le partitionnement dans lequel on a fusionné les classes des arguments.
	\end{itemize}
	\index{type \textsf{UnionFind}}
\end{defn}

\begin{exm}
	On réalise le \textit{pseudo-code} ci-dessous.
	\begin{itemize}
		\item $p \gets \texttt{initialise\_partition}([1, 2, 3, 4, 5])$\/ $\leadsto$ $\{\{1\}, \{2\}, \{3\}, \{4\}, \{5\}\}$
		\item $\texttt{find}(p, 1) = 1$\/ 
		\item $\texttt{union}(p, 1, 3)$\/ $\leadsto$ $\{\{1,3\}, \{2\}, \{4\}, \{5\}\}$
		\item $\texttt{find}(p, 1) = \texttt{find}(p, 3)$
	\end{itemize}
\end{exm}

On implémente ce type abstrait en \textsc{OCaml}.

\begin{rmk}[Niveau zéro -- listes de liste]~
	\begin{lstlisting}[language=caml,caption=Implémentation du type \textsf{UnionFind} en \textsc{OCaml}]
type 'a t = 'a list list

let initialise_partition (l: 'a list): 'a t =
	List.map (fun x -> [ x ] ) l

let rec find (p: 'a t) (x: 'a): 'a =
	match p with
	| classe :: classes ->
			if List.mem x classe then List.hd classe
			else find classes x
	| [] -> raise Not_Found

let est_equiv (p: 'a t) (x: 'a) (y: 'a): bool = 
	(find p x) = (find p y)

let rec extrait_liste (x: 'a) (p: 'a t): 'a list * 'a p =
	match p with
	| classe :: classes ->
			if List.mem x classe then (classe, classes)
			else
				let cl, cls' = extrait_liste x classes in
				(cl, classe :: cls')
	| [] -> raise Not_Found

let union (p: 'a t) (x: 'a) (y: 'a): 'a t =
	if est_equiv p x y then p
	else
		let cx, p' = extrait_liste x p in
		let cy, p'' = extrait_liste y p' in
		(cx @ cy) :: p''
	\end{lstlisting}
\end{rmk}

\begin{rmk}[Niveau un -- tableau de classes]
	Dans la case du tableau, on inscrit le numéro de sa classe.
	Pour \texttt{find}, on prend le premier ayant la même classe.
	Pour \texttt{union}, on re-numérote vers un numéro commun.
	Par exemple, \[
		\begin{array}{|c|c|c|c|c|c|}
			\hline
			0 & 1 & 0 & 0 & 1 & 2\\ \hline
			0 & 1 & 2 & 3 & 4 & 5 \\ \hline
		\end{array}\quad\quad\longleftrightarrow\quad\quad\{\{0,2,3\},\{1,4\},\{5\}\}
	.\]
\end{rmk}

\begin{rmk}[Niveau deux -- tableau de représentants]
	Dans les cases du tableau, on écrit le représentant de la classe de $i$.
	Pour \texttt{find}, on lit la case.
	Pour \texttt{union}, on re-numérote vers un numéro commun.
	Par exemple, \[
		\begin{array}{|c|c|c|c|c|c|}
			\hline
			2 & 4 & 2 & 2 & 4 & 5\\ \hline
			0 & 1 & 2 & 3 & 4 & 5 \\ \hline
		\end{array}\quad\quad\longleftrightarrow\quad\quad\{\{0,2,3\},\{1,4\},\{5\}\}
	.\]
\end{rmk}

\begin{rmk}[Niveau trois -- arbres]
	Pour $\texttt{union}(0, 1)$, on cherche le représentant de 0 (2) puis celui de 1 (4). On fait pointer 4 vers 2.
	Pour la suite de l'implémentation, \textit{c.f.}\ \textsc{dm}$_3$.

	\begin{figure}[H]
		\centering
		\tikzfig{ex-unionfind-arbres}
		\caption{Représentation par des arbres}
	\end{figure}
\end{rmk}

Avec cette nouvelle structure, on peut maintenant revenir sur l'algorithme de \textsc{Kruskal}.

\begin{algorithm}[H]
	\centering
	\begin{algorithmic}[1]
		\Entree Un graphe $G = (S, A, c)$\/ un graphe non orienté, pondéré
		\Sortie Un \textsc{acpm}
		\State Soit $(e_i)_{i\in\llbracket 1,m \rrbracket}$\/ un tri des arrêtes par coût croissant
		\State $f \gets 0$\/ \Comment{Nombre d'\textsl{\texttt{union}} effectuées}
		\State $p \gets \texttt{initialise\_partition}(S)$\/ 
		\State $I \gets 0$\/ 
		\State $B \gets \O$\/ 
		\While{$f < n - 1$}
			\State $\{x,y\} \gets e_I$\/ 
			\If{$\texttt{find}(p, x) \neq \texttt{find}(p, y)$}
				\State $p \gets \texttt{union}(p, x, y)$\/ 
				\State $B \gets B \cup \{\!\{x,y\}\!\}$\/ 
				\State $f \gets f + 1$\/ 
			\EndIf
			\State $I \gets I + 1$\/
		\EndWhile
		\State\Return $(S,B)$
	\end{algorithmic}
	\caption{Algorithme de \textsc{Kruskal} -- version 2}
\end{algorithm}

\paragraph{Étude de complexité.}
Notons $C_{\texttt{find}}^n$\/ un majorant du coût de \texttt{find} sur une structure contenant $n$\/ éléments, notons $C_{\texttt{union}}^n$\/ un majorant du coût de \texttt{union} sur une structure contenant $n$\/ éléments, et notons $C_{\texttt{init}}^n$\/ un majorant du coût de \texttt{init} sur une structure contenant $n$\/ éléments.
La complexité de cet algorithme est de \[
	\mathcal{O}\big(C_{\texttt{init}}^n + 2m\:C_{\texttt{find}}^n + n\: C_{\texttt{union}}^n + m \log_2 m\big)
.\]

\section{Couplage dans un graphe biparti}

\begin{defn}[Couplage]
	On appelle \textit{couplage} d'un graphe non orienté $G = (S, A)$, la donnée d'un sous-ensemble $C \subseteq A$\/ tel que \[
		\forall \{x,y\}, \{x',y'\} \in C,\:
		\quad\quad \{x,y\} \cap \{x',y'\} \neq \O
		\implies
		\{x,y\}  = \{x', y'\}
	.\]

	\index{graphe!couplage}
\end{defn}

\begin{figure}[H]
	\centering
	\tikzfig{ex-couplage}
	\caption{Exemple de couplage}
\end{figure}

\begin{exm}
	On réutilise l'exemple ci-dessous dans toute la section.
	L'ensemble $C = \{\{a,2\}, \{b,3\}\}$\/ est un couplage.
	Mais, l'ensemble $C' = \{\{a,1\}, \{a,2\}\}$\/ n'en est pas un.
\end{exm}

\begin{defn}
	Un couplage est dit \textit{maximal} s'il est maximal pour l'inclusion ($\subseteq$).
	Un couplage est dit \textit{maximum} si son cardinal est maximal.
	\index{graphe!couplage!maximal}
	\index{graphe!couplage!maximum}
\end{defn}

\begin{exm}
	Dans l'exemple précédent, 
	\begin{itemize}
		\item le couplage $C = \{\{a,2\}, \{b,3\}\}$\/ n'est ni maximal, ni maximum ;
		\item le couplage $C' = \{\{a,2\}, \{b, 3\}, \{d, 4\}\}$\/ est maximal mais pas maximum ;
		\item le couplage $C'' = \{\{a,1\}, \{b,3\}, \{c,2\}, \{d,4\}\}$\/ est maximum.
	\end{itemize}
\end{exm}

\begin{rmk}
	Dans toute la suite, on ne considère que des graphes bipartis.
\end{rmk}

\begin{defn}
	Étant donné un graphe biparti $G = (S, A)$\/ et un couplage $C$, un sommet $x$\/ est dit \textit{libre} dès lors que \[
		\forall \{y,z\} \in C,\: x \not\in \{y,z\}
	.\]
	Une chaîne élémentaire\footnotemark $(c_0, c_1, \ldots, c_{2p+1})$\/ est dit \textit{augmentante} si
	\begin{itemize}
		\item $c_0$\/ et $c_{2n+1}$\/ sont libres ;
		\item $\forall i \in \llbracket 0,p \rrbracket$, $\{c_{2i}, c_{2i+1}\}\in A \setminus C$ ;
		\item $\forall i \in \llbracket 0,p-1 \rrbracket$, $\{c_{2i+1}, c_{2i+2}\} \in C$.
	\end{itemize}
\end{defn}
\footnotetext{\textit{i.e.}\ une chaîne sans boucles.}

\begin{exm}~
	\begin{figure}[H]
		\centering
		\tikzfig{ex-chaine-augmentante}
		\caption{Chaîne augmentante}
	\end{figure}
\end{exm}

\begin{exm}
	Dans l'exemple de cette section, $(d, 4)$\/ et $(c, 2, a, 1)$\/ sont deux chaînes augmentantes.
\end{exm}

\begin{prop}
	Étant donné un graphe biparti $G = (S, A)$\/ avec $S = S_1 \cupdot S_2$ (partitionnement du graphe biparti), un couplage $C$\/ est maximum si, et seulement s'il n'admet pas de chaînes augmentantes.
\end{prop}

\begin{prv}
	\begin{itemize}
		\item[``$\implies$'']
			Soit $C$\/ un couplage admettant une chaîne augmentante. Montrons que $C$\/ n'est pas maximum.
			Soit la chaîne augmentante\footnotemark \[
				c_0 \to c_1 \Rightarrow c_2 \to c_3 \Rightarrow c_4 \to \cdots \to c_{2p - 1} \Rightarrow c_{2p} \to c_{2p+1}
			.\]
			On considère alors le couplage \[
				C' = \Big(C \setminus \big\{\{c_{2i+1},c_{2i+2} \} \mid i \in \llbracket 0,p-1 \rrbracket\big\}\Big) \cup \big\{\{c_{2i},c_{2i+1}\}  \mid i \in \llbracket 0,p \rrbracket\big\}
			.\]
			On transforme donc la chaîne en \[
				c_0 \Rightarrow c_1 \to c_2 \Rightarrow c_3 \to \cdots \to c_{2p-1} \to c_{2p} \Rightarrow c_{2p+1}
			.\]
			C'est bien un couplage, et $\Card(C') = \Card C + 1$. $C$\/ n'est donc pas un couplage maximum.
		\item[``$\impliedby$'']
			Soit $C$\/ un couplage non maximum. Montrons que $C$\/ admet une chaîne augmentante. Soit $M$\/ un couplage maximum, et $D = C \mathrel{\triangle} M = (C \setminus M) \cupdot (M \setminus C)$.
			On a $\Card C < \Card M$\/ et $\Card(C \setminus M) < \Card(M \setminus C)$.
			On remarque que, si $c_0 \to c_1 \to c_2 \to \cdots \to c_{p-1}\to c_p$\/ est une chaîne de $D$, (si $c_0 \to c_1 \in C \setminus M$\/ et $c_1 \to c_2 \in C \setminus M$\/ donc $c_1$\/ est dans deux arrêtes distinctes d'un couplage $C$, ce qui est absurde ; de même pour les autres arrêtes). Ainsi, 2 arrêtes consécutives ne sont pas dans la même composante de l'union $(C \setminus M) \cupdot (M \setminus C)$.
			Considérons la relation d'équivalence $\sim$\/ sur $D$\/ définie par $\{x,y\} \sim \{z,t\} \iffdef$ il existe une chaîne de $D$\/ utilisant l'arrête $\{x,y\}$\/ et l'arrête $\{z,t\}$.
			Soit le partitionnement $D_1, \ldots, D_q$\/ de $D$\/ par $\sim$.
			Par inégalité de cardinal, il existe un $D_i$\/ tel que \[
				\Card \{e \in D_i  \mid e \in C\} < \Card \{e \in D_i  \mid e \in M\}
			.\] L'ensemble $D_i$\/ contient alors une chaîne augmentante.
	\end{itemize}
\end{prv}
\footnotetext{On représente $\Rightarrow$ pour les arrêtes dans le couplage $C$.}

		\begin{prv}[par récurrence sur $n$, la largeur de la matrice]
	\begin{itemize}
		\item Si $n = 1$, alors la matrice $A = (a_{11})$\/ est déjà triangulaire.
		\item On suppose le polynôme caractéristique $\chi_A$\/ de la matrice scindé dans $\mathds{K}[X]$, d'où il a au moins une racine dans $\mathds{K}$. D'où, la matrice $A$\/ a au moins une valeur propre $\lambda_1 \in \mathds{K}$. Il existe donc un vecteur non nul $\vec{\varepsilon}_1$\/ tel que $A \cdot \vec{\varepsilon}_1 = \lambda_1\,\vec{\varepsilon}_1$. On complète $(\vec{\varepsilon}_1)$\/ en une base de $\mathds{K}^n$\/ : $(\vec{\varepsilon}_1, \vec{e}_2, \ldots, \vec{e}_n)$. En changent de base, il existe une matrice inversible $P$\/ telle que \[
			A' = P^{-1}\cdot A\cdot P = 
			\begin{pNiceArray}[last-row,last-col]{c|ccc}
				\lambda_1 & *&\Ldots&*&\varepsilon_1\\ \hline
				0 & \Block{3-3}{B}&&&e_1\\
				\Vdots&&&&\Vdots\\
				0&&&&e_n\\
				f(\vec{\varepsilon}_1)&f(\vec{e}_1)&\ldots&f(\vec{e}_n)
			\end{pNiceArray}
		.\]
		Comme le polynôme caractéristique est invariant par changement de base, on en déduit que \[
			\chi_A(x) = \chi_{A'}(x) = \left|
			\begin{array}{c|c}
				x-\lambda_1 &*\\ \hline
				0&xI_{n-1} - B\\
			\end{array} \right| = (x-\lambda_1) \cdot \Pi(x)
		.\]
		Or, comme $\chi_A$\/ est scindé, $\Pi(x)$\/ est aussi scindé.
		Or, $\Pi(x) = \det(xI_{n-1} - B)$ d'où $B$\/ est trigonalisable.
	\end{itemize}
\end{prv}

\begin{crlr}
	Toute matrice de $\mathscr{M}_{n,n}(\C)$\/ est trigonalisable.
\end{crlr}

\begin{exo}\relax
	{\slshape Soit une matrice carrée $A \in \mathscr{M}_{n,n}(\mathds{K})$ (où $\mathds{K}$\/ est $\R$\/ ou $\C$). Montrer que
		\begin{align*}
			(1)\quad\text{la matrice } A \text{ est nilpotente}
			\iff& \text{ le polynôme caractéristique de } A \text{ est } \chi_A(X) = X^n\quad(2)\\
			\iff& \text{ la matrice } A \text{ est trigonalisable avec des zéros}\\
			&\text{ sur sa diagonale} \quad(3)
		\end{align*}
	}

	On montre $``\,(1) \implies (2),"$ $``\,(2) \implies (3)\,"$\/ puis $``\,(3) \implies (1)."$

	\begin{itemize}
		\item[$``\,(3) \implies(1)\,"$] Il existe donc une matrice inversible $P$\/ telle que $T = P^{-1}\cdot A\cdot P$\/ et $T$\/ est une matrice trigonalisable. Or, à chaque produit $A^n \cdot A$, une \guillemotleft~sur-diagonalse~\guillemotright\  de zéros supplémentaires. D'où, à partir d'un certain rang $p$, on a $A^p = 0$. La matrice $A$\/ est donc nilpotente.
		\item[$``\,(2) \implies(3)\,"$] On sait que $\chi_A = X^n = (X-0)^n$\/ est scindé, d'où $A$\/ est trigonalisable.
			Il existe donc une matrice inversible $P$\/ telle que \[
				P^{-1}\cdot A\cdot P = A' = \begin{pmatrix}
					\lambda_1 & * & \ldots & *\\
					0 & \ddots&\ddots&\vdots\\
					\vdots&\ddots&\ddots&*\\
					0&\ldots&0&\lambda_n
				\end{pmatrix}
			.\]
			Et donc $\chi_{A'}(x) = (x-\lambda_1)(x-\lambda_2) \cdots (x-\lambda_n)$.
			Or, le polynôme caractéristique est invariant par changement de base, d'où $\lambda_1 = \lambda_2 = \cdots = \lambda_n$.
		\item[$``\,(1)\implies(2)\,"$] On passe dans $\C$\/ alors $\chi_A$\/ est scindé dans $\C$. D'où, il existe $(\lambda_1, \lambda_2, \ldots, \lambda_n) \in \C^n$\/ tels que \[
			\chi_A(X) = (X - \lambda_1) (X - \lambda_2) \cdots (X-\lambda_n)
		.\]
		D'où, chaque $\lambda_i$\/ est une valeur propre \ul{complexe} de la matrice $A$. Or $A$\/ est nilpotente, d'où, par définition, il existe $p \in \N$\/ tel que $A^p = 0$. Les scalaires $\lambda_i$\/ sont dans le spectre de $A$\/ : en effet, il existe un vecteur colonne $X$\/ non nul tel que $A\cdot X = \lambda_i\,X$, d'où $A^2 \cdot X = A\cdot AX = A\cdot \lambda_iX = \lambda_i^2 X$. De même, $A^3 \cdot X = A \cdot A^2 \cdot X = A \cdot \lambda_i^2 X = \lambda_i^2 (A\cdot X) = \lambda_i^3 X$.
		Et, de \guillemotleft~proche en proche~\guillemotright, on a donc \[
			\forall k \in \N,\:A^k\cdot X = \lambda_i^k X
		.\]
		En particulier, si $k=p$, on a $0 = 0\cdot X = A^p\cdot X = \lambda_i^pX$. D'où $\lambda_i^p X = 0$. Or, $X \neq 0$, d'où $\lambda_i^p = 0$\/ et donc $\lambda_i = 0$.
		Finalement, $\chi_A(X) = (X-\lambda_1) \cdots (X-\lambda_n) = (X-0)\cdots(X-0) = X^n  \in \C[X]$. On a donc $\chi_A(X) \in \R[X]$.
	\end{itemize}
\end{exo}

		Une isométrie $f$\/ est, ou bien une rotation d'un angle $\theta$ autour de l'axe $\Vect(\vec{w})$, ou bien la composée d'une rotation d'angle $\theta$\/ autour de l'axe $\Vect(\vec{w})$\/ et d'une symétrie par rapport au plan $\Vect(\vec{u}, \vec{v})$.

\begin{prv}[{\color{cyan}tarte à la crème}\null]
	Comme $f$\/ est une isométrie, on a, pour tout vecteur $\vec{x} \in E$, $\|f(\vec{x})\| = \|\vec{x}\|$.
	Or, il existe un vecteur $\vec{x}$\/ non nul tel qu'il existe $\lambda \in \R$\/ tel que $f(\vec{x}) = \lambda \vec{x}$, d'où $\|\lambda \vec{x}\| = \|\vec{x}\|$, donc $|\lambda|\cdot \|\vec{x}\| = 1 \cdot \|\vec{x}\|$. On en déduit que $\lambda \in \{-1,1\}$\/ car $\|\vec{x}\| \neq 0$.
	La suite de la démonstration est dans le poly.
\end{prv}

\begin{thm}
	Soit $E$\/ un espace euclidien, et soit $f$\/ une isométrie : $E$\/ est la somme directe et orthogonale de $\Ker(\id_E - f)$, de $\Ker(-\id_E - f)$, et/ou de plans $P_i$\/ stables par $f$\/ sur lesquels $f$\/ induit une rotation.
\end{thm}

\begin{crlr}
	Si $f$\/ est une isométrie d'un espace euclidien $E$, alors il existe $(p, q) \in \N^2$, des réels $\theta_1, \ldots, \theta_k$, et une base $\mathcal{B}$\/ orthonormée de $E$\/ tels que \[
		[\:f\:]_\mathcal{B} = \begin{pNiceMatrix}
			R_{\theta_1} & \Block{3-4}{(0)} & & & \\
			\Block{4-3}{(0)}& \ddots & & &\\
			& & R_{\theta_k} & & \\
			&&& I_p &\\
			&&&& -I_q
		\end{pNiceMatrix} 
	.\] 
\end{crlr}


		\subsection{Algorithme de {\scshape Berry-Sethi}\/ : les langages réguliers sont reconnaissables}

\begin{exm}
	On considère l'expression régulière $aab(a \mid b)^*$. On numérote les lettres : $a_1a_2b_1(a_3 \mid b_2)^*$, avec  \[
		\varphi : \left(
			\begin{array}{ccc}
				a_1&\mapsto&a\\
				a_2&\mapsto&a\\
				a_3&\mapsto&a\\
				b_1&\mapsto&b\\
				b_2&\mapsto&b\\
			\end{array}
		\right)
	.\]

	\begin{table}[H]
		\centering
		\[
			\begin{array}{c|c|c|c|c}
				&\Lambda&S&P&F\\\hline
				a_1&\O&a_1&a_1&\O\\
				a_2&\O&a_2&a_2&\O\\
				a_1\cdot a_2&\O&a_2&a_1&a_1a_2\\
				b_1&\O&b_1&b_1&\O\\
				a_1a_2b_1&\O&b_1&a_1&a_1a_2,a_2b_1\\
				a_3&\O&a_3&a_3&\O\\
				b_2&\O&b_2&b_2&\O\\
				a_3 \mid b_2&\O&a_3,b_2&a_3,b_2&\O\\
				(a_3 \mid b_2)^*&\varepsilon&a_3,b_2&a_3,b_2&a_3b_2,b_2a_3,a_3a_3,b_2b_2\\
				a_1a_2b_1(a_3 \mid b_2)^*&\O&a_3,b_2ab_1&a_1&a_3b_2,b_2a_3,a_3a_3,b_2b_2,a_1a_2,a_2b_1,b_1a_3,b_1b_2
			\end{array}
		\]
		\caption{$\Lambda$, $S$, $P$\/ et $F$\/ pour les différents mots reconnus}
		\label{tab:l,s,p,f}
	\end{table}
	On crée donc l'automate ci-dessous.
	\begin{figure}[H]
		\centering
		\tikzfig{automate-numerote}
		\caption{Automate déduit de la table \ref{tab:l,s,p,f}}
		\label{aut:num}
	\end{figure}
	On applique la fonction $\varphi$\/ a tous les états et transitions pour obtenir l'automate ci-dessous. Cet algorithme reconnaît le langage $aab(a \mid b)^*$.
	\begin{figure}[H]
		\centering
		\tikzfig{automate-non-numerote}
		\caption{Application de $\varphi$\/ à l'automate de la figure \ref{aut:num}}
	\end{figure}
\end{exm}

\begin{thm}
	Tout langage régulier est reconnaissable.
	De plus, on a un algorithme qui calcule un automate le reconnaissant, à partir de sa représentation sous forme d'expression régulière.
\end{thm}

\begin{algo}[\scshape Berry-Sethi]
	Entrée : Une expression régulière $e$\/ \\
	Sortie : Un automate reconnaissant $\mathcal{L}(e)$\/ \\
	\begin{enumerate}
		\item On linéarise $e$\/ en $f$\/ avec une fonction $\varphi$\/ telle que $f_\varphi = e$.
		\item On calcule inductivement $\Lambda(f)$, $S(f)$, $P(f)$, et $F(f)$.
		\item On fabrique $\mathcal{A} = (\Sigma, \mathcal{Q}, I, F, \delta)$\/ un automate reconnaissant $\mathcal{L}(f)$.
		\item On retourne $\mathcal{A}_\varphi$.
	\end{enumerate}
	\todo{refaire la mise en page pour les algorithmes}
\end{algo}

\subsection{Les langages reconnaissables sont réguliers}

On fait le \guillemotleft~sens inverse~\guillemotright\ : à partir d'un automate, comment en déduire le langage reconnu par cet automate ?

L'idée est de supprimer les états un à un.
Premièrement, on rassemble les états initiaux en les reliant à un état {\raisebox{-5pt}{\tikz\node [style=new style 0] at (0, 0) {\clap{$i$}};}}, et de même, on relie les états finaux à {\raisebox{-5pt}{\tikz\node [style=new style 0] at (0, 0) {\clap{$f$}};}}.
Pour une suite d'états, on concatène les lettres reconnus sur chaque transition :
\begin{figure}[H]
	\centering
	\tikzfig{automate-succession}
	\caption{Succession d'états}
\end{figure}
De même, lors de \guillemotleft~branches~\guillemotright\ en parallèles, on les concatène avec un $ \mid $.
En appliquant cet algorithme à l'automate précédent, on a \[
	(aab)  \cdot \Big((\varepsilon \mid aa^*)  \mid (b \mid aa^*b)\cdot (b \mid aa^*b)^* (aa^* \mid \varepsilon)\Big)
.\]

\begin{defn}
	Un automate généralisé est un quintuplet $(\Sigma, \mathcal{Q}, I, F, \delta)$\/ où
	\begin{itemize}
		\item $\Sigma$\/ est un alphabet ;
		\item $\mathcal{Q}$\/ est un ensemble fini ;
		\item $I \subseteq  \mathcal{Q}$\/ ;
		\item $F \subseteq \mathcal{Q}$\/ ;
		\item $\delta \subseteq \mathcal{Q} \times \Reg(\Sigma) \times \mathcal{Q}$, avec \[
			\forall r \in \Reg(\Sigma),\:\forall (q,q') \in \mathcal{Q}^2,\:\Card(\{(q,r,q') \in \delta\}) \le 1
		.\]
	\end{itemize}
\end{defn}

\begin{defn}[Langage reconnu par un automate généralisé]
	Soit $(\Sigma, \mathcal{Q}, I, F, \delta)$\/ un automate généralisé.
	On dit qu'un mot $w$\/ est reconnu par l'automate s'il existe une suite \[
		q_0 \xrightarrow{r_1} q_1 \xrightarrow{r_2} q_2 \to \cdots \to q_{n-1}\xrightarrow{r_n} q_n
	\] et $(u_i)_{i \in \left\llbracket 1,n \right\rrbracket}$\/ tels que $\forall i \in \left\llbracket 1,n \right\rrbracket$, $u_i \in \mathcal{L}(r_i)$\/ et $w = u_1 \cdot u_2 \cdot \ldots \cdot u_n$.
\end{defn}

\begin{defn}
	Un automate généralisé $(\Sigma, \mathcal{Q}, I, F, \delta)$\/ est dit \guillemotleft~bien détouré\footnotemark~\guillemotright\ si $I = \{i\}$ et $F = \{f\}$, avec $i \neq f$, tels que $i$\/ n'a pas de transitions entrantes et $f$\/ n'a pas de transitions sortantes.
\end{defn}
\footnotetext{Cette notation n'est pas officielle.}

\begin{lem}
	Tout automate généralisé est équivalent à un automate généralisé \guillemotleft~bien détouré.~\guillemotright\@ En effet, soit $\mathcal{A} = (\Sigma, \mathcal{Q}, I, F, \delta)$\/ un automate généralisé.
	Soit $i \not\in \mathcal{Q}$\/ et $f \not\in \mathcal{Q}$.
	On pose $\Sigma' = \Sigma$, $I' = \{i\}$, $F' = \{f\}$, $\mathcal{Q}' = \mathcal{Q} \cupdot \{i,f\}$\/ et \[
		\delta' = \delta \cupdot \{(i,\varepsilon,q)  \mid q \in I\} \cupdot \{(q,\varepsilon,f)  \mid q \in F\}
	.\] Alors, l'automate $\mathcal{A}' = (\Sigma', \mathcal{Q}', I', F', \delta')$\/ est équivalent à $\mathcal{A}$\/ et \guillemotleft~bien détouré.~\guillemotright
\end{lem}

\begin{lem}
	Soit $\mathcal{A} = (\Sigma, \mathcal{Q}, I, F, \delta)$\/ un automate généralisé \guillemotleft~bien détouré~\guillemotright\ tel que $|\mathcal{Q}| \ge 3$. Alors il existe un automate généralisé \guillemotleft~bien détouré~\guillemotright\ $\mathcal{A}'  =(\Sigma, \mathcal{Q}', I, F, \delta')$\/ avec $\mathcal{Q}' \subsetneq \mathcal{Q}$\/ et $\mathcal{L}(\mathcal{A}) = \mathcal{L}(\mathcal{A}')$.
\end{lem}

\begin{prv}
	Étant donné qu'il existe au plus une transition entre chaque pair d'état $(q, q') \in \mathcal{Q}^2$, il est possible de le représenter au moyen d'une fonction de transition \[
		T : \mathcal{Q} \times \mathcal{Q} \longrightarrow \Reg(\Sigma)
	.\]
	\todo{Recopier la def de $T$}
	Soit $q \in \mathcal{Q} \setminus \{i,f\}$. Soit alors $T'$\/ défini, pour $(q_a, q_b) \in \mathcal{Q} \setminus \{q\}$, par \[
		T'(q_a, q_b) = T(q_a, q_b)  \mid T(q_a, q) \cdot T(q,q)^* \cdot T(q,q_b)
	.\] On considère l'automate $\mathcal{Q}' = \mathcal{Q} \setminus \{q\}$\/ et $\delta'$\/ construit à partir de $T'$.
\end{prv}

\begin{exm}
	On considère l'automate ci-dessous.
	\begin{figure}[H]
		\centering
		\tikzfig{automate-11}
		\caption{Automate exemple}
		\label{aut:a11}
	\end{figure}
	La fonction $T$\/ peut être représentée dans la table ci-dessous.
	\begin{table}[H]
		\centering
		\begin{tabular}{c|ccc}
			&0&1&2\\\hline
			0&$\O$&$a \mid b$&$\varepsilon$ \\
			1&$\O$\/ &$\O$\/ &$\O$\/ \\
			2&$\O$\/ &$\O$\/ &$a^*$\/\\
		\end{tabular}
		\caption{Fonction $T$\/ équivalente à l'automate de la figure \ref{aut:a11}}
	\end{table}
\end{exm}

\begin{exm}
	On applique l'algorithme à l'automate suivant.
	\begin{figure}[H]
		\centering
		\tikzfig{automate-12}
		\tikzfig{automate-12b}
		\tikzfig{automate-12c}
		\tikzfig{automate-12d}
		\tikzfig{automate-12e}
		\tikzfig{automate-12f}
		\caption{Application de l'algorithme à un exemple}
	\end{figure}
	On a donc que le langage de l'automate initial est \[
		\mathcal{L}(d^* (aa) \mid (ac \mid b)(cc)^*(b \mid ca)(d \mid e)^*)
	.\]
\end{exm}

\begin{thm}
	Un langage reconnaissable est régulier.
\end{thm}

\begin{prv}
	On itère le lemme précédent depuis un automate généralisé $\mathcal{A}$\/ jusqu'à obtention d'un automate comme celui ci-dessous.
	\begin{figure}[H]
		\centering
		\tikzfig{automate-13}
		\caption{Automate résultat de l'application du lemme}
	\end{figure}
	On a alors $\mathcal{L}(\mathcal{A}) = \mathcal{L}(r)$.
\end{prv}


		\begin{prop}
	Soit $E$\/ un $\mathds{K}$-espace vectoriel de dimension finie, et soit $u : E \to E$\/ un endomorphisme. Si $u$\/ est diagonalisable, et $F$\/ est un sous-espace vectoriel stable par $u$, alors $u\big|_F$\/ est aussi diagonalisable.
\end{prop}

\begin{prv}
	On suppose $u$\/ diagonalisable.
	D'où $u$\/ possède un polynôme annulateur $P$\/ scindé à racine simple.
	Alors $P(u) = 0_{\mathscr{L}(E)}$, d'où $\forall \vec{x} \in E$, $P(u)(\vec{x}) = \vec{0}_E$, et donc $\forall \vec{x} \in F$, $P\big(u\big|_F\big)(\vec{x}) = \vec{0}_E = \vec{0}_F$.
	Et donc, le polynôme $P$\/ est annulateur de $u\big|_F$\/ et il est scindé à racines simples. D'où $u\big|_F$\/ est diagonalisable.
\end{prv}

\begin{exo}
	{\slshape Soit $A$\/ une matrice diagonale par blocs. Montrer que $A$\/ est diagonalisable si et seulement si chaque bloc est diagonalisable.} \[
		\begin{bNiceArray}{c|c|c|c}[last-col]
			B_1&0&0&0&\begin{array}{l}\varepsilon_1\\\vdots\\\varepsilon_{d_1}\\\end{array}\\ \hline
			0&B_2&0&0&\begin{array}{l}\varepsilon_{d_1+1}\\\vdots\\\varepsilon_{d_1+d_2}\\\end{array}\\ \hline
			0&0&\ddots&0&\quad\vdots\\ \hline
			0&0&0&B_r&\begin{array}{l}\varepsilon_{d_1+\cdots + d_{r-1} + 1}\\\vdots\\\varepsilon_{d_1 + \cdots + d_r}.\\\end{array}\\
		\end{bNiceArray}
	\]
	\begin{itemize}
		\item[``$\impliedby$'']
			Soit $u$\/ l'endomorphisme tel que $\big[u\big]_\mathscr{B} = A$, où $\mathscr{B} = (\varepsilon_1, \ldots, \varepsilon_d,\ldots, \varepsilon_{d_1+\cdots + d_{r-1} + 1}, \ldots, \varepsilon_{d_1 + \cdots + d_r})$.
			Chaque sous-espace vectoriel $F_i$\/ est stable par $u$\/ car la matrice est diagonale par blocs.
			Or, chaque bloc est diagonalisable, d'où chaque $u\big|_{F_i}$\/ est diagonalisable.
			Il existe donc une base de $F_i$\/ formée de vecteurs propres de $u$. En concaténant ces bases, on obtient une base de $F$\/ formée de vecteurs propres de $u$.

			Autre méthode :
			Chaque bloc $B_i$\/ est diagonalisable, d'où $\forall i$, $\exists P_i \in \mathrm{GL}_{d_i}(\mathds{K})$, $P_i^{-1}\cdot B_i \cdot P_i = D_i$\/ diagonale.
			On pose \[
				P =
				\left(\begin{array}{c|c|c|c}
						P_1&0&\ldots&0\\ \hline
						0&P_2&\ddots&\vdots\\ \hline
						\vdots&\ddots&\ddots&0\\ \hline
						0&\ldots&0&P_r
				\end{array}\right)
			.\]
			Et donc \[
				P^{-1}\cdot A \cdot P = \left( 
				\begin{array}{c|c|c|c}
					D_1&0&\ldots&0\\ \hline
					0&D_2&\ddots&\vdots\\ \hline
					\vdots&\ddots&\ddots&0\\ \hline
					0&\ldots&0&D_r
				\end{array}\right)
			.\]
		\item[``$\implies$'']
			Réciproquement, pour tout $i$, on a $B_i = \Big[u\big|_{F_i}\Big]_{(\varepsilon_{d_1 + \cdots + d_{i-1} + 1},\ldots,\varepsilon_{d_1 + \cdots + d_i})}$.\\
			Or $u$\/ est diagonalisable, donc tout endomorphisme induit par $u$\/ sur un sous-espace vectoriel stable est diagonalisable. Et donc, chaque bloc est diagonalisable.
	\end{itemize}
\end{exo}



		\clearpage
		\setcounter{section}{0}		\renewcommand{\thesection}{\llap{Annexe }\thechapter.\Alph{section}}
		\renewcommand{\thesectionnum}{\Alph{section}}
		\section{Comment prouver la correction d'un programme ?}

Avec $\Sigma = \{a,b\}$. Comment montrer qu'un mot a au moins un $a$\/ et un nombre pair de $b$.

\begin{figure}[H]
	\centering
	\tikzfig{annexe-a-automate-1}
	\caption{Automate reconnaissant les mots valides}
\end{figure}

On veut montrer que \[
	P_w : \text{\guillemotleft~}\forall w \in \Sigma^*,\, \forall q \in \mathcal{Q},\:(\text{il existe une exécution par $w$\/ menant à $q$}) \iff w \text{ satisfait } I_q\text{~\guillemotright}
\]
où \[
	I_{\substack{(v,\\\vrt\in\\\mathds{B}}\substack{r)\\\vrt\in\\\{0,1\}}} : \quad
		(|w|_a \ge 1 \iff v)\:\text{et}\:(r = |w|_b\:\text{mod}\:2)
.\]
On le montre par récurrence sur la longueur de $w$\/ : 

\begin{itemize}
	\item[``$\implies$'']
		\begin{itemize}
			\item Pour $w = \varepsilon$, alors  montrons que $\forall q \in \mathcal{Q}$, il existe un exécution menant à $q$\/ étiquetée par $w$\/ (noté $\xrightarrow[\mathcal{A}]{w}q$) si et seulement si $w$\/ satisfait $I_q$.
				\begin{itemize}
					\item $\xrightarrow[\mathcal{A}]{\varepsilon}({\bfm F}, 0)$\/ est vrai, de plus $\varepsilon$\/ satisfait $I_{({\bfm F}, 0)}$\/ ;
					\item sinon si $q \neq ({\bfm F}, 0)$, alors $\xrightarrow[\mathcal{A}]{\varepsilon}q$\/ est fausse, de plus $\varepsilon$\/ ne satisfait pas $I_q$.
				\end{itemize}
			\item Supposons maintenant $P_w$\/ vrai pour tout mot $w$\/ de taille $n$. Soit $w = w_1\ldots w_nw_{n+1}$\/ un mot de taille $n+1$. Notons $\ubar{w} = w_1\ldots w_n$.
				Montrons que $P_w$\/ est vrai. Soit $q \in \mathcal{Q}$. Supposons $\xrightarrow[\mathcal{A}]{w} q$.
				\begin{itemize}
					\item Si $q = ({\bfm F}, 0)$\/ et $w_{n+1} = b$. On a donc $\xrightarrow[\mathcal{A}]{\ubar{w}}({\bfm F},1)$, et, par hypothèse de récurrence, $\ubar{w}$\/ satisfait. Donc $|\ubar{w}|_a = 0$\/ et $|\ubar{w}|_b \equiv 1 \mod 2$\/ donc $|w|_a = 0$\/ et $|w|_b \equiv 0 \mod 2$\/ donc $w$\/ satisfait $I_{({\bfm F}, 0)}$.
					\item De même pour les autres cas.
				\end{itemize}
		\end{itemize}
	\item[``$\impliedby$''] Réciproquement, supposons que $w$\/ satisfait $I_q$.
		\begin{itemize}
			\item Si $w = ({\bfm V},0)$\/ et $w_{n+1} = a$. Alors,
				\begin{itemize}
					\item si $|\ubar{w}|_a = 0$, alors $\ubar{w} $\/ satisfait $I_{({\bfm F},0)}$. Par hypothèse de récurrence, on a donc $\xrightarrow[\mathcal{A}]{\ubar{w}}({\bfm F},0)$\/ et donc $\xrightarrow[\mathcal{A}]{w}({\bfm V},0)$.
					\item si $|\ubar{w}|_b \ge 1$, alors $\ubar{w}$\/ satisfait $I_{({\bfm V},0)}$\/ donc $\xrightarrow[\mathcal{A}]{\ubar{w}}({\bfm V},0)$\/ et donc $\xrightarrow[\mathcal{A}]{\ubar{w}}({\bfm V},0)$.
				\end{itemize}
			\item De même pour les autres cas.
		\end{itemize}
\end{itemize}
On a donc bien \[
	\forall w \in \Sigma^*,\forall q \in \mathcal{Q},\:\xrightarrow[\mathcal{A}]w q \iff w \text{ satisfait } I_q
.\] Finalement,
\begin{align*}
	\mathcal{L}(\mathcal{A}) &= \{w \in \Sigma^* \mid \exists f \in F,\:\xrightarrow[\mathcal{A}]{w} f\} \\
	&= \{w \in \Sigma^*  \mid \xrightarrow[\mathcal{A}]w ({\bfm V},0)\}  \\
	&= \{w \in \Sigma^*  \mid w \text{ satisfait } I_{({\bfm V},0)}\} \\
	&= \{w \in \Sigma^*  \mid |w|_a \ge 1 \text{ et } |w|_b \equiv 0 \mod 2\} \\
\end{align*}

		\section{\scshape Hors-programme}

\begin{defn}
	On appelle monoïde un ensemble $M$\/ muni d'une loi ``$\cdot$'' interne associative admettant un élément neutre $1_M$.
\end{defn}

\begin{defn}
	Étant donné deux monoïdes $M$\/ et $N$, on appelle morphisme de monoïdes une fonction $\mu : M \to N$\/ telle que
	\begin{enumerate}
		\item $\mu(1_M) = 1_N$\/ ;
		\item $\mu(x \cdot_M y) = \mu(x) \cdot_N \mu(y)$.
	\end{enumerate}
\end{defn}

\begin{exm}
	$|\:\cdot\:|\:: (\Sigma^*, \cdot) \to (\N, +)$\/ est un morphisme de monoïdes.
\end{exm}

\begin{defn}
	Un langage $L$\/ est dit reconnu par un monoïde $M$, un morphisme $\mu : \Sigma^* \to M$\/ et un ensemble $P \subseteq M$\/ si $L = \mu^{-1}(P)$.
\end{defn}

\begin{exm}
	L'ensemble $\{a^{n^3} \mid n \in \N\}$\/ est reconnu par le morphisme $|\:\cdot\:|$\/ et l'ensemble $P = \{n^3  \mid n \in \N\}$.
\end{exm}

\begin{thm}
	Un langage est régulier si et seulement s'il est reconnu par un monoïde fini.
\end{thm}

\begin{exm}
	L'ensemble $\{a^{2n}  \mid n \in \N\}$\/ est un langage régulier. En effet, on a $M = \ZdZ$, $P = \{0\}$\/ et \begin{align*}
		\mu: \Sigma^* &\longrightarrow \ZdZ \\
		w &\longmapsto |w|\ \mathrm{mod}\ 2.
	\end{align*}

	\begin{figure}[H]
		\centering
		\tikzfig{automate-monoide-1}
		\caption{Automate reconnaissant $\mu^{-1}(P) = L$}
	\end{figure}
\end{exm}

\begin{prv}
	\begin{itemize}
		\item[``$\implies$''] Soit $L \in \wp(\Sigma^*)$\/ reconnu par un monoïde $M$ fini, un morphisme $\mu$\/ et un ensemble $P$ : $L = \mu^{-1}(P)$. Posons $\mathcal{A} = (\Sigma', \mathcal{Q}, I, F, \delta)$\/ avec

			\vspace{-5mm}
			\begin{multicols}{4}
				\[\Sigma' = \Sigma\]
				\[\mathcal{Q} = M\]
				\[I = \{1_M\}\]
				\[F = P\]
			\end{multicols}
			\vspace{-7mm}\[
				\delta = \{(q,\ell, q') \in \mathcal{Q} \times \Sigma \times \mathcal{Q}  \mid q \cdot \mu(\ell) = q'\}.
			\]
			Montrons que $\mathcal{L}(\mathcal{A}) = L$.
			Soit $w \in \mathcal{L}(\mathcal{A})$. Il existe une exécution acceptante \[
				1_M = q_0 \xrightarrow{w_1} q_1 \to \cdots \xrightarrow{w_n} q_n \in P
			.\] Or, $\mu(w_1\ldots w_n) = \prod_{i=1}^n \mu(w_i) = q_0 \prod_{i=1}^n \mu(w_i) = q_0 \mu(w_1) \cdot \prod_{i=1}^n \mu(w_i) = q_1 \prod_{i=1}^n \mu(w_i) = q_n \in P$.
	\end{itemize}
\end{prv}


	}
	\def\addmacros#1{#1}
}

{
	\chap[2]{Algorithmes probabilistes}
	\minitoc
	\renewcommand{\cwd}{../cours/chap02/}
	\addmacros{
		\section{(Ne pas) être diagonalisable}

\begin{defn}
	Soit une matrice carrée $A$. On dit que $A$\/ est {\it diagonalisable}\/ s'il existe une matrice inversible~$P \in \mathrm{GL}_n(\mathds{K})$\/ telle que $P^{-1}\cdot A\cdot P$\/ est diagonale.
\end{defn}

\begin{exo}
	\begin{enumerate}
		\item Montrons que la matrice $B = {7\: 1\choose 0\:7}$\/ n'est pas diagonalisable.
			Par l'absurde : on suppose qu'il existe $P \in \mathrm{GL}_2(\R)$\/ et $(\lambda_1, \lambda_2) \in \R^2$\/ tels que \[
				P^{-1} \cdot B \cdot P = \begin{bmatrix}
					\lambda_1 & 0\\
					0&\lambda_2
				\end{bmatrix}
			.\] On applique la trace $\tr$\/ et le déterminant $\det$\/ :
			\begin{gather*}
				\tr(B) = \tr{\lambda_1\:0\choose 0\:\lambda_2} \quad\text{d'où}\quad \lambda_1 + \lambda_2 = 7 + 7 = 14 = \s\\
				\det(B) = \det{\lambda_1\:0\choose 0\:\lambda_2} \quad\text{d'où}\quad \lambda_1 \times \lambda_2 = 7 \times 7 = 49 = p
			\end{gather*}
			D'où $\lambda_1$\/ et $\lambda_2$\/ sont des solutions de l'équation $X^2 - \s X + p = 0$. Or
			\begin{align*}
				X^2 - \s X + p = 0 \iff& X^2 - 14X + 49 = 0\\
				\iff& (X-7)^2 = 0\\
				\iff& X = 7.
			\end{align*}
			D'où 
			\begin{align*}
				B = P P^{-1} B P P^{-1} = P \begin{pmatrix}
					7&0\\
					0&7
				\end{pmatrix} P^{-1} = P \cdot 7I_2\cdot P^{-1} = 7I_2.
			\end{align*}
			La matrice $B$\/ n'est donc pas diagonalisable.

			De même, montrons que la matrice $A$\/ n'est pas diagonalisable. On remarque que \[
				A \cdot \mat{1\\1\\1} = \begin{pmatrix}
					0&1&2\\
					1&0&2\\
					0&0&3
				\end{pmatrix} \begin{pmatrix}
					1\\1\\1
				\end{pmatrix} = \begin{pmatrix}
					3\\3\\3
				\end{pmatrix} = 3\begin{pmatrix}
					1\\1\\1
				\end{pmatrix} 
			.\] Ainsi, \[
				P^{-1}\cdot A\cdot P = \begin{pmatrix}
					3&0&0\\
					0&?&0\\
					0&0&?
				\end{pmatrix}\qquad\text{où}\qquad P = \begin{pmatrix}
					1&?&?\\
					1&?&?\\
					1&?&?
				\end{pmatrix}
			.\] De même, $A\left( \substack{1\\1\\0} \right) = 1 \times \left( \substack{1\\1\\0} \right)$. D'où \[
				P^{-1}\cdot A\cdot P = \begin{pmatrix}
					3&0&0\\
					0&1&0\\
					0&0&?
				\end{pmatrix}\qquad\text{où}\qquad P = \begin{pmatrix}
					1&1&?\\
					1&1&?\\
					1&0&?
				\end{pmatrix}
			.\] Finalement, on en conclut que \[
				P = \begin{pmatrix}
					3&0&0\\
					0&1&0\\
					0&0&-1
				\end{pmatrix} \qquad \text{et}\qquad P^{-1}\cdot A\cdot P = \begin{pmatrix}
					1&1&1\\
					1&1&-1\\
					1&0&0
				\end{pmatrix} = D
			.\]
			De plus, la matrice $P$\/ est inversible car $\det P \neq 0$.
		\item Pour calculer $A^n$, on pourrait chercher un polynôme annulateur $Q$\/ de $A$, et on exprime $X^n = Q \times T_n + R_n$, et donc $A^n = R_n(A)$.
			Mais, on peut également diagonaliser $A$\/ (si elle est diagonalisable).
			Ainsi,  \[
				D^n = (P^{-1}\cdot A\cdot P)^n = P^{-1}\cdot A\cdot \cancel P\cdot \cancel{P^{-1}} \cdot \ldots\cdot \cancel{P^{-1}} \cdot A \cdot P = P^{-1}\cdot  A^n\cdot P
			.\] D'où $A^n = P \cdot D^n \cdot P^{-1}$. Or, \[
				D^n = \begin{pmatrix}
					3&0&0\\
					0&1&0\\
					0&0&-1
				\end{pmatrix}^n = \begin{pmatrix}
					3^n&0&0\\
					0&1^n&0\\
					0&0&(-1)^n
				\end{pmatrix}
			.\]
			On calcule donc $A^{n}$\/ en calculant l'inverse de $P$\/ : \[
				A^n = \begin{pmatrix}
					1&1&1\\
					1&1&-1\\
					1&0&0
				\end{pmatrix} \begin{pmatrix}
					3^n&0&0\\
					0&1^n&0\\
					0&0&(-1)^n
				\end{pmatrix} \cdot P^{-1}
			.\]
		\item
			\begin{align*}
				\begin{rcases*}
					\hfill u_{n+1} = v_n + 2w_n\\
					\hfill v_{n+1} = u_n + 2w_n\\
					\hfill w_{n+1} = 3w_n
				\end{rcases*} \iff& \begin{pmatrix}
					u_{n+1}\\v_{n+1}\\w_{n+1}
				\end{pmatrix} = \begin{pmatrix}
					0&1&2\\
					1&0&2\\
					0&0&3
				\end{pmatrix} \begin{pmatrix}
					u_n\\ v_n\\ w_n
				\end{pmatrix}\\
				\iff& U_{n+1} = A\cdot U_n\\
				\iff& U'_{n+1} = D \cdot U'_{n}
			\end{align*}
			où $D = P^{-1} \cdot A \cdot P$, $U'_{n+1} = P\cdot U_{n+1}$\/ et $U'_n = P\cdot U_n$.
			\begin{align*}
				\phantom{\begin{rcases*}
					\hfill mm_{n+1} = v_n + 2w_n\\
					\hfill v_{n+1} = u_n + 2w_n\\
					\hfill w_{n+1} = 3w_n
				\end{rcases*}} \iff&
				\begin{pmatrix}
					u'_{n+1}\\v'_{n+1}\\w'_{n+1}
				\end{pmatrix} = \begin{pmatrix}
					3&0&0\\
					0&1&0\\
					0&0&-1
				\end{pmatrix} \cdot \begin{pmatrix}
					u'_n\\
					v'_n\\
					w'_n
				\end{pmatrix}\\
				\iff& \begin{cases}
					u'_{n+1} = 3u'_n\\
					v'_{n+1} = v'_n\\
					w'_{n+1} = -w'_n
				\end{cases}\\
				\iff& \begin{cases}
					u'_n = K\times  3^n\\
					v'_n = L\\
					w'_n = M \times (-1)^n
				\end{cases}
			\end{align*}
			Ainsi, \[
				\begin{pmatrix}
					u_n\\v_n\\w_n
				\end{pmatrix} = \underbrace{\begin{pmatrix}
					1&1&1\\
					1&1&-1\\
					1&0&0
				\end{pmatrix}}_P \cdot \begin{pmatrix}
					K\times 3^n\\
					L\\
					M\times (-1)^n
				\end{pmatrix}
			.\] D'où $u_n = K\cdot 3^n + L + M \cdot (-1)^n$, $v_n = K\times 3^n + L - M \cdot (-1)^n$\/ et $w_n = K\cdot 3^n$, où les constantes $K$, $L$\/ et $M$\/ sont des constantes fixées par les conditions initiales.
		\item
			\begin{align*}
				\begin{rcases*}
					\hfill x'(t) = y(t) + 2z(t)\\
					\hfill y'(t) = x(t) + 2z(t)\\
					\hfill z'(t) = 3z(t)
				\end{rcases*} \iff& \begin{pmatrix}
					x'(t)\\
					y'(t)\\
					z'(t)
				\end{pmatrix} = \begin{pmatrix}
					0&1&2\\
					1&0&2\\
					0&0&3
				\end{pmatrix} \cdot \begin{pmatrix}
					x(t)\\
					y(t)\\
					z(t)
				\end{pmatrix}\\
				\iff& X'(t) = A\cdot X(t)\\
				\iff& U'(t) = D \cdot U(t) \text{ avec } D = P^{-1} \cdot A\cdot P \text{ et } X(t) = P\cdot U(t)\\
				\iff& \begin{pmatrix}
					u'(t)\\
					v'(t)\\
					w'(t)
				\end{pmatrix} = \begin{pmatrix}
					3&0&0\\
					0&1&0\\
					0&0&-1
				\end{pmatrix} \cdot \begin{pmatrix}
					u(t)\\
					v(t)\\
					w(t)
				\end{pmatrix}\\
				\iff& \begin{cases}
					u'(t) = 3u(t)\\
					v'(t) = v(t)\\
					w'(t) = -w(t)
				\end{cases}\\
				\iff& \begin{cases}
					u(t) = K \cdot \mathrm{e}^{3t}\\
					v(t) = L \cdot \mathrm{e}^{t}\\
					w(t) = M \cdot \mathrm{e}^{-t}
				\end{cases}
			\end{align*}
			Ainsi \[
				\begin{pmatrix}
					x(t)\\
					y(t)\\
					z(t)
				\end{pmatrix} = \underbrace{\begin{pmatrix}
					1&1&1\\
					1&1&-1\\
					1&0&0
				\end{pmatrix}}_P \cdot \begin{pmatrix}
					K \times \mathrm{e}^{3t}\\
					L \cdot \mathrm{e}^{t}\\
					M \cdot \mathrm{e}^{-t}
				\end{pmatrix}
			.\] 
			D'où $x(t) = K\cdot \mathrm{e}^{3t} + L \cdot \mathrm{e}^{t} + M \cdot \mathrm{e}^{-t}$, $y(t) = K \cdot \mathrm{e}^{3t} + L \cdot \mathrm{e}^{t} - M \cdot \mathrm{e}^{-t}$\/ et $z(t) = K\cdot \mathrm{e}^{3t}$. Les constantes $K$, $L$\/ et $M$\/ peuvent être déterminées à partir des conditions initiales.
	\end{enumerate}
\end{exo}

\begin{rmkn}[équations différentielles]
	On considère l'équation différentielle $(*)$ : $x'(t) = \lambda \cdot x(t)$.
	Les fonctions $x : t \mapsto K\cdot \mathrm{e}^{\lambda t}$\/ sont des solutions de cette équation. On peut utiliser la méthode de {\sc Lagrange}\/ : la méthode de la~\guillemotleft~variation de la constante.~\guillemotright\@ On cherche des solutions sous la forme $x(t) = k(t) \cdot \mathrm{e}^{\lambda t}$ (vision du~physicien). D'où $k(t) = x(t) / \mathrm{e}^{\lambda t}$\/ (vision du mathématicien). De plus, $x'(t) = k'(t) \mathrm{e}^{\lambda t} + k(t) \lambda \mathrm{e}^{\lambda t}$.
	Ainsi, on injecte ce $k(t)$\/ dans l'équation différentielle :
	\begin{align*}
		(*) \iff& k'(t) \mathrm{e}^{\lambda t} + k(t) \lambda \mathrm{e}^{\lambda t} = \lambda k(t)\mathrm{e}^{\lambda t}\\
		\iff& k'(t) \mathrm{e}^{\lambda t} = 0\\
		\iff& k'(t) = 0\\
		\iff& \exists K \in \R\,\:k(t) = K.
	\end{align*}
	Les solutions trouvées dans l'exercice précédent sont donc les uniques solutions du système d'équations différentielles.

	De même, pour résoudre une équation différentielle avec 2\tsup{nd} membre de la forme \[
		(**) : \qquad x'(t) - \lambda \cdot x(t) = b(t)
	.\]
	La fonction $t \mapsto x(t)$\/ est une solution de l'équation {\sc sans}\/ 2\tsup{nd} membre si et seulement si \[
		\exists K \in \R,\:\forall t \in \R,\quad x(t) = K \cdot \mathrm{e}^{\lambda t}
	.\]
	\begin{center}
		\slshape Comment résoudre l'équation différentielle {\scshape avec}\/ 2\tsup{nd} membre si on connaît la solution générale de l'équation {\scshape sans}\/ 2\tsup{nd} membre ?
	\end{center}
	On utilise la méthode le la variation de la constante.
	Soit $x(t) = k(t) \cdot \mathrm{e}^{\lambda t}$. Ainsi, en injectant cette expression de $x$\/ dans l'équation $(**)$, on trouve
	\begin{align*}
		(**) \iff& k'(t) \mathrm{e}^{\lambda t} + k(t) \cdot \lambda \mathrm{e}^{\lambda t} = \lambda k(t) \mathrm{e}^{\lambda t} + b(t)\\
		\iff& k'(t) \mathrm{e}^{\lambda t} = b(t)\\
		\iff& k'(t) = b(t) \cdot \mathrm{e}^{-\lambda t}\\
		\iff& k(t) = \int_{0}^{t} b(u)\cdot \mathrm{e}^{-\lambda u}~\mathrm{d}u + K\\
		\iff& x(t) = \left( \int_{0}^{t} b(u) \cdot \mathrm{e}^{-\lambda u}~\mathrm{d}u + K \right) \mathrm{e}^{\lambda t}\\
		\iff& x(t) = \underbrace{\int_{0}^{t} b(u) \cdot \mathrm{e}^{\lambda (t-u)}~\mathrm{d}u}_{\text{solution particulière}} + \underbrace{K \cdot \mathrm{e}^{\lambda t}}_{\substack{\text{solution}\\\text{générale}\\\text{de $(*)$}}}.
	\end{align*}
\end{rmkn}

		\begin{exm}
	On pose $f$, le sinus cardinal :  \begin{align*}
		f: \R^* &\longrightarrow \R \\
		t &\longmapsto \frac{\sin t}{t}.
	\end{align*}
	\begin{figure}[H]
		\centering
		\begin{asy}
			import graph;
			size(10cm);
			draw((-10, 0) -- (10, 0), Arrow(TeXHead));
			draw((0, -3) -- (0, 5), Arrow(TeXHead));
			real f(real x) {
				if(x == 0) { return 3; }
				else {return 3*sin(x) / x;}
			}
			draw(graph(f, -10, 10), magenta);
		\end{asy}
		\caption{Sinus cardinal}
	\end{figure}

	La fonction $f$\/ est continue sur ${]0,8]}$\/ mais $\lim_{t\to 0} \frac{\sin t}{t} = 1$. D'où $\int_{0}^{8} \frac{\sin t}{t}~\mathrm{d}t$\/ est faussement impropre en $0$\/ et donc convergente.


	Mais attention ! On ne dit pas \guillemotleft~{\color{red}soit $f : t \mapsto \frac{1}{t}$. L'intégrale $\int_{8}^{+\infty} \frac{1}{t}~\mathrm{d}t$\/ est faussement impropre en $+\infty$\/ car $\lim_{t\to +\infty}\frac{1}{t} = 0$}.~\guillemotright
\end{exm}

\section{Intégrer les $\mathbf{\sim}$, $\po$, et \textit{O}}

\begin{thm}
	\hfill$\O$\hfill\null
\end{thm}

\begin{thm}
	Le 2.\ n'est pas la réciproque du 1.\ mais la contraposée.
\end{thm}

\begin{prop}
	\hfill$\O$\hfill\null
\end{prop}

\begin{exm}
	On considère l'intégrale $\int_{2}^{+\infty} \frac{1}{t^2+ \cos t}~\mathrm{d}t$, c'est une intégrale impropre en $+\infty$.
	On recherche un équivalent de $\frac{1}{t^2 + \cos t}$\/ en $+\infty$ : \[
		\frac{1}{t^2 + \cos t} \simi_{t\to +\infty} \frac{1}{t^2}
	\] qui ne change pas de signe. Or, $\int_{2}^{+\infty} \frac{1}{t^2}~\mathrm{d}t$\/ converge car c'est une intégrale de {\sc Riemann}\/ avec $\alpha = 2 > 1$.
	On en déduit que l'intégrale $I$\/ converge.

	On procède autrement : \[
		0 \le \frac{1}{t^2 + \cos t} \le \frac{1}{t^2 - 1}
	.\] Or, $\int_{2}^{+\infty} \frac{1}{t^2 - 1}~\mathrm{d}t$\/ converge car
	\begin{align*}
		\int_{2}^{x} \frac{1}{t^2 - 1}~\mathrm{d}t &= \int_{2}^{x} \left( \frac{\sfrac12}{t-1} - \frac{\sfrac12}{t+1} \right) ~\mathrm{d}t \\
		&= \frac{1}{2} \int_{2}^{x} \frac{1}{t-1}~\mathrm{d}t - \frac{1}{2}\int_{2}^{x} \frac{1}{t+1}~\mathrm{d}t \\
		&= \frac{1}{2} \Big[\ln|t-1|\Big]_2^x - \frac{1}{2}\Big[\ln |t+1|\Big]_2^x \\
	\end{align*}
	D'où \[
		\int_{2}^{x} \frac{1}{t^2 - 1}~\mathrm{d}t = \frac{1}{2} \left[ \ln\left| \frac{t-1}{t+1} \right| \right]_2^x = \frac{1}{2}\ln \left| \frac{x-1}{x+1} \right| + \frac{1}{2} \ln 3 \tendsto{x\to +\infty} \frac{1}{2} \ln 3
	.\] donc l'intégrale $I$\/ converge et $I \le \frac{1}{2} \ln_3$.
\end{exm}

\begin{exo}
	\begin{enumerate}
		\item L'intégrale $I = \int_{0}^{1} \frac{\sin t}{t^2}~\mathrm{d}t$\/ est impropre en 0. On utilise un équivalent : $\sin t \simi_{t\to 0} t$\/ qui ne change pas de signe. Or, $\int_{0}^{t} \frac{1}{t}~\mathrm{d}t$\/ diverge (par critère de {\sc Riemann}). Donc $I$\/ diverge.
			
			L'intégrale $J = \int_{1}^{+\infty} \sin \frac{1}{t}~\mathrm{d}t$\/ est généralisée en $+\infty$. On cherche un équivalent en $+\infty$\/ : \[
				\sin \frac{1}{t} \simi_{t\to +\infty} \frac{1}{t}
			\] qui ne change pas de signe. Or, $\int_{1}^{+\infty} \frac{1}{t}~\mathrm{d}t$\/ diverge par critère de {\sc Riemann}. On en déduit que $J$\/ diverge également.
		\item L'intégrale $\int_{0}^{+\infty} \frac{1}{t^2}~\mathrm{d}t$\/ est impropre, {\bf et}\/ en 0, {\bf et}\/ en $+\infty$. Le théorème ne marche donc pas.
			En effet $t\mapsto \frac{1}{t^2}$\/ n'est pas continue par morceaux en 0, ce qui était le cas pour $t\mapsto \frac{1}{1+t^2}$.
	\end{enumerate}
\end{exo}

\begin{rmkn}[Retour sur la {\sc remarque}\/ 5]
	L'intégrale $\int_{0}^{+\infty} \frac{1}{\ln(1+t)}~\mathrm{d}t$\/ est impropre en 0 {\bf et}\/ en $+\infty$. $\int_{0}^{+\infty} \frac{1}{\ln(1+t)}~\mathrm{d}t$\/ converge si et seulement si $\int_{0}^{7} \frac{1}{\ln(1+t)}~\mathrm{d}t$\/ {\bf et}\/ $\int_{7}^{+\infty} \frac{1}{\ln(1+t)}~\mathrm{d}t$\/ convergent.
	Et si elles convergent \[
		\int_{0}^{+\infty} \frac{1}{\ln(1+t)}~\mathrm{d}t = \int_{0}^{7} \frac{1}{\ln(1+t)}~\mathrm{d}t + \int_{7}^{+\infty} \frac{1}{\ln(1+t)}~\mathrm{d}t
	.\]
	On n'utilise pas deux barrières en même temps. Sinon, les intégrales doublement impropres peuvent, et converger, et diverger.
\end{rmkn}

\begin{prop}[avec $\sim$]
	Si $f(t) \simi_{t\to b} g(t)$\/ qui ne change pas de signe. Alors,
	\begin{itemize}
		\item ou bien $\ds\int_{a}^{b} f(t)~\mathrm{d}t$\/ et $\ds\int_{a}^{b} g(t)~\mathrm{d}t$\/ convergent et $\ds \int_{x}^{b} f(t)~\mathrm{d}t \simi_{x\to b} \int_{x}^{b} g(t)~\mathrm{d}t$.
		\item ou bien $\ds\int_{a}^{b} f(t)~\mathrm{d}t$\/ et $\ds\int_{a}^{b} g(t)~\mathrm{d}t$\/ divergent et $\ds\int_{a}^{x} f(t)~\mathrm{d}t \simi_{x\to b} \int_{a}^{x} g(t)~\mathrm{d}t$.
	\end{itemize}
	Cette proposition est équivalente à le {\sc lemme}\/ 13 sur les séries.
\end{prop}



		\begin{prop}
	La relation $\preceq$\/ est un \textit{pré-ordre} :
	\begin{itemize}
		\item $\preceq $\/ est réflective ;
		\item $\preceq $\/ est transitive.
	\end{itemize}
\end{prop}

\begin{prv}
	Soit $Q$\/ un problème de décision.
	\begin{itemize}
		\item $Q \preceq Q$\/ par la fonction identité, qui est totale et calculable.
		\item Soient $Q$, $R$\/ et $S$\/ trois problèmes de décision tels que $Q \preceq R$\/ et $R \preceq S$. Soit donc $f_1$\/ la réduction de $Q$\/ à $R$, et $f_2$\/ la réduction de $R$\/ à $S$. Soit $f = f_2 \circ f_1 : \mathcal{E}_Q \to \mathcal{E}_S$. La fonction $f$\/ est totale comme composée de fonctions totales, $f$\/ est calculable comme composée de fonctions calculables. De plus,
			\begin{align*}
				\forall e \in \mathcal{E}_Q,\qquad f(e) \in S^+ \iff& f_2(f_1(e)) \in S^+\\
				\iff& f_1(e) \in R^+\\
				\iff& e \in Q^+
			\end{align*}
	\end{itemize}
\end{prv}

\section{Classe \textbf{P} et \textbf{NP}}

Pour répondre à un problème, on peut le résoudre par des algorithmes plus ou moins rapides. Mais, l'objectif de cette section est de montrer que certains problèmes ne peuvent se résoudre que par des algorithmes lents, et que l'on ne peut pas faire mieux.

\begin{defn}
	Le modèle de calcul impose une représentation des entrées par chaînes de caractères. Cela induit donc une notion de \textit{taille d'entrée}, qui est la longueur de la chaîne de caractères.
	\index{taille d'entrée}
\end{defn}


\subsection{Complexité d'une machine}

\begin{defn}
	Étant donné une machine $\mathcal{M}$ et une entrée $w \in \Sigma^*$, on note $C^\mathcal{M}(w)$\/ le nombre d'opérations élémentaires effectuées lors de l'appel de $\mathcal{M}$\/ sur $w$. Lorsque $\smash{w \xrightarrow[\mathcal{M}]{} {\circlearrowleft}}$, on définit $C^\mathcal{M} = +\infty$.

	Pour $n \in \N$, on définit alors \[
		C^\mathcal{M}_n = \max \{ C^\mathcal{M}(w)  \mid w \in \Sigma^n \}
	.\]
	\index{machine!nombre d'opérations élémentaires!($C^\mathcal{M}(w)$)}
	\index{machine!nombre d'opérations élémentaires!maximal pour un mot de taille $n$\/ ($C^\mathcal{M}_n$)}
\end{defn}

\begin{rmk}
	On a, $\forall n \in \N$, $C_n^\mathcal{M} \in \bar{\N} = \N \cup \{+\infty\}$.
\end{rmk}

\begin{defn}
	Soit $f : \N\to \N$\/ une fonction totale et calculable. On note $\textsc{Time}(f)$\/ l'ensemble des machines $\mathcal{M}$\/ telles que
	\begin{itemize}
		\item $\mathcal{M}$\/ s'arrête sur toute entrée ;
		\item $\big(C_n^\mathcal{M}\big)_{n \in \N} = \mathcal{O}\big(\big(f(n)\big)_{n \in \N}\big)$.
	\end{itemize}
	\index{machine!ensemble $\textsc{Time}(f)$}
\end{defn}

\subsection{Classe \textbf{P}}

\begin{defn}
	On dit d'une machine $\mathcal{M}$\/ qu'elle est de \textit{complexité polynômiale} dès lors qu'il existe $k \in \N$\/ tel que $\mathcal{M} \in \textsc{Time}(n^k)$.
	\index{machine!de complexité polynômiale}
\end{defn}

\begin{defn}
	On dit d'une fonction (partielle ou non), qu'elle est \textit{calculable en temps polynômial} dès lors qu'il existe une machine $\mathcal{M}$\/ de complexité polynômiale la calculant.
	\index{fonction!calculable!en temps polynômial}
\end{defn}

\begin{exm}
	\begin{itemize}
		\item l'identité ($n \mapsto n$)
		\item la fonction successeur ($n\mapsto n+1$)
	\end{itemize}
\end{exm}


		\clearpage
		\setcounter{section}{0}		\renewcommand{\thesection}{\llap{Annexe }\thechapter.\Alph{section}}
		\renewcommand{\thesectionnum}{\Alph{section}}
		\section{Comment prouver la correction d'un programme ?}

Avec $\Sigma = \{a,b\}$. Comment montrer qu'un mot a au moins un $a$\/ et un nombre pair de $b$.

\begin{figure}[H]
	\centering
	\tikzfig{annexe-a-automate-1}
	\caption{Automate reconnaissant les mots valides}
\end{figure}

On veut montrer que \[
	P_w : \text{\guillemotleft~}\forall w \in \Sigma^*,\, \forall q \in \mathcal{Q},\:(\text{il existe une exécution par $w$\/ menant à $q$}) \iff w \text{ satisfait } I_q\text{~\guillemotright}
\]
où \[
	I_{\substack{(v,\\\vrt\in\\\mathds{B}}\substack{r)\\\vrt\in\\\{0,1\}}} : \quad
		(|w|_a \ge 1 \iff v)\:\text{et}\:(r = |w|_b\:\text{mod}\:2)
.\]
On le montre par récurrence sur la longueur de $w$\/ : 

\begin{itemize}
	\item[``$\implies$'']
		\begin{itemize}
			\item Pour $w = \varepsilon$, alors  montrons que $\forall q \in \mathcal{Q}$, il existe un exécution menant à $q$\/ étiquetée par $w$\/ (noté $\xrightarrow[\mathcal{A}]{w}q$) si et seulement si $w$\/ satisfait $I_q$.
				\begin{itemize}
					\item $\xrightarrow[\mathcal{A}]{\varepsilon}({\bfm F}, 0)$\/ est vrai, de plus $\varepsilon$\/ satisfait $I_{({\bfm F}, 0)}$\/ ;
					\item sinon si $q \neq ({\bfm F}, 0)$, alors $\xrightarrow[\mathcal{A}]{\varepsilon}q$\/ est fausse, de plus $\varepsilon$\/ ne satisfait pas $I_q$.
				\end{itemize}
			\item Supposons maintenant $P_w$\/ vrai pour tout mot $w$\/ de taille $n$. Soit $w = w_1\ldots w_nw_{n+1}$\/ un mot de taille $n+1$. Notons $\ubar{w} = w_1\ldots w_n$.
				Montrons que $P_w$\/ est vrai. Soit $q \in \mathcal{Q}$. Supposons $\xrightarrow[\mathcal{A}]{w} q$.
				\begin{itemize}
					\item Si $q = ({\bfm F}, 0)$\/ et $w_{n+1} = b$. On a donc $\xrightarrow[\mathcal{A}]{\ubar{w}}({\bfm F},1)$, et, par hypothèse de récurrence, $\ubar{w}$\/ satisfait. Donc $|\ubar{w}|_a = 0$\/ et $|\ubar{w}|_b \equiv 1 \mod 2$\/ donc $|w|_a = 0$\/ et $|w|_b \equiv 0 \mod 2$\/ donc $w$\/ satisfait $I_{({\bfm F}, 0)}$.
					\item De même pour les autres cas.
				\end{itemize}
		\end{itemize}
	\item[``$\impliedby$''] Réciproquement, supposons que $w$\/ satisfait $I_q$.
		\begin{itemize}
			\item Si $w = ({\bfm V},0)$\/ et $w_{n+1} = a$. Alors,
				\begin{itemize}
					\item si $|\ubar{w}|_a = 0$, alors $\ubar{w} $\/ satisfait $I_{({\bfm F},0)}$. Par hypothèse de récurrence, on a donc $\xrightarrow[\mathcal{A}]{\ubar{w}}({\bfm F},0)$\/ et donc $\xrightarrow[\mathcal{A}]{w}({\bfm V},0)$.
					\item si $|\ubar{w}|_b \ge 1$, alors $\ubar{w}$\/ satisfait $I_{({\bfm V},0)}$\/ donc $\xrightarrow[\mathcal{A}]{\ubar{w}}({\bfm V},0)$\/ et donc $\xrightarrow[\mathcal{A}]{\ubar{w}}({\bfm V},0)$.
				\end{itemize}
			\item De même pour les autres cas.
		\end{itemize}
\end{itemize}
On a donc bien \[
	\forall w \in \Sigma^*,\forall q \in \mathcal{Q},\:\xrightarrow[\mathcal{A}]w q \iff w \text{ satisfait } I_q
.\] Finalement,
\begin{align*}
	\mathcal{L}(\mathcal{A}) &= \{w \in \Sigma^* \mid \exists f \in F,\:\xrightarrow[\mathcal{A}]{w} f\} \\
	&= \{w \in \Sigma^*  \mid \xrightarrow[\mathcal{A}]w ({\bfm V},0)\}  \\
	&= \{w \in \Sigma^*  \mid w \text{ satisfait } I_{({\bfm V},0)}\} \\
	&= \{w \in \Sigma^*  \mid |w|_a \ge 1 \text{ et } |w|_b \equiv 0 \mod 2\} \\
\end{align*}

	}
	\def\addmacros#1{#1}
}

{
	\chap[3]{Apprentissage}
	\minitoc
	\renewcommand{\cwd}{../cours/chap03/}
	\addmacros{
		\section{(Ne pas) être diagonalisable}

\begin{defn}
	Soit une matrice carrée $A$. On dit que $A$\/ est {\it diagonalisable}\/ s'il existe une matrice inversible~$P \in \mathrm{GL}_n(\mathds{K})$\/ telle que $P^{-1}\cdot A\cdot P$\/ est diagonale.
\end{defn}

\begin{exo}
	\begin{enumerate}
		\item Montrons que la matrice $B = {7\: 1\choose 0\:7}$\/ n'est pas diagonalisable.
			Par l'absurde : on suppose qu'il existe $P \in \mathrm{GL}_2(\R)$\/ et $(\lambda_1, \lambda_2) \in \R^2$\/ tels que \[
				P^{-1} \cdot B \cdot P = \begin{bmatrix}
					\lambda_1 & 0\\
					0&\lambda_2
				\end{bmatrix}
			.\] On applique la trace $\tr$\/ et le déterminant $\det$\/ :
			\begin{gather*}
				\tr(B) = \tr{\lambda_1\:0\choose 0\:\lambda_2} \quad\text{d'où}\quad \lambda_1 + \lambda_2 = 7 + 7 = 14 = \s\\
				\det(B) = \det{\lambda_1\:0\choose 0\:\lambda_2} \quad\text{d'où}\quad \lambda_1 \times \lambda_2 = 7 \times 7 = 49 = p
			\end{gather*}
			D'où $\lambda_1$\/ et $\lambda_2$\/ sont des solutions de l'équation $X^2 - \s X + p = 0$. Or
			\begin{align*}
				X^2 - \s X + p = 0 \iff& X^2 - 14X + 49 = 0\\
				\iff& (X-7)^2 = 0\\
				\iff& X = 7.
			\end{align*}
			D'où 
			\begin{align*}
				B = P P^{-1} B P P^{-1} = P \begin{pmatrix}
					7&0\\
					0&7
				\end{pmatrix} P^{-1} = P \cdot 7I_2\cdot P^{-1} = 7I_2.
			\end{align*}
			La matrice $B$\/ n'est donc pas diagonalisable.

			De même, montrons que la matrice $A$\/ n'est pas diagonalisable. On remarque que \[
				A \cdot \mat{1\\1\\1} = \begin{pmatrix}
					0&1&2\\
					1&0&2\\
					0&0&3
				\end{pmatrix} \begin{pmatrix}
					1\\1\\1
				\end{pmatrix} = \begin{pmatrix}
					3\\3\\3
				\end{pmatrix} = 3\begin{pmatrix}
					1\\1\\1
				\end{pmatrix} 
			.\] Ainsi, \[
				P^{-1}\cdot A\cdot P = \begin{pmatrix}
					3&0&0\\
					0&?&0\\
					0&0&?
				\end{pmatrix}\qquad\text{où}\qquad P = \begin{pmatrix}
					1&?&?\\
					1&?&?\\
					1&?&?
				\end{pmatrix}
			.\] De même, $A\left( \substack{1\\1\\0} \right) = 1 \times \left( \substack{1\\1\\0} \right)$. D'où \[
				P^{-1}\cdot A\cdot P = \begin{pmatrix}
					3&0&0\\
					0&1&0\\
					0&0&?
				\end{pmatrix}\qquad\text{où}\qquad P = \begin{pmatrix}
					1&1&?\\
					1&1&?\\
					1&0&?
				\end{pmatrix}
			.\] Finalement, on en conclut que \[
				P = \begin{pmatrix}
					3&0&0\\
					0&1&0\\
					0&0&-1
				\end{pmatrix} \qquad \text{et}\qquad P^{-1}\cdot A\cdot P = \begin{pmatrix}
					1&1&1\\
					1&1&-1\\
					1&0&0
				\end{pmatrix} = D
			.\]
			De plus, la matrice $P$\/ est inversible car $\det P \neq 0$.
		\item Pour calculer $A^n$, on pourrait chercher un polynôme annulateur $Q$\/ de $A$, et on exprime $X^n = Q \times T_n + R_n$, et donc $A^n = R_n(A)$.
			Mais, on peut également diagonaliser $A$\/ (si elle est diagonalisable).
			Ainsi,  \[
				D^n = (P^{-1}\cdot A\cdot P)^n = P^{-1}\cdot A\cdot \cancel P\cdot \cancel{P^{-1}} \cdot \ldots\cdot \cancel{P^{-1}} \cdot A \cdot P = P^{-1}\cdot  A^n\cdot P
			.\] D'où $A^n = P \cdot D^n \cdot P^{-1}$. Or, \[
				D^n = \begin{pmatrix}
					3&0&0\\
					0&1&0\\
					0&0&-1
				\end{pmatrix}^n = \begin{pmatrix}
					3^n&0&0\\
					0&1^n&0\\
					0&0&(-1)^n
				\end{pmatrix}
			.\]
			On calcule donc $A^{n}$\/ en calculant l'inverse de $P$\/ : \[
				A^n = \begin{pmatrix}
					1&1&1\\
					1&1&-1\\
					1&0&0
				\end{pmatrix} \begin{pmatrix}
					3^n&0&0\\
					0&1^n&0\\
					0&0&(-1)^n
				\end{pmatrix} \cdot P^{-1}
			.\]
		\item
			\begin{align*}
				\begin{rcases*}
					\hfill u_{n+1} = v_n + 2w_n\\
					\hfill v_{n+1} = u_n + 2w_n\\
					\hfill w_{n+1} = 3w_n
				\end{rcases*} \iff& \begin{pmatrix}
					u_{n+1}\\v_{n+1}\\w_{n+1}
				\end{pmatrix} = \begin{pmatrix}
					0&1&2\\
					1&0&2\\
					0&0&3
				\end{pmatrix} \begin{pmatrix}
					u_n\\ v_n\\ w_n
				\end{pmatrix}\\
				\iff& U_{n+1} = A\cdot U_n\\
				\iff& U'_{n+1} = D \cdot U'_{n}
			\end{align*}
			où $D = P^{-1} \cdot A \cdot P$, $U'_{n+1} = P\cdot U_{n+1}$\/ et $U'_n = P\cdot U_n$.
			\begin{align*}
				\phantom{\begin{rcases*}
					\hfill mm_{n+1} = v_n + 2w_n\\
					\hfill v_{n+1} = u_n + 2w_n\\
					\hfill w_{n+1} = 3w_n
				\end{rcases*}} \iff&
				\begin{pmatrix}
					u'_{n+1}\\v'_{n+1}\\w'_{n+1}
				\end{pmatrix} = \begin{pmatrix}
					3&0&0\\
					0&1&0\\
					0&0&-1
				\end{pmatrix} \cdot \begin{pmatrix}
					u'_n\\
					v'_n\\
					w'_n
				\end{pmatrix}\\
				\iff& \begin{cases}
					u'_{n+1} = 3u'_n\\
					v'_{n+1} = v'_n\\
					w'_{n+1} = -w'_n
				\end{cases}\\
				\iff& \begin{cases}
					u'_n = K\times  3^n\\
					v'_n = L\\
					w'_n = M \times (-1)^n
				\end{cases}
			\end{align*}
			Ainsi, \[
				\begin{pmatrix}
					u_n\\v_n\\w_n
				\end{pmatrix} = \underbrace{\begin{pmatrix}
					1&1&1\\
					1&1&-1\\
					1&0&0
				\end{pmatrix}}_P \cdot \begin{pmatrix}
					K\times 3^n\\
					L\\
					M\times (-1)^n
				\end{pmatrix}
			.\] D'où $u_n = K\cdot 3^n + L + M \cdot (-1)^n$, $v_n = K\times 3^n + L - M \cdot (-1)^n$\/ et $w_n = K\cdot 3^n$, où les constantes $K$, $L$\/ et $M$\/ sont des constantes fixées par les conditions initiales.
		\item
			\begin{align*}
				\begin{rcases*}
					\hfill x'(t) = y(t) + 2z(t)\\
					\hfill y'(t) = x(t) + 2z(t)\\
					\hfill z'(t) = 3z(t)
				\end{rcases*} \iff& \begin{pmatrix}
					x'(t)\\
					y'(t)\\
					z'(t)
				\end{pmatrix} = \begin{pmatrix}
					0&1&2\\
					1&0&2\\
					0&0&3
				\end{pmatrix} \cdot \begin{pmatrix}
					x(t)\\
					y(t)\\
					z(t)
				\end{pmatrix}\\
				\iff& X'(t) = A\cdot X(t)\\
				\iff& U'(t) = D \cdot U(t) \text{ avec } D = P^{-1} \cdot A\cdot P \text{ et } X(t) = P\cdot U(t)\\
				\iff& \begin{pmatrix}
					u'(t)\\
					v'(t)\\
					w'(t)
				\end{pmatrix} = \begin{pmatrix}
					3&0&0\\
					0&1&0\\
					0&0&-1
				\end{pmatrix} \cdot \begin{pmatrix}
					u(t)\\
					v(t)\\
					w(t)
				\end{pmatrix}\\
				\iff& \begin{cases}
					u'(t) = 3u(t)\\
					v'(t) = v(t)\\
					w'(t) = -w(t)
				\end{cases}\\
				\iff& \begin{cases}
					u(t) = K \cdot \mathrm{e}^{3t}\\
					v(t) = L \cdot \mathrm{e}^{t}\\
					w(t) = M \cdot \mathrm{e}^{-t}
				\end{cases}
			\end{align*}
			Ainsi \[
				\begin{pmatrix}
					x(t)\\
					y(t)\\
					z(t)
				\end{pmatrix} = \underbrace{\begin{pmatrix}
					1&1&1\\
					1&1&-1\\
					1&0&0
				\end{pmatrix}}_P \cdot \begin{pmatrix}
					K \times \mathrm{e}^{3t}\\
					L \cdot \mathrm{e}^{t}\\
					M \cdot \mathrm{e}^{-t}
				\end{pmatrix}
			.\] 
			D'où $x(t) = K\cdot \mathrm{e}^{3t} + L \cdot \mathrm{e}^{t} + M \cdot \mathrm{e}^{-t}$, $y(t) = K \cdot \mathrm{e}^{3t} + L \cdot \mathrm{e}^{t} - M \cdot \mathrm{e}^{-t}$\/ et $z(t) = K\cdot \mathrm{e}^{3t}$. Les constantes $K$, $L$\/ et $M$\/ peuvent être déterminées à partir des conditions initiales.
	\end{enumerate}
\end{exo}

\begin{rmkn}[équations différentielles]
	On considère l'équation différentielle $(*)$ : $x'(t) = \lambda \cdot x(t)$.
	Les fonctions $x : t \mapsto K\cdot \mathrm{e}^{\lambda t}$\/ sont des solutions de cette équation. On peut utiliser la méthode de {\sc Lagrange}\/ : la méthode de la~\guillemotleft~variation de la constante.~\guillemotright\@ On cherche des solutions sous la forme $x(t) = k(t) \cdot \mathrm{e}^{\lambda t}$ (vision du~physicien). D'où $k(t) = x(t) / \mathrm{e}^{\lambda t}$\/ (vision du mathématicien). De plus, $x'(t) = k'(t) \mathrm{e}^{\lambda t} + k(t) \lambda \mathrm{e}^{\lambda t}$.
	Ainsi, on injecte ce $k(t)$\/ dans l'équation différentielle :
	\begin{align*}
		(*) \iff& k'(t) \mathrm{e}^{\lambda t} + k(t) \lambda \mathrm{e}^{\lambda t} = \lambda k(t)\mathrm{e}^{\lambda t}\\
		\iff& k'(t) \mathrm{e}^{\lambda t} = 0\\
		\iff& k'(t) = 0\\
		\iff& \exists K \in \R\,\:k(t) = K.
	\end{align*}
	Les solutions trouvées dans l'exercice précédent sont donc les uniques solutions du système d'équations différentielles.

	De même, pour résoudre une équation différentielle avec 2\tsup{nd} membre de la forme \[
		(**) : \qquad x'(t) - \lambda \cdot x(t) = b(t)
	.\]
	La fonction $t \mapsto x(t)$\/ est une solution de l'équation {\sc sans}\/ 2\tsup{nd} membre si et seulement si \[
		\exists K \in \R,\:\forall t \in \R,\quad x(t) = K \cdot \mathrm{e}^{\lambda t}
	.\]
	\begin{center}
		\slshape Comment résoudre l'équation différentielle {\scshape avec}\/ 2\tsup{nd} membre si on connaît la solution générale de l'équation {\scshape sans}\/ 2\tsup{nd} membre ?
	\end{center}
	On utilise la méthode le la variation de la constante.
	Soit $x(t) = k(t) \cdot \mathrm{e}^{\lambda t}$. Ainsi, en injectant cette expression de $x$\/ dans l'équation $(**)$, on trouve
	\begin{align*}
		(**) \iff& k'(t) \mathrm{e}^{\lambda t} + k(t) \cdot \lambda \mathrm{e}^{\lambda t} = \lambda k(t) \mathrm{e}^{\lambda t} + b(t)\\
		\iff& k'(t) \mathrm{e}^{\lambda t} = b(t)\\
		\iff& k'(t) = b(t) \cdot \mathrm{e}^{-\lambda t}\\
		\iff& k(t) = \int_{0}^{t} b(u)\cdot \mathrm{e}^{-\lambda u}~\mathrm{d}u + K\\
		\iff& x(t) = \left( \int_{0}^{t} b(u) \cdot \mathrm{e}^{-\lambda u}~\mathrm{d}u + K \right) \mathrm{e}^{\lambda t}\\
		\iff& x(t) = \underbrace{\int_{0}^{t} b(u) \cdot \mathrm{e}^{\lambda (t-u)}~\mathrm{d}u}_{\text{solution particulière}} + \underbrace{K \cdot \mathrm{e}^{\lambda t}}_{\substack{\text{solution}\\\text{générale}\\\text{de $(*)$}}}.
	\end{align*}
\end{rmkn}

		\begin{exm}
	On pose $f$, le sinus cardinal :  \begin{align*}
		f: \R^* &\longrightarrow \R \\
		t &\longmapsto \frac{\sin t}{t}.
	\end{align*}
	\begin{figure}[H]
		\centering
		\begin{asy}
			import graph;
			size(10cm);
			draw((-10, 0) -- (10, 0), Arrow(TeXHead));
			draw((0, -3) -- (0, 5), Arrow(TeXHead));
			real f(real x) {
				if(x == 0) { return 3; }
				else {return 3*sin(x) / x;}
			}
			draw(graph(f, -10, 10), magenta);
		\end{asy}
		\caption{Sinus cardinal}
	\end{figure}

	La fonction $f$\/ est continue sur ${]0,8]}$\/ mais $\lim_{t\to 0} \frac{\sin t}{t} = 1$. D'où $\int_{0}^{8} \frac{\sin t}{t}~\mathrm{d}t$\/ est faussement impropre en $0$\/ et donc convergente.


	Mais attention ! On ne dit pas \guillemotleft~{\color{red}soit $f : t \mapsto \frac{1}{t}$. L'intégrale $\int_{8}^{+\infty} \frac{1}{t}~\mathrm{d}t$\/ est faussement impropre en $+\infty$\/ car $\lim_{t\to +\infty}\frac{1}{t} = 0$}.~\guillemotright
\end{exm}

\section{Intégrer les $\mathbf{\sim}$, $\po$, et \textit{O}}

\begin{thm}
	\hfill$\O$\hfill\null
\end{thm}

\begin{thm}
	Le 2.\ n'est pas la réciproque du 1.\ mais la contraposée.
\end{thm}

\begin{prop}
	\hfill$\O$\hfill\null
\end{prop}

\begin{exm}
	On considère l'intégrale $\int_{2}^{+\infty} \frac{1}{t^2+ \cos t}~\mathrm{d}t$, c'est une intégrale impropre en $+\infty$.
	On recherche un équivalent de $\frac{1}{t^2 + \cos t}$\/ en $+\infty$ : \[
		\frac{1}{t^2 + \cos t} \simi_{t\to +\infty} \frac{1}{t^2}
	\] qui ne change pas de signe. Or, $\int_{2}^{+\infty} \frac{1}{t^2}~\mathrm{d}t$\/ converge car c'est une intégrale de {\sc Riemann}\/ avec $\alpha = 2 > 1$.
	On en déduit que l'intégrale $I$\/ converge.

	On procède autrement : \[
		0 \le \frac{1}{t^2 + \cos t} \le \frac{1}{t^2 - 1}
	.\] Or, $\int_{2}^{+\infty} \frac{1}{t^2 - 1}~\mathrm{d}t$\/ converge car
	\begin{align*}
		\int_{2}^{x} \frac{1}{t^2 - 1}~\mathrm{d}t &= \int_{2}^{x} \left( \frac{\sfrac12}{t-1} - \frac{\sfrac12}{t+1} \right) ~\mathrm{d}t \\
		&= \frac{1}{2} \int_{2}^{x} \frac{1}{t-1}~\mathrm{d}t - \frac{1}{2}\int_{2}^{x} \frac{1}{t+1}~\mathrm{d}t \\
		&= \frac{1}{2} \Big[\ln|t-1|\Big]_2^x - \frac{1}{2}\Big[\ln |t+1|\Big]_2^x \\
	\end{align*}
	D'où \[
		\int_{2}^{x} \frac{1}{t^2 - 1}~\mathrm{d}t = \frac{1}{2} \left[ \ln\left| \frac{t-1}{t+1} \right| \right]_2^x = \frac{1}{2}\ln \left| \frac{x-1}{x+1} \right| + \frac{1}{2} \ln 3 \tendsto{x\to +\infty} \frac{1}{2} \ln 3
	.\] donc l'intégrale $I$\/ converge et $I \le \frac{1}{2} \ln_3$.
\end{exm}

\begin{exo}
	\begin{enumerate}
		\item L'intégrale $I = \int_{0}^{1} \frac{\sin t}{t^2}~\mathrm{d}t$\/ est impropre en 0. On utilise un équivalent : $\sin t \simi_{t\to 0} t$\/ qui ne change pas de signe. Or, $\int_{0}^{t} \frac{1}{t}~\mathrm{d}t$\/ diverge (par critère de {\sc Riemann}). Donc $I$\/ diverge.
			
			L'intégrale $J = \int_{1}^{+\infty} \sin \frac{1}{t}~\mathrm{d}t$\/ est généralisée en $+\infty$. On cherche un équivalent en $+\infty$\/ : \[
				\sin \frac{1}{t} \simi_{t\to +\infty} \frac{1}{t}
			\] qui ne change pas de signe. Or, $\int_{1}^{+\infty} \frac{1}{t}~\mathrm{d}t$\/ diverge par critère de {\sc Riemann}. On en déduit que $J$\/ diverge également.
		\item L'intégrale $\int_{0}^{+\infty} \frac{1}{t^2}~\mathrm{d}t$\/ est impropre, {\bf et}\/ en 0, {\bf et}\/ en $+\infty$. Le théorème ne marche donc pas.
			En effet $t\mapsto \frac{1}{t^2}$\/ n'est pas continue par morceaux en 0, ce qui était le cas pour $t\mapsto \frac{1}{1+t^2}$.
	\end{enumerate}
\end{exo}

\begin{rmkn}[Retour sur la {\sc remarque}\/ 5]
	L'intégrale $\int_{0}^{+\infty} \frac{1}{\ln(1+t)}~\mathrm{d}t$\/ est impropre en 0 {\bf et}\/ en $+\infty$. $\int_{0}^{+\infty} \frac{1}{\ln(1+t)}~\mathrm{d}t$\/ converge si et seulement si $\int_{0}^{7} \frac{1}{\ln(1+t)}~\mathrm{d}t$\/ {\bf et}\/ $\int_{7}^{+\infty} \frac{1}{\ln(1+t)}~\mathrm{d}t$\/ convergent.
	Et si elles convergent \[
		\int_{0}^{+\infty} \frac{1}{\ln(1+t)}~\mathrm{d}t = \int_{0}^{7} \frac{1}{\ln(1+t)}~\mathrm{d}t + \int_{7}^{+\infty} \frac{1}{\ln(1+t)}~\mathrm{d}t
	.\]
	On n'utilise pas deux barrières en même temps. Sinon, les intégrales doublement impropres peuvent, et converger, et diverger.
\end{rmkn}

\begin{prop}[avec $\sim$]
	Si $f(t) \simi_{t\to b} g(t)$\/ qui ne change pas de signe. Alors,
	\begin{itemize}
		\item ou bien $\ds\int_{a}^{b} f(t)~\mathrm{d}t$\/ et $\ds\int_{a}^{b} g(t)~\mathrm{d}t$\/ convergent et $\ds \int_{x}^{b} f(t)~\mathrm{d}t \simi_{x\to b} \int_{x}^{b} g(t)~\mathrm{d}t$.
		\item ou bien $\ds\int_{a}^{b} f(t)~\mathrm{d}t$\/ et $\ds\int_{a}^{b} g(t)~\mathrm{d}t$\/ divergent et $\ds\int_{a}^{x} f(t)~\mathrm{d}t \simi_{x\to b} \int_{a}^{x} g(t)~\mathrm{d}t$.
	\end{itemize}
	Cette proposition est équivalente à le {\sc lemme}\/ 13 sur les séries.
\end{prop}



		\begin{prop}
	La relation $\preceq$\/ est un \textit{pré-ordre} :
	\begin{itemize}
		\item $\preceq $\/ est réflective ;
		\item $\preceq $\/ est transitive.
	\end{itemize}
\end{prop}

\begin{prv}
	Soit $Q$\/ un problème de décision.
	\begin{itemize}
		\item $Q \preceq Q$\/ par la fonction identité, qui est totale et calculable.
		\item Soient $Q$, $R$\/ et $S$\/ trois problèmes de décision tels que $Q \preceq R$\/ et $R \preceq S$. Soit donc $f_1$\/ la réduction de $Q$\/ à $R$, et $f_2$\/ la réduction de $R$\/ à $S$. Soit $f = f_2 \circ f_1 : \mathcal{E}_Q \to \mathcal{E}_S$. La fonction $f$\/ est totale comme composée de fonctions totales, $f$\/ est calculable comme composée de fonctions calculables. De plus,
			\begin{align*}
				\forall e \in \mathcal{E}_Q,\qquad f(e) \in S^+ \iff& f_2(f_1(e)) \in S^+\\
				\iff& f_1(e) \in R^+\\
				\iff& e \in Q^+
			\end{align*}
	\end{itemize}
\end{prv}

\section{Classe \textbf{P} et \textbf{NP}}

Pour répondre à un problème, on peut le résoudre par des algorithmes plus ou moins rapides. Mais, l'objectif de cette section est de montrer que certains problèmes ne peuvent se résoudre que par des algorithmes lents, et que l'on ne peut pas faire mieux.

\begin{defn}
	Le modèle de calcul impose une représentation des entrées par chaînes de caractères. Cela induit donc une notion de \textit{taille d'entrée}, qui est la longueur de la chaîne de caractères.
	\index{taille d'entrée}
\end{defn}


\subsection{Complexité d'une machine}

\begin{defn}
	Étant donné une machine $\mathcal{M}$ et une entrée $w \in \Sigma^*$, on note $C^\mathcal{M}(w)$\/ le nombre d'opérations élémentaires effectuées lors de l'appel de $\mathcal{M}$\/ sur $w$. Lorsque $\smash{w \xrightarrow[\mathcal{M}]{} {\circlearrowleft}}$, on définit $C^\mathcal{M} = +\infty$.

	Pour $n \in \N$, on définit alors \[
		C^\mathcal{M}_n = \max \{ C^\mathcal{M}(w)  \mid w \in \Sigma^n \}
	.\]
	\index{machine!nombre d'opérations élémentaires!($C^\mathcal{M}(w)$)}
	\index{machine!nombre d'opérations élémentaires!maximal pour un mot de taille $n$\/ ($C^\mathcal{M}_n$)}
\end{defn}

\begin{rmk}
	On a, $\forall n \in \N$, $C_n^\mathcal{M} \in \bar{\N} = \N \cup \{+\infty\}$.
\end{rmk}

\begin{defn}
	Soit $f : \N\to \N$\/ une fonction totale et calculable. On note $\textsc{Time}(f)$\/ l'ensemble des machines $\mathcal{M}$\/ telles que
	\begin{itemize}
		\item $\mathcal{M}$\/ s'arrête sur toute entrée ;
		\item $\big(C_n^\mathcal{M}\big)_{n \in \N} = \mathcal{O}\big(\big(f(n)\big)_{n \in \N}\big)$.
	\end{itemize}
	\index{machine!ensemble $\textsc{Time}(f)$}
\end{defn}

\subsection{Classe \textbf{P}}

\begin{defn}
	On dit d'une machine $\mathcal{M}$\/ qu'elle est de \textit{complexité polynômiale} dès lors qu'il existe $k \in \N$\/ tel que $\mathcal{M} \in \textsc{Time}(n^k)$.
	\index{machine!de complexité polynômiale}
\end{defn}

\begin{defn}
	On dit d'une fonction (partielle ou non), qu'elle est \textit{calculable en temps polynômial} dès lors qu'il existe une machine $\mathcal{M}$\/ de complexité polynômiale la calculant.
	\index{fonction!calculable!en temps polynômial}
\end{defn}

\begin{exm}
	\begin{itemize}
		\item l'identité ($n \mapsto n$)
		\item la fonction successeur ($n\mapsto n+1$)
	\end{itemize}
\end{exm}


	}
	\def\addmacros#1{#1}
}

{
	\chap[4]{Calculabilité, Décidabilité, Complexité}
	\minitoc
	\renewcommand{\cwd}{../cours/chap04/}
	\addmacros{
		\section{(Ne pas) être diagonalisable}

\begin{defn}
	Soit une matrice carrée $A$. On dit que $A$\/ est {\it diagonalisable}\/ s'il existe une matrice inversible~$P \in \mathrm{GL}_n(\mathds{K})$\/ telle que $P^{-1}\cdot A\cdot P$\/ est diagonale.
\end{defn}

\begin{exo}
	\begin{enumerate}
		\item Montrons que la matrice $B = {7\: 1\choose 0\:7}$\/ n'est pas diagonalisable.
			Par l'absurde : on suppose qu'il existe $P \in \mathrm{GL}_2(\R)$\/ et $(\lambda_1, \lambda_2) \in \R^2$\/ tels que \[
				P^{-1} \cdot B \cdot P = \begin{bmatrix}
					\lambda_1 & 0\\
					0&\lambda_2
				\end{bmatrix}
			.\] On applique la trace $\tr$\/ et le déterminant $\det$\/ :
			\begin{gather*}
				\tr(B) = \tr{\lambda_1\:0\choose 0\:\lambda_2} \quad\text{d'où}\quad \lambda_1 + \lambda_2 = 7 + 7 = 14 = \s\\
				\det(B) = \det{\lambda_1\:0\choose 0\:\lambda_2} \quad\text{d'où}\quad \lambda_1 \times \lambda_2 = 7 \times 7 = 49 = p
			\end{gather*}
			D'où $\lambda_1$\/ et $\lambda_2$\/ sont des solutions de l'équation $X^2 - \s X + p = 0$. Or
			\begin{align*}
				X^2 - \s X + p = 0 \iff& X^2 - 14X + 49 = 0\\
				\iff& (X-7)^2 = 0\\
				\iff& X = 7.
			\end{align*}
			D'où 
			\begin{align*}
				B = P P^{-1} B P P^{-1} = P \begin{pmatrix}
					7&0\\
					0&7
				\end{pmatrix} P^{-1} = P \cdot 7I_2\cdot P^{-1} = 7I_2.
			\end{align*}
			La matrice $B$\/ n'est donc pas diagonalisable.

			De même, montrons que la matrice $A$\/ n'est pas diagonalisable. On remarque que \[
				A \cdot \mat{1\\1\\1} = \begin{pmatrix}
					0&1&2\\
					1&0&2\\
					0&0&3
				\end{pmatrix} \begin{pmatrix}
					1\\1\\1
				\end{pmatrix} = \begin{pmatrix}
					3\\3\\3
				\end{pmatrix} = 3\begin{pmatrix}
					1\\1\\1
				\end{pmatrix} 
			.\] Ainsi, \[
				P^{-1}\cdot A\cdot P = \begin{pmatrix}
					3&0&0\\
					0&?&0\\
					0&0&?
				\end{pmatrix}\qquad\text{où}\qquad P = \begin{pmatrix}
					1&?&?\\
					1&?&?\\
					1&?&?
				\end{pmatrix}
			.\] De même, $A\left( \substack{1\\1\\0} \right) = 1 \times \left( \substack{1\\1\\0} \right)$. D'où \[
				P^{-1}\cdot A\cdot P = \begin{pmatrix}
					3&0&0\\
					0&1&0\\
					0&0&?
				\end{pmatrix}\qquad\text{où}\qquad P = \begin{pmatrix}
					1&1&?\\
					1&1&?\\
					1&0&?
				\end{pmatrix}
			.\] Finalement, on en conclut que \[
				P = \begin{pmatrix}
					3&0&0\\
					0&1&0\\
					0&0&-1
				\end{pmatrix} \qquad \text{et}\qquad P^{-1}\cdot A\cdot P = \begin{pmatrix}
					1&1&1\\
					1&1&-1\\
					1&0&0
				\end{pmatrix} = D
			.\]
			De plus, la matrice $P$\/ est inversible car $\det P \neq 0$.
		\item Pour calculer $A^n$, on pourrait chercher un polynôme annulateur $Q$\/ de $A$, et on exprime $X^n = Q \times T_n + R_n$, et donc $A^n = R_n(A)$.
			Mais, on peut également diagonaliser $A$\/ (si elle est diagonalisable).
			Ainsi,  \[
				D^n = (P^{-1}\cdot A\cdot P)^n = P^{-1}\cdot A\cdot \cancel P\cdot \cancel{P^{-1}} \cdot \ldots\cdot \cancel{P^{-1}} \cdot A \cdot P = P^{-1}\cdot  A^n\cdot P
			.\] D'où $A^n = P \cdot D^n \cdot P^{-1}$. Or, \[
				D^n = \begin{pmatrix}
					3&0&0\\
					0&1&0\\
					0&0&-1
				\end{pmatrix}^n = \begin{pmatrix}
					3^n&0&0\\
					0&1^n&0\\
					0&0&(-1)^n
				\end{pmatrix}
			.\]
			On calcule donc $A^{n}$\/ en calculant l'inverse de $P$\/ : \[
				A^n = \begin{pmatrix}
					1&1&1\\
					1&1&-1\\
					1&0&0
				\end{pmatrix} \begin{pmatrix}
					3^n&0&0\\
					0&1^n&0\\
					0&0&(-1)^n
				\end{pmatrix} \cdot P^{-1}
			.\]
		\item
			\begin{align*}
				\begin{rcases*}
					\hfill u_{n+1} = v_n + 2w_n\\
					\hfill v_{n+1} = u_n + 2w_n\\
					\hfill w_{n+1} = 3w_n
				\end{rcases*} \iff& \begin{pmatrix}
					u_{n+1}\\v_{n+1}\\w_{n+1}
				\end{pmatrix} = \begin{pmatrix}
					0&1&2\\
					1&0&2\\
					0&0&3
				\end{pmatrix} \begin{pmatrix}
					u_n\\ v_n\\ w_n
				\end{pmatrix}\\
				\iff& U_{n+1} = A\cdot U_n\\
				\iff& U'_{n+1} = D \cdot U'_{n}
			\end{align*}
			où $D = P^{-1} \cdot A \cdot P$, $U'_{n+1} = P\cdot U_{n+1}$\/ et $U'_n = P\cdot U_n$.
			\begin{align*}
				\phantom{\begin{rcases*}
					\hfill mm_{n+1} = v_n + 2w_n\\
					\hfill v_{n+1} = u_n + 2w_n\\
					\hfill w_{n+1} = 3w_n
				\end{rcases*}} \iff&
				\begin{pmatrix}
					u'_{n+1}\\v'_{n+1}\\w'_{n+1}
				\end{pmatrix} = \begin{pmatrix}
					3&0&0\\
					0&1&0\\
					0&0&-1
				\end{pmatrix} \cdot \begin{pmatrix}
					u'_n\\
					v'_n\\
					w'_n
				\end{pmatrix}\\
				\iff& \begin{cases}
					u'_{n+1} = 3u'_n\\
					v'_{n+1} = v'_n\\
					w'_{n+1} = -w'_n
				\end{cases}\\
				\iff& \begin{cases}
					u'_n = K\times  3^n\\
					v'_n = L\\
					w'_n = M \times (-1)^n
				\end{cases}
			\end{align*}
			Ainsi, \[
				\begin{pmatrix}
					u_n\\v_n\\w_n
				\end{pmatrix} = \underbrace{\begin{pmatrix}
					1&1&1\\
					1&1&-1\\
					1&0&0
				\end{pmatrix}}_P \cdot \begin{pmatrix}
					K\times 3^n\\
					L\\
					M\times (-1)^n
				\end{pmatrix}
			.\] D'où $u_n = K\cdot 3^n + L + M \cdot (-1)^n$, $v_n = K\times 3^n + L - M \cdot (-1)^n$\/ et $w_n = K\cdot 3^n$, où les constantes $K$, $L$\/ et $M$\/ sont des constantes fixées par les conditions initiales.
		\item
			\begin{align*}
				\begin{rcases*}
					\hfill x'(t) = y(t) + 2z(t)\\
					\hfill y'(t) = x(t) + 2z(t)\\
					\hfill z'(t) = 3z(t)
				\end{rcases*} \iff& \begin{pmatrix}
					x'(t)\\
					y'(t)\\
					z'(t)
				\end{pmatrix} = \begin{pmatrix}
					0&1&2\\
					1&0&2\\
					0&0&3
				\end{pmatrix} \cdot \begin{pmatrix}
					x(t)\\
					y(t)\\
					z(t)
				\end{pmatrix}\\
				\iff& X'(t) = A\cdot X(t)\\
				\iff& U'(t) = D \cdot U(t) \text{ avec } D = P^{-1} \cdot A\cdot P \text{ et } X(t) = P\cdot U(t)\\
				\iff& \begin{pmatrix}
					u'(t)\\
					v'(t)\\
					w'(t)
				\end{pmatrix} = \begin{pmatrix}
					3&0&0\\
					0&1&0\\
					0&0&-1
				\end{pmatrix} \cdot \begin{pmatrix}
					u(t)\\
					v(t)\\
					w(t)
				\end{pmatrix}\\
				\iff& \begin{cases}
					u'(t) = 3u(t)\\
					v'(t) = v(t)\\
					w'(t) = -w(t)
				\end{cases}\\
				\iff& \begin{cases}
					u(t) = K \cdot \mathrm{e}^{3t}\\
					v(t) = L \cdot \mathrm{e}^{t}\\
					w(t) = M \cdot \mathrm{e}^{-t}
				\end{cases}
			\end{align*}
			Ainsi \[
				\begin{pmatrix}
					x(t)\\
					y(t)\\
					z(t)
				\end{pmatrix} = \underbrace{\begin{pmatrix}
					1&1&1\\
					1&1&-1\\
					1&0&0
				\end{pmatrix}}_P \cdot \begin{pmatrix}
					K \times \mathrm{e}^{3t}\\
					L \cdot \mathrm{e}^{t}\\
					M \cdot \mathrm{e}^{-t}
				\end{pmatrix}
			.\] 
			D'où $x(t) = K\cdot \mathrm{e}^{3t} + L \cdot \mathrm{e}^{t} + M \cdot \mathrm{e}^{-t}$, $y(t) = K \cdot \mathrm{e}^{3t} + L \cdot \mathrm{e}^{t} - M \cdot \mathrm{e}^{-t}$\/ et $z(t) = K\cdot \mathrm{e}^{3t}$. Les constantes $K$, $L$\/ et $M$\/ peuvent être déterminées à partir des conditions initiales.
	\end{enumerate}
\end{exo}

\begin{rmkn}[équations différentielles]
	On considère l'équation différentielle $(*)$ : $x'(t) = \lambda \cdot x(t)$.
	Les fonctions $x : t \mapsto K\cdot \mathrm{e}^{\lambda t}$\/ sont des solutions de cette équation. On peut utiliser la méthode de {\sc Lagrange}\/ : la méthode de la~\guillemotleft~variation de la constante.~\guillemotright\@ On cherche des solutions sous la forme $x(t) = k(t) \cdot \mathrm{e}^{\lambda t}$ (vision du~physicien). D'où $k(t) = x(t) / \mathrm{e}^{\lambda t}$\/ (vision du mathématicien). De plus, $x'(t) = k'(t) \mathrm{e}^{\lambda t} + k(t) \lambda \mathrm{e}^{\lambda t}$.
	Ainsi, on injecte ce $k(t)$\/ dans l'équation différentielle :
	\begin{align*}
		(*) \iff& k'(t) \mathrm{e}^{\lambda t} + k(t) \lambda \mathrm{e}^{\lambda t} = \lambda k(t)\mathrm{e}^{\lambda t}\\
		\iff& k'(t) \mathrm{e}^{\lambda t} = 0\\
		\iff& k'(t) = 0\\
		\iff& \exists K \in \R\,\:k(t) = K.
	\end{align*}
	Les solutions trouvées dans l'exercice précédent sont donc les uniques solutions du système d'équations différentielles.

	De même, pour résoudre une équation différentielle avec 2\tsup{nd} membre de la forme \[
		(**) : \qquad x'(t) - \lambda \cdot x(t) = b(t)
	.\]
	La fonction $t \mapsto x(t)$\/ est une solution de l'équation {\sc sans}\/ 2\tsup{nd} membre si et seulement si \[
		\exists K \in \R,\:\forall t \in \R,\quad x(t) = K \cdot \mathrm{e}^{\lambda t}
	.\]
	\begin{center}
		\slshape Comment résoudre l'équation différentielle {\scshape avec}\/ 2\tsup{nd} membre si on connaît la solution générale de l'équation {\scshape sans}\/ 2\tsup{nd} membre ?
	\end{center}
	On utilise la méthode le la variation de la constante.
	Soit $x(t) = k(t) \cdot \mathrm{e}^{\lambda t}$. Ainsi, en injectant cette expression de $x$\/ dans l'équation $(**)$, on trouve
	\begin{align*}
		(**) \iff& k'(t) \mathrm{e}^{\lambda t} + k(t) \cdot \lambda \mathrm{e}^{\lambda t} = \lambda k(t) \mathrm{e}^{\lambda t} + b(t)\\
		\iff& k'(t) \mathrm{e}^{\lambda t} = b(t)\\
		\iff& k'(t) = b(t) \cdot \mathrm{e}^{-\lambda t}\\
		\iff& k(t) = \int_{0}^{t} b(u)\cdot \mathrm{e}^{-\lambda u}~\mathrm{d}u + K\\
		\iff& x(t) = \left( \int_{0}^{t} b(u) \cdot \mathrm{e}^{-\lambda u}~\mathrm{d}u + K \right) \mathrm{e}^{\lambda t}\\
		\iff& x(t) = \underbrace{\int_{0}^{t} b(u) \cdot \mathrm{e}^{\lambda (t-u)}~\mathrm{d}u}_{\text{solution particulière}} + \underbrace{K \cdot \mathrm{e}^{\lambda t}}_{\substack{\text{solution}\\\text{générale}\\\text{de $(*)$}}}.
	\end{align*}
\end{rmkn}

		\begin{exm}
	On pose $f$, le sinus cardinal :  \begin{align*}
		f: \R^* &\longrightarrow \R \\
		t &\longmapsto \frac{\sin t}{t}.
	\end{align*}
	\begin{figure}[H]
		\centering
		\begin{asy}
			import graph;
			size(10cm);
			draw((-10, 0) -- (10, 0), Arrow(TeXHead));
			draw((0, -3) -- (0, 5), Arrow(TeXHead));
			real f(real x) {
				if(x == 0) { return 3; }
				else {return 3*sin(x) / x;}
			}
			draw(graph(f, -10, 10), magenta);
		\end{asy}
		\caption{Sinus cardinal}
	\end{figure}

	La fonction $f$\/ est continue sur ${]0,8]}$\/ mais $\lim_{t\to 0} \frac{\sin t}{t} = 1$. D'où $\int_{0}^{8} \frac{\sin t}{t}~\mathrm{d}t$\/ est faussement impropre en $0$\/ et donc convergente.


	Mais attention ! On ne dit pas \guillemotleft~{\color{red}soit $f : t \mapsto \frac{1}{t}$. L'intégrale $\int_{8}^{+\infty} \frac{1}{t}~\mathrm{d}t$\/ est faussement impropre en $+\infty$\/ car $\lim_{t\to +\infty}\frac{1}{t} = 0$}.~\guillemotright
\end{exm}

\section{Intégrer les $\mathbf{\sim}$, $\po$, et \textit{O}}

\begin{thm}
	\hfill$\O$\hfill\null
\end{thm}

\begin{thm}
	Le 2.\ n'est pas la réciproque du 1.\ mais la contraposée.
\end{thm}

\begin{prop}
	\hfill$\O$\hfill\null
\end{prop}

\begin{exm}
	On considère l'intégrale $\int_{2}^{+\infty} \frac{1}{t^2+ \cos t}~\mathrm{d}t$, c'est une intégrale impropre en $+\infty$.
	On recherche un équivalent de $\frac{1}{t^2 + \cos t}$\/ en $+\infty$ : \[
		\frac{1}{t^2 + \cos t} \simi_{t\to +\infty} \frac{1}{t^2}
	\] qui ne change pas de signe. Or, $\int_{2}^{+\infty} \frac{1}{t^2}~\mathrm{d}t$\/ converge car c'est une intégrale de {\sc Riemann}\/ avec $\alpha = 2 > 1$.
	On en déduit que l'intégrale $I$\/ converge.

	On procède autrement : \[
		0 \le \frac{1}{t^2 + \cos t} \le \frac{1}{t^2 - 1}
	.\] Or, $\int_{2}^{+\infty} \frac{1}{t^2 - 1}~\mathrm{d}t$\/ converge car
	\begin{align*}
		\int_{2}^{x} \frac{1}{t^2 - 1}~\mathrm{d}t &= \int_{2}^{x} \left( \frac{\sfrac12}{t-1} - \frac{\sfrac12}{t+1} \right) ~\mathrm{d}t \\
		&= \frac{1}{2} \int_{2}^{x} \frac{1}{t-1}~\mathrm{d}t - \frac{1}{2}\int_{2}^{x} \frac{1}{t+1}~\mathrm{d}t \\
		&= \frac{1}{2} \Big[\ln|t-1|\Big]_2^x - \frac{1}{2}\Big[\ln |t+1|\Big]_2^x \\
	\end{align*}
	D'où \[
		\int_{2}^{x} \frac{1}{t^2 - 1}~\mathrm{d}t = \frac{1}{2} \left[ \ln\left| \frac{t-1}{t+1} \right| \right]_2^x = \frac{1}{2}\ln \left| \frac{x-1}{x+1} \right| + \frac{1}{2} \ln 3 \tendsto{x\to +\infty} \frac{1}{2} \ln 3
	.\] donc l'intégrale $I$\/ converge et $I \le \frac{1}{2} \ln_3$.
\end{exm}

\begin{exo}
	\begin{enumerate}
		\item L'intégrale $I = \int_{0}^{1} \frac{\sin t}{t^2}~\mathrm{d}t$\/ est impropre en 0. On utilise un équivalent : $\sin t \simi_{t\to 0} t$\/ qui ne change pas de signe. Or, $\int_{0}^{t} \frac{1}{t}~\mathrm{d}t$\/ diverge (par critère de {\sc Riemann}). Donc $I$\/ diverge.
			
			L'intégrale $J = \int_{1}^{+\infty} \sin \frac{1}{t}~\mathrm{d}t$\/ est généralisée en $+\infty$. On cherche un équivalent en $+\infty$\/ : \[
				\sin \frac{1}{t} \simi_{t\to +\infty} \frac{1}{t}
			\] qui ne change pas de signe. Or, $\int_{1}^{+\infty} \frac{1}{t}~\mathrm{d}t$\/ diverge par critère de {\sc Riemann}. On en déduit que $J$\/ diverge également.
		\item L'intégrale $\int_{0}^{+\infty} \frac{1}{t^2}~\mathrm{d}t$\/ est impropre, {\bf et}\/ en 0, {\bf et}\/ en $+\infty$. Le théorème ne marche donc pas.
			En effet $t\mapsto \frac{1}{t^2}$\/ n'est pas continue par morceaux en 0, ce qui était le cas pour $t\mapsto \frac{1}{1+t^2}$.
	\end{enumerate}
\end{exo}

\begin{rmkn}[Retour sur la {\sc remarque}\/ 5]
	L'intégrale $\int_{0}^{+\infty} \frac{1}{\ln(1+t)}~\mathrm{d}t$\/ est impropre en 0 {\bf et}\/ en $+\infty$. $\int_{0}^{+\infty} \frac{1}{\ln(1+t)}~\mathrm{d}t$\/ converge si et seulement si $\int_{0}^{7} \frac{1}{\ln(1+t)}~\mathrm{d}t$\/ {\bf et}\/ $\int_{7}^{+\infty} \frac{1}{\ln(1+t)}~\mathrm{d}t$\/ convergent.
	Et si elles convergent \[
		\int_{0}^{+\infty} \frac{1}{\ln(1+t)}~\mathrm{d}t = \int_{0}^{7} \frac{1}{\ln(1+t)}~\mathrm{d}t + \int_{7}^{+\infty} \frac{1}{\ln(1+t)}~\mathrm{d}t
	.\]
	On n'utilise pas deux barrières en même temps. Sinon, les intégrales doublement impropres peuvent, et converger, et diverger.
\end{rmkn}

\begin{prop}[avec $\sim$]
	Si $f(t) \simi_{t\to b} g(t)$\/ qui ne change pas de signe. Alors,
	\begin{itemize}
		\item ou bien $\ds\int_{a}^{b} f(t)~\mathrm{d}t$\/ et $\ds\int_{a}^{b} g(t)~\mathrm{d}t$\/ convergent et $\ds \int_{x}^{b} f(t)~\mathrm{d}t \simi_{x\to b} \int_{x}^{b} g(t)~\mathrm{d}t$.
		\item ou bien $\ds\int_{a}^{b} f(t)~\mathrm{d}t$\/ et $\ds\int_{a}^{b} g(t)~\mathrm{d}t$\/ divergent et $\ds\int_{a}^{x} f(t)~\mathrm{d}t \simi_{x\to b} \int_{a}^{x} g(t)~\mathrm{d}t$.
	\end{itemize}
	Cette proposition est équivalente à le {\sc lemme}\/ 13 sur les séries.
\end{prop}



		\begin{prop}
	La relation $\preceq$\/ est un \textit{pré-ordre} :
	\begin{itemize}
		\item $\preceq $\/ est réflective ;
		\item $\preceq $\/ est transitive.
	\end{itemize}
\end{prop}

\begin{prv}
	Soit $Q$\/ un problème de décision.
	\begin{itemize}
		\item $Q \preceq Q$\/ par la fonction identité, qui est totale et calculable.
		\item Soient $Q$, $R$\/ et $S$\/ trois problèmes de décision tels que $Q \preceq R$\/ et $R \preceq S$. Soit donc $f_1$\/ la réduction de $Q$\/ à $R$, et $f_2$\/ la réduction de $R$\/ à $S$. Soit $f = f_2 \circ f_1 : \mathcal{E}_Q \to \mathcal{E}_S$. La fonction $f$\/ est totale comme composée de fonctions totales, $f$\/ est calculable comme composée de fonctions calculables. De plus,
			\begin{align*}
				\forall e \in \mathcal{E}_Q,\qquad f(e) \in S^+ \iff& f_2(f_1(e)) \in S^+\\
				\iff& f_1(e) \in R^+\\
				\iff& e \in Q^+
			\end{align*}
	\end{itemize}
\end{prv}

\section{Classe \textbf{P} et \textbf{NP}}

Pour répondre à un problème, on peut le résoudre par des algorithmes plus ou moins rapides. Mais, l'objectif de cette section est de montrer que certains problèmes ne peuvent se résoudre que par des algorithmes lents, et que l'on ne peut pas faire mieux.

\begin{defn}
	Le modèle de calcul impose une représentation des entrées par chaînes de caractères. Cela induit donc une notion de \textit{taille d'entrée}, qui est la longueur de la chaîne de caractères.
	\index{taille d'entrée}
\end{defn}


\subsection{Complexité d'une machine}

\begin{defn}
	Étant donné une machine $\mathcal{M}$ et une entrée $w \in \Sigma^*$, on note $C^\mathcal{M}(w)$\/ le nombre d'opérations élémentaires effectuées lors de l'appel de $\mathcal{M}$\/ sur $w$. Lorsque $\smash{w \xrightarrow[\mathcal{M}]{} {\circlearrowleft}}$, on définit $C^\mathcal{M} = +\infty$.

	Pour $n \in \N$, on définit alors \[
		C^\mathcal{M}_n = \max \{ C^\mathcal{M}(w)  \mid w \in \Sigma^n \}
	.\]
	\index{machine!nombre d'opérations élémentaires!($C^\mathcal{M}(w)$)}
	\index{machine!nombre d'opérations élémentaires!maximal pour un mot de taille $n$\/ ($C^\mathcal{M}_n$)}
\end{defn}

\begin{rmk}
	On a, $\forall n \in \N$, $C_n^\mathcal{M} \in \bar{\N} = \N \cup \{+\infty\}$.
\end{rmk}

\begin{defn}
	Soit $f : \N\to \N$\/ une fonction totale et calculable. On note $\textsc{Time}(f)$\/ l'ensemble des machines $\mathcal{M}$\/ telles que
	\begin{itemize}
		\item $\mathcal{M}$\/ s'arrête sur toute entrée ;
		\item $\big(C_n^\mathcal{M}\big)_{n \in \N} = \mathcal{O}\big(\big(f(n)\big)_{n \in \N}\big)$.
	\end{itemize}
	\index{machine!ensemble $\textsc{Time}(f)$}
\end{defn}

\subsection{Classe \textbf{P}}

\begin{defn}
	On dit d'une machine $\mathcal{M}$\/ qu'elle est de \textit{complexité polynômiale} dès lors qu'il existe $k \in \N$\/ tel que $\mathcal{M} \in \textsc{Time}(n^k)$.
	\index{machine!de complexité polynômiale}
\end{defn}

\begin{defn}
	On dit d'une fonction (partielle ou non), qu'elle est \textit{calculable en temps polynômial} dès lors qu'il existe une machine $\mathcal{M}$\/ de complexité polynômiale la calculant.
	\index{fonction!calculable!en temps polynômial}
\end{defn}

\begin{exm}
	\begin{itemize}
		\item l'identité ($n \mapsto n$)
		\item la fonction successeur ($n\mapsto n+1$)
	\end{itemize}
\end{exm}


		\section{Arbres couvrants de poids minimum}

\begin{exm}
	On considère le graphe ci-dessous.
	\begin{figure}[H]
		\centering
		\tikzfig{ex-graphe-pondere}
		\caption{Arbre pondéré}
	\end{figure}
	On cherche à \guillemotleft~supprimer~\guillemotright\ des arrêtes de ce graphe afin d'avoir un poids total minimum, tout en conservant la connexité du graphe.
	Une structure assurant cette condition est un arbre.

	Pour résoudre ce problème, on part du graphe vide, et on ajoute les arrêtes les moins coûteuses en premier.
\end{exm}

\begin{defn}[Arbre]
	Soit $G = (S,A)$\/ un graphe non-orienté. On dit que $G$\/ est un \textit{arbre} si $G$\/ est connexe et acyclique.
	\index{arbre}
\end{defn}

\begin{defn}[Arbre couvrant]
	Étant donné un graphe non orienté pondéré par poids positifs $G = (S, A, c)$,\footnotemark\ on dit de $G' = (S', A')$\/ que c'est un \textit{arbre couvrant} de $G$\/ si $S' = S$\/ et $A' \subseteq A$, et $G'$\/ est un arbre.
	\index{arbre!couvrant}
\end{defn}
\footnotetext{on dit que $c$\/ est la fonction de pondération de ce graphe}

\begin{defn}[Arbre couvrant de poids minimum]
	Étant donné un graphe non orienté pondéré $G = (S, A, c)$\/ et un arbre couvrant $T = (S', A')$, on appelle \textit{poids} de l'arbre $T$\/ la valeur $\sum_{a \in A'} c(a)$.
	\index{arbre!couvrant!poids}

	Si $G$\/ est connexe, il admet au moins un arbre couvrant, on peut définir l'\textit{arbre couvrant de poids minimum} (\textit{\textsc{acpm}}).
	\index{arbre!couvrant!de poids minimum}
\end{defn}

On définir alors le problème \[
	\textsc{acpm}\text{\footnotemark}
	\begin{cases}
		\text{\textbf{Entrée}}&: G = (S, A, c) \text{ connexe}\\
		\text{\textbf{Sortie}}&: \text{ le poids de l'arbre couvrant de poids minimum}.
	\end{cases}
\]
\footnotetext{Arbre Couvrant de Poids Minimum}

\begin{algorithm}[H]
	\centering
	\begin{algorithmic}[1]
		\Entree $G = (S, A, c)$\/ un graphe connexe
		\Sortie Un arbre couvrant de poids minimum
		\State $B \gets \O$\/ 
		\State $U \gets \O$\/
		\While{il existe $u$\/ et $v$\/ tels que $u \nsim_B v$}
			\State Soit $\{x,y\} \in A \setminus U$\/ de poids minimal
			\If{$x \sim_B y$}
				\State $U \gets \big\{\!\{x,y\}\!\big\} \cup U$
			\Else
				\State $U \gets \big\{\!\{x,y\}\!\big\} \cup U$
				\State $B \gets \big\{\!\{x,y\}\!\big\} \cup B$
			\EndIf
		\EndWhile
		\State\Return $T = (S,B)$\/
	\end{algorithmic}
	\caption{Algorithme de \textsc{Kruskal}}
\end{algorithm}

\begin{prop}
	L'algorithme de \textsc{Kruskal} est correct.
\end{prop}

\begin{prv}
	\begin{enumerate}
		\item Il existe un arbre couvrant de poids minimum utilisant les arrêtes de $B$ ;
		\item $B \subseteq U \subseteq A$\/ ;
		\item $\forall \{u,v\} \in U$, $u \sim_B v$.
	\end{enumerate}
	Ces trois propriétés sont invariantes.
	\begin{description}
		\item[Initialement] $B = \O = U$, donc \textsc{ok}.
		\item[Propagation] Soient $\ubar{B}$\/ et $\ubar{U}$\/ (resp.\ $\bar{B}, \bar{U}$) les valeurs de $B$\/ et $U$\/ avant (resp.\ après) une itération de boucle. Supposons que $\ubar{B}$\/ et $\ubar{U}$\/ satisfont les propriétés 1, 2 et 3. Montrons que $\bar{B}$\/ et $\bar{U}$\/ les satisfont aussi.
			\begin{enumerate}
				\item[2.] On a $\{x,y\} \in A$\/ et $\ubar{B} \subseteq \ubar{U} \subseteq A$, donc \[
						\bar{B} \subseteq \ubar{B} \cup \{\!\{x,y\}\!\} \subseteq \ubar{U} \cup \{\!\{x,y\}\!\} \subseteq A
					.\]
				\item[3.] Soit $\{u,v\} \in \bar{U}$.
					\begin{itemize}
						\item Si $\{u,v\} \in \ubar{U}$, alors de 3, $u \sim_{\ubar{B}} v$. Or, $\ubar{B} \subseteq \bar{B}$\/ et donc $u \sim_{\bar{B}} v$.
						\item Sinon, $\{ u,v\} = \{x,y\}$, alors $x = u$\/ et $v = y$.
							\begin{itemize}
								\item Sous-cas 1 : $\bar{B} = \ubar{B} \cup \{\!\{x,y\}\!\}$, alors $x \sim_{\bar{B}} y$.
								\item Sous-cas 2 : $\bar{B} = \ubar{B}$, alors par condition du \textbf{si}, $x \sim_{\ubar{B}} y$\/ et donc $x \sim_{\bar{B}} y$.
							\end{itemize}
					\end{itemize}
				\item[1.]
					Soit $\mathcal{T}$\/ un \textsc{acpm} contenant $\ubar{B}$.
					\begin{itemize}
						\item Cas 1 : $\bar{B} = \ubar{B}$, \textsc{ok}
						\item Cas 2 : $\bar{B} = \ubar{B} \cup \{\!\{x,y\}\!\}$.
							\begin{itemize}
								\item Sous-cas 1 : $\{x,y\} \in \mathcal{T}$, alors $\mathcal{T}$\/ est un \textsc{acpm} qui contient $\bar{B}$.
								\item Sous-cas 2 : $\{x,y\}  \not\in \mathcal{T}$, $\mathcal{T}$\/ est un arbre couvrant, donc il contient une chaîne de $x$\/ à $y$\/ : \[
											\{\overset{\substack{x\\[-1mm]\vrt=}}{x_0},x_1\},\{x_1,x_2\},\ldots,\{x_{n-1},\underset{\substack{\vrt=\\y}}{x_n}\}
									.\]
									Or, $\forall i \in \llbracket 1,n-1 \rrbracket$, $x_i \sim_{\ubar{B}} x_{i+1}$. Par transitivité, on a donc $x = x_0 \sim_{\ubar{B}} x_n = y$, ce qui n'est pas le cas.
									Il existe donc $i_0 \in \llbracket 0,n-1 \rrbracket$, tel que $x_{i_0} \nsim_{\ubar{B}} x_{i_0 + 1}$\/ et donc $\{x_{i_0}, x_{i_0 + 1}\} \not\in \ubar{U}$. D'où, d'après 3, on a $\{x_{i_0}, x_{i_0 + 1}\}  \not\in \ubar{B}$
									Considérons alors $\mathcal{T}' = \big(\mathcal{T} \setminus \{\!\{x_{i_0},x_{i_0+1}\}\!\}\big)  \cup \{\!\{x,y\}\!\}$. Montrons que $\mathcal{T}'$\/ est un \textsc{acpm} contenant $B$, en commençant par montrer que c'est un arbre couvrant. L'arbre $\mathcal{T}'$\/ a $n-1$\/ arrêtes (autant que $\mathcal{T}$). Montrons que $\mathcal{T}'$\/ est connexe.
									Soit $(a,b) \in S^2$. $\mathcal{T}$\/ est connexe, soit donc une chaîne \[
										C : \quad a = u_0, u_1, \ldots, u_n = b
									\] de $\mathcal{T}$. Si la chaîne $C$\/ n'utilise pas l'arrête $\{x_{i_0},x_{i_0+1}\}$, alors $C$\/ est une chaîne de $\mathcal{T}'$. Sinon, on pose $\mu$\/ et $\tau$\/ tels que \[
									\underbrace{a,\ldots,x_{i_0}}_{\mu},\underbrace{x_{i_0+1},\ldots,b}_{\tau}
									.\]
									Soit alors la chaîne
									\begin{align*}
										\overbrace{a,\ldots,x_{i_0}}^{\mu},x_{i_0-1},x_{i_0-2},\ldots,x_0 = x,&\\
										\underbrace{b,\ldots,x_{i_0 + 1}}_{\tau},x_{i_0+2},\ldots,x_{n-1},x_n=y&
									\end{align*}
									qui est dans $\mathcal{T}'$.
									Montrons que le poids est minimum. Notons $P(\mathcal{T})$\/ le poids de l'arbre. On a donc \[
										P(\mathcal{T}') = P(\mathcal{T}) + c(\{x,y\}) - c(\{x_{i_0},x_{i_0+1}\})
									.\] Par choix glouton, ($\{x_{i_0}, x_{i_0+1} \not\in \ubar{U}\}$), $c(\{x,y\}) \le c(\{x_{i_0},x_{i_0+1}\})$\/ donc $P(\mathcal{T}') \le P(\mathcal{T})$, et $\mathcal{T}$\/ étant de poids min, $P(\mathcal{T}') = P(\mathcal{T})$\/ et $\mathcal{T}'$\/ est un \textsc{acpm} contenant $\bar{B}$.
							\end{itemize}
					\end{itemize}
			\end{enumerate}
	\end{description}

	Les invariants le sont.
\end{prv}

À la fin, $B$\/ induit un graphe connexe et $B$\/ est contenu dans un \textsc{acpm}, c'en est donc un.


		\paragraph{Une structure pour la gestion des partitions : \textsf{UnionFind}.}

\begin{defn}[Type de données abstrait \textsf{UnionFind}]
	On définit le type de données abstrait \textsf{UnionFind} comme contenant
	\begin{itemize}
		\item un type \texttt{t} de partitions ;
		\item un type \texttt{elem} des éléments manipulés par les partitions ;
		\item $\texttt{initialise\_partition} : \texttt{elem list} \to \texttt{t}$\/ retournant le partitionnement dans lequel chaque élément est seul dans sa classe ;
		\item $\texttt{find} : (\texttt{t} \mathbin{\texttt{*}} \texttt{elem}) \to \texttt{elem}$\/ retournant un représentant de la classe de l'élément. Si deux éléments $x$\/ et $y$\/ sont dans la même classe, dans le partitionnement $p$, alors $\texttt{find}(p,x) = \texttt{find}(p,y)$ ;
		\item $\texttt{union} : (\texttt{t} \mathbin{\texttt{*}} \texttt{elem} \mathbin{\texttt{*}} \texttt{elem}) \to \texttt{t}$\/ retourne le partitionnement dans lequel on a fusionné les classes des arguments.
	\end{itemize}
	\index{type \textsf{UnionFind}}
\end{defn}

\begin{exm}
	On réalise le \textit{pseudo-code} ci-dessous.
	\begin{itemize}
		\item $p \gets \texttt{initialise\_partition}([1, 2, 3, 4, 5])$\/ $\leadsto$ $\{\{1\}, \{2\}, \{3\}, \{4\}, \{5\}\}$
		\item $\texttt{find}(p, 1) = 1$\/ 
		\item $\texttt{union}(p, 1, 3)$\/ $\leadsto$ $\{\{1,3\}, \{2\}, \{4\}, \{5\}\}$
		\item $\texttt{find}(p, 1) = \texttt{find}(p, 3)$
	\end{itemize}
\end{exm}

On implémente ce type abstrait en \textsc{OCaml}.

\begin{rmk}[Niveau zéro -- listes de liste]~
	\begin{lstlisting}[language=caml,caption=Implémentation du type \textsf{UnionFind} en \textsc{OCaml}]
type 'a t = 'a list list

let initialise_partition (l: 'a list): 'a t =
	List.map (fun x -> [ x ] ) l

let rec find (p: 'a t) (x: 'a): 'a =
	match p with
	| classe :: classes ->
			if List.mem x classe then List.hd classe
			else find classes x
	| [] -> raise Not_Found

let est_equiv (p: 'a t) (x: 'a) (y: 'a): bool = 
	(find p x) = (find p y)

let rec extrait_liste (x: 'a) (p: 'a t): 'a list * 'a p =
	match p with
	| classe :: classes ->
			if List.mem x classe then (classe, classes)
			else
				let cl, cls' = extrait_liste x classes in
				(cl, classe :: cls')
	| [] -> raise Not_Found

let union (p: 'a t) (x: 'a) (y: 'a): 'a t =
	if est_equiv p x y then p
	else
		let cx, p' = extrait_liste x p in
		let cy, p'' = extrait_liste y p' in
		(cx @ cy) :: p''
	\end{lstlisting}
\end{rmk}

\begin{rmk}[Niveau un -- tableau de classes]
	Dans la case du tableau, on inscrit le numéro de sa classe.
	Pour \texttt{find}, on prend le premier ayant la même classe.
	Pour \texttt{union}, on re-numérote vers un numéro commun.
	Par exemple, \[
		\begin{array}{|c|c|c|c|c|c|}
			\hline
			0 & 1 & 0 & 0 & 1 & 2\\ \hline
			0 & 1 & 2 & 3 & 4 & 5 \\ \hline
		\end{array}\quad\quad\longleftrightarrow\quad\quad\{\{0,2,3\},\{1,4\},\{5\}\}
	.\]
\end{rmk}

\begin{rmk}[Niveau deux -- tableau de représentants]
	Dans les cases du tableau, on écrit le représentant de la classe de $i$.
	Pour \texttt{find}, on lit la case.
	Pour \texttt{union}, on re-numérote vers un numéro commun.
	Par exemple, \[
		\begin{array}{|c|c|c|c|c|c|}
			\hline
			2 & 4 & 2 & 2 & 4 & 5\\ \hline
			0 & 1 & 2 & 3 & 4 & 5 \\ \hline
		\end{array}\quad\quad\longleftrightarrow\quad\quad\{\{0,2,3\},\{1,4\},\{5\}\}
	.\]
\end{rmk}

\begin{rmk}[Niveau trois -- arbres]
	Pour $\texttt{union}(0, 1)$, on cherche le représentant de 0 (2) puis celui de 1 (4). On fait pointer 4 vers 2.
	Pour la suite de l'implémentation, \textit{c.f.}\ \textsc{dm}$_3$.

	\begin{figure}[H]
		\centering
		\tikzfig{ex-unionfind-arbres}
		\caption{Représentation par des arbres}
	\end{figure}
\end{rmk}

Avec cette nouvelle structure, on peut maintenant revenir sur l'algorithme de \textsc{Kruskal}.

\begin{algorithm}[H]
	\centering
	\begin{algorithmic}[1]
		\Entree Un graphe $G = (S, A, c)$\/ un graphe non orienté, pondéré
		\Sortie Un \textsc{acpm}
		\State Soit $(e_i)_{i\in\llbracket 1,m \rrbracket}$\/ un tri des arrêtes par coût croissant
		\State $f \gets 0$\/ \Comment{Nombre d'\textsl{\texttt{union}} effectuées}
		\State $p \gets \texttt{initialise\_partition}(S)$\/ 
		\State $I \gets 0$\/ 
		\State $B \gets \O$\/ 
		\While{$f < n - 1$}
			\State $\{x,y\} \gets e_I$\/ 
			\If{$\texttt{find}(p, x) \neq \texttt{find}(p, y)$}
				\State $p \gets \texttt{union}(p, x, y)$\/ 
				\State $B \gets B \cup \{\!\{x,y\}\!\}$\/ 
				\State $f \gets f + 1$\/ 
			\EndIf
			\State $I \gets I + 1$\/
		\EndWhile
		\State\Return $(S,B)$
	\end{algorithmic}
	\caption{Algorithme de \textsc{Kruskal} -- version 2}
\end{algorithm}

\paragraph{Étude de complexité.}
Notons $C_{\texttt{find}}^n$\/ un majorant du coût de \texttt{find} sur une structure contenant $n$\/ éléments, notons $C_{\texttt{union}}^n$\/ un majorant du coût de \texttt{union} sur une structure contenant $n$\/ éléments, et notons $C_{\texttt{init}}^n$\/ un majorant du coût de \texttt{init} sur une structure contenant $n$\/ éléments.
La complexité de cet algorithme est de \[
	\mathcal{O}\big(C_{\texttt{init}}^n + 2m\:C_{\texttt{find}}^n + n\: C_{\texttt{union}}^n + m \log_2 m\big)
.\]

\section{Couplage dans un graphe biparti}

\begin{defn}[Couplage]
	On appelle \textit{couplage} d'un graphe non orienté $G = (S, A)$, la donnée d'un sous-ensemble $C \subseteq A$\/ tel que \[
		\forall \{x,y\}, \{x',y'\} \in C,\:
		\quad\quad \{x,y\} \cap \{x',y'\} \neq \O
		\implies
		\{x,y\}  = \{x', y'\}
	.\]

	\index{graphe!couplage}
\end{defn}

\begin{figure}[H]
	\centering
	\tikzfig{ex-couplage}
	\caption{Exemple de couplage}
\end{figure}

\begin{exm}
	On réutilise l'exemple ci-dessous dans toute la section.
	L'ensemble $C = \{\{a,2\}, \{b,3\}\}$\/ est un couplage.
	Mais, l'ensemble $C' = \{\{a,1\}, \{a,2\}\}$\/ n'en est pas un.
\end{exm}

\begin{defn}
	Un couplage est dit \textit{maximal} s'il est maximal pour l'inclusion ($\subseteq$).
	Un couplage est dit \textit{maximum} si son cardinal est maximal.
	\index{graphe!couplage!maximal}
	\index{graphe!couplage!maximum}
\end{defn}

\begin{exm}
	Dans l'exemple précédent, 
	\begin{itemize}
		\item le couplage $C = \{\{a,2\}, \{b,3\}\}$\/ n'est ni maximal, ni maximum ;
		\item le couplage $C' = \{\{a,2\}, \{b, 3\}, \{d, 4\}\}$\/ est maximal mais pas maximum ;
		\item le couplage $C'' = \{\{a,1\}, \{b,3\}, \{c,2\}, \{d,4\}\}$\/ est maximum.
	\end{itemize}
\end{exm}

\begin{rmk}
	Dans toute la suite, on ne considère que des graphes bipartis.
\end{rmk}

\begin{defn}
	Étant donné un graphe biparti $G = (S, A)$\/ et un couplage $C$, un sommet $x$\/ est dit \textit{libre} dès lors que \[
		\forall \{y,z\} \in C,\: x \not\in \{y,z\}
	.\]
	Une chaîne élémentaire\footnotemark $(c_0, c_1, \ldots, c_{2p+1})$\/ est dit \textit{augmentante} si
	\begin{itemize}
		\item $c_0$\/ et $c_{2n+1}$\/ sont libres ;
		\item $\forall i \in \llbracket 0,p \rrbracket$, $\{c_{2i}, c_{2i+1}\}\in A \setminus C$ ;
		\item $\forall i \in \llbracket 0,p-1 \rrbracket$, $\{c_{2i+1}, c_{2i+2}\} \in C$.
	\end{itemize}
\end{defn}
\footnotetext{\textit{i.e.}\ une chaîne sans boucles.}

\begin{exm}~
	\begin{figure}[H]
		\centering
		\tikzfig{ex-chaine-augmentante}
		\caption{Chaîne augmentante}
	\end{figure}
\end{exm}

\begin{exm}
	Dans l'exemple de cette section, $(d, 4)$\/ et $(c, 2, a, 1)$\/ sont deux chaînes augmentantes.
\end{exm}

\begin{prop}
	Étant donné un graphe biparti $G = (S, A)$\/ avec $S = S_1 \cupdot S_2$ (partitionnement du graphe biparti), un couplage $C$\/ est maximum si, et seulement s'il n'admet pas de chaînes augmentantes.
\end{prop}

\begin{prv}
	\begin{itemize}
		\item[``$\implies$'']
			Soit $C$\/ un couplage admettant une chaîne augmentante. Montrons que $C$\/ n'est pas maximum.
			Soit la chaîne augmentante\footnotemark \[
				c_0 \to c_1 \Rightarrow c_2 \to c_3 \Rightarrow c_4 \to \cdots \to c_{2p - 1} \Rightarrow c_{2p} \to c_{2p+1}
			.\]
			On considère alors le couplage \[
				C' = \Big(C \setminus \big\{\{c_{2i+1},c_{2i+2} \} \mid i \in \llbracket 0,p-1 \rrbracket\big\}\Big) \cup \big\{\{c_{2i},c_{2i+1}\}  \mid i \in \llbracket 0,p \rrbracket\big\}
			.\]
			On transforme donc la chaîne en \[
				c_0 \Rightarrow c_1 \to c_2 \Rightarrow c_3 \to \cdots \to c_{2p-1} \to c_{2p} \Rightarrow c_{2p+1}
			.\]
			C'est bien un couplage, et $\Card(C') = \Card C + 1$. $C$\/ n'est donc pas un couplage maximum.
		\item[``$\impliedby$'']
			Soit $C$\/ un couplage non maximum. Montrons que $C$\/ admet une chaîne augmentante. Soit $M$\/ un couplage maximum, et $D = C \mathrel{\triangle} M = (C \setminus M) \cupdot (M \setminus C)$.
			On a $\Card C < \Card M$\/ et $\Card(C \setminus M) < \Card(M \setminus C)$.
			On remarque que, si $c_0 \to c_1 \to c_2 \to \cdots \to c_{p-1}\to c_p$\/ est une chaîne de $D$, (si $c_0 \to c_1 \in C \setminus M$\/ et $c_1 \to c_2 \in C \setminus M$\/ donc $c_1$\/ est dans deux arrêtes distinctes d'un couplage $C$, ce qui est absurde ; de même pour les autres arrêtes). Ainsi, 2 arrêtes consécutives ne sont pas dans la même composante de l'union $(C \setminus M) \cupdot (M \setminus C)$.
			Considérons la relation d'équivalence $\sim$\/ sur $D$\/ définie par $\{x,y\} \sim \{z,t\} \iffdef$ il existe une chaîne de $D$\/ utilisant l'arrête $\{x,y\}$\/ et l'arrête $\{z,t\}$.
			Soit le partitionnement $D_1, \ldots, D_q$\/ de $D$\/ par $\sim$.
			Par inégalité de cardinal, il existe un $D_i$\/ tel que \[
				\Card \{e \in D_i  \mid e \in C\} < \Card \{e \in D_i  \mid e \in M\}
			.\] L'ensemble $D_i$\/ contient alors une chaîne augmentante.
	\end{itemize}
\end{prv}
\footnotetext{On représente $\Rightarrow$ pour les arrêtes dans le couplage $C$.}

	}
	\def\addmacros#1{#1}
}

{
	\chap[5]{Trois exemples d'algorithmes de graphes}
	\minitoc
	\renewcommand{\cwd}{../cours/chap05/}
	\addmacros{
		\section{(Ne pas) être diagonalisable}

\begin{defn}
	Soit une matrice carrée $A$. On dit que $A$\/ est {\it diagonalisable}\/ s'il existe une matrice inversible~$P \in \mathrm{GL}_n(\mathds{K})$\/ telle que $P^{-1}\cdot A\cdot P$\/ est diagonale.
\end{defn}

\begin{exo}
	\begin{enumerate}
		\item Montrons que la matrice $B = {7\: 1\choose 0\:7}$\/ n'est pas diagonalisable.
			Par l'absurde : on suppose qu'il existe $P \in \mathrm{GL}_2(\R)$\/ et $(\lambda_1, \lambda_2) \in \R^2$\/ tels que \[
				P^{-1} \cdot B \cdot P = \begin{bmatrix}
					\lambda_1 & 0\\
					0&\lambda_2
				\end{bmatrix}
			.\] On applique la trace $\tr$\/ et le déterminant $\det$\/ :
			\begin{gather*}
				\tr(B) = \tr{\lambda_1\:0\choose 0\:\lambda_2} \quad\text{d'où}\quad \lambda_1 + \lambda_2 = 7 + 7 = 14 = \s\\
				\det(B) = \det{\lambda_1\:0\choose 0\:\lambda_2} \quad\text{d'où}\quad \lambda_1 \times \lambda_2 = 7 \times 7 = 49 = p
			\end{gather*}
			D'où $\lambda_1$\/ et $\lambda_2$\/ sont des solutions de l'équation $X^2 - \s X + p = 0$. Or
			\begin{align*}
				X^2 - \s X + p = 0 \iff& X^2 - 14X + 49 = 0\\
				\iff& (X-7)^2 = 0\\
				\iff& X = 7.
			\end{align*}
			D'où 
			\begin{align*}
				B = P P^{-1} B P P^{-1} = P \begin{pmatrix}
					7&0\\
					0&7
				\end{pmatrix} P^{-1} = P \cdot 7I_2\cdot P^{-1} = 7I_2.
			\end{align*}
			La matrice $B$\/ n'est donc pas diagonalisable.

			De même, montrons que la matrice $A$\/ n'est pas diagonalisable. On remarque que \[
				A \cdot \mat{1\\1\\1} = \begin{pmatrix}
					0&1&2\\
					1&0&2\\
					0&0&3
				\end{pmatrix} \begin{pmatrix}
					1\\1\\1
				\end{pmatrix} = \begin{pmatrix}
					3\\3\\3
				\end{pmatrix} = 3\begin{pmatrix}
					1\\1\\1
				\end{pmatrix} 
			.\] Ainsi, \[
				P^{-1}\cdot A\cdot P = \begin{pmatrix}
					3&0&0\\
					0&?&0\\
					0&0&?
				\end{pmatrix}\qquad\text{où}\qquad P = \begin{pmatrix}
					1&?&?\\
					1&?&?\\
					1&?&?
				\end{pmatrix}
			.\] De même, $A\left( \substack{1\\1\\0} \right) = 1 \times \left( \substack{1\\1\\0} \right)$. D'où \[
				P^{-1}\cdot A\cdot P = \begin{pmatrix}
					3&0&0\\
					0&1&0\\
					0&0&?
				\end{pmatrix}\qquad\text{où}\qquad P = \begin{pmatrix}
					1&1&?\\
					1&1&?\\
					1&0&?
				\end{pmatrix}
			.\] Finalement, on en conclut que \[
				P = \begin{pmatrix}
					3&0&0\\
					0&1&0\\
					0&0&-1
				\end{pmatrix} \qquad \text{et}\qquad P^{-1}\cdot A\cdot P = \begin{pmatrix}
					1&1&1\\
					1&1&-1\\
					1&0&0
				\end{pmatrix} = D
			.\]
			De plus, la matrice $P$\/ est inversible car $\det P \neq 0$.
		\item Pour calculer $A^n$, on pourrait chercher un polynôme annulateur $Q$\/ de $A$, et on exprime $X^n = Q \times T_n + R_n$, et donc $A^n = R_n(A)$.
			Mais, on peut également diagonaliser $A$\/ (si elle est diagonalisable).
			Ainsi,  \[
				D^n = (P^{-1}\cdot A\cdot P)^n = P^{-1}\cdot A\cdot \cancel P\cdot \cancel{P^{-1}} \cdot \ldots\cdot \cancel{P^{-1}} \cdot A \cdot P = P^{-1}\cdot  A^n\cdot P
			.\] D'où $A^n = P \cdot D^n \cdot P^{-1}$. Or, \[
				D^n = \begin{pmatrix}
					3&0&0\\
					0&1&0\\
					0&0&-1
				\end{pmatrix}^n = \begin{pmatrix}
					3^n&0&0\\
					0&1^n&0\\
					0&0&(-1)^n
				\end{pmatrix}
			.\]
			On calcule donc $A^{n}$\/ en calculant l'inverse de $P$\/ : \[
				A^n = \begin{pmatrix}
					1&1&1\\
					1&1&-1\\
					1&0&0
				\end{pmatrix} \begin{pmatrix}
					3^n&0&0\\
					0&1^n&0\\
					0&0&(-1)^n
				\end{pmatrix} \cdot P^{-1}
			.\]
		\item
			\begin{align*}
				\begin{rcases*}
					\hfill u_{n+1} = v_n + 2w_n\\
					\hfill v_{n+1} = u_n + 2w_n\\
					\hfill w_{n+1} = 3w_n
				\end{rcases*} \iff& \begin{pmatrix}
					u_{n+1}\\v_{n+1}\\w_{n+1}
				\end{pmatrix} = \begin{pmatrix}
					0&1&2\\
					1&0&2\\
					0&0&3
				\end{pmatrix} \begin{pmatrix}
					u_n\\ v_n\\ w_n
				\end{pmatrix}\\
				\iff& U_{n+1} = A\cdot U_n\\
				\iff& U'_{n+1} = D \cdot U'_{n}
			\end{align*}
			où $D = P^{-1} \cdot A \cdot P$, $U'_{n+1} = P\cdot U_{n+1}$\/ et $U'_n = P\cdot U_n$.
			\begin{align*}
				\phantom{\begin{rcases*}
					\hfill mm_{n+1} = v_n + 2w_n\\
					\hfill v_{n+1} = u_n + 2w_n\\
					\hfill w_{n+1} = 3w_n
				\end{rcases*}} \iff&
				\begin{pmatrix}
					u'_{n+1}\\v'_{n+1}\\w'_{n+1}
				\end{pmatrix} = \begin{pmatrix}
					3&0&0\\
					0&1&0\\
					0&0&-1
				\end{pmatrix} \cdot \begin{pmatrix}
					u'_n\\
					v'_n\\
					w'_n
				\end{pmatrix}\\
				\iff& \begin{cases}
					u'_{n+1} = 3u'_n\\
					v'_{n+1} = v'_n\\
					w'_{n+1} = -w'_n
				\end{cases}\\
				\iff& \begin{cases}
					u'_n = K\times  3^n\\
					v'_n = L\\
					w'_n = M \times (-1)^n
				\end{cases}
			\end{align*}
			Ainsi, \[
				\begin{pmatrix}
					u_n\\v_n\\w_n
				\end{pmatrix} = \underbrace{\begin{pmatrix}
					1&1&1\\
					1&1&-1\\
					1&0&0
				\end{pmatrix}}_P \cdot \begin{pmatrix}
					K\times 3^n\\
					L\\
					M\times (-1)^n
				\end{pmatrix}
			.\] D'où $u_n = K\cdot 3^n + L + M \cdot (-1)^n$, $v_n = K\times 3^n + L - M \cdot (-1)^n$\/ et $w_n = K\cdot 3^n$, où les constantes $K$, $L$\/ et $M$\/ sont des constantes fixées par les conditions initiales.
		\item
			\begin{align*}
				\begin{rcases*}
					\hfill x'(t) = y(t) + 2z(t)\\
					\hfill y'(t) = x(t) + 2z(t)\\
					\hfill z'(t) = 3z(t)
				\end{rcases*} \iff& \begin{pmatrix}
					x'(t)\\
					y'(t)\\
					z'(t)
				\end{pmatrix} = \begin{pmatrix}
					0&1&2\\
					1&0&2\\
					0&0&3
				\end{pmatrix} \cdot \begin{pmatrix}
					x(t)\\
					y(t)\\
					z(t)
				\end{pmatrix}\\
				\iff& X'(t) = A\cdot X(t)\\
				\iff& U'(t) = D \cdot U(t) \text{ avec } D = P^{-1} \cdot A\cdot P \text{ et } X(t) = P\cdot U(t)\\
				\iff& \begin{pmatrix}
					u'(t)\\
					v'(t)\\
					w'(t)
				\end{pmatrix} = \begin{pmatrix}
					3&0&0\\
					0&1&0\\
					0&0&-1
				\end{pmatrix} \cdot \begin{pmatrix}
					u(t)\\
					v(t)\\
					w(t)
				\end{pmatrix}\\
				\iff& \begin{cases}
					u'(t) = 3u(t)\\
					v'(t) = v(t)\\
					w'(t) = -w(t)
				\end{cases}\\
				\iff& \begin{cases}
					u(t) = K \cdot \mathrm{e}^{3t}\\
					v(t) = L \cdot \mathrm{e}^{t}\\
					w(t) = M \cdot \mathrm{e}^{-t}
				\end{cases}
			\end{align*}
			Ainsi \[
				\begin{pmatrix}
					x(t)\\
					y(t)\\
					z(t)
				\end{pmatrix} = \underbrace{\begin{pmatrix}
					1&1&1\\
					1&1&-1\\
					1&0&0
				\end{pmatrix}}_P \cdot \begin{pmatrix}
					K \times \mathrm{e}^{3t}\\
					L \cdot \mathrm{e}^{t}\\
					M \cdot \mathrm{e}^{-t}
				\end{pmatrix}
			.\] 
			D'où $x(t) = K\cdot \mathrm{e}^{3t} + L \cdot \mathrm{e}^{t} + M \cdot \mathrm{e}^{-t}$, $y(t) = K \cdot \mathrm{e}^{3t} + L \cdot \mathrm{e}^{t} - M \cdot \mathrm{e}^{-t}$\/ et $z(t) = K\cdot \mathrm{e}^{3t}$. Les constantes $K$, $L$\/ et $M$\/ peuvent être déterminées à partir des conditions initiales.
	\end{enumerate}
\end{exo}

\begin{rmkn}[équations différentielles]
	On considère l'équation différentielle $(*)$ : $x'(t) = \lambda \cdot x(t)$.
	Les fonctions $x : t \mapsto K\cdot \mathrm{e}^{\lambda t}$\/ sont des solutions de cette équation. On peut utiliser la méthode de {\sc Lagrange}\/ : la méthode de la~\guillemotleft~variation de la constante.~\guillemotright\@ On cherche des solutions sous la forme $x(t) = k(t) \cdot \mathrm{e}^{\lambda t}$ (vision du~physicien). D'où $k(t) = x(t) / \mathrm{e}^{\lambda t}$\/ (vision du mathématicien). De plus, $x'(t) = k'(t) \mathrm{e}^{\lambda t} + k(t) \lambda \mathrm{e}^{\lambda t}$.
	Ainsi, on injecte ce $k(t)$\/ dans l'équation différentielle :
	\begin{align*}
		(*) \iff& k'(t) \mathrm{e}^{\lambda t} + k(t) \lambda \mathrm{e}^{\lambda t} = \lambda k(t)\mathrm{e}^{\lambda t}\\
		\iff& k'(t) \mathrm{e}^{\lambda t} = 0\\
		\iff& k'(t) = 0\\
		\iff& \exists K \in \R\,\:k(t) = K.
	\end{align*}
	Les solutions trouvées dans l'exercice précédent sont donc les uniques solutions du système d'équations différentielles.

	De même, pour résoudre une équation différentielle avec 2\tsup{nd} membre de la forme \[
		(**) : \qquad x'(t) - \lambda \cdot x(t) = b(t)
	.\]
	La fonction $t \mapsto x(t)$\/ est une solution de l'équation {\sc sans}\/ 2\tsup{nd} membre si et seulement si \[
		\exists K \in \R,\:\forall t \in \R,\quad x(t) = K \cdot \mathrm{e}^{\lambda t}
	.\]
	\begin{center}
		\slshape Comment résoudre l'équation différentielle {\scshape avec}\/ 2\tsup{nd} membre si on connaît la solution générale de l'équation {\scshape sans}\/ 2\tsup{nd} membre ?
	\end{center}
	On utilise la méthode le la variation de la constante.
	Soit $x(t) = k(t) \cdot \mathrm{e}^{\lambda t}$. Ainsi, en injectant cette expression de $x$\/ dans l'équation $(**)$, on trouve
	\begin{align*}
		(**) \iff& k'(t) \mathrm{e}^{\lambda t} + k(t) \cdot \lambda \mathrm{e}^{\lambda t} = \lambda k(t) \mathrm{e}^{\lambda t} + b(t)\\
		\iff& k'(t) \mathrm{e}^{\lambda t} = b(t)\\
		\iff& k'(t) = b(t) \cdot \mathrm{e}^{-\lambda t}\\
		\iff& k(t) = \int_{0}^{t} b(u)\cdot \mathrm{e}^{-\lambda u}~\mathrm{d}u + K\\
		\iff& x(t) = \left( \int_{0}^{t} b(u) \cdot \mathrm{e}^{-\lambda u}~\mathrm{d}u + K \right) \mathrm{e}^{\lambda t}\\
		\iff& x(t) = \underbrace{\int_{0}^{t} b(u) \cdot \mathrm{e}^{\lambda (t-u)}~\mathrm{d}u}_{\text{solution particulière}} + \underbrace{K \cdot \mathrm{e}^{\lambda t}}_{\substack{\text{solution}\\\text{générale}\\\text{de $(*)$}}}.
	\end{align*}
\end{rmkn}

		\begin{exm}
	On pose $f$, le sinus cardinal :  \begin{align*}
		f: \R^* &\longrightarrow \R \\
		t &\longmapsto \frac{\sin t}{t}.
	\end{align*}
	\begin{figure}[H]
		\centering
		\begin{asy}
			import graph;
			size(10cm);
			draw((-10, 0) -- (10, 0), Arrow(TeXHead));
			draw((0, -3) -- (0, 5), Arrow(TeXHead));
			real f(real x) {
				if(x == 0) { return 3; }
				else {return 3*sin(x) / x;}
			}
			draw(graph(f, -10, 10), magenta);
		\end{asy}
		\caption{Sinus cardinal}
	\end{figure}

	La fonction $f$\/ est continue sur ${]0,8]}$\/ mais $\lim_{t\to 0} \frac{\sin t}{t} = 1$. D'où $\int_{0}^{8} \frac{\sin t}{t}~\mathrm{d}t$\/ est faussement impropre en $0$\/ et donc convergente.


	Mais attention ! On ne dit pas \guillemotleft~{\color{red}soit $f : t \mapsto \frac{1}{t}$. L'intégrale $\int_{8}^{+\infty} \frac{1}{t}~\mathrm{d}t$\/ est faussement impropre en $+\infty$\/ car $\lim_{t\to +\infty}\frac{1}{t} = 0$}.~\guillemotright
\end{exm}

\section{Intégrer les $\mathbf{\sim}$, $\po$, et \textit{O}}

\begin{thm}
	\hfill$\O$\hfill\null
\end{thm}

\begin{thm}
	Le 2.\ n'est pas la réciproque du 1.\ mais la contraposée.
\end{thm}

\begin{prop}
	\hfill$\O$\hfill\null
\end{prop}

\begin{exm}
	On considère l'intégrale $\int_{2}^{+\infty} \frac{1}{t^2+ \cos t}~\mathrm{d}t$, c'est une intégrale impropre en $+\infty$.
	On recherche un équivalent de $\frac{1}{t^2 + \cos t}$\/ en $+\infty$ : \[
		\frac{1}{t^2 + \cos t} \simi_{t\to +\infty} \frac{1}{t^2}
	\] qui ne change pas de signe. Or, $\int_{2}^{+\infty} \frac{1}{t^2}~\mathrm{d}t$\/ converge car c'est une intégrale de {\sc Riemann}\/ avec $\alpha = 2 > 1$.
	On en déduit que l'intégrale $I$\/ converge.

	On procède autrement : \[
		0 \le \frac{1}{t^2 + \cos t} \le \frac{1}{t^2 - 1}
	.\] Or, $\int_{2}^{+\infty} \frac{1}{t^2 - 1}~\mathrm{d}t$\/ converge car
	\begin{align*}
		\int_{2}^{x} \frac{1}{t^2 - 1}~\mathrm{d}t &= \int_{2}^{x} \left( \frac{\sfrac12}{t-1} - \frac{\sfrac12}{t+1} \right) ~\mathrm{d}t \\
		&= \frac{1}{2} \int_{2}^{x} \frac{1}{t-1}~\mathrm{d}t - \frac{1}{2}\int_{2}^{x} \frac{1}{t+1}~\mathrm{d}t \\
		&= \frac{1}{2} \Big[\ln|t-1|\Big]_2^x - \frac{1}{2}\Big[\ln |t+1|\Big]_2^x \\
	\end{align*}
	D'où \[
		\int_{2}^{x} \frac{1}{t^2 - 1}~\mathrm{d}t = \frac{1}{2} \left[ \ln\left| \frac{t-1}{t+1} \right| \right]_2^x = \frac{1}{2}\ln \left| \frac{x-1}{x+1} \right| + \frac{1}{2} \ln 3 \tendsto{x\to +\infty} \frac{1}{2} \ln 3
	.\] donc l'intégrale $I$\/ converge et $I \le \frac{1}{2} \ln_3$.
\end{exm}

\begin{exo}
	\begin{enumerate}
		\item L'intégrale $I = \int_{0}^{1} \frac{\sin t}{t^2}~\mathrm{d}t$\/ est impropre en 0. On utilise un équivalent : $\sin t \simi_{t\to 0} t$\/ qui ne change pas de signe. Or, $\int_{0}^{t} \frac{1}{t}~\mathrm{d}t$\/ diverge (par critère de {\sc Riemann}). Donc $I$\/ diverge.
			
			L'intégrale $J = \int_{1}^{+\infty} \sin \frac{1}{t}~\mathrm{d}t$\/ est généralisée en $+\infty$. On cherche un équivalent en $+\infty$\/ : \[
				\sin \frac{1}{t} \simi_{t\to +\infty} \frac{1}{t}
			\] qui ne change pas de signe. Or, $\int_{1}^{+\infty} \frac{1}{t}~\mathrm{d}t$\/ diverge par critère de {\sc Riemann}. On en déduit que $J$\/ diverge également.
		\item L'intégrale $\int_{0}^{+\infty} \frac{1}{t^2}~\mathrm{d}t$\/ est impropre, {\bf et}\/ en 0, {\bf et}\/ en $+\infty$. Le théorème ne marche donc pas.
			En effet $t\mapsto \frac{1}{t^2}$\/ n'est pas continue par morceaux en 0, ce qui était le cas pour $t\mapsto \frac{1}{1+t^2}$.
	\end{enumerate}
\end{exo}

\begin{rmkn}[Retour sur la {\sc remarque}\/ 5]
	L'intégrale $\int_{0}^{+\infty} \frac{1}{\ln(1+t)}~\mathrm{d}t$\/ est impropre en 0 {\bf et}\/ en $+\infty$. $\int_{0}^{+\infty} \frac{1}{\ln(1+t)}~\mathrm{d}t$\/ converge si et seulement si $\int_{0}^{7} \frac{1}{\ln(1+t)}~\mathrm{d}t$\/ {\bf et}\/ $\int_{7}^{+\infty} \frac{1}{\ln(1+t)}~\mathrm{d}t$\/ convergent.
	Et si elles convergent \[
		\int_{0}^{+\infty} \frac{1}{\ln(1+t)}~\mathrm{d}t = \int_{0}^{7} \frac{1}{\ln(1+t)}~\mathrm{d}t + \int_{7}^{+\infty} \frac{1}{\ln(1+t)}~\mathrm{d}t
	.\]
	On n'utilise pas deux barrières en même temps. Sinon, les intégrales doublement impropres peuvent, et converger, et diverger.
\end{rmkn}

\begin{prop}[avec $\sim$]
	Si $f(t) \simi_{t\to b} g(t)$\/ qui ne change pas de signe. Alors,
	\begin{itemize}
		\item ou bien $\ds\int_{a}^{b} f(t)~\mathrm{d}t$\/ et $\ds\int_{a}^{b} g(t)~\mathrm{d}t$\/ convergent et $\ds \int_{x}^{b} f(t)~\mathrm{d}t \simi_{x\to b} \int_{x}^{b} g(t)~\mathrm{d}t$.
		\item ou bien $\ds\int_{a}^{b} f(t)~\mathrm{d}t$\/ et $\ds\int_{a}^{b} g(t)~\mathrm{d}t$\/ divergent et $\ds\int_{a}^{x} f(t)~\mathrm{d}t \simi_{x\to b} \int_{a}^{x} g(t)~\mathrm{d}t$.
	\end{itemize}
	Cette proposition est équivalente à le {\sc lemme}\/ 13 sur les séries.
\end{prop}



		\begin{prop}
	La relation $\preceq$\/ est un \textit{pré-ordre} :
	\begin{itemize}
		\item $\preceq $\/ est réflective ;
		\item $\preceq $\/ est transitive.
	\end{itemize}
\end{prop}

\begin{prv}
	Soit $Q$\/ un problème de décision.
	\begin{itemize}
		\item $Q \preceq Q$\/ par la fonction identité, qui est totale et calculable.
		\item Soient $Q$, $R$\/ et $S$\/ trois problèmes de décision tels que $Q \preceq R$\/ et $R \preceq S$. Soit donc $f_1$\/ la réduction de $Q$\/ à $R$, et $f_2$\/ la réduction de $R$\/ à $S$. Soit $f = f_2 \circ f_1 : \mathcal{E}_Q \to \mathcal{E}_S$. La fonction $f$\/ est totale comme composée de fonctions totales, $f$\/ est calculable comme composée de fonctions calculables. De plus,
			\begin{align*}
				\forall e \in \mathcal{E}_Q,\qquad f(e) \in S^+ \iff& f_2(f_1(e)) \in S^+\\
				\iff& f_1(e) \in R^+\\
				\iff& e \in Q^+
			\end{align*}
	\end{itemize}
\end{prv}

\section{Classe \textbf{P} et \textbf{NP}}

Pour répondre à un problème, on peut le résoudre par des algorithmes plus ou moins rapides. Mais, l'objectif de cette section est de montrer que certains problèmes ne peuvent se résoudre que par des algorithmes lents, et que l'on ne peut pas faire mieux.

\begin{defn}
	Le modèle de calcul impose une représentation des entrées par chaînes de caractères. Cela induit donc une notion de \textit{taille d'entrée}, qui est la longueur de la chaîne de caractères.
	\index{taille d'entrée}
\end{defn}


\subsection{Complexité d'une machine}

\begin{defn}
	Étant donné une machine $\mathcal{M}$ et une entrée $w \in \Sigma^*$, on note $C^\mathcal{M}(w)$\/ le nombre d'opérations élémentaires effectuées lors de l'appel de $\mathcal{M}$\/ sur $w$. Lorsque $\smash{w \xrightarrow[\mathcal{M}]{} {\circlearrowleft}}$, on définit $C^\mathcal{M} = +\infty$.

	Pour $n \in \N$, on définit alors \[
		C^\mathcal{M}_n = \max \{ C^\mathcal{M}(w)  \mid w \in \Sigma^n \}
	.\]
	\index{machine!nombre d'opérations élémentaires!($C^\mathcal{M}(w)$)}
	\index{machine!nombre d'opérations élémentaires!maximal pour un mot de taille $n$\/ ($C^\mathcal{M}_n$)}
\end{defn}

\begin{rmk}
	On a, $\forall n \in \N$, $C_n^\mathcal{M} \in \bar{\N} = \N \cup \{+\infty\}$.
\end{rmk}

\begin{defn}
	Soit $f : \N\to \N$\/ une fonction totale et calculable. On note $\textsc{Time}(f)$\/ l'ensemble des machines $\mathcal{M}$\/ telles que
	\begin{itemize}
		\item $\mathcal{M}$\/ s'arrête sur toute entrée ;
		\item $\big(C_n^\mathcal{M}\big)_{n \in \N} = \mathcal{O}\big(\big(f(n)\big)_{n \in \N}\big)$.
	\end{itemize}
	\index{machine!ensemble $\textsc{Time}(f)$}
\end{defn}

\subsection{Classe \textbf{P}}

\begin{defn}
	On dit d'une machine $\mathcal{M}$\/ qu'elle est de \textit{complexité polynômiale} dès lors qu'il existe $k \in \N$\/ tel que $\mathcal{M} \in \textsc{Time}(n^k)$.
	\index{machine!de complexité polynômiale}
\end{defn}

\begin{defn}
	On dit d'une fonction (partielle ou non), qu'elle est \textit{calculable en temps polynômial} dès lors qu'il existe une machine $\mathcal{M}$\/ de complexité polynômiale la calculant.
	\index{fonction!calculable!en temps polynômial}
\end{defn}

\begin{exm}
	\begin{itemize}
		\item l'identité ($n \mapsto n$)
		\item la fonction successeur ($n\mapsto n+1$)
	\end{itemize}
\end{exm}


		\section{Arbres couvrants de poids minimum}

\begin{exm}
	On considère le graphe ci-dessous.
	\begin{figure}[H]
		\centering
		\tikzfig{ex-graphe-pondere}
		\caption{Arbre pondéré}
	\end{figure}
	On cherche à \guillemotleft~supprimer~\guillemotright\ des arrêtes de ce graphe afin d'avoir un poids total minimum, tout en conservant la connexité du graphe.
	Une structure assurant cette condition est un arbre.

	Pour résoudre ce problème, on part du graphe vide, et on ajoute les arrêtes les moins coûteuses en premier.
\end{exm}

\begin{defn}[Arbre]
	Soit $G = (S,A)$\/ un graphe non-orienté. On dit que $G$\/ est un \textit{arbre} si $G$\/ est connexe et acyclique.
	\index{arbre}
\end{defn}

\begin{defn}[Arbre couvrant]
	Étant donné un graphe non orienté pondéré par poids positifs $G = (S, A, c)$,\footnotemark\ on dit de $G' = (S', A')$\/ que c'est un \textit{arbre couvrant} de $G$\/ si $S' = S$\/ et $A' \subseteq A$, et $G'$\/ est un arbre.
	\index{arbre!couvrant}
\end{defn}
\footnotetext{on dit que $c$\/ est la fonction de pondération de ce graphe}

\begin{defn}[Arbre couvrant de poids minimum]
	Étant donné un graphe non orienté pondéré $G = (S, A, c)$\/ et un arbre couvrant $T = (S', A')$, on appelle \textit{poids} de l'arbre $T$\/ la valeur $\sum_{a \in A'} c(a)$.
	\index{arbre!couvrant!poids}

	Si $G$\/ est connexe, il admet au moins un arbre couvrant, on peut définir l'\textit{arbre couvrant de poids minimum} (\textit{\textsc{acpm}}).
	\index{arbre!couvrant!de poids minimum}
\end{defn}

On définir alors le problème \[
	\textsc{acpm}\text{\footnotemark}
	\begin{cases}
		\text{\textbf{Entrée}}&: G = (S, A, c) \text{ connexe}\\
		\text{\textbf{Sortie}}&: \text{ le poids de l'arbre couvrant de poids minimum}.
	\end{cases}
\]
\footnotetext{Arbre Couvrant de Poids Minimum}

\begin{algorithm}[H]
	\centering
	\begin{algorithmic}[1]
		\Entree $G = (S, A, c)$\/ un graphe connexe
		\Sortie Un arbre couvrant de poids minimum
		\State $B \gets \O$\/ 
		\State $U \gets \O$\/
		\While{il existe $u$\/ et $v$\/ tels que $u \nsim_B v$}
			\State Soit $\{x,y\} \in A \setminus U$\/ de poids minimal
			\If{$x \sim_B y$}
				\State $U \gets \big\{\!\{x,y\}\!\big\} \cup U$
			\Else
				\State $U \gets \big\{\!\{x,y\}\!\big\} \cup U$
				\State $B \gets \big\{\!\{x,y\}\!\big\} \cup B$
			\EndIf
		\EndWhile
		\State\Return $T = (S,B)$\/
	\end{algorithmic}
	\caption{Algorithme de \textsc{Kruskal}}
\end{algorithm}

\begin{prop}
	L'algorithme de \textsc{Kruskal} est correct.
\end{prop}

\begin{prv}
	\begin{enumerate}
		\item Il existe un arbre couvrant de poids minimum utilisant les arrêtes de $B$ ;
		\item $B \subseteq U \subseteq A$\/ ;
		\item $\forall \{u,v\} \in U$, $u \sim_B v$.
	\end{enumerate}
	Ces trois propriétés sont invariantes.
	\begin{description}
		\item[Initialement] $B = \O = U$, donc \textsc{ok}.
		\item[Propagation] Soient $\ubar{B}$\/ et $\ubar{U}$\/ (resp.\ $\bar{B}, \bar{U}$) les valeurs de $B$\/ et $U$\/ avant (resp.\ après) une itération de boucle. Supposons que $\ubar{B}$\/ et $\ubar{U}$\/ satisfont les propriétés 1, 2 et 3. Montrons que $\bar{B}$\/ et $\bar{U}$\/ les satisfont aussi.
			\begin{enumerate}
				\item[2.] On a $\{x,y\} \in A$\/ et $\ubar{B} \subseteq \ubar{U} \subseteq A$, donc \[
						\bar{B} \subseteq \ubar{B} \cup \{\!\{x,y\}\!\} \subseteq \ubar{U} \cup \{\!\{x,y\}\!\} \subseteq A
					.\]
				\item[3.] Soit $\{u,v\} \in \bar{U}$.
					\begin{itemize}
						\item Si $\{u,v\} \in \ubar{U}$, alors de 3, $u \sim_{\ubar{B}} v$. Or, $\ubar{B} \subseteq \bar{B}$\/ et donc $u \sim_{\bar{B}} v$.
						\item Sinon, $\{ u,v\} = \{x,y\}$, alors $x = u$\/ et $v = y$.
							\begin{itemize}
								\item Sous-cas 1 : $\bar{B} = \ubar{B} \cup \{\!\{x,y\}\!\}$, alors $x \sim_{\bar{B}} y$.
								\item Sous-cas 2 : $\bar{B} = \ubar{B}$, alors par condition du \textbf{si}, $x \sim_{\ubar{B}} y$\/ et donc $x \sim_{\bar{B}} y$.
							\end{itemize}
					\end{itemize}
				\item[1.]
					Soit $\mathcal{T}$\/ un \textsc{acpm} contenant $\ubar{B}$.
					\begin{itemize}
						\item Cas 1 : $\bar{B} = \ubar{B}$, \textsc{ok}
						\item Cas 2 : $\bar{B} = \ubar{B} \cup \{\!\{x,y\}\!\}$.
							\begin{itemize}
								\item Sous-cas 1 : $\{x,y\} \in \mathcal{T}$, alors $\mathcal{T}$\/ est un \textsc{acpm} qui contient $\bar{B}$.
								\item Sous-cas 2 : $\{x,y\}  \not\in \mathcal{T}$, $\mathcal{T}$\/ est un arbre couvrant, donc il contient une chaîne de $x$\/ à $y$\/ : \[
											\{\overset{\substack{x\\[-1mm]\vrt=}}{x_0},x_1\},\{x_1,x_2\},\ldots,\{x_{n-1},\underset{\substack{\vrt=\\y}}{x_n}\}
									.\]
									Or, $\forall i \in \llbracket 1,n-1 \rrbracket$, $x_i \sim_{\ubar{B}} x_{i+1}$. Par transitivité, on a donc $x = x_0 \sim_{\ubar{B}} x_n = y$, ce qui n'est pas le cas.
									Il existe donc $i_0 \in \llbracket 0,n-1 \rrbracket$, tel que $x_{i_0} \nsim_{\ubar{B}} x_{i_0 + 1}$\/ et donc $\{x_{i_0}, x_{i_0 + 1}\} \not\in \ubar{U}$. D'où, d'après 3, on a $\{x_{i_0}, x_{i_0 + 1}\}  \not\in \ubar{B}$
									Considérons alors $\mathcal{T}' = \big(\mathcal{T} \setminus \{\!\{x_{i_0},x_{i_0+1}\}\!\}\big)  \cup \{\!\{x,y\}\!\}$. Montrons que $\mathcal{T}'$\/ est un \textsc{acpm} contenant $B$, en commençant par montrer que c'est un arbre couvrant. L'arbre $\mathcal{T}'$\/ a $n-1$\/ arrêtes (autant que $\mathcal{T}$). Montrons que $\mathcal{T}'$\/ est connexe.
									Soit $(a,b) \in S^2$. $\mathcal{T}$\/ est connexe, soit donc une chaîne \[
										C : \quad a = u_0, u_1, \ldots, u_n = b
									\] de $\mathcal{T}$. Si la chaîne $C$\/ n'utilise pas l'arrête $\{x_{i_0},x_{i_0+1}\}$, alors $C$\/ est une chaîne de $\mathcal{T}'$. Sinon, on pose $\mu$\/ et $\tau$\/ tels que \[
									\underbrace{a,\ldots,x_{i_0}}_{\mu},\underbrace{x_{i_0+1},\ldots,b}_{\tau}
									.\]
									Soit alors la chaîne
									\begin{align*}
										\overbrace{a,\ldots,x_{i_0}}^{\mu},x_{i_0-1},x_{i_0-2},\ldots,x_0 = x,&\\
										\underbrace{b,\ldots,x_{i_0 + 1}}_{\tau},x_{i_0+2},\ldots,x_{n-1},x_n=y&
									\end{align*}
									qui est dans $\mathcal{T}'$.
									Montrons que le poids est minimum. Notons $P(\mathcal{T})$\/ le poids de l'arbre. On a donc \[
										P(\mathcal{T}') = P(\mathcal{T}) + c(\{x,y\}) - c(\{x_{i_0},x_{i_0+1}\})
									.\] Par choix glouton, ($\{x_{i_0}, x_{i_0+1} \not\in \ubar{U}\}$), $c(\{x,y\}) \le c(\{x_{i_0},x_{i_0+1}\})$\/ donc $P(\mathcal{T}') \le P(\mathcal{T})$, et $\mathcal{T}$\/ étant de poids min, $P(\mathcal{T}') = P(\mathcal{T})$\/ et $\mathcal{T}'$\/ est un \textsc{acpm} contenant $\bar{B}$.
							\end{itemize}
					\end{itemize}
			\end{enumerate}
	\end{description}

	Les invariants le sont.
\end{prv}

À la fin, $B$\/ induit un graphe connexe et $B$\/ est contenu dans un \textsc{acpm}, c'en est donc un.


		\paragraph{Une structure pour la gestion des partitions : \textsf{UnionFind}.}

\begin{defn}[Type de données abstrait \textsf{UnionFind}]
	On définit le type de données abstrait \textsf{UnionFind} comme contenant
	\begin{itemize}
		\item un type \texttt{t} de partitions ;
		\item un type \texttt{elem} des éléments manipulés par les partitions ;
		\item $\texttt{initialise\_partition} : \texttt{elem list} \to \texttt{t}$\/ retournant le partitionnement dans lequel chaque élément est seul dans sa classe ;
		\item $\texttt{find} : (\texttt{t} \mathbin{\texttt{*}} \texttt{elem}) \to \texttt{elem}$\/ retournant un représentant de la classe de l'élément. Si deux éléments $x$\/ et $y$\/ sont dans la même classe, dans le partitionnement $p$, alors $\texttt{find}(p,x) = \texttt{find}(p,y)$ ;
		\item $\texttt{union} : (\texttt{t} \mathbin{\texttt{*}} \texttt{elem} \mathbin{\texttt{*}} \texttt{elem}) \to \texttt{t}$\/ retourne le partitionnement dans lequel on a fusionné les classes des arguments.
	\end{itemize}
	\index{type \textsf{UnionFind}}
\end{defn}

\begin{exm}
	On réalise le \textit{pseudo-code} ci-dessous.
	\begin{itemize}
		\item $p \gets \texttt{initialise\_partition}([1, 2, 3, 4, 5])$\/ $\leadsto$ $\{\{1\}, \{2\}, \{3\}, \{4\}, \{5\}\}$
		\item $\texttt{find}(p, 1) = 1$\/ 
		\item $\texttt{union}(p, 1, 3)$\/ $\leadsto$ $\{\{1,3\}, \{2\}, \{4\}, \{5\}\}$
		\item $\texttt{find}(p, 1) = \texttt{find}(p, 3)$
	\end{itemize}
\end{exm}

On implémente ce type abstrait en \textsc{OCaml}.

\begin{rmk}[Niveau zéro -- listes de liste]~
	\begin{lstlisting}[language=caml,caption=Implémentation du type \textsf{UnionFind} en \textsc{OCaml}]
type 'a t = 'a list list

let initialise_partition (l: 'a list): 'a t =
	List.map (fun x -> [ x ] ) l

let rec find (p: 'a t) (x: 'a): 'a =
	match p with
	| classe :: classes ->
			if List.mem x classe then List.hd classe
			else find classes x
	| [] -> raise Not_Found

let est_equiv (p: 'a t) (x: 'a) (y: 'a): bool = 
	(find p x) = (find p y)

let rec extrait_liste (x: 'a) (p: 'a t): 'a list * 'a p =
	match p with
	| classe :: classes ->
			if List.mem x classe then (classe, classes)
			else
				let cl, cls' = extrait_liste x classes in
				(cl, classe :: cls')
	| [] -> raise Not_Found

let union (p: 'a t) (x: 'a) (y: 'a): 'a t =
	if est_equiv p x y then p
	else
		let cx, p' = extrait_liste x p in
		let cy, p'' = extrait_liste y p' in
		(cx @ cy) :: p''
	\end{lstlisting}
\end{rmk}

\begin{rmk}[Niveau un -- tableau de classes]
	Dans la case du tableau, on inscrit le numéro de sa classe.
	Pour \texttt{find}, on prend le premier ayant la même classe.
	Pour \texttt{union}, on re-numérote vers un numéro commun.
	Par exemple, \[
		\begin{array}{|c|c|c|c|c|c|}
			\hline
			0 & 1 & 0 & 0 & 1 & 2\\ \hline
			0 & 1 & 2 & 3 & 4 & 5 \\ \hline
		\end{array}\quad\quad\longleftrightarrow\quad\quad\{\{0,2,3\},\{1,4\},\{5\}\}
	.\]
\end{rmk}

\begin{rmk}[Niveau deux -- tableau de représentants]
	Dans les cases du tableau, on écrit le représentant de la classe de $i$.
	Pour \texttt{find}, on lit la case.
	Pour \texttt{union}, on re-numérote vers un numéro commun.
	Par exemple, \[
		\begin{array}{|c|c|c|c|c|c|}
			\hline
			2 & 4 & 2 & 2 & 4 & 5\\ \hline
			0 & 1 & 2 & 3 & 4 & 5 \\ \hline
		\end{array}\quad\quad\longleftrightarrow\quad\quad\{\{0,2,3\},\{1,4\},\{5\}\}
	.\]
\end{rmk}

\begin{rmk}[Niveau trois -- arbres]
	Pour $\texttt{union}(0, 1)$, on cherche le représentant de 0 (2) puis celui de 1 (4). On fait pointer 4 vers 2.
	Pour la suite de l'implémentation, \textit{c.f.}\ \textsc{dm}$_3$.

	\begin{figure}[H]
		\centering
		\tikzfig{ex-unionfind-arbres}
		\caption{Représentation par des arbres}
	\end{figure}
\end{rmk}

Avec cette nouvelle structure, on peut maintenant revenir sur l'algorithme de \textsc{Kruskal}.

\begin{algorithm}[H]
	\centering
	\begin{algorithmic}[1]
		\Entree Un graphe $G = (S, A, c)$\/ un graphe non orienté, pondéré
		\Sortie Un \textsc{acpm}
		\State Soit $(e_i)_{i\in\llbracket 1,m \rrbracket}$\/ un tri des arrêtes par coût croissant
		\State $f \gets 0$\/ \Comment{Nombre d'\textsl{\texttt{union}} effectuées}
		\State $p \gets \texttt{initialise\_partition}(S)$\/ 
		\State $I \gets 0$\/ 
		\State $B \gets \O$\/ 
		\While{$f < n - 1$}
			\State $\{x,y\} \gets e_I$\/ 
			\If{$\texttt{find}(p, x) \neq \texttt{find}(p, y)$}
				\State $p \gets \texttt{union}(p, x, y)$\/ 
				\State $B \gets B \cup \{\!\{x,y\}\!\}$\/ 
				\State $f \gets f + 1$\/ 
			\EndIf
			\State $I \gets I + 1$\/
		\EndWhile
		\State\Return $(S,B)$
	\end{algorithmic}
	\caption{Algorithme de \textsc{Kruskal} -- version 2}
\end{algorithm}

\paragraph{Étude de complexité.}
Notons $C_{\texttt{find}}^n$\/ un majorant du coût de \texttt{find} sur une structure contenant $n$\/ éléments, notons $C_{\texttt{union}}^n$\/ un majorant du coût de \texttt{union} sur une structure contenant $n$\/ éléments, et notons $C_{\texttt{init}}^n$\/ un majorant du coût de \texttt{init} sur une structure contenant $n$\/ éléments.
La complexité de cet algorithme est de \[
	\mathcal{O}\big(C_{\texttt{init}}^n + 2m\:C_{\texttt{find}}^n + n\: C_{\texttt{union}}^n + m \log_2 m\big)
.\]

\section{Couplage dans un graphe biparti}

\begin{defn}[Couplage]
	On appelle \textit{couplage} d'un graphe non orienté $G = (S, A)$, la donnée d'un sous-ensemble $C \subseteq A$\/ tel que \[
		\forall \{x,y\}, \{x',y'\} \in C,\:
		\quad\quad \{x,y\} \cap \{x',y'\} \neq \O
		\implies
		\{x,y\}  = \{x', y'\}
	.\]

	\index{graphe!couplage}
\end{defn}

\begin{figure}[H]
	\centering
	\tikzfig{ex-couplage}
	\caption{Exemple de couplage}
\end{figure}

\begin{exm}
	On réutilise l'exemple ci-dessous dans toute la section.
	L'ensemble $C = \{\{a,2\}, \{b,3\}\}$\/ est un couplage.
	Mais, l'ensemble $C' = \{\{a,1\}, \{a,2\}\}$\/ n'en est pas un.
\end{exm}

\begin{defn}
	Un couplage est dit \textit{maximal} s'il est maximal pour l'inclusion ($\subseteq$).
	Un couplage est dit \textit{maximum} si son cardinal est maximal.
	\index{graphe!couplage!maximal}
	\index{graphe!couplage!maximum}
\end{defn}

\begin{exm}
	Dans l'exemple précédent, 
	\begin{itemize}
		\item le couplage $C = \{\{a,2\}, \{b,3\}\}$\/ n'est ni maximal, ni maximum ;
		\item le couplage $C' = \{\{a,2\}, \{b, 3\}, \{d, 4\}\}$\/ est maximal mais pas maximum ;
		\item le couplage $C'' = \{\{a,1\}, \{b,3\}, \{c,2\}, \{d,4\}\}$\/ est maximum.
	\end{itemize}
\end{exm}

\begin{rmk}
	Dans toute la suite, on ne considère que des graphes bipartis.
\end{rmk}

\begin{defn}
	Étant donné un graphe biparti $G = (S, A)$\/ et un couplage $C$, un sommet $x$\/ est dit \textit{libre} dès lors que \[
		\forall \{y,z\} \in C,\: x \not\in \{y,z\}
	.\]
	Une chaîne élémentaire\footnotemark $(c_0, c_1, \ldots, c_{2p+1})$\/ est dit \textit{augmentante} si
	\begin{itemize}
		\item $c_0$\/ et $c_{2n+1}$\/ sont libres ;
		\item $\forall i \in \llbracket 0,p \rrbracket$, $\{c_{2i}, c_{2i+1}\}\in A \setminus C$ ;
		\item $\forall i \in \llbracket 0,p-1 \rrbracket$, $\{c_{2i+1}, c_{2i+2}\} \in C$.
	\end{itemize}
\end{defn}
\footnotetext{\textit{i.e.}\ une chaîne sans boucles.}

\begin{exm}~
	\begin{figure}[H]
		\centering
		\tikzfig{ex-chaine-augmentante}
		\caption{Chaîne augmentante}
	\end{figure}
\end{exm}

\begin{exm}
	Dans l'exemple de cette section, $(d, 4)$\/ et $(c, 2, a, 1)$\/ sont deux chaînes augmentantes.
\end{exm}

\begin{prop}
	Étant donné un graphe biparti $G = (S, A)$\/ avec $S = S_1 \cupdot S_2$ (partitionnement du graphe biparti), un couplage $C$\/ est maximum si, et seulement s'il n'admet pas de chaînes augmentantes.
\end{prop}

\begin{prv}
	\begin{itemize}
		\item[``$\implies$'']
			Soit $C$\/ un couplage admettant une chaîne augmentante. Montrons que $C$\/ n'est pas maximum.
			Soit la chaîne augmentante\footnotemark \[
				c_0 \to c_1 \Rightarrow c_2 \to c_3 \Rightarrow c_4 \to \cdots \to c_{2p - 1} \Rightarrow c_{2p} \to c_{2p+1}
			.\]
			On considère alors le couplage \[
				C' = \Big(C \setminus \big\{\{c_{2i+1},c_{2i+2} \} \mid i \in \llbracket 0,p-1 \rrbracket\big\}\Big) \cup \big\{\{c_{2i},c_{2i+1}\}  \mid i \in \llbracket 0,p \rrbracket\big\}
			.\]
			On transforme donc la chaîne en \[
				c_0 \Rightarrow c_1 \to c_2 \Rightarrow c_3 \to \cdots \to c_{2p-1} \to c_{2p} \Rightarrow c_{2p+1}
			.\]
			C'est bien un couplage, et $\Card(C') = \Card C + 1$. $C$\/ n'est donc pas un couplage maximum.
		\item[``$\impliedby$'']
			Soit $C$\/ un couplage non maximum. Montrons que $C$\/ admet une chaîne augmentante. Soit $M$\/ un couplage maximum, et $D = C \mathrel{\triangle} M = (C \setminus M) \cupdot (M \setminus C)$.
			On a $\Card C < \Card M$\/ et $\Card(C \setminus M) < \Card(M \setminus C)$.
			On remarque que, si $c_0 \to c_1 \to c_2 \to \cdots \to c_{p-1}\to c_p$\/ est une chaîne de $D$, (si $c_0 \to c_1 \in C \setminus M$\/ et $c_1 \to c_2 \in C \setminus M$\/ donc $c_1$\/ est dans deux arrêtes distinctes d'un couplage $C$, ce qui est absurde ; de même pour les autres arrêtes). Ainsi, 2 arrêtes consécutives ne sont pas dans la même composante de l'union $(C \setminus M) \cupdot (M \setminus C)$.
			Considérons la relation d'équivalence $\sim$\/ sur $D$\/ définie par $\{x,y\} \sim \{z,t\} \iffdef$ il existe une chaîne de $D$\/ utilisant l'arrête $\{x,y\}$\/ et l'arrête $\{z,t\}$.
			Soit le partitionnement $D_1, \ldots, D_q$\/ de $D$\/ par $\sim$.
			Par inégalité de cardinal, il existe un $D_i$\/ tel que \[
				\Card \{e \in D_i  \mid e \in C\} < \Card \{e \in D_i  \mid e \in M\}
			.\] L'ensemble $D_i$\/ contient alors une chaîne augmentante.
	\end{itemize}
\end{prv}
\footnotetext{On représente $\Rightarrow$ pour les arrêtes dans le couplage $C$.}

		\begin{prv}[par récurrence sur $n$, la largeur de la matrice]
	\begin{itemize}
		\item Si $n = 1$, alors la matrice $A = (a_{11})$\/ est déjà triangulaire.
		\item On suppose le polynôme caractéristique $\chi_A$\/ de la matrice scindé dans $\mathds{K}[X]$, d'où il a au moins une racine dans $\mathds{K}$. D'où, la matrice $A$\/ a au moins une valeur propre $\lambda_1 \in \mathds{K}$. Il existe donc un vecteur non nul $\vec{\varepsilon}_1$\/ tel que $A \cdot \vec{\varepsilon}_1 = \lambda_1\,\vec{\varepsilon}_1$. On complète $(\vec{\varepsilon}_1)$\/ en une base de $\mathds{K}^n$\/ : $(\vec{\varepsilon}_1, \vec{e}_2, \ldots, \vec{e}_n)$. En changent de base, il existe une matrice inversible $P$\/ telle que \[
			A' = P^{-1}\cdot A\cdot P = 
			\begin{pNiceArray}[last-row,last-col]{c|ccc}
				\lambda_1 & *&\Ldots&*&\varepsilon_1\\ \hline
				0 & \Block{3-3}{B}&&&e_1\\
				\Vdots&&&&\Vdots\\
				0&&&&e_n\\
				f(\vec{\varepsilon}_1)&f(\vec{e}_1)&\ldots&f(\vec{e}_n)
			\end{pNiceArray}
		.\]
		Comme le polynôme caractéristique est invariant par changement de base, on en déduit que \[
			\chi_A(x) = \chi_{A'}(x) = \left|
			\begin{array}{c|c}
				x-\lambda_1 &*\\ \hline
				0&xI_{n-1} - B\\
			\end{array} \right| = (x-\lambda_1) \cdot \Pi(x)
		.\]
		Or, comme $\chi_A$\/ est scindé, $\Pi(x)$\/ est aussi scindé.
		Or, $\Pi(x) = \det(xI_{n-1} - B)$ d'où $B$\/ est trigonalisable.
	\end{itemize}
\end{prv}

\begin{crlr}
	Toute matrice de $\mathscr{M}_{n,n}(\C)$\/ est trigonalisable.
\end{crlr}

\begin{exo}\relax
	{\slshape Soit une matrice carrée $A \in \mathscr{M}_{n,n}(\mathds{K})$ (où $\mathds{K}$\/ est $\R$\/ ou $\C$). Montrer que
		\begin{align*}
			(1)\quad\text{la matrice } A \text{ est nilpotente}
			\iff& \text{ le polynôme caractéristique de } A \text{ est } \chi_A(X) = X^n\quad(2)\\
			\iff& \text{ la matrice } A \text{ est trigonalisable avec des zéros}\\
			&\text{ sur sa diagonale} \quad(3)
		\end{align*}
	}

	On montre $``\,(1) \implies (2),"$ $``\,(2) \implies (3)\,"$\/ puis $``\,(3) \implies (1)."$

	\begin{itemize}
		\item[$``\,(3) \implies(1)\,"$] Il existe donc une matrice inversible $P$\/ telle que $T = P^{-1}\cdot A\cdot P$\/ et $T$\/ est une matrice trigonalisable. Or, à chaque produit $A^n \cdot A$, une \guillemotleft~sur-diagonalse~\guillemotright\  de zéros supplémentaires. D'où, à partir d'un certain rang $p$, on a $A^p = 0$. La matrice $A$\/ est donc nilpotente.
		\item[$``\,(2) \implies(3)\,"$] On sait que $\chi_A = X^n = (X-0)^n$\/ est scindé, d'où $A$\/ est trigonalisable.
			Il existe donc une matrice inversible $P$\/ telle que \[
				P^{-1}\cdot A\cdot P = A' = \begin{pmatrix}
					\lambda_1 & * & \ldots & *\\
					0 & \ddots&\ddots&\vdots\\
					\vdots&\ddots&\ddots&*\\
					0&\ldots&0&\lambda_n
				\end{pmatrix}
			.\]
			Et donc $\chi_{A'}(x) = (x-\lambda_1)(x-\lambda_2) \cdots (x-\lambda_n)$.
			Or, le polynôme caractéristique est invariant par changement de base, d'où $\lambda_1 = \lambda_2 = \cdots = \lambda_n$.
		\item[$``\,(1)\implies(2)\,"$] On passe dans $\C$\/ alors $\chi_A$\/ est scindé dans $\C$. D'où, il existe $(\lambda_1, \lambda_2, \ldots, \lambda_n) \in \C^n$\/ tels que \[
			\chi_A(X) = (X - \lambda_1) (X - \lambda_2) \cdots (X-\lambda_n)
		.\]
		D'où, chaque $\lambda_i$\/ est une valeur propre \ul{complexe} de la matrice $A$. Or $A$\/ est nilpotente, d'où, par définition, il existe $p \in \N$\/ tel que $A^p = 0$. Les scalaires $\lambda_i$\/ sont dans le spectre de $A$\/ : en effet, il existe un vecteur colonne $X$\/ non nul tel que $A\cdot X = \lambda_i\,X$, d'où $A^2 \cdot X = A\cdot AX = A\cdot \lambda_iX = \lambda_i^2 X$. De même, $A^3 \cdot X = A \cdot A^2 \cdot X = A \cdot \lambda_i^2 X = \lambda_i^2 (A\cdot X) = \lambda_i^3 X$.
		Et, de \guillemotleft~proche en proche~\guillemotright, on a donc \[
			\forall k \in \N,\:A^k\cdot X = \lambda_i^k X
		.\]
		En particulier, si $k=p$, on a $0 = 0\cdot X = A^p\cdot X = \lambda_i^pX$. D'où $\lambda_i^p X = 0$. Or, $X \neq 0$, d'où $\lambda_i^p = 0$\/ et donc $\lambda_i = 0$.
		Finalement, $\chi_A(X) = (X-\lambda_1) \cdots (X-\lambda_n) = (X-0)\cdots(X-0) = X^n  \in \C[X]$. On a donc $\chi_A(X) \in \R[X]$.
	\end{itemize}
\end{exo}

		\clearpage
		\setcounter{section}{0}		\renewcommand{\thesection}{\llap{Annexe }\thechapter.\Alph{section}}
		\renewcommand{\thesectionnum}{\Alph{section}}
		\section{Comment prouver la correction d'un programme ?}

Avec $\Sigma = \{a,b\}$. Comment montrer qu'un mot a au moins un $a$\/ et un nombre pair de $b$.

\begin{figure}[H]
	\centering
	\tikzfig{annexe-a-automate-1}
	\caption{Automate reconnaissant les mots valides}
\end{figure}

On veut montrer que \[
	P_w : \text{\guillemotleft~}\forall w \in \Sigma^*,\, \forall q \in \mathcal{Q},\:(\text{il existe une exécution par $w$\/ menant à $q$}) \iff w \text{ satisfait } I_q\text{~\guillemotright}
\]
où \[
	I_{\substack{(v,\\\vrt\in\\\mathds{B}}\substack{r)\\\vrt\in\\\{0,1\}}} : \quad
		(|w|_a \ge 1 \iff v)\:\text{et}\:(r = |w|_b\:\text{mod}\:2)
.\]
On le montre par récurrence sur la longueur de $w$\/ : 

\begin{itemize}
	\item[``$\implies$'']
		\begin{itemize}
			\item Pour $w = \varepsilon$, alors  montrons que $\forall q \in \mathcal{Q}$, il existe un exécution menant à $q$\/ étiquetée par $w$\/ (noté $\xrightarrow[\mathcal{A}]{w}q$) si et seulement si $w$\/ satisfait $I_q$.
				\begin{itemize}
					\item $\xrightarrow[\mathcal{A}]{\varepsilon}({\bfm F}, 0)$\/ est vrai, de plus $\varepsilon$\/ satisfait $I_{({\bfm F}, 0)}$\/ ;
					\item sinon si $q \neq ({\bfm F}, 0)$, alors $\xrightarrow[\mathcal{A}]{\varepsilon}q$\/ est fausse, de plus $\varepsilon$\/ ne satisfait pas $I_q$.
				\end{itemize}
			\item Supposons maintenant $P_w$\/ vrai pour tout mot $w$\/ de taille $n$. Soit $w = w_1\ldots w_nw_{n+1}$\/ un mot de taille $n+1$. Notons $\ubar{w} = w_1\ldots w_n$.
				Montrons que $P_w$\/ est vrai. Soit $q \in \mathcal{Q}$. Supposons $\xrightarrow[\mathcal{A}]{w} q$.
				\begin{itemize}
					\item Si $q = ({\bfm F}, 0)$\/ et $w_{n+1} = b$. On a donc $\xrightarrow[\mathcal{A}]{\ubar{w}}({\bfm F},1)$, et, par hypothèse de récurrence, $\ubar{w}$\/ satisfait. Donc $|\ubar{w}|_a = 0$\/ et $|\ubar{w}|_b \equiv 1 \mod 2$\/ donc $|w|_a = 0$\/ et $|w|_b \equiv 0 \mod 2$\/ donc $w$\/ satisfait $I_{({\bfm F}, 0)}$.
					\item De même pour les autres cas.
				\end{itemize}
		\end{itemize}
	\item[``$\impliedby$''] Réciproquement, supposons que $w$\/ satisfait $I_q$.
		\begin{itemize}
			\item Si $w = ({\bfm V},0)$\/ et $w_{n+1} = a$. Alors,
				\begin{itemize}
					\item si $|\ubar{w}|_a = 0$, alors $\ubar{w} $\/ satisfait $I_{({\bfm F},0)}$. Par hypothèse de récurrence, on a donc $\xrightarrow[\mathcal{A}]{\ubar{w}}({\bfm F},0)$\/ et donc $\xrightarrow[\mathcal{A}]{w}({\bfm V},0)$.
					\item si $|\ubar{w}|_b \ge 1$, alors $\ubar{w}$\/ satisfait $I_{({\bfm V},0)}$\/ donc $\xrightarrow[\mathcal{A}]{\ubar{w}}({\bfm V},0)$\/ et donc $\xrightarrow[\mathcal{A}]{\ubar{w}}({\bfm V},0)$.
				\end{itemize}
			\item De même pour les autres cas.
		\end{itemize}
\end{itemize}
On a donc bien \[
	\forall w \in \Sigma^*,\forall q \in \mathcal{Q},\:\xrightarrow[\mathcal{A}]w q \iff w \text{ satisfait } I_q
.\] Finalement,
\begin{align*}
	\mathcal{L}(\mathcal{A}) &= \{w \in \Sigma^* \mid \exists f \in F,\:\xrightarrow[\mathcal{A}]{w} f\} \\
	&= \{w \in \Sigma^*  \mid \xrightarrow[\mathcal{A}]w ({\bfm V},0)\}  \\
	&= \{w \in \Sigma^*  \mid w \text{ satisfait } I_{({\bfm V},0)}\} \\
	&= \{w \in \Sigma^*  \mid |w|_a \ge 1 \text{ et } |w|_b \equiv 0 \mod 2\} \\
\end{align*}

	}
	\def\addmacros#1{#1}
}

{
	\chap[6]{Preuves}
	\minitoc
	\renewcommand{\cwd}{../cours/chap06/}
	\addmacros{
		\section{(Ne pas) être diagonalisable}

\begin{defn}
	Soit une matrice carrée $A$. On dit que $A$\/ est {\it diagonalisable}\/ s'il existe une matrice inversible~$P \in \mathrm{GL}_n(\mathds{K})$\/ telle que $P^{-1}\cdot A\cdot P$\/ est diagonale.
\end{defn}

\begin{exo}
	\begin{enumerate}
		\item Montrons que la matrice $B = {7\: 1\choose 0\:7}$\/ n'est pas diagonalisable.
			Par l'absurde : on suppose qu'il existe $P \in \mathrm{GL}_2(\R)$\/ et $(\lambda_1, \lambda_2) \in \R^2$\/ tels que \[
				P^{-1} \cdot B \cdot P = \begin{bmatrix}
					\lambda_1 & 0\\
					0&\lambda_2
				\end{bmatrix}
			.\] On applique la trace $\tr$\/ et le déterminant $\det$\/ :
			\begin{gather*}
				\tr(B) = \tr{\lambda_1\:0\choose 0\:\lambda_2} \quad\text{d'où}\quad \lambda_1 + \lambda_2 = 7 + 7 = 14 = \s\\
				\det(B) = \det{\lambda_1\:0\choose 0\:\lambda_2} \quad\text{d'où}\quad \lambda_1 \times \lambda_2 = 7 \times 7 = 49 = p
			\end{gather*}
			D'où $\lambda_1$\/ et $\lambda_2$\/ sont des solutions de l'équation $X^2 - \s X + p = 0$. Or
			\begin{align*}
				X^2 - \s X + p = 0 \iff& X^2 - 14X + 49 = 0\\
				\iff& (X-7)^2 = 0\\
				\iff& X = 7.
			\end{align*}
			D'où 
			\begin{align*}
				B = P P^{-1} B P P^{-1} = P \begin{pmatrix}
					7&0\\
					0&7
				\end{pmatrix} P^{-1} = P \cdot 7I_2\cdot P^{-1} = 7I_2.
			\end{align*}
			La matrice $B$\/ n'est donc pas diagonalisable.

			De même, montrons que la matrice $A$\/ n'est pas diagonalisable. On remarque que \[
				A \cdot \mat{1\\1\\1} = \begin{pmatrix}
					0&1&2\\
					1&0&2\\
					0&0&3
				\end{pmatrix} \begin{pmatrix}
					1\\1\\1
				\end{pmatrix} = \begin{pmatrix}
					3\\3\\3
				\end{pmatrix} = 3\begin{pmatrix}
					1\\1\\1
				\end{pmatrix} 
			.\] Ainsi, \[
				P^{-1}\cdot A\cdot P = \begin{pmatrix}
					3&0&0\\
					0&?&0\\
					0&0&?
				\end{pmatrix}\qquad\text{où}\qquad P = \begin{pmatrix}
					1&?&?\\
					1&?&?\\
					1&?&?
				\end{pmatrix}
			.\] De même, $A\left( \substack{1\\1\\0} \right) = 1 \times \left( \substack{1\\1\\0} \right)$. D'où \[
				P^{-1}\cdot A\cdot P = \begin{pmatrix}
					3&0&0\\
					0&1&0\\
					0&0&?
				\end{pmatrix}\qquad\text{où}\qquad P = \begin{pmatrix}
					1&1&?\\
					1&1&?\\
					1&0&?
				\end{pmatrix}
			.\] Finalement, on en conclut que \[
				P = \begin{pmatrix}
					3&0&0\\
					0&1&0\\
					0&0&-1
				\end{pmatrix} \qquad \text{et}\qquad P^{-1}\cdot A\cdot P = \begin{pmatrix}
					1&1&1\\
					1&1&-1\\
					1&0&0
				\end{pmatrix} = D
			.\]
			De plus, la matrice $P$\/ est inversible car $\det P \neq 0$.
		\item Pour calculer $A^n$, on pourrait chercher un polynôme annulateur $Q$\/ de $A$, et on exprime $X^n = Q \times T_n + R_n$, et donc $A^n = R_n(A)$.
			Mais, on peut également diagonaliser $A$\/ (si elle est diagonalisable).
			Ainsi,  \[
				D^n = (P^{-1}\cdot A\cdot P)^n = P^{-1}\cdot A\cdot \cancel P\cdot \cancel{P^{-1}} \cdot \ldots\cdot \cancel{P^{-1}} \cdot A \cdot P = P^{-1}\cdot  A^n\cdot P
			.\] D'où $A^n = P \cdot D^n \cdot P^{-1}$. Or, \[
				D^n = \begin{pmatrix}
					3&0&0\\
					0&1&0\\
					0&0&-1
				\end{pmatrix}^n = \begin{pmatrix}
					3^n&0&0\\
					0&1^n&0\\
					0&0&(-1)^n
				\end{pmatrix}
			.\]
			On calcule donc $A^{n}$\/ en calculant l'inverse de $P$\/ : \[
				A^n = \begin{pmatrix}
					1&1&1\\
					1&1&-1\\
					1&0&0
				\end{pmatrix} \begin{pmatrix}
					3^n&0&0\\
					0&1^n&0\\
					0&0&(-1)^n
				\end{pmatrix} \cdot P^{-1}
			.\]
		\item
			\begin{align*}
				\begin{rcases*}
					\hfill u_{n+1} = v_n + 2w_n\\
					\hfill v_{n+1} = u_n + 2w_n\\
					\hfill w_{n+1} = 3w_n
				\end{rcases*} \iff& \begin{pmatrix}
					u_{n+1}\\v_{n+1}\\w_{n+1}
				\end{pmatrix} = \begin{pmatrix}
					0&1&2\\
					1&0&2\\
					0&0&3
				\end{pmatrix} \begin{pmatrix}
					u_n\\ v_n\\ w_n
				\end{pmatrix}\\
				\iff& U_{n+1} = A\cdot U_n\\
				\iff& U'_{n+1} = D \cdot U'_{n}
			\end{align*}
			où $D = P^{-1} \cdot A \cdot P$, $U'_{n+1} = P\cdot U_{n+1}$\/ et $U'_n = P\cdot U_n$.
			\begin{align*}
				\phantom{\begin{rcases*}
					\hfill mm_{n+1} = v_n + 2w_n\\
					\hfill v_{n+1} = u_n + 2w_n\\
					\hfill w_{n+1} = 3w_n
				\end{rcases*}} \iff&
				\begin{pmatrix}
					u'_{n+1}\\v'_{n+1}\\w'_{n+1}
				\end{pmatrix} = \begin{pmatrix}
					3&0&0\\
					0&1&0\\
					0&0&-1
				\end{pmatrix} \cdot \begin{pmatrix}
					u'_n\\
					v'_n\\
					w'_n
				\end{pmatrix}\\
				\iff& \begin{cases}
					u'_{n+1} = 3u'_n\\
					v'_{n+1} = v'_n\\
					w'_{n+1} = -w'_n
				\end{cases}\\
				\iff& \begin{cases}
					u'_n = K\times  3^n\\
					v'_n = L\\
					w'_n = M \times (-1)^n
				\end{cases}
			\end{align*}
			Ainsi, \[
				\begin{pmatrix}
					u_n\\v_n\\w_n
				\end{pmatrix} = \underbrace{\begin{pmatrix}
					1&1&1\\
					1&1&-1\\
					1&0&0
				\end{pmatrix}}_P \cdot \begin{pmatrix}
					K\times 3^n\\
					L\\
					M\times (-1)^n
				\end{pmatrix}
			.\] D'où $u_n = K\cdot 3^n + L + M \cdot (-1)^n$, $v_n = K\times 3^n + L - M \cdot (-1)^n$\/ et $w_n = K\cdot 3^n$, où les constantes $K$, $L$\/ et $M$\/ sont des constantes fixées par les conditions initiales.
		\item
			\begin{align*}
				\begin{rcases*}
					\hfill x'(t) = y(t) + 2z(t)\\
					\hfill y'(t) = x(t) + 2z(t)\\
					\hfill z'(t) = 3z(t)
				\end{rcases*} \iff& \begin{pmatrix}
					x'(t)\\
					y'(t)\\
					z'(t)
				\end{pmatrix} = \begin{pmatrix}
					0&1&2\\
					1&0&2\\
					0&0&3
				\end{pmatrix} \cdot \begin{pmatrix}
					x(t)\\
					y(t)\\
					z(t)
				\end{pmatrix}\\
				\iff& X'(t) = A\cdot X(t)\\
				\iff& U'(t) = D \cdot U(t) \text{ avec } D = P^{-1} \cdot A\cdot P \text{ et } X(t) = P\cdot U(t)\\
				\iff& \begin{pmatrix}
					u'(t)\\
					v'(t)\\
					w'(t)
				\end{pmatrix} = \begin{pmatrix}
					3&0&0\\
					0&1&0\\
					0&0&-1
				\end{pmatrix} \cdot \begin{pmatrix}
					u(t)\\
					v(t)\\
					w(t)
				\end{pmatrix}\\
				\iff& \begin{cases}
					u'(t) = 3u(t)\\
					v'(t) = v(t)\\
					w'(t) = -w(t)
				\end{cases}\\
				\iff& \begin{cases}
					u(t) = K \cdot \mathrm{e}^{3t}\\
					v(t) = L \cdot \mathrm{e}^{t}\\
					w(t) = M \cdot \mathrm{e}^{-t}
				\end{cases}
			\end{align*}
			Ainsi \[
				\begin{pmatrix}
					x(t)\\
					y(t)\\
					z(t)
				\end{pmatrix} = \underbrace{\begin{pmatrix}
					1&1&1\\
					1&1&-1\\
					1&0&0
				\end{pmatrix}}_P \cdot \begin{pmatrix}
					K \times \mathrm{e}^{3t}\\
					L \cdot \mathrm{e}^{t}\\
					M \cdot \mathrm{e}^{-t}
				\end{pmatrix}
			.\] 
			D'où $x(t) = K\cdot \mathrm{e}^{3t} + L \cdot \mathrm{e}^{t} + M \cdot \mathrm{e}^{-t}$, $y(t) = K \cdot \mathrm{e}^{3t} + L \cdot \mathrm{e}^{t} - M \cdot \mathrm{e}^{-t}$\/ et $z(t) = K\cdot \mathrm{e}^{3t}$. Les constantes $K$, $L$\/ et $M$\/ peuvent être déterminées à partir des conditions initiales.
	\end{enumerate}
\end{exo}

\begin{rmkn}[équations différentielles]
	On considère l'équation différentielle $(*)$ : $x'(t) = \lambda \cdot x(t)$.
	Les fonctions $x : t \mapsto K\cdot \mathrm{e}^{\lambda t}$\/ sont des solutions de cette équation. On peut utiliser la méthode de {\sc Lagrange}\/ : la méthode de la~\guillemotleft~variation de la constante.~\guillemotright\@ On cherche des solutions sous la forme $x(t) = k(t) \cdot \mathrm{e}^{\lambda t}$ (vision du~physicien). D'où $k(t) = x(t) / \mathrm{e}^{\lambda t}$\/ (vision du mathématicien). De plus, $x'(t) = k'(t) \mathrm{e}^{\lambda t} + k(t) \lambda \mathrm{e}^{\lambda t}$.
	Ainsi, on injecte ce $k(t)$\/ dans l'équation différentielle :
	\begin{align*}
		(*) \iff& k'(t) \mathrm{e}^{\lambda t} + k(t) \lambda \mathrm{e}^{\lambda t} = \lambda k(t)\mathrm{e}^{\lambda t}\\
		\iff& k'(t) \mathrm{e}^{\lambda t} = 0\\
		\iff& k'(t) = 0\\
		\iff& \exists K \in \R\,\:k(t) = K.
	\end{align*}
	Les solutions trouvées dans l'exercice précédent sont donc les uniques solutions du système d'équations différentielles.

	De même, pour résoudre une équation différentielle avec 2\tsup{nd} membre de la forme \[
		(**) : \qquad x'(t) - \lambda \cdot x(t) = b(t)
	.\]
	La fonction $t \mapsto x(t)$\/ est une solution de l'équation {\sc sans}\/ 2\tsup{nd} membre si et seulement si \[
		\exists K \in \R,\:\forall t \in \R,\quad x(t) = K \cdot \mathrm{e}^{\lambda t}
	.\]
	\begin{center}
		\slshape Comment résoudre l'équation différentielle {\scshape avec}\/ 2\tsup{nd} membre si on connaît la solution générale de l'équation {\scshape sans}\/ 2\tsup{nd} membre ?
	\end{center}
	On utilise la méthode le la variation de la constante.
	Soit $x(t) = k(t) \cdot \mathrm{e}^{\lambda t}$. Ainsi, en injectant cette expression de $x$\/ dans l'équation $(**)$, on trouve
	\begin{align*}
		(**) \iff& k'(t) \mathrm{e}^{\lambda t} + k(t) \cdot \lambda \mathrm{e}^{\lambda t} = \lambda k(t) \mathrm{e}^{\lambda t} + b(t)\\
		\iff& k'(t) \mathrm{e}^{\lambda t} = b(t)\\
		\iff& k'(t) = b(t) \cdot \mathrm{e}^{-\lambda t}\\
		\iff& k(t) = \int_{0}^{t} b(u)\cdot \mathrm{e}^{-\lambda u}~\mathrm{d}u + K\\
		\iff& x(t) = \left( \int_{0}^{t} b(u) \cdot \mathrm{e}^{-\lambda u}~\mathrm{d}u + K \right) \mathrm{e}^{\lambda t}\\
		\iff& x(t) = \underbrace{\int_{0}^{t} b(u) \cdot \mathrm{e}^{\lambda (t-u)}~\mathrm{d}u}_{\text{solution particulière}} + \underbrace{K \cdot \mathrm{e}^{\lambda t}}_{\substack{\text{solution}\\\text{générale}\\\text{de $(*)$}}}.
	\end{align*}
\end{rmkn}

		\begin{exm}
	On pose $f$, le sinus cardinal :  \begin{align*}
		f: \R^* &\longrightarrow \R \\
		t &\longmapsto \frac{\sin t}{t}.
	\end{align*}
	\begin{figure}[H]
		\centering
		\begin{asy}
			import graph;
			size(10cm);
			draw((-10, 0) -- (10, 0), Arrow(TeXHead));
			draw((0, -3) -- (0, 5), Arrow(TeXHead));
			real f(real x) {
				if(x == 0) { return 3; }
				else {return 3*sin(x) / x;}
			}
			draw(graph(f, -10, 10), magenta);
		\end{asy}
		\caption{Sinus cardinal}
	\end{figure}

	La fonction $f$\/ est continue sur ${]0,8]}$\/ mais $\lim_{t\to 0} \frac{\sin t}{t} = 1$. D'où $\int_{0}^{8} \frac{\sin t}{t}~\mathrm{d}t$\/ est faussement impropre en $0$\/ et donc convergente.


	Mais attention ! On ne dit pas \guillemotleft~{\color{red}soit $f : t \mapsto \frac{1}{t}$. L'intégrale $\int_{8}^{+\infty} \frac{1}{t}~\mathrm{d}t$\/ est faussement impropre en $+\infty$\/ car $\lim_{t\to +\infty}\frac{1}{t} = 0$}.~\guillemotright
\end{exm}

\section{Intégrer les $\mathbf{\sim}$, $\po$, et \textit{O}}

\begin{thm}
	\hfill$\O$\hfill\null
\end{thm}

\begin{thm}
	Le 2.\ n'est pas la réciproque du 1.\ mais la contraposée.
\end{thm}

\begin{prop}
	\hfill$\O$\hfill\null
\end{prop}

\begin{exm}
	On considère l'intégrale $\int_{2}^{+\infty} \frac{1}{t^2+ \cos t}~\mathrm{d}t$, c'est une intégrale impropre en $+\infty$.
	On recherche un équivalent de $\frac{1}{t^2 + \cos t}$\/ en $+\infty$ : \[
		\frac{1}{t^2 + \cos t} \simi_{t\to +\infty} \frac{1}{t^2}
	\] qui ne change pas de signe. Or, $\int_{2}^{+\infty} \frac{1}{t^2}~\mathrm{d}t$\/ converge car c'est une intégrale de {\sc Riemann}\/ avec $\alpha = 2 > 1$.
	On en déduit que l'intégrale $I$\/ converge.

	On procède autrement : \[
		0 \le \frac{1}{t^2 + \cos t} \le \frac{1}{t^2 - 1}
	.\] Or, $\int_{2}^{+\infty} \frac{1}{t^2 - 1}~\mathrm{d}t$\/ converge car
	\begin{align*}
		\int_{2}^{x} \frac{1}{t^2 - 1}~\mathrm{d}t &= \int_{2}^{x} \left( \frac{\sfrac12}{t-1} - \frac{\sfrac12}{t+1} \right) ~\mathrm{d}t \\
		&= \frac{1}{2} \int_{2}^{x} \frac{1}{t-1}~\mathrm{d}t - \frac{1}{2}\int_{2}^{x} \frac{1}{t+1}~\mathrm{d}t \\
		&= \frac{1}{2} \Big[\ln|t-1|\Big]_2^x - \frac{1}{2}\Big[\ln |t+1|\Big]_2^x \\
	\end{align*}
	D'où \[
		\int_{2}^{x} \frac{1}{t^2 - 1}~\mathrm{d}t = \frac{1}{2} \left[ \ln\left| \frac{t-1}{t+1} \right| \right]_2^x = \frac{1}{2}\ln \left| \frac{x-1}{x+1} \right| + \frac{1}{2} \ln 3 \tendsto{x\to +\infty} \frac{1}{2} \ln 3
	.\] donc l'intégrale $I$\/ converge et $I \le \frac{1}{2} \ln_3$.
\end{exm}

\begin{exo}
	\begin{enumerate}
		\item L'intégrale $I = \int_{0}^{1} \frac{\sin t}{t^2}~\mathrm{d}t$\/ est impropre en 0. On utilise un équivalent : $\sin t \simi_{t\to 0} t$\/ qui ne change pas de signe. Or, $\int_{0}^{t} \frac{1}{t}~\mathrm{d}t$\/ diverge (par critère de {\sc Riemann}). Donc $I$\/ diverge.
			
			L'intégrale $J = \int_{1}^{+\infty} \sin \frac{1}{t}~\mathrm{d}t$\/ est généralisée en $+\infty$. On cherche un équivalent en $+\infty$\/ : \[
				\sin \frac{1}{t} \simi_{t\to +\infty} \frac{1}{t}
			\] qui ne change pas de signe. Or, $\int_{1}^{+\infty} \frac{1}{t}~\mathrm{d}t$\/ diverge par critère de {\sc Riemann}. On en déduit que $J$\/ diverge également.
		\item L'intégrale $\int_{0}^{+\infty} \frac{1}{t^2}~\mathrm{d}t$\/ est impropre, {\bf et}\/ en 0, {\bf et}\/ en $+\infty$. Le théorème ne marche donc pas.
			En effet $t\mapsto \frac{1}{t^2}$\/ n'est pas continue par morceaux en 0, ce qui était le cas pour $t\mapsto \frac{1}{1+t^2}$.
	\end{enumerate}
\end{exo}

\begin{rmkn}[Retour sur la {\sc remarque}\/ 5]
	L'intégrale $\int_{0}^{+\infty} \frac{1}{\ln(1+t)}~\mathrm{d}t$\/ est impropre en 0 {\bf et}\/ en $+\infty$. $\int_{0}^{+\infty} \frac{1}{\ln(1+t)}~\mathrm{d}t$\/ converge si et seulement si $\int_{0}^{7} \frac{1}{\ln(1+t)}~\mathrm{d}t$\/ {\bf et}\/ $\int_{7}^{+\infty} \frac{1}{\ln(1+t)}~\mathrm{d}t$\/ convergent.
	Et si elles convergent \[
		\int_{0}^{+\infty} \frac{1}{\ln(1+t)}~\mathrm{d}t = \int_{0}^{7} \frac{1}{\ln(1+t)}~\mathrm{d}t + \int_{7}^{+\infty} \frac{1}{\ln(1+t)}~\mathrm{d}t
	.\]
	On n'utilise pas deux barrières en même temps. Sinon, les intégrales doublement impropres peuvent, et converger, et diverger.
\end{rmkn}

\begin{prop}[avec $\sim$]
	Si $f(t) \simi_{t\to b} g(t)$\/ qui ne change pas de signe. Alors,
	\begin{itemize}
		\item ou bien $\ds\int_{a}^{b} f(t)~\mathrm{d}t$\/ et $\ds\int_{a}^{b} g(t)~\mathrm{d}t$\/ convergent et $\ds \int_{x}^{b} f(t)~\mathrm{d}t \simi_{x\to b} \int_{x}^{b} g(t)~\mathrm{d}t$.
		\item ou bien $\ds\int_{a}^{b} f(t)~\mathrm{d}t$\/ et $\ds\int_{a}^{b} g(t)~\mathrm{d}t$\/ divergent et $\ds\int_{a}^{x} f(t)~\mathrm{d}t \simi_{x\to b} \int_{a}^{x} g(t)~\mathrm{d}t$.
	\end{itemize}
	Cette proposition est équivalente à le {\sc lemme}\/ 13 sur les séries.
\end{prop}



		\begin{prop}
	La relation $\preceq$\/ est un \textit{pré-ordre} :
	\begin{itemize}
		\item $\preceq $\/ est réflective ;
		\item $\preceq $\/ est transitive.
	\end{itemize}
\end{prop}

\begin{prv}
	Soit $Q$\/ un problème de décision.
	\begin{itemize}
		\item $Q \preceq Q$\/ par la fonction identité, qui est totale et calculable.
		\item Soient $Q$, $R$\/ et $S$\/ trois problèmes de décision tels que $Q \preceq R$\/ et $R \preceq S$. Soit donc $f_1$\/ la réduction de $Q$\/ à $R$, et $f_2$\/ la réduction de $R$\/ à $S$. Soit $f = f_2 \circ f_1 : \mathcal{E}_Q \to \mathcal{E}_S$. La fonction $f$\/ est totale comme composée de fonctions totales, $f$\/ est calculable comme composée de fonctions calculables. De plus,
			\begin{align*}
				\forall e \in \mathcal{E}_Q,\qquad f(e) \in S^+ \iff& f_2(f_1(e)) \in S^+\\
				\iff& f_1(e) \in R^+\\
				\iff& e \in Q^+
			\end{align*}
	\end{itemize}
\end{prv}

\section{Classe \textbf{P} et \textbf{NP}}

Pour répondre à un problème, on peut le résoudre par des algorithmes plus ou moins rapides. Mais, l'objectif de cette section est de montrer que certains problèmes ne peuvent se résoudre que par des algorithmes lents, et que l'on ne peut pas faire mieux.

\begin{defn}
	Le modèle de calcul impose une représentation des entrées par chaînes de caractères. Cela induit donc une notion de \textit{taille d'entrée}, qui est la longueur de la chaîne de caractères.
	\index{taille d'entrée}
\end{defn}


\subsection{Complexité d'une machine}

\begin{defn}
	Étant donné une machine $\mathcal{M}$ et une entrée $w \in \Sigma^*$, on note $C^\mathcal{M}(w)$\/ le nombre d'opérations élémentaires effectuées lors de l'appel de $\mathcal{M}$\/ sur $w$. Lorsque $\smash{w \xrightarrow[\mathcal{M}]{} {\circlearrowleft}}$, on définit $C^\mathcal{M} = +\infty$.

	Pour $n \in \N$, on définit alors \[
		C^\mathcal{M}_n = \max \{ C^\mathcal{M}(w)  \mid w \in \Sigma^n \}
	.\]
	\index{machine!nombre d'opérations élémentaires!($C^\mathcal{M}(w)$)}
	\index{machine!nombre d'opérations élémentaires!maximal pour un mot de taille $n$\/ ($C^\mathcal{M}_n$)}
\end{defn}

\begin{rmk}
	On a, $\forall n \in \N$, $C_n^\mathcal{M} \in \bar{\N} = \N \cup \{+\infty\}$.
\end{rmk}

\begin{defn}
	Soit $f : \N\to \N$\/ une fonction totale et calculable. On note $\textsc{Time}(f)$\/ l'ensemble des machines $\mathcal{M}$\/ telles que
	\begin{itemize}
		\item $\mathcal{M}$\/ s'arrête sur toute entrée ;
		\item $\big(C_n^\mathcal{M}\big)_{n \in \N} = \mathcal{O}\big(\big(f(n)\big)_{n \in \N}\big)$.
	\end{itemize}
	\index{machine!ensemble $\textsc{Time}(f)$}
\end{defn}

\subsection{Classe \textbf{P}}

\begin{defn}
	On dit d'une machine $\mathcal{M}$\/ qu'elle est de \textit{complexité polynômiale} dès lors qu'il existe $k \in \N$\/ tel que $\mathcal{M} \in \textsc{Time}(n^k)$.
	\index{machine!de complexité polynômiale}
\end{defn}

\begin{defn}
	On dit d'une fonction (partielle ou non), qu'elle est \textit{calculable en temps polynômial} dès lors qu'il existe une machine $\mathcal{M}$\/ de complexité polynômiale la calculant.
	\index{fonction!calculable!en temps polynômial}
\end{defn}

\begin{exm}
	\begin{itemize}
		\item l'identité ($n \mapsto n$)
		\item la fonction successeur ($n\mapsto n+1$)
	\end{itemize}
\end{exm}


		\section{Arbres couvrants de poids minimum}

\begin{exm}
	On considère le graphe ci-dessous.
	\begin{figure}[H]
		\centering
		\tikzfig{ex-graphe-pondere}
		\caption{Arbre pondéré}
	\end{figure}
	On cherche à \guillemotleft~supprimer~\guillemotright\ des arrêtes de ce graphe afin d'avoir un poids total minimum, tout en conservant la connexité du graphe.
	Une structure assurant cette condition est un arbre.

	Pour résoudre ce problème, on part du graphe vide, et on ajoute les arrêtes les moins coûteuses en premier.
\end{exm}

\begin{defn}[Arbre]
	Soit $G = (S,A)$\/ un graphe non-orienté. On dit que $G$\/ est un \textit{arbre} si $G$\/ est connexe et acyclique.
	\index{arbre}
\end{defn}

\begin{defn}[Arbre couvrant]
	Étant donné un graphe non orienté pondéré par poids positifs $G = (S, A, c)$,\footnotemark\ on dit de $G' = (S', A')$\/ que c'est un \textit{arbre couvrant} de $G$\/ si $S' = S$\/ et $A' \subseteq A$, et $G'$\/ est un arbre.
	\index{arbre!couvrant}
\end{defn}
\footnotetext{on dit que $c$\/ est la fonction de pondération de ce graphe}

\begin{defn}[Arbre couvrant de poids minimum]
	Étant donné un graphe non orienté pondéré $G = (S, A, c)$\/ et un arbre couvrant $T = (S', A')$, on appelle \textit{poids} de l'arbre $T$\/ la valeur $\sum_{a \in A'} c(a)$.
	\index{arbre!couvrant!poids}

	Si $G$\/ est connexe, il admet au moins un arbre couvrant, on peut définir l'\textit{arbre couvrant de poids minimum} (\textit{\textsc{acpm}}).
	\index{arbre!couvrant!de poids minimum}
\end{defn}

On définir alors le problème \[
	\textsc{acpm}\text{\footnotemark}
	\begin{cases}
		\text{\textbf{Entrée}}&: G = (S, A, c) \text{ connexe}\\
		\text{\textbf{Sortie}}&: \text{ le poids de l'arbre couvrant de poids minimum}.
	\end{cases}
\]
\footnotetext{Arbre Couvrant de Poids Minimum}

\begin{algorithm}[H]
	\centering
	\begin{algorithmic}[1]
		\Entree $G = (S, A, c)$\/ un graphe connexe
		\Sortie Un arbre couvrant de poids minimum
		\State $B \gets \O$\/ 
		\State $U \gets \O$\/
		\While{il existe $u$\/ et $v$\/ tels que $u \nsim_B v$}
			\State Soit $\{x,y\} \in A \setminus U$\/ de poids minimal
			\If{$x \sim_B y$}
				\State $U \gets \big\{\!\{x,y\}\!\big\} \cup U$
			\Else
				\State $U \gets \big\{\!\{x,y\}\!\big\} \cup U$
				\State $B \gets \big\{\!\{x,y\}\!\big\} \cup B$
			\EndIf
		\EndWhile
		\State\Return $T = (S,B)$\/
	\end{algorithmic}
	\caption{Algorithme de \textsc{Kruskal}}
\end{algorithm}

\begin{prop}
	L'algorithme de \textsc{Kruskal} est correct.
\end{prop}

\begin{prv}
	\begin{enumerate}
		\item Il existe un arbre couvrant de poids minimum utilisant les arrêtes de $B$ ;
		\item $B \subseteq U \subseteq A$\/ ;
		\item $\forall \{u,v\} \in U$, $u \sim_B v$.
	\end{enumerate}
	Ces trois propriétés sont invariantes.
	\begin{description}
		\item[Initialement] $B = \O = U$, donc \textsc{ok}.
		\item[Propagation] Soient $\ubar{B}$\/ et $\ubar{U}$\/ (resp.\ $\bar{B}, \bar{U}$) les valeurs de $B$\/ et $U$\/ avant (resp.\ après) une itération de boucle. Supposons que $\ubar{B}$\/ et $\ubar{U}$\/ satisfont les propriétés 1, 2 et 3. Montrons que $\bar{B}$\/ et $\bar{U}$\/ les satisfont aussi.
			\begin{enumerate}
				\item[2.] On a $\{x,y\} \in A$\/ et $\ubar{B} \subseteq \ubar{U} \subseteq A$, donc \[
						\bar{B} \subseteq \ubar{B} \cup \{\!\{x,y\}\!\} \subseteq \ubar{U} \cup \{\!\{x,y\}\!\} \subseteq A
					.\]
				\item[3.] Soit $\{u,v\} \in \bar{U}$.
					\begin{itemize}
						\item Si $\{u,v\} \in \ubar{U}$, alors de 3, $u \sim_{\ubar{B}} v$. Or, $\ubar{B} \subseteq \bar{B}$\/ et donc $u \sim_{\bar{B}} v$.
						\item Sinon, $\{ u,v\} = \{x,y\}$, alors $x = u$\/ et $v = y$.
							\begin{itemize}
								\item Sous-cas 1 : $\bar{B} = \ubar{B} \cup \{\!\{x,y\}\!\}$, alors $x \sim_{\bar{B}} y$.
								\item Sous-cas 2 : $\bar{B} = \ubar{B}$, alors par condition du \textbf{si}, $x \sim_{\ubar{B}} y$\/ et donc $x \sim_{\bar{B}} y$.
							\end{itemize}
					\end{itemize}
				\item[1.]
					Soit $\mathcal{T}$\/ un \textsc{acpm} contenant $\ubar{B}$.
					\begin{itemize}
						\item Cas 1 : $\bar{B} = \ubar{B}$, \textsc{ok}
						\item Cas 2 : $\bar{B} = \ubar{B} \cup \{\!\{x,y\}\!\}$.
							\begin{itemize}
								\item Sous-cas 1 : $\{x,y\} \in \mathcal{T}$, alors $\mathcal{T}$\/ est un \textsc{acpm} qui contient $\bar{B}$.
								\item Sous-cas 2 : $\{x,y\}  \not\in \mathcal{T}$, $\mathcal{T}$\/ est un arbre couvrant, donc il contient une chaîne de $x$\/ à $y$\/ : \[
											\{\overset{\substack{x\\[-1mm]\vrt=}}{x_0},x_1\},\{x_1,x_2\},\ldots,\{x_{n-1},\underset{\substack{\vrt=\\y}}{x_n}\}
									.\]
									Or, $\forall i \in \llbracket 1,n-1 \rrbracket$, $x_i \sim_{\ubar{B}} x_{i+1}$. Par transitivité, on a donc $x = x_0 \sim_{\ubar{B}} x_n = y$, ce qui n'est pas le cas.
									Il existe donc $i_0 \in \llbracket 0,n-1 \rrbracket$, tel que $x_{i_0} \nsim_{\ubar{B}} x_{i_0 + 1}$\/ et donc $\{x_{i_0}, x_{i_0 + 1}\} \not\in \ubar{U}$. D'où, d'après 3, on a $\{x_{i_0}, x_{i_0 + 1}\}  \not\in \ubar{B}$
									Considérons alors $\mathcal{T}' = \big(\mathcal{T} \setminus \{\!\{x_{i_0},x_{i_0+1}\}\!\}\big)  \cup \{\!\{x,y\}\!\}$. Montrons que $\mathcal{T}'$\/ est un \textsc{acpm} contenant $B$, en commençant par montrer que c'est un arbre couvrant. L'arbre $\mathcal{T}'$\/ a $n-1$\/ arrêtes (autant que $\mathcal{T}$). Montrons que $\mathcal{T}'$\/ est connexe.
									Soit $(a,b) \in S^2$. $\mathcal{T}$\/ est connexe, soit donc une chaîne \[
										C : \quad a = u_0, u_1, \ldots, u_n = b
									\] de $\mathcal{T}$. Si la chaîne $C$\/ n'utilise pas l'arrête $\{x_{i_0},x_{i_0+1}\}$, alors $C$\/ est une chaîne de $\mathcal{T}'$. Sinon, on pose $\mu$\/ et $\tau$\/ tels que \[
									\underbrace{a,\ldots,x_{i_0}}_{\mu},\underbrace{x_{i_0+1},\ldots,b}_{\tau}
									.\]
									Soit alors la chaîne
									\begin{align*}
										\overbrace{a,\ldots,x_{i_0}}^{\mu},x_{i_0-1},x_{i_0-2},\ldots,x_0 = x,&\\
										\underbrace{b,\ldots,x_{i_0 + 1}}_{\tau},x_{i_0+2},\ldots,x_{n-1},x_n=y&
									\end{align*}
									qui est dans $\mathcal{T}'$.
									Montrons que le poids est minimum. Notons $P(\mathcal{T})$\/ le poids de l'arbre. On a donc \[
										P(\mathcal{T}') = P(\mathcal{T}) + c(\{x,y\}) - c(\{x_{i_0},x_{i_0+1}\})
									.\] Par choix glouton, ($\{x_{i_0}, x_{i_0+1} \not\in \ubar{U}\}$), $c(\{x,y\}) \le c(\{x_{i_0},x_{i_0+1}\})$\/ donc $P(\mathcal{T}') \le P(\mathcal{T})$, et $\mathcal{T}$\/ étant de poids min, $P(\mathcal{T}') = P(\mathcal{T})$\/ et $\mathcal{T}'$\/ est un \textsc{acpm} contenant $\bar{B}$.
							\end{itemize}
					\end{itemize}
			\end{enumerate}
	\end{description}

	Les invariants le sont.
\end{prv}

À la fin, $B$\/ induit un graphe connexe et $B$\/ est contenu dans un \textsc{acpm}, c'en est donc un.


		\paragraph{Une structure pour la gestion des partitions : \textsf{UnionFind}.}

\begin{defn}[Type de données abstrait \textsf{UnionFind}]
	On définit le type de données abstrait \textsf{UnionFind} comme contenant
	\begin{itemize}
		\item un type \texttt{t} de partitions ;
		\item un type \texttt{elem} des éléments manipulés par les partitions ;
		\item $\texttt{initialise\_partition} : \texttt{elem list} \to \texttt{t}$\/ retournant le partitionnement dans lequel chaque élément est seul dans sa classe ;
		\item $\texttt{find} : (\texttt{t} \mathbin{\texttt{*}} \texttt{elem}) \to \texttt{elem}$\/ retournant un représentant de la classe de l'élément. Si deux éléments $x$\/ et $y$\/ sont dans la même classe, dans le partitionnement $p$, alors $\texttt{find}(p,x) = \texttt{find}(p,y)$ ;
		\item $\texttt{union} : (\texttt{t} \mathbin{\texttt{*}} \texttt{elem} \mathbin{\texttt{*}} \texttt{elem}) \to \texttt{t}$\/ retourne le partitionnement dans lequel on a fusionné les classes des arguments.
	\end{itemize}
	\index{type \textsf{UnionFind}}
\end{defn}

\begin{exm}
	On réalise le \textit{pseudo-code} ci-dessous.
	\begin{itemize}
		\item $p \gets \texttt{initialise\_partition}([1, 2, 3, 4, 5])$\/ $\leadsto$ $\{\{1\}, \{2\}, \{3\}, \{4\}, \{5\}\}$
		\item $\texttt{find}(p, 1) = 1$\/ 
		\item $\texttt{union}(p, 1, 3)$\/ $\leadsto$ $\{\{1,3\}, \{2\}, \{4\}, \{5\}\}$
		\item $\texttt{find}(p, 1) = \texttt{find}(p, 3)$
	\end{itemize}
\end{exm}

On implémente ce type abstrait en \textsc{OCaml}.

\begin{rmk}[Niveau zéro -- listes de liste]~
	\begin{lstlisting}[language=caml,caption=Implémentation du type \textsf{UnionFind} en \textsc{OCaml}]
type 'a t = 'a list list

let initialise_partition (l: 'a list): 'a t =
	List.map (fun x -> [ x ] ) l

let rec find (p: 'a t) (x: 'a): 'a =
	match p with
	| classe :: classes ->
			if List.mem x classe then List.hd classe
			else find classes x
	| [] -> raise Not_Found

let est_equiv (p: 'a t) (x: 'a) (y: 'a): bool = 
	(find p x) = (find p y)

let rec extrait_liste (x: 'a) (p: 'a t): 'a list * 'a p =
	match p with
	| classe :: classes ->
			if List.mem x classe then (classe, classes)
			else
				let cl, cls' = extrait_liste x classes in
				(cl, classe :: cls')
	| [] -> raise Not_Found

let union (p: 'a t) (x: 'a) (y: 'a): 'a t =
	if est_equiv p x y then p
	else
		let cx, p' = extrait_liste x p in
		let cy, p'' = extrait_liste y p' in
		(cx @ cy) :: p''
	\end{lstlisting}
\end{rmk}

\begin{rmk}[Niveau un -- tableau de classes]
	Dans la case du tableau, on inscrit le numéro de sa classe.
	Pour \texttt{find}, on prend le premier ayant la même classe.
	Pour \texttt{union}, on re-numérote vers un numéro commun.
	Par exemple, \[
		\begin{array}{|c|c|c|c|c|c|}
			\hline
			0 & 1 & 0 & 0 & 1 & 2\\ \hline
			0 & 1 & 2 & 3 & 4 & 5 \\ \hline
		\end{array}\quad\quad\longleftrightarrow\quad\quad\{\{0,2,3\},\{1,4\},\{5\}\}
	.\]
\end{rmk}

\begin{rmk}[Niveau deux -- tableau de représentants]
	Dans les cases du tableau, on écrit le représentant de la classe de $i$.
	Pour \texttt{find}, on lit la case.
	Pour \texttt{union}, on re-numérote vers un numéro commun.
	Par exemple, \[
		\begin{array}{|c|c|c|c|c|c|}
			\hline
			2 & 4 & 2 & 2 & 4 & 5\\ \hline
			0 & 1 & 2 & 3 & 4 & 5 \\ \hline
		\end{array}\quad\quad\longleftrightarrow\quad\quad\{\{0,2,3\},\{1,4\},\{5\}\}
	.\]
\end{rmk}

\begin{rmk}[Niveau trois -- arbres]
	Pour $\texttt{union}(0, 1)$, on cherche le représentant de 0 (2) puis celui de 1 (4). On fait pointer 4 vers 2.
	Pour la suite de l'implémentation, \textit{c.f.}\ \textsc{dm}$_3$.

	\begin{figure}[H]
		\centering
		\tikzfig{ex-unionfind-arbres}
		\caption{Représentation par des arbres}
	\end{figure}
\end{rmk}

Avec cette nouvelle structure, on peut maintenant revenir sur l'algorithme de \textsc{Kruskal}.

\begin{algorithm}[H]
	\centering
	\begin{algorithmic}[1]
		\Entree Un graphe $G = (S, A, c)$\/ un graphe non orienté, pondéré
		\Sortie Un \textsc{acpm}
		\State Soit $(e_i)_{i\in\llbracket 1,m \rrbracket}$\/ un tri des arrêtes par coût croissant
		\State $f \gets 0$\/ \Comment{Nombre d'\textsl{\texttt{union}} effectuées}
		\State $p \gets \texttt{initialise\_partition}(S)$\/ 
		\State $I \gets 0$\/ 
		\State $B \gets \O$\/ 
		\While{$f < n - 1$}
			\State $\{x,y\} \gets e_I$\/ 
			\If{$\texttt{find}(p, x) \neq \texttt{find}(p, y)$}
				\State $p \gets \texttt{union}(p, x, y)$\/ 
				\State $B \gets B \cup \{\!\{x,y\}\!\}$\/ 
				\State $f \gets f + 1$\/ 
			\EndIf
			\State $I \gets I + 1$\/
		\EndWhile
		\State\Return $(S,B)$
	\end{algorithmic}
	\caption{Algorithme de \textsc{Kruskal} -- version 2}
\end{algorithm}

\paragraph{Étude de complexité.}
Notons $C_{\texttt{find}}^n$\/ un majorant du coût de \texttt{find} sur une structure contenant $n$\/ éléments, notons $C_{\texttt{union}}^n$\/ un majorant du coût de \texttt{union} sur une structure contenant $n$\/ éléments, et notons $C_{\texttt{init}}^n$\/ un majorant du coût de \texttt{init} sur une structure contenant $n$\/ éléments.
La complexité de cet algorithme est de \[
	\mathcal{O}\big(C_{\texttt{init}}^n + 2m\:C_{\texttt{find}}^n + n\: C_{\texttt{union}}^n + m \log_2 m\big)
.\]

\section{Couplage dans un graphe biparti}

\begin{defn}[Couplage]
	On appelle \textit{couplage} d'un graphe non orienté $G = (S, A)$, la donnée d'un sous-ensemble $C \subseteq A$\/ tel que \[
		\forall \{x,y\}, \{x',y'\} \in C,\:
		\quad\quad \{x,y\} \cap \{x',y'\} \neq \O
		\implies
		\{x,y\}  = \{x', y'\}
	.\]

	\index{graphe!couplage}
\end{defn}

\begin{figure}[H]
	\centering
	\tikzfig{ex-couplage}
	\caption{Exemple de couplage}
\end{figure}

\begin{exm}
	On réutilise l'exemple ci-dessous dans toute la section.
	L'ensemble $C = \{\{a,2\}, \{b,3\}\}$\/ est un couplage.
	Mais, l'ensemble $C' = \{\{a,1\}, \{a,2\}\}$\/ n'en est pas un.
\end{exm}

\begin{defn}
	Un couplage est dit \textit{maximal} s'il est maximal pour l'inclusion ($\subseteq$).
	Un couplage est dit \textit{maximum} si son cardinal est maximal.
	\index{graphe!couplage!maximal}
	\index{graphe!couplage!maximum}
\end{defn}

\begin{exm}
	Dans l'exemple précédent, 
	\begin{itemize}
		\item le couplage $C = \{\{a,2\}, \{b,3\}\}$\/ n'est ni maximal, ni maximum ;
		\item le couplage $C' = \{\{a,2\}, \{b, 3\}, \{d, 4\}\}$\/ est maximal mais pas maximum ;
		\item le couplage $C'' = \{\{a,1\}, \{b,3\}, \{c,2\}, \{d,4\}\}$\/ est maximum.
	\end{itemize}
\end{exm}

\begin{rmk}
	Dans toute la suite, on ne considère que des graphes bipartis.
\end{rmk}

\begin{defn}
	Étant donné un graphe biparti $G = (S, A)$\/ et un couplage $C$, un sommet $x$\/ est dit \textit{libre} dès lors que \[
		\forall \{y,z\} \in C,\: x \not\in \{y,z\}
	.\]
	Une chaîne élémentaire\footnotemark $(c_0, c_1, \ldots, c_{2p+1})$\/ est dit \textit{augmentante} si
	\begin{itemize}
		\item $c_0$\/ et $c_{2n+1}$\/ sont libres ;
		\item $\forall i \in \llbracket 0,p \rrbracket$, $\{c_{2i}, c_{2i+1}\}\in A \setminus C$ ;
		\item $\forall i \in \llbracket 0,p-1 \rrbracket$, $\{c_{2i+1}, c_{2i+2}\} \in C$.
	\end{itemize}
\end{defn}
\footnotetext{\textit{i.e.}\ une chaîne sans boucles.}

\begin{exm}~
	\begin{figure}[H]
		\centering
		\tikzfig{ex-chaine-augmentante}
		\caption{Chaîne augmentante}
	\end{figure}
\end{exm}

\begin{exm}
	Dans l'exemple de cette section, $(d, 4)$\/ et $(c, 2, a, 1)$\/ sont deux chaînes augmentantes.
\end{exm}

\begin{prop}
	Étant donné un graphe biparti $G = (S, A)$\/ avec $S = S_1 \cupdot S_2$ (partitionnement du graphe biparti), un couplage $C$\/ est maximum si, et seulement s'il n'admet pas de chaînes augmentantes.
\end{prop}

\begin{prv}
	\begin{itemize}
		\item[``$\implies$'']
			Soit $C$\/ un couplage admettant une chaîne augmentante. Montrons que $C$\/ n'est pas maximum.
			Soit la chaîne augmentante\footnotemark \[
				c_0 \to c_1 \Rightarrow c_2 \to c_3 \Rightarrow c_4 \to \cdots \to c_{2p - 1} \Rightarrow c_{2p} \to c_{2p+1}
			.\]
			On considère alors le couplage \[
				C' = \Big(C \setminus \big\{\{c_{2i+1},c_{2i+2} \} \mid i \in \llbracket 0,p-1 \rrbracket\big\}\Big) \cup \big\{\{c_{2i},c_{2i+1}\}  \mid i \in \llbracket 0,p \rrbracket\big\}
			.\]
			On transforme donc la chaîne en \[
				c_0 \Rightarrow c_1 \to c_2 \Rightarrow c_3 \to \cdots \to c_{2p-1} \to c_{2p} \Rightarrow c_{2p+1}
			.\]
			C'est bien un couplage, et $\Card(C') = \Card C + 1$. $C$\/ n'est donc pas un couplage maximum.
		\item[``$\impliedby$'']
			Soit $C$\/ un couplage non maximum. Montrons que $C$\/ admet une chaîne augmentante. Soit $M$\/ un couplage maximum, et $D = C \mathrel{\triangle} M = (C \setminus M) \cupdot (M \setminus C)$.
			On a $\Card C < \Card M$\/ et $\Card(C \setminus M) < \Card(M \setminus C)$.
			On remarque que, si $c_0 \to c_1 \to c_2 \to \cdots \to c_{p-1}\to c_p$\/ est une chaîne de $D$, (si $c_0 \to c_1 \in C \setminus M$\/ et $c_1 \to c_2 \in C \setminus M$\/ donc $c_1$\/ est dans deux arrêtes distinctes d'un couplage $C$, ce qui est absurde ; de même pour les autres arrêtes). Ainsi, 2 arrêtes consécutives ne sont pas dans la même composante de l'union $(C \setminus M) \cupdot (M \setminus C)$.
			Considérons la relation d'équivalence $\sim$\/ sur $D$\/ définie par $\{x,y\} \sim \{z,t\} \iffdef$ il existe une chaîne de $D$\/ utilisant l'arrête $\{x,y\}$\/ et l'arrête $\{z,t\}$.
			Soit le partitionnement $D_1, \ldots, D_q$\/ de $D$\/ par $\sim$.
			Par inégalité de cardinal, il existe un $D_i$\/ tel que \[
				\Card \{e \in D_i  \mid e \in C\} < \Card \{e \in D_i  \mid e \in M\}
			.\] L'ensemble $D_i$\/ contient alors une chaîne augmentante.
	\end{itemize}
\end{prv}
\footnotetext{On représente $\Rightarrow$ pour les arrêtes dans le couplage $C$.}

		\addrecap
	}
	\def\addmacros#1{#1}
}

{
	\chap[7]{Tentative de réponse à la \textbf{NP}-complétude}
	\minitoc
	\renewcommand{\cwd}{../cours/chap07/}
	\addmacros{
		\section{(Ne pas) être diagonalisable}

\begin{defn}
	Soit une matrice carrée $A$. On dit que $A$\/ est {\it diagonalisable}\/ s'il existe une matrice inversible~$P \in \mathrm{GL}_n(\mathds{K})$\/ telle que $P^{-1}\cdot A\cdot P$\/ est diagonale.
\end{defn}

\begin{exo}
	\begin{enumerate}
		\item Montrons que la matrice $B = {7\: 1\choose 0\:7}$\/ n'est pas diagonalisable.
			Par l'absurde : on suppose qu'il existe $P \in \mathrm{GL}_2(\R)$\/ et $(\lambda_1, \lambda_2) \in \R^2$\/ tels que \[
				P^{-1} \cdot B \cdot P = \begin{bmatrix}
					\lambda_1 & 0\\
					0&\lambda_2
				\end{bmatrix}
			.\] On applique la trace $\tr$\/ et le déterminant $\det$\/ :
			\begin{gather*}
				\tr(B) = \tr{\lambda_1\:0\choose 0\:\lambda_2} \quad\text{d'où}\quad \lambda_1 + \lambda_2 = 7 + 7 = 14 = \s\\
				\det(B) = \det{\lambda_1\:0\choose 0\:\lambda_2} \quad\text{d'où}\quad \lambda_1 \times \lambda_2 = 7 \times 7 = 49 = p
			\end{gather*}
			D'où $\lambda_1$\/ et $\lambda_2$\/ sont des solutions de l'équation $X^2 - \s X + p = 0$. Or
			\begin{align*}
				X^2 - \s X + p = 0 \iff& X^2 - 14X + 49 = 0\\
				\iff& (X-7)^2 = 0\\
				\iff& X = 7.
			\end{align*}
			D'où 
			\begin{align*}
				B = P P^{-1} B P P^{-1} = P \begin{pmatrix}
					7&0\\
					0&7
				\end{pmatrix} P^{-1} = P \cdot 7I_2\cdot P^{-1} = 7I_2.
			\end{align*}
			La matrice $B$\/ n'est donc pas diagonalisable.

			De même, montrons que la matrice $A$\/ n'est pas diagonalisable. On remarque que \[
				A \cdot \mat{1\\1\\1} = \begin{pmatrix}
					0&1&2\\
					1&0&2\\
					0&0&3
				\end{pmatrix} \begin{pmatrix}
					1\\1\\1
				\end{pmatrix} = \begin{pmatrix}
					3\\3\\3
				\end{pmatrix} = 3\begin{pmatrix}
					1\\1\\1
				\end{pmatrix} 
			.\] Ainsi, \[
				P^{-1}\cdot A\cdot P = \begin{pmatrix}
					3&0&0\\
					0&?&0\\
					0&0&?
				\end{pmatrix}\qquad\text{où}\qquad P = \begin{pmatrix}
					1&?&?\\
					1&?&?\\
					1&?&?
				\end{pmatrix}
			.\] De même, $A\left( \substack{1\\1\\0} \right) = 1 \times \left( \substack{1\\1\\0} \right)$. D'où \[
				P^{-1}\cdot A\cdot P = \begin{pmatrix}
					3&0&0\\
					0&1&0\\
					0&0&?
				\end{pmatrix}\qquad\text{où}\qquad P = \begin{pmatrix}
					1&1&?\\
					1&1&?\\
					1&0&?
				\end{pmatrix}
			.\] Finalement, on en conclut que \[
				P = \begin{pmatrix}
					3&0&0\\
					0&1&0\\
					0&0&-1
				\end{pmatrix} \qquad \text{et}\qquad P^{-1}\cdot A\cdot P = \begin{pmatrix}
					1&1&1\\
					1&1&-1\\
					1&0&0
				\end{pmatrix} = D
			.\]
			De plus, la matrice $P$\/ est inversible car $\det P \neq 0$.
		\item Pour calculer $A^n$, on pourrait chercher un polynôme annulateur $Q$\/ de $A$, et on exprime $X^n = Q \times T_n + R_n$, et donc $A^n = R_n(A)$.
			Mais, on peut également diagonaliser $A$\/ (si elle est diagonalisable).
			Ainsi,  \[
				D^n = (P^{-1}\cdot A\cdot P)^n = P^{-1}\cdot A\cdot \cancel P\cdot \cancel{P^{-1}} \cdot \ldots\cdot \cancel{P^{-1}} \cdot A \cdot P = P^{-1}\cdot  A^n\cdot P
			.\] D'où $A^n = P \cdot D^n \cdot P^{-1}$. Or, \[
				D^n = \begin{pmatrix}
					3&0&0\\
					0&1&0\\
					0&0&-1
				\end{pmatrix}^n = \begin{pmatrix}
					3^n&0&0\\
					0&1^n&0\\
					0&0&(-1)^n
				\end{pmatrix}
			.\]
			On calcule donc $A^{n}$\/ en calculant l'inverse de $P$\/ : \[
				A^n = \begin{pmatrix}
					1&1&1\\
					1&1&-1\\
					1&0&0
				\end{pmatrix} \begin{pmatrix}
					3^n&0&0\\
					0&1^n&0\\
					0&0&(-1)^n
				\end{pmatrix} \cdot P^{-1}
			.\]
		\item
			\begin{align*}
				\begin{rcases*}
					\hfill u_{n+1} = v_n + 2w_n\\
					\hfill v_{n+1} = u_n + 2w_n\\
					\hfill w_{n+1} = 3w_n
				\end{rcases*} \iff& \begin{pmatrix}
					u_{n+1}\\v_{n+1}\\w_{n+1}
				\end{pmatrix} = \begin{pmatrix}
					0&1&2\\
					1&0&2\\
					0&0&3
				\end{pmatrix} \begin{pmatrix}
					u_n\\ v_n\\ w_n
				\end{pmatrix}\\
				\iff& U_{n+1} = A\cdot U_n\\
				\iff& U'_{n+1} = D \cdot U'_{n}
			\end{align*}
			où $D = P^{-1} \cdot A \cdot P$, $U'_{n+1} = P\cdot U_{n+1}$\/ et $U'_n = P\cdot U_n$.
			\begin{align*}
				\phantom{\begin{rcases*}
					\hfill mm_{n+1} = v_n + 2w_n\\
					\hfill v_{n+1} = u_n + 2w_n\\
					\hfill w_{n+1} = 3w_n
				\end{rcases*}} \iff&
				\begin{pmatrix}
					u'_{n+1}\\v'_{n+1}\\w'_{n+1}
				\end{pmatrix} = \begin{pmatrix}
					3&0&0\\
					0&1&0\\
					0&0&-1
				\end{pmatrix} \cdot \begin{pmatrix}
					u'_n\\
					v'_n\\
					w'_n
				\end{pmatrix}\\
				\iff& \begin{cases}
					u'_{n+1} = 3u'_n\\
					v'_{n+1} = v'_n\\
					w'_{n+1} = -w'_n
				\end{cases}\\
				\iff& \begin{cases}
					u'_n = K\times  3^n\\
					v'_n = L\\
					w'_n = M \times (-1)^n
				\end{cases}
			\end{align*}
			Ainsi, \[
				\begin{pmatrix}
					u_n\\v_n\\w_n
				\end{pmatrix} = \underbrace{\begin{pmatrix}
					1&1&1\\
					1&1&-1\\
					1&0&0
				\end{pmatrix}}_P \cdot \begin{pmatrix}
					K\times 3^n\\
					L\\
					M\times (-1)^n
				\end{pmatrix}
			.\] D'où $u_n = K\cdot 3^n + L + M \cdot (-1)^n$, $v_n = K\times 3^n + L - M \cdot (-1)^n$\/ et $w_n = K\cdot 3^n$, où les constantes $K$, $L$\/ et $M$\/ sont des constantes fixées par les conditions initiales.
		\item
			\begin{align*}
				\begin{rcases*}
					\hfill x'(t) = y(t) + 2z(t)\\
					\hfill y'(t) = x(t) + 2z(t)\\
					\hfill z'(t) = 3z(t)
				\end{rcases*} \iff& \begin{pmatrix}
					x'(t)\\
					y'(t)\\
					z'(t)
				\end{pmatrix} = \begin{pmatrix}
					0&1&2\\
					1&0&2\\
					0&0&3
				\end{pmatrix} \cdot \begin{pmatrix}
					x(t)\\
					y(t)\\
					z(t)
				\end{pmatrix}\\
				\iff& X'(t) = A\cdot X(t)\\
				\iff& U'(t) = D \cdot U(t) \text{ avec } D = P^{-1} \cdot A\cdot P \text{ et } X(t) = P\cdot U(t)\\
				\iff& \begin{pmatrix}
					u'(t)\\
					v'(t)\\
					w'(t)
				\end{pmatrix} = \begin{pmatrix}
					3&0&0\\
					0&1&0\\
					0&0&-1
				\end{pmatrix} \cdot \begin{pmatrix}
					u(t)\\
					v(t)\\
					w(t)
				\end{pmatrix}\\
				\iff& \begin{cases}
					u'(t) = 3u(t)\\
					v'(t) = v(t)\\
					w'(t) = -w(t)
				\end{cases}\\
				\iff& \begin{cases}
					u(t) = K \cdot \mathrm{e}^{3t}\\
					v(t) = L \cdot \mathrm{e}^{t}\\
					w(t) = M \cdot \mathrm{e}^{-t}
				\end{cases}
			\end{align*}
			Ainsi \[
				\begin{pmatrix}
					x(t)\\
					y(t)\\
					z(t)
				\end{pmatrix} = \underbrace{\begin{pmatrix}
					1&1&1\\
					1&1&-1\\
					1&0&0
				\end{pmatrix}}_P \cdot \begin{pmatrix}
					K \times \mathrm{e}^{3t}\\
					L \cdot \mathrm{e}^{t}\\
					M \cdot \mathrm{e}^{-t}
				\end{pmatrix}
			.\] 
			D'où $x(t) = K\cdot \mathrm{e}^{3t} + L \cdot \mathrm{e}^{t} + M \cdot \mathrm{e}^{-t}$, $y(t) = K \cdot \mathrm{e}^{3t} + L \cdot \mathrm{e}^{t} - M \cdot \mathrm{e}^{-t}$\/ et $z(t) = K\cdot \mathrm{e}^{3t}$. Les constantes $K$, $L$\/ et $M$\/ peuvent être déterminées à partir des conditions initiales.
	\end{enumerate}
\end{exo}

\begin{rmkn}[équations différentielles]
	On considère l'équation différentielle $(*)$ : $x'(t) = \lambda \cdot x(t)$.
	Les fonctions $x : t \mapsto K\cdot \mathrm{e}^{\lambda t}$\/ sont des solutions de cette équation. On peut utiliser la méthode de {\sc Lagrange}\/ : la méthode de la~\guillemotleft~variation de la constante.~\guillemotright\@ On cherche des solutions sous la forme $x(t) = k(t) \cdot \mathrm{e}^{\lambda t}$ (vision du~physicien). D'où $k(t) = x(t) / \mathrm{e}^{\lambda t}$\/ (vision du mathématicien). De plus, $x'(t) = k'(t) \mathrm{e}^{\lambda t} + k(t) \lambda \mathrm{e}^{\lambda t}$.
	Ainsi, on injecte ce $k(t)$\/ dans l'équation différentielle :
	\begin{align*}
		(*) \iff& k'(t) \mathrm{e}^{\lambda t} + k(t) \lambda \mathrm{e}^{\lambda t} = \lambda k(t)\mathrm{e}^{\lambda t}\\
		\iff& k'(t) \mathrm{e}^{\lambda t} = 0\\
		\iff& k'(t) = 0\\
		\iff& \exists K \in \R\,\:k(t) = K.
	\end{align*}
	Les solutions trouvées dans l'exercice précédent sont donc les uniques solutions du système d'équations différentielles.

	De même, pour résoudre une équation différentielle avec 2\tsup{nd} membre de la forme \[
		(**) : \qquad x'(t) - \lambda \cdot x(t) = b(t)
	.\]
	La fonction $t \mapsto x(t)$\/ est une solution de l'équation {\sc sans}\/ 2\tsup{nd} membre si et seulement si \[
		\exists K \in \R,\:\forall t \in \R,\quad x(t) = K \cdot \mathrm{e}^{\lambda t}
	.\]
	\begin{center}
		\slshape Comment résoudre l'équation différentielle {\scshape avec}\/ 2\tsup{nd} membre si on connaît la solution générale de l'équation {\scshape sans}\/ 2\tsup{nd} membre ?
	\end{center}
	On utilise la méthode le la variation de la constante.
	Soit $x(t) = k(t) \cdot \mathrm{e}^{\lambda t}$. Ainsi, en injectant cette expression de $x$\/ dans l'équation $(**)$, on trouve
	\begin{align*}
		(**) \iff& k'(t) \mathrm{e}^{\lambda t} + k(t) \cdot \lambda \mathrm{e}^{\lambda t} = \lambda k(t) \mathrm{e}^{\lambda t} + b(t)\\
		\iff& k'(t) \mathrm{e}^{\lambda t} = b(t)\\
		\iff& k'(t) = b(t) \cdot \mathrm{e}^{-\lambda t}\\
		\iff& k(t) = \int_{0}^{t} b(u)\cdot \mathrm{e}^{-\lambda u}~\mathrm{d}u + K\\
		\iff& x(t) = \left( \int_{0}^{t} b(u) \cdot \mathrm{e}^{-\lambda u}~\mathrm{d}u + K \right) \mathrm{e}^{\lambda t}\\
		\iff& x(t) = \underbrace{\int_{0}^{t} b(u) \cdot \mathrm{e}^{\lambda (t-u)}~\mathrm{d}u}_{\text{solution particulière}} + \underbrace{K \cdot \mathrm{e}^{\lambda t}}_{\substack{\text{solution}\\\text{générale}\\\text{de $(*)$}}}.
	\end{align*}
\end{rmkn}

		\begin{exm}
	On pose $f$, le sinus cardinal :  \begin{align*}
		f: \R^* &\longrightarrow \R \\
		t &\longmapsto \frac{\sin t}{t}.
	\end{align*}
	\begin{figure}[H]
		\centering
		\begin{asy}
			import graph;
			size(10cm);
			draw((-10, 0) -- (10, 0), Arrow(TeXHead));
			draw((0, -3) -- (0, 5), Arrow(TeXHead));
			real f(real x) {
				if(x == 0) { return 3; }
				else {return 3*sin(x) / x;}
			}
			draw(graph(f, -10, 10), magenta);
		\end{asy}
		\caption{Sinus cardinal}
	\end{figure}

	La fonction $f$\/ est continue sur ${]0,8]}$\/ mais $\lim_{t\to 0} \frac{\sin t}{t} = 1$. D'où $\int_{0}^{8} \frac{\sin t}{t}~\mathrm{d}t$\/ est faussement impropre en $0$\/ et donc convergente.


	Mais attention ! On ne dit pas \guillemotleft~{\color{red}soit $f : t \mapsto \frac{1}{t}$. L'intégrale $\int_{8}^{+\infty} \frac{1}{t}~\mathrm{d}t$\/ est faussement impropre en $+\infty$\/ car $\lim_{t\to +\infty}\frac{1}{t} = 0$}.~\guillemotright
\end{exm}

\section{Intégrer les $\mathbf{\sim}$, $\po$, et \textit{O}}

\begin{thm}
	\hfill$\O$\hfill\null
\end{thm}

\begin{thm}
	Le 2.\ n'est pas la réciproque du 1.\ mais la contraposée.
\end{thm}

\begin{prop}
	\hfill$\O$\hfill\null
\end{prop}

\begin{exm}
	On considère l'intégrale $\int_{2}^{+\infty} \frac{1}{t^2+ \cos t}~\mathrm{d}t$, c'est une intégrale impropre en $+\infty$.
	On recherche un équivalent de $\frac{1}{t^2 + \cos t}$\/ en $+\infty$ : \[
		\frac{1}{t^2 + \cos t} \simi_{t\to +\infty} \frac{1}{t^2}
	\] qui ne change pas de signe. Or, $\int_{2}^{+\infty} \frac{1}{t^2}~\mathrm{d}t$\/ converge car c'est une intégrale de {\sc Riemann}\/ avec $\alpha = 2 > 1$.
	On en déduit que l'intégrale $I$\/ converge.

	On procède autrement : \[
		0 \le \frac{1}{t^2 + \cos t} \le \frac{1}{t^2 - 1}
	.\] Or, $\int_{2}^{+\infty} \frac{1}{t^2 - 1}~\mathrm{d}t$\/ converge car
	\begin{align*}
		\int_{2}^{x} \frac{1}{t^2 - 1}~\mathrm{d}t &= \int_{2}^{x} \left( \frac{\sfrac12}{t-1} - \frac{\sfrac12}{t+1} \right) ~\mathrm{d}t \\
		&= \frac{1}{2} \int_{2}^{x} \frac{1}{t-1}~\mathrm{d}t - \frac{1}{2}\int_{2}^{x} \frac{1}{t+1}~\mathrm{d}t \\
		&= \frac{1}{2} \Big[\ln|t-1|\Big]_2^x - \frac{1}{2}\Big[\ln |t+1|\Big]_2^x \\
	\end{align*}
	D'où \[
		\int_{2}^{x} \frac{1}{t^2 - 1}~\mathrm{d}t = \frac{1}{2} \left[ \ln\left| \frac{t-1}{t+1} \right| \right]_2^x = \frac{1}{2}\ln \left| \frac{x-1}{x+1} \right| + \frac{1}{2} \ln 3 \tendsto{x\to +\infty} \frac{1}{2} \ln 3
	.\] donc l'intégrale $I$\/ converge et $I \le \frac{1}{2} \ln_3$.
\end{exm}

\begin{exo}
	\begin{enumerate}
		\item L'intégrale $I = \int_{0}^{1} \frac{\sin t}{t^2}~\mathrm{d}t$\/ est impropre en 0. On utilise un équivalent : $\sin t \simi_{t\to 0} t$\/ qui ne change pas de signe. Or, $\int_{0}^{t} \frac{1}{t}~\mathrm{d}t$\/ diverge (par critère de {\sc Riemann}). Donc $I$\/ diverge.
			
			L'intégrale $J = \int_{1}^{+\infty} \sin \frac{1}{t}~\mathrm{d}t$\/ est généralisée en $+\infty$. On cherche un équivalent en $+\infty$\/ : \[
				\sin \frac{1}{t} \simi_{t\to +\infty} \frac{1}{t}
			\] qui ne change pas de signe. Or, $\int_{1}^{+\infty} \frac{1}{t}~\mathrm{d}t$\/ diverge par critère de {\sc Riemann}. On en déduit que $J$\/ diverge également.
		\item L'intégrale $\int_{0}^{+\infty} \frac{1}{t^2}~\mathrm{d}t$\/ est impropre, {\bf et}\/ en 0, {\bf et}\/ en $+\infty$. Le théorème ne marche donc pas.
			En effet $t\mapsto \frac{1}{t^2}$\/ n'est pas continue par morceaux en 0, ce qui était le cas pour $t\mapsto \frac{1}{1+t^2}$.
	\end{enumerate}
\end{exo}

\begin{rmkn}[Retour sur la {\sc remarque}\/ 5]
	L'intégrale $\int_{0}^{+\infty} \frac{1}{\ln(1+t)}~\mathrm{d}t$\/ est impropre en 0 {\bf et}\/ en $+\infty$. $\int_{0}^{+\infty} \frac{1}{\ln(1+t)}~\mathrm{d}t$\/ converge si et seulement si $\int_{0}^{7} \frac{1}{\ln(1+t)}~\mathrm{d}t$\/ {\bf et}\/ $\int_{7}^{+\infty} \frac{1}{\ln(1+t)}~\mathrm{d}t$\/ convergent.
	Et si elles convergent \[
		\int_{0}^{+\infty} \frac{1}{\ln(1+t)}~\mathrm{d}t = \int_{0}^{7} \frac{1}{\ln(1+t)}~\mathrm{d}t + \int_{7}^{+\infty} \frac{1}{\ln(1+t)}~\mathrm{d}t
	.\]
	On n'utilise pas deux barrières en même temps. Sinon, les intégrales doublement impropres peuvent, et converger, et diverger.
\end{rmkn}

\begin{prop}[avec $\sim$]
	Si $f(t) \simi_{t\to b} g(t)$\/ qui ne change pas de signe. Alors,
	\begin{itemize}
		\item ou bien $\ds\int_{a}^{b} f(t)~\mathrm{d}t$\/ et $\ds\int_{a}^{b} g(t)~\mathrm{d}t$\/ convergent et $\ds \int_{x}^{b} f(t)~\mathrm{d}t \simi_{x\to b} \int_{x}^{b} g(t)~\mathrm{d}t$.
		\item ou bien $\ds\int_{a}^{b} f(t)~\mathrm{d}t$\/ et $\ds\int_{a}^{b} g(t)~\mathrm{d}t$\/ divergent et $\ds\int_{a}^{x} f(t)~\mathrm{d}t \simi_{x\to b} \int_{a}^{x} g(t)~\mathrm{d}t$.
	\end{itemize}
	Cette proposition est équivalente à le {\sc lemme}\/ 13 sur les séries.
\end{prop}



		\begin{prop}
	La relation $\preceq$\/ est un \textit{pré-ordre} :
	\begin{itemize}
		\item $\preceq $\/ est réflective ;
		\item $\preceq $\/ est transitive.
	\end{itemize}
\end{prop}

\begin{prv}
	Soit $Q$\/ un problème de décision.
	\begin{itemize}
		\item $Q \preceq Q$\/ par la fonction identité, qui est totale et calculable.
		\item Soient $Q$, $R$\/ et $S$\/ trois problèmes de décision tels que $Q \preceq R$\/ et $R \preceq S$. Soit donc $f_1$\/ la réduction de $Q$\/ à $R$, et $f_2$\/ la réduction de $R$\/ à $S$. Soit $f = f_2 \circ f_1 : \mathcal{E}_Q \to \mathcal{E}_S$. La fonction $f$\/ est totale comme composée de fonctions totales, $f$\/ est calculable comme composée de fonctions calculables. De plus,
			\begin{align*}
				\forall e \in \mathcal{E}_Q,\qquad f(e) \in S^+ \iff& f_2(f_1(e)) \in S^+\\
				\iff& f_1(e) \in R^+\\
				\iff& e \in Q^+
			\end{align*}
	\end{itemize}
\end{prv}

\section{Classe \textbf{P} et \textbf{NP}}

Pour répondre à un problème, on peut le résoudre par des algorithmes plus ou moins rapides. Mais, l'objectif de cette section est de montrer que certains problèmes ne peuvent se résoudre que par des algorithmes lents, et que l'on ne peut pas faire mieux.

\begin{defn}
	Le modèle de calcul impose une représentation des entrées par chaînes de caractères. Cela induit donc une notion de \textit{taille d'entrée}, qui est la longueur de la chaîne de caractères.
	\index{taille d'entrée}
\end{defn}


\subsection{Complexité d'une machine}

\begin{defn}
	Étant donné une machine $\mathcal{M}$ et une entrée $w \in \Sigma^*$, on note $C^\mathcal{M}(w)$\/ le nombre d'opérations élémentaires effectuées lors de l'appel de $\mathcal{M}$\/ sur $w$. Lorsque $\smash{w \xrightarrow[\mathcal{M}]{} {\circlearrowleft}}$, on définit $C^\mathcal{M} = +\infty$.

	Pour $n \in \N$, on définit alors \[
		C^\mathcal{M}_n = \max \{ C^\mathcal{M}(w)  \mid w \in \Sigma^n \}
	.\]
	\index{machine!nombre d'opérations élémentaires!($C^\mathcal{M}(w)$)}
	\index{machine!nombre d'opérations élémentaires!maximal pour un mot de taille $n$\/ ($C^\mathcal{M}_n$)}
\end{defn}

\begin{rmk}
	On a, $\forall n \in \N$, $C_n^\mathcal{M} \in \bar{\N} = \N \cup \{+\infty\}$.
\end{rmk}

\begin{defn}
	Soit $f : \N\to \N$\/ une fonction totale et calculable. On note $\textsc{Time}(f)$\/ l'ensemble des machines $\mathcal{M}$\/ telles que
	\begin{itemize}
		\item $\mathcal{M}$\/ s'arrête sur toute entrée ;
		\item $\big(C_n^\mathcal{M}\big)_{n \in \N} = \mathcal{O}\big(\big(f(n)\big)_{n \in \N}\big)$.
	\end{itemize}
	\index{machine!ensemble $\textsc{Time}(f)$}
\end{defn}

\subsection{Classe \textbf{P}}

\begin{defn}
	On dit d'une machine $\mathcal{M}$\/ qu'elle est de \textit{complexité polynômiale} dès lors qu'il existe $k \in \N$\/ tel que $\mathcal{M} \in \textsc{Time}(n^k)$.
	\index{machine!de complexité polynômiale}
\end{defn}

\begin{defn}
	On dit d'une fonction (partielle ou non), qu'elle est \textit{calculable en temps polynômial} dès lors qu'il existe une machine $\mathcal{M}$\/ de complexité polynômiale la calculant.
	\index{fonction!calculable!en temps polynômial}
\end{defn}

\begin{exm}
	\begin{itemize}
		\item l'identité ($n \mapsto n$)
		\item la fonction successeur ($n\mapsto n+1$)
	\end{itemize}
\end{exm}


		\clearpage
		\setcounter{section}{0}		\renewcommand{\thesection}{\llap{Annexe }\thechapter.\Alph{section}}
		\renewcommand{\thesectionnum}{\Alph{section}}
		\section{Comment prouver la correction d'un programme ?}

Avec $\Sigma = \{a,b\}$. Comment montrer qu'un mot a au moins un $a$\/ et un nombre pair de $b$.

\begin{figure}[H]
	\centering
	\tikzfig{annexe-a-automate-1}
	\caption{Automate reconnaissant les mots valides}
\end{figure}

On veut montrer que \[
	P_w : \text{\guillemotleft~}\forall w \in \Sigma^*,\, \forall q \in \mathcal{Q},\:(\text{il existe une exécution par $w$\/ menant à $q$}) \iff w \text{ satisfait } I_q\text{~\guillemotright}
\]
où \[
	I_{\substack{(v,\\\vrt\in\\\mathds{B}}\substack{r)\\\vrt\in\\\{0,1\}}} : \quad
		(|w|_a \ge 1 \iff v)\:\text{et}\:(r = |w|_b\:\text{mod}\:2)
.\]
On le montre par récurrence sur la longueur de $w$\/ : 

\begin{itemize}
	\item[``$\implies$'']
		\begin{itemize}
			\item Pour $w = \varepsilon$, alors  montrons que $\forall q \in \mathcal{Q}$, il existe un exécution menant à $q$\/ étiquetée par $w$\/ (noté $\xrightarrow[\mathcal{A}]{w}q$) si et seulement si $w$\/ satisfait $I_q$.
				\begin{itemize}
					\item $\xrightarrow[\mathcal{A}]{\varepsilon}({\bfm F}, 0)$\/ est vrai, de plus $\varepsilon$\/ satisfait $I_{({\bfm F}, 0)}$\/ ;
					\item sinon si $q \neq ({\bfm F}, 0)$, alors $\xrightarrow[\mathcal{A}]{\varepsilon}q$\/ est fausse, de plus $\varepsilon$\/ ne satisfait pas $I_q$.
				\end{itemize}
			\item Supposons maintenant $P_w$\/ vrai pour tout mot $w$\/ de taille $n$. Soit $w = w_1\ldots w_nw_{n+1}$\/ un mot de taille $n+1$. Notons $\ubar{w} = w_1\ldots w_n$.
				Montrons que $P_w$\/ est vrai. Soit $q \in \mathcal{Q}$. Supposons $\xrightarrow[\mathcal{A}]{w} q$.
				\begin{itemize}
					\item Si $q = ({\bfm F}, 0)$\/ et $w_{n+1} = b$. On a donc $\xrightarrow[\mathcal{A}]{\ubar{w}}({\bfm F},1)$, et, par hypothèse de récurrence, $\ubar{w}$\/ satisfait. Donc $|\ubar{w}|_a = 0$\/ et $|\ubar{w}|_b \equiv 1 \mod 2$\/ donc $|w|_a = 0$\/ et $|w|_b \equiv 0 \mod 2$\/ donc $w$\/ satisfait $I_{({\bfm F}, 0)}$.
					\item De même pour les autres cas.
				\end{itemize}
		\end{itemize}
	\item[``$\impliedby$''] Réciproquement, supposons que $w$\/ satisfait $I_q$.
		\begin{itemize}
			\item Si $w = ({\bfm V},0)$\/ et $w_{n+1} = a$. Alors,
				\begin{itemize}
					\item si $|\ubar{w}|_a = 0$, alors $\ubar{w} $\/ satisfait $I_{({\bfm F},0)}$. Par hypothèse de récurrence, on a donc $\xrightarrow[\mathcal{A}]{\ubar{w}}({\bfm F},0)$\/ et donc $\xrightarrow[\mathcal{A}]{w}({\bfm V},0)$.
					\item si $|\ubar{w}|_b \ge 1$, alors $\ubar{w}$\/ satisfait $I_{({\bfm V},0)}$\/ donc $\xrightarrow[\mathcal{A}]{\ubar{w}}({\bfm V},0)$\/ et donc $\xrightarrow[\mathcal{A}]{\ubar{w}}({\bfm V},0)$.
				\end{itemize}
			\item De même pour les autres cas.
		\end{itemize}
\end{itemize}
On a donc bien \[
	\forall w \in \Sigma^*,\forall q \in \mathcal{Q},\:\xrightarrow[\mathcal{A}]w q \iff w \text{ satisfait } I_q
.\] Finalement,
\begin{align*}
	\mathcal{L}(\mathcal{A}) &= \{w \in \Sigma^* \mid \exists f \in F,\:\xrightarrow[\mathcal{A}]{w} f\} \\
	&= \{w \in \Sigma^*  \mid \xrightarrow[\mathcal{A}]w ({\bfm V},0)\}  \\
	&= \{w \in \Sigma^*  \mid w \text{ satisfait } I_{({\bfm V},0)}\} \\
	&= \{w \in \Sigma^*  \mid |w|_a \ge 1 \text{ et } |w|_b \equiv 0 \mod 2\} \\
\end{align*}

	}
	\def\addmacros#1{#1}
}

{
	\chap[8]{Jeux}
	\minitoc
	\renewcommand{\cwd}{../cours/chap08/}
	\addmacros{
		\section{(Ne pas) être diagonalisable}

\begin{defn}
	Soit une matrice carrée $A$. On dit que $A$\/ est {\it diagonalisable}\/ s'il existe une matrice inversible~$P \in \mathrm{GL}_n(\mathds{K})$\/ telle que $P^{-1}\cdot A\cdot P$\/ est diagonale.
\end{defn}

\begin{exo}
	\begin{enumerate}
		\item Montrons que la matrice $B = {7\: 1\choose 0\:7}$\/ n'est pas diagonalisable.
			Par l'absurde : on suppose qu'il existe $P \in \mathrm{GL}_2(\R)$\/ et $(\lambda_1, \lambda_2) \in \R^2$\/ tels que \[
				P^{-1} \cdot B \cdot P = \begin{bmatrix}
					\lambda_1 & 0\\
					0&\lambda_2
				\end{bmatrix}
			.\] On applique la trace $\tr$\/ et le déterminant $\det$\/ :
			\begin{gather*}
				\tr(B) = \tr{\lambda_1\:0\choose 0\:\lambda_2} \quad\text{d'où}\quad \lambda_1 + \lambda_2 = 7 + 7 = 14 = \s\\
				\det(B) = \det{\lambda_1\:0\choose 0\:\lambda_2} \quad\text{d'où}\quad \lambda_1 \times \lambda_2 = 7 \times 7 = 49 = p
			\end{gather*}
			D'où $\lambda_1$\/ et $\lambda_2$\/ sont des solutions de l'équation $X^2 - \s X + p = 0$. Or
			\begin{align*}
				X^2 - \s X + p = 0 \iff& X^2 - 14X + 49 = 0\\
				\iff& (X-7)^2 = 0\\
				\iff& X = 7.
			\end{align*}
			D'où 
			\begin{align*}
				B = P P^{-1} B P P^{-1} = P \begin{pmatrix}
					7&0\\
					0&7
				\end{pmatrix} P^{-1} = P \cdot 7I_2\cdot P^{-1} = 7I_2.
			\end{align*}
			La matrice $B$\/ n'est donc pas diagonalisable.

			De même, montrons que la matrice $A$\/ n'est pas diagonalisable. On remarque que \[
				A \cdot \mat{1\\1\\1} = \begin{pmatrix}
					0&1&2\\
					1&0&2\\
					0&0&3
				\end{pmatrix} \begin{pmatrix}
					1\\1\\1
				\end{pmatrix} = \begin{pmatrix}
					3\\3\\3
				\end{pmatrix} = 3\begin{pmatrix}
					1\\1\\1
				\end{pmatrix} 
			.\] Ainsi, \[
				P^{-1}\cdot A\cdot P = \begin{pmatrix}
					3&0&0\\
					0&?&0\\
					0&0&?
				\end{pmatrix}\qquad\text{où}\qquad P = \begin{pmatrix}
					1&?&?\\
					1&?&?\\
					1&?&?
				\end{pmatrix}
			.\] De même, $A\left( \substack{1\\1\\0} \right) = 1 \times \left( \substack{1\\1\\0} \right)$. D'où \[
				P^{-1}\cdot A\cdot P = \begin{pmatrix}
					3&0&0\\
					0&1&0\\
					0&0&?
				\end{pmatrix}\qquad\text{où}\qquad P = \begin{pmatrix}
					1&1&?\\
					1&1&?\\
					1&0&?
				\end{pmatrix}
			.\] Finalement, on en conclut que \[
				P = \begin{pmatrix}
					3&0&0\\
					0&1&0\\
					0&0&-1
				\end{pmatrix} \qquad \text{et}\qquad P^{-1}\cdot A\cdot P = \begin{pmatrix}
					1&1&1\\
					1&1&-1\\
					1&0&0
				\end{pmatrix} = D
			.\]
			De plus, la matrice $P$\/ est inversible car $\det P \neq 0$.
		\item Pour calculer $A^n$, on pourrait chercher un polynôme annulateur $Q$\/ de $A$, et on exprime $X^n = Q \times T_n + R_n$, et donc $A^n = R_n(A)$.
			Mais, on peut également diagonaliser $A$\/ (si elle est diagonalisable).
			Ainsi,  \[
				D^n = (P^{-1}\cdot A\cdot P)^n = P^{-1}\cdot A\cdot \cancel P\cdot \cancel{P^{-1}} \cdot \ldots\cdot \cancel{P^{-1}} \cdot A \cdot P = P^{-1}\cdot  A^n\cdot P
			.\] D'où $A^n = P \cdot D^n \cdot P^{-1}$. Or, \[
				D^n = \begin{pmatrix}
					3&0&0\\
					0&1&0\\
					0&0&-1
				\end{pmatrix}^n = \begin{pmatrix}
					3^n&0&0\\
					0&1^n&0\\
					0&0&(-1)^n
				\end{pmatrix}
			.\]
			On calcule donc $A^{n}$\/ en calculant l'inverse de $P$\/ : \[
				A^n = \begin{pmatrix}
					1&1&1\\
					1&1&-1\\
					1&0&0
				\end{pmatrix} \begin{pmatrix}
					3^n&0&0\\
					0&1^n&0\\
					0&0&(-1)^n
				\end{pmatrix} \cdot P^{-1}
			.\]
		\item
			\begin{align*}
				\begin{rcases*}
					\hfill u_{n+1} = v_n + 2w_n\\
					\hfill v_{n+1} = u_n + 2w_n\\
					\hfill w_{n+1} = 3w_n
				\end{rcases*} \iff& \begin{pmatrix}
					u_{n+1}\\v_{n+1}\\w_{n+1}
				\end{pmatrix} = \begin{pmatrix}
					0&1&2\\
					1&0&2\\
					0&0&3
				\end{pmatrix} \begin{pmatrix}
					u_n\\ v_n\\ w_n
				\end{pmatrix}\\
				\iff& U_{n+1} = A\cdot U_n\\
				\iff& U'_{n+1} = D \cdot U'_{n}
			\end{align*}
			où $D = P^{-1} \cdot A \cdot P$, $U'_{n+1} = P\cdot U_{n+1}$\/ et $U'_n = P\cdot U_n$.
			\begin{align*}
				\phantom{\begin{rcases*}
					\hfill mm_{n+1} = v_n + 2w_n\\
					\hfill v_{n+1} = u_n + 2w_n\\
					\hfill w_{n+1} = 3w_n
				\end{rcases*}} \iff&
				\begin{pmatrix}
					u'_{n+1}\\v'_{n+1}\\w'_{n+1}
				\end{pmatrix} = \begin{pmatrix}
					3&0&0\\
					0&1&0\\
					0&0&-1
				\end{pmatrix} \cdot \begin{pmatrix}
					u'_n\\
					v'_n\\
					w'_n
				\end{pmatrix}\\
				\iff& \begin{cases}
					u'_{n+1} = 3u'_n\\
					v'_{n+1} = v'_n\\
					w'_{n+1} = -w'_n
				\end{cases}\\
				\iff& \begin{cases}
					u'_n = K\times  3^n\\
					v'_n = L\\
					w'_n = M \times (-1)^n
				\end{cases}
			\end{align*}
			Ainsi, \[
				\begin{pmatrix}
					u_n\\v_n\\w_n
				\end{pmatrix} = \underbrace{\begin{pmatrix}
					1&1&1\\
					1&1&-1\\
					1&0&0
				\end{pmatrix}}_P \cdot \begin{pmatrix}
					K\times 3^n\\
					L\\
					M\times (-1)^n
				\end{pmatrix}
			.\] D'où $u_n = K\cdot 3^n + L + M \cdot (-1)^n$, $v_n = K\times 3^n + L - M \cdot (-1)^n$\/ et $w_n = K\cdot 3^n$, où les constantes $K$, $L$\/ et $M$\/ sont des constantes fixées par les conditions initiales.
		\item
			\begin{align*}
				\begin{rcases*}
					\hfill x'(t) = y(t) + 2z(t)\\
					\hfill y'(t) = x(t) + 2z(t)\\
					\hfill z'(t) = 3z(t)
				\end{rcases*} \iff& \begin{pmatrix}
					x'(t)\\
					y'(t)\\
					z'(t)
				\end{pmatrix} = \begin{pmatrix}
					0&1&2\\
					1&0&2\\
					0&0&3
				\end{pmatrix} \cdot \begin{pmatrix}
					x(t)\\
					y(t)\\
					z(t)
				\end{pmatrix}\\
				\iff& X'(t) = A\cdot X(t)\\
				\iff& U'(t) = D \cdot U(t) \text{ avec } D = P^{-1} \cdot A\cdot P \text{ et } X(t) = P\cdot U(t)\\
				\iff& \begin{pmatrix}
					u'(t)\\
					v'(t)\\
					w'(t)
				\end{pmatrix} = \begin{pmatrix}
					3&0&0\\
					0&1&0\\
					0&0&-1
				\end{pmatrix} \cdot \begin{pmatrix}
					u(t)\\
					v(t)\\
					w(t)
				\end{pmatrix}\\
				\iff& \begin{cases}
					u'(t) = 3u(t)\\
					v'(t) = v(t)\\
					w'(t) = -w(t)
				\end{cases}\\
				\iff& \begin{cases}
					u(t) = K \cdot \mathrm{e}^{3t}\\
					v(t) = L \cdot \mathrm{e}^{t}\\
					w(t) = M \cdot \mathrm{e}^{-t}
				\end{cases}
			\end{align*}
			Ainsi \[
				\begin{pmatrix}
					x(t)\\
					y(t)\\
					z(t)
				\end{pmatrix} = \underbrace{\begin{pmatrix}
					1&1&1\\
					1&1&-1\\
					1&0&0
				\end{pmatrix}}_P \cdot \begin{pmatrix}
					K \times \mathrm{e}^{3t}\\
					L \cdot \mathrm{e}^{t}\\
					M \cdot \mathrm{e}^{-t}
				\end{pmatrix}
			.\] 
			D'où $x(t) = K\cdot \mathrm{e}^{3t} + L \cdot \mathrm{e}^{t} + M \cdot \mathrm{e}^{-t}$, $y(t) = K \cdot \mathrm{e}^{3t} + L \cdot \mathrm{e}^{t} - M \cdot \mathrm{e}^{-t}$\/ et $z(t) = K\cdot \mathrm{e}^{3t}$. Les constantes $K$, $L$\/ et $M$\/ peuvent être déterminées à partir des conditions initiales.
	\end{enumerate}
\end{exo}

\begin{rmkn}[équations différentielles]
	On considère l'équation différentielle $(*)$ : $x'(t) = \lambda \cdot x(t)$.
	Les fonctions $x : t \mapsto K\cdot \mathrm{e}^{\lambda t}$\/ sont des solutions de cette équation. On peut utiliser la méthode de {\sc Lagrange}\/ : la méthode de la~\guillemotleft~variation de la constante.~\guillemotright\@ On cherche des solutions sous la forme $x(t) = k(t) \cdot \mathrm{e}^{\lambda t}$ (vision du~physicien). D'où $k(t) = x(t) / \mathrm{e}^{\lambda t}$\/ (vision du mathématicien). De plus, $x'(t) = k'(t) \mathrm{e}^{\lambda t} + k(t) \lambda \mathrm{e}^{\lambda t}$.
	Ainsi, on injecte ce $k(t)$\/ dans l'équation différentielle :
	\begin{align*}
		(*) \iff& k'(t) \mathrm{e}^{\lambda t} + k(t) \lambda \mathrm{e}^{\lambda t} = \lambda k(t)\mathrm{e}^{\lambda t}\\
		\iff& k'(t) \mathrm{e}^{\lambda t} = 0\\
		\iff& k'(t) = 0\\
		\iff& \exists K \in \R\,\:k(t) = K.
	\end{align*}
	Les solutions trouvées dans l'exercice précédent sont donc les uniques solutions du système d'équations différentielles.

	De même, pour résoudre une équation différentielle avec 2\tsup{nd} membre de la forme \[
		(**) : \qquad x'(t) - \lambda \cdot x(t) = b(t)
	.\]
	La fonction $t \mapsto x(t)$\/ est une solution de l'équation {\sc sans}\/ 2\tsup{nd} membre si et seulement si \[
		\exists K \in \R,\:\forall t \in \R,\quad x(t) = K \cdot \mathrm{e}^{\lambda t}
	.\]
	\begin{center}
		\slshape Comment résoudre l'équation différentielle {\scshape avec}\/ 2\tsup{nd} membre si on connaît la solution générale de l'équation {\scshape sans}\/ 2\tsup{nd} membre ?
	\end{center}
	On utilise la méthode le la variation de la constante.
	Soit $x(t) = k(t) \cdot \mathrm{e}^{\lambda t}$. Ainsi, en injectant cette expression de $x$\/ dans l'équation $(**)$, on trouve
	\begin{align*}
		(**) \iff& k'(t) \mathrm{e}^{\lambda t} + k(t) \cdot \lambda \mathrm{e}^{\lambda t} = \lambda k(t) \mathrm{e}^{\lambda t} + b(t)\\
		\iff& k'(t) \mathrm{e}^{\lambda t} = b(t)\\
		\iff& k'(t) = b(t) \cdot \mathrm{e}^{-\lambda t}\\
		\iff& k(t) = \int_{0}^{t} b(u)\cdot \mathrm{e}^{-\lambda u}~\mathrm{d}u + K\\
		\iff& x(t) = \left( \int_{0}^{t} b(u) \cdot \mathrm{e}^{-\lambda u}~\mathrm{d}u + K \right) \mathrm{e}^{\lambda t}\\
		\iff& x(t) = \underbrace{\int_{0}^{t} b(u) \cdot \mathrm{e}^{\lambda (t-u)}~\mathrm{d}u}_{\text{solution particulière}} + \underbrace{K \cdot \mathrm{e}^{\lambda t}}_{\substack{\text{solution}\\\text{générale}\\\text{de $(*)$}}}.
	\end{align*}
\end{rmkn}

		\begin{exm}
	On pose $f$, le sinus cardinal :  \begin{align*}
		f: \R^* &\longrightarrow \R \\
		t &\longmapsto \frac{\sin t}{t}.
	\end{align*}
	\begin{figure}[H]
		\centering
		\begin{asy}
			import graph;
			size(10cm);
			draw((-10, 0) -- (10, 0), Arrow(TeXHead));
			draw((0, -3) -- (0, 5), Arrow(TeXHead));
			real f(real x) {
				if(x == 0) { return 3; }
				else {return 3*sin(x) / x;}
			}
			draw(graph(f, -10, 10), magenta);
		\end{asy}
		\caption{Sinus cardinal}
	\end{figure}

	La fonction $f$\/ est continue sur ${]0,8]}$\/ mais $\lim_{t\to 0} \frac{\sin t}{t} = 1$. D'où $\int_{0}^{8} \frac{\sin t}{t}~\mathrm{d}t$\/ est faussement impropre en $0$\/ et donc convergente.


	Mais attention ! On ne dit pas \guillemotleft~{\color{red}soit $f : t \mapsto \frac{1}{t}$. L'intégrale $\int_{8}^{+\infty} \frac{1}{t}~\mathrm{d}t$\/ est faussement impropre en $+\infty$\/ car $\lim_{t\to +\infty}\frac{1}{t} = 0$}.~\guillemotright
\end{exm}

\section{Intégrer les $\mathbf{\sim}$, $\po$, et \textit{O}}

\begin{thm}
	\hfill$\O$\hfill\null
\end{thm}

\begin{thm}
	Le 2.\ n'est pas la réciproque du 1.\ mais la contraposée.
\end{thm}

\begin{prop}
	\hfill$\O$\hfill\null
\end{prop}

\begin{exm}
	On considère l'intégrale $\int_{2}^{+\infty} \frac{1}{t^2+ \cos t}~\mathrm{d}t$, c'est une intégrale impropre en $+\infty$.
	On recherche un équivalent de $\frac{1}{t^2 + \cos t}$\/ en $+\infty$ : \[
		\frac{1}{t^2 + \cos t} \simi_{t\to +\infty} \frac{1}{t^2}
	\] qui ne change pas de signe. Or, $\int_{2}^{+\infty} \frac{1}{t^2}~\mathrm{d}t$\/ converge car c'est une intégrale de {\sc Riemann}\/ avec $\alpha = 2 > 1$.
	On en déduit que l'intégrale $I$\/ converge.

	On procède autrement : \[
		0 \le \frac{1}{t^2 + \cos t} \le \frac{1}{t^2 - 1}
	.\] Or, $\int_{2}^{+\infty} \frac{1}{t^2 - 1}~\mathrm{d}t$\/ converge car
	\begin{align*}
		\int_{2}^{x} \frac{1}{t^2 - 1}~\mathrm{d}t &= \int_{2}^{x} \left( \frac{\sfrac12}{t-1} - \frac{\sfrac12}{t+1} \right) ~\mathrm{d}t \\
		&= \frac{1}{2} \int_{2}^{x} \frac{1}{t-1}~\mathrm{d}t - \frac{1}{2}\int_{2}^{x} \frac{1}{t+1}~\mathrm{d}t \\
		&= \frac{1}{2} \Big[\ln|t-1|\Big]_2^x - \frac{1}{2}\Big[\ln |t+1|\Big]_2^x \\
	\end{align*}
	D'où \[
		\int_{2}^{x} \frac{1}{t^2 - 1}~\mathrm{d}t = \frac{1}{2} \left[ \ln\left| \frac{t-1}{t+1} \right| \right]_2^x = \frac{1}{2}\ln \left| \frac{x-1}{x+1} \right| + \frac{1}{2} \ln 3 \tendsto{x\to +\infty} \frac{1}{2} \ln 3
	.\] donc l'intégrale $I$\/ converge et $I \le \frac{1}{2} \ln_3$.
\end{exm}

\begin{exo}
	\begin{enumerate}
		\item L'intégrale $I = \int_{0}^{1} \frac{\sin t}{t^2}~\mathrm{d}t$\/ est impropre en 0. On utilise un équivalent : $\sin t \simi_{t\to 0} t$\/ qui ne change pas de signe. Or, $\int_{0}^{t} \frac{1}{t}~\mathrm{d}t$\/ diverge (par critère de {\sc Riemann}). Donc $I$\/ diverge.
			
			L'intégrale $J = \int_{1}^{+\infty} \sin \frac{1}{t}~\mathrm{d}t$\/ est généralisée en $+\infty$. On cherche un équivalent en $+\infty$\/ : \[
				\sin \frac{1}{t} \simi_{t\to +\infty} \frac{1}{t}
			\] qui ne change pas de signe. Or, $\int_{1}^{+\infty} \frac{1}{t}~\mathrm{d}t$\/ diverge par critère de {\sc Riemann}. On en déduit que $J$\/ diverge également.
		\item L'intégrale $\int_{0}^{+\infty} \frac{1}{t^2}~\mathrm{d}t$\/ est impropre, {\bf et}\/ en 0, {\bf et}\/ en $+\infty$. Le théorème ne marche donc pas.
			En effet $t\mapsto \frac{1}{t^2}$\/ n'est pas continue par morceaux en 0, ce qui était le cas pour $t\mapsto \frac{1}{1+t^2}$.
	\end{enumerate}
\end{exo}

\begin{rmkn}[Retour sur la {\sc remarque}\/ 5]
	L'intégrale $\int_{0}^{+\infty} \frac{1}{\ln(1+t)}~\mathrm{d}t$\/ est impropre en 0 {\bf et}\/ en $+\infty$. $\int_{0}^{+\infty} \frac{1}{\ln(1+t)}~\mathrm{d}t$\/ converge si et seulement si $\int_{0}^{7} \frac{1}{\ln(1+t)}~\mathrm{d}t$\/ {\bf et}\/ $\int_{7}^{+\infty} \frac{1}{\ln(1+t)}~\mathrm{d}t$\/ convergent.
	Et si elles convergent \[
		\int_{0}^{+\infty} \frac{1}{\ln(1+t)}~\mathrm{d}t = \int_{0}^{7} \frac{1}{\ln(1+t)}~\mathrm{d}t + \int_{7}^{+\infty} \frac{1}{\ln(1+t)}~\mathrm{d}t
	.\]
	On n'utilise pas deux barrières en même temps. Sinon, les intégrales doublement impropres peuvent, et converger, et diverger.
\end{rmkn}

\begin{prop}[avec $\sim$]
	Si $f(t) \simi_{t\to b} g(t)$\/ qui ne change pas de signe. Alors,
	\begin{itemize}
		\item ou bien $\ds\int_{a}^{b} f(t)~\mathrm{d}t$\/ et $\ds\int_{a}^{b} g(t)~\mathrm{d}t$\/ convergent et $\ds \int_{x}^{b} f(t)~\mathrm{d}t \simi_{x\to b} \int_{x}^{b} g(t)~\mathrm{d}t$.
		\item ou bien $\ds\int_{a}^{b} f(t)~\mathrm{d}t$\/ et $\ds\int_{a}^{b} g(t)~\mathrm{d}t$\/ divergent et $\ds\int_{a}^{x} f(t)~\mathrm{d}t \simi_{x\to b} \int_{a}^{x} g(t)~\mathrm{d}t$.
	\end{itemize}
	Cette proposition est équivalente à le {\sc lemme}\/ 13 sur les séries.
\end{prop}



		\begin{prop}
	La relation $\preceq$\/ est un \textit{pré-ordre} :
	\begin{itemize}
		\item $\preceq $\/ est réflective ;
		\item $\preceq $\/ est transitive.
	\end{itemize}
\end{prop}

\begin{prv}
	Soit $Q$\/ un problème de décision.
	\begin{itemize}
		\item $Q \preceq Q$\/ par la fonction identité, qui est totale et calculable.
		\item Soient $Q$, $R$\/ et $S$\/ trois problèmes de décision tels que $Q \preceq R$\/ et $R \preceq S$. Soit donc $f_1$\/ la réduction de $Q$\/ à $R$, et $f_2$\/ la réduction de $R$\/ à $S$. Soit $f = f_2 \circ f_1 : \mathcal{E}_Q \to \mathcal{E}_S$. La fonction $f$\/ est totale comme composée de fonctions totales, $f$\/ est calculable comme composée de fonctions calculables. De plus,
			\begin{align*}
				\forall e \in \mathcal{E}_Q,\qquad f(e) \in S^+ \iff& f_2(f_1(e)) \in S^+\\
				\iff& f_1(e) \in R^+\\
				\iff& e \in Q^+
			\end{align*}
	\end{itemize}
\end{prv}

\section{Classe \textbf{P} et \textbf{NP}}

Pour répondre à un problème, on peut le résoudre par des algorithmes plus ou moins rapides. Mais, l'objectif de cette section est de montrer que certains problèmes ne peuvent se résoudre que par des algorithmes lents, et que l'on ne peut pas faire mieux.

\begin{defn}
	Le modèle de calcul impose une représentation des entrées par chaînes de caractères. Cela induit donc une notion de \textit{taille d'entrée}, qui est la longueur de la chaîne de caractères.
	\index{taille d'entrée}
\end{defn}


\subsection{Complexité d'une machine}

\begin{defn}
	Étant donné une machine $\mathcal{M}$ et une entrée $w \in \Sigma^*$, on note $C^\mathcal{M}(w)$\/ le nombre d'opérations élémentaires effectuées lors de l'appel de $\mathcal{M}$\/ sur $w$. Lorsque $\smash{w \xrightarrow[\mathcal{M}]{} {\circlearrowleft}}$, on définit $C^\mathcal{M} = +\infty$.

	Pour $n \in \N$, on définit alors \[
		C^\mathcal{M}_n = \max \{ C^\mathcal{M}(w)  \mid w \in \Sigma^n \}
	.\]
	\index{machine!nombre d'opérations élémentaires!($C^\mathcal{M}(w)$)}
	\index{machine!nombre d'opérations élémentaires!maximal pour un mot de taille $n$\/ ($C^\mathcal{M}_n$)}
\end{defn}

\begin{rmk}
	On a, $\forall n \in \N$, $C_n^\mathcal{M} \in \bar{\N} = \N \cup \{+\infty\}$.
\end{rmk}

\begin{defn}
	Soit $f : \N\to \N$\/ une fonction totale et calculable. On note $\textsc{Time}(f)$\/ l'ensemble des machines $\mathcal{M}$\/ telles que
	\begin{itemize}
		\item $\mathcal{M}$\/ s'arrête sur toute entrée ;
		\item $\big(C_n^\mathcal{M}\big)_{n \in \N} = \mathcal{O}\big(\big(f(n)\big)_{n \in \N}\big)$.
	\end{itemize}
	\index{machine!ensemble $\textsc{Time}(f)$}
\end{defn}

\subsection{Classe \textbf{P}}

\begin{defn}
	On dit d'une machine $\mathcal{M}$\/ qu'elle est de \textit{complexité polynômiale} dès lors qu'il existe $k \in \N$\/ tel que $\mathcal{M} \in \textsc{Time}(n^k)$.
	\index{machine!de complexité polynômiale}
\end{defn}

\begin{defn}
	On dit d'une fonction (partielle ou non), qu'elle est \textit{calculable en temps polynômial} dès lors qu'il existe une machine $\mathcal{M}$\/ de complexité polynômiale la calculant.
	\index{fonction!calculable!en temps polynômial}
\end{defn}

\begin{exm}
	\begin{itemize}
		\item l'identité ($n \mapsto n$)
		\item la fonction successeur ($n\mapsto n+1$)
	\end{itemize}
\end{exm}


	}
	\def\addmacros#1{#1}
}

{
	\chap[9]{Grammaires non contextuelles}
	\minitoc
	\renewcommand{\cwd}{../cours/chap09/}
	\addmacros{
		\section{(Ne pas) être diagonalisable}

\begin{defn}
	Soit une matrice carrée $A$. On dit que $A$\/ est {\it diagonalisable}\/ s'il existe une matrice inversible~$P \in \mathrm{GL}_n(\mathds{K})$\/ telle que $P^{-1}\cdot A\cdot P$\/ est diagonale.
\end{defn}

\begin{exo}
	\begin{enumerate}
		\item Montrons que la matrice $B = {7\: 1\choose 0\:7}$\/ n'est pas diagonalisable.
			Par l'absurde : on suppose qu'il existe $P \in \mathrm{GL}_2(\R)$\/ et $(\lambda_1, \lambda_2) \in \R^2$\/ tels que \[
				P^{-1} \cdot B \cdot P = \begin{bmatrix}
					\lambda_1 & 0\\
					0&\lambda_2
				\end{bmatrix}
			.\] On applique la trace $\tr$\/ et le déterminant $\det$\/ :
			\begin{gather*}
				\tr(B) = \tr{\lambda_1\:0\choose 0\:\lambda_2} \quad\text{d'où}\quad \lambda_1 + \lambda_2 = 7 + 7 = 14 = \s\\
				\det(B) = \det{\lambda_1\:0\choose 0\:\lambda_2} \quad\text{d'où}\quad \lambda_1 \times \lambda_2 = 7 \times 7 = 49 = p
			\end{gather*}
			D'où $\lambda_1$\/ et $\lambda_2$\/ sont des solutions de l'équation $X^2 - \s X + p = 0$. Or
			\begin{align*}
				X^2 - \s X + p = 0 \iff& X^2 - 14X + 49 = 0\\
				\iff& (X-7)^2 = 0\\
				\iff& X = 7.
			\end{align*}
			D'où 
			\begin{align*}
				B = P P^{-1} B P P^{-1} = P \begin{pmatrix}
					7&0\\
					0&7
				\end{pmatrix} P^{-1} = P \cdot 7I_2\cdot P^{-1} = 7I_2.
			\end{align*}
			La matrice $B$\/ n'est donc pas diagonalisable.

			De même, montrons que la matrice $A$\/ n'est pas diagonalisable. On remarque que \[
				A \cdot \mat{1\\1\\1} = \begin{pmatrix}
					0&1&2\\
					1&0&2\\
					0&0&3
				\end{pmatrix} \begin{pmatrix}
					1\\1\\1
				\end{pmatrix} = \begin{pmatrix}
					3\\3\\3
				\end{pmatrix} = 3\begin{pmatrix}
					1\\1\\1
				\end{pmatrix} 
			.\] Ainsi, \[
				P^{-1}\cdot A\cdot P = \begin{pmatrix}
					3&0&0\\
					0&?&0\\
					0&0&?
				\end{pmatrix}\qquad\text{où}\qquad P = \begin{pmatrix}
					1&?&?\\
					1&?&?\\
					1&?&?
				\end{pmatrix}
			.\] De même, $A\left( \substack{1\\1\\0} \right) = 1 \times \left( \substack{1\\1\\0} \right)$. D'où \[
				P^{-1}\cdot A\cdot P = \begin{pmatrix}
					3&0&0\\
					0&1&0\\
					0&0&?
				\end{pmatrix}\qquad\text{où}\qquad P = \begin{pmatrix}
					1&1&?\\
					1&1&?\\
					1&0&?
				\end{pmatrix}
			.\] Finalement, on en conclut que \[
				P = \begin{pmatrix}
					3&0&0\\
					0&1&0\\
					0&0&-1
				\end{pmatrix} \qquad \text{et}\qquad P^{-1}\cdot A\cdot P = \begin{pmatrix}
					1&1&1\\
					1&1&-1\\
					1&0&0
				\end{pmatrix} = D
			.\]
			De plus, la matrice $P$\/ est inversible car $\det P \neq 0$.
		\item Pour calculer $A^n$, on pourrait chercher un polynôme annulateur $Q$\/ de $A$, et on exprime $X^n = Q \times T_n + R_n$, et donc $A^n = R_n(A)$.
			Mais, on peut également diagonaliser $A$\/ (si elle est diagonalisable).
			Ainsi,  \[
				D^n = (P^{-1}\cdot A\cdot P)^n = P^{-1}\cdot A\cdot \cancel P\cdot \cancel{P^{-1}} \cdot \ldots\cdot \cancel{P^{-1}} \cdot A \cdot P = P^{-1}\cdot  A^n\cdot P
			.\] D'où $A^n = P \cdot D^n \cdot P^{-1}$. Or, \[
				D^n = \begin{pmatrix}
					3&0&0\\
					0&1&0\\
					0&0&-1
				\end{pmatrix}^n = \begin{pmatrix}
					3^n&0&0\\
					0&1^n&0\\
					0&0&(-1)^n
				\end{pmatrix}
			.\]
			On calcule donc $A^{n}$\/ en calculant l'inverse de $P$\/ : \[
				A^n = \begin{pmatrix}
					1&1&1\\
					1&1&-1\\
					1&0&0
				\end{pmatrix} \begin{pmatrix}
					3^n&0&0\\
					0&1^n&0\\
					0&0&(-1)^n
				\end{pmatrix} \cdot P^{-1}
			.\]
		\item
			\begin{align*}
				\begin{rcases*}
					\hfill u_{n+1} = v_n + 2w_n\\
					\hfill v_{n+1} = u_n + 2w_n\\
					\hfill w_{n+1} = 3w_n
				\end{rcases*} \iff& \begin{pmatrix}
					u_{n+1}\\v_{n+1}\\w_{n+1}
				\end{pmatrix} = \begin{pmatrix}
					0&1&2\\
					1&0&2\\
					0&0&3
				\end{pmatrix} \begin{pmatrix}
					u_n\\ v_n\\ w_n
				\end{pmatrix}\\
				\iff& U_{n+1} = A\cdot U_n\\
				\iff& U'_{n+1} = D \cdot U'_{n}
			\end{align*}
			où $D = P^{-1} \cdot A \cdot P$, $U'_{n+1} = P\cdot U_{n+1}$\/ et $U'_n = P\cdot U_n$.
			\begin{align*}
				\phantom{\begin{rcases*}
					\hfill mm_{n+1} = v_n + 2w_n\\
					\hfill v_{n+1} = u_n + 2w_n\\
					\hfill w_{n+1} = 3w_n
				\end{rcases*}} \iff&
				\begin{pmatrix}
					u'_{n+1}\\v'_{n+1}\\w'_{n+1}
				\end{pmatrix} = \begin{pmatrix}
					3&0&0\\
					0&1&0\\
					0&0&-1
				\end{pmatrix} \cdot \begin{pmatrix}
					u'_n\\
					v'_n\\
					w'_n
				\end{pmatrix}\\
				\iff& \begin{cases}
					u'_{n+1} = 3u'_n\\
					v'_{n+1} = v'_n\\
					w'_{n+1} = -w'_n
				\end{cases}\\
				\iff& \begin{cases}
					u'_n = K\times  3^n\\
					v'_n = L\\
					w'_n = M \times (-1)^n
				\end{cases}
			\end{align*}
			Ainsi, \[
				\begin{pmatrix}
					u_n\\v_n\\w_n
				\end{pmatrix} = \underbrace{\begin{pmatrix}
					1&1&1\\
					1&1&-1\\
					1&0&0
				\end{pmatrix}}_P \cdot \begin{pmatrix}
					K\times 3^n\\
					L\\
					M\times (-1)^n
				\end{pmatrix}
			.\] D'où $u_n = K\cdot 3^n + L + M \cdot (-1)^n$, $v_n = K\times 3^n + L - M \cdot (-1)^n$\/ et $w_n = K\cdot 3^n$, où les constantes $K$, $L$\/ et $M$\/ sont des constantes fixées par les conditions initiales.
		\item
			\begin{align*}
				\begin{rcases*}
					\hfill x'(t) = y(t) + 2z(t)\\
					\hfill y'(t) = x(t) + 2z(t)\\
					\hfill z'(t) = 3z(t)
				\end{rcases*} \iff& \begin{pmatrix}
					x'(t)\\
					y'(t)\\
					z'(t)
				\end{pmatrix} = \begin{pmatrix}
					0&1&2\\
					1&0&2\\
					0&0&3
				\end{pmatrix} \cdot \begin{pmatrix}
					x(t)\\
					y(t)\\
					z(t)
				\end{pmatrix}\\
				\iff& X'(t) = A\cdot X(t)\\
				\iff& U'(t) = D \cdot U(t) \text{ avec } D = P^{-1} \cdot A\cdot P \text{ et } X(t) = P\cdot U(t)\\
				\iff& \begin{pmatrix}
					u'(t)\\
					v'(t)\\
					w'(t)
				\end{pmatrix} = \begin{pmatrix}
					3&0&0\\
					0&1&0\\
					0&0&-1
				\end{pmatrix} \cdot \begin{pmatrix}
					u(t)\\
					v(t)\\
					w(t)
				\end{pmatrix}\\
				\iff& \begin{cases}
					u'(t) = 3u(t)\\
					v'(t) = v(t)\\
					w'(t) = -w(t)
				\end{cases}\\
				\iff& \begin{cases}
					u(t) = K \cdot \mathrm{e}^{3t}\\
					v(t) = L \cdot \mathrm{e}^{t}\\
					w(t) = M \cdot \mathrm{e}^{-t}
				\end{cases}
			\end{align*}
			Ainsi \[
				\begin{pmatrix}
					x(t)\\
					y(t)\\
					z(t)
				\end{pmatrix} = \underbrace{\begin{pmatrix}
					1&1&1\\
					1&1&-1\\
					1&0&0
				\end{pmatrix}}_P \cdot \begin{pmatrix}
					K \times \mathrm{e}^{3t}\\
					L \cdot \mathrm{e}^{t}\\
					M \cdot \mathrm{e}^{-t}
				\end{pmatrix}
			.\] 
			D'où $x(t) = K\cdot \mathrm{e}^{3t} + L \cdot \mathrm{e}^{t} + M \cdot \mathrm{e}^{-t}$, $y(t) = K \cdot \mathrm{e}^{3t} + L \cdot \mathrm{e}^{t} - M \cdot \mathrm{e}^{-t}$\/ et $z(t) = K\cdot \mathrm{e}^{3t}$. Les constantes $K$, $L$\/ et $M$\/ peuvent être déterminées à partir des conditions initiales.
	\end{enumerate}
\end{exo}

\begin{rmkn}[équations différentielles]
	On considère l'équation différentielle $(*)$ : $x'(t) = \lambda \cdot x(t)$.
	Les fonctions $x : t \mapsto K\cdot \mathrm{e}^{\lambda t}$\/ sont des solutions de cette équation. On peut utiliser la méthode de {\sc Lagrange}\/ : la méthode de la~\guillemotleft~variation de la constante.~\guillemotright\@ On cherche des solutions sous la forme $x(t) = k(t) \cdot \mathrm{e}^{\lambda t}$ (vision du~physicien). D'où $k(t) = x(t) / \mathrm{e}^{\lambda t}$\/ (vision du mathématicien). De plus, $x'(t) = k'(t) \mathrm{e}^{\lambda t} + k(t) \lambda \mathrm{e}^{\lambda t}$.
	Ainsi, on injecte ce $k(t)$\/ dans l'équation différentielle :
	\begin{align*}
		(*) \iff& k'(t) \mathrm{e}^{\lambda t} + k(t) \lambda \mathrm{e}^{\lambda t} = \lambda k(t)\mathrm{e}^{\lambda t}\\
		\iff& k'(t) \mathrm{e}^{\lambda t} = 0\\
		\iff& k'(t) = 0\\
		\iff& \exists K \in \R\,\:k(t) = K.
	\end{align*}
	Les solutions trouvées dans l'exercice précédent sont donc les uniques solutions du système d'équations différentielles.

	De même, pour résoudre une équation différentielle avec 2\tsup{nd} membre de la forme \[
		(**) : \qquad x'(t) - \lambda \cdot x(t) = b(t)
	.\]
	La fonction $t \mapsto x(t)$\/ est une solution de l'équation {\sc sans}\/ 2\tsup{nd} membre si et seulement si \[
		\exists K \in \R,\:\forall t \in \R,\quad x(t) = K \cdot \mathrm{e}^{\lambda t}
	.\]
	\begin{center}
		\slshape Comment résoudre l'équation différentielle {\scshape avec}\/ 2\tsup{nd} membre si on connaît la solution générale de l'équation {\scshape sans}\/ 2\tsup{nd} membre ?
	\end{center}
	On utilise la méthode le la variation de la constante.
	Soit $x(t) = k(t) \cdot \mathrm{e}^{\lambda t}$. Ainsi, en injectant cette expression de $x$\/ dans l'équation $(**)$, on trouve
	\begin{align*}
		(**) \iff& k'(t) \mathrm{e}^{\lambda t} + k(t) \cdot \lambda \mathrm{e}^{\lambda t} = \lambda k(t) \mathrm{e}^{\lambda t} + b(t)\\
		\iff& k'(t) \mathrm{e}^{\lambda t} = b(t)\\
		\iff& k'(t) = b(t) \cdot \mathrm{e}^{-\lambda t}\\
		\iff& k(t) = \int_{0}^{t} b(u)\cdot \mathrm{e}^{-\lambda u}~\mathrm{d}u + K\\
		\iff& x(t) = \left( \int_{0}^{t} b(u) \cdot \mathrm{e}^{-\lambda u}~\mathrm{d}u + K \right) \mathrm{e}^{\lambda t}\\
		\iff& x(t) = \underbrace{\int_{0}^{t} b(u) \cdot \mathrm{e}^{\lambda (t-u)}~\mathrm{d}u}_{\text{solution particulière}} + \underbrace{K \cdot \mathrm{e}^{\lambda t}}_{\substack{\text{solution}\\\text{générale}\\\text{de $(*)$}}}.
	\end{align*}
\end{rmkn}

		\begin{exm}
	On pose $f$, le sinus cardinal :  \begin{align*}
		f: \R^* &\longrightarrow \R \\
		t &\longmapsto \frac{\sin t}{t}.
	\end{align*}
	\begin{figure}[H]
		\centering
		\begin{asy}
			import graph;
			size(10cm);
			draw((-10, 0) -- (10, 0), Arrow(TeXHead));
			draw((0, -3) -- (0, 5), Arrow(TeXHead));
			real f(real x) {
				if(x == 0) { return 3; }
				else {return 3*sin(x) / x;}
			}
			draw(graph(f, -10, 10), magenta);
		\end{asy}
		\caption{Sinus cardinal}
	\end{figure}

	La fonction $f$\/ est continue sur ${]0,8]}$\/ mais $\lim_{t\to 0} \frac{\sin t}{t} = 1$. D'où $\int_{0}^{8} \frac{\sin t}{t}~\mathrm{d}t$\/ est faussement impropre en $0$\/ et donc convergente.


	Mais attention ! On ne dit pas \guillemotleft~{\color{red}soit $f : t \mapsto \frac{1}{t}$. L'intégrale $\int_{8}^{+\infty} \frac{1}{t}~\mathrm{d}t$\/ est faussement impropre en $+\infty$\/ car $\lim_{t\to +\infty}\frac{1}{t} = 0$}.~\guillemotright
\end{exm}

\section{Intégrer les $\mathbf{\sim}$, $\po$, et \textit{O}}

\begin{thm}
	\hfill$\O$\hfill\null
\end{thm}

\begin{thm}
	Le 2.\ n'est pas la réciproque du 1.\ mais la contraposée.
\end{thm}

\begin{prop}
	\hfill$\O$\hfill\null
\end{prop}

\begin{exm}
	On considère l'intégrale $\int_{2}^{+\infty} \frac{1}{t^2+ \cos t}~\mathrm{d}t$, c'est une intégrale impropre en $+\infty$.
	On recherche un équivalent de $\frac{1}{t^2 + \cos t}$\/ en $+\infty$ : \[
		\frac{1}{t^2 + \cos t} \simi_{t\to +\infty} \frac{1}{t^2}
	\] qui ne change pas de signe. Or, $\int_{2}^{+\infty} \frac{1}{t^2}~\mathrm{d}t$\/ converge car c'est une intégrale de {\sc Riemann}\/ avec $\alpha = 2 > 1$.
	On en déduit que l'intégrale $I$\/ converge.

	On procède autrement : \[
		0 \le \frac{1}{t^2 + \cos t} \le \frac{1}{t^2 - 1}
	.\] Or, $\int_{2}^{+\infty} \frac{1}{t^2 - 1}~\mathrm{d}t$\/ converge car
	\begin{align*}
		\int_{2}^{x} \frac{1}{t^2 - 1}~\mathrm{d}t &= \int_{2}^{x} \left( \frac{\sfrac12}{t-1} - \frac{\sfrac12}{t+1} \right) ~\mathrm{d}t \\
		&= \frac{1}{2} \int_{2}^{x} \frac{1}{t-1}~\mathrm{d}t - \frac{1}{2}\int_{2}^{x} \frac{1}{t+1}~\mathrm{d}t \\
		&= \frac{1}{2} \Big[\ln|t-1|\Big]_2^x - \frac{1}{2}\Big[\ln |t+1|\Big]_2^x \\
	\end{align*}
	D'où \[
		\int_{2}^{x} \frac{1}{t^2 - 1}~\mathrm{d}t = \frac{1}{2} \left[ \ln\left| \frac{t-1}{t+1} \right| \right]_2^x = \frac{1}{2}\ln \left| \frac{x-1}{x+1} \right| + \frac{1}{2} \ln 3 \tendsto{x\to +\infty} \frac{1}{2} \ln 3
	.\] donc l'intégrale $I$\/ converge et $I \le \frac{1}{2} \ln_3$.
\end{exm}

\begin{exo}
	\begin{enumerate}
		\item L'intégrale $I = \int_{0}^{1} \frac{\sin t}{t^2}~\mathrm{d}t$\/ est impropre en 0. On utilise un équivalent : $\sin t \simi_{t\to 0} t$\/ qui ne change pas de signe. Or, $\int_{0}^{t} \frac{1}{t}~\mathrm{d}t$\/ diverge (par critère de {\sc Riemann}). Donc $I$\/ diverge.
			
			L'intégrale $J = \int_{1}^{+\infty} \sin \frac{1}{t}~\mathrm{d}t$\/ est généralisée en $+\infty$. On cherche un équivalent en $+\infty$\/ : \[
				\sin \frac{1}{t} \simi_{t\to +\infty} \frac{1}{t}
			\] qui ne change pas de signe. Or, $\int_{1}^{+\infty} \frac{1}{t}~\mathrm{d}t$\/ diverge par critère de {\sc Riemann}. On en déduit que $J$\/ diverge également.
		\item L'intégrale $\int_{0}^{+\infty} \frac{1}{t^2}~\mathrm{d}t$\/ est impropre, {\bf et}\/ en 0, {\bf et}\/ en $+\infty$. Le théorème ne marche donc pas.
			En effet $t\mapsto \frac{1}{t^2}$\/ n'est pas continue par morceaux en 0, ce qui était le cas pour $t\mapsto \frac{1}{1+t^2}$.
	\end{enumerate}
\end{exo}

\begin{rmkn}[Retour sur la {\sc remarque}\/ 5]
	L'intégrale $\int_{0}^{+\infty} \frac{1}{\ln(1+t)}~\mathrm{d}t$\/ est impropre en 0 {\bf et}\/ en $+\infty$. $\int_{0}^{+\infty} \frac{1}{\ln(1+t)}~\mathrm{d}t$\/ converge si et seulement si $\int_{0}^{7} \frac{1}{\ln(1+t)}~\mathrm{d}t$\/ {\bf et}\/ $\int_{7}^{+\infty} \frac{1}{\ln(1+t)}~\mathrm{d}t$\/ convergent.
	Et si elles convergent \[
		\int_{0}^{+\infty} \frac{1}{\ln(1+t)}~\mathrm{d}t = \int_{0}^{7} \frac{1}{\ln(1+t)}~\mathrm{d}t + \int_{7}^{+\infty} \frac{1}{\ln(1+t)}~\mathrm{d}t
	.\]
	On n'utilise pas deux barrières en même temps. Sinon, les intégrales doublement impropres peuvent, et converger, et diverger.
\end{rmkn}

\begin{prop}[avec $\sim$]
	Si $f(t) \simi_{t\to b} g(t)$\/ qui ne change pas de signe. Alors,
	\begin{itemize}
		\item ou bien $\ds\int_{a}^{b} f(t)~\mathrm{d}t$\/ et $\ds\int_{a}^{b} g(t)~\mathrm{d}t$\/ convergent et $\ds \int_{x}^{b} f(t)~\mathrm{d}t \simi_{x\to b} \int_{x}^{b} g(t)~\mathrm{d}t$.
		\item ou bien $\ds\int_{a}^{b} f(t)~\mathrm{d}t$\/ et $\ds\int_{a}^{b} g(t)~\mathrm{d}t$\/ divergent et $\ds\int_{a}^{x} f(t)~\mathrm{d}t \simi_{x\to b} \int_{a}^{x} g(t)~\mathrm{d}t$.
	\end{itemize}
	Cette proposition est équivalente à le {\sc lemme}\/ 13 sur les séries.
\end{prop}



		\begin{prop}
	La relation $\preceq$\/ est un \textit{pré-ordre} :
	\begin{itemize}
		\item $\preceq $\/ est réflective ;
		\item $\preceq $\/ est transitive.
	\end{itemize}
\end{prop}

\begin{prv}
	Soit $Q$\/ un problème de décision.
	\begin{itemize}
		\item $Q \preceq Q$\/ par la fonction identité, qui est totale et calculable.
		\item Soient $Q$, $R$\/ et $S$\/ trois problèmes de décision tels que $Q \preceq R$\/ et $R \preceq S$. Soit donc $f_1$\/ la réduction de $Q$\/ à $R$, et $f_2$\/ la réduction de $R$\/ à $S$. Soit $f = f_2 \circ f_1 : \mathcal{E}_Q \to \mathcal{E}_S$. La fonction $f$\/ est totale comme composée de fonctions totales, $f$\/ est calculable comme composée de fonctions calculables. De plus,
			\begin{align*}
				\forall e \in \mathcal{E}_Q,\qquad f(e) \in S^+ \iff& f_2(f_1(e)) \in S^+\\
				\iff& f_1(e) \in R^+\\
				\iff& e \in Q^+
			\end{align*}
	\end{itemize}
\end{prv}

\section{Classe \textbf{P} et \textbf{NP}}

Pour répondre à un problème, on peut le résoudre par des algorithmes plus ou moins rapides. Mais, l'objectif de cette section est de montrer que certains problèmes ne peuvent se résoudre que par des algorithmes lents, et que l'on ne peut pas faire mieux.

\begin{defn}
	Le modèle de calcul impose une représentation des entrées par chaînes de caractères. Cela induit donc une notion de \textit{taille d'entrée}, qui est la longueur de la chaîne de caractères.
	\index{taille d'entrée}
\end{defn}


\subsection{Complexité d'une machine}

\begin{defn}
	Étant donné une machine $\mathcal{M}$ et une entrée $w \in \Sigma^*$, on note $C^\mathcal{M}(w)$\/ le nombre d'opérations élémentaires effectuées lors de l'appel de $\mathcal{M}$\/ sur $w$. Lorsque $\smash{w \xrightarrow[\mathcal{M}]{} {\circlearrowleft}}$, on définit $C^\mathcal{M} = +\infty$.

	Pour $n \in \N$, on définit alors \[
		C^\mathcal{M}_n = \max \{ C^\mathcal{M}(w)  \mid w \in \Sigma^n \}
	.\]
	\index{machine!nombre d'opérations élémentaires!($C^\mathcal{M}(w)$)}
	\index{machine!nombre d'opérations élémentaires!maximal pour un mot de taille $n$\/ ($C^\mathcal{M}_n$)}
\end{defn}

\begin{rmk}
	On a, $\forall n \in \N$, $C_n^\mathcal{M} \in \bar{\N} = \N \cup \{+\infty\}$.
\end{rmk}

\begin{defn}
	Soit $f : \N\to \N$\/ une fonction totale et calculable. On note $\textsc{Time}(f)$\/ l'ensemble des machines $\mathcal{M}$\/ telles que
	\begin{itemize}
		\item $\mathcal{M}$\/ s'arrête sur toute entrée ;
		\item $\big(C_n^\mathcal{M}\big)_{n \in \N} = \mathcal{O}\big(\big(f(n)\big)_{n \in \N}\big)$.
	\end{itemize}
	\index{machine!ensemble $\textsc{Time}(f)$}
\end{defn}

\subsection{Classe \textbf{P}}

\begin{defn}
	On dit d'une machine $\mathcal{M}$\/ qu'elle est de \textit{complexité polynômiale} dès lors qu'il existe $k \in \N$\/ tel que $\mathcal{M} \in \textsc{Time}(n^k)$.
	\index{machine!de complexité polynômiale}
\end{defn}

\begin{defn}
	On dit d'une fonction (partielle ou non), qu'elle est \textit{calculable en temps polynômial} dès lors qu'il existe une machine $\mathcal{M}$\/ de complexité polynômiale la calculant.
	\index{fonction!calculable!en temps polynômial}
\end{defn}

\begin{exm}
	\begin{itemize}
		\item l'identité ($n \mapsto n$)
		\item la fonction successeur ($n\mapsto n+1$)
	\end{itemize}
\end{exm}


		\section{Arbres couvrants de poids minimum}

\begin{exm}
	On considère le graphe ci-dessous.
	\begin{figure}[H]
		\centering
		\tikzfig{ex-graphe-pondere}
		\caption{Arbre pondéré}
	\end{figure}
	On cherche à \guillemotleft~supprimer~\guillemotright\ des arrêtes de ce graphe afin d'avoir un poids total minimum, tout en conservant la connexité du graphe.
	Une structure assurant cette condition est un arbre.

	Pour résoudre ce problème, on part du graphe vide, et on ajoute les arrêtes les moins coûteuses en premier.
\end{exm}

\begin{defn}[Arbre]
	Soit $G = (S,A)$\/ un graphe non-orienté. On dit que $G$\/ est un \textit{arbre} si $G$\/ est connexe et acyclique.
	\index{arbre}
\end{defn}

\begin{defn}[Arbre couvrant]
	Étant donné un graphe non orienté pondéré par poids positifs $G = (S, A, c)$,\footnotemark\ on dit de $G' = (S', A')$\/ que c'est un \textit{arbre couvrant} de $G$\/ si $S' = S$\/ et $A' \subseteq A$, et $G'$\/ est un arbre.
	\index{arbre!couvrant}
\end{defn}
\footnotetext{on dit que $c$\/ est la fonction de pondération de ce graphe}

\begin{defn}[Arbre couvrant de poids minimum]
	Étant donné un graphe non orienté pondéré $G = (S, A, c)$\/ et un arbre couvrant $T = (S', A')$, on appelle \textit{poids} de l'arbre $T$\/ la valeur $\sum_{a \in A'} c(a)$.
	\index{arbre!couvrant!poids}

	Si $G$\/ est connexe, il admet au moins un arbre couvrant, on peut définir l'\textit{arbre couvrant de poids minimum} (\textit{\textsc{acpm}}).
	\index{arbre!couvrant!de poids minimum}
\end{defn}

On définir alors le problème \[
	\textsc{acpm}\text{\footnotemark}
	\begin{cases}
		\text{\textbf{Entrée}}&: G = (S, A, c) \text{ connexe}\\
		\text{\textbf{Sortie}}&: \text{ le poids de l'arbre couvrant de poids minimum}.
	\end{cases}
\]
\footnotetext{Arbre Couvrant de Poids Minimum}

\begin{algorithm}[H]
	\centering
	\begin{algorithmic}[1]
		\Entree $G = (S, A, c)$\/ un graphe connexe
		\Sortie Un arbre couvrant de poids minimum
		\State $B \gets \O$\/ 
		\State $U \gets \O$\/
		\While{il existe $u$\/ et $v$\/ tels que $u \nsim_B v$}
			\State Soit $\{x,y\} \in A \setminus U$\/ de poids minimal
			\If{$x \sim_B y$}
				\State $U \gets \big\{\!\{x,y\}\!\big\} \cup U$
			\Else
				\State $U \gets \big\{\!\{x,y\}\!\big\} \cup U$
				\State $B \gets \big\{\!\{x,y\}\!\big\} \cup B$
			\EndIf
		\EndWhile
		\State\Return $T = (S,B)$\/
	\end{algorithmic}
	\caption{Algorithme de \textsc{Kruskal}}
\end{algorithm}

\begin{prop}
	L'algorithme de \textsc{Kruskal} est correct.
\end{prop}

\begin{prv}
	\begin{enumerate}
		\item Il existe un arbre couvrant de poids minimum utilisant les arrêtes de $B$ ;
		\item $B \subseteq U \subseteq A$\/ ;
		\item $\forall \{u,v\} \in U$, $u \sim_B v$.
	\end{enumerate}
	Ces trois propriétés sont invariantes.
	\begin{description}
		\item[Initialement] $B = \O = U$, donc \textsc{ok}.
		\item[Propagation] Soient $\ubar{B}$\/ et $\ubar{U}$\/ (resp.\ $\bar{B}, \bar{U}$) les valeurs de $B$\/ et $U$\/ avant (resp.\ après) une itération de boucle. Supposons que $\ubar{B}$\/ et $\ubar{U}$\/ satisfont les propriétés 1, 2 et 3. Montrons que $\bar{B}$\/ et $\bar{U}$\/ les satisfont aussi.
			\begin{enumerate}
				\item[2.] On a $\{x,y\} \in A$\/ et $\ubar{B} \subseteq \ubar{U} \subseteq A$, donc \[
						\bar{B} \subseteq \ubar{B} \cup \{\!\{x,y\}\!\} \subseteq \ubar{U} \cup \{\!\{x,y\}\!\} \subseteq A
					.\]
				\item[3.] Soit $\{u,v\} \in \bar{U}$.
					\begin{itemize}
						\item Si $\{u,v\} \in \ubar{U}$, alors de 3, $u \sim_{\ubar{B}} v$. Or, $\ubar{B} \subseteq \bar{B}$\/ et donc $u \sim_{\bar{B}} v$.
						\item Sinon, $\{ u,v\} = \{x,y\}$, alors $x = u$\/ et $v = y$.
							\begin{itemize}
								\item Sous-cas 1 : $\bar{B} = \ubar{B} \cup \{\!\{x,y\}\!\}$, alors $x \sim_{\bar{B}} y$.
								\item Sous-cas 2 : $\bar{B} = \ubar{B}$, alors par condition du \textbf{si}, $x \sim_{\ubar{B}} y$\/ et donc $x \sim_{\bar{B}} y$.
							\end{itemize}
					\end{itemize}
				\item[1.]
					Soit $\mathcal{T}$\/ un \textsc{acpm} contenant $\ubar{B}$.
					\begin{itemize}
						\item Cas 1 : $\bar{B} = \ubar{B}$, \textsc{ok}
						\item Cas 2 : $\bar{B} = \ubar{B} \cup \{\!\{x,y\}\!\}$.
							\begin{itemize}
								\item Sous-cas 1 : $\{x,y\} \in \mathcal{T}$, alors $\mathcal{T}$\/ est un \textsc{acpm} qui contient $\bar{B}$.
								\item Sous-cas 2 : $\{x,y\}  \not\in \mathcal{T}$, $\mathcal{T}$\/ est un arbre couvrant, donc il contient une chaîne de $x$\/ à $y$\/ : \[
											\{\overset{\substack{x\\[-1mm]\vrt=}}{x_0},x_1\},\{x_1,x_2\},\ldots,\{x_{n-1},\underset{\substack{\vrt=\\y}}{x_n}\}
									.\]
									Or, $\forall i \in \llbracket 1,n-1 \rrbracket$, $x_i \sim_{\ubar{B}} x_{i+1}$. Par transitivité, on a donc $x = x_0 \sim_{\ubar{B}} x_n = y$, ce qui n'est pas le cas.
									Il existe donc $i_0 \in \llbracket 0,n-1 \rrbracket$, tel que $x_{i_0} \nsim_{\ubar{B}} x_{i_0 + 1}$\/ et donc $\{x_{i_0}, x_{i_0 + 1}\} \not\in \ubar{U}$. D'où, d'après 3, on a $\{x_{i_0}, x_{i_0 + 1}\}  \not\in \ubar{B}$
									Considérons alors $\mathcal{T}' = \big(\mathcal{T} \setminus \{\!\{x_{i_0},x_{i_0+1}\}\!\}\big)  \cup \{\!\{x,y\}\!\}$. Montrons que $\mathcal{T}'$\/ est un \textsc{acpm} contenant $B$, en commençant par montrer que c'est un arbre couvrant. L'arbre $\mathcal{T}'$\/ a $n-1$\/ arrêtes (autant que $\mathcal{T}$). Montrons que $\mathcal{T}'$\/ est connexe.
									Soit $(a,b) \in S^2$. $\mathcal{T}$\/ est connexe, soit donc une chaîne \[
										C : \quad a = u_0, u_1, \ldots, u_n = b
									\] de $\mathcal{T}$. Si la chaîne $C$\/ n'utilise pas l'arrête $\{x_{i_0},x_{i_0+1}\}$, alors $C$\/ est une chaîne de $\mathcal{T}'$. Sinon, on pose $\mu$\/ et $\tau$\/ tels que \[
									\underbrace{a,\ldots,x_{i_0}}_{\mu},\underbrace{x_{i_0+1},\ldots,b}_{\tau}
									.\]
									Soit alors la chaîne
									\begin{align*}
										\overbrace{a,\ldots,x_{i_0}}^{\mu},x_{i_0-1},x_{i_0-2},\ldots,x_0 = x,&\\
										\underbrace{b,\ldots,x_{i_0 + 1}}_{\tau},x_{i_0+2},\ldots,x_{n-1},x_n=y&
									\end{align*}
									qui est dans $\mathcal{T}'$.
									Montrons que le poids est minimum. Notons $P(\mathcal{T})$\/ le poids de l'arbre. On a donc \[
										P(\mathcal{T}') = P(\mathcal{T}) + c(\{x,y\}) - c(\{x_{i_0},x_{i_0+1}\})
									.\] Par choix glouton, ($\{x_{i_0}, x_{i_0+1} \not\in \ubar{U}\}$), $c(\{x,y\}) \le c(\{x_{i_0},x_{i_0+1}\})$\/ donc $P(\mathcal{T}') \le P(\mathcal{T})$, et $\mathcal{T}$\/ étant de poids min, $P(\mathcal{T}') = P(\mathcal{T})$\/ et $\mathcal{T}'$\/ est un \textsc{acpm} contenant $\bar{B}$.
							\end{itemize}
					\end{itemize}
			\end{enumerate}
	\end{description}

	Les invariants le sont.
\end{prv}

À la fin, $B$\/ induit un graphe connexe et $B$\/ est contenu dans un \textsc{acpm}, c'en est donc un.


		\paragraph{Une structure pour la gestion des partitions : \textsf{UnionFind}.}

\begin{defn}[Type de données abstrait \textsf{UnionFind}]
	On définit le type de données abstrait \textsf{UnionFind} comme contenant
	\begin{itemize}
		\item un type \texttt{t} de partitions ;
		\item un type \texttt{elem} des éléments manipulés par les partitions ;
		\item $\texttt{initialise\_partition} : \texttt{elem list} \to \texttt{t}$\/ retournant le partitionnement dans lequel chaque élément est seul dans sa classe ;
		\item $\texttt{find} : (\texttt{t} \mathbin{\texttt{*}} \texttt{elem}) \to \texttt{elem}$\/ retournant un représentant de la classe de l'élément. Si deux éléments $x$\/ et $y$\/ sont dans la même classe, dans le partitionnement $p$, alors $\texttt{find}(p,x) = \texttt{find}(p,y)$ ;
		\item $\texttt{union} : (\texttt{t} \mathbin{\texttt{*}} \texttt{elem} \mathbin{\texttt{*}} \texttt{elem}) \to \texttt{t}$\/ retourne le partitionnement dans lequel on a fusionné les classes des arguments.
	\end{itemize}
	\index{type \textsf{UnionFind}}
\end{defn}

\begin{exm}
	On réalise le \textit{pseudo-code} ci-dessous.
	\begin{itemize}
		\item $p \gets \texttt{initialise\_partition}([1, 2, 3, 4, 5])$\/ $\leadsto$ $\{\{1\}, \{2\}, \{3\}, \{4\}, \{5\}\}$
		\item $\texttt{find}(p, 1) = 1$\/ 
		\item $\texttt{union}(p, 1, 3)$\/ $\leadsto$ $\{\{1,3\}, \{2\}, \{4\}, \{5\}\}$
		\item $\texttt{find}(p, 1) = \texttt{find}(p, 3)$
	\end{itemize}
\end{exm}

On implémente ce type abstrait en \textsc{OCaml}.

\begin{rmk}[Niveau zéro -- listes de liste]~
	\begin{lstlisting}[language=caml,caption=Implémentation du type \textsf{UnionFind} en \textsc{OCaml}]
type 'a t = 'a list list

let initialise_partition (l: 'a list): 'a t =
	List.map (fun x -> [ x ] ) l

let rec find (p: 'a t) (x: 'a): 'a =
	match p with
	| classe :: classes ->
			if List.mem x classe then List.hd classe
			else find classes x
	| [] -> raise Not_Found

let est_equiv (p: 'a t) (x: 'a) (y: 'a): bool = 
	(find p x) = (find p y)

let rec extrait_liste (x: 'a) (p: 'a t): 'a list * 'a p =
	match p with
	| classe :: classes ->
			if List.mem x classe then (classe, classes)
			else
				let cl, cls' = extrait_liste x classes in
				(cl, classe :: cls')
	| [] -> raise Not_Found

let union (p: 'a t) (x: 'a) (y: 'a): 'a t =
	if est_equiv p x y then p
	else
		let cx, p' = extrait_liste x p in
		let cy, p'' = extrait_liste y p' in
		(cx @ cy) :: p''
	\end{lstlisting}
\end{rmk}

\begin{rmk}[Niveau un -- tableau de classes]
	Dans la case du tableau, on inscrit le numéro de sa classe.
	Pour \texttt{find}, on prend le premier ayant la même classe.
	Pour \texttt{union}, on re-numérote vers un numéro commun.
	Par exemple, \[
		\begin{array}{|c|c|c|c|c|c|}
			\hline
			0 & 1 & 0 & 0 & 1 & 2\\ \hline
			0 & 1 & 2 & 3 & 4 & 5 \\ \hline
		\end{array}\quad\quad\longleftrightarrow\quad\quad\{\{0,2,3\},\{1,4\},\{5\}\}
	.\]
\end{rmk}

\begin{rmk}[Niveau deux -- tableau de représentants]
	Dans les cases du tableau, on écrit le représentant de la classe de $i$.
	Pour \texttt{find}, on lit la case.
	Pour \texttt{union}, on re-numérote vers un numéro commun.
	Par exemple, \[
		\begin{array}{|c|c|c|c|c|c|}
			\hline
			2 & 4 & 2 & 2 & 4 & 5\\ \hline
			0 & 1 & 2 & 3 & 4 & 5 \\ \hline
		\end{array}\quad\quad\longleftrightarrow\quad\quad\{\{0,2,3\},\{1,4\},\{5\}\}
	.\]
\end{rmk}

\begin{rmk}[Niveau trois -- arbres]
	Pour $\texttt{union}(0, 1)$, on cherche le représentant de 0 (2) puis celui de 1 (4). On fait pointer 4 vers 2.
	Pour la suite de l'implémentation, \textit{c.f.}\ \textsc{dm}$_3$.

	\begin{figure}[H]
		\centering
		\tikzfig{ex-unionfind-arbres}
		\caption{Représentation par des arbres}
	\end{figure}
\end{rmk}

Avec cette nouvelle structure, on peut maintenant revenir sur l'algorithme de \textsc{Kruskal}.

\begin{algorithm}[H]
	\centering
	\begin{algorithmic}[1]
		\Entree Un graphe $G = (S, A, c)$\/ un graphe non orienté, pondéré
		\Sortie Un \textsc{acpm}
		\State Soit $(e_i)_{i\in\llbracket 1,m \rrbracket}$\/ un tri des arrêtes par coût croissant
		\State $f \gets 0$\/ \Comment{Nombre d'\textsl{\texttt{union}} effectuées}
		\State $p \gets \texttt{initialise\_partition}(S)$\/ 
		\State $I \gets 0$\/ 
		\State $B \gets \O$\/ 
		\While{$f < n - 1$}
			\State $\{x,y\} \gets e_I$\/ 
			\If{$\texttt{find}(p, x) \neq \texttt{find}(p, y)$}
				\State $p \gets \texttt{union}(p, x, y)$\/ 
				\State $B \gets B \cup \{\!\{x,y\}\!\}$\/ 
				\State $f \gets f + 1$\/ 
			\EndIf
			\State $I \gets I + 1$\/
		\EndWhile
		\State\Return $(S,B)$
	\end{algorithmic}
	\caption{Algorithme de \textsc{Kruskal} -- version 2}
\end{algorithm}

\paragraph{Étude de complexité.}
Notons $C_{\texttt{find}}^n$\/ un majorant du coût de \texttt{find} sur une structure contenant $n$\/ éléments, notons $C_{\texttt{union}}^n$\/ un majorant du coût de \texttt{union} sur une structure contenant $n$\/ éléments, et notons $C_{\texttt{init}}^n$\/ un majorant du coût de \texttt{init} sur une structure contenant $n$\/ éléments.
La complexité de cet algorithme est de \[
	\mathcal{O}\big(C_{\texttt{init}}^n + 2m\:C_{\texttt{find}}^n + n\: C_{\texttt{union}}^n + m \log_2 m\big)
.\]

\section{Couplage dans un graphe biparti}

\begin{defn}[Couplage]
	On appelle \textit{couplage} d'un graphe non orienté $G = (S, A)$, la donnée d'un sous-ensemble $C \subseteq A$\/ tel que \[
		\forall \{x,y\}, \{x',y'\} \in C,\:
		\quad\quad \{x,y\} \cap \{x',y'\} \neq \O
		\implies
		\{x,y\}  = \{x', y'\}
	.\]

	\index{graphe!couplage}
\end{defn}

\begin{figure}[H]
	\centering
	\tikzfig{ex-couplage}
	\caption{Exemple de couplage}
\end{figure}

\begin{exm}
	On réutilise l'exemple ci-dessous dans toute la section.
	L'ensemble $C = \{\{a,2\}, \{b,3\}\}$\/ est un couplage.
	Mais, l'ensemble $C' = \{\{a,1\}, \{a,2\}\}$\/ n'en est pas un.
\end{exm}

\begin{defn}
	Un couplage est dit \textit{maximal} s'il est maximal pour l'inclusion ($\subseteq$).
	Un couplage est dit \textit{maximum} si son cardinal est maximal.
	\index{graphe!couplage!maximal}
	\index{graphe!couplage!maximum}
\end{defn}

\begin{exm}
	Dans l'exemple précédent, 
	\begin{itemize}
		\item le couplage $C = \{\{a,2\}, \{b,3\}\}$\/ n'est ni maximal, ni maximum ;
		\item le couplage $C' = \{\{a,2\}, \{b, 3\}, \{d, 4\}\}$\/ est maximal mais pas maximum ;
		\item le couplage $C'' = \{\{a,1\}, \{b,3\}, \{c,2\}, \{d,4\}\}$\/ est maximum.
	\end{itemize}
\end{exm}

\begin{rmk}
	Dans toute la suite, on ne considère que des graphes bipartis.
\end{rmk}

\begin{defn}
	Étant donné un graphe biparti $G = (S, A)$\/ et un couplage $C$, un sommet $x$\/ est dit \textit{libre} dès lors que \[
		\forall \{y,z\} \in C,\: x \not\in \{y,z\}
	.\]
	Une chaîne élémentaire\footnotemark $(c_0, c_1, \ldots, c_{2p+1})$\/ est dit \textit{augmentante} si
	\begin{itemize}
		\item $c_0$\/ et $c_{2n+1}$\/ sont libres ;
		\item $\forall i \in \llbracket 0,p \rrbracket$, $\{c_{2i}, c_{2i+1}\}\in A \setminus C$ ;
		\item $\forall i \in \llbracket 0,p-1 \rrbracket$, $\{c_{2i+1}, c_{2i+2}\} \in C$.
	\end{itemize}
\end{defn}
\footnotetext{\textit{i.e.}\ une chaîne sans boucles.}

\begin{exm}~
	\begin{figure}[H]
		\centering
		\tikzfig{ex-chaine-augmentante}
		\caption{Chaîne augmentante}
	\end{figure}
\end{exm}

\begin{exm}
	Dans l'exemple de cette section, $(d, 4)$\/ et $(c, 2, a, 1)$\/ sont deux chaînes augmentantes.
\end{exm}

\begin{prop}
	Étant donné un graphe biparti $G = (S, A)$\/ avec $S = S_1 \cupdot S_2$ (partitionnement du graphe biparti), un couplage $C$\/ est maximum si, et seulement s'il n'admet pas de chaînes augmentantes.
\end{prop}

\begin{prv}
	\begin{itemize}
		\item[``$\implies$'']
			Soit $C$\/ un couplage admettant une chaîne augmentante. Montrons que $C$\/ n'est pas maximum.
			Soit la chaîne augmentante\footnotemark \[
				c_0 \to c_1 \Rightarrow c_2 \to c_3 \Rightarrow c_4 \to \cdots \to c_{2p - 1} \Rightarrow c_{2p} \to c_{2p+1}
			.\]
			On considère alors le couplage \[
				C' = \Big(C \setminus \big\{\{c_{2i+1},c_{2i+2} \} \mid i \in \llbracket 0,p-1 \rrbracket\big\}\Big) \cup \big\{\{c_{2i},c_{2i+1}\}  \mid i \in \llbracket 0,p \rrbracket\big\}
			.\]
			On transforme donc la chaîne en \[
				c_0 \Rightarrow c_1 \to c_2 \Rightarrow c_3 \to \cdots \to c_{2p-1} \to c_{2p} \Rightarrow c_{2p+1}
			.\]
			C'est bien un couplage, et $\Card(C') = \Card C + 1$. $C$\/ n'est donc pas un couplage maximum.
		\item[``$\impliedby$'']
			Soit $C$\/ un couplage non maximum. Montrons que $C$\/ admet une chaîne augmentante. Soit $M$\/ un couplage maximum, et $D = C \mathrel{\triangle} M = (C \setminus M) \cupdot (M \setminus C)$.
			On a $\Card C < \Card M$\/ et $\Card(C \setminus M) < \Card(M \setminus C)$.
			On remarque que, si $c_0 \to c_1 \to c_2 \to \cdots \to c_{p-1}\to c_p$\/ est une chaîne de $D$, (si $c_0 \to c_1 \in C \setminus M$\/ et $c_1 \to c_2 \in C \setminus M$\/ donc $c_1$\/ est dans deux arrêtes distinctes d'un couplage $C$, ce qui est absurde ; de même pour les autres arrêtes). Ainsi, 2 arrêtes consécutives ne sont pas dans la même composante de l'union $(C \setminus M) \cupdot (M \setminus C)$.
			Considérons la relation d'équivalence $\sim$\/ sur $D$\/ définie par $\{x,y\} \sim \{z,t\} \iffdef$ il existe une chaîne de $D$\/ utilisant l'arrête $\{x,y\}$\/ et l'arrête $\{z,t\}$.
			Soit le partitionnement $D_1, \ldots, D_q$\/ de $D$\/ par $\sim$.
			Par inégalité de cardinal, il existe un $D_i$\/ tel que \[
				\Card \{e \in D_i  \mid e \in C\} < \Card \{e \in D_i  \mid e \in M\}
			.\] L'ensemble $D_i$\/ contient alors une chaîne augmentante.
	\end{itemize}
\end{prv}
\footnotetext{On représente $\Rightarrow$ pour les arrêtes dans le couplage $C$.}

	}
	\def\addmacros#1{#1}
}

{
	\chap[10]{Concurrence}
	\minitoc
	\renewcommand{\cwd}{../cours/chap10/}
	\addmacros{
		\section{(Ne pas) être diagonalisable}

\begin{defn}
	Soit une matrice carrée $A$. On dit que $A$\/ est {\it diagonalisable}\/ s'il existe une matrice inversible~$P \in \mathrm{GL}_n(\mathds{K})$\/ telle que $P^{-1}\cdot A\cdot P$\/ est diagonale.
\end{defn}

\begin{exo}
	\begin{enumerate}
		\item Montrons que la matrice $B = {7\: 1\choose 0\:7}$\/ n'est pas diagonalisable.
			Par l'absurde : on suppose qu'il existe $P \in \mathrm{GL}_2(\R)$\/ et $(\lambda_1, \lambda_2) \in \R^2$\/ tels que \[
				P^{-1} \cdot B \cdot P = \begin{bmatrix}
					\lambda_1 & 0\\
					0&\lambda_2
				\end{bmatrix}
			.\] On applique la trace $\tr$\/ et le déterminant $\det$\/ :
			\begin{gather*}
				\tr(B) = \tr{\lambda_1\:0\choose 0\:\lambda_2} \quad\text{d'où}\quad \lambda_1 + \lambda_2 = 7 + 7 = 14 = \s\\
				\det(B) = \det{\lambda_1\:0\choose 0\:\lambda_2} \quad\text{d'où}\quad \lambda_1 \times \lambda_2 = 7 \times 7 = 49 = p
			\end{gather*}
			D'où $\lambda_1$\/ et $\lambda_2$\/ sont des solutions de l'équation $X^2 - \s X + p = 0$. Or
			\begin{align*}
				X^2 - \s X + p = 0 \iff& X^2 - 14X + 49 = 0\\
				\iff& (X-7)^2 = 0\\
				\iff& X = 7.
			\end{align*}
			D'où 
			\begin{align*}
				B = P P^{-1} B P P^{-1} = P \begin{pmatrix}
					7&0\\
					0&7
				\end{pmatrix} P^{-1} = P \cdot 7I_2\cdot P^{-1} = 7I_2.
			\end{align*}
			La matrice $B$\/ n'est donc pas diagonalisable.

			De même, montrons que la matrice $A$\/ n'est pas diagonalisable. On remarque que \[
				A \cdot \mat{1\\1\\1} = \begin{pmatrix}
					0&1&2\\
					1&0&2\\
					0&0&3
				\end{pmatrix} \begin{pmatrix}
					1\\1\\1
				\end{pmatrix} = \begin{pmatrix}
					3\\3\\3
				\end{pmatrix} = 3\begin{pmatrix}
					1\\1\\1
				\end{pmatrix} 
			.\] Ainsi, \[
				P^{-1}\cdot A\cdot P = \begin{pmatrix}
					3&0&0\\
					0&?&0\\
					0&0&?
				\end{pmatrix}\qquad\text{où}\qquad P = \begin{pmatrix}
					1&?&?\\
					1&?&?\\
					1&?&?
				\end{pmatrix}
			.\] De même, $A\left( \substack{1\\1\\0} \right) = 1 \times \left( \substack{1\\1\\0} \right)$. D'où \[
				P^{-1}\cdot A\cdot P = \begin{pmatrix}
					3&0&0\\
					0&1&0\\
					0&0&?
				\end{pmatrix}\qquad\text{où}\qquad P = \begin{pmatrix}
					1&1&?\\
					1&1&?\\
					1&0&?
				\end{pmatrix}
			.\] Finalement, on en conclut que \[
				P = \begin{pmatrix}
					3&0&0\\
					0&1&0\\
					0&0&-1
				\end{pmatrix} \qquad \text{et}\qquad P^{-1}\cdot A\cdot P = \begin{pmatrix}
					1&1&1\\
					1&1&-1\\
					1&0&0
				\end{pmatrix} = D
			.\]
			De plus, la matrice $P$\/ est inversible car $\det P \neq 0$.
		\item Pour calculer $A^n$, on pourrait chercher un polynôme annulateur $Q$\/ de $A$, et on exprime $X^n = Q \times T_n + R_n$, et donc $A^n = R_n(A)$.
			Mais, on peut également diagonaliser $A$\/ (si elle est diagonalisable).
			Ainsi,  \[
				D^n = (P^{-1}\cdot A\cdot P)^n = P^{-1}\cdot A\cdot \cancel P\cdot \cancel{P^{-1}} \cdot \ldots\cdot \cancel{P^{-1}} \cdot A \cdot P = P^{-1}\cdot  A^n\cdot P
			.\] D'où $A^n = P \cdot D^n \cdot P^{-1}$. Or, \[
				D^n = \begin{pmatrix}
					3&0&0\\
					0&1&0\\
					0&0&-1
				\end{pmatrix}^n = \begin{pmatrix}
					3^n&0&0\\
					0&1^n&0\\
					0&0&(-1)^n
				\end{pmatrix}
			.\]
			On calcule donc $A^{n}$\/ en calculant l'inverse de $P$\/ : \[
				A^n = \begin{pmatrix}
					1&1&1\\
					1&1&-1\\
					1&0&0
				\end{pmatrix} \begin{pmatrix}
					3^n&0&0\\
					0&1^n&0\\
					0&0&(-1)^n
				\end{pmatrix} \cdot P^{-1}
			.\]
		\item
			\begin{align*}
				\begin{rcases*}
					\hfill u_{n+1} = v_n + 2w_n\\
					\hfill v_{n+1} = u_n + 2w_n\\
					\hfill w_{n+1} = 3w_n
				\end{rcases*} \iff& \begin{pmatrix}
					u_{n+1}\\v_{n+1}\\w_{n+1}
				\end{pmatrix} = \begin{pmatrix}
					0&1&2\\
					1&0&2\\
					0&0&3
				\end{pmatrix} \begin{pmatrix}
					u_n\\ v_n\\ w_n
				\end{pmatrix}\\
				\iff& U_{n+1} = A\cdot U_n\\
				\iff& U'_{n+1} = D \cdot U'_{n}
			\end{align*}
			où $D = P^{-1} \cdot A \cdot P$, $U'_{n+1} = P\cdot U_{n+1}$\/ et $U'_n = P\cdot U_n$.
			\begin{align*}
				\phantom{\begin{rcases*}
					\hfill mm_{n+1} = v_n + 2w_n\\
					\hfill v_{n+1} = u_n + 2w_n\\
					\hfill w_{n+1} = 3w_n
				\end{rcases*}} \iff&
				\begin{pmatrix}
					u'_{n+1}\\v'_{n+1}\\w'_{n+1}
				\end{pmatrix} = \begin{pmatrix}
					3&0&0\\
					0&1&0\\
					0&0&-1
				\end{pmatrix} \cdot \begin{pmatrix}
					u'_n\\
					v'_n\\
					w'_n
				\end{pmatrix}\\
				\iff& \begin{cases}
					u'_{n+1} = 3u'_n\\
					v'_{n+1} = v'_n\\
					w'_{n+1} = -w'_n
				\end{cases}\\
				\iff& \begin{cases}
					u'_n = K\times  3^n\\
					v'_n = L\\
					w'_n = M \times (-1)^n
				\end{cases}
			\end{align*}
			Ainsi, \[
				\begin{pmatrix}
					u_n\\v_n\\w_n
				\end{pmatrix} = \underbrace{\begin{pmatrix}
					1&1&1\\
					1&1&-1\\
					1&0&0
				\end{pmatrix}}_P \cdot \begin{pmatrix}
					K\times 3^n\\
					L\\
					M\times (-1)^n
				\end{pmatrix}
			.\] D'où $u_n = K\cdot 3^n + L + M \cdot (-1)^n$, $v_n = K\times 3^n + L - M \cdot (-1)^n$\/ et $w_n = K\cdot 3^n$, où les constantes $K$, $L$\/ et $M$\/ sont des constantes fixées par les conditions initiales.
		\item
			\begin{align*}
				\begin{rcases*}
					\hfill x'(t) = y(t) + 2z(t)\\
					\hfill y'(t) = x(t) + 2z(t)\\
					\hfill z'(t) = 3z(t)
				\end{rcases*} \iff& \begin{pmatrix}
					x'(t)\\
					y'(t)\\
					z'(t)
				\end{pmatrix} = \begin{pmatrix}
					0&1&2\\
					1&0&2\\
					0&0&3
				\end{pmatrix} \cdot \begin{pmatrix}
					x(t)\\
					y(t)\\
					z(t)
				\end{pmatrix}\\
				\iff& X'(t) = A\cdot X(t)\\
				\iff& U'(t) = D \cdot U(t) \text{ avec } D = P^{-1} \cdot A\cdot P \text{ et } X(t) = P\cdot U(t)\\
				\iff& \begin{pmatrix}
					u'(t)\\
					v'(t)\\
					w'(t)
				\end{pmatrix} = \begin{pmatrix}
					3&0&0\\
					0&1&0\\
					0&0&-1
				\end{pmatrix} \cdot \begin{pmatrix}
					u(t)\\
					v(t)\\
					w(t)
				\end{pmatrix}\\
				\iff& \begin{cases}
					u'(t) = 3u(t)\\
					v'(t) = v(t)\\
					w'(t) = -w(t)
				\end{cases}\\
				\iff& \begin{cases}
					u(t) = K \cdot \mathrm{e}^{3t}\\
					v(t) = L \cdot \mathrm{e}^{t}\\
					w(t) = M \cdot \mathrm{e}^{-t}
				\end{cases}
			\end{align*}
			Ainsi \[
				\begin{pmatrix}
					x(t)\\
					y(t)\\
					z(t)
				\end{pmatrix} = \underbrace{\begin{pmatrix}
					1&1&1\\
					1&1&-1\\
					1&0&0
				\end{pmatrix}}_P \cdot \begin{pmatrix}
					K \times \mathrm{e}^{3t}\\
					L \cdot \mathrm{e}^{t}\\
					M \cdot \mathrm{e}^{-t}
				\end{pmatrix}
			.\] 
			D'où $x(t) = K\cdot \mathrm{e}^{3t} + L \cdot \mathrm{e}^{t} + M \cdot \mathrm{e}^{-t}$, $y(t) = K \cdot \mathrm{e}^{3t} + L \cdot \mathrm{e}^{t} - M \cdot \mathrm{e}^{-t}$\/ et $z(t) = K\cdot \mathrm{e}^{3t}$. Les constantes $K$, $L$\/ et $M$\/ peuvent être déterminées à partir des conditions initiales.
	\end{enumerate}
\end{exo}

\begin{rmkn}[équations différentielles]
	On considère l'équation différentielle $(*)$ : $x'(t) = \lambda \cdot x(t)$.
	Les fonctions $x : t \mapsto K\cdot \mathrm{e}^{\lambda t}$\/ sont des solutions de cette équation. On peut utiliser la méthode de {\sc Lagrange}\/ : la méthode de la~\guillemotleft~variation de la constante.~\guillemotright\@ On cherche des solutions sous la forme $x(t) = k(t) \cdot \mathrm{e}^{\lambda t}$ (vision du~physicien). D'où $k(t) = x(t) / \mathrm{e}^{\lambda t}$\/ (vision du mathématicien). De plus, $x'(t) = k'(t) \mathrm{e}^{\lambda t} + k(t) \lambda \mathrm{e}^{\lambda t}$.
	Ainsi, on injecte ce $k(t)$\/ dans l'équation différentielle :
	\begin{align*}
		(*) \iff& k'(t) \mathrm{e}^{\lambda t} + k(t) \lambda \mathrm{e}^{\lambda t} = \lambda k(t)\mathrm{e}^{\lambda t}\\
		\iff& k'(t) \mathrm{e}^{\lambda t} = 0\\
		\iff& k'(t) = 0\\
		\iff& \exists K \in \R\,\:k(t) = K.
	\end{align*}
	Les solutions trouvées dans l'exercice précédent sont donc les uniques solutions du système d'équations différentielles.

	De même, pour résoudre une équation différentielle avec 2\tsup{nd} membre de la forme \[
		(**) : \qquad x'(t) - \lambda \cdot x(t) = b(t)
	.\]
	La fonction $t \mapsto x(t)$\/ est une solution de l'équation {\sc sans}\/ 2\tsup{nd} membre si et seulement si \[
		\exists K \in \R,\:\forall t \in \R,\quad x(t) = K \cdot \mathrm{e}^{\lambda t}
	.\]
	\begin{center}
		\slshape Comment résoudre l'équation différentielle {\scshape avec}\/ 2\tsup{nd} membre si on connaît la solution générale de l'équation {\scshape sans}\/ 2\tsup{nd} membre ?
	\end{center}
	On utilise la méthode le la variation de la constante.
	Soit $x(t) = k(t) \cdot \mathrm{e}^{\lambda t}$. Ainsi, en injectant cette expression de $x$\/ dans l'équation $(**)$, on trouve
	\begin{align*}
		(**) \iff& k'(t) \mathrm{e}^{\lambda t} + k(t) \cdot \lambda \mathrm{e}^{\lambda t} = \lambda k(t) \mathrm{e}^{\lambda t} + b(t)\\
		\iff& k'(t) \mathrm{e}^{\lambda t} = b(t)\\
		\iff& k'(t) = b(t) \cdot \mathrm{e}^{-\lambda t}\\
		\iff& k(t) = \int_{0}^{t} b(u)\cdot \mathrm{e}^{-\lambda u}~\mathrm{d}u + K\\
		\iff& x(t) = \left( \int_{0}^{t} b(u) \cdot \mathrm{e}^{-\lambda u}~\mathrm{d}u + K \right) \mathrm{e}^{\lambda t}\\
		\iff& x(t) = \underbrace{\int_{0}^{t} b(u) \cdot \mathrm{e}^{\lambda (t-u)}~\mathrm{d}u}_{\text{solution particulière}} + \underbrace{K \cdot \mathrm{e}^{\lambda t}}_{\substack{\text{solution}\\\text{générale}\\\text{de $(*)$}}}.
	\end{align*}
\end{rmkn}

		\begin{exm}
	On pose $f$, le sinus cardinal :  \begin{align*}
		f: \R^* &\longrightarrow \R \\
		t &\longmapsto \frac{\sin t}{t}.
	\end{align*}
	\begin{figure}[H]
		\centering
		\begin{asy}
			import graph;
			size(10cm);
			draw((-10, 0) -- (10, 0), Arrow(TeXHead));
			draw((0, -3) -- (0, 5), Arrow(TeXHead));
			real f(real x) {
				if(x == 0) { return 3; }
				else {return 3*sin(x) / x;}
			}
			draw(graph(f, -10, 10), magenta);
		\end{asy}
		\caption{Sinus cardinal}
	\end{figure}

	La fonction $f$\/ est continue sur ${]0,8]}$\/ mais $\lim_{t\to 0} \frac{\sin t}{t} = 1$. D'où $\int_{0}^{8} \frac{\sin t}{t}~\mathrm{d}t$\/ est faussement impropre en $0$\/ et donc convergente.


	Mais attention ! On ne dit pas \guillemotleft~{\color{red}soit $f : t \mapsto \frac{1}{t}$. L'intégrale $\int_{8}^{+\infty} \frac{1}{t}~\mathrm{d}t$\/ est faussement impropre en $+\infty$\/ car $\lim_{t\to +\infty}\frac{1}{t} = 0$}.~\guillemotright
\end{exm}

\section{Intégrer les $\mathbf{\sim}$, $\po$, et \textit{O}}

\begin{thm}
	\hfill$\O$\hfill\null
\end{thm}

\begin{thm}
	Le 2.\ n'est pas la réciproque du 1.\ mais la contraposée.
\end{thm}

\begin{prop}
	\hfill$\O$\hfill\null
\end{prop}

\begin{exm}
	On considère l'intégrale $\int_{2}^{+\infty} \frac{1}{t^2+ \cos t}~\mathrm{d}t$, c'est une intégrale impropre en $+\infty$.
	On recherche un équivalent de $\frac{1}{t^2 + \cos t}$\/ en $+\infty$ : \[
		\frac{1}{t^2 + \cos t} \simi_{t\to +\infty} \frac{1}{t^2}
	\] qui ne change pas de signe. Or, $\int_{2}^{+\infty} \frac{1}{t^2}~\mathrm{d}t$\/ converge car c'est une intégrale de {\sc Riemann}\/ avec $\alpha = 2 > 1$.
	On en déduit que l'intégrale $I$\/ converge.

	On procède autrement : \[
		0 \le \frac{1}{t^2 + \cos t} \le \frac{1}{t^2 - 1}
	.\] Or, $\int_{2}^{+\infty} \frac{1}{t^2 - 1}~\mathrm{d}t$\/ converge car
	\begin{align*}
		\int_{2}^{x} \frac{1}{t^2 - 1}~\mathrm{d}t &= \int_{2}^{x} \left( \frac{\sfrac12}{t-1} - \frac{\sfrac12}{t+1} \right) ~\mathrm{d}t \\
		&= \frac{1}{2} \int_{2}^{x} \frac{1}{t-1}~\mathrm{d}t - \frac{1}{2}\int_{2}^{x} \frac{1}{t+1}~\mathrm{d}t \\
		&= \frac{1}{2} \Big[\ln|t-1|\Big]_2^x - \frac{1}{2}\Big[\ln |t+1|\Big]_2^x \\
	\end{align*}
	D'où \[
		\int_{2}^{x} \frac{1}{t^2 - 1}~\mathrm{d}t = \frac{1}{2} \left[ \ln\left| \frac{t-1}{t+1} \right| \right]_2^x = \frac{1}{2}\ln \left| \frac{x-1}{x+1} \right| + \frac{1}{2} \ln 3 \tendsto{x\to +\infty} \frac{1}{2} \ln 3
	.\] donc l'intégrale $I$\/ converge et $I \le \frac{1}{2} \ln_3$.
\end{exm}

\begin{exo}
	\begin{enumerate}
		\item L'intégrale $I = \int_{0}^{1} \frac{\sin t}{t^2}~\mathrm{d}t$\/ est impropre en 0. On utilise un équivalent : $\sin t \simi_{t\to 0} t$\/ qui ne change pas de signe. Or, $\int_{0}^{t} \frac{1}{t}~\mathrm{d}t$\/ diverge (par critère de {\sc Riemann}). Donc $I$\/ diverge.
			
			L'intégrale $J = \int_{1}^{+\infty} \sin \frac{1}{t}~\mathrm{d}t$\/ est généralisée en $+\infty$. On cherche un équivalent en $+\infty$\/ : \[
				\sin \frac{1}{t} \simi_{t\to +\infty} \frac{1}{t}
			\] qui ne change pas de signe. Or, $\int_{1}^{+\infty} \frac{1}{t}~\mathrm{d}t$\/ diverge par critère de {\sc Riemann}. On en déduit que $J$\/ diverge également.
		\item L'intégrale $\int_{0}^{+\infty} \frac{1}{t^2}~\mathrm{d}t$\/ est impropre, {\bf et}\/ en 0, {\bf et}\/ en $+\infty$. Le théorème ne marche donc pas.
			En effet $t\mapsto \frac{1}{t^2}$\/ n'est pas continue par morceaux en 0, ce qui était le cas pour $t\mapsto \frac{1}{1+t^2}$.
	\end{enumerate}
\end{exo}

\begin{rmkn}[Retour sur la {\sc remarque}\/ 5]
	L'intégrale $\int_{0}^{+\infty} \frac{1}{\ln(1+t)}~\mathrm{d}t$\/ est impropre en 0 {\bf et}\/ en $+\infty$. $\int_{0}^{+\infty} \frac{1}{\ln(1+t)}~\mathrm{d}t$\/ converge si et seulement si $\int_{0}^{7} \frac{1}{\ln(1+t)}~\mathrm{d}t$\/ {\bf et}\/ $\int_{7}^{+\infty} \frac{1}{\ln(1+t)}~\mathrm{d}t$\/ convergent.
	Et si elles convergent \[
		\int_{0}^{+\infty} \frac{1}{\ln(1+t)}~\mathrm{d}t = \int_{0}^{7} \frac{1}{\ln(1+t)}~\mathrm{d}t + \int_{7}^{+\infty} \frac{1}{\ln(1+t)}~\mathrm{d}t
	.\]
	On n'utilise pas deux barrières en même temps. Sinon, les intégrales doublement impropres peuvent, et converger, et diverger.
\end{rmkn}

\begin{prop}[avec $\sim$]
	Si $f(t) \simi_{t\to b} g(t)$\/ qui ne change pas de signe. Alors,
	\begin{itemize}
		\item ou bien $\ds\int_{a}^{b} f(t)~\mathrm{d}t$\/ et $\ds\int_{a}^{b} g(t)~\mathrm{d}t$\/ convergent et $\ds \int_{x}^{b} f(t)~\mathrm{d}t \simi_{x\to b} \int_{x}^{b} g(t)~\mathrm{d}t$.
		\item ou bien $\ds\int_{a}^{b} f(t)~\mathrm{d}t$\/ et $\ds\int_{a}^{b} g(t)~\mathrm{d}t$\/ divergent et $\ds\int_{a}^{x} f(t)~\mathrm{d}t \simi_{x\to b} \int_{a}^{x} g(t)~\mathrm{d}t$.
	\end{itemize}
	Cette proposition est équivalente à le {\sc lemme}\/ 13 sur les séries.
\end{prop}



		\begin{prop}
	La relation $\preceq$\/ est un \textit{pré-ordre} :
	\begin{itemize}
		\item $\preceq $\/ est réflective ;
		\item $\preceq $\/ est transitive.
	\end{itemize}
\end{prop}

\begin{prv}
	Soit $Q$\/ un problème de décision.
	\begin{itemize}
		\item $Q \preceq Q$\/ par la fonction identité, qui est totale et calculable.
		\item Soient $Q$, $R$\/ et $S$\/ trois problèmes de décision tels que $Q \preceq R$\/ et $R \preceq S$. Soit donc $f_1$\/ la réduction de $Q$\/ à $R$, et $f_2$\/ la réduction de $R$\/ à $S$. Soit $f = f_2 \circ f_1 : \mathcal{E}_Q \to \mathcal{E}_S$. La fonction $f$\/ est totale comme composée de fonctions totales, $f$\/ est calculable comme composée de fonctions calculables. De plus,
			\begin{align*}
				\forall e \in \mathcal{E}_Q,\qquad f(e) \in S^+ \iff& f_2(f_1(e)) \in S^+\\
				\iff& f_1(e) \in R^+\\
				\iff& e \in Q^+
			\end{align*}
	\end{itemize}
\end{prv}

\section{Classe \textbf{P} et \textbf{NP}}

Pour répondre à un problème, on peut le résoudre par des algorithmes plus ou moins rapides. Mais, l'objectif de cette section est de montrer que certains problèmes ne peuvent se résoudre que par des algorithmes lents, et que l'on ne peut pas faire mieux.

\begin{defn}
	Le modèle de calcul impose une représentation des entrées par chaînes de caractères. Cela induit donc une notion de \textit{taille d'entrée}, qui est la longueur de la chaîne de caractères.
	\index{taille d'entrée}
\end{defn}


\subsection{Complexité d'une machine}

\begin{defn}
	Étant donné une machine $\mathcal{M}$ et une entrée $w \in \Sigma^*$, on note $C^\mathcal{M}(w)$\/ le nombre d'opérations élémentaires effectuées lors de l'appel de $\mathcal{M}$\/ sur $w$. Lorsque $\smash{w \xrightarrow[\mathcal{M}]{} {\circlearrowleft}}$, on définit $C^\mathcal{M} = +\infty$.

	Pour $n \in \N$, on définit alors \[
		C^\mathcal{M}_n = \max \{ C^\mathcal{M}(w)  \mid w \in \Sigma^n \}
	.\]
	\index{machine!nombre d'opérations élémentaires!($C^\mathcal{M}(w)$)}
	\index{machine!nombre d'opérations élémentaires!maximal pour un mot de taille $n$\/ ($C^\mathcal{M}_n$)}
\end{defn}

\begin{rmk}
	On a, $\forall n \in \N$, $C_n^\mathcal{M} \in \bar{\N} = \N \cup \{+\infty\}$.
\end{rmk}

\begin{defn}
	Soit $f : \N\to \N$\/ une fonction totale et calculable. On note $\textsc{Time}(f)$\/ l'ensemble des machines $\mathcal{M}$\/ telles que
	\begin{itemize}
		\item $\mathcal{M}$\/ s'arrête sur toute entrée ;
		\item $\big(C_n^\mathcal{M}\big)_{n \in \N} = \mathcal{O}\big(\big(f(n)\big)_{n \in \N}\big)$.
	\end{itemize}
	\index{machine!ensemble $\textsc{Time}(f)$}
\end{defn}

\subsection{Classe \textbf{P}}

\begin{defn}
	On dit d'une machine $\mathcal{M}$\/ qu'elle est de \textit{complexité polynômiale} dès lors qu'il existe $k \in \N$\/ tel que $\mathcal{M} \in \textsc{Time}(n^k)$.
	\index{machine!de complexité polynômiale}
\end{defn}

\begin{defn}
	On dit d'une fonction (partielle ou non), qu'elle est \textit{calculable en temps polynômial} dès lors qu'il existe une machine $\mathcal{M}$\/ de complexité polynômiale la calculant.
	\index{fonction!calculable!en temps polynômial}
\end{defn}

\begin{exm}
	\begin{itemize}
		\item l'identité ($n \mapsto n$)
		\item la fonction successeur ($n\mapsto n+1$)
	\end{itemize}
\end{exm}


		\section{Arbres couvrants de poids minimum}

\begin{exm}
	On considère le graphe ci-dessous.
	\begin{figure}[H]
		\centering
		\tikzfig{ex-graphe-pondere}
		\caption{Arbre pondéré}
	\end{figure}
	On cherche à \guillemotleft~supprimer~\guillemotright\ des arrêtes de ce graphe afin d'avoir un poids total minimum, tout en conservant la connexité du graphe.
	Une structure assurant cette condition est un arbre.

	Pour résoudre ce problème, on part du graphe vide, et on ajoute les arrêtes les moins coûteuses en premier.
\end{exm}

\begin{defn}[Arbre]
	Soit $G = (S,A)$\/ un graphe non-orienté. On dit que $G$\/ est un \textit{arbre} si $G$\/ est connexe et acyclique.
	\index{arbre}
\end{defn}

\begin{defn}[Arbre couvrant]
	Étant donné un graphe non orienté pondéré par poids positifs $G = (S, A, c)$,\footnotemark\ on dit de $G' = (S', A')$\/ que c'est un \textit{arbre couvrant} de $G$\/ si $S' = S$\/ et $A' \subseteq A$, et $G'$\/ est un arbre.
	\index{arbre!couvrant}
\end{defn}
\footnotetext{on dit que $c$\/ est la fonction de pondération de ce graphe}

\begin{defn}[Arbre couvrant de poids minimum]
	Étant donné un graphe non orienté pondéré $G = (S, A, c)$\/ et un arbre couvrant $T = (S', A')$, on appelle \textit{poids} de l'arbre $T$\/ la valeur $\sum_{a \in A'} c(a)$.
	\index{arbre!couvrant!poids}

	Si $G$\/ est connexe, il admet au moins un arbre couvrant, on peut définir l'\textit{arbre couvrant de poids minimum} (\textit{\textsc{acpm}}).
	\index{arbre!couvrant!de poids minimum}
\end{defn}

On définir alors le problème \[
	\textsc{acpm}\text{\footnotemark}
	\begin{cases}
		\text{\textbf{Entrée}}&: G = (S, A, c) \text{ connexe}\\
		\text{\textbf{Sortie}}&: \text{ le poids de l'arbre couvrant de poids minimum}.
	\end{cases}
\]
\footnotetext{Arbre Couvrant de Poids Minimum}

\begin{algorithm}[H]
	\centering
	\begin{algorithmic}[1]
		\Entree $G = (S, A, c)$\/ un graphe connexe
		\Sortie Un arbre couvrant de poids minimum
		\State $B \gets \O$\/ 
		\State $U \gets \O$\/
		\While{il existe $u$\/ et $v$\/ tels que $u \nsim_B v$}
			\State Soit $\{x,y\} \in A \setminus U$\/ de poids minimal
			\If{$x \sim_B y$}
				\State $U \gets \big\{\!\{x,y\}\!\big\} \cup U$
			\Else
				\State $U \gets \big\{\!\{x,y\}\!\big\} \cup U$
				\State $B \gets \big\{\!\{x,y\}\!\big\} \cup B$
			\EndIf
		\EndWhile
		\State\Return $T = (S,B)$\/
	\end{algorithmic}
	\caption{Algorithme de \textsc{Kruskal}}
\end{algorithm}

\begin{prop}
	L'algorithme de \textsc{Kruskal} est correct.
\end{prop}

\begin{prv}
	\begin{enumerate}
		\item Il existe un arbre couvrant de poids minimum utilisant les arrêtes de $B$ ;
		\item $B \subseteq U \subseteq A$\/ ;
		\item $\forall \{u,v\} \in U$, $u \sim_B v$.
	\end{enumerate}
	Ces trois propriétés sont invariantes.
	\begin{description}
		\item[Initialement] $B = \O = U$, donc \textsc{ok}.
		\item[Propagation] Soient $\ubar{B}$\/ et $\ubar{U}$\/ (resp.\ $\bar{B}, \bar{U}$) les valeurs de $B$\/ et $U$\/ avant (resp.\ après) une itération de boucle. Supposons que $\ubar{B}$\/ et $\ubar{U}$\/ satisfont les propriétés 1, 2 et 3. Montrons que $\bar{B}$\/ et $\bar{U}$\/ les satisfont aussi.
			\begin{enumerate}
				\item[2.] On a $\{x,y\} \in A$\/ et $\ubar{B} \subseteq \ubar{U} \subseteq A$, donc \[
						\bar{B} \subseteq \ubar{B} \cup \{\!\{x,y\}\!\} \subseteq \ubar{U} \cup \{\!\{x,y\}\!\} \subseteq A
					.\]
				\item[3.] Soit $\{u,v\} \in \bar{U}$.
					\begin{itemize}
						\item Si $\{u,v\} \in \ubar{U}$, alors de 3, $u \sim_{\ubar{B}} v$. Or, $\ubar{B} \subseteq \bar{B}$\/ et donc $u \sim_{\bar{B}} v$.
						\item Sinon, $\{ u,v\} = \{x,y\}$, alors $x = u$\/ et $v = y$.
							\begin{itemize}
								\item Sous-cas 1 : $\bar{B} = \ubar{B} \cup \{\!\{x,y\}\!\}$, alors $x \sim_{\bar{B}} y$.
								\item Sous-cas 2 : $\bar{B} = \ubar{B}$, alors par condition du \textbf{si}, $x \sim_{\ubar{B}} y$\/ et donc $x \sim_{\bar{B}} y$.
							\end{itemize}
					\end{itemize}
				\item[1.]
					Soit $\mathcal{T}$\/ un \textsc{acpm} contenant $\ubar{B}$.
					\begin{itemize}
						\item Cas 1 : $\bar{B} = \ubar{B}$, \textsc{ok}
						\item Cas 2 : $\bar{B} = \ubar{B} \cup \{\!\{x,y\}\!\}$.
							\begin{itemize}
								\item Sous-cas 1 : $\{x,y\} \in \mathcal{T}$, alors $\mathcal{T}$\/ est un \textsc{acpm} qui contient $\bar{B}$.
								\item Sous-cas 2 : $\{x,y\}  \not\in \mathcal{T}$, $\mathcal{T}$\/ est un arbre couvrant, donc il contient une chaîne de $x$\/ à $y$\/ : \[
											\{\overset{\substack{x\\[-1mm]\vrt=}}{x_0},x_1\},\{x_1,x_2\},\ldots,\{x_{n-1},\underset{\substack{\vrt=\\y}}{x_n}\}
									.\]
									Or, $\forall i \in \llbracket 1,n-1 \rrbracket$, $x_i \sim_{\ubar{B}} x_{i+1}$. Par transitivité, on a donc $x = x_0 \sim_{\ubar{B}} x_n = y$, ce qui n'est pas le cas.
									Il existe donc $i_0 \in \llbracket 0,n-1 \rrbracket$, tel que $x_{i_0} \nsim_{\ubar{B}} x_{i_0 + 1}$\/ et donc $\{x_{i_0}, x_{i_0 + 1}\} \not\in \ubar{U}$. D'où, d'après 3, on a $\{x_{i_0}, x_{i_0 + 1}\}  \not\in \ubar{B}$
									Considérons alors $\mathcal{T}' = \big(\mathcal{T} \setminus \{\!\{x_{i_0},x_{i_0+1}\}\!\}\big)  \cup \{\!\{x,y\}\!\}$. Montrons que $\mathcal{T}'$\/ est un \textsc{acpm} contenant $B$, en commençant par montrer que c'est un arbre couvrant. L'arbre $\mathcal{T}'$\/ a $n-1$\/ arrêtes (autant que $\mathcal{T}$). Montrons que $\mathcal{T}'$\/ est connexe.
									Soit $(a,b) \in S^2$. $\mathcal{T}$\/ est connexe, soit donc une chaîne \[
										C : \quad a = u_0, u_1, \ldots, u_n = b
									\] de $\mathcal{T}$. Si la chaîne $C$\/ n'utilise pas l'arrête $\{x_{i_0},x_{i_0+1}\}$, alors $C$\/ est une chaîne de $\mathcal{T}'$. Sinon, on pose $\mu$\/ et $\tau$\/ tels que \[
									\underbrace{a,\ldots,x_{i_0}}_{\mu},\underbrace{x_{i_0+1},\ldots,b}_{\tau}
									.\]
									Soit alors la chaîne
									\begin{align*}
										\overbrace{a,\ldots,x_{i_0}}^{\mu},x_{i_0-1},x_{i_0-2},\ldots,x_0 = x,&\\
										\underbrace{b,\ldots,x_{i_0 + 1}}_{\tau},x_{i_0+2},\ldots,x_{n-1},x_n=y&
									\end{align*}
									qui est dans $\mathcal{T}'$.
									Montrons que le poids est minimum. Notons $P(\mathcal{T})$\/ le poids de l'arbre. On a donc \[
										P(\mathcal{T}') = P(\mathcal{T}) + c(\{x,y\}) - c(\{x_{i_0},x_{i_0+1}\})
									.\] Par choix glouton, ($\{x_{i_0}, x_{i_0+1} \not\in \ubar{U}\}$), $c(\{x,y\}) \le c(\{x_{i_0},x_{i_0+1}\})$\/ donc $P(\mathcal{T}') \le P(\mathcal{T})$, et $\mathcal{T}$\/ étant de poids min, $P(\mathcal{T}') = P(\mathcal{T})$\/ et $\mathcal{T}'$\/ est un \textsc{acpm} contenant $\bar{B}$.
							\end{itemize}
					\end{itemize}
			\end{enumerate}
	\end{description}

	Les invariants le sont.
\end{prv}

À la fin, $B$\/ induit un graphe connexe et $B$\/ est contenu dans un \textsc{acpm}, c'en est donc un.


	}
	\def\addmacros#1{#1}
}



\part{Travaux Dirigés}
\def\prefix{\textsc{td}}
\renewcommand{\chaptername}{Travaux Dirigés}


{
	\td[1]{Ordre \& Induction}
	\minitoc
	\renewcommand{\cwd}{../td/td01/}
	\addmacros{
		\section{Quelques problèmes décidables}

\begin{enumerate}
	\item Soit $f : \R \to \R$.
		\begin{itemize}
			\item Si $f$\/ admet un zéro, on pose $\mathcal{M} = \texttt{fun}\ \texttt{s}\ \to \texttt{true}$.
			\item Si $f$\/ n'admet pas un zéro, on pose $\mathcal{M} = \texttt{fun}\ \texttt{s}\ \to \texttt{false}$.
		\end{itemize}
		Alors, $\mathcal{M}$\/ décide \textsc{Zero}$_f$.
	\item Soit $\mathcal{M}$\/ une machine, et soit $w \in \Sigma^*$.
		\begin{itemize}
			\item Si $\mathcal{M}$\/ se termine sur l'entrée $w$, alors on pose $\mathcal{M}' = \texttt{fun}\ \texttt{s} \to \texttt{true}$.
			\item Si $\mathcal{M}$\/ ne se termine pas sur l'entrée $w$, alors on pose $\mathcal{M}' = \texttt{fun}\ \texttt{s} \to \texttt{false}$.
		\end{itemize}
		Alors, $\mathcal{M}'$\/ décide \textsc{Arrêt}$_{\mathcal{M},w}$.
	\item Le problème est trivialement vrai. En effet, soit $M \in \mathcal{O}$, de la forme
		\begin{lstlisting}[language=caml]
let m (s: string): string =
	%*$\langle$\textrm{code}$\rangle$*) 
		\end{lstlisting}
		On crée la machine $\mathcal{N}$\/ ci-dessous.
		\begin{lstlisting}[language=caml]
let n (s: string): string =
	if true then
		%*$\langle$\textrm{code}$\rangle$*) 
	else
		%*$\langle$\textrm{code}$\rangle$*) 
		\end{lstlisting}
		On a $\texttt{m} \neq \texttt{n}$, mais $\mathcal{L}(\texttt{m}) = \mathcal{L}(\texttt{n})$, donc le problème est vrai sur toute entrée et la fonction $\texttt{fun}\ \texttt{s} \to \texttt{true}$\/ répond au problème.
\end{enumerate}

		\section{Ensembles définis inductivement}

La correction est disponible sur \textit{cahier-de-prepa}.

\begin{comment}
	\begin{exm}
		Avec $S = \N$, $\mathcal{B} = \{0, 2\} $, $A_1 = \{0\}$\/ et \begin{align*}
			f_1: A_1 \times \N &\longrightarrow \N \\
			(0, x) &\longmapsto x + 4.
		\end{align*}

		On a \[
			X \supseteq \{0, 2, 4, 6, 8, 10, \ldots, 20, \ldots\} = 2\N
		.\]
	\end{exm}
	\begin{exm}
		Avec $S$\/ l'ensemble des langages sur $\Sigma$, $\mathcal{B} = \{\O\} \cup \bigl\{\{a\}\:\big|\: a \in \Sigma \bigr\}$, et
		\begin{multicols}{3}
			\begin{align*}
				f_1: S \times S &\longrightarrow S \\
				(L_1, L_2) &\longmapsto L_1 \cup L_2,
			\end{align*}
			\begin{align*}
				f_2: S \times S &\longrightarrow S \\
				(L_1, L_2) &\longmapsto L_1 \cdot L_2,
			\end{align*}
			\begin{align*}
				f_3: S &\longrightarrow S \\
				L &\longmapsto L^*.
			\end{align*}
		\end{multicols}
	\end{exm}

\begin{enumerate}
	\item Soit $\mathcal{A} = \{X \subseteq S  \mid X \supseteq \mathcal{B} \mathrel{\text{et}} X \text{ est stable par } f_i\}$. On a $S \in \mathcal{A}$\/ et donc $\mathcal{A} \neq \O$. De plus, soit \[
			Y = \{x \in S  \mid \forall X \in \mathcal{A},\,x \in X\} = \bigcap_{X \in \mathcal{A}} X
		.\]
		Soit $b \in \mathcal{B}$, on a $\forall X \in A,\, b \in X$. D'où $b \in Y$\/ par intersection. On en déduit que $\mathcal{B} \subseteq Y$.

		Soit $i \in \left\llbracket 1,m \right\rrbracket$. Soit $(x_1, \ldots, x_{n_i}) \in Y^{n_i}$\/ et soit $a \in A_i$. Montrons que $f_i(a, x_1, \ldots, x_{n_i}) \in Y$.
		Or, soit $X \in \mathcal{A}$, on a $(x_1, \ldots, x_{n_i}) \in X^{n_i}$\/ donc $f_i(a, x_1, \ldots, x_{n_i}) \in X$. Ceci étant vrai pour tout $X \in \mathcal{A}$, on a $f_i(a, x_1, \ldots, x_{n_i}) \in Y$\/ donc $Y$\/ est stable par $f_i$\/ par tout $i \in \left\llbracket 1,m \right\rrbracket$\/ et donc $Y \in \mathcal{A}$.
		On a également $Y \subseteq X$\/ pour tout $X \in \mathcal{A}$. On en déduit que $Y$\/ est le plus petit élément (pour l'inclusion) de $\mathcal{A}$.
	\item On pose $X_0  = \mathcal{B}$\/ et \[
			X_{n+1} = X_n \cup \big\{ f_i(a, x_1, \ldots, x_{n_i})  \mid a \in A_i,\,(x_1, \ldots, x_{n_i}) \in (X_n)^{n_i},\,i \in \left\llbracket 1,m \right\rrbracket\big\}
		.\]
		Soit $X = \bigcup_{n \in \N} X_n$. Soit $Y$\/ l'ensemble défini par induction à partir de $\mathcal{B}$\/ et des $(f_i)_{i\in\left\llbracket 1,n \right\rrbracket}$. Montrons que $X = Y$.
		On montre que $X$\/ est le plus petit élément (pour l'inclusion) de $\mathcal{A}$\/ et on conclut par unicité du minimum (avec la question précédente).
		Par définition de la suite $(X_n)_{n\in\N}$, elle est croissante (au sens de l'inclusion).
		Montrons à présent, par récurrence, la propriété ci-dessous : $P_n : ``X_n \subseteq Y."$
		\begin{itemize}
			\item Par définition de $Y$, on a $X_0 = \mathcal{B} \subseteq Y$.
			\item Soit 
		\end{itemize}
\end{enumerate}

\subsection{Un théorème d'induction}

\begin{enumerate}
	\item[3.] Soit $Z = \{x \in S  \mid P(x) \text{ vraie}\:\}$.
		Montrons que $\mathcal{X}\subseteq Z$.
		On remarque que $\mathcal{X} \supseteq \mathcal{B}$\/ ; $\mathcal{X}$\/ est stable par $f_i$. On en conclut que $Z \supseteq \mathcal{X}$ et donc $\forall x \in \mathcal{X},\,P(x)$\/ est vraie.
\end{enumerate}

\begin{exm}
	Soit $\mathcal{X}$\/ défini par induction par $\mathcal{B} = \{0, 2\}$\/ et \begin{align*}
		f: \N &\longrightarrow \N \\
		n &\longmapsto n + 2.
	\end{align*}
	Montrons que $\forall n \in \mathcal{X}$, $x$\/ est pair.

	On sait que $0$\/ est pair, $2$\/ est pair ; et, \[
		\forall x,y \in \mathcal{X},\, (x \text{ pair} \land y \text{ pair}) \implies f(x, y) \text{ pair}
	.\]

	On en déduit que \[
		\forall n \in \mathcal{X},\,x \text{ est pair}.
	.\]
\end{exm}
\end{comment}

		\section{Tableaux dynamiques}

\begin{enumerate}[start=3]
	\item On trouve une complexité amortie en $n^2$. À rédiger.
	\item Au lieu de diviser quand $r < n / 2$, mais quand $r < n / 4$.
\end{enumerate}

		\section{Barres en triangle}


On note $a(x)$ le côté du triangule équilatéral, et donc $x = a(x) \sqrt{3}  / 2$.

On calcule le flux $\Phi$ magnétique : \[
	\Phi = B \frac{a x}{2} = B x^2 / \sqrt{3}
.\]
Ainsi, d'après la loi de \textsc{Faraday}, on a \[
	e = - \frac{\mathrm{d}\Phi}{\mathrm{d}t} = - \frac{2B}{\sqrt{3}}  \: x\,\dot{x}
.\]
Or, par loi d'\textsc{Ohm}, $i = e / R(x)$.
Et, on connait la resistance du circuit $R(x) = 3 a(x) / \gamma S$.
Alors, 
\begin{align*}
	i(x) &= \frac{-2B\,x\,\dot{x}}{\sqrt{3} \cdot \frac{3a(x)}{\gamma S}} = - \frac{2B \gamma S}{3 \sqrt{3}} \cdot \frac{x\,\dot{x}}{2 \frac{x}{\sqrt{3}}}\\
	&= - B \gamma S \dot{x} / 3. \\
\end{align*}
On calcule donc la force de \textsc{Laplace} :
\begin{align*}
	\vec{F}_{\mathcal{L}} &= i \cdot [\mathrm{CD}] \cdot B \cdot \vec{e}_x\\
	&= i \cdot \frac{2\dot{x}}{\sqrt{3}} B \vec{e}_x \\
	&= -\underbrace{\frac{2 B^2 \gamma S}{3\sqrt{3}}}_\alpha  \cdot x \dot{x} \\
\end{align*}

D'après le \textsc{pfd}, on a donc \[
	m \ddot{x} = - \alpha x \dot{x} \text{ d'où }  \ddot{x} = - \frac{\alpha}{m} x\dot{x} 
.\] où $m = \rho S L$. 

On intègre les deux côtés de l'équation, \[
	[\dot{x}]_0^{t_\mathrm{f}} = -\frac{\alpha}{m} \cdot \left[ \frac{x^2}{2} \right]_0^{t_\mathrm{f}}
.\]
D'où, \[
	0 - v_0 = \frac{-\alpha}{m} \cdot x_\mathrm{f}^2 / 2
.\] On en conclut \[
	x_\mathrm{f} = \sqrt{\frac{2m}{\alpha} v_0} 
.\] 


		\begin{multicols}{2}
	\section{Implications}
	\begin{enumerate}
		\item
			\[
				\begin{prooftree}
					\infer 0[Ax]{p,q \vdash q}
					\infer 1[$\to$i]{q \vdash p \to q}
				\end{prooftree}
			\]
		\item
			\[
				\begin{prooftree}
					\infer 0[Ax]{p, p\land q\vdash p\land q}
					\infer 1[$\land$e,d]{p, p \land q \vdash q}
					\infer 1[$\to$i]{p \land q \vdash p \to q}
				\end{prooftree}
			\]
		\item 
			\[
				\begin{prooftree}
					\infer 0[Ax]{p, p\to q \vdash p}
					\infer 0[Ax]{p, p\to q \vdash p \to q}
					\infer 0[Ax]{p, p\to q \vdash p}
					\infer 2[$\to$e]{p, p \to q \vdash q}
					\infer 2[$\land$i]{p, p \to q \vdash p \land q}
					\infer 1[$\to$i]{p \to q \vdash p\to (p \land q)}
				\end{prooftree}
			\]
		\item
			\[
				\begin{prooftree}
					\infer 0[Ax]{\lnot q, p\to q, p \vdash p \to q}
					\infer 0[Ax]{\lnot q, p\to q, p \vdash p}
					\infer 2[$\to$e]{\lnot q, p\to q, p \vdash q}
					\infer 0[Ax]{\lnot q, p\to q, p \vdash \lnot q}
					\infer 2[$\lnot$e]{\lnot q, p\to q, p \vdash \bot}
					\infer 1[$\lnot$i]{\lnot q, p\to q \vdash\lnot p}
					\infer 1[$\to$i]{p \to q \vdash \lnot q \to \lnot p}
				\end{prooftree}
			\]
		\item 
			\[
				\begin{prooftree}
					\infer 0[Ax]{p \land q, p\to r \vdash p \land q}
					\infer 1[$\land$e,g]{p \land q, p\to r \vdash p}
					\infer 0[Ax]{p \land q, p\to r \vdash p \to r}
					\infer 2[$\to$e]{p \land q, p \to r \vdash r}
					\infer 1[$\to$i]{p \to r \vdash (p \land q) \to r}
				\end{prooftree}
			\] 
		\item
			\[
				\begin{prooftree}
					\infer 0[Ax]{p,q\vdash p}
					\infer 1[$\to$i]{p \vdash q \to p}
					\infer 1[$\to$i]{\vdash p \to (q \to p)}
				\end{prooftree}
			\]
		\item
			\[
				\begin{prooftree}
					\infer 0[Ax]{p \to q, p \vdash p\to q}
					\infer 0[Ax]{p\to q,p \vdash p}
					\infer 2[$\to$e]{p, p\to q \vdash q}
					\infer 1[$\to$i]{p \vdash (p \to q) \to q}
				\end{prooftree}
			\] 
	\end{enumerate}
\end{multicols}

		\section{Langage de \textsc{Dyck}}

\begin{enumerate}
	\item On suppose ce langage reconnaissable par un automate à $n$ états. On considère le mot $w = {\red(}^n \cdot {\red)}^n$, donc $|w| \ge n$.
		Ainsi, il existe $x$, $y$ et $z$ trois mots tels que $w = xyz$, $|xy| \le n$, $y \neq \varepsilon$ et $\forall p \in \N,\: x y^p z \in \mathcal{L}(\mathcal{G})$.
		Soit alors $p \in \llbracket 1,n-1 \rrbracket$ et $q \in \llbracket 1,n -p \rrbracket$ tels que $x = {\red(}^p$, $y = {\red(}^q$ et $z = {\red(}^{n-q-p} \cdot {\red)}^n$.
		Ainsi, $xy \in \mathcal{L}(\mathcal{G})$, ce qui est absurde. On en déduit que $\mathcal{L}(\mathcal{G})$ n'est pas reconnaissable, il n'est donc pas régulier.
	\item On pose $\mathcal{G} = (\Sigma, \{\mathrm{S}\}, \{\mathrm{S} \to \red( \mathrm{S}\red)  \mid \mathrm{SS}  \mid \varepsilon\}, \mathrm{S})$.
	\item
		\begin{itemize}
			\item On le montre par induction.
				\begin{itemize}
					\item \textbf{Cas $\mathrm{S} \to \varepsilon$.}
						On a $|\varepsilon|_{\red(} = 0 = |\varepsilon|_{\red)}$.
					\item \textbf{Cas $\mathrm{S} \to \red( \mathrm{S}\red)$.}
						Soit $u \in \mathcal{L}(\mathcal{G})$ avec $|u|_{\red(} = |u|_{\red)} = n$. Ainsi, $|\red(u\red)|_{\red(} = |\red(u\red)|_{\red)} = n + 1$.
					\item \textbf{Cas $\mathrm{S}\to \mathrm{SS}$.}
						Soient $u$ et $v$ deux mots de $\mathcal{L}(\mathcal{G})$ tels que $|u|_{\red(} = |u|_{\red)} = n$ et $|v|_{\red(} = |v|_{\red)} = m$.
						Alors, $|u\cdot v|_{\red(} = |v\cdot u|_{\red)} = n + m$.
				\end{itemize}
			\item Montrons par induction $\mathcal{P}_{u}$ : \guillemotleft~pour tout $v$ préfixe de $u$, $|v|_{\red(} \ge |v|_{\red)}$.~\guillemotright\@ 
				\begin{itemize}
					\item \textbf{Cas $\mathrm{S} \to \varepsilon$.}
						Le seul préfixe de $\varepsilon$ est $\varepsilon$, et on a bien $|\varepsilon|_{\red(} = 0 \ge 0 = |\varepsilon|_{\red)}$.
					\item \textbf{Cas $\mathrm{S}\to \red( \mathrm{S} \red)$.}
						Soit $u$ un mot de $\mathcal{L}(\mathcal{G})$ vérifiant $\mathcal{P}_u$.
						Soit $v$ un préfixe de $\red(u\red)$.
						On procède par induction sur $v$.
						\begin{itemize}
							\item \textbf{Cas $v = \varepsilon$ ou $\red($.} \textsc{ok}.
							\item \textbf{Cas $\red(\tilde{u}$,} où $\tilde{u}$ est un préfixe de $u$.
								Par hypothèse d'induction, $|\tilde{u}|_{\red(} \ge |\tilde{u}|_{\red)}$ donc $|\red(\tilde{u}|_{\red(} = |\red(\tilde{u}|_{\red)}$.
							\item \textbf{Cas $\red(u\red)$.} Par hypothèse d'induction, $|u|_{\red(} \ge |u|_{\red)}$ donc $|\red(u\red)|_{\red(}\ge |\red(u\red)|_{\red)}$.
						\end{itemize}
				\end{itemize}
		\end{itemize}
	\item On note $\overline w^j= \big| w_{\llbracket 0,j \rrbracket} \big|_{\red(} - \big| w_{\llbracket 0,j \rrbracket} \big|_{\red)}$.
		Alors les deux conditions se traduisent par $\overline w^{|w|} = 0$ et $\forall i \in \llbracket 0,|w|-1 \rrbracket$, $\overline w^i \ge 0$.
\end{enumerate}

		\section{$\mathcal{N}$}

\newcommand{\ofact}{\charfusion[\mathbin]{\bigcirc}{\scriptstyle!}}

\begin{enumerate}
	\item On définit par induction la fonction suivante \begin{align*}
			\oplus: \mathcal{N}^2 &\longrightarrow \mathcal{N} \\
			(\mathbf{S}(x),y) &\longmapsto \oplus(x, \mathbf{S}(y))\\
			(\mathbf{0}, x) &\longmapsto x.
		\end{align*}
	\item Soit $(x,y) \in \mathcal{N}^2$.
		\begin{itemize}
			\item Si $f(x) = 0$, alors $\oplus(x,y) = y$\/ et donc $f(\oplus(x,y)) = f(y) = f(x) + f(y)$.
			\item Si $f(x) \ge 1$, alors  $x = \mathbf{S}(z)$\/ avec $z \in \mathcal{N}$. Ainsi, $\oplus(x,y) = \oplus(z, \mathbf{S}(y))$. Or, $f(z) = f(x) - 1 \le f(x)$. Et donc, par définition de $\oplus$\/ puis par hypothèse d'induction, on a $f(\oplus(x,y)) = f(\oplus(z, \mathbf{S}(y)) = f(z) + f(\mathbf{S}(y))$. On en déduit que $f(\oplus(x,y)) = f(x) - 1 + f(y) + 1 = f(x) + f(y)$.
		\end{itemize}
		Par induction, on a bien $\forall (x,y) \in \mathcal{N}^2,\:f\big({\oplus}(x,y)\big) = f(x) + f(y)$.
	\item On définit par induction la fonction suivante \begin{align*}
			\otimes: \mathcal{N}^2 &\longrightarrow \mathcal{N} \\
			(\mathbf{S}(x), y) &\longmapsto {\oplus}\big(y, {\otimes}(x,y)\big)\\
			(\mathbf{0}, y) &\longmapsto \mathbf{0}.
		\end{align*}
	\item Soit $(x,y) \in \mathcal{N}^2$.
		\begin{itemize}
			\item Si $f(x) = 0$, alors $\otimes(x,y) = \mathbf{0}$, et donc $f(\otimes(x,y)) = 0 = f(x) \times f(y)$.
			\item Si $f(x) \ge 1$, alors $x = \mathbf{S}(z)$\/ avec $z \in \mathcal{N}$. Ainsi, par définition de $\otimes$, on a $\otimes(x,y) = \oplus(y, \otimes(z,y))$. Or, par hypothèse d'induction, $f(\otimes(z,y)) = f(z) \times f(y)$\/ (car $f(z) < f(x)$), et donc $f(\otimes(x,y)) = f(y) + f(\otimes(z,y)) = f(y) + f(z) \times f(y) = f(y) \times (1 + f(z)) = f(y) \times f(x)$.
		\end{itemize}
		Par induction, on a bien $\forall (x,y) \in \mathcal{N}^2,\:f\big({\otimes}(x,y)\big) = f(x) \times f(y)$.
	\item On définit par induction la fonction suivante \begin{align*}
			\ofact : \mathcal{N} &\longrightarrow \mathcal{N} \\
			\mathbf{0} &\longmapsto \mathbf{S}(\mathbf{0})\\
			\mathbf{S}(x) &\longmapsto {\otimes}\big(\mathbf{S}(x), \ofact(x)\big).
		\end{align*}
	\item Soit $x \in \mathcal{N}$.
		\begin{itemize}
			\item Si $f(x)= 0$, alors $\ofact(x) = \mathbf{S}(\mathbf{0})$\/ par définition, et donc $f(\ofact(x)) = 1 = 0! = f(x) !\:$.
			\item Si $f(x) \ge 1$, alors $x = \mathbf{S}(z)$\/ avec $z \in \mathcal{N}$. Ainsi, par définition de $\ofact$, on a $\ofact(x) = \otimes(x, \ofact(z))$, et donc, par hypothèse de récurrence, $f(\ofact(x)) = f(x) \times f(\ofact(z)) = f(x) \times \big(f(z)!\big)$. Or, comme $f(z) = f(x) - 1$, on a donc $f(\ofact(x)) = f(x) \times \big(f(x) - 1\big)! = f(x)!$\:.
		\end{itemize}
		Par induction, on a bien $\forall x \in \mathcal{N},\:f\big({\ofact}(x)\big) = f(x)!\:$.
\end{enumerate}

		\documentclass[a4paper]{article}

\usepackage[margin=1in]{geometry}
\usepackage[utf8]{inputenc}
\usepackage[T1]{fontenc}
\usepackage{mathrsfs}
\usepackage{textcomp}
\usepackage[french]{babel}
\usepackage{amsmath}
\usepackage{amssymb}
\usepackage{cancel}
\usepackage{frcursive}
\usepackage[inline]{asymptote}
\usepackage{tikz}
\usepackage[european,straightvoltages,europeanresistors]{circuitikz}
\usepackage{tikz-cd}
\usepackage{tkz-tab}
\usepackage[b]{esvect}
\usepackage[framemethod=TikZ]{mdframed}
\usepackage{centernot}
\usepackage{diagbox}
\usepackage{dsfont}
\usepackage{fancyhdr}
\usepackage{float}
\usepackage{graphicx}
\usepackage{listings}
\usepackage{multicol}
\usepackage{nicematrix}
\usepackage{pdflscape}
\usepackage{stmaryrd}
\usepackage{xfrac}
\usepackage{hep-math-font}
\usepackage{amsthm}
\usepackage{thmtools}
\usepackage{indentfirst}
\usepackage[framemethod=TikZ]{mdframed}
\usepackage{accents}
\usepackage{soulutf8}
\usepackage{mathtools}
\usepackage{bodegraph}
\usepackage{slashbox}
\usepackage{enumitem}
\usepackage{calligra}
\usepackage{cinzel}
\usepackage{BOONDOX-calo}

% Tikz
\usetikzlibrary{babel}
\usetikzlibrary{positioning}
\usetikzlibrary{calc}

% global settings
\frenchspacing
\reversemarginpar
\setuldepth{a}

%\everymath{\displaystyle}

\frenchbsetup{StandardLists=true}

\def\asydir{asy}

%\sisetup{exponent-product=\cdot,output-decimal-marker={,},separate-uncertainty,range-phrase=\;à\;,locale=FR}

\setlength{\parskip}{1em}

\theoremstyle{definition}

% Changing math
\let\emptyset\varnothing
\let\ge\geqslant
\let\le\leqslant
\let\preceq\preccurlyeq
\let\succeq\succcurlyeq
\let\ds\displaystyle
\let\ts\textstyle

\newcommand{\C}{\mathds{C}}
\newcommand{\R}{\mathds{R}}
\newcommand{\Z}{\mathds{Z}}
\newcommand{\N}{\mathds{N}}
\newcommand{\Q}{\mathds{Q}}

\renewcommand{\O}{\emptyset}

\newcommand\ubar[1]{\underaccent{\bar}{#1}}

\renewcommand\Re{\expandafter\mathfrak{Re}}
\renewcommand\Im{\expandafter\mathfrak{Im}}

\let\slantedpartial\partial
\DeclareRobustCommand{\partial}{\text{\rotatebox[origin=t]{20}{\scalebox{0.95}[1]{$\slantedpartial$}}}\hspace{-1pt}}

% merging two maths characters w/ \charfusion
\makeatletter
\def\moverlay{\mathpalette\mov@rlay}
\def\mov@rlay#1#2{\leavevmode\vtop{%
   \baselineskip\z@skip \lineskiplimit-\maxdimen
   \ialign{\hfil$\m@th#1##$\hfil\cr#2\crcr}}}
\newcommand{\charfusion}[3][\mathord]{
    #1{\ifx#1\mathop\vphantom{#2}\fi
        \mathpalette\mov@rlay{#2\cr#3}
      }
    \ifx#1\mathop\expandafter\displaylimits\fi}
\makeatother

% custom math commands
\newcommand{\T}{{\!\!\,\top}}
\newcommand{\avrt}[1]{\rotatebox{-90}{$#1$}}
\newcommand{\bigcupdot}{\charfusion[\mathop]{\bigcup}{\cdot}}
\newcommand{\cupdot}{\charfusion[\mathbin]{\cup}{\cdot}}
%\newcommand{\danger}{{\large\fontencoding{U}\fontfamily{futs}\selectfont\char 66\relax}\;}
\newcommand{\tendsto}[1]{\xrightarrow[#1]{}}
\newcommand{\vrt}[1]{\rotatebox{90}{$#1$}}
\newcommand{\tsup}[1]{\textsuperscript{\underline{#1}}}
\newcommand{\tsub}[1]{\textsubscript{#1}}

\renewcommand{\mod}[1]{~\left[ #1 \right]}
\renewcommand{\t}{{}^t\!}
\newcommand{\s}{\text{\calligra s}}

% custom units / constants
%\DeclareSIUnit{\litre}{\ell}
\let\hbar\hslash

% header / footer
\pagestyle{fancy}
\fancyhead{} \fancyfoot{}
\fancyfoot[C]{\thepage}

% fonts
\let\sc\scshape
\let\bf\bfseries
\let\it\itshape
\let\sl\slshape

% custom math operators
\let\th\relax
\let\det\relax
\DeclareMathOperator*{\codim}{codim}
\DeclareMathOperator*{\dom}{dom}
\DeclareMathOperator*{\gO}{O}
\DeclareMathOperator*{\po}{\text{\cursive o}}
\DeclareMathOperator*{\sgn}{sgn}
\DeclareMathOperator*{\simi}{\sim}
\DeclareMathOperator{\Arccos}{Arccos}
\DeclareMathOperator{\Arcsin}{Arcsin}
\DeclareMathOperator{\Arctan}{Arctan}
\DeclareMathOperator{\Argsh}{Argsh}
\DeclareMathOperator{\Arg}{Arg}
\DeclareMathOperator{\Aut}{Aut}
\DeclareMathOperator{\Card}{Card}
\DeclareMathOperator{\Cl}{\mathcal{C}\!\ell}
\DeclareMathOperator{\Cov}{Cov}
\DeclareMathOperator{\Ker}{Ker}
\DeclareMathOperator{\Mat}{Mat}
\DeclareMathOperator{\PGCD}{PGCD}
\DeclareMathOperator{\PPCM}{PPCM}
\DeclareMathOperator{\Supp}{Supp}
\DeclareMathOperator{\Vect}{Vect}
\DeclareMathOperator{\argmax}{argmax}
\DeclareMathOperator{\argmin}{argmin}
\DeclareMathOperator{\ch}{ch}
\DeclareMathOperator{\com}{com}
\DeclareMathOperator{\cotan}{cotan}
\DeclareMathOperator{\det}{det}
\DeclareMathOperator{\id}{id}
\DeclareMathOperator{\rg}{rg}
\DeclareMathOperator{\rk}{rk}
\DeclareMathOperator{\sh}{sh}
\DeclareMathOperator{\th}{th}
\DeclareMathOperator{\tr}{tr}

% colors and page style
\definecolor{truewhite}{HTML}{ffffff}
\definecolor{white}{HTML}{faf4ed}
\definecolor{trueblack}{HTML}{000000}
\definecolor{black}{HTML}{575279}
\definecolor{mauve}{HTML}{907aa9}
\definecolor{blue}{HTML}{286983}
\definecolor{red}{HTML}{d7827e}
\definecolor{yellow}{HTML}{ea9d34}
\definecolor{gray}{HTML}{9893a5}
\definecolor{grey}{HTML}{9893a5}
\definecolor{green}{HTML}{a0d971}

\pagecolor{white}
\color{black}

\begin{asydef}
	settings.prc = false;
	settings.render=0;

	white = rgb("faf4ed");
	black = rgb("575279");
	blue = rgb("286983");
	red = rgb("d7827e");
	yellow = rgb("f6c177");
	orange = rgb("ea9d34");
	gray = rgb("9893a5");
	grey = rgb("9893a5");
	deepcyan = rgb("56949f");
	pink = rgb("b4637a");
	magenta = rgb("eb6f92");
	green = rgb("a0d971");
	purple = rgb("907aa9");

	defaultpen(black + fontsize(8pt));

	import three;
	currentlight = nolight;
\end{asydef}

% theorems, proofs, ...

\mdfsetup{skipabove=1em,skipbelow=1em, innertopmargin=6pt, innerbottommargin=6pt,}

\declaretheoremstyle[
	headfont=\normalfont\itshape,
	numbered=no,
	postheadspace=\newline,
	headpunct={:},
	qed=\qedsymbol]{demstyle}

\declaretheorem[style=demstyle, name=Démonstration]{dem}

\newcommand\veczero{\kern-1.2pt\vec{\kern1.2pt 0}} % \vec{0} looks weird since the `0' isn't italicized

\makeatletter
\renewcommand{\title}[2]{
	\AtBeginDocument{
		\begin{titlepage}
			\begin{center}
				\vspace{10cm}
				{\Large \sc Chapitre #1}\\
				\vspace{1cm}
				{\Huge \calligra #2}\\
				\vfill
				Hugo {\sc Salou} MPI${}^{\star}$\\
				{\small Dernière mise à jour le \@date }
			\end{center}
		\end{titlepage}
	}
}

\newcommand{\titletp}[4]{
	\AtBeginDocument{
		\begin{titlepage}
			\begin{center}
				\vspace{10cm}
				{\Large \sc tp #1}\\
				\vspace{1cm}
				{\Huge \textsc{\textit{#2}}}\\
				\vfill
				{#3}\textit{MPI}${}^{\star}$\\
			\end{center}
		\end{titlepage}
	}
	\fancyfoot{}\fancyhead{}
	\fancyfoot[R]{#4 \textit{MPI}${}^{\star}$}
	\fancyhead[C]{{\sc tp #1} : #2}
	\fancyhead[R]{\thepage}
}

\newcommand{\titletd}[2]{
	\AtBeginDocument{
		\begin{titlepage}
			\begin{center}
				\vspace{10cm}
				{\Large \sc td #1}\\
				\vspace{1cm}
				{\Huge \calligra #2}\\
				\vfill
				Hugo {\sc Salou} MPI${}^{\star}$\\
				{\small Dernière mise à jour le \@date }
			\end{center}
		\end{titlepage}
	}
}
\makeatother

\newcommand{\sign}{
	\null
	\vfill
	\begin{center}
		{
			\fontfamily{ccr}\selectfont
			\textit{\textbf{\.{\"i}}}
		}
	\end{center}
	\vfill
	\null
}

\renewcommand{\thefootnote}{\emph{\alph{footnote}}}

% figure support
\usepackage{import}
\usepackage{xifthen}
\pdfminorversion=7
\usepackage{pdfpages}
\usepackage{transparent}
\newcommand{\incfig}[1]{%
	\def\svgwidth{\columnwidth}
	\import{./figures/}{#1.pdf_tex}
}

\pdfsuppresswarningpagegroup=1
\ctikzset{tripoles/european not symbol=circle}

\newcommand{\missingpart}{{\large\color{red} Il manque quelque chose ici\ldots}}


\fancyhead[R]{Hugo {\sc Salou}\/ MPI}
\fancyhead[L]{TD\textsubscript4 -- Exercice 8}

\begin{document}
	\let\thesection\relax
	
\begin{comment}
\section{Exercice 9}

\slshape
Soit la matrice $A = \begin{pmatrix}
	1&1&-1\\
	2&3&-4\\
	4&1&-4
\end{pmatrix}$.
\begin{enumerate}
	\item Déterminer le spectre de la matrice $A$\/ et trouver une matrice $P$\/ inversible telle que $P^{-1} A P$\/ est diagonale.
	\item Soit $B$\/ une matrice de taille $3\times 3$\/ qui commentent avec $A$\/ ($AB = BA$). Montrer que $B$\/ est diagonale.
\end{enumerate}
\upshape

\begin{enumerate}
	\item On sait, tout d'abord, que, pour $x \in \R$,
		\begin{align*}
			\chi_A(x) = \det(x\,I_n - A) &=
			\begin{vmatrix}
				x - 1 &- 1 & 1\\
				-2&x-3&4\\
				-4&-1&x + 4
			\end{vmatrix}\\
			&= \\
		\end{align*}
\end{enumerate}
\end{comment}

\section{Exercice 8}

\begin{enumerate}
	\item Soit un vecteur non nul $\vec{x} \in \Ker(\lambda {\id} - {u \circ v})$. Ainsi, $u(v(\vec{x})) = \lambda \vec{x}$. Et, donc $v(u(v(\vec{x}))) = \lambda v(\vec{x})$. On a donc $v(\vec{x}) \in \Ker(\lambda {\id} - {v  \circ u})$.
		Or, si $\lambda \neq 0$, on a $v(\vec{x}) \neq \vec{0}$\/ ; en effet, si $v(\vec{x}) = \vec{0}$, alors $u \circ v(\vec{x}) = \vec{0} = \lambda \vec{x}$\/ et donc $\vec{x} = \vec{0}$, ce ne serait donc pas un vecteur propre de $u \circ v$\/ : une contradiction. On en déduit que $v(\vec{x})$\/ est un vecteur propre de $u \circ v$\/ associé à la valeur propre $\lambda$.
	\item On pose donc $\lambda = 0$, une valeur propre de $u  \circ v$. L'endomorphisme $u \circ v$\/ n'est donc pas injectif, donc bijectif. On sait donc, comme $E$\/ est de dimension finie, que $\det(u \circ v) = 0$. Or $\det (u \circ v) = \det u \times \det v = \det(v  \circ u)$. Et donc $\det(v  \circ u) = 0$, $v  \circ u$\/ n'est donc pas bijectif, donc injectif. Et donc, on a $0 \in \Sp(v  \circ u)$.
	\item Soit $P \in \R[X]$, et soit $Q$\/ une primitive de $P$.
		\begin{align*}
			P \in \Ker (u  \circ v) \iff& \Big(\int_{0}^X P(t)~\mathrm{d}t\Big)' = 0\\
			\iff& \big(Q(X) - Q(0)\big)' = 0\\
			\iff& Q'(X) = 0\\
			\iff& P(X) = 0
		\end{align*}
		On en déduit que $\Ker (u \circ v) = \{0\}$.

		Également,
		\begin{align*}
			P \in \Ker(v  \circ u) \iff& \int_{0}^{X} P'(t)~\mathrm{d}t = 0\\
			\iff& P(X) - P(0) = 0\\
			\iff& P(X) = P(0)\\
			\iff& \deg P \le 0\\
			\iff& P \in \R_0[X]
		\end{align*}
		On en déduit que $\Ker(v \circ u) = \R_0[X]$.
\end{enumerate}


\end{document}

	}
	\def\addmacros#1{#1}
}
{
	\td[2]{Logique propositionnelle}
	\minitoc
	\renewcommand{\cwd}{../td/td02/}
	\addmacros{
		\section{Quelques problèmes décidables}

\begin{enumerate}
	\item Soit $f : \R \to \R$.
		\begin{itemize}
			\item Si $f$\/ admet un zéro, on pose $\mathcal{M} = \texttt{fun}\ \texttt{s}\ \to \texttt{true}$.
			\item Si $f$\/ n'admet pas un zéro, on pose $\mathcal{M} = \texttt{fun}\ \texttt{s}\ \to \texttt{false}$.
		\end{itemize}
		Alors, $\mathcal{M}$\/ décide \textsc{Zero}$_f$.
	\item Soit $\mathcal{M}$\/ une machine, et soit $w \in \Sigma^*$.
		\begin{itemize}
			\item Si $\mathcal{M}$\/ se termine sur l'entrée $w$, alors on pose $\mathcal{M}' = \texttt{fun}\ \texttt{s} \to \texttt{true}$.
			\item Si $\mathcal{M}$\/ ne se termine pas sur l'entrée $w$, alors on pose $\mathcal{M}' = \texttt{fun}\ \texttt{s} \to \texttt{false}$.
		\end{itemize}
		Alors, $\mathcal{M}'$\/ décide \textsc{Arrêt}$_{\mathcal{M},w}$.
	\item Le problème est trivialement vrai. En effet, soit $M \in \mathcal{O}$, de la forme
		\begin{lstlisting}[language=caml]
let m (s: string): string =
	%*$\langle$\textrm{code}$\rangle$*) 
		\end{lstlisting}
		On crée la machine $\mathcal{N}$\/ ci-dessous.
		\begin{lstlisting}[language=caml]
let n (s: string): string =
	if true then
		%*$\langle$\textrm{code}$\rangle$*) 
	else
		%*$\langle$\textrm{code}$\rangle$*) 
		\end{lstlisting}
		On a $\texttt{m} \neq \texttt{n}$, mais $\mathcal{L}(\texttt{m}) = \mathcal{L}(\texttt{n})$, donc le problème est vrai sur toute entrée et la fonction $\texttt{fun}\ \texttt{s} \to \texttt{true}$\/ répond au problème.
\end{enumerate}

		\section{Ensembles définis inductivement}

La correction est disponible sur \textit{cahier-de-prepa}.

\begin{comment}
	\begin{exm}
		Avec $S = \N$, $\mathcal{B} = \{0, 2\} $, $A_1 = \{0\}$\/ et \begin{align*}
			f_1: A_1 \times \N &\longrightarrow \N \\
			(0, x) &\longmapsto x + 4.
		\end{align*}

		On a \[
			X \supseteq \{0, 2, 4, 6, 8, 10, \ldots, 20, \ldots\} = 2\N
		.\]
	\end{exm}
	\begin{exm}
		Avec $S$\/ l'ensemble des langages sur $\Sigma$, $\mathcal{B} = \{\O\} \cup \bigl\{\{a\}\:\big|\: a \in \Sigma \bigr\}$, et
		\begin{multicols}{3}
			\begin{align*}
				f_1: S \times S &\longrightarrow S \\
				(L_1, L_2) &\longmapsto L_1 \cup L_2,
			\end{align*}
			\begin{align*}
				f_2: S \times S &\longrightarrow S \\
				(L_1, L_2) &\longmapsto L_1 \cdot L_2,
			\end{align*}
			\begin{align*}
				f_3: S &\longrightarrow S \\
				L &\longmapsto L^*.
			\end{align*}
		\end{multicols}
	\end{exm}

\begin{enumerate}
	\item Soit $\mathcal{A} = \{X \subseteq S  \mid X \supseteq \mathcal{B} \mathrel{\text{et}} X \text{ est stable par } f_i\}$. On a $S \in \mathcal{A}$\/ et donc $\mathcal{A} \neq \O$. De plus, soit \[
			Y = \{x \in S  \mid \forall X \in \mathcal{A},\,x \in X\} = \bigcap_{X \in \mathcal{A}} X
		.\]
		Soit $b \in \mathcal{B}$, on a $\forall X \in A,\, b \in X$. D'où $b \in Y$\/ par intersection. On en déduit que $\mathcal{B} \subseteq Y$.

		Soit $i \in \left\llbracket 1,m \right\rrbracket$. Soit $(x_1, \ldots, x_{n_i}) \in Y^{n_i}$\/ et soit $a \in A_i$. Montrons que $f_i(a, x_1, \ldots, x_{n_i}) \in Y$.
		Or, soit $X \in \mathcal{A}$, on a $(x_1, \ldots, x_{n_i}) \in X^{n_i}$\/ donc $f_i(a, x_1, \ldots, x_{n_i}) \in X$. Ceci étant vrai pour tout $X \in \mathcal{A}$, on a $f_i(a, x_1, \ldots, x_{n_i}) \in Y$\/ donc $Y$\/ est stable par $f_i$\/ par tout $i \in \left\llbracket 1,m \right\rrbracket$\/ et donc $Y \in \mathcal{A}$.
		On a également $Y \subseteq X$\/ pour tout $X \in \mathcal{A}$. On en déduit que $Y$\/ est le plus petit élément (pour l'inclusion) de $\mathcal{A}$.
	\item On pose $X_0  = \mathcal{B}$\/ et \[
			X_{n+1} = X_n \cup \big\{ f_i(a, x_1, \ldots, x_{n_i})  \mid a \in A_i,\,(x_1, \ldots, x_{n_i}) \in (X_n)^{n_i},\,i \in \left\llbracket 1,m \right\rrbracket\big\}
		.\]
		Soit $X = \bigcup_{n \in \N} X_n$. Soit $Y$\/ l'ensemble défini par induction à partir de $\mathcal{B}$\/ et des $(f_i)_{i\in\left\llbracket 1,n \right\rrbracket}$. Montrons que $X = Y$.
		On montre que $X$\/ est le plus petit élément (pour l'inclusion) de $\mathcal{A}$\/ et on conclut par unicité du minimum (avec la question précédente).
		Par définition de la suite $(X_n)_{n\in\N}$, elle est croissante (au sens de l'inclusion).
		Montrons à présent, par récurrence, la propriété ci-dessous : $P_n : ``X_n \subseteq Y."$
		\begin{itemize}
			\item Par définition de $Y$, on a $X_0 = \mathcal{B} \subseteq Y$.
			\item Soit 
		\end{itemize}
\end{enumerate}

\subsection{Un théorème d'induction}

\begin{enumerate}
	\item[3.] Soit $Z = \{x \in S  \mid P(x) \text{ vraie}\:\}$.
		Montrons que $\mathcal{X}\subseteq Z$.
		On remarque que $\mathcal{X} \supseteq \mathcal{B}$\/ ; $\mathcal{X}$\/ est stable par $f_i$. On en conclut que $Z \supseteq \mathcal{X}$ et donc $\forall x \in \mathcal{X},\,P(x)$\/ est vraie.
\end{enumerate}

\begin{exm}
	Soit $\mathcal{X}$\/ défini par induction par $\mathcal{B} = \{0, 2\}$\/ et \begin{align*}
		f: \N &\longrightarrow \N \\
		n &\longmapsto n + 2.
	\end{align*}
	Montrons que $\forall n \in \mathcal{X}$, $x$\/ est pair.

	On sait que $0$\/ est pair, $2$\/ est pair ; et, \[
		\forall x,y \in \mathcal{X},\, (x \text{ pair} \land y \text{ pair}) \implies f(x, y) \text{ pair}
	.\]

	On en déduit que \[
		\forall n \in \mathcal{X},\,x \text{ est pair}.
	.\]
\end{exm}
\end{comment}

		\section{Tableaux dynamiques}

\begin{enumerate}[start=3]
	\item On trouve une complexité amortie en $n^2$. À rédiger.
	\item Au lieu de diviser quand $r < n / 2$, mais quand $r < n / 4$.
\end{enumerate}

		\section{Barres en triangle}


On note $a(x)$ le côté du triangule équilatéral, et donc $x = a(x) \sqrt{3}  / 2$.

On calcule le flux $\Phi$ magnétique : \[
	\Phi = B \frac{a x}{2} = B x^2 / \sqrt{3}
.\]
Ainsi, d'après la loi de \textsc{Faraday}, on a \[
	e = - \frac{\mathrm{d}\Phi}{\mathrm{d}t} = - \frac{2B}{\sqrt{3}}  \: x\,\dot{x}
.\]
Or, par loi d'\textsc{Ohm}, $i = e / R(x)$.
Et, on connait la resistance du circuit $R(x) = 3 a(x) / \gamma S$.
Alors, 
\begin{align*}
	i(x) &= \frac{-2B\,x\,\dot{x}}{\sqrt{3} \cdot \frac{3a(x)}{\gamma S}} = - \frac{2B \gamma S}{3 \sqrt{3}} \cdot \frac{x\,\dot{x}}{2 \frac{x}{\sqrt{3}}}\\
	&= - B \gamma S \dot{x} / 3. \\
\end{align*}
On calcule donc la force de \textsc{Laplace} :
\begin{align*}
	\vec{F}_{\mathcal{L}} &= i \cdot [\mathrm{CD}] \cdot B \cdot \vec{e}_x\\
	&= i \cdot \frac{2\dot{x}}{\sqrt{3}} B \vec{e}_x \\
	&= -\underbrace{\frac{2 B^2 \gamma S}{3\sqrt{3}}}_\alpha  \cdot x \dot{x} \\
\end{align*}

D'après le \textsc{pfd}, on a donc \[
	m \ddot{x} = - \alpha x \dot{x} \text{ d'où }  \ddot{x} = - \frac{\alpha}{m} x\dot{x} 
.\] où $m = \rho S L$. 

On intègre les deux côtés de l'équation, \[
	[\dot{x}]_0^{t_\mathrm{f}} = -\frac{\alpha}{m} \cdot \left[ \frac{x^2}{2} \right]_0^{t_\mathrm{f}}
.\]
D'où, \[
	0 - v_0 = \frac{-\alpha}{m} \cdot x_\mathrm{f}^2 / 2
.\] On en conclut \[
	x_\mathrm{f} = \sqrt{\frac{2m}{\alpha} v_0} 
.\] 


		\begin{multicols}{2}
	\section{Implications}
	\begin{enumerate}
		\item
			\[
				\begin{prooftree}
					\infer 0[Ax]{p,q \vdash q}
					\infer 1[$\to$i]{q \vdash p \to q}
				\end{prooftree}
			\]
		\item
			\[
				\begin{prooftree}
					\infer 0[Ax]{p, p\land q\vdash p\land q}
					\infer 1[$\land$e,d]{p, p \land q \vdash q}
					\infer 1[$\to$i]{p \land q \vdash p \to q}
				\end{prooftree}
			\]
		\item 
			\[
				\begin{prooftree}
					\infer 0[Ax]{p, p\to q \vdash p}
					\infer 0[Ax]{p, p\to q \vdash p \to q}
					\infer 0[Ax]{p, p\to q \vdash p}
					\infer 2[$\to$e]{p, p \to q \vdash q}
					\infer 2[$\land$i]{p, p \to q \vdash p \land q}
					\infer 1[$\to$i]{p \to q \vdash p\to (p \land q)}
				\end{prooftree}
			\]
		\item
			\[
				\begin{prooftree}
					\infer 0[Ax]{\lnot q, p\to q, p \vdash p \to q}
					\infer 0[Ax]{\lnot q, p\to q, p \vdash p}
					\infer 2[$\to$e]{\lnot q, p\to q, p \vdash q}
					\infer 0[Ax]{\lnot q, p\to q, p \vdash \lnot q}
					\infer 2[$\lnot$e]{\lnot q, p\to q, p \vdash \bot}
					\infer 1[$\lnot$i]{\lnot q, p\to q \vdash\lnot p}
					\infer 1[$\to$i]{p \to q \vdash \lnot q \to \lnot p}
				\end{prooftree}
			\]
		\item 
			\[
				\begin{prooftree}
					\infer 0[Ax]{p \land q, p\to r \vdash p \land q}
					\infer 1[$\land$e,g]{p \land q, p\to r \vdash p}
					\infer 0[Ax]{p \land q, p\to r \vdash p \to r}
					\infer 2[$\to$e]{p \land q, p \to r \vdash r}
					\infer 1[$\to$i]{p \to r \vdash (p \land q) \to r}
				\end{prooftree}
			\] 
		\item
			\[
				\begin{prooftree}
					\infer 0[Ax]{p,q\vdash p}
					\infer 1[$\to$i]{p \vdash q \to p}
					\infer 1[$\to$i]{\vdash p \to (q \to p)}
				\end{prooftree}
			\]
		\item
			\[
				\begin{prooftree}
					\infer 0[Ax]{p \to q, p \vdash p\to q}
					\infer 0[Ax]{p\to q,p \vdash p}
					\infer 2[$\to$e]{p, p\to q \vdash q}
					\infer 1[$\to$i]{p \vdash (p \to q) \to q}
				\end{prooftree}
			\] 
	\end{enumerate}
\end{multicols}

		\section{Langage de \textsc{Dyck}}

\begin{enumerate}
	\item On suppose ce langage reconnaissable par un automate à $n$ états. On considère le mot $w = {\red(}^n \cdot {\red)}^n$, donc $|w| \ge n$.
		Ainsi, il existe $x$, $y$ et $z$ trois mots tels que $w = xyz$, $|xy| \le n$, $y \neq \varepsilon$ et $\forall p \in \N,\: x y^p z \in \mathcal{L}(\mathcal{G})$.
		Soit alors $p \in \llbracket 1,n-1 \rrbracket$ et $q \in \llbracket 1,n -p \rrbracket$ tels que $x = {\red(}^p$, $y = {\red(}^q$ et $z = {\red(}^{n-q-p} \cdot {\red)}^n$.
		Ainsi, $xy \in \mathcal{L}(\mathcal{G})$, ce qui est absurde. On en déduit que $\mathcal{L}(\mathcal{G})$ n'est pas reconnaissable, il n'est donc pas régulier.
	\item On pose $\mathcal{G} = (\Sigma, \{\mathrm{S}\}, \{\mathrm{S} \to \red( \mathrm{S}\red)  \mid \mathrm{SS}  \mid \varepsilon\}, \mathrm{S})$.
	\item
		\begin{itemize}
			\item On le montre par induction.
				\begin{itemize}
					\item \textbf{Cas $\mathrm{S} \to \varepsilon$.}
						On a $|\varepsilon|_{\red(} = 0 = |\varepsilon|_{\red)}$.
					\item \textbf{Cas $\mathrm{S} \to \red( \mathrm{S}\red)$.}
						Soit $u \in \mathcal{L}(\mathcal{G})$ avec $|u|_{\red(} = |u|_{\red)} = n$. Ainsi, $|\red(u\red)|_{\red(} = |\red(u\red)|_{\red)} = n + 1$.
					\item \textbf{Cas $\mathrm{S}\to \mathrm{SS}$.}
						Soient $u$ et $v$ deux mots de $\mathcal{L}(\mathcal{G})$ tels que $|u|_{\red(} = |u|_{\red)} = n$ et $|v|_{\red(} = |v|_{\red)} = m$.
						Alors, $|u\cdot v|_{\red(} = |v\cdot u|_{\red)} = n + m$.
				\end{itemize}
			\item Montrons par induction $\mathcal{P}_{u}$ : \guillemotleft~pour tout $v$ préfixe de $u$, $|v|_{\red(} \ge |v|_{\red)}$.~\guillemotright\@ 
				\begin{itemize}
					\item \textbf{Cas $\mathrm{S} \to \varepsilon$.}
						Le seul préfixe de $\varepsilon$ est $\varepsilon$, et on a bien $|\varepsilon|_{\red(} = 0 \ge 0 = |\varepsilon|_{\red)}$.
					\item \textbf{Cas $\mathrm{S}\to \red( \mathrm{S} \red)$.}
						Soit $u$ un mot de $\mathcal{L}(\mathcal{G})$ vérifiant $\mathcal{P}_u$.
						Soit $v$ un préfixe de $\red(u\red)$.
						On procède par induction sur $v$.
						\begin{itemize}
							\item \textbf{Cas $v = \varepsilon$ ou $\red($.} \textsc{ok}.
							\item \textbf{Cas $\red(\tilde{u}$,} où $\tilde{u}$ est un préfixe de $u$.
								Par hypothèse d'induction, $|\tilde{u}|_{\red(} \ge |\tilde{u}|_{\red)}$ donc $|\red(\tilde{u}|_{\red(} = |\red(\tilde{u}|_{\red)}$.
							\item \textbf{Cas $\red(u\red)$.} Par hypothèse d'induction, $|u|_{\red(} \ge |u|_{\red)}$ donc $|\red(u\red)|_{\red(}\ge |\red(u\red)|_{\red)}$.
						\end{itemize}
				\end{itemize}
		\end{itemize}
	\item On note $\overline w^j= \big| w_{\llbracket 0,j \rrbracket} \big|_{\red(} - \big| w_{\llbracket 0,j \rrbracket} \big|_{\red)}$.
		Alors les deux conditions se traduisent par $\overline w^{|w|} = 0$ et $\forall i \in \llbracket 0,|w|-1 \rrbracket$, $\overline w^i \ge 0$.
\end{enumerate}

		\section{$\mathcal{N}$}

\newcommand{\ofact}{\charfusion[\mathbin]{\bigcirc}{\scriptstyle!}}

\begin{enumerate}
	\item On définit par induction la fonction suivante \begin{align*}
			\oplus: \mathcal{N}^2 &\longrightarrow \mathcal{N} \\
			(\mathbf{S}(x),y) &\longmapsto \oplus(x, \mathbf{S}(y))\\
			(\mathbf{0}, x) &\longmapsto x.
		\end{align*}
	\item Soit $(x,y) \in \mathcal{N}^2$.
		\begin{itemize}
			\item Si $f(x) = 0$, alors $\oplus(x,y) = y$\/ et donc $f(\oplus(x,y)) = f(y) = f(x) + f(y)$.
			\item Si $f(x) \ge 1$, alors  $x = \mathbf{S}(z)$\/ avec $z \in \mathcal{N}$. Ainsi, $\oplus(x,y) = \oplus(z, \mathbf{S}(y))$. Or, $f(z) = f(x) - 1 \le f(x)$. Et donc, par définition de $\oplus$\/ puis par hypothèse d'induction, on a $f(\oplus(x,y)) = f(\oplus(z, \mathbf{S}(y)) = f(z) + f(\mathbf{S}(y))$. On en déduit que $f(\oplus(x,y)) = f(x) - 1 + f(y) + 1 = f(x) + f(y)$.
		\end{itemize}
		Par induction, on a bien $\forall (x,y) \in \mathcal{N}^2,\:f\big({\oplus}(x,y)\big) = f(x) + f(y)$.
	\item On définit par induction la fonction suivante \begin{align*}
			\otimes: \mathcal{N}^2 &\longrightarrow \mathcal{N} \\
			(\mathbf{S}(x), y) &\longmapsto {\oplus}\big(y, {\otimes}(x,y)\big)\\
			(\mathbf{0}, y) &\longmapsto \mathbf{0}.
		\end{align*}
	\item Soit $(x,y) \in \mathcal{N}^2$.
		\begin{itemize}
			\item Si $f(x) = 0$, alors $\otimes(x,y) = \mathbf{0}$, et donc $f(\otimes(x,y)) = 0 = f(x) \times f(y)$.
			\item Si $f(x) \ge 1$, alors $x = \mathbf{S}(z)$\/ avec $z \in \mathcal{N}$. Ainsi, par définition de $\otimes$, on a $\otimes(x,y) = \oplus(y, \otimes(z,y))$. Or, par hypothèse d'induction, $f(\otimes(z,y)) = f(z) \times f(y)$\/ (car $f(z) < f(x)$), et donc $f(\otimes(x,y)) = f(y) + f(\otimes(z,y)) = f(y) + f(z) \times f(y) = f(y) \times (1 + f(z)) = f(y) \times f(x)$.
		\end{itemize}
		Par induction, on a bien $\forall (x,y) \in \mathcal{N}^2,\:f\big({\otimes}(x,y)\big) = f(x) \times f(y)$.
	\item On définit par induction la fonction suivante \begin{align*}
			\ofact : \mathcal{N} &\longrightarrow \mathcal{N} \\
			\mathbf{0} &\longmapsto \mathbf{S}(\mathbf{0})\\
			\mathbf{S}(x) &\longmapsto {\otimes}\big(\mathbf{S}(x), \ofact(x)\big).
		\end{align*}
	\item Soit $x \in \mathcal{N}$.
		\begin{itemize}
			\item Si $f(x)= 0$, alors $\ofact(x) = \mathbf{S}(\mathbf{0})$\/ par définition, et donc $f(\ofact(x)) = 1 = 0! = f(x) !\:$.
			\item Si $f(x) \ge 1$, alors $x = \mathbf{S}(z)$\/ avec $z \in \mathcal{N}$. Ainsi, par définition de $\ofact$, on a $\ofact(x) = \otimes(x, \ofact(z))$, et donc, par hypothèse de récurrence, $f(\ofact(x)) = f(x) \times f(\ofact(z)) = f(x) \times \big(f(z)!\big)$. Or, comme $f(z) = f(x) - 1$, on a donc $f(\ofact(x)) = f(x) \times \big(f(x) - 1\big)! = f(x)!$\:.
		\end{itemize}
		Par induction, on a bien $\forall x \in \mathcal{N},\:f\big({\ofact}(x)\big) = f(x)!\:$.
\end{enumerate}

		\documentclass[a4paper]{article}

\usepackage[margin=1in]{geometry}
\usepackage[utf8]{inputenc}
\usepackage[T1]{fontenc}
\usepackage{mathrsfs}
\usepackage{textcomp}
\usepackage[french]{babel}
\usepackage{amsmath}
\usepackage{amssymb}
\usepackage{cancel}
\usepackage{frcursive}
\usepackage[inline]{asymptote}
\usepackage{tikz}
\usepackage[european,straightvoltages,europeanresistors]{circuitikz}
\usepackage{tikz-cd}
\usepackage{tkz-tab}
\usepackage[b]{esvect}
\usepackage[framemethod=TikZ]{mdframed}
\usepackage{centernot}
\usepackage{diagbox}
\usepackage{dsfont}
\usepackage{fancyhdr}
\usepackage{float}
\usepackage{graphicx}
\usepackage{listings}
\usepackage{multicol}
\usepackage{nicematrix}
\usepackage{pdflscape}
\usepackage{stmaryrd}
\usepackage{xfrac}
\usepackage{hep-math-font}
\usepackage{amsthm}
\usepackage{thmtools}
\usepackage{indentfirst}
\usepackage[framemethod=TikZ]{mdframed}
\usepackage{accents}
\usepackage{soulutf8}
\usepackage{mathtools}
\usepackage{bodegraph}
\usepackage{slashbox}
\usepackage{enumitem}
\usepackage{calligra}
\usepackage{cinzel}
\usepackage{BOONDOX-calo}

% Tikz
\usetikzlibrary{babel}
\usetikzlibrary{positioning}
\usetikzlibrary{calc}

% global settings
\frenchspacing
\reversemarginpar
\setuldepth{a}

%\everymath{\displaystyle}

\frenchbsetup{StandardLists=true}

\def\asydir{asy}

%\sisetup{exponent-product=\cdot,output-decimal-marker={,},separate-uncertainty,range-phrase=\;à\;,locale=FR}

\setlength{\parskip}{1em}

\theoremstyle{definition}

% Changing math
\let\emptyset\varnothing
\let\ge\geqslant
\let\le\leqslant
\let\preceq\preccurlyeq
\let\succeq\succcurlyeq
\let\ds\displaystyle
\let\ts\textstyle

\newcommand{\C}{\mathds{C}}
\newcommand{\R}{\mathds{R}}
\newcommand{\Z}{\mathds{Z}}
\newcommand{\N}{\mathds{N}}
\newcommand{\Q}{\mathds{Q}}

\renewcommand{\O}{\emptyset}

\newcommand\ubar[1]{\underaccent{\bar}{#1}}

\renewcommand\Re{\expandafter\mathfrak{Re}}
\renewcommand\Im{\expandafter\mathfrak{Im}}

\let\slantedpartial\partial
\DeclareRobustCommand{\partial}{\text{\rotatebox[origin=t]{20}{\scalebox{0.95}[1]{$\slantedpartial$}}}\hspace{-1pt}}

% merging two maths characters w/ \charfusion
\makeatletter
\def\moverlay{\mathpalette\mov@rlay}
\def\mov@rlay#1#2{\leavevmode\vtop{%
   \baselineskip\z@skip \lineskiplimit-\maxdimen
   \ialign{\hfil$\m@th#1##$\hfil\cr#2\crcr}}}
\newcommand{\charfusion}[3][\mathord]{
    #1{\ifx#1\mathop\vphantom{#2}\fi
        \mathpalette\mov@rlay{#2\cr#3}
      }
    \ifx#1\mathop\expandafter\displaylimits\fi}
\makeatother

% custom math commands
\newcommand{\T}{{\!\!\,\top}}
\newcommand{\avrt}[1]{\rotatebox{-90}{$#1$}}
\newcommand{\bigcupdot}{\charfusion[\mathop]{\bigcup}{\cdot}}
\newcommand{\cupdot}{\charfusion[\mathbin]{\cup}{\cdot}}
%\newcommand{\danger}{{\large\fontencoding{U}\fontfamily{futs}\selectfont\char 66\relax}\;}
\newcommand{\tendsto}[1]{\xrightarrow[#1]{}}
\newcommand{\vrt}[1]{\rotatebox{90}{$#1$}}
\newcommand{\tsup}[1]{\textsuperscript{\underline{#1}}}
\newcommand{\tsub}[1]{\textsubscript{#1}}

\renewcommand{\mod}[1]{~\left[ #1 \right]}
\renewcommand{\t}{{}^t\!}
\newcommand{\s}{\text{\calligra s}}

% custom units / constants
%\DeclareSIUnit{\litre}{\ell}
\let\hbar\hslash

% header / footer
\pagestyle{fancy}
\fancyhead{} \fancyfoot{}
\fancyfoot[C]{\thepage}

% fonts
\let\sc\scshape
\let\bf\bfseries
\let\it\itshape
\let\sl\slshape

% custom math operators
\let\th\relax
\let\det\relax
\DeclareMathOperator*{\codim}{codim}
\DeclareMathOperator*{\dom}{dom}
\DeclareMathOperator*{\gO}{O}
\DeclareMathOperator*{\po}{\text{\cursive o}}
\DeclareMathOperator*{\sgn}{sgn}
\DeclareMathOperator*{\simi}{\sim}
\DeclareMathOperator{\Arccos}{Arccos}
\DeclareMathOperator{\Arcsin}{Arcsin}
\DeclareMathOperator{\Arctan}{Arctan}
\DeclareMathOperator{\Argsh}{Argsh}
\DeclareMathOperator{\Arg}{Arg}
\DeclareMathOperator{\Aut}{Aut}
\DeclareMathOperator{\Card}{Card}
\DeclareMathOperator{\Cl}{\mathcal{C}\!\ell}
\DeclareMathOperator{\Cov}{Cov}
\DeclareMathOperator{\Ker}{Ker}
\DeclareMathOperator{\Mat}{Mat}
\DeclareMathOperator{\PGCD}{PGCD}
\DeclareMathOperator{\PPCM}{PPCM}
\DeclareMathOperator{\Supp}{Supp}
\DeclareMathOperator{\Vect}{Vect}
\DeclareMathOperator{\argmax}{argmax}
\DeclareMathOperator{\argmin}{argmin}
\DeclareMathOperator{\ch}{ch}
\DeclareMathOperator{\com}{com}
\DeclareMathOperator{\cotan}{cotan}
\DeclareMathOperator{\det}{det}
\DeclareMathOperator{\id}{id}
\DeclareMathOperator{\rg}{rg}
\DeclareMathOperator{\rk}{rk}
\DeclareMathOperator{\sh}{sh}
\DeclareMathOperator{\th}{th}
\DeclareMathOperator{\tr}{tr}

% colors and page style
\definecolor{truewhite}{HTML}{ffffff}
\definecolor{white}{HTML}{faf4ed}
\definecolor{trueblack}{HTML}{000000}
\definecolor{black}{HTML}{575279}
\definecolor{mauve}{HTML}{907aa9}
\definecolor{blue}{HTML}{286983}
\definecolor{red}{HTML}{d7827e}
\definecolor{yellow}{HTML}{ea9d34}
\definecolor{gray}{HTML}{9893a5}
\definecolor{grey}{HTML}{9893a5}
\definecolor{green}{HTML}{a0d971}

\pagecolor{white}
\color{black}

\begin{asydef}
	settings.prc = false;
	settings.render=0;

	white = rgb("faf4ed");
	black = rgb("575279");
	blue = rgb("286983");
	red = rgb("d7827e");
	yellow = rgb("f6c177");
	orange = rgb("ea9d34");
	gray = rgb("9893a5");
	grey = rgb("9893a5");
	deepcyan = rgb("56949f");
	pink = rgb("b4637a");
	magenta = rgb("eb6f92");
	green = rgb("a0d971");
	purple = rgb("907aa9");

	defaultpen(black + fontsize(8pt));

	import three;
	currentlight = nolight;
\end{asydef}

% theorems, proofs, ...

\mdfsetup{skipabove=1em,skipbelow=1em, innertopmargin=6pt, innerbottommargin=6pt,}

\declaretheoremstyle[
	headfont=\normalfont\itshape,
	numbered=no,
	postheadspace=\newline,
	headpunct={:},
	qed=\qedsymbol]{demstyle}

\declaretheorem[style=demstyle, name=Démonstration]{dem}

\newcommand\veczero{\kern-1.2pt\vec{\kern1.2pt 0}} % \vec{0} looks weird since the `0' isn't italicized

\makeatletter
\renewcommand{\title}[2]{
	\AtBeginDocument{
		\begin{titlepage}
			\begin{center}
				\vspace{10cm}
				{\Large \sc Chapitre #1}\\
				\vspace{1cm}
				{\Huge \calligra #2}\\
				\vfill
				Hugo {\sc Salou} MPI${}^{\star}$\\
				{\small Dernière mise à jour le \@date }
			\end{center}
		\end{titlepage}
	}
}

\newcommand{\titletp}[4]{
	\AtBeginDocument{
		\begin{titlepage}
			\begin{center}
				\vspace{10cm}
				{\Large \sc tp #1}\\
				\vspace{1cm}
				{\Huge \textsc{\textit{#2}}}\\
				\vfill
				{#3}\textit{MPI}${}^{\star}$\\
			\end{center}
		\end{titlepage}
	}
	\fancyfoot{}\fancyhead{}
	\fancyfoot[R]{#4 \textit{MPI}${}^{\star}$}
	\fancyhead[C]{{\sc tp #1} : #2}
	\fancyhead[R]{\thepage}
}

\newcommand{\titletd}[2]{
	\AtBeginDocument{
		\begin{titlepage}
			\begin{center}
				\vspace{10cm}
				{\Large \sc td #1}\\
				\vspace{1cm}
				{\Huge \calligra #2}\\
				\vfill
				Hugo {\sc Salou} MPI${}^{\star}$\\
				{\small Dernière mise à jour le \@date }
			\end{center}
		\end{titlepage}
	}
}
\makeatother

\newcommand{\sign}{
	\null
	\vfill
	\begin{center}
		{
			\fontfamily{ccr}\selectfont
			\textit{\textbf{\.{\"i}}}
		}
	\end{center}
	\vfill
	\null
}

\renewcommand{\thefootnote}{\emph{\alph{footnote}}}

% figure support
\usepackage{import}
\usepackage{xifthen}
\pdfminorversion=7
\usepackage{pdfpages}
\usepackage{transparent}
\newcommand{\incfig}[1]{%
	\def\svgwidth{\columnwidth}
	\import{./figures/}{#1.pdf_tex}
}

\pdfsuppresswarningpagegroup=1
\ctikzset{tripoles/european not symbol=circle}

\newcommand{\missingpart}{{\large\color{red} Il manque quelque chose ici\ldots}}


\fancyhead[R]{Hugo {\sc Salou}\/ MPI}
\fancyhead[L]{TD\textsubscript4 -- Exercice 8}

\begin{document}
	\let\thesection\relax
	
\begin{comment}
\section{Exercice 9}

\slshape
Soit la matrice $A = \begin{pmatrix}
	1&1&-1\\
	2&3&-4\\
	4&1&-4
\end{pmatrix}$.
\begin{enumerate}
	\item Déterminer le spectre de la matrice $A$\/ et trouver une matrice $P$\/ inversible telle que $P^{-1} A P$\/ est diagonale.
	\item Soit $B$\/ une matrice de taille $3\times 3$\/ qui commentent avec $A$\/ ($AB = BA$). Montrer que $B$\/ est diagonale.
\end{enumerate}
\upshape

\begin{enumerate}
	\item On sait, tout d'abord, que, pour $x \in \R$,
		\begin{align*}
			\chi_A(x) = \det(x\,I_n - A) &=
			\begin{vmatrix}
				x - 1 &- 1 & 1\\
				-2&x-3&4\\
				-4&-1&x + 4
			\end{vmatrix}\\
			&= \\
		\end{align*}
\end{enumerate}
\end{comment}

\section{Exercice 8}

\begin{enumerate}
	\item Soit un vecteur non nul $\vec{x} \in \Ker(\lambda {\id} - {u \circ v})$. Ainsi, $u(v(\vec{x})) = \lambda \vec{x}$. Et, donc $v(u(v(\vec{x}))) = \lambda v(\vec{x})$. On a donc $v(\vec{x}) \in \Ker(\lambda {\id} - {v  \circ u})$.
		Or, si $\lambda \neq 0$, on a $v(\vec{x}) \neq \vec{0}$\/ ; en effet, si $v(\vec{x}) = \vec{0}$, alors $u \circ v(\vec{x}) = \vec{0} = \lambda \vec{x}$\/ et donc $\vec{x} = \vec{0}$, ce ne serait donc pas un vecteur propre de $u \circ v$\/ : une contradiction. On en déduit que $v(\vec{x})$\/ est un vecteur propre de $u \circ v$\/ associé à la valeur propre $\lambda$.
	\item On pose donc $\lambda = 0$, une valeur propre de $u  \circ v$. L'endomorphisme $u \circ v$\/ n'est donc pas injectif, donc bijectif. On sait donc, comme $E$\/ est de dimension finie, que $\det(u \circ v) = 0$. Or $\det (u \circ v) = \det u \times \det v = \det(v  \circ u)$. Et donc $\det(v  \circ u) = 0$, $v  \circ u$\/ n'est donc pas bijectif, donc injectif. Et donc, on a $0 \in \Sp(v  \circ u)$.
	\item Soit $P \in \R[X]$, et soit $Q$\/ une primitive de $P$.
		\begin{align*}
			P \in \Ker (u  \circ v) \iff& \Big(\int_{0}^X P(t)~\mathrm{d}t\Big)' = 0\\
			\iff& \big(Q(X) - Q(0)\big)' = 0\\
			\iff& Q'(X) = 0\\
			\iff& P(X) = 0
		\end{align*}
		On en déduit que $\Ker (u \circ v) = \{0\}$.

		Également,
		\begin{align*}
			P \in \Ker(v  \circ u) \iff& \int_{0}^{X} P'(t)~\mathrm{d}t = 0\\
			\iff& P(X) - P(0) = 0\\
			\iff& P(X) = P(0)\\
			\iff& \deg P \le 0\\
			\iff& P \in \R_0[X]
		\end{align*}
		On en déduit que $\Ker(v \circ u) = \R_0[X]$.
\end{enumerate}


\end{document}

	}
	\def\addmacros#1{#1}
}
{
	\td[3]{Langages et expressions régulières}
	\minitoc
	\renewcommand{\cwd}{../td/td03/}
	\addmacros{
		\section{Quelques problèmes décidables}

\begin{enumerate}
	\item Soit $f : \R \to \R$.
		\begin{itemize}
			\item Si $f$\/ admet un zéro, on pose $\mathcal{M} = \texttt{fun}\ \texttt{s}\ \to \texttt{true}$.
			\item Si $f$\/ n'admet pas un zéro, on pose $\mathcal{M} = \texttt{fun}\ \texttt{s}\ \to \texttt{false}$.
		\end{itemize}
		Alors, $\mathcal{M}$\/ décide \textsc{Zero}$_f$.
	\item Soit $\mathcal{M}$\/ une machine, et soit $w \in \Sigma^*$.
		\begin{itemize}
			\item Si $\mathcal{M}$\/ se termine sur l'entrée $w$, alors on pose $\mathcal{M}' = \texttt{fun}\ \texttt{s} \to \texttt{true}$.
			\item Si $\mathcal{M}$\/ ne se termine pas sur l'entrée $w$, alors on pose $\mathcal{M}' = \texttt{fun}\ \texttt{s} \to \texttt{false}$.
		\end{itemize}
		Alors, $\mathcal{M}'$\/ décide \textsc{Arrêt}$_{\mathcal{M},w}$.
	\item Le problème est trivialement vrai. En effet, soit $M \in \mathcal{O}$, de la forme
		\begin{lstlisting}[language=caml]
let m (s: string): string =
	%*$\langle$\textrm{code}$\rangle$*) 
		\end{lstlisting}
		On crée la machine $\mathcal{N}$\/ ci-dessous.
		\begin{lstlisting}[language=caml]
let n (s: string): string =
	if true then
		%*$\langle$\textrm{code}$\rangle$*) 
	else
		%*$\langle$\textrm{code}$\rangle$*) 
		\end{lstlisting}
		On a $\texttt{m} \neq \texttt{n}$, mais $\mathcal{L}(\texttt{m}) = \mathcal{L}(\texttt{n})$, donc le problème est vrai sur toute entrée et la fonction $\texttt{fun}\ \texttt{s} \to \texttt{true}$\/ répond au problème.
\end{enumerate}

		\section{Exercice 10}

\begin{enumerate}
	\item On a $u_6 \leadsto \hat{v}_\mathrm{e}$, $u_3 \leadsto \hat{v}_\mathrm{c}$\/ (composante continue), $u_2 \leadsto \hat{v}_\mathrm{d}$\/ (discontinuités), $u_4 \leadsto \hat{v}_\mathrm{a}$\/ (composante continue), $u_5 \leadsto \hat{v}_\mathrm{f}$\/ (nombre de fréquences) et $u_1 \leadsto \hat{v}_\mathrm{b}$.
	\item
		Le filtre donnant le signal $\hat{v}_\mathrm{g}$\/ est un passe-bandes (les hautes et basses fréquences sont éliminées) et c'est un filtre non linéaire (de nouvelles fréquences apparaissent).

		Le filtre donnant le signal $\hat{v}_\mathrm{h}$\/ est un filtre passe-bas.

		Le filtre donnant le signal $\hat{v}_\mathrm{i}$\/ est un filtre passe-haut dont sa fréquence de coupure est inférieure à $1\:\mathrm{kHz}$.
\end{enumerate}


		\section{Déterminisation 1}

\begin{enumerate}
	\item \tikzfig{ex11-1}
	\item \tikzfig{ex11-2}
\end{enumerate}

		\section{Déterminisation 2}

\begin{enumerate}
	\item \tikzfig{ex12-1}
	\item \tikzfig{ex12-2}
	\item \tikzfig{ex12-3}
\end{enumerate}

		\section{Ensembles définis inductivement}

La correction est disponible sur \textit{cahier-de-prepa}.

\begin{comment}
	\begin{exm}
		Avec $S = \N$, $\mathcal{B} = \{0, 2\} $, $A_1 = \{0\}$\/ et \begin{align*}
			f_1: A_1 \times \N &\longrightarrow \N \\
			(0, x) &\longmapsto x + 4.
		\end{align*}

		On a \[
			X \supseteq \{0, 2, 4, 6, 8, 10, \ldots, 20, \ldots\} = 2\N
		.\]
	\end{exm}
	\begin{exm}
		Avec $S$\/ l'ensemble des langages sur $\Sigma$, $\mathcal{B} = \{\O\} \cup \bigl\{\{a\}\:\big|\: a \in \Sigma \bigr\}$, et
		\begin{multicols}{3}
			\begin{align*}
				f_1: S \times S &\longrightarrow S \\
				(L_1, L_2) &\longmapsto L_1 \cup L_2,
			\end{align*}
			\begin{align*}
				f_2: S \times S &\longrightarrow S \\
				(L_1, L_2) &\longmapsto L_1 \cdot L_2,
			\end{align*}
			\begin{align*}
				f_3: S &\longrightarrow S \\
				L &\longmapsto L^*.
			\end{align*}
		\end{multicols}
	\end{exm}

\begin{enumerate}
	\item Soit $\mathcal{A} = \{X \subseteq S  \mid X \supseteq \mathcal{B} \mathrel{\text{et}} X \text{ est stable par } f_i\}$. On a $S \in \mathcal{A}$\/ et donc $\mathcal{A} \neq \O$. De plus, soit \[
			Y = \{x \in S  \mid \forall X \in \mathcal{A},\,x \in X\} = \bigcap_{X \in \mathcal{A}} X
		.\]
		Soit $b \in \mathcal{B}$, on a $\forall X \in A,\, b \in X$. D'où $b \in Y$\/ par intersection. On en déduit que $\mathcal{B} \subseteq Y$.

		Soit $i \in \left\llbracket 1,m \right\rrbracket$. Soit $(x_1, \ldots, x_{n_i}) \in Y^{n_i}$\/ et soit $a \in A_i$. Montrons que $f_i(a, x_1, \ldots, x_{n_i}) \in Y$.
		Or, soit $X \in \mathcal{A}$, on a $(x_1, \ldots, x_{n_i}) \in X^{n_i}$\/ donc $f_i(a, x_1, \ldots, x_{n_i}) \in X$. Ceci étant vrai pour tout $X \in \mathcal{A}$, on a $f_i(a, x_1, \ldots, x_{n_i}) \in Y$\/ donc $Y$\/ est stable par $f_i$\/ par tout $i \in \left\llbracket 1,m \right\rrbracket$\/ et donc $Y \in \mathcal{A}$.
		On a également $Y \subseteq X$\/ pour tout $X \in \mathcal{A}$. On en déduit que $Y$\/ est le plus petit élément (pour l'inclusion) de $\mathcal{A}$.
	\item On pose $X_0  = \mathcal{B}$\/ et \[
			X_{n+1} = X_n \cup \big\{ f_i(a, x_1, \ldots, x_{n_i})  \mid a \in A_i,\,(x_1, \ldots, x_{n_i}) \in (X_n)^{n_i},\,i \in \left\llbracket 1,m \right\rrbracket\big\}
		.\]
		Soit $X = \bigcup_{n \in \N} X_n$. Soit $Y$\/ l'ensemble défini par induction à partir de $\mathcal{B}$\/ et des $(f_i)_{i\in\left\llbracket 1,n \right\rrbracket}$. Montrons que $X = Y$.
		On montre que $X$\/ est le plus petit élément (pour l'inclusion) de $\mathcal{A}$\/ et on conclut par unicité du minimum (avec la question précédente).
		Par définition de la suite $(X_n)_{n\in\N}$, elle est croissante (au sens de l'inclusion).
		Montrons à présent, par récurrence, la propriété ci-dessous : $P_n : ``X_n \subseteq Y."$
		\begin{itemize}
			\item Par définition de $Y$, on a $X_0 = \mathcal{B} \subseteq Y$.
			\item Soit 
		\end{itemize}
\end{enumerate}

\subsection{Un théorème d'induction}

\begin{enumerate}
	\item[3.] Soit $Z = \{x \in S  \mid P(x) \text{ vraie}\:\}$.
		Montrons que $\mathcal{X}\subseteq Z$.
		On remarque que $\mathcal{X} \supseteq \mathcal{B}$\/ ; $\mathcal{X}$\/ est stable par $f_i$. On en conclut que $Z \supseteq \mathcal{X}$ et donc $\forall x \in \mathcal{X},\,P(x)$\/ est vraie.
\end{enumerate}

\begin{exm}
	Soit $\mathcal{X}$\/ défini par induction par $\mathcal{B} = \{0, 2\}$\/ et \begin{align*}
		f: \N &\longrightarrow \N \\
		n &\longmapsto n + 2.
	\end{align*}
	Montrons que $\forall n \in \mathcal{X}$, $x$\/ est pair.

	On sait que $0$\/ est pair, $2$\/ est pair ; et, \[
		\forall x,y \in \mathcal{X},\, (x \text{ pair} \land y \text{ pair}) \implies f(x, y) \text{ pair}
	.\]

	On en déduit que \[
		\forall n \in \mathcal{X},\,x \text{ est pair}.
	.\]
\end{exm}
\end{comment}

		\section{Tableaux dynamiques}

\begin{enumerate}[start=3]
	\item On trouve une complexité amortie en $n^2$. À rédiger.
	\item Au lieu de diviser quand $r < n / 2$, mais quand $r < n / 4$.
\end{enumerate}

		\section{Barres en triangle}


On note $a(x)$ le côté du triangule équilatéral, et donc $x = a(x) \sqrt{3}  / 2$.

On calcule le flux $\Phi$ magnétique : \[
	\Phi = B \frac{a x}{2} = B x^2 / \sqrt{3}
.\]
Ainsi, d'après la loi de \textsc{Faraday}, on a \[
	e = - \frac{\mathrm{d}\Phi}{\mathrm{d}t} = - \frac{2B}{\sqrt{3}}  \: x\,\dot{x}
.\]
Or, par loi d'\textsc{Ohm}, $i = e / R(x)$.
Et, on connait la resistance du circuit $R(x) = 3 a(x) / \gamma S$.
Alors, 
\begin{align*}
	i(x) &= \frac{-2B\,x\,\dot{x}}{\sqrt{3} \cdot \frac{3a(x)}{\gamma S}} = - \frac{2B \gamma S}{3 \sqrt{3}} \cdot \frac{x\,\dot{x}}{2 \frac{x}{\sqrt{3}}}\\
	&= - B \gamma S \dot{x} / 3. \\
\end{align*}
On calcule donc la force de \textsc{Laplace} :
\begin{align*}
	\vec{F}_{\mathcal{L}} &= i \cdot [\mathrm{CD}] \cdot B \cdot \vec{e}_x\\
	&= i \cdot \frac{2\dot{x}}{\sqrt{3}} B \vec{e}_x \\
	&= -\underbrace{\frac{2 B^2 \gamma S}{3\sqrt{3}}}_\alpha  \cdot x \dot{x} \\
\end{align*}

D'après le \textsc{pfd}, on a donc \[
	m \ddot{x} = - \alpha x \dot{x} \text{ d'où }  \ddot{x} = - \frac{\alpha}{m} x\dot{x} 
.\] où $m = \rho S L$. 

On intègre les deux côtés de l'équation, \[
	[\dot{x}]_0^{t_\mathrm{f}} = -\frac{\alpha}{m} \cdot \left[ \frac{x^2}{2} \right]_0^{t_\mathrm{f}}
.\]
D'où, \[
	0 - v_0 = \frac{-\alpha}{m} \cdot x_\mathrm{f}^2 / 2
.\] On en conclut \[
	x_\mathrm{f} = \sqrt{\frac{2m}{\alpha} v_0} 
.\] 


		\begin{multicols}{2}
	\section{Implications}
	\begin{enumerate}
		\item
			\[
				\begin{prooftree}
					\infer 0[Ax]{p,q \vdash q}
					\infer 1[$\to$i]{q \vdash p \to q}
				\end{prooftree}
			\]
		\item
			\[
				\begin{prooftree}
					\infer 0[Ax]{p, p\land q\vdash p\land q}
					\infer 1[$\land$e,d]{p, p \land q \vdash q}
					\infer 1[$\to$i]{p \land q \vdash p \to q}
				\end{prooftree}
			\]
		\item 
			\[
				\begin{prooftree}
					\infer 0[Ax]{p, p\to q \vdash p}
					\infer 0[Ax]{p, p\to q \vdash p \to q}
					\infer 0[Ax]{p, p\to q \vdash p}
					\infer 2[$\to$e]{p, p \to q \vdash q}
					\infer 2[$\land$i]{p, p \to q \vdash p \land q}
					\infer 1[$\to$i]{p \to q \vdash p\to (p \land q)}
				\end{prooftree}
			\]
		\item
			\[
				\begin{prooftree}
					\infer 0[Ax]{\lnot q, p\to q, p \vdash p \to q}
					\infer 0[Ax]{\lnot q, p\to q, p \vdash p}
					\infer 2[$\to$e]{\lnot q, p\to q, p \vdash q}
					\infer 0[Ax]{\lnot q, p\to q, p \vdash \lnot q}
					\infer 2[$\lnot$e]{\lnot q, p\to q, p \vdash \bot}
					\infer 1[$\lnot$i]{\lnot q, p\to q \vdash\lnot p}
					\infer 1[$\to$i]{p \to q \vdash \lnot q \to \lnot p}
				\end{prooftree}
			\]
		\item 
			\[
				\begin{prooftree}
					\infer 0[Ax]{p \land q, p\to r \vdash p \land q}
					\infer 1[$\land$e,g]{p \land q, p\to r \vdash p}
					\infer 0[Ax]{p \land q, p\to r \vdash p \to r}
					\infer 2[$\to$e]{p \land q, p \to r \vdash r}
					\infer 1[$\to$i]{p \to r \vdash (p \land q) \to r}
				\end{prooftree}
			\] 
		\item
			\[
				\begin{prooftree}
					\infer 0[Ax]{p,q\vdash p}
					\infer 1[$\to$i]{p \vdash q \to p}
					\infer 1[$\to$i]{\vdash p \to (q \to p)}
				\end{prooftree}
			\]
		\item
			\[
				\begin{prooftree}
					\infer 0[Ax]{p \to q, p \vdash p\to q}
					\infer 0[Ax]{p\to q,p \vdash p}
					\infer 2[$\to$e]{p, p\to q \vdash q}
					\infer 1[$\to$i]{p \vdash (p \to q) \to q}
				\end{prooftree}
			\] 
	\end{enumerate}
\end{multicols}

		\section{Langage de \textsc{Dyck}}

\begin{enumerate}
	\item On suppose ce langage reconnaissable par un automate à $n$ états. On considère le mot $w = {\red(}^n \cdot {\red)}^n$, donc $|w| \ge n$.
		Ainsi, il existe $x$, $y$ et $z$ trois mots tels que $w = xyz$, $|xy| \le n$, $y \neq \varepsilon$ et $\forall p \in \N,\: x y^p z \in \mathcal{L}(\mathcal{G})$.
		Soit alors $p \in \llbracket 1,n-1 \rrbracket$ et $q \in \llbracket 1,n -p \rrbracket$ tels que $x = {\red(}^p$, $y = {\red(}^q$ et $z = {\red(}^{n-q-p} \cdot {\red)}^n$.
		Ainsi, $xy \in \mathcal{L}(\mathcal{G})$, ce qui est absurde. On en déduit que $\mathcal{L}(\mathcal{G})$ n'est pas reconnaissable, il n'est donc pas régulier.
	\item On pose $\mathcal{G} = (\Sigma, \{\mathrm{S}\}, \{\mathrm{S} \to \red( \mathrm{S}\red)  \mid \mathrm{SS}  \mid \varepsilon\}, \mathrm{S})$.
	\item
		\begin{itemize}
			\item On le montre par induction.
				\begin{itemize}
					\item \textbf{Cas $\mathrm{S} \to \varepsilon$.}
						On a $|\varepsilon|_{\red(} = 0 = |\varepsilon|_{\red)}$.
					\item \textbf{Cas $\mathrm{S} \to \red( \mathrm{S}\red)$.}
						Soit $u \in \mathcal{L}(\mathcal{G})$ avec $|u|_{\red(} = |u|_{\red)} = n$. Ainsi, $|\red(u\red)|_{\red(} = |\red(u\red)|_{\red)} = n + 1$.
					\item \textbf{Cas $\mathrm{S}\to \mathrm{SS}$.}
						Soient $u$ et $v$ deux mots de $\mathcal{L}(\mathcal{G})$ tels que $|u|_{\red(} = |u|_{\red)} = n$ et $|v|_{\red(} = |v|_{\red)} = m$.
						Alors, $|u\cdot v|_{\red(} = |v\cdot u|_{\red)} = n + m$.
				\end{itemize}
			\item Montrons par induction $\mathcal{P}_{u}$ : \guillemotleft~pour tout $v$ préfixe de $u$, $|v|_{\red(} \ge |v|_{\red)}$.~\guillemotright\@ 
				\begin{itemize}
					\item \textbf{Cas $\mathrm{S} \to \varepsilon$.}
						Le seul préfixe de $\varepsilon$ est $\varepsilon$, et on a bien $|\varepsilon|_{\red(} = 0 \ge 0 = |\varepsilon|_{\red)}$.
					\item \textbf{Cas $\mathrm{S}\to \red( \mathrm{S} \red)$.}
						Soit $u$ un mot de $\mathcal{L}(\mathcal{G})$ vérifiant $\mathcal{P}_u$.
						Soit $v$ un préfixe de $\red(u\red)$.
						On procède par induction sur $v$.
						\begin{itemize}
							\item \textbf{Cas $v = \varepsilon$ ou $\red($.} \textsc{ok}.
							\item \textbf{Cas $\red(\tilde{u}$,} où $\tilde{u}$ est un préfixe de $u$.
								Par hypothèse d'induction, $|\tilde{u}|_{\red(} \ge |\tilde{u}|_{\red)}$ donc $|\red(\tilde{u}|_{\red(} = |\red(\tilde{u}|_{\red)}$.
							\item \textbf{Cas $\red(u\red)$.} Par hypothèse d'induction, $|u|_{\red(} \ge |u|_{\red)}$ donc $|\red(u\red)|_{\red(}\ge |\red(u\red)|_{\red)}$.
						\end{itemize}
				\end{itemize}
		\end{itemize}
	\item On note $\overline w^j= \big| w_{\llbracket 0,j \rrbracket} \big|_{\red(} - \big| w_{\llbracket 0,j \rrbracket} \big|_{\red)}$.
		Alors les deux conditions se traduisent par $\overline w^{|w|} = 0$ et $\forall i \in \llbracket 0,|w|-1 \rrbracket$, $\overline w^i \ge 0$.
\end{enumerate}

		\section{$\mathcal{N}$}

\newcommand{\ofact}{\charfusion[\mathbin]{\bigcirc}{\scriptstyle!}}

\begin{enumerate}
	\item On définit par induction la fonction suivante \begin{align*}
			\oplus: \mathcal{N}^2 &\longrightarrow \mathcal{N} \\
			(\mathbf{S}(x),y) &\longmapsto \oplus(x, \mathbf{S}(y))\\
			(\mathbf{0}, x) &\longmapsto x.
		\end{align*}
	\item Soit $(x,y) \in \mathcal{N}^2$.
		\begin{itemize}
			\item Si $f(x) = 0$, alors $\oplus(x,y) = y$\/ et donc $f(\oplus(x,y)) = f(y) = f(x) + f(y)$.
			\item Si $f(x) \ge 1$, alors  $x = \mathbf{S}(z)$\/ avec $z \in \mathcal{N}$. Ainsi, $\oplus(x,y) = \oplus(z, \mathbf{S}(y))$. Or, $f(z) = f(x) - 1 \le f(x)$. Et donc, par définition de $\oplus$\/ puis par hypothèse d'induction, on a $f(\oplus(x,y)) = f(\oplus(z, \mathbf{S}(y)) = f(z) + f(\mathbf{S}(y))$. On en déduit que $f(\oplus(x,y)) = f(x) - 1 + f(y) + 1 = f(x) + f(y)$.
		\end{itemize}
		Par induction, on a bien $\forall (x,y) \in \mathcal{N}^2,\:f\big({\oplus}(x,y)\big) = f(x) + f(y)$.
	\item On définit par induction la fonction suivante \begin{align*}
			\otimes: \mathcal{N}^2 &\longrightarrow \mathcal{N} \\
			(\mathbf{S}(x), y) &\longmapsto {\oplus}\big(y, {\otimes}(x,y)\big)\\
			(\mathbf{0}, y) &\longmapsto \mathbf{0}.
		\end{align*}
	\item Soit $(x,y) \in \mathcal{N}^2$.
		\begin{itemize}
			\item Si $f(x) = 0$, alors $\otimes(x,y) = \mathbf{0}$, et donc $f(\otimes(x,y)) = 0 = f(x) \times f(y)$.
			\item Si $f(x) \ge 1$, alors $x = \mathbf{S}(z)$\/ avec $z \in \mathcal{N}$. Ainsi, par définition de $\otimes$, on a $\otimes(x,y) = \oplus(y, \otimes(z,y))$. Or, par hypothèse d'induction, $f(\otimes(z,y)) = f(z) \times f(y)$\/ (car $f(z) < f(x)$), et donc $f(\otimes(x,y)) = f(y) + f(\otimes(z,y)) = f(y) + f(z) \times f(y) = f(y) \times (1 + f(z)) = f(y) \times f(x)$.
		\end{itemize}
		Par induction, on a bien $\forall (x,y) \in \mathcal{N}^2,\:f\big({\otimes}(x,y)\big) = f(x) \times f(y)$.
	\item On définit par induction la fonction suivante \begin{align*}
			\ofact : \mathcal{N} &\longrightarrow \mathcal{N} \\
			\mathbf{0} &\longmapsto \mathbf{S}(\mathbf{0})\\
			\mathbf{S}(x) &\longmapsto {\otimes}\big(\mathbf{S}(x), \ofact(x)\big).
		\end{align*}
	\item Soit $x \in \mathcal{N}$.
		\begin{itemize}
			\item Si $f(x)= 0$, alors $\ofact(x) = \mathbf{S}(\mathbf{0})$\/ par définition, et donc $f(\ofact(x)) = 1 = 0! = f(x) !\:$.
			\item Si $f(x) \ge 1$, alors $x = \mathbf{S}(z)$\/ avec $z \in \mathcal{N}$. Ainsi, par définition de $\ofact$, on a $\ofact(x) = \otimes(x, \ofact(z))$, et donc, par hypothèse de récurrence, $f(\ofact(x)) = f(x) \times f(\ofact(z)) = f(x) \times \big(f(z)!\big)$. Or, comme $f(z) = f(x) - 1$, on a donc $f(\ofact(x)) = f(x) \times \big(f(x) - 1\big)! = f(x)!$\:.
		\end{itemize}
		Par induction, on a bien $\forall x \in \mathcal{N},\:f\big({\ofact}(x)\big) = f(x)!\:$.
\end{enumerate}

		\documentclass[a4paper]{article}

\usepackage[margin=1in]{geometry}
\usepackage[utf8]{inputenc}
\usepackage[T1]{fontenc}
\usepackage{mathrsfs}
\usepackage{textcomp}
\usepackage[french]{babel}
\usepackage{amsmath}
\usepackage{amssymb}
\usepackage{cancel}
\usepackage{frcursive}
\usepackage[inline]{asymptote}
\usepackage{tikz}
\usepackage[european,straightvoltages,europeanresistors]{circuitikz}
\usepackage{tikz-cd}
\usepackage{tkz-tab}
\usepackage[b]{esvect}
\usepackage[framemethod=TikZ]{mdframed}
\usepackage{centernot}
\usepackage{diagbox}
\usepackage{dsfont}
\usepackage{fancyhdr}
\usepackage{float}
\usepackage{graphicx}
\usepackage{listings}
\usepackage{multicol}
\usepackage{nicematrix}
\usepackage{pdflscape}
\usepackage{stmaryrd}
\usepackage{xfrac}
\usepackage{hep-math-font}
\usepackage{amsthm}
\usepackage{thmtools}
\usepackage{indentfirst}
\usepackage[framemethod=TikZ]{mdframed}
\usepackage{accents}
\usepackage{soulutf8}
\usepackage{mathtools}
\usepackage{bodegraph}
\usepackage{slashbox}
\usepackage{enumitem}
\usepackage{calligra}
\usepackage{cinzel}
\usepackage{BOONDOX-calo}

% Tikz
\usetikzlibrary{babel}
\usetikzlibrary{positioning}
\usetikzlibrary{calc}

% global settings
\frenchspacing
\reversemarginpar
\setuldepth{a}

%\everymath{\displaystyle}

\frenchbsetup{StandardLists=true}

\def\asydir{asy}

%\sisetup{exponent-product=\cdot,output-decimal-marker={,},separate-uncertainty,range-phrase=\;à\;,locale=FR}

\setlength{\parskip}{1em}

\theoremstyle{definition}

% Changing math
\let\emptyset\varnothing
\let\ge\geqslant
\let\le\leqslant
\let\preceq\preccurlyeq
\let\succeq\succcurlyeq
\let\ds\displaystyle
\let\ts\textstyle

\newcommand{\C}{\mathds{C}}
\newcommand{\R}{\mathds{R}}
\newcommand{\Z}{\mathds{Z}}
\newcommand{\N}{\mathds{N}}
\newcommand{\Q}{\mathds{Q}}

\renewcommand{\O}{\emptyset}

\newcommand\ubar[1]{\underaccent{\bar}{#1}}

\renewcommand\Re{\expandafter\mathfrak{Re}}
\renewcommand\Im{\expandafter\mathfrak{Im}}

\let\slantedpartial\partial
\DeclareRobustCommand{\partial}{\text{\rotatebox[origin=t]{20}{\scalebox{0.95}[1]{$\slantedpartial$}}}\hspace{-1pt}}

% merging two maths characters w/ \charfusion
\makeatletter
\def\moverlay{\mathpalette\mov@rlay}
\def\mov@rlay#1#2{\leavevmode\vtop{%
   \baselineskip\z@skip \lineskiplimit-\maxdimen
   \ialign{\hfil$\m@th#1##$\hfil\cr#2\crcr}}}
\newcommand{\charfusion}[3][\mathord]{
    #1{\ifx#1\mathop\vphantom{#2}\fi
        \mathpalette\mov@rlay{#2\cr#3}
      }
    \ifx#1\mathop\expandafter\displaylimits\fi}
\makeatother

% custom math commands
\newcommand{\T}{{\!\!\,\top}}
\newcommand{\avrt}[1]{\rotatebox{-90}{$#1$}}
\newcommand{\bigcupdot}{\charfusion[\mathop]{\bigcup}{\cdot}}
\newcommand{\cupdot}{\charfusion[\mathbin]{\cup}{\cdot}}
%\newcommand{\danger}{{\large\fontencoding{U}\fontfamily{futs}\selectfont\char 66\relax}\;}
\newcommand{\tendsto}[1]{\xrightarrow[#1]{}}
\newcommand{\vrt}[1]{\rotatebox{90}{$#1$}}
\newcommand{\tsup}[1]{\textsuperscript{\underline{#1}}}
\newcommand{\tsub}[1]{\textsubscript{#1}}

\renewcommand{\mod}[1]{~\left[ #1 \right]}
\renewcommand{\t}{{}^t\!}
\newcommand{\s}{\text{\calligra s}}

% custom units / constants
%\DeclareSIUnit{\litre}{\ell}
\let\hbar\hslash

% header / footer
\pagestyle{fancy}
\fancyhead{} \fancyfoot{}
\fancyfoot[C]{\thepage}

% fonts
\let\sc\scshape
\let\bf\bfseries
\let\it\itshape
\let\sl\slshape

% custom math operators
\let\th\relax
\let\det\relax
\DeclareMathOperator*{\codim}{codim}
\DeclareMathOperator*{\dom}{dom}
\DeclareMathOperator*{\gO}{O}
\DeclareMathOperator*{\po}{\text{\cursive o}}
\DeclareMathOperator*{\sgn}{sgn}
\DeclareMathOperator*{\simi}{\sim}
\DeclareMathOperator{\Arccos}{Arccos}
\DeclareMathOperator{\Arcsin}{Arcsin}
\DeclareMathOperator{\Arctan}{Arctan}
\DeclareMathOperator{\Argsh}{Argsh}
\DeclareMathOperator{\Arg}{Arg}
\DeclareMathOperator{\Aut}{Aut}
\DeclareMathOperator{\Card}{Card}
\DeclareMathOperator{\Cl}{\mathcal{C}\!\ell}
\DeclareMathOperator{\Cov}{Cov}
\DeclareMathOperator{\Ker}{Ker}
\DeclareMathOperator{\Mat}{Mat}
\DeclareMathOperator{\PGCD}{PGCD}
\DeclareMathOperator{\PPCM}{PPCM}
\DeclareMathOperator{\Supp}{Supp}
\DeclareMathOperator{\Vect}{Vect}
\DeclareMathOperator{\argmax}{argmax}
\DeclareMathOperator{\argmin}{argmin}
\DeclareMathOperator{\ch}{ch}
\DeclareMathOperator{\com}{com}
\DeclareMathOperator{\cotan}{cotan}
\DeclareMathOperator{\det}{det}
\DeclareMathOperator{\id}{id}
\DeclareMathOperator{\rg}{rg}
\DeclareMathOperator{\rk}{rk}
\DeclareMathOperator{\sh}{sh}
\DeclareMathOperator{\th}{th}
\DeclareMathOperator{\tr}{tr}

% colors and page style
\definecolor{truewhite}{HTML}{ffffff}
\definecolor{white}{HTML}{faf4ed}
\definecolor{trueblack}{HTML}{000000}
\definecolor{black}{HTML}{575279}
\definecolor{mauve}{HTML}{907aa9}
\definecolor{blue}{HTML}{286983}
\definecolor{red}{HTML}{d7827e}
\definecolor{yellow}{HTML}{ea9d34}
\definecolor{gray}{HTML}{9893a5}
\definecolor{grey}{HTML}{9893a5}
\definecolor{green}{HTML}{a0d971}

\pagecolor{white}
\color{black}

\begin{asydef}
	settings.prc = false;
	settings.render=0;

	white = rgb("faf4ed");
	black = rgb("575279");
	blue = rgb("286983");
	red = rgb("d7827e");
	yellow = rgb("f6c177");
	orange = rgb("ea9d34");
	gray = rgb("9893a5");
	grey = rgb("9893a5");
	deepcyan = rgb("56949f");
	pink = rgb("b4637a");
	magenta = rgb("eb6f92");
	green = rgb("a0d971");
	purple = rgb("907aa9");

	defaultpen(black + fontsize(8pt));

	import three;
	currentlight = nolight;
\end{asydef}

% theorems, proofs, ...

\mdfsetup{skipabove=1em,skipbelow=1em, innertopmargin=6pt, innerbottommargin=6pt,}

\declaretheoremstyle[
	headfont=\normalfont\itshape,
	numbered=no,
	postheadspace=\newline,
	headpunct={:},
	qed=\qedsymbol]{demstyle}

\declaretheorem[style=demstyle, name=Démonstration]{dem}

\newcommand\veczero{\kern-1.2pt\vec{\kern1.2pt 0}} % \vec{0} looks weird since the `0' isn't italicized

\makeatletter
\renewcommand{\title}[2]{
	\AtBeginDocument{
		\begin{titlepage}
			\begin{center}
				\vspace{10cm}
				{\Large \sc Chapitre #1}\\
				\vspace{1cm}
				{\Huge \calligra #2}\\
				\vfill
				Hugo {\sc Salou} MPI${}^{\star}$\\
				{\small Dernière mise à jour le \@date }
			\end{center}
		\end{titlepage}
	}
}

\newcommand{\titletp}[4]{
	\AtBeginDocument{
		\begin{titlepage}
			\begin{center}
				\vspace{10cm}
				{\Large \sc tp #1}\\
				\vspace{1cm}
				{\Huge \textsc{\textit{#2}}}\\
				\vfill
				{#3}\textit{MPI}${}^{\star}$\\
			\end{center}
		\end{titlepage}
	}
	\fancyfoot{}\fancyhead{}
	\fancyfoot[R]{#4 \textit{MPI}${}^{\star}$}
	\fancyhead[C]{{\sc tp #1} : #2}
	\fancyhead[R]{\thepage}
}

\newcommand{\titletd}[2]{
	\AtBeginDocument{
		\begin{titlepage}
			\begin{center}
				\vspace{10cm}
				{\Large \sc td #1}\\
				\vspace{1cm}
				{\Huge \calligra #2}\\
				\vfill
				Hugo {\sc Salou} MPI${}^{\star}$\\
				{\small Dernière mise à jour le \@date }
			\end{center}
		\end{titlepage}
	}
}
\makeatother

\newcommand{\sign}{
	\null
	\vfill
	\begin{center}
		{
			\fontfamily{ccr}\selectfont
			\textit{\textbf{\.{\"i}}}
		}
	\end{center}
	\vfill
	\null
}

\renewcommand{\thefootnote}{\emph{\alph{footnote}}}

% figure support
\usepackage{import}
\usepackage{xifthen}
\pdfminorversion=7
\usepackage{pdfpages}
\usepackage{transparent}
\newcommand{\incfig}[1]{%
	\def\svgwidth{\columnwidth}
	\import{./figures/}{#1.pdf_tex}
}

\pdfsuppresswarningpagegroup=1
\ctikzset{tripoles/european not symbol=circle}

\newcommand{\missingpart}{{\large\color{red} Il manque quelque chose ici\ldots}}


\fancyhead[R]{Hugo {\sc Salou}\/ MPI}
\fancyhead[L]{TD\textsubscript4 -- Exercice 8}

\begin{document}
	\let\thesection\relax
	
\begin{comment}
\section{Exercice 9}

\slshape
Soit la matrice $A = \begin{pmatrix}
	1&1&-1\\
	2&3&-4\\
	4&1&-4
\end{pmatrix}$.
\begin{enumerate}
	\item Déterminer le spectre de la matrice $A$\/ et trouver une matrice $P$\/ inversible telle que $P^{-1} A P$\/ est diagonale.
	\item Soit $B$\/ une matrice de taille $3\times 3$\/ qui commentent avec $A$\/ ($AB = BA$). Montrer que $B$\/ est diagonale.
\end{enumerate}
\upshape

\begin{enumerate}
	\item On sait, tout d'abord, que, pour $x \in \R$,
		\begin{align*}
			\chi_A(x) = \det(x\,I_n - A) &=
			\begin{vmatrix}
				x - 1 &- 1 & 1\\
				-2&x-3&4\\
				-4&-1&x + 4
			\end{vmatrix}\\
			&= \\
		\end{align*}
\end{enumerate}
\end{comment}

\section{Exercice 8}

\begin{enumerate}
	\item Soit un vecteur non nul $\vec{x} \in \Ker(\lambda {\id} - {u \circ v})$. Ainsi, $u(v(\vec{x})) = \lambda \vec{x}$. Et, donc $v(u(v(\vec{x}))) = \lambda v(\vec{x})$. On a donc $v(\vec{x}) \in \Ker(\lambda {\id} - {v  \circ u})$.
		Or, si $\lambda \neq 0$, on a $v(\vec{x}) \neq \vec{0}$\/ ; en effet, si $v(\vec{x}) = \vec{0}$, alors $u \circ v(\vec{x}) = \vec{0} = \lambda \vec{x}$\/ et donc $\vec{x} = \vec{0}$, ce ne serait donc pas un vecteur propre de $u \circ v$\/ : une contradiction. On en déduit que $v(\vec{x})$\/ est un vecteur propre de $u \circ v$\/ associé à la valeur propre $\lambda$.
	\item On pose donc $\lambda = 0$, une valeur propre de $u  \circ v$. L'endomorphisme $u \circ v$\/ n'est donc pas injectif, donc bijectif. On sait donc, comme $E$\/ est de dimension finie, que $\det(u \circ v) = 0$. Or $\det (u \circ v) = \det u \times \det v = \det(v  \circ u)$. Et donc $\det(v  \circ u) = 0$, $v  \circ u$\/ n'est donc pas bijectif, donc injectif. Et donc, on a $0 \in \Sp(v  \circ u)$.
	\item Soit $P \in \R[X]$, et soit $Q$\/ une primitive de $P$.
		\begin{align*}
			P \in \Ker (u  \circ v) \iff& \Big(\int_{0}^X P(t)~\mathrm{d}t\Big)' = 0\\
			\iff& \big(Q(X) - Q(0)\big)' = 0\\
			\iff& Q'(X) = 0\\
			\iff& P(X) = 0
		\end{align*}
		On en déduit que $\Ker (u \circ v) = \{0\}$.

		Également,
		\begin{align*}
			P \in \Ker(v  \circ u) \iff& \int_{0}^{X} P'(t)~\mathrm{d}t = 0\\
			\iff& P(X) - P(0) = 0\\
			\iff& P(X) = P(0)\\
			\iff& \deg P \le 0\\
			\iff& P \in \R_0[X]
		\end{align*}
		On en déduit que $\Ker(v \circ u) = \R_0[X]$.
\end{enumerate}


\end{document}

		\section{Complétion d'automate}

\begin{enumerate}
	\item Non, cet automate n'est pas complet. Par exemple, la lecture d'un $b$\/ à l'état 1 est impossible.
	\item Cet automate reconnaît le langage $L = \mathcal{L}\big(a \cdot b\cdot (a \mid b)^*\big)$.
	\item~

		\begin{figure}[H]
			\centering
			\tikzfig{automate-ex9}
			\caption{Automate complet équivalent à $\mathcal{A}$}
		\end{figure}
\end{enumerate}



		\section{Exercice supplémentaire 1}

\slshape
\begin{enumerate}
	\item Montrer que l'ensemble des langages reconnaissables est stable par complémentaire.
	\item Montrer que l'ensemble des langages reconnaissables est stable par intersection.
\end{enumerate}
\upshape

\begin{enumerate}
	\item Soient $\mathcal{A} = (\Sigma, \mathcal{Q}, I, F, \delta)$\/ et $\mathcal{A}' = (\Sigma, \mathcal{Q}', I', F', \delta')$\/ deux automates déterministes complets, tels que $\mathcal{L}(\mathcal{A}) = \mathcal{L}(\mathcal{A}')$. Alors \[
		\mathcal{L}\big(\Sigma, \mathcal{Q}', I', \mathcal{Q}'\setminus F', \delta')\big) = \Sigma^* \setminus \mathcal{L}(\mathcal{A})
	.\]
	\item On utilise les lois de {\sc De Morgan}\/ en passant au complémentaire les deux automates, puis l'union (que l'on a vu en cours), et on repasse au complémentaire.
\end{enumerate}





	}
	\def\addmacros#1{#1}
}
{
	\td[4]{Langages et expressions régulières (2)}
	\minitoc
	\renewcommand{\cwd}{../td/td04/}
	\addmacros{
		\section{Exercice 1 : Vérification d'égalité polynomiale}

\begin{enumerate}
	\item Étant donnés deux tableaux représentant deux polynômes, on peut calculer leurs produit en concaténant ce tableau. La complexité du produit de polynômes avec cet algorithme est en $\mathcal{O}(nm)$\/ où $n$\/ est le degré du 1er polynôme, et $m$\/ est le degré du second. En effet, \textit{dans le pire des cas}, tous les polynômes représentant les deux polynômes sont des monômes, or, la concaténation étant en $\mathcal{O}(nm)$ (pour un tableau de taille $n$\/ et un de taille $m$). D'où la complexité en $\mathcal{O}(nm)$.
	\item Afin d'évaluer ces polynômes, on utilise l'algorithme de \textsc{Horner}, qui est en $\mathcal{O}(n)$, donc en temps linéaire.
	\item En développant ces polynômes, la complexité serait en $\mathcal{O}(n^3)$. En effet, la multiplication de deux polynômes de degrés $n$\/ a une complexité en $\mathcal{O}(n^2)$. D'où la complexité en $\mathcal{O}(n^3)$\/ pour la multiplication de deux polynômes ayant chacun un degré $n$.
	\item Un polynôme de degré $n$\/ a, au plus, $n$\/ racines. D'où, le polynôme $P - Q$, a au plus $n$\/ racines (où $n = \max(\deg P, \deg Q)$). Ainsi, s'il a $n+1$\/ racines, c'est alors le polynôme nul, et donc $P = Q$.
		\begin{algorithm}[H]
			\centering
			\begin{algorithmic}
				\State {\bf Entrée}\/ : $P = (P_{i})_{i \in \left\llbracket 1,m \right\rrbracket }$\/ et $Q = (Q_j)_{j \in \left\llbracket 1,p \right\rrbracket }$\/ deux polynômes
				\State $n \gets \deg P$\/
				\For{$i \in \left\llbracket 0,n \right\rrbracket$}
					\If{$P(i) \neq Q(i)$} \Comment{Avec l'algorithme de \textsc{Horner}, évaluation en $\mathcal{O}(n)$}
						\State \Return {\sc Non}\/
					\EndIf
				\EndFor
				\State \Return {\sc Oui}\/
			\end{algorithmic}
			\caption{Algorithme déterministe pout tester l'égalité polynomiale en $\mathcal{O}(n^2)$}
		\end{algorithm}

	\item~
		\begin{algorithm}[H]
			\centering
			\begin{algorithmic}[1]
				\Entree $P = (P_{i})_{i \in \left\llbracket 1,n \right\rrbracket }$\/ et $Q = (Q_j)_{j \in \left\llbracket 1,n \right\rrbracket }$\/ deux polynômes, et $k \in \N$\/ un entier
				\State $x \gets \mathcal{U}(\left\llbracket 1,k\times n \right\rrbracket)$\/ \phantom{$\frac00$}
				\If{$P(x) \neq Q(x)$}
					\State\Return {\sc Non}\/
				\EndIf
				\State\Return {\sc Oui}\/
			\end{algorithmic}
			\caption{Algorithme probabiliste pout tester l'égalité polynomiale en $\mathcal{O}(n)$}
		\end{algorithm}

		Soit $X$\/ la variable aléatoire de $\mathcal{U}(\left\llbracket 1,k\times n \right\rrbracket)$.
		L'événement ``$P \neq Q$\/ mais l'algorithme retourne {\sc Oui}'' arrive si $X \in \{j \in \left\llbracket 1,kn \right\rrbracket  \mid P(j) = Q(j)\} = A$. Or $|A| \le n$, et $A \subseteq \left\llbracket 1,kn \right\rrbracket$. Ainsi, l'événement a une probabilité de $\frac{1}{k}$.
\end{enumerate}

		\section{Suppression des $\varepsilon$-transitions}

\begin{figure}[H]
	\centering
	\tikzfig{ex2}
\end{figure}

		\section{Déterminisation d'automates avec $\varepsilon$-transitions}

Pour les deux automates, on commence par supprimer les $\varepsilon$-transitions, puis on le déterminise.

\begin{enumerate}
	\item L'automate équivalent sans $\varepsilon$-transitions est le suivant.
		\begin{figure}[H]
			\centering
			\tikzfig{ex3-1a}
		\end{figure}
		Une fois déterminisé, on obtient l'automate ci-dessous.
		\begin{figure}[H]
			\centering
			\tikzfig{ex3-1a-det}
		\end{figure}
	\item L'automate équivalent, sans $\varepsilon$-transitions, est le suivant.
		\begin{figure}[H]
			\centering
			\tikzfig{ex3-1b}
		\end{figure}
		Une fois déterminisé, on obtient l'automate ci-dessous.
		\begin{figure}[H]
			\centering
			\tikzfig{ex3-1b-det}
		\end{figure}
\end{enumerate}


		\section{Exercice 4}

\paragraph{Q.\ 1}

\begin{algo}
	\textsl{Entrée} : Un automate $\mathcal{A}$\/ ;\\
	\textsl{Sortie} : $\mathcal{L}(\mathcal{A}) = \O$\/ ;\\
	On fait un parcours en largeur depuis les états initiaux et on regarde si on atteint un état final.
\end{algo}

\begin{algo}[Nathan F.]
	{\sl Entrée} : Deux automates $\mathcal{A}$\/ et $\mathcal{B}$\/ \\
	{\sl Sortie} : $\mathcal{L}(\mathcal{A}) = \mathcal{L}(\mathcal{B})$\/ ;
	Soit $\mathcal{C}$\/ l'automate reconnaissant $\mathcal{L}(\mathcal{A}) \mathbin\triangle \mathcal{L}(\mathcal{B})$. On retourne $\mathcal{L}(\mathcal{C}) \mathrel{\ds\mathop=^?} \O$\/ à l'aide de l'algorithme précédent.
\end{algo}

Autre possibilité, on procède par double inclusion :

\begin{algo}[$\subseteq$]
	{\sl Entrée} : Deux automates $\mathcal{A}$\/ et $\mathcal{B}$\/ \\
	{\sl Sortie} : $\mathcal{L}(\mathcal{A}) \subseteq \mathcal{L}(\mathcal{B})$\/ ;
	On retourne $\mathcal{A} \setminus \mathcal{B} \mathrel{\ds\mathop=^?} \O$.
\end{algo}

\paragraph{Q.\ 2}
L'algorithme reconnaissant $\mathcal{L}(\mathcal{A}) \mathrel\triangle \mathcal{L}(\mathcal{B})$\/ doit être déterminisé, sa complexité est donc au moins de $2^n$.

		\section{Exercice 5}

\paragraph{1.}

On a $e = a(ab \mid b^*)  \mid a$, $f = a_1(a_2b_1 \mid b_2^*)  \mid a_3$\/et $f_\varphi = e$\/ où \[
	\varphi : \left(\begin{array}{ccc}
		\forall i,\:a_i&\longmapsto&a\\
		\forall i,\:b_i&\longmapsto&b
	\end{array}\right)
.\]
D'où
\[
	\begin{array}{c|c|c|c|c}
		&\Lambda&P&S&F\\\hline
		a_1&\O&a_1&a_1&\O\\
		a_2&\O&a_2&a_2&\O\\
		b_2^*&\varepsilon&b_2&b_2&b_1b_2\\
		a_2b_1 \mid b_2^*&\varepsilon&a_2,b_2&b_1,b_2&a_2b_1,b_2b_2\\
		a_3&\O&a_3&a_3&\O\\
		a_1(a_2b_1 \mid b_2^*)&\O&a_1&b_1,b_2,a_1&a_1a_2,a_1b_2,a_2b_1,b_2b_2\\
		f&\O&a_1,a_3&b_1,b_2,a_3,a_1&a_1a_2,a_1b_2,a_3b_1,b_2b_2
	\end{array}
.\]

Automate à faire\ldots

\paragraph{2.}
On pose $e = (\varepsilon  \mid a)^* \cdot ab\cdot (a \mid b)^*$\/ et $f = (\varepsilon  \mid a_1)^* \cdot a_2b_1\cdot (a_3 \mid b_2)^*$\/ et \[
	\varphi : \left(\begin{array}{ccc}
		\forall i,\:a_i&\longmapsto&a\\
		\forall i,\:b_i&\longmapsto&b
	\end{array}\right)
\] d'où $f_\varphi = e$.

	}
	\def\addmacros#1{#1}
}
{
	\td[5]{Langages et expressions régulières (3)}
	\minitoc
	\renewcommand{\cwd}{../td/td05/}
	\addmacros{
		\section{Exercice 1 : Vérification d'égalité polynomiale}

\begin{enumerate}
	\item Étant donnés deux tableaux représentant deux polynômes, on peut calculer leurs produit en concaténant ce tableau. La complexité du produit de polynômes avec cet algorithme est en $\mathcal{O}(nm)$\/ où $n$\/ est le degré du 1er polynôme, et $m$\/ est le degré du second. En effet, \textit{dans le pire des cas}, tous les polynômes représentant les deux polynômes sont des monômes, or, la concaténation étant en $\mathcal{O}(nm)$ (pour un tableau de taille $n$\/ et un de taille $m$). D'où la complexité en $\mathcal{O}(nm)$.
	\item Afin d'évaluer ces polynômes, on utilise l'algorithme de \textsc{Horner}, qui est en $\mathcal{O}(n)$, donc en temps linéaire.
	\item En développant ces polynômes, la complexité serait en $\mathcal{O}(n^3)$. En effet, la multiplication de deux polynômes de degrés $n$\/ a une complexité en $\mathcal{O}(n^2)$. D'où la complexité en $\mathcal{O}(n^3)$\/ pour la multiplication de deux polynômes ayant chacun un degré $n$.
	\item Un polynôme de degré $n$\/ a, au plus, $n$\/ racines. D'où, le polynôme $P - Q$, a au plus $n$\/ racines (où $n = \max(\deg P, \deg Q)$). Ainsi, s'il a $n+1$\/ racines, c'est alors le polynôme nul, et donc $P = Q$.
		\begin{algorithm}[H]
			\centering
			\begin{algorithmic}
				\State {\bf Entrée}\/ : $P = (P_{i})_{i \in \left\llbracket 1,m \right\rrbracket }$\/ et $Q = (Q_j)_{j \in \left\llbracket 1,p \right\rrbracket }$\/ deux polynômes
				\State $n \gets \deg P$\/
				\For{$i \in \left\llbracket 0,n \right\rrbracket$}
					\If{$P(i) \neq Q(i)$} \Comment{Avec l'algorithme de \textsc{Horner}, évaluation en $\mathcal{O}(n)$}
						\State \Return {\sc Non}\/
					\EndIf
				\EndFor
				\State \Return {\sc Oui}\/
			\end{algorithmic}
			\caption{Algorithme déterministe pout tester l'égalité polynomiale en $\mathcal{O}(n^2)$}
		\end{algorithm}

	\item~
		\begin{algorithm}[H]
			\centering
			\begin{algorithmic}[1]
				\Entree $P = (P_{i})_{i \in \left\llbracket 1,n \right\rrbracket }$\/ et $Q = (Q_j)_{j \in \left\llbracket 1,n \right\rrbracket }$\/ deux polynômes, et $k \in \N$\/ un entier
				\State $x \gets \mathcal{U}(\left\llbracket 1,k\times n \right\rrbracket)$\/ \phantom{$\frac00$}
				\If{$P(x) \neq Q(x)$}
					\State\Return {\sc Non}\/
				\EndIf
				\State\Return {\sc Oui}\/
			\end{algorithmic}
			\caption{Algorithme probabiliste pout tester l'égalité polynomiale en $\mathcal{O}(n)$}
		\end{algorithm}

		Soit $X$\/ la variable aléatoire de $\mathcal{U}(\left\llbracket 1,k\times n \right\rrbracket)$.
		L'événement ``$P \neq Q$\/ mais l'algorithme retourne {\sc Oui}'' arrive si $X \in \{j \in \left\llbracket 1,kn \right\rrbracket  \mid P(j) = Q(j)\} = A$. Or $|A| \le n$, et $A \subseteq \left\llbracket 1,kn \right\rrbracket$. Ainsi, l'événement a une probabilité de $\frac{1}{k}$.
\end{enumerate}

		\section{Exercice 4}

\paragraph{Q.\ 1}

\begin{algo}
	\textsl{Entrée} : Un automate $\mathcal{A}$\/ ;\\
	\textsl{Sortie} : $\mathcal{L}(\mathcal{A}) = \O$\/ ;\\
	On fait un parcours en largeur depuis les états initiaux et on regarde si on atteint un état final.
\end{algo}

\begin{algo}[Nathan F.]
	{\sl Entrée} : Deux automates $\mathcal{A}$\/ et $\mathcal{B}$\/ \\
	{\sl Sortie} : $\mathcal{L}(\mathcal{A}) = \mathcal{L}(\mathcal{B})$\/ ;
	Soit $\mathcal{C}$\/ l'automate reconnaissant $\mathcal{L}(\mathcal{A}) \mathbin\triangle \mathcal{L}(\mathcal{B})$. On retourne $\mathcal{L}(\mathcal{C}) \mathrel{\ds\mathop=^?} \O$\/ à l'aide de l'algorithme précédent.
\end{algo}

Autre possibilité, on procède par double inclusion :

\begin{algo}[$\subseteq$]
	{\sl Entrée} : Deux automates $\mathcal{A}$\/ et $\mathcal{B}$\/ \\
	{\sl Sortie} : $\mathcal{L}(\mathcal{A}) \subseteq \mathcal{L}(\mathcal{B})$\/ ;
	On retourne $\mathcal{A} \setminus \mathcal{B} \mathrel{\ds\mathop=^?} \O$.
\end{algo}

\paragraph{Q.\ 2}
L'algorithme reconnaissant $\mathcal{L}(\mathcal{A}) \mathrel\triangle \mathcal{L}(\mathcal{B})$\/ doit être déterminisé, sa complexité est donc au moins de $2^n$.

		\section{Exercice 5}

\paragraph{1.}

On a $e = a(ab \mid b^*)  \mid a$, $f = a_1(a_2b_1 \mid b_2^*)  \mid a_3$\/et $f_\varphi = e$\/ où \[
	\varphi : \left(\begin{array}{ccc}
		\forall i,\:a_i&\longmapsto&a\\
		\forall i,\:b_i&\longmapsto&b
	\end{array}\right)
.\]
D'où
\[
	\begin{array}{c|c|c|c|c}
		&\Lambda&P&S&F\\\hline
		a_1&\O&a_1&a_1&\O\\
		a_2&\O&a_2&a_2&\O\\
		b_2^*&\varepsilon&b_2&b_2&b_1b_2\\
		a_2b_1 \mid b_2^*&\varepsilon&a_2,b_2&b_1,b_2&a_2b_1,b_2b_2\\
		a_3&\O&a_3&a_3&\O\\
		a_1(a_2b_1 \mid b_2^*)&\O&a_1&b_1,b_2,a_1&a_1a_2,a_1b_2,a_2b_1,b_2b_2\\
		f&\O&a_1,a_3&b_1,b_2,a_3,a_1&a_1a_2,a_1b_2,a_3b_1,b_2b_2
	\end{array}
.\]

Automate à faire\ldots

\paragraph{2.}
On pose $e = (\varepsilon  \mid a)^* \cdot ab\cdot (a \mid b)^*$\/ et $f = (\varepsilon  \mid a_1)^* \cdot a_2b_1\cdot (a_3 \mid b_2)^*$\/ et \[
	\varphi : \left(\begin{array}{ccc}
		\forall i,\:a_i&\longmapsto&a\\
		\forall i,\:b_i&\longmapsto&b
	\end{array}\right)
\] d'où $f_\varphi = e$.

		\section{Exercice 6 : Langages reconnaissables ou non}

\paragraph{Q.\ 7}
{\slshape Le carré d'un langage est le langage $L_2 = \{u\cdot u \mid u \in L\}$. Si $L$\/ est reconnaissable, $L_2$\/ est-il nécessairement reconnaissable ?}

Avec $\Sigma = \{a,b\}$, soit $L = \mathcal{L}(a^* \cdot b^*)$. On a donc $L_2 = \{a^n \cdot b^m \cdot a^n \cdot b^m  \mid (n,m) \in \N^2\}$. Supposons $L_2$\/ reconnaissable. Soit $\mathcal{A}$\/ un automate à $n$\/ états reconnaissant $L_2$.
On pose $u = a^{2n} \cdot b^n \cdot a^{2n} \cdot b^n \in L_2$. D'après le lemme de l'étoile, il existe $(x,y,z) \in (\Sigma^*)^3$\/ tel que $u = x\cdot y\cdot z$, $|xy| \le n$, $\mathcal{L}(x\cdot y^* \cdot z) \subseteq L_2$, et $y \neq \varepsilon$. Ainsi, il existe $m \in \left\llbracket 1,n \right\rrbracket$\/ et $p \in \left\llbracket 1,n \right\rrbracket$\/ tels que $y = a^{m}$, $x = a^{p}$\/ et $z = a^{2n-m-p} \cdot  b^{n} \cdot a^{2n} \cdot b^n$. Et alors, $x\cdot y^2\cdot z = a^{p}\cdot a^{2m} \cdot a^{n-m-p} \cdot  b^n \cdot a^{2n}\cdot b^n = a^{2n+m} \cdot b^n \cdot a^{2n} \cdot b^n \not\in L_2$.

\paragraph{Q.\ 5} {\slshape Le langage $L_5 = \{a^{n^3}  \mid n \in \N\}$\/ est-il reconnaissable ?}
Soit $\mathcal{A}$\/ un automate à $N$\/ états, et soit $u = a^{N^3}$.
D'après le lemme de l'étoile, il existe $(x,y,z) \in (\Sigma^*)^3$\/ tel que $u = x\cdot y\cdot z$, $|xy| \le N$, $\mathcal{L}(x\cdot y^*\cdot z) \subseteq L_5$\/ et $y \neq \varepsilon$.
D'où $x\cdot y^{0}\cdot z \in L$, et donc $a^{N^3 - i} \in L$, avec $i\le N$. Or, $\forall k \in \N,\:N^3 - i \neq k^3$, ce qui est absurde.


	}
	\def\addmacros#1{#1}
}
{
	\td[6]{Algorithmes probabilistes}
	\minitoc
	\renewcommand{\cwd}{../td/td06/}
	\addmacros{
		\section{Exercice 1 : Vérification d'égalité polynomiale}

\begin{enumerate}
	\item Étant donnés deux tableaux représentant deux polynômes, on peut calculer leurs produit en concaténant ce tableau. La complexité du produit de polynômes avec cet algorithme est en $\mathcal{O}(nm)$\/ où $n$\/ est le degré du 1er polynôme, et $m$\/ est le degré du second. En effet, \textit{dans le pire des cas}, tous les polynômes représentant les deux polynômes sont des monômes, or, la concaténation étant en $\mathcal{O}(nm)$ (pour un tableau de taille $n$\/ et un de taille $m$). D'où la complexité en $\mathcal{O}(nm)$.
	\item Afin d'évaluer ces polynômes, on utilise l'algorithme de \textsc{Horner}, qui est en $\mathcal{O}(n)$, donc en temps linéaire.
	\item En développant ces polynômes, la complexité serait en $\mathcal{O}(n^3)$. En effet, la multiplication de deux polynômes de degrés $n$\/ a une complexité en $\mathcal{O}(n^2)$. D'où la complexité en $\mathcal{O}(n^3)$\/ pour la multiplication de deux polynômes ayant chacun un degré $n$.
	\item Un polynôme de degré $n$\/ a, au plus, $n$\/ racines. D'où, le polynôme $P - Q$, a au plus $n$\/ racines (où $n = \max(\deg P, \deg Q)$). Ainsi, s'il a $n+1$\/ racines, c'est alors le polynôme nul, et donc $P = Q$.
		\begin{algorithm}[H]
			\centering
			\begin{algorithmic}
				\State {\bf Entrée}\/ : $P = (P_{i})_{i \in \left\llbracket 1,m \right\rrbracket }$\/ et $Q = (Q_j)_{j \in \left\llbracket 1,p \right\rrbracket }$\/ deux polynômes
				\State $n \gets \deg P$\/
				\For{$i \in \left\llbracket 0,n \right\rrbracket$}
					\If{$P(i) \neq Q(i)$} \Comment{Avec l'algorithme de \textsc{Horner}, évaluation en $\mathcal{O}(n)$}
						\State \Return {\sc Non}\/
					\EndIf
				\EndFor
				\State \Return {\sc Oui}\/
			\end{algorithmic}
			\caption{Algorithme déterministe pout tester l'égalité polynomiale en $\mathcal{O}(n^2)$}
		\end{algorithm}

	\item~
		\begin{algorithm}[H]
			\centering
			\begin{algorithmic}[1]
				\Entree $P = (P_{i})_{i \in \left\llbracket 1,n \right\rrbracket }$\/ et $Q = (Q_j)_{j \in \left\llbracket 1,n \right\rrbracket }$\/ deux polynômes, et $k \in \N$\/ un entier
				\State $x \gets \mathcal{U}(\left\llbracket 1,k\times n \right\rrbracket)$\/ \phantom{$\frac00$}
				\If{$P(x) \neq Q(x)$}
					\State\Return {\sc Non}\/
				\EndIf
				\State\Return {\sc Oui}\/
			\end{algorithmic}
			\caption{Algorithme probabiliste pout tester l'égalité polynomiale en $\mathcal{O}(n)$}
		\end{algorithm}

		Soit $X$\/ la variable aléatoire de $\mathcal{U}(\left\llbracket 1,k\times n \right\rrbracket)$.
		L'événement ``$P \neq Q$\/ mais l'algorithme retourne {\sc Oui}'' arrive si $X \in \{j \in \left\llbracket 1,kn \right\rrbracket  \mid P(j) = Q(j)\} = A$. Or $|A| \le n$, et $A \subseteq \left\llbracket 1,kn \right\rrbracket$. Ainsi, l'événement a une probabilité de $\frac{1}{k}$.
\end{enumerate}

		\section{Suppression des $\varepsilon$-transitions}

\begin{figure}[H]
	\centering
	\tikzfig{ex2}
\end{figure}

		\section{Déterminisation d'automates avec $\varepsilon$-transitions}

Pour les deux automates, on commence par supprimer les $\varepsilon$-transitions, puis on le déterminise.

\begin{enumerate}
	\item L'automate équivalent sans $\varepsilon$-transitions est le suivant.
		\begin{figure}[H]
			\centering
			\tikzfig{ex3-1a}
		\end{figure}
		Une fois déterminisé, on obtient l'automate ci-dessous.
		\begin{figure}[H]
			\centering
			\tikzfig{ex3-1a-det}
		\end{figure}
	\item L'automate équivalent, sans $\varepsilon$-transitions, est le suivant.
		\begin{figure}[H]
			\centering
			\tikzfig{ex3-1b}
		\end{figure}
		Une fois déterminisé, on obtient l'automate ci-dessous.
		\begin{figure}[H]
			\centering
			\tikzfig{ex3-1b-det}
		\end{figure}
\end{enumerate}


	}
	\def\addmacros#1{#1}
}
{
	\td[7]{Décidabilité, Calculabilité}
	\minitoc
	\renewcommand{\cwd}{../td/td07/}
	\addmacros{
		\section{Quelques problèmes décidables}

\begin{enumerate}
	\item Soit $f : \R \to \R$.
		\begin{itemize}
			\item Si $f$\/ admet un zéro, on pose $\mathcal{M} = \texttt{fun}\ \texttt{s}\ \to \texttt{true}$.
			\item Si $f$\/ n'admet pas un zéro, on pose $\mathcal{M} = \texttt{fun}\ \texttt{s}\ \to \texttt{false}$.
		\end{itemize}
		Alors, $\mathcal{M}$\/ décide \textsc{Zero}$_f$.
	\item Soit $\mathcal{M}$\/ une machine, et soit $w \in \Sigma^*$.
		\begin{itemize}
			\item Si $\mathcal{M}$\/ se termine sur l'entrée $w$, alors on pose $\mathcal{M}' = \texttt{fun}\ \texttt{s} \to \texttt{true}$.
			\item Si $\mathcal{M}$\/ ne se termine pas sur l'entrée $w$, alors on pose $\mathcal{M}' = \texttt{fun}\ \texttt{s} \to \texttt{false}$.
		\end{itemize}
		Alors, $\mathcal{M}'$\/ décide \textsc{Arrêt}$_{\mathcal{M},w}$.
	\item Le problème est trivialement vrai. En effet, soit $M \in \mathcal{O}$, de la forme
		\begin{lstlisting}[language=caml]
let m (s: string): string =
	%*$\langle$\textrm{code}$\rangle$*) 
		\end{lstlisting}
		On crée la machine $\mathcal{N}$\/ ci-dessous.
		\begin{lstlisting}[language=caml]
let n (s: string): string =
	if true then
		%*$\langle$\textrm{code}$\rangle$*) 
	else
		%*$\langle$\textrm{code}$\rangle$*) 
		\end{lstlisting}
		On a $\texttt{m} \neq \texttt{n}$, mais $\mathcal{L}(\texttt{m}) = \mathcal{L}(\texttt{n})$, donc le problème est vrai sur toute entrée et la fonction $\texttt{fun}\ \texttt{s} \to \texttt{true}$\/ répond au problème.
\end{enumerate}

		\section{Ensembles définis inductivement}

La correction est disponible sur \textit{cahier-de-prepa}.

\begin{comment}
	\begin{exm}
		Avec $S = \N$, $\mathcal{B} = \{0, 2\} $, $A_1 = \{0\}$\/ et \begin{align*}
			f_1: A_1 \times \N &\longrightarrow \N \\
			(0, x) &\longmapsto x + 4.
		\end{align*}

		On a \[
			X \supseteq \{0, 2, 4, 6, 8, 10, \ldots, 20, \ldots\} = 2\N
		.\]
	\end{exm}
	\begin{exm}
		Avec $S$\/ l'ensemble des langages sur $\Sigma$, $\mathcal{B} = \{\O\} \cup \bigl\{\{a\}\:\big|\: a \in \Sigma \bigr\}$, et
		\begin{multicols}{3}
			\begin{align*}
				f_1: S \times S &\longrightarrow S \\
				(L_1, L_2) &\longmapsto L_1 \cup L_2,
			\end{align*}
			\begin{align*}
				f_2: S \times S &\longrightarrow S \\
				(L_1, L_2) &\longmapsto L_1 \cdot L_2,
			\end{align*}
			\begin{align*}
				f_3: S &\longrightarrow S \\
				L &\longmapsto L^*.
			\end{align*}
		\end{multicols}
	\end{exm}

\begin{enumerate}
	\item Soit $\mathcal{A} = \{X \subseteq S  \mid X \supseteq \mathcal{B} \mathrel{\text{et}} X \text{ est stable par } f_i\}$. On a $S \in \mathcal{A}$\/ et donc $\mathcal{A} \neq \O$. De plus, soit \[
			Y = \{x \in S  \mid \forall X \in \mathcal{A},\,x \in X\} = \bigcap_{X \in \mathcal{A}} X
		.\]
		Soit $b \in \mathcal{B}$, on a $\forall X \in A,\, b \in X$. D'où $b \in Y$\/ par intersection. On en déduit que $\mathcal{B} \subseteq Y$.

		Soit $i \in \left\llbracket 1,m \right\rrbracket$. Soit $(x_1, \ldots, x_{n_i}) \in Y^{n_i}$\/ et soit $a \in A_i$. Montrons que $f_i(a, x_1, \ldots, x_{n_i}) \in Y$.
		Or, soit $X \in \mathcal{A}$, on a $(x_1, \ldots, x_{n_i}) \in X^{n_i}$\/ donc $f_i(a, x_1, \ldots, x_{n_i}) \in X$. Ceci étant vrai pour tout $X \in \mathcal{A}$, on a $f_i(a, x_1, \ldots, x_{n_i}) \in Y$\/ donc $Y$\/ est stable par $f_i$\/ par tout $i \in \left\llbracket 1,m \right\rrbracket$\/ et donc $Y \in \mathcal{A}$.
		On a également $Y \subseteq X$\/ pour tout $X \in \mathcal{A}$. On en déduit que $Y$\/ est le plus petit élément (pour l'inclusion) de $\mathcal{A}$.
	\item On pose $X_0  = \mathcal{B}$\/ et \[
			X_{n+1} = X_n \cup \big\{ f_i(a, x_1, \ldots, x_{n_i})  \mid a \in A_i,\,(x_1, \ldots, x_{n_i}) \in (X_n)^{n_i},\,i \in \left\llbracket 1,m \right\rrbracket\big\}
		.\]
		Soit $X = \bigcup_{n \in \N} X_n$. Soit $Y$\/ l'ensemble défini par induction à partir de $\mathcal{B}$\/ et des $(f_i)_{i\in\left\llbracket 1,n \right\rrbracket}$. Montrons que $X = Y$.
		On montre que $X$\/ est le plus petit élément (pour l'inclusion) de $\mathcal{A}$\/ et on conclut par unicité du minimum (avec la question précédente).
		Par définition de la suite $(X_n)_{n\in\N}$, elle est croissante (au sens de l'inclusion).
		Montrons à présent, par récurrence, la propriété ci-dessous : $P_n : ``X_n \subseteq Y."$
		\begin{itemize}
			\item Par définition de $Y$, on a $X_0 = \mathcal{B} \subseteq Y$.
			\item Soit 
		\end{itemize}
\end{enumerate}

\subsection{Un théorème d'induction}

\begin{enumerate}
	\item[3.] Soit $Z = \{x \in S  \mid P(x) \text{ vraie}\:\}$.
		Montrons que $\mathcal{X}\subseteq Z$.
		On remarque que $\mathcal{X} \supseteq \mathcal{B}$\/ ; $\mathcal{X}$\/ est stable par $f_i$. On en conclut que $Z \supseteq \mathcal{X}$ et donc $\forall x \in \mathcal{X},\,P(x)$\/ est vraie.
\end{enumerate}

\begin{exm}
	Soit $\mathcal{X}$\/ défini par induction par $\mathcal{B} = \{0, 2\}$\/ et \begin{align*}
		f: \N &\longrightarrow \N \\
		n &\longmapsto n + 2.
	\end{align*}
	Montrons que $\forall n \in \mathcal{X}$, $x$\/ est pair.

	On sait que $0$\/ est pair, $2$\/ est pair ; et, \[
		\forall x,y \in \mathcal{X},\, (x \text{ pair} \land y \text{ pair}) \implies f(x, y) \text{ pair}
	.\]

	On en déduit que \[
		\forall n \in \mathcal{X},\,x \text{ est pair}.
	.\]
\end{exm}
\end{comment}

		\section{Tableaux dynamiques}

\begin{enumerate}[start=3]
	\item On trouve une complexité amortie en $n^2$. À rédiger.
	\item Au lieu de diviser quand $r < n / 2$, mais quand $r < n / 4$.
\end{enumerate}

		\section{Barres en triangle}


On note $a(x)$ le côté du triangule équilatéral, et donc $x = a(x) \sqrt{3}  / 2$.

On calcule le flux $\Phi$ magnétique : \[
	\Phi = B \frac{a x}{2} = B x^2 / \sqrt{3}
.\]
Ainsi, d'après la loi de \textsc{Faraday}, on a \[
	e = - \frac{\mathrm{d}\Phi}{\mathrm{d}t} = - \frac{2B}{\sqrt{3}}  \: x\,\dot{x}
.\]
Or, par loi d'\textsc{Ohm}, $i = e / R(x)$.
Et, on connait la resistance du circuit $R(x) = 3 a(x) / \gamma S$.
Alors, 
\begin{align*}
	i(x) &= \frac{-2B\,x\,\dot{x}}{\sqrt{3} \cdot \frac{3a(x)}{\gamma S}} = - \frac{2B \gamma S}{3 \sqrt{3}} \cdot \frac{x\,\dot{x}}{2 \frac{x}{\sqrt{3}}}\\
	&= - B \gamma S \dot{x} / 3. \\
\end{align*}
On calcule donc la force de \textsc{Laplace} :
\begin{align*}
	\vec{F}_{\mathcal{L}} &= i \cdot [\mathrm{CD}] \cdot B \cdot \vec{e}_x\\
	&= i \cdot \frac{2\dot{x}}{\sqrt{3}} B \vec{e}_x \\
	&= -\underbrace{\frac{2 B^2 \gamma S}{3\sqrt{3}}}_\alpha  \cdot x \dot{x} \\
\end{align*}

D'après le \textsc{pfd}, on a donc \[
	m \ddot{x} = - \alpha x \dot{x} \text{ d'où }  \ddot{x} = - \frac{\alpha}{m} x\dot{x} 
.\] où $m = \rho S L$. 

On intègre les deux côtés de l'équation, \[
	[\dot{x}]_0^{t_\mathrm{f}} = -\frac{\alpha}{m} \cdot \left[ \frac{x^2}{2} \right]_0^{t_\mathrm{f}}
.\]
D'où, \[
	0 - v_0 = \frac{-\alpha}{m} \cdot x_\mathrm{f}^2 / 2
.\] On en conclut \[
	x_\mathrm{f} = \sqrt{\frac{2m}{\alpha} v_0} 
.\] 


		\begin{multicols}{2}
	\section{Implications}
	\begin{enumerate}
		\item
			\[
				\begin{prooftree}
					\infer 0[Ax]{p,q \vdash q}
					\infer 1[$\to$i]{q \vdash p \to q}
				\end{prooftree}
			\]
		\item
			\[
				\begin{prooftree}
					\infer 0[Ax]{p, p\land q\vdash p\land q}
					\infer 1[$\land$e,d]{p, p \land q \vdash q}
					\infer 1[$\to$i]{p \land q \vdash p \to q}
				\end{prooftree}
			\]
		\item 
			\[
				\begin{prooftree}
					\infer 0[Ax]{p, p\to q \vdash p}
					\infer 0[Ax]{p, p\to q \vdash p \to q}
					\infer 0[Ax]{p, p\to q \vdash p}
					\infer 2[$\to$e]{p, p \to q \vdash q}
					\infer 2[$\land$i]{p, p \to q \vdash p \land q}
					\infer 1[$\to$i]{p \to q \vdash p\to (p \land q)}
				\end{prooftree}
			\]
		\item
			\[
				\begin{prooftree}
					\infer 0[Ax]{\lnot q, p\to q, p \vdash p \to q}
					\infer 0[Ax]{\lnot q, p\to q, p \vdash p}
					\infer 2[$\to$e]{\lnot q, p\to q, p \vdash q}
					\infer 0[Ax]{\lnot q, p\to q, p \vdash \lnot q}
					\infer 2[$\lnot$e]{\lnot q, p\to q, p \vdash \bot}
					\infer 1[$\lnot$i]{\lnot q, p\to q \vdash\lnot p}
					\infer 1[$\to$i]{p \to q \vdash \lnot q \to \lnot p}
				\end{prooftree}
			\]
		\item 
			\[
				\begin{prooftree}
					\infer 0[Ax]{p \land q, p\to r \vdash p \land q}
					\infer 1[$\land$e,g]{p \land q, p\to r \vdash p}
					\infer 0[Ax]{p \land q, p\to r \vdash p \to r}
					\infer 2[$\to$e]{p \land q, p \to r \vdash r}
					\infer 1[$\to$i]{p \to r \vdash (p \land q) \to r}
				\end{prooftree}
			\] 
		\item
			\[
				\begin{prooftree}
					\infer 0[Ax]{p,q\vdash p}
					\infer 1[$\to$i]{p \vdash q \to p}
					\infer 1[$\to$i]{\vdash p \to (q \to p)}
				\end{prooftree}
			\]
		\item
			\[
				\begin{prooftree}
					\infer 0[Ax]{p \to q, p \vdash p\to q}
					\infer 0[Ax]{p\to q,p \vdash p}
					\infer 2[$\to$e]{p, p\to q \vdash q}
					\infer 1[$\to$i]{p \vdash (p \to q) \to q}
				\end{prooftree}
			\] 
	\end{enumerate}
\end{multicols}

		\section{Langage de \textsc{Dyck}}

\begin{enumerate}
	\item On suppose ce langage reconnaissable par un automate à $n$ états. On considère le mot $w = {\red(}^n \cdot {\red)}^n$, donc $|w| \ge n$.
		Ainsi, il existe $x$, $y$ et $z$ trois mots tels que $w = xyz$, $|xy| \le n$, $y \neq \varepsilon$ et $\forall p \in \N,\: x y^p z \in \mathcal{L}(\mathcal{G})$.
		Soit alors $p \in \llbracket 1,n-1 \rrbracket$ et $q \in \llbracket 1,n -p \rrbracket$ tels que $x = {\red(}^p$, $y = {\red(}^q$ et $z = {\red(}^{n-q-p} \cdot {\red)}^n$.
		Ainsi, $xy \in \mathcal{L}(\mathcal{G})$, ce qui est absurde. On en déduit que $\mathcal{L}(\mathcal{G})$ n'est pas reconnaissable, il n'est donc pas régulier.
	\item On pose $\mathcal{G} = (\Sigma, \{\mathrm{S}\}, \{\mathrm{S} \to \red( \mathrm{S}\red)  \mid \mathrm{SS}  \mid \varepsilon\}, \mathrm{S})$.
	\item
		\begin{itemize}
			\item On le montre par induction.
				\begin{itemize}
					\item \textbf{Cas $\mathrm{S} \to \varepsilon$.}
						On a $|\varepsilon|_{\red(} = 0 = |\varepsilon|_{\red)}$.
					\item \textbf{Cas $\mathrm{S} \to \red( \mathrm{S}\red)$.}
						Soit $u \in \mathcal{L}(\mathcal{G})$ avec $|u|_{\red(} = |u|_{\red)} = n$. Ainsi, $|\red(u\red)|_{\red(} = |\red(u\red)|_{\red)} = n + 1$.
					\item \textbf{Cas $\mathrm{S}\to \mathrm{SS}$.}
						Soient $u$ et $v$ deux mots de $\mathcal{L}(\mathcal{G})$ tels que $|u|_{\red(} = |u|_{\red)} = n$ et $|v|_{\red(} = |v|_{\red)} = m$.
						Alors, $|u\cdot v|_{\red(} = |v\cdot u|_{\red)} = n + m$.
				\end{itemize}
			\item Montrons par induction $\mathcal{P}_{u}$ : \guillemotleft~pour tout $v$ préfixe de $u$, $|v|_{\red(} \ge |v|_{\red)}$.~\guillemotright\@ 
				\begin{itemize}
					\item \textbf{Cas $\mathrm{S} \to \varepsilon$.}
						Le seul préfixe de $\varepsilon$ est $\varepsilon$, et on a bien $|\varepsilon|_{\red(} = 0 \ge 0 = |\varepsilon|_{\red)}$.
					\item \textbf{Cas $\mathrm{S}\to \red( \mathrm{S} \red)$.}
						Soit $u$ un mot de $\mathcal{L}(\mathcal{G})$ vérifiant $\mathcal{P}_u$.
						Soit $v$ un préfixe de $\red(u\red)$.
						On procède par induction sur $v$.
						\begin{itemize}
							\item \textbf{Cas $v = \varepsilon$ ou $\red($.} \textsc{ok}.
							\item \textbf{Cas $\red(\tilde{u}$,} où $\tilde{u}$ est un préfixe de $u$.
								Par hypothèse d'induction, $|\tilde{u}|_{\red(} \ge |\tilde{u}|_{\red)}$ donc $|\red(\tilde{u}|_{\red(} = |\red(\tilde{u}|_{\red)}$.
							\item \textbf{Cas $\red(u\red)$.} Par hypothèse d'induction, $|u|_{\red(} \ge |u|_{\red)}$ donc $|\red(u\red)|_{\red(}\ge |\red(u\red)|_{\red)}$.
						\end{itemize}
				\end{itemize}
		\end{itemize}
	\item On note $\overline w^j= \big| w_{\llbracket 0,j \rrbracket} \big|_{\red(} - \big| w_{\llbracket 0,j \rrbracket} \big|_{\red)}$.
		Alors les deux conditions se traduisent par $\overline w^{|w|} = 0$ et $\forall i \in \llbracket 0,|w|-1 \rrbracket$, $\overline w^i \ge 0$.
\end{enumerate}

		\section{$\mathcal{N}$}

\newcommand{\ofact}{\charfusion[\mathbin]{\bigcirc}{\scriptstyle!}}

\begin{enumerate}
	\item On définit par induction la fonction suivante \begin{align*}
			\oplus: \mathcal{N}^2 &\longrightarrow \mathcal{N} \\
			(\mathbf{S}(x),y) &\longmapsto \oplus(x, \mathbf{S}(y))\\
			(\mathbf{0}, x) &\longmapsto x.
		\end{align*}
	\item Soit $(x,y) \in \mathcal{N}^2$.
		\begin{itemize}
			\item Si $f(x) = 0$, alors $\oplus(x,y) = y$\/ et donc $f(\oplus(x,y)) = f(y) = f(x) + f(y)$.
			\item Si $f(x) \ge 1$, alors  $x = \mathbf{S}(z)$\/ avec $z \in \mathcal{N}$. Ainsi, $\oplus(x,y) = \oplus(z, \mathbf{S}(y))$. Or, $f(z) = f(x) - 1 \le f(x)$. Et donc, par définition de $\oplus$\/ puis par hypothèse d'induction, on a $f(\oplus(x,y)) = f(\oplus(z, \mathbf{S}(y)) = f(z) + f(\mathbf{S}(y))$. On en déduit que $f(\oplus(x,y)) = f(x) - 1 + f(y) + 1 = f(x) + f(y)$.
		\end{itemize}
		Par induction, on a bien $\forall (x,y) \in \mathcal{N}^2,\:f\big({\oplus}(x,y)\big) = f(x) + f(y)$.
	\item On définit par induction la fonction suivante \begin{align*}
			\otimes: \mathcal{N}^2 &\longrightarrow \mathcal{N} \\
			(\mathbf{S}(x), y) &\longmapsto {\oplus}\big(y, {\otimes}(x,y)\big)\\
			(\mathbf{0}, y) &\longmapsto \mathbf{0}.
		\end{align*}
	\item Soit $(x,y) \in \mathcal{N}^2$.
		\begin{itemize}
			\item Si $f(x) = 0$, alors $\otimes(x,y) = \mathbf{0}$, et donc $f(\otimes(x,y)) = 0 = f(x) \times f(y)$.
			\item Si $f(x) \ge 1$, alors $x = \mathbf{S}(z)$\/ avec $z \in \mathcal{N}$. Ainsi, par définition de $\otimes$, on a $\otimes(x,y) = \oplus(y, \otimes(z,y))$. Or, par hypothèse d'induction, $f(\otimes(z,y)) = f(z) \times f(y)$\/ (car $f(z) < f(x)$), et donc $f(\otimes(x,y)) = f(y) + f(\otimes(z,y)) = f(y) + f(z) \times f(y) = f(y) \times (1 + f(z)) = f(y) \times f(x)$.
		\end{itemize}
		Par induction, on a bien $\forall (x,y) \in \mathcal{N}^2,\:f\big({\otimes}(x,y)\big) = f(x) \times f(y)$.
	\item On définit par induction la fonction suivante \begin{align*}
			\ofact : \mathcal{N} &\longrightarrow \mathcal{N} \\
			\mathbf{0} &\longmapsto \mathbf{S}(\mathbf{0})\\
			\mathbf{S}(x) &\longmapsto {\otimes}\big(\mathbf{S}(x), \ofact(x)\big).
		\end{align*}
	\item Soit $x \in \mathcal{N}$.
		\begin{itemize}
			\item Si $f(x)= 0$, alors $\ofact(x) = \mathbf{S}(\mathbf{0})$\/ par définition, et donc $f(\ofact(x)) = 1 = 0! = f(x) !\:$.
			\item Si $f(x) \ge 1$, alors $x = \mathbf{S}(z)$\/ avec $z \in \mathcal{N}$. Ainsi, par définition de $\ofact$, on a $\ofact(x) = \otimes(x, \ofact(z))$, et donc, par hypothèse de récurrence, $f(\ofact(x)) = f(x) \times f(\ofact(z)) = f(x) \times \big(f(z)!\big)$. Or, comme $f(z) = f(x) - 1$, on a donc $f(\ofact(x)) = f(x) \times \big(f(x) - 1\big)! = f(x)!$\:.
		\end{itemize}
		Par induction, on a bien $\forall x \in \mathcal{N},\:f\big({\ofact}(x)\big) = f(x)!\:$.
\end{enumerate}

		\documentclass[a4paper]{article}

\usepackage[margin=1in]{geometry}
\usepackage[utf8]{inputenc}
\usepackage[T1]{fontenc}
\usepackage{mathrsfs}
\usepackage{textcomp}
\usepackage[french]{babel}
\usepackage{amsmath}
\usepackage{amssymb}
\usepackage{cancel}
\usepackage{frcursive}
\usepackage[inline]{asymptote}
\usepackage{tikz}
\usepackage[european,straightvoltages,europeanresistors]{circuitikz}
\usepackage{tikz-cd}
\usepackage{tkz-tab}
\usepackage[b]{esvect}
\usepackage[framemethod=TikZ]{mdframed}
\usepackage{centernot}
\usepackage{diagbox}
\usepackage{dsfont}
\usepackage{fancyhdr}
\usepackage{float}
\usepackage{graphicx}
\usepackage{listings}
\usepackage{multicol}
\usepackage{nicematrix}
\usepackage{pdflscape}
\usepackage{stmaryrd}
\usepackage{xfrac}
\usepackage{hep-math-font}
\usepackage{amsthm}
\usepackage{thmtools}
\usepackage{indentfirst}
\usepackage[framemethod=TikZ]{mdframed}
\usepackage{accents}
\usepackage{soulutf8}
\usepackage{mathtools}
\usepackage{bodegraph}
\usepackage{slashbox}
\usepackage{enumitem}
\usepackage{calligra}
\usepackage{cinzel}
\usepackage{BOONDOX-calo}

% Tikz
\usetikzlibrary{babel}
\usetikzlibrary{positioning}
\usetikzlibrary{calc}

% global settings
\frenchspacing
\reversemarginpar
\setuldepth{a}

%\everymath{\displaystyle}

\frenchbsetup{StandardLists=true}

\def\asydir{asy}

%\sisetup{exponent-product=\cdot,output-decimal-marker={,},separate-uncertainty,range-phrase=\;à\;,locale=FR}

\setlength{\parskip}{1em}

\theoremstyle{definition}

% Changing math
\let\emptyset\varnothing
\let\ge\geqslant
\let\le\leqslant
\let\preceq\preccurlyeq
\let\succeq\succcurlyeq
\let\ds\displaystyle
\let\ts\textstyle

\newcommand{\C}{\mathds{C}}
\newcommand{\R}{\mathds{R}}
\newcommand{\Z}{\mathds{Z}}
\newcommand{\N}{\mathds{N}}
\newcommand{\Q}{\mathds{Q}}

\renewcommand{\O}{\emptyset}

\newcommand\ubar[1]{\underaccent{\bar}{#1}}

\renewcommand\Re{\expandafter\mathfrak{Re}}
\renewcommand\Im{\expandafter\mathfrak{Im}}

\let\slantedpartial\partial
\DeclareRobustCommand{\partial}{\text{\rotatebox[origin=t]{20}{\scalebox{0.95}[1]{$\slantedpartial$}}}\hspace{-1pt}}

% merging two maths characters w/ \charfusion
\makeatletter
\def\moverlay{\mathpalette\mov@rlay}
\def\mov@rlay#1#2{\leavevmode\vtop{%
   \baselineskip\z@skip \lineskiplimit-\maxdimen
   \ialign{\hfil$\m@th#1##$\hfil\cr#2\crcr}}}
\newcommand{\charfusion}[3][\mathord]{
    #1{\ifx#1\mathop\vphantom{#2}\fi
        \mathpalette\mov@rlay{#2\cr#3}
      }
    \ifx#1\mathop\expandafter\displaylimits\fi}
\makeatother

% custom math commands
\newcommand{\T}{{\!\!\,\top}}
\newcommand{\avrt}[1]{\rotatebox{-90}{$#1$}}
\newcommand{\bigcupdot}{\charfusion[\mathop]{\bigcup}{\cdot}}
\newcommand{\cupdot}{\charfusion[\mathbin]{\cup}{\cdot}}
%\newcommand{\danger}{{\large\fontencoding{U}\fontfamily{futs}\selectfont\char 66\relax}\;}
\newcommand{\tendsto}[1]{\xrightarrow[#1]{}}
\newcommand{\vrt}[1]{\rotatebox{90}{$#1$}}
\newcommand{\tsup}[1]{\textsuperscript{\underline{#1}}}
\newcommand{\tsub}[1]{\textsubscript{#1}}

\renewcommand{\mod}[1]{~\left[ #1 \right]}
\renewcommand{\t}{{}^t\!}
\newcommand{\s}{\text{\calligra s}}

% custom units / constants
%\DeclareSIUnit{\litre}{\ell}
\let\hbar\hslash

% header / footer
\pagestyle{fancy}
\fancyhead{} \fancyfoot{}
\fancyfoot[C]{\thepage}

% fonts
\let\sc\scshape
\let\bf\bfseries
\let\it\itshape
\let\sl\slshape

% custom math operators
\let\th\relax
\let\det\relax
\DeclareMathOperator*{\codim}{codim}
\DeclareMathOperator*{\dom}{dom}
\DeclareMathOperator*{\gO}{O}
\DeclareMathOperator*{\po}{\text{\cursive o}}
\DeclareMathOperator*{\sgn}{sgn}
\DeclareMathOperator*{\simi}{\sim}
\DeclareMathOperator{\Arccos}{Arccos}
\DeclareMathOperator{\Arcsin}{Arcsin}
\DeclareMathOperator{\Arctan}{Arctan}
\DeclareMathOperator{\Argsh}{Argsh}
\DeclareMathOperator{\Arg}{Arg}
\DeclareMathOperator{\Aut}{Aut}
\DeclareMathOperator{\Card}{Card}
\DeclareMathOperator{\Cl}{\mathcal{C}\!\ell}
\DeclareMathOperator{\Cov}{Cov}
\DeclareMathOperator{\Ker}{Ker}
\DeclareMathOperator{\Mat}{Mat}
\DeclareMathOperator{\PGCD}{PGCD}
\DeclareMathOperator{\PPCM}{PPCM}
\DeclareMathOperator{\Supp}{Supp}
\DeclareMathOperator{\Vect}{Vect}
\DeclareMathOperator{\argmax}{argmax}
\DeclareMathOperator{\argmin}{argmin}
\DeclareMathOperator{\ch}{ch}
\DeclareMathOperator{\com}{com}
\DeclareMathOperator{\cotan}{cotan}
\DeclareMathOperator{\det}{det}
\DeclareMathOperator{\id}{id}
\DeclareMathOperator{\rg}{rg}
\DeclareMathOperator{\rk}{rk}
\DeclareMathOperator{\sh}{sh}
\DeclareMathOperator{\th}{th}
\DeclareMathOperator{\tr}{tr}

% colors and page style
\definecolor{truewhite}{HTML}{ffffff}
\definecolor{white}{HTML}{faf4ed}
\definecolor{trueblack}{HTML}{000000}
\definecolor{black}{HTML}{575279}
\definecolor{mauve}{HTML}{907aa9}
\definecolor{blue}{HTML}{286983}
\definecolor{red}{HTML}{d7827e}
\definecolor{yellow}{HTML}{ea9d34}
\definecolor{gray}{HTML}{9893a5}
\definecolor{grey}{HTML}{9893a5}
\definecolor{green}{HTML}{a0d971}

\pagecolor{white}
\color{black}

\begin{asydef}
	settings.prc = false;
	settings.render=0;

	white = rgb("faf4ed");
	black = rgb("575279");
	blue = rgb("286983");
	red = rgb("d7827e");
	yellow = rgb("f6c177");
	orange = rgb("ea9d34");
	gray = rgb("9893a5");
	grey = rgb("9893a5");
	deepcyan = rgb("56949f");
	pink = rgb("b4637a");
	magenta = rgb("eb6f92");
	green = rgb("a0d971");
	purple = rgb("907aa9");

	defaultpen(black + fontsize(8pt));

	import three;
	currentlight = nolight;
\end{asydef}

% theorems, proofs, ...

\mdfsetup{skipabove=1em,skipbelow=1em, innertopmargin=6pt, innerbottommargin=6pt,}

\declaretheoremstyle[
	headfont=\normalfont\itshape,
	numbered=no,
	postheadspace=\newline,
	headpunct={:},
	qed=\qedsymbol]{demstyle}

\declaretheorem[style=demstyle, name=Démonstration]{dem}

\newcommand\veczero{\kern-1.2pt\vec{\kern1.2pt 0}} % \vec{0} looks weird since the `0' isn't italicized

\makeatletter
\renewcommand{\title}[2]{
	\AtBeginDocument{
		\begin{titlepage}
			\begin{center}
				\vspace{10cm}
				{\Large \sc Chapitre #1}\\
				\vspace{1cm}
				{\Huge \calligra #2}\\
				\vfill
				Hugo {\sc Salou} MPI${}^{\star}$\\
				{\small Dernière mise à jour le \@date }
			\end{center}
		\end{titlepage}
	}
}

\newcommand{\titletp}[4]{
	\AtBeginDocument{
		\begin{titlepage}
			\begin{center}
				\vspace{10cm}
				{\Large \sc tp #1}\\
				\vspace{1cm}
				{\Huge \textsc{\textit{#2}}}\\
				\vfill
				{#3}\textit{MPI}${}^{\star}$\\
			\end{center}
		\end{titlepage}
	}
	\fancyfoot{}\fancyhead{}
	\fancyfoot[R]{#4 \textit{MPI}${}^{\star}$}
	\fancyhead[C]{{\sc tp #1} : #2}
	\fancyhead[R]{\thepage}
}

\newcommand{\titletd}[2]{
	\AtBeginDocument{
		\begin{titlepage}
			\begin{center}
				\vspace{10cm}
				{\Large \sc td #1}\\
				\vspace{1cm}
				{\Huge \calligra #2}\\
				\vfill
				Hugo {\sc Salou} MPI${}^{\star}$\\
				{\small Dernière mise à jour le \@date }
			\end{center}
		\end{titlepage}
	}
}
\makeatother

\newcommand{\sign}{
	\null
	\vfill
	\begin{center}
		{
			\fontfamily{ccr}\selectfont
			\textit{\textbf{\.{\"i}}}
		}
	\end{center}
	\vfill
	\null
}

\renewcommand{\thefootnote}{\emph{\alph{footnote}}}

% figure support
\usepackage{import}
\usepackage{xifthen}
\pdfminorversion=7
\usepackage{pdfpages}
\usepackage{transparent}
\newcommand{\incfig}[1]{%
	\def\svgwidth{\columnwidth}
	\import{./figures/}{#1.pdf_tex}
}

\pdfsuppresswarningpagegroup=1
\ctikzset{tripoles/european not symbol=circle}

\newcommand{\missingpart}{{\large\color{red} Il manque quelque chose ici\ldots}}


\fancyhead[R]{Hugo {\sc Salou}\/ MPI}
\fancyhead[L]{TD\textsubscript4 -- Exercice 8}

\begin{document}
	\let\thesection\relax
	
\begin{comment}
\section{Exercice 9}

\slshape
Soit la matrice $A = \begin{pmatrix}
	1&1&-1\\
	2&3&-4\\
	4&1&-4
\end{pmatrix}$.
\begin{enumerate}
	\item Déterminer le spectre de la matrice $A$\/ et trouver une matrice $P$\/ inversible telle que $P^{-1} A P$\/ est diagonale.
	\item Soit $B$\/ une matrice de taille $3\times 3$\/ qui commentent avec $A$\/ ($AB = BA$). Montrer que $B$\/ est diagonale.
\end{enumerate}
\upshape

\begin{enumerate}
	\item On sait, tout d'abord, que, pour $x \in \R$,
		\begin{align*}
			\chi_A(x) = \det(x\,I_n - A) &=
			\begin{vmatrix}
				x - 1 &- 1 & 1\\
				-2&x-3&4\\
				-4&-1&x + 4
			\end{vmatrix}\\
			&= \\
		\end{align*}
\end{enumerate}
\end{comment}

\section{Exercice 8}

\begin{enumerate}
	\item Soit un vecteur non nul $\vec{x} \in \Ker(\lambda {\id} - {u \circ v})$. Ainsi, $u(v(\vec{x})) = \lambda \vec{x}$. Et, donc $v(u(v(\vec{x}))) = \lambda v(\vec{x})$. On a donc $v(\vec{x}) \in \Ker(\lambda {\id} - {v  \circ u})$.
		Or, si $\lambda \neq 0$, on a $v(\vec{x}) \neq \vec{0}$\/ ; en effet, si $v(\vec{x}) = \vec{0}$, alors $u \circ v(\vec{x}) = \vec{0} = \lambda \vec{x}$\/ et donc $\vec{x} = \vec{0}$, ce ne serait donc pas un vecteur propre de $u \circ v$\/ : une contradiction. On en déduit que $v(\vec{x})$\/ est un vecteur propre de $u \circ v$\/ associé à la valeur propre $\lambda$.
	\item On pose donc $\lambda = 0$, une valeur propre de $u  \circ v$. L'endomorphisme $u \circ v$\/ n'est donc pas injectif, donc bijectif. On sait donc, comme $E$\/ est de dimension finie, que $\det(u \circ v) = 0$. Or $\det (u \circ v) = \det u \times \det v = \det(v  \circ u)$. Et donc $\det(v  \circ u) = 0$, $v  \circ u$\/ n'est donc pas bijectif, donc injectif. Et donc, on a $0 \in \Sp(v  \circ u)$.
	\item Soit $P \in \R[X]$, et soit $Q$\/ une primitive de $P$.
		\begin{align*}
			P \in \Ker (u  \circ v) \iff& \Big(\int_{0}^X P(t)~\mathrm{d}t\Big)' = 0\\
			\iff& \big(Q(X) - Q(0)\big)' = 0\\
			\iff& Q'(X) = 0\\
			\iff& P(X) = 0
		\end{align*}
		On en déduit que $\Ker (u \circ v) = \{0\}$.

		Également,
		\begin{align*}
			P \in \Ker(v  \circ u) \iff& \int_{0}^{X} P'(t)~\mathrm{d}t = 0\\
			\iff& P(X) - P(0) = 0\\
			\iff& P(X) = P(0)\\
			\iff& \deg P \le 0\\
			\iff& P \in \R_0[X]
		\end{align*}
		On en déduit que $\Ker(v \circ u) = \R_0[X]$.
\end{enumerate}


\end{document}

	}
	\def\addmacros#1{#1}
}
{
	\td[8]{Classe \textbf{P}, classe \textbf{NP}}
	\minitoc
	\renewcommand{\cwd}{../td/td08/}
	\addmacros{
		\section{Quelques problèmes décidables}

\begin{enumerate}
	\item Soit $f : \R \to \R$.
		\begin{itemize}
			\item Si $f$\/ admet un zéro, on pose $\mathcal{M} = \texttt{fun}\ \texttt{s}\ \to \texttt{true}$.
			\item Si $f$\/ n'admet pas un zéro, on pose $\mathcal{M} = \texttt{fun}\ \texttt{s}\ \to \texttt{false}$.
		\end{itemize}
		Alors, $\mathcal{M}$\/ décide \textsc{Zero}$_f$.
	\item Soit $\mathcal{M}$\/ une machine, et soit $w \in \Sigma^*$.
		\begin{itemize}
			\item Si $\mathcal{M}$\/ se termine sur l'entrée $w$, alors on pose $\mathcal{M}' = \texttt{fun}\ \texttt{s} \to \texttt{true}$.
			\item Si $\mathcal{M}$\/ ne se termine pas sur l'entrée $w$, alors on pose $\mathcal{M}' = \texttt{fun}\ \texttt{s} \to \texttt{false}$.
		\end{itemize}
		Alors, $\mathcal{M}'$\/ décide \textsc{Arrêt}$_{\mathcal{M},w}$.
	\item Le problème est trivialement vrai. En effet, soit $M \in \mathcal{O}$, de la forme
		\begin{lstlisting}[language=caml]
let m (s: string): string =
	%*$\langle$\textrm{code}$\rangle$*) 
		\end{lstlisting}
		On crée la machine $\mathcal{N}$\/ ci-dessous.
		\begin{lstlisting}[language=caml]
let n (s: string): string =
	if true then
		%*$\langle$\textrm{code}$\rangle$*) 
	else
		%*$\langle$\textrm{code}$\rangle$*) 
		\end{lstlisting}
		On a $\texttt{m} \neq \texttt{n}$, mais $\mathcal{L}(\texttt{m}) = \mathcal{L}(\texttt{n})$, donc le problème est vrai sur toute entrée et la fonction $\texttt{fun}\ \texttt{s} \to \texttt{true}$\/ répond au problème.
\end{enumerate}

		\section{Ensembles définis inductivement}

La correction est disponible sur \textit{cahier-de-prepa}.

\begin{comment}
	\begin{exm}
		Avec $S = \N$, $\mathcal{B} = \{0, 2\} $, $A_1 = \{0\}$\/ et \begin{align*}
			f_1: A_1 \times \N &\longrightarrow \N \\
			(0, x) &\longmapsto x + 4.
		\end{align*}

		On a \[
			X \supseteq \{0, 2, 4, 6, 8, 10, \ldots, 20, \ldots\} = 2\N
		.\]
	\end{exm}
	\begin{exm}
		Avec $S$\/ l'ensemble des langages sur $\Sigma$, $\mathcal{B} = \{\O\} \cup \bigl\{\{a\}\:\big|\: a \in \Sigma \bigr\}$, et
		\begin{multicols}{3}
			\begin{align*}
				f_1: S \times S &\longrightarrow S \\
				(L_1, L_2) &\longmapsto L_1 \cup L_2,
			\end{align*}
			\begin{align*}
				f_2: S \times S &\longrightarrow S \\
				(L_1, L_2) &\longmapsto L_1 \cdot L_2,
			\end{align*}
			\begin{align*}
				f_3: S &\longrightarrow S \\
				L &\longmapsto L^*.
			\end{align*}
		\end{multicols}
	\end{exm}

\begin{enumerate}
	\item Soit $\mathcal{A} = \{X \subseteq S  \mid X \supseteq \mathcal{B} \mathrel{\text{et}} X \text{ est stable par } f_i\}$. On a $S \in \mathcal{A}$\/ et donc $\mathcal{A} \neq \O$. De plus, soit \[
			Y = \{x \in S  \mid \forall X \in \mathcal{A},\,x \in X\} = \bigcap_{X \in \mathcal{A}} X
		.\]
		Soit $b \in \mathcal{B}$, on a $\forall X \in A,\, b \in X$. D'où $b \in Y$\/ par intersection. On en déduit que $\mathcal{B} \subseteq Y$.

		Soit $i \in \left\llbracket 1,m \right\rrbracket$. Soit $(x_1, \ldots, x_{n_i}) \in Y^{n_i}$\/ et soit $a \in A_i$. Montrons que $f_i(a, x_1, \ldots, x_{n_i}) \in Y$.
		Or, soit $X \in \mathcal{A}$, on a $(x_1, \ldots, x_{n_i}) \in X^{n_i}$\/ donc $f_i(a, x_1, \ldots, x_{n_i}) \in X$. Ceci étant vrai pour tout $X \in \mathcal{A}$, on a $f_i(a, x_1, \ldots, x_{n_i}) \in Y$\/ donc $Y$\/ est stable par $f_i$\/ par tout $i \in \left\llbracket 1,m \right\rrbracket$\/ et donc $Y \in \mathcal{A}$.
		On a également $Y \subseteq X$\/ pour tout $X \in \mathcal{A}$. On en déduit que $Y$\/ est le plus petit élément (pour l'inclusion) de $\mathcal{A}$.
	\item On pose $X_0  = \mathcal{B}$\/ et \[
			X_{n+1} = X_n \cup \big\{ f_i(a, x_1, \ldots, x_{n_i})  \mid a \in A_i,\,(x_1, \ldots, x_{n_i}) \in (X_n)^{n_i},\,i \in \left\llbracket 1,m \right\rrbracket\big\}
		.\]
		Soit $X = \bigcup_{n \in \N} X_n$. Soit $Y$\/ l'ensemble défini par induction à partir de $\mathcal{B}$\/ et des $(f_i)_{i\in\left\llbracket 1,n \right\rrbracket}$. Montrons que $X = Y$.
		On montre que $X$\/ est le plus petit élément (pour l'inclusion) de $\mathcal{A}$\/ et on conclut par unicité du minimum (avec la question précédente).
		Par définition de la suite $(X_n)_{n\in\N}$, elle est croissante (au sens de l'inclusion).
		Montrons à présent, par récurrence, la propriété ci-dessous : $P_n : ``X_n \subseteq Y."$
		\begin{itemize}
			\item Par définition de $Y$, on a $X_0 = \mathcal{B} \subseteq Y$.
			\item Soit 
		\end{itemize}
\end{enumerate}

\subsection{Un théorème d'induction}

\begin{enumerate}
	\item[3.] Soit $Z = \{x \in S  \mid P(x) \text{ vraie}\:\}$.
		Montrons que $\mathcal{X}\subseteq Z$.
		On remarque que $\mathcal{X} \supseteq \mathcal{B}$\/ ; $\mathcal{X}$\/ est stable par $f_i$. On en conclut que $Z \supseteq \mathcal{X}$ et donc $\forall x \in \mathcal{X},\,P(x)$\/ est vraie.
\end{enumerate}

\begin{exm}
	Soit $\mathcal{X}$\/ défini par induction par $\mathcal{B} = \{0, 2\}$\/ et \begin{align*}
		f: \N &\longrightarrow \N \\
		n &\longmapsto n + 2.
	\end{align*}
	Montrons que $\forall n \in \mathcal{X}$, $x$\/ est pair.

	On sait que $0$\/ est pair, $2$\/ est pair ; et, \[
		\forall x,y \in \mathcal{X},\, (x \text{ pair} \land y \text{ pair}) \implies f(x, y) \text{ pair}
	.\]

	On en déduit que \[
		\forall n \in \mathcal{X},\,x \text{ est pair}.
	.\]
\end{exm}
\end{comment}

		\section{Tableaux dynamiques}

\begin{enumerate}[start=3]
	\item On trouve une complexité amortie en $n^2$. À rédiger.
	\item Au lieu de diviser quand $r < n / 2$, mais quand $r < n / 4$.
\end{enumerate}

		\section{Barres en triangle}


On note $a(x)$ le côté du triangule équilatéral, et donc $x = a(x) \sqrt{3}  / 2$.

On calcule le flux $\Phi$ magnétique : \[
	\Phi = B \frac{a x}{2} = B x^2 / \sqrt{3}
.\]
Ainsi, d'après la loi de \textsc{Faraday}, on a \[
	e = - \frac{\mathrm{d}\Phi}{\mathrm{d}t} = - \frac{2B}{\sqrt{3}}  \: x\,\dot{x}
.\]
Or, par loi d'\textsc{Ohm}, $i = e / R(x)$.
Et, on connait la resistance du circuit $R(x) = 3 a(x) / \gamma S$.
Alors, 
\begin{align*}
	i(x) &= \frac{-2B\,x\,\dot{x}}{\sqrt{3} \cdot \frac{3a(x)}{\gamma S}} = - \frac{2B \gamma S}{3 \sqrt{3}} \cdot \frac{x\,\dot{x}}{2 \frac{x}{\sqrt{3}}}\\
	&= - B \gamma S \dot{x} / 3. \\
\end{align*}
On calcule donc la force de \textsc{Laplace} :
\begin{align*}
	\vec{F}_{\mathcal{L}} &= i \cdot [\mathrm{CD}] \cdot B \cdot \vec{e}_x\\
	&= i \cdot \frac{2\dot{x}}{\sqrt{3}} B \vec{e}_x \\
	&= -\underbrace{\frac{2 B^2 \gamma S}{3\sqrt{3}}}_\alpha  \cdot x \dot{x} \\
\end{align*}

D'après le \textsc{pfd}, on a donc \[
	m \ddot{x} = - \alpha x \dot{x} \text{ d'où }  \ddot{x} = - \frac{\alpha}{m} x\dot{x} 
.\] où $m = \rho S L$. 

On intègre les deux côtés de l'équation, \[
	[\dot{x}]_0^{t_\mathrm{f}} = -\frac{\alpha}{m} \cdot \left[ \frac{x^2}{2} \right]_0^{t_\mathrm{f}}
.\]
D'où, \[
	0 - v_0 = \frac{-\alpha}{m} \cdot x_\mathrm{f}^2 / 2
.\] On en conclut \[
	x_\mathrm{f} = \sqrt{\frac{2m}{\alpha} v_0} 
.\] 


	}
	\def\addmacros#1{#1}
}
{
	\td[9]{Algorithmique des graphes}
	\minitoc
	\renewcommand{\cwd}{../td/td09/}
	\addmacros{
		\section{Quelques problèmes décidables}

\begin{enumerate}
	\item Soit $f : \R \to \R$.
		\begin{itemize}
			\item Si $f$\/ admet un zéro, on pose $\mathcal{M} = \texttt{fun}\ \texttt{s}\ \to \texttt{true}$.
			\item Si $f$\/ n'admet pas un zéro, on pose $\mathcal{M} = \texttt{fun}\ \texttt{s}\ \to \texttt{false}$.
		\end{itemize}
		Alors, $\mathcal{M}$\/ décide \textsc{Zero}$_f$.
	\item Soit $\mathcal{M}$\/ une machine, et soit $w \in \Sigma^*$.
		\begin{itemize}
			\item Si $\mathcal{M}$\/ se termine sur l'entrée $w$, alors on pose $\mathcal{M}' = \texttt{fun}\ \texttt{s} \to \texttt{true}$.
			\item Si $\mathcal{M}$\/ ne se termine pas sur l'entrée $w$, alors on pose $\mathcal{M}' = \texttt{fun}\ \texttt{s} \to \texttt{false}$.
		\end{itemize}
		Alors, $\mathcal{M}'$\/ décide \textsc{Arrêt}$_{\mathcal{M},w}$.
	\item Le problème est trivialement vrai. En effet, soit $M \in \mathcal{O}$, de la forme
		\begin{lstlisting}[language=caml]
let m (s: string): string =
	%*$\langle$\textrm{code}$\rangle$*) 
		\end{lstlisting}
		On crée la machine $\mathcal{N}$\/ ci-dessous.
		\begin{lstlisting}[language=caml]
let n (s: string): string =
	if true then
		%*$\langle$\textrm{code}$\rangle$*) 
	else
		%*$\langle$\textrm{code}$\rangle$*) 
		\end{lstlisting}
		On a $\texttt{m} \neq \texttt{n}$, mais $\mathcal{L}(\texttt{m}) = \mathcal{L}(\texttt{n})$, donc le problème est vrai sur toute entrée et la fonction $\texttt{fun}\ \texttt{s} \to \texttt{true}$\/ répond au problème.
\end{enumerate}

		\section{Ensembles définis inductivement}

La correction est disponible sur \textit{cahier-de-prepa}.

\begin{comment}
	\begin{exm}
		Avec $S = \N$, $\mathcal{B} = \{0, 2\} $, $A_1 = \{0\}$\/ et \begin{align*}
			f_1: A_1 \times \N &\longrightarrow \N \\
			(0, x) &\longmapsto x + 4.
		\end{align*}

		On a \[
			X \supseteq \{0, 2, 4, 6, 8, 10, \ldots, 20, \ldots\} = 2\N
		.\]
	\end{exm}
	\begin{exm}
		Avec $S$\/ l'ensemble des langages sur $\Sigma$, $\mathcal{B} = \{\O\} \cup \bigl\{\{a\}\:\big|\: a \in \Sigma \bigr\}$, et
		\begin{multicols}{3}
			\begin{align*}
				f_1: S \times S &\longrightarrow S \\
				(L_1, L_2) &\longmapsto L_1 \cup L_2,
			\end{align*}
			\begin{align*}
				f_2: S \times S &\longrightarrow S \\
				(L_1, L_2) &\longmapsto L_1 \cdot L_2,
			\end{align*}
			\begin{align*}
				f_3: S &\longrightarrow S \\
				L &\longmapsto L^*.
			\end{align*}
		\end{multicols}
	\end{exm}

\begin{enumerate}
	\item Soit $\mathcal{A} = \{X \subseteq S  \mid X \supseteq \mathcal{B} \mathrel{\text{et}} X \text{ est stable par } f_i\}$. On a $S \in \mathcal{A}$\/ et donc $\mathcal{A} \neq \O$. De plus, soit \[
			Y = \{x \in S  \mid \forall X \in \mathcal{A},\,x \in X\} = \bigcap_{X \in \mathcal{A}} X
		.\]
		Soit $b \in \mathcal{B}$, on a $\forall X \in A,\, b \in X$. D'où $b \in Y$\/ par intersection. On en déduit que $\mathcal{B} \subseteq Y$.

		Soit $i \in \left\llbracket 1,m \right\rrbracket$. Soit $(x_1, \ldots, x_{n_i}) \in Y^{n_i}$\/ et soit $a \in A_i$. Montrons que $f_i(a, x_1, \ldots, x_{n_i}) \in Y$.
		Or, soit $X \in \mathcal{A}$, on a $(x_1, \ldots, x_{n_i}) \in X^{n_i}$\/ donc $f_i(a, x_1, \ldots, x_{n_i}) \in X$. Ceci étant vrai pour tout $X \in \mathcal{A}$, on a $f_i(a, x_1, \ldots, x_{n_i}) \in Y$\/ donc $Y$\/ est stable par $f_i$\/ par tout $i \in \left\llbracket 1,m \right\rrbracket$\/ et donc $Y \in \mathcal{A}$.
		On a également $Y \subseteq X$\/ pour tout $X \in \mathcal{A}$. On en déduit que $Y$\/ est le plus petit élément (pour l'inclusion) de $\mathcal{A}$.
	\item On pose $X_0  = \mathcal{B}$\/ et \[
			X_{n+1} = X_n \cup \big\{ f_i(a, x_1, \ldots, x_{n_i})  \mid a \in A_i,\,(x_1, \ldots, x_{n_i}) \in (X_n)^{n_i},\,i \in \left\llbracket 1,m \right\rrbracket\big\}
		.\]
		Soit $X = \bigcup_{n \in \N} X_n$. Soit $Y$\/ l'ensemble défini par induction à partir de $\mathcal{B}$\/ et des $(f_i)_{i\in\left\llbracket 1,n \right\rrbracket}$. Montrons que $X = Y$.
		On montre que $X$\/ est le plus petit élément (pour l'inclusion) de $\mathcal{A}$\/ et on conclut par unicité du minimum (avec la question précédente).
		Par définition de la suite $(X_n)_{n\in\N}$, elle est croissante (au sens de l'inclusion).
		Montrons à présent, par récurrence, la propriété ci-dessous : $P_n : ``X_n \subseteq Y."$
		\begin{itemize}
			\item Par définition de $Y$, on a $X_0 = \mathcal{B} \subseteq Y$.
			\item Soit 
		\end{itemize}
\end{enumerate}

\subsection{Un théorème d'induction}

\begin{enumerate}
	\item[3.] Soit $Z = \{x \in S  \mid P(x) \text{ vraie}\:\}$.
		Montrons que $\mathcal{X}\subseteq Z$.
		On remarque que $\mathcal{X} \supseteq \mathcal{B}$\/ ; $\mathcal{X}$\/ est stable par $f_i$. On en conclut que $Z \supseteq \mathcal{X}$ et donc $\forall x \in \mathcal{X},\,P(x)$\/ est vraie.
\end{enumerate}

\begin{exm}
	Soit $\mathcal{X}$\/ défini par induction par $\mathcal{B} = \{0, 2\}$\/ et \begin{align*}
		f: \N &\longrightarrow \N \\
		n &\longmapsto n + 2.
	\end{align*}
	Montrons que $\forall n \in \mathcal{X}$, $x$\/ est pair.

	On sait que $0$\/ est pair, $2$\/ est pair ; et, \[
		\forall x,y \in \mathcal{X},\, (x \text{ pair} \land y \text{ pair}) \implies f(x, y) \text{ pair}
	.\]

	On en déduit que \[
		\forall n \in \mathcal{X},\,x \text{ est pair}.
	.\]
\end{exm}
\end{comment}

		\section{Tableaux dynamiques}

\begin{enumerate}[start=3]
	\item On trouve une complexité amortie en $n^2$. À rédiger.
	\item Au lieu de diviser quand $r < n / 2$, mais quand $r < n / 4$.
\end{enumerate}

		\section{Barres en triangle}


On note $a(x)$ le côté du triangule équilatéral, et donc $x = a(x) \sqrt{3}  / 2$.

On calcule le flux $\Phi$ magnétique : \[
	\Phi = B \frac{a x}{2} = B x^2 / \sqrt{3}
.\]
Ainsi, d'après la loi de \textsc{Faraday}, on a \[
	e = - \frac{\mathrm{d}\Phi}{\mathrm{d}t} = - \frac{2B}{\sqrt{3}}  \: x\,\dot{x}
.\]
Or, par loi d'\textsc{Ohm}, $i = e / R(x)$.
Et, on connait la resistance du circuit $R(x) = 3 a(x) / \gamma S$.
Alors, 
\begin{align*}
	i(x) &= \frac{-2B\,x\,\dot{x}}{\sqrt{3} \cdot \frac{3a(x)}{\gamma S}} = - \frac{2B \gamma S}{3 \sqrt{3}} \cdot \frac{x\,\dot{x}}{2 \frac{x}{\sqrt{3}}}\\
	&= - B \gamma S \dot{x} / 3. \\
\end{align*}
On calcule donc la force de \textsc{Laplace} :
\begin{align*}
	\vec{F}_{\mathcal{L}} &= i \cdot [\mathrm{CD}] \cdot B \cdot \vec{e}_x\\
	&= i \cdot \frac{2\dot{x}}{\sqrt{3}} B \vec{e}_x \\
	&= -\underbrace{\frac{2 B^2 \gamma S}{3\sqrt{3}}}_\alpha  \cdot x \dot{x} \\
\end{align*}

D'après le \textsc{pfd}, on a donc \[
	m \ddot{x} = - \alpha x \dot{x} \text{ d'où }  \ddot{x} = - \frac{\alpha}{m} x\dot{x} 
.\] où $m = \rho S L$. 

On intègre les deux côtés de l'équation, \[
	[\dot{x}]_0^{t_\mathrm{f}} = -\frac{\alpha}{m} \cdot \left[ \frac{x^2}{2} \right]_0^{t_\mathrm{f}}
.\]
D'où, \[
	0 - v_0 = \frac{-\alpha}{m} \cdot x_\mathrm{f}^2 / 2
.\] On en conclut \[
	x_\mathrm{f} = \sqrt{\frac{2m}{\alpha} v_0} 
.\] 


		\begin{multicols}{2}
	\section{Implications}
	\begin{enumerate}
		\item
			\[
				\begin{prooftree}
					\infer 0[Ax]{p,q \vdash q}
					\infer 1[$\to$i]{q \vdash p \to q}
				\end{prooftree}
			\]
		\item
			\[
				\begin{prooftree}
					\infer 0[Ax]{p, p\land q\vdash p\land q}
					\infer 1[$\land$e,d]{p, p \land q \vdash q}
					\infer 1[$\to$i]{p \land q \vdash p \to q}
				\end{prooftree}
			\]
		\item 
			\[
				\begin{prooftree}
					\infer 0[Ax]{p, p\to q \vdash p}
					\infer 0[Ax]{p, p\to q \vdash p \to q}
					\infer 0[Ax]{p, p\to q \vdash p}
					\infer 2[$\to$e]{p, p \to q \vdash q}
					\infer 2[$\land$i]{p, p \to q \vdash p \land q}
					\infer 1[$\to$i]{p \to q \vdash p\to (p \land q)}
				\end{prooftree}
			\]
		\item
			\[
				\begin{prooftree}
					\infer 0[Ax]{\lnot q, p\to q, p \vdash p \to q}
					\infer 0[Ax]{\lnot q, p\to q, p \vdash p}
					\infer 2[$\to$e]{\lnot q, p\to q, p \vdash q}
					\infer 0[Ax]{\lnot q, p\to q, p \vdash \lnot q}
					\infer 2[$\lnot$e]{\lnot q, p\to q, p \vdash \bot}
					\infer 1[$\lnot$i]{\lnot q, p\to q \vdash\lnot p}
					\infer 1[$\to$i]{p \to q \vdash \lnot q \to \lnot p}
				\end{prooftree}
			\]
		\item 
			\[
				\begin{prooftree}
					\infer 0[Ax]{p \land q, p\to r \vdash p \land q}
					\infer 1[$\land$e,g]{p \land q, p\to r \vdash p}
					\infer 0[Ax]{p \land q, p\to r \vdash p \to r}
					\infer 2[$\to$e]{p \land q, p \to r \vdash r}
					\infer 1[$\to$i]{p \to r \vdash (p \land q) \to r}
				\end{prooftree}
			\] 
		\item
			\[
				\begin{prooftree}
					\infer 0[Ax]{p,q\vdash p}
					\infer 1[$\to$i]{p \vdash q \to p}
					\infer 1[$\to$i]{\vdash p \to (q \to p)}
				\end{prooftree}
			\]
		\item
			\[
				\begin{prooftree}
					\infer 0[Ax]{p \to q, p \vdash p\to q}
					\infer 0[Ax]{p\to q,p \vdash p}
					\infer 2[$\to$e]{p, p\to q \vdash q}
					\infer 1[$\to$i]{p \vdash (p \to q) \to q}
				\end{prooftree}
			\] 
	\end{enumerate}
\end{multicols}

	}
	\def\addmacros#1{#1}
}
{
	\td[10]{Preuves en logique propositionnelle}
	\minitoc
	\renewcommand{\cwd}{../td/td10/}
	\addmacros{
		\begin{landscape}
			\begin{multicols}{2}
				\section{Quelques problèmes décidables}

\begin{enumerate}
	\item Soit $f : \R \to \R$.
		\begin{itemize}
			\item Si $f$\/ admet un zéro, on pose $\mathcal{M} = \texttt{fun}\ \texttt{s}\ \to \texttt{true}$.
			\item Si $f$\/ n'admet pas un zéro, on pose $\mathcal{M} = \texttt{fun}\ \texttt{s}\ \to \texttt{false}$.
		\end{itemize}
		Alors, $\mathcal{M}$\/ décide \textsc{Zero}$_f$.
	\item Soit $\mathcal{M}$\/ une machine, et soit $w \in \Sigma^*$.
		\begin{itemize}
			\item Si $\mathcal{M}$\/ se termine sur l'entrée $w$, alors on pose $\mathcal{M}' = \texttt{fun}\ \texttt{s} \to \texttt{true}$.
			\item Si $\mathcal{M}$\/ ne se termine pas sur l'entrée $w$, alors on pose $\mathcal{M}' = \texttt{fun}\ \texttt{s} \to \texttt{false}$.
		\end{itemize}
		Alors, $\mathcal{M}'$\/ décide \textsc{Arrêt}$_{\mathcal{M},w}$.
	\item Le problème est trivialement vrai. En effet, soit $M \in \mathcal{O}$, de la forme
		\begin{lstlisting}[language=caml]
let m (s: string): string =
	%*$\langle$\textrm{code}$\rangle$*) 
		\end{lstlisting}
		On crée la machine $\mathcal{N}$\/ ci-dessous.
		\begin{lstlisting}[language=caml]
let n (s: string): string =
	if true then
		%*$\langle$\textrm{code}$\rangle$*) 
	else
		%*$\langle$\textrm{code}$\rangle$*) 
		\end{lstlisting}
		On a $\texttt{m} \neq \texttt{n}$, mais $\mathcal{L}(\texttt{m}) = \mathcal{L}(\texttt{n})$, donc le problème est vrai sur toute entrée et la fonction $\texttt{fun}\ \texttt{s} \to \texttt{true}$\/ répond au problème.
\end{enumerate}

				\section{Ensembles définis inductivement}

La correction est disponible sur \textit{cahier-de-prepa}.

\begin{comment}
	\begin{exm}
		Avec $S = \N$, $\mathcal{B} = \{0, 2\} $, $A_1 = \{0\}$\/ et \begin{align*}
			f_1: A_1 \times \N &\longrightarrow \N \\
			(0, x) &\longmapsto x + 4.
		\end{align*}

		On a \[
			X \supseteq \{0, 2, 4, 6, 8, 10, \ldots, 20, \ldots\} = 2\N
		.\]
	\end{exm}
	\begin{exm}
		Avec $S$\/ l'ensemble des langages sur $\Sigma$, $\mathcal{B} = \{\O\} \cup \bigl\{\{a\}\:\big|\: a \in \Sigma \bigr\}$, et
		\begin{multicols}{3}
			\begin{align*}
				f_1: S \times S &\longrightarrow S \\
				(L_1, L_2) &\longmapsto L_1 \cup L_2,
			\end{align*}
			\begin{align*}
				f_2: S \times S &\longrightarrow S \\
				(L_1, L_2) &\longmapsto L_1 \cdot L_2,
			\end{align*}
			\begin{align*}
				f_3: S &\longrightarrow S \\
				L &\longmapsto L^*.
			\end{align*}
		\end{multicols}
	\end{exm}

\begin{enumerate}
	\item Soit $\mathcal{A} = \{X \subseteq S  \mid X \supseteq \mathcal{B} \mathrel{\text{et}} X \text{ est stable par } f_i\}$. On a $S \in \mathcal{A}$\/ et donc $\mathcal{A} \neq \O$. De plus, soit \[
			Y = \{x \in S  \mid \forall X \in \mathcal{A},\,x \in X\} = \bigcap_{X \in \mathcal{A}} X
		.\]
		Soit $b \in \mathcal{B}$, on a $\forall X \in A,\, b \in X$. D'où $b \in Y$\/ par intersection. On en déduit que $\mathcal{B} \subseteq Y$.

		Soit $i \in \left\llbracket 1,m \right\rrbracket$. Soit $(x_1, \ldots, x_{n_i}) \in Y^{n_i}$\/ et soit $a \in A_i$. Montrons que $f_i(a, x_1, \ldots, x_{n_i}) \in Y$.
		Or, soit $X \in \mathcal{A}$, on a $(x_1, \ldots, x_{n_i}) \in X^{n_i}$\/ donc $f_i(a, x_1, \ldots, x_{n_i}) \in X$. Ceci étant vrai pour tout $X \in \mathcal{A}$, on a $f_i(a, x_1, \ldots, x_{n_i}) \in Y$\/ donc $Y$\/ est stable par $f_i$\/ par tout $i \in \left\llbracket 1,m \right\rrbracket$\/ et donc $Y \in \mathcal{A}$.
		On a également $Y \subseteq X$\/ pour tout $X \in \mathcal{A}$. On en déduit que $Y$\/ est le plus petit élément (pour l'inclusion) de $\mathcal{A}$.
	\item On pose $X_0  = \mathcal{B}$\/ et \[
			X_{n+1} = X_n \cup \big\{ f_i(a, x_1, \ldots, x_{n_i})  \mid a \in A_i,\,(x_1, \ldots, x_{n_i}) \in (X_n)^{n_i},\,i \in \left\llbracket 1,m \right\rrbracket\big\}
		.\]
		Soit $X = \bigcup_{n \in \N} X_n$. Soit $Y$\/ l'ensemble défini par induction à partir de $\mathcal{B}$\/ et des $(f_i)_{i\in\left\llbracket 1,n \right\rrbracket}$. Montrons que $X = Y$.
		On montre que $X$\/ est le plus petit élément (pour l'inclusion) de $\mathcal{A}$\/ et on conclut par unicité du minimum (avec la question précédente).
		Par définition de la suite $(X_n)_{n\in\N}$, elle est croissante (au sens de l'inclusion).
		Montrons à présent, par récurrence, la propriété ci-dessous : $P_n : ``X_n \subseteq Y."$
		\begin{itemize}
			\item Par définition de $Y$, on a $X_0 = \mathcal{B} \subseteq Y$.
			\item Soit 
		\end{itemize}
\end{enumerate}

\subsection{Un théorème d'induction}

\begin{enumerate}
	\item[3.] Soit $Z = \{x \in S  \mid P(x) \text{ vraie}\:\}$.
		Montrons que $\mathcal{X}\subseteq Z$.
		On remarque que $\mathcal{X} \supseteq \mathcal{B}$\/ ; $\mathcal{X}$\/ est stable par $f_i$. On en conclut que $Z \supseteq \mathcal{X}$ et donc $\forall x \in \mathcal{X},\,P(x)$\/ est vraie.
\end{enumerate}

\begin{exm}
	Soit $\mathcal{X}$\/ défini par induction par $\mathcal{B} = \{0, 2\}$\/ et \begin{align*}
		f: \N &\longrightarrow \N \\
		n &\longmapsto n + 2.
	\end{align*}
	Montrons que $\forall n \in \mathcal{X}$, $x$\/ est pair.

	On sait que $0$\/ est pair, $2$\/ est pair ; et, \[
		\forall x,y \in \mathcal{X},\, (x \text{ pair} \land y \text{ pair}) \implies f(x, y) \text{ pair}
	.\]

	On en déduit que \[
		\forall n \in \mathcal{X},\,x \text{ est pair}.
	.\]
\end{exm}
\end{comment}

			\end{multicols}
			\section{Tableaux dynamiques}

\begin{enumerate}[start=3]
	\item On trouve une complexité amortie en $n^2$. À rédiger.
	\item Au lieu de diviser quand $r < n / 2$, mais quand $r < n / 4$.
\end{enumerate}

			\section{Barres en triangle}


On note $a(x)$ le côté du triangule équilatéral, et donc $x = a(x) \sqrt{3}  / 2$.

On calcule le flux $\Phi$ magnétique : \[
	\Phi = B \frac{a x}{2} = B x^2 / \sqrt{3}
.\]
Ainsi, d'après la loi de \textsc{Faraday}, on a \[
	e = - \frac{\mathrm{d}\Phi}{\mathrm{d}t} = - \frac{2B}{\sqrt{3}}  \: x\,\dot{x}
.\]
Or, par loi d'\textsc{Ohm}, $i = e / R(x)$.
Et, on connait la resistance du circuit $R(x) = 3 a(x) / \gamma S$.
Alors, 
\begin{align*}
	i(x) &= \frac{-2B\,x\,\dot{x}}{\sqrt{3} \cdot \frac{3a(x)}{\gamma S}} = - \frac{2B \gamma S}{3 \sqrt{3}} \cdot \frac{x\,\dot{x}}{2 \frac{x}{\sqrt{3}}}\\
	&= - B \gamma S \dot{x} / 3. \\
\end{align*}
On calcule donc la force de \textsc{Laplace} :
\begin{align*}
	\vec{F}_{\mathcal{L}} &= i \cdot [\mathrm{CD}] \cdot B \cdot \vec{e}_x\\
	&= i \cdot \frac{2\dot{x}}{\sqrt{3}} B \vec{e}_x \\
	&= -\underbrace{\frac{2 B^2 \gamma S}{3\sqrt{3}}}_\alpha  \cdot x \dot{x} \\
\end{align*}

D'après le \textsc{pfd}, on a donc \[
	m \ddot{x} = - \alpha x \dot{x} \text{ d'où }  \ddot{x} = - \frac{\alpha}{m} x\dot{x} 
.\] où $m = \rho S L$. 

On intègre les deux côtés de l'équation, \[
	[\dot{x}]_0^{t_\mathrm{f}} = -\frac{\alpha}{m} \cdot \left[ \frac{x^2}{2} \right]_0^{t_\mathrm{f}}
.\]
D'où, \[
	0 - v_0 = \frac{-\alpha}{m} \cdot x_\mathrm{f}^2 / 2
.\] On en conclut \[
	x_\mathrm{f} = \sqrt{\frac{2m}{\alpha} v_0} 
.\] 


			\begin{multicols}{2}
	\section{Implications}
	\begin{enumerate}
		\item
			\[
				\begin{prooftree}
					\infer 0[Ax]{p,q \vdash q}
					\infer 1[$\to$i]{q \vdash p \to q}
				\end{prooftree}
			\]
		\item
			\[
				\begin{prooftree}
					\infer 0[Ax]{p, p\land q\vdash p\land q}
					\infer 1[$\land$e,d]{p, p \land q \vdash q}
					\infer 1[$\to$i]{p \land q \vdash p \to q}
				\end{prooftree}
			\]
		\item 
			\[
				\begin{prooftree}
					\infer 0[Ax]{p, p\to q \vdash p}
					\infer 0[Ax]{p, p\to q \vdash p \to q}
					\infer 0[Ax]{p, p\to q \vdash p}
					\infer 2[$\to$e]{p, p \to q \vdash q}
					\infer 2[$\land$i]{p, p \to q \vdash p \land q}
					\infer 1[$\to$i]{p \to q \vdash p\to (p \land q)}
				\end{prooftree}
			\]
		\item
			\[
				\begin{prooftree}
					\infer 0[Ax]{\lnot q, p\to q, p \vdash p \to q}
					\infer 0[Ax]{\lnot q, p\to q, p \vdash p}
					\infer 2[$\to$e]{\lnot q, p\to q, p \vdash q}
					\infer 0[Ax]{\lnot q, p\to q, p \vdash \lnot q}
					\infer 2[$\lnot$e]{\lnot q, p\to q, p \vdash \bot}
					\infer 1[$\lnot$i]{\lnot q, p\to q \vdash\lnot p}
					\infer 1[$\to$i]{p \to q \vdash \lnot q \to \lnot p}
				\end{prooftree}
			\]
		\item 
			\[
				\begin{prooftree}
					\infer 0[Ax]{p \land q, p\to r \vdash p \land q}
					\infer 1[$\land$e,g]{p \land q, p\to r \vdash p}
					\infer 0[Ax]{p \land q, p\to r \vdash p \to r}
					\infer 2[$\to$e]{p \land q, p \to r \vdash r}
					\infer 1[$\to$i]{p \to r \vdash (p \land q) \to r}
				\end{prooftree}
			\] 
		\item
			\[
				\begin{prooftree}
					\infer 0[Ax]{p,q\vdash p}
					\infer 1[$\to$i]{p \vdash q \to p}
					\infer 1[$\to$i]{\vdash p \to (q \to p)}
				\end{prooftree}
			\]
		\item
			\[
				\begin{prooftree}
					\infer 0[Ax]{p \to q, p \vdash p\to q}
					\infer 0[Ax]{p\to q,p \vdash p}
					\infer 2[$\to$e]{p, p\to q \vdash q}
					\infer 1[$\to$i]{p \vdash (p \to q) \to q}
				\end{prooftree}
			\] 
	\end{enumerate}
\end{multicols}

			\section{Langage de \textsc{Dyck}}

\begin{enumerate}
	\item On suppose ce langage reconnaissable par un automate à $n$ états. On considère le mot $w = {\red(}^n \cdot {\red)}^n$, donc $|w| \ge n$.
		Ainsi, il existe $x$, $y$ et $z$ trois mots tels que $w = xyz$, $|xy| \le n$, $y \neq \varepsilon$ et $\forall p \in \N,\: x y^p z \in \mathcal{L}(\mathcal{G})$.
		Soit alors $p \in \llbracket 1,n-1 \rrbracket$ et $q \in \llbracket 1,n -p \rrbracket$ tels que $x = {\red(}^p$, $y = {\red(}^q$ et $z = {\red(}^{n-q-p} \cdot {\red)}^n$.
		Ainsi, $xy \in \mathcal{L}(\mathcal{G})$, ce qui est absurde. On en déduit que $\mathcal{L}(\mathcal{G})$ n'est pas reconnaissable, il n'est donc pas régulier.
	\item On pose $\mathcal{G} = (\Sigma, \{\mathrm{S}\}, \{\mathrm{S} \to \red( \mathrm{S}\red)  \mid \mathrm{SS}  \mid \varepsilon\}, \mathrm{S})$.
	\item
		\begin{itemize}
			\item On le montre par induction.
				\begin{itemize}
					\item \textbf{Cas $\mathrm{S} \to \varepsilon$.}
						On a $|\varepsilon|_{\red(} = 0 = |\varepsilon|_{\red)}$.
					\item \textbf{Cas $\mathrm{S} \to \red( \mathrm{S}\red)$.}
						Soit $u \in \mathcal{L}(\mathcal{G})$ avec $|u|_{\red(} = |u|_{\red)} = n$. Ainsi, $|\red(u\red)|_{\red(} = |\red(u\red)|_{\red)} = n + 1$.
					\item \textbf{Cas $\mathrm{S}\to \mathrm{SS}$.}
						Soient $u$ et $v$ deux mots de $\mathcal{L}(\mathcal{G})$ tels que $|u|_{\red(} = |u|_{\red)} = n$ et $|v|_{\red(} = |v|_{\red)} = m$.
						Alors, $|u\cdot v|_{\red(} = |v\cdot u|_{\red)} = n + m$.
				\end{itemize}
			\item Montrons par induction $\mathcal{P}_{u}$ : \guillemotleft~pour tout $v$ préfixe de $u$, $|v|_{\red(} \ge |v|_{\red)}$.~\guillemotright\@ 
				\begin{itemize}
					\item \textbf{Cas $\mathrm{S} \to \varepsilon$.}
						Le seul préfixe de $\varepsilon$ est $\varepsilon$, et on a bien $|\varepsilon|_{\red(} = 0 \ge 0 = |\varepsilon|_{\red)}$.
					\item \textbf{Cas $\mathrm{S}\to \red( \mathrm{S} \red)$.}
						Soit $u$ un mot de $\mathcal{L}(\mathcal{G})$ vérifiant $\mathcal{P}_u$.
						Soit $v$ un préfixe de $\red(u\red)$.
						On procède par induction sur $v$.
						\begin{itemize}
							\item \textbf{Cas $v = \varepsilon$ ou $\red($.} \textsc{ok}.
							\item \textbf{Cas $\red(\tilde{u}$,} où $\tilde{u}$ est un préfixe de $u$.
								Par hypothèse d'induction, $|\tilde{u}|_{\red(} \ge |\tilde{u}|_{\red)}$ donc $|\red(\tilde{u}|_{\red(} = |\red(\tilde{u}|_{\red)}$.
							\item \textbf{Cas $\red(u\red)$.} Par hypothèse d'induction, $|u|_{\red(} \ge |u|_{\red)}$ donc $|\red(u\red)|_{\red(}\ge |\red(u\red)|_{\red)}$.
						\end{itemize}
				\end{itemize}
		\end{itemize}
	\item On note $\overline w^j= \big| w_{\llbracket 0,j \rrbracket} \big|_{\red(} - \big| w_{\llbracket 0,j \rrbracket} \big|_{\red)}$.
		Alors les deux conditions se traduisent par $\overline w^{|w|} = 0$ et $\forall i \in \llbracket 0,|w|-1 \rrbracket$, $\overline w^i \ge 0$.
\end{enumerate}

			\section{$\mathcal{N}$}

\newcommand{\ofact}{\charfusion[\mathbin]{\bigcirc}{\scriptstyle!}}

\begin{enumerate}
	\item On définit par induction la fonction suivante \begin{align*}
			\oplus: \mathcal{N}^2 &\longrightarrow \mathcal{N} \\
			(\mathbf{S}(x),y) &\longmapsto \oplus(x, \mathbf{S}(y))\\
			(\mathbf{0}, x) &\longmapsto x.
		\end{align*}
	\item Soit $(x,y) \in \mathcal{N}^2$.
		\begin{itemize}
			\item Si $f(x) = 0$, alors $\oplus(x,y) = y$\/ et donc $f(\oplus(x,y)) = f(y) = f(x) + f(y)$.
			\item Si $f(x) \ge 1$, alors  $x = \mathbf{S}(z)$\/ avec $z \in \mathcal{N}$. Ainsi, $\oplus(x,y) = \oplus(z, \mathbf{S}(y))$. Or, $f(z) = f(x) - 1 \le f(x)$. Et donc, par définition de $\oplus$\/ puis par hypothèse d'induction, on a $f(\oplus(x,y)) = f(\oplus(z, \mathbf{S}(y)) = f(z) + f(\mathbf{S}(y))$. On en déduit que $f(\oplus(x,y)) = f(x) - 1 + f(y) + 1 = f(x) + f(y)$.
		\end{itemize}
		Par induction, on a bien $\forall (x,y) \in \mathcal{N}^2,\:f\big({\oplus}(x,y)\big) = f(x) + f(y)$.
	\item On définit par induction la fonction suivante \begin{align*}
			\otimes: \mathcal{N}^2 &\longrightarrow \mathcal{N} \\
			(\mathbf{S}(x), y) &\longmapsto {\oplus}\big(y, {\otimes}(x,y)\big)\\
			(\mathbf{0}, y) &\longmapsto \mathbf{0}.
		\end{align*}
	\item Soit $(x,y) \in \mathcal{N}^2$.
		\begin{itemize}
			\item Si $f(x) = 0$, alors $\otimes(x,y) = \mathbf{0}$, et donc $f(\otimes(x,y)) = 0 = f(x) \times f(y)$.
			\item Si $f(x) \ge 1$, alors $x = \mathbf{S}(z)$\/ avec $z \in \mathcal{N}$. Ainsi, par définition de $\otimes$, on a $\otimes(x,y) = \oplus(y, \otimes(z,y))$. Or, par hypothèse d'induction, $f(\otimes(z,y)) = f(z) \times f(y)$\/ (car $f(z) < f(x)$), et donc $f(\otimes(x,y)) = f(y) + f(\otimes(z,y)) = f(y) + f(z) \times f(y) = f(y) \times (1 + f(z)) = f(y) \times f(x)$.
		\end{itemize}
		Par induction, on a bien $\forall (x,y) \in \mathcal{N}^2,\:f\big({\otimes}(x,y)\big) = f(x) \times f(y)$.
	\item On définit par induction la fonction suivante \begin{align*}
			\ofact : \mathcal{N} &\longrightarrow \mathcal{N} \\
			\mathbf{0} &\longmapsto \mathbf{S}(\mathbf{0})\\
			\mathbf{S}(x) &\longmapsto {\otimes}\big(\mathbf{S}(x), \ofact(x)\big).
		\end{align*}
	\item Soit $x \in \mathcal{N}$.
		\begin{itemize}
			\item Si $f(x)= 0$, alors $\ofact(x) = \mathbf{S}(\mathbf{0})$\/ par définition, et donc $f(\ofact(x)) = 1 = 0! = f(x) !\:$.
			\item Si $f(x) \ge 1$, alors $x = \mathbf{S}(z)$\/ avec $z \in \mathcal{N}$. Ainsi, par définition de $\ofact$, on a $\ofact(x) = \otimes(x, \ofact(z))$, et donc, par hypothèse de récurrence, $f(\ofact(x)) = f(x) \times f(\ofact(z)) = f(x) \times \big(f(z)!\big)$. Or, comme $f(z) = f(x) - 1$, on a donc $f(\ofact(x)) = f(x) \times \big(f(x) - 1\big)! = f(x)!$\:.
		\end{itemize}
		Par induction, on a bien $\forall x \in \mathcal{N},\:f\big({\ofact}(x)\big) = f(x)!\:$.
\end{enumerate}

			\documentclass[a4paper]{article}

\usepackage[margin=1in]{geometry}
\usepackage[utf8]{inputenc}
\usepackage[T1]{fontenc}
\usepackage{mathrsfs}
\usepackage{textcomp}
\usepackage[french]{babel}
\usepackage{amsmath}
\usepackage{amssymb}
\usepackage{cancel}
\usepackage{frcursive}
\usepackage[inline]{asymptote}
\usepackage{tikz}
\usepackage[european,straightvoltages,europeanresistors]{circuitikz}
\usepackage{tikz-cd}
\usepackage{tkz-tab}
\usepackage[b]{esvect}
\usepackage[framemethod=TikZ]{mdframed}
\usepackage{centernot}
\usepackage{diagbox}
\usepackage{dsfont}
\usepackage{fancyhdr}
\usepackage{float}
\usepackage{graphicx}
\usepackage{listings}
\usepackage{multicol}
\usepackage{nicematrix}
\usepackage{pdflscape}
\usepackage{stmaryrd}
\usepackage{xfrac}
\usepackage{hep-math-font}
\usepackage{amsthm}
\usepackage{thmtools}
\usepackage{indentfirst}
\usepackage[framemethod=TikZ]{mdframed}
\usepackage{accents}
\usepackage{soulutf8}
\usepackage{mathtools}
\usepackage{bodegraph}
\usepackage{slashbox}
\usepackage{enumitem}
\usepackage{calligra}
\usepackage{cinzel}
\usepackage{BOONDOX-calo}

% Tikz
\usetikzlibrary{babel}
\usetikzlibrary{positioning}
\usetikzlibrary{calc}

% global settings
\frenchspacing
\reversemarginpar
\setuldepth{a}

%\everymath{\displaystyle}

\frenchbsetup{StandardLists=true}

\def\asydir{asy}

%\sisetup{exponent-product=\cdot,output-decimal-marker={,},separate-uncertainty,range-phrase=\;à\;,locale=FR}

\setlength{\parskip}{1em}

\theoremstyle{definition}

% Changing math
\let\emptyset\varnothing
\let\ge\geqslant
\let\le\leqslant
\let\preceq\preccurlyeq
\let\succeq\succcurlyeq
\let\ds\displaystyle
\let\ts\textstyle

\newcommand{\C}{\mathds{C}}
\newcommand{\R}{\mathds{R}}
\newcommand{\Z}{\mathds{Z}}
\newcommand{\N}{\mathds{N}}
\newcommand{\Q}{\mathds{Q}}

\renewcommand{\O}{\emptyset}

\newcommand\ubar[1]{\underaccent{\bar}{#1}}

\renewcommand\Re{\expandafter\mathfrak{Re}}
\renewcommand\Im{\expandafter\mathfrak{Im}}

\let\slantedpartial\partial
\DeclareRobustCommand{\partial}{\text{\rotatebox[origin=t]{20}{\scalebox{0.95}[1]{$\slantedpartial$}}}\hspace{-1pt}}

% merging two maths characters w/ \charfusion
\makeatletter
\def\moverlay{\mathpalette\mov@rlay}
\def\mov@rlay#1#2{\leavevmode\vtop{%
   \baselineskip\z@skip \lineskiplimit-\maxdimen
   \ialign{\hfil$\m@th#1##$\hfil\cr#2\crcr}}}
\newcommand{\charfusion}[3][\mathord]{
    #1{\ifx#1\mathop\vphantom{#2}\fi
        \mathpalette\mov@rlay{#2\cr#3}
      }
    \ifx#1\mathop\expandafter\displaylimits\fi}
\makeatother

% custom math commands
\newcommand{\T}{{\!\!\,\top}}
\newcommand{\avrt}[1]{\rotatebox{-90}{$#1$}}
\newcommand{\bigcupdot}{\charfusion[\mathop]{\bigcup}{\cdot}}
\newcommand{\cupdot}{\charfusion[\mathbin]{\cup}{\cdot}}
%\newcommand{\danger}{{\large\fontencoding{U}\fontfamily{futs}\selectfont\char 66\relax}\;}
\newcommand{\tendsto}[1]{\xrightarrow[#1]{}}
\newcommand{\vrt}[1]{\rotatebox{90}{$#1$}}
\newcommand{\tsup}[1]{\textsuperscript{\underline{#1}}}
\newcommand{\tsub}[1]{\textsubscript{#1}}

\renewcommand{\mod}[1]{~\left[ #1 \right]}
\renewcommand{\t}{{}^t\!}
\newcommand{\s}{\text{\calligra s}}

% custom units / constants
%\DeclareSIUnit{\litre}{\ell}
\let\hbar\hslash

% header / footer
\pagestyle{fancy}
\fancyhead{} \fancyfoot{}
\fancyfoot[C]{\thepage}

% fonts
\let\sc\scshape
\let\bf\bfseries
\let\it\itshape
\let\sl\slshape

% custom math operators
\let\th\relax
\let\det\relax
\DeclareMathOperator*{\codim}{codim}
\DeclareMathOperator*{\dom}{dom}
\DeclareMathOperator*{\gO}{O}
\DeclareMathOperator*{\po}{\text{\cursive o}}
\DeclareMathOperator*{\sgn}{sgn}
\DeclareMathOperator*{\simi}{\sim}
\DeclareMathOperator{\Arccos}{Arccos}
\DeclareMathOperator{\Arcsin}{Arcsin}
\DeclareMathOperator{\Arctan}{Arctan}
\DeclareMathOperator{\Argsh}{Argsh}
\DeclareMathOperator{\Arg}{Arg}
\DeclareMathOperator{\Aut}{Aut}
\DeclareMathOperator{\Card}{Card}
\DeclareMathOperator{\Cl}{\mathcal{C}\!\ell}
\DeclareMathOperator{\Cov}{Cov}
\DeclareMathOperator{\Ker}{Ker}
\DeclareMathOperator{\Mat}{Mat}
\DeclareMathOperator{\PGCD}{PGCD}
\DeclareMathOperator{\PPCM}{PPCM}
\DeclareMathOperator{\Supp}{Supp}
\DeclareMathOperator{\Vect}{Vect}
\DeclareMathOperator{\argmax}{argmax}
\DeclareMathOperator{\argmin}{argmin}
\DeclareMathOperator{\ch}{ch}
\DeclareMathOperator{\com}{com}
\DeclareMathOperator{\cotan}{cotan}
\DeclareMathOperator{\det}{det}
\DeclareMathOperator{\id}{id}
\DeclareMathOperator{\rg}{rg}
\DeclareMathOperator{\rk}{rk}
\DeclareMathOperator{\sh}{sh}
\DeclareMathOperator{\th}{th}
\DeclareMathOperator{\tr}{tr}

% colors and page style
\definecolor{truewhite}{HTML}{ffffff}
\definecolor{white}{HTML}{faf4ed}
\definecolor{trueblack}{HTML}{000000}
\definecolor{black}{HTML}{575279}
\definecolor{mauve}{HTML}{907aa9}
\definecolor{blue}{HTML}{286983}
\definecolor{red}{HTML}{d7827e}
\definecolor{yellow}{HTML}{ea9d34}
\definecolor{gray}{HTML}{9893a5}
\definecolor{grey}{HTML}{9893a5}
\definecolor{green}{HTML}{a0d971}

\pagecolor{white}
\color{black}

\begin{asydef}
	settings.prc = false;
	settings.render=0;

	white = rgb("faf4ed");
	black = rgb("575279");
	blue = rgb("286983");
	red = rgb("d7827e");
	yellow = rgb("f6c177");
	orange = rgb("ea9d34");
	gray = rgb("9893a5");
	grey = rgb("9893a5");
	deepcyan = rgb("56949f");
	pink = rgb("b4637a");
	magenta = rgb("eb6f92");
	green = rgb("a0d971");
	purple = rgb("907aa9");

	defaultpen(black + fontsize(8pt));

	import three;
	currentlight = nolight;
\end{asydef}

% theorems, proofs, ...

\mdfsetup{skipabove=1em,skipbelow=1em, innertopmargin=6pt, innerbottommargin=6pt,}

\declaretheoremstyle[
	headfont=\normalfont\itshape,
	numbered=no,
	postheadspace=\newline,
	headpunct={:},
	qed=\qedsymbol]{demstyle}

\declaretheorem[style=demstyle, name=Démonstration]{dem}

\newcommand\veczero{\kern-1.2pt\vec{\kern1.2pt 0}} % \vec{0} looks weird since the `0' isn't italicized

\makeatletter
\renewcommand{\title}[2]{
	\AtBeginDocument{
		\begin{titlepage}
			\begin{center}
				\vspace{10cm}
				{\Large \sc Chapitre #1}\\
				\vspace{1cm}
				{\Huge \calligra #2}\\
				\vfill
				Hugo {\sc Salou} MPI${}^{\star}$\\
				{\small Dernière mise à jour le \@date }
			\end{center}
		\end{titlepage}
	}
}

\newcommand{\titletp}[4]{
	\AtBeginDocument{
		\begin{titlepage}
			\begin{center}
				\vspace{10cm}
				{\Large \sc tp #1}\\
				\vspace{1cm}
				{\Huge \textsc{\textit{#2}}}\\
				\vfill
				{#3}\textit{MPI}${}^{\star}$\\
			\end{center}
		\end{titlepage}
	}
	\fancyfoot{}\fancyhead{}
	\fancyfoot[R]{#4 \textit{MPI}${}^{\star}$}
	\fancyhead[C]{{\sc tp #1} : #2}
	\fancyhead[R]{\thepage}
}

\newcommand{\titletd}[2]{
	\AtBeginDocument{
		\begin{titlepage}
			\begin{center}
				\vspace{10cm}
				{\Large \sc td #1}\\
				\vspace{1cm}
				{\Huge \calligra #2}\\
				\vfill
				Hugo {\sc Salou} MPI${}^{\star}$\\
				{\small Dernière mise à jour le \@date }
			\end{center}
		\end{titlepage}
	}
}
\makeatother

\newcommand{\sign}{
	\null
	\vfill
	\begin{center}
		{
			\fontfamily{ccr}\selectfont
			\textit{\textbf{\.{\"i}}}
		}
	\end{center}
	\vfill
	\null
}

\renewcommand{\thefootnote}{\emph{\alph{footnote}}}

% figure support
\usepackage{import}
\usepackage{xifthen}
\pdfminorversion=7
\usepackage{pdfpages}
\usepackage{transparent}
\newcommand{\incfig}[1]{%
	\def\svgwidth{\columnwidth}
	\import{./figures/}{#1.pdf_tex}
}

\pdfsuppresswarningpagegroup=1
\ctikzset{tripoles/european not symbol=circle}

\newcommand{\missingpart}{{\large\color{red} Il manque quelque chose ici\ldots}}


\fancyhead[R]{Hugo {\sc Salou}\/ MPI}
\fancyhead[L]{TD\textsubscript4 -- Exercice 8}

\begin{document}
	\let\thesection\relax
	
\begin{comment}
\section{Exercice 9}

\slshape
Soit la matrice $A = \begin{pmatrix}
	1&1&-1\\
	2&3&-4\\
	4&1&-4
\end{pmatrix}$.
\begin{enumerate}
	\item Déterminer le spectre de la matrice $A$\/ et trouver une matrice $P$\/ inversible telle que $P^{-1} A P$\/ est diagonale.
	\item Soit $B$\/ une matrice de taille $3\times 3$\/ qui commentent avec $A$\/ ($AB = BA$). Montrer que $B$\/ est diagonale.
\end{enumerate}
\upshape

\begin{enumerate}
	\item On sait, tout d'abord, que, pour $x \in \R$,
		\begin{align*}
			\chi_A(x) = \det(x\,I_n - A) &=
			\begin{vmatrix}
				x - 1 &- 1 & 1\\
				-2&x-3&4\\
				-4&-1&x + 4
			\end{vmatrix}\\
			&= \\
		\end{align*}
\end{enumerate}
\end{comment}

\section{Exercice 8}

\begin{enumerate}
	\item Soit un vecteur non nul $\vec{x} \in \Ker(\lambda {\id} - {u \circ v})$. Ainsi, $u(v(\vec{x})) = \lambda \vec{x}$. Et, donc $v(u(v(\vec{x}))) = \lambda v(\vec{x})$. On a donc $v(\vec{x}) \in \Ker(\lambda {\id} - {v  \circ u})$.
		Or, si $\lambda \neq 0$, on a $v(\vec{x}) \neq \vec{0}$\/ ; en effet, si $v(\vec{x}) = \vec{0}$, alors $u \circ v(\vec{x}) = \vec{0} = \lambda \vec{x}$\/ et donc $\vec{x} = \vec{0}$, ce ne serait donc pas un vecteur propre de $u \circ v$\/ : une contradiction. On en déduit que $v(\vec{x})$\/ est un vecteur propre de $u \circ v$\/ associé à la valeur propre $\lambda$.
	\item On pose donc $\lambda = 0$, une valeur propre de $u  \circ v$. L'endomorphisme $u \circ v$\/ n'est donc pas injectif, donc bijectif. On sait donc, comme $E$\/ est de dimension finie, que $\det(u \circ v) = 0$. Or $\det (u \circ v) = \det u \times \det v = \det(v  \circ u)$. Et donc $\det(v  \circ u) = 0$, $v  \circ u$\/ n'est donc pas bijectif, donc injectif. Et donc, on a $0 \in \Sp(v  \circ u)$.
	\item Soit $P \in \R[X]$, et soit $Q$\/ une primitive de $P$.
		\begin{align*}
			P \in \Ker (u  \circ v) \iff& \Big(\int_{0}^X P(t)~\mathrm{d}t\Big)' = 0\\
			\iff& \big(Q(X) - Q(0)\big)' = 0\\
			\iff& Q'(X) = 0\\
			\iff& P(X) = 0
		\end{align*}
		On en déduit que $\Ker (u \circ v) = \{0\}$.

		Également,
		\begin{align*}
			P \in \Ker(v  \circ u) \iff& \int_{0}^{X} P'(t)~\mathrm{d}t = 0\\
			\iff& P(X) - P(0) = 0\\
			\iff& P(X) = P(0)\\
			\iff& \deg P \le 0\\
			\iff& P \in \R_0[X]
		\end{align*}
		On en déduit que $\Ker(v \circ u) = \R_0[X]$.
\end{enumerate}


\end{document}

			\section{Complétion d'automate}

\begin{enumerate}
	\item Non, cet automate n'est pas complet. Par exemple, la lecture d'un $b$\/ à l'état 1 est impossible.
	\item Cet automate reconnaît le langage $L = \mathcal{L}\big(a \cdot b\cdot (a \mid b)^*\big)$.
	\item~

		\begin{figure}[H]
			\centering
			\tikzfig{automate-ex9}
			\caption{Automate complet équivalent à $\mathcal{A}$}
		\end{figure}
\end{enumerate}



			\section{Exercice 10}

\begin{enumerate}
	\item On a $u_6 \leadsto \hat{v}_\mathrm{e}$, $u_3 \leadsto \hat{v}_\mathrm{c}$\/ (composante continue), $u_2 \leadsto \hat{v}_\mathrm{d}$\/ (discontinuités), $u_4 \leadsto \hat{v}_\mathrm{a}$\/ (composante continue), $u_5 \leadsto \hat{v}_\mathrm{f}$\/ (nombre de fréquences) et $u_1 \leadsto \hat{v}_\mathrm{b}$.
	\item
		Le filtre donnant le signal $\hat{v}_\mathrm{g}$\/ est un passe-bandes (les hautes et basses fréquences sont éliminées) et c'est un filtre non linéaire (de nouvelles fréquences apparaissent).

		Le filtre donnant le signal $\hat{v}_\mathrm{h}$\/ est un filtre passe-bas.

		Le filtre donnant le signal $\hat{v}_\mathrm{i}$\/ est un filtre passe-haut dont sa fréquence de coupure est inférieure à $1\:\mathrm{kHz}$.
\end{enumerate}


			\section{Déterminisation 1}

\begin{enumerate}
	\item \tikzfig{ex11-1}
	\item \tikzfig{ex11-2}
\end{enumerate}

			\section{Déterminisation 2}

\begin{enumerate}
	\item \tikzfig{ex12-1}
	\item \tikzfig{ex12-2}
	\item \tikzfig{ex12-3}
\end{enumerate}

			\section{Exercice 14}

\begin{enumerate}
	\item Grâce aux impédances d'entrée infinies, on a $u_r = r i_D(t)$\/ et $u_1 = K_\text{m}\,r\,i_D(t)\,u_D(t)$. C'est un Wattmètre car le produit $i_D(t) \times u_D(t)$\/ correspond à la puissance reçue par $D$ (car elle est positive). Ce Wattmètre est analogique car $i_D(t)$\/ et $u_D(t)$\/ sont des grandeurs qui varient dans un interval continu.

	\item On a $\omega_\text{c} = \frac{1}{RC}$\/ d'où $f_\text{c} = \frac{1}{2\pi\,R\,C}\circeq \frac{1}{0{,}6}\:\mathrm{Hz} \circeq 2\:\mathrm{Hz}$.

		La fonction de ce circuit est de calculer la moyenne d'un signal (c'est donc un moyenneur) : il filtre les fréquences supérieures à quelques $\:\mathrm{Hz}$. On n'obtient donc que la composante continue comme montré sur la figure ci-dessous.

		\begin{figure}[H]
			\centering
			\begin{asy}
				import graph;
				size(7cm);
				real f(real x) { return (-atan((x - 2)*3)/pi + 1/2)*7; }
				draw((-1, 0) -- (15, 0), Arrow(TeXHead));
				draw((0, -1) -- (0, 8), Arrow(TeXHead));
				draw(graph(f, 0, 15), green);
				draw((5,0)--(5,5), orange);
				draw((0,0)--(0,5), orange);
				dot((5,5), orange);
				dot((0,5), orange);
			\end{asy}
			\caption{Moyenneur}
		\end{figure}
		
		{\color{red} On ne passe pas par les complexes} mais on revient aux définitions réelles d'un signal sinusoïdal :
		\begin{align*}
			u_1(t) &= K_\text{m} r\,i_D(t)\,u_D(t) \\
			&= K_\text{m}\,r\,I_D\cos(2\pi f + \varphi_i)\,U_D \cos(2\pi f t + \varphi_u) \\
			&= K_\text{m}\,r\,I_D\,U_D \times \frac{1}{2}\Big(\cos(2\pi\times 2f t + \varphi_i + \varphi_u) - \cos(\varphi_u - \varphi_i)\Big). \\
		\end{align*}
		On en déduit donc que \[
			u_s(t) \sim K_\text{m}\,r\,\underbrace{\frac{1}{2}\,U_D\,I_D\cos(\varphi_u - \varphi_i)}_{\langle p(t)\rangle}
		.\]
	\item
		\begin{itemize}
			\item Si $D = R$, on a
				\[
					\langle p(t) \rangle = \frac{1}{2}\,U_D\,I_D \cos(\overbrace{\varphi_u - \varphi_i}^{0}) = \frac{1}{2} R I_D^2 = \frac{U_D^2}{2R}
				.\] 
			\item Si $D = C$, on a $\ubar{u}_C = \ubar{Z}_C \ubar{i}_C = \frac{1}{\mathrm{j}C\omega} \ubar{i}_C$. Or, $\Arg(\ubar{Z}_C) = \Arg\left( \frac{\ubar{u}}{\ubar{i}} \right) = \Arg(\ubar{u}) - \Arg(\ubar{i}) = \varphi_u - \varphi_i = -\frac{\pi}{2}$. On en conclut donc que $\langle p(t) \rangle = 0$.
			\item Si $D = L$, on a $\Arg(\ubar{Z}_L) = \varphi_u - \varphi_i = \frac{\pi}{2}$\/ et donc $\langle p(t) \rangle  = 0$.
		\end{itemize}
\end{enumerate}

		\end{landscape}
	}
	\def\addmacros#1{#1}
}
{
	\td[11]{Preuves}
	\minitoc
	\renewcommand{\cwd}{../td/td11/}
	\addmacros{
		\section{Quelques problèmes décidables}

\begin{enumerate}
	\item Soit $f : \R \to \R$.
		\begin{itemize}
			\item Si $f$\/ admet un zéro, on pose $\mathcal{M} = \texttt{fun}\ \texttt{s}\ \to \texttt{true}$.
			\item Si $f$\/ n'admet pas un zéro, on pose $\mathcal{M} = \texttt{fun}\ \texttt{s}\ \to \texttt{false}$.
		\end{itemize}
		Alors, $\mathcal{M}$\/ décide \textsc{Zero}$_f$.
	\item Soit $\mathcal{M}$\/ une machine, et soit $w \in \Sigma^*$.
		\begin{itemize}
			\item Si $\mathcal{M}$\/ se termine sur l'entrée $w$, alors on pose $\mathcal{M}' = \texttt{fun}\ \texttt{s} \to \texttt{true}$.
			\item Si $\mathcal{M}$\/ ne se termine pas sur l'entrée $w$, alors on pose $\mathcal{M}' = \texttt{fun}\ \texttt{s} \to \texttt{false}$.
		\end{itemize}
		Alors, $\mathcal{M}'$\/ décide \textsc{Arrêt}$_{\mathcal{M},w}$.
	\item Le problème est trivialement vrai. En effet, soit $M \in \mathcal{O}$, de la forme
		\begin{lstlisting}[language=caml]
let m (s: string): string =
	%*$\langle$\textrm{code}$\rangle$*) 
		\end{lstlisting}
		On crée la machine $\mathcal{N}$\/ ci-dessous.
		\begin{lstlisting}[language=caml]
let n (s: string): string =
	if true then
		%*$\langle$\textrm{code}$\rangle$*) 
	else
		%*$\langle$\textrm{code}$\rangle$*) 
		\end{lstlisting}
		On a $\texttt{m} \neq \texttt{n}$, mais $\mathcal{L}(\texttt{m}) = \mathcal{L}(\texttt{n})$, donc le problème est vrai sur toute entrée et la fonction $\texttt{fun}\ \texttt{s} \to \texttt{true}$\/ répond au problème.
\end{enumerate}

		\section{Ensembles définis inductivement}

La correction est disponible sur \textit{cahier-de-prepa}.

\begin{comment}
	\begin{exm}
		Avec $S = \N$, $\mathcal{B} = \{0, 2\} $, $A_1 = \{0\}$\/ et \begin{align*}
			f_1: A_1 \times \N &\longrightarrow \N \\
			(0, x) &\longmapsto x + 4.
		\end{align*}

		On a \[
			X \supseteq \{0, 2, 4, 6, 8, 10, \ldots, 20, \ldots\} = 2\N
		.\]
	\end{exm}
	\begin{exm}
		Avec $S$\/ l'ensemble des langages sur $\Sigma$, $\mathcal{B} = \{\O\} \cup \bigl\{\{a\}\:\big|\: a \in \Sigma \bigr\}$, et
		\begin{multicols}{3}
			\begin{align*}
				f_1: S \times S &\longrightarrow S \\
				(L_1, L_2) &\longmapsto L_1 \cup L_2,
			\end{align*}
			\begin{align*}
				f_2: S \times S &\longrightarrow S \\
				(L_1, L_2) &\longmapsto L_1 \cdot L_2,
			\end{align*}
			\begin{align*}
				f_3: S &\longrightarrow S \\
				L &\longmapsto L^*.
			\end{align*}
		\end{multicols}
	\end{exm}

\begin{enumerate}
	\item Soit $\mathcal{A} = \{X \subseteq S  \mid X \supseteq \mathcal{B} \mathrel{\text{et}} X \text{ est stable par } f_i\}$. On a $S \in \mathcal{A}$\/ et donc $\mathcal{A} \neq \O$. De plus, soit \[
			Y = \{x \in S  \mid \forall X \in \mathcal{A},\,x \in X\} = \bigcap_{X \in \mathcal{A}} X
		.\]
		Soit $b \in \mathcal{B}$, on a $\forall X \in A,\, b \in X$. D'où $b \in Y$\/ par intersection. On en déduit que $\mathcal{B} \subseteq Y$.

		Soit $i \in \left\llbracket 1,m \right\rrbracket$. Soit $(x_1, \ldots, x_{n_i}) \in Y^{n_i}$\/ et soit $a \in A_i$. Montrons que $f_i(a, x_1, \ldots, x_{n_i}) \in Y$.
		Or, soit $X \in \mathcal{A}$, on a $(x_1, \ldots, x_{n_i}) \in X^{n_i}$\/ donc $f_i(a, x_1, \ldots, x_{n_i}) \in X$. Ceci étant vrai pour tout $X \in \mathcal{A}$, on a $f_i(a, x_1, \ldots, x_{n_i}) \in Y$\/ donc $Y$\/ est stable par $f_i$\/ par tout $i \in \left\llbracket 1,m \right\rrbracket$\/ et donc $Y \in \mathcal{A}$.
		On a également $Y \subseteq X$\/ pour tout $X \in \mathcal{A}$. On en déduit que $Y$\/ est le plus petit élément (pour l'inclusion) de $\mathcal{A}$.
	\item On pose $X_0  = \mathcal{B}$\/ et \[
			X_{n+1} = X_n \cup \big\{ f_i(a, x_1, \ldots, x_{n_i})  \mid a \in A_i,\,(x_1, \ldots, x_{n_i}) \in (X_n)^{n_i},\,i \in \left\llbracket 1,m \right\rrbracket\big\}
		.\]
		Soit $X = \bigcup_{n \in \N} X_n$. Soit $Y$\/ l'ensemble défini par induction à partir de $\mathcal{B}$\/ et des $(f_i)_{i\in\left\llbracket 1,n \right\rrbracket}$. Montrons que $X = Y$.
		On montre que $X$\/ est le plus petit élément (pour l'inclusion) de $\mathcal{A}$\/ et on conclut par unicité du minimum (avec la question précédente).
		Par définition de la suite $(X_n)_{n\in\N}$, elle est croissante (au sens de l'inclusion).
		Montrons à présent, par récurrence, la propriété ci-dessous : $P_n : ``X_n \subseteq Y."$
		\begin{itemize}
			\item Par définition de $Y$, on a $X_0 = \mathcal{B} \subseteq Y$.
			\item Soit 
		\end{itemize}
\end{enumerate}

\subsection{Un théorème d'induction}

\begin{enumerate}
	\item[3.] Soit $Z = \{x \in S  \mid P(x) \text{ vraie}\:\}$.
		Montrons que $\mathcal{X}\subseteq Z$.
		On remarque que $\mathcal{X} \supseteq \mathcal{B}$\/ ; $\mathcal{X}$\/ est stable par $f_i$. On en conclut que $Z \supseteq \mathcal{X}$ et donc $\forall x \in \mathcal{X},\,P(x)$\/ est vraie.
\end{enumerate}

\begin{exm}
	Soit $\mathcal{X}$\/ défini par induction par $\mathcal{B} = \{0, 2\}$\/ et \begin{align*}
		f: \N &\longrightarrow \N \\
		n &\longmapsto n + 2.
	\end{align*}
	Montrons que $\forall n \in \mathcal{X}$, $x$\/ est pair.

	On sait que $0$\/ est pair, $2$\/ est pair ; et, \[
		\forall x,y \in \mathcal{X},\, (x \text{ pair} \land y \text{ pair}) \implies f(x, y) \text{ pair}
	.\]

	On en déduit que \[
		\forall n \in \mathcal{X},\,x \text{ est pair}.
	.\]
\end{exm}
\end{comment}

		\section{Tableaux dynamiques}

\begin{enumerate}[start=3]
	\item On trouve une complexité amortie en $n^2$. À rédiger.
	\item Au lieu de diviser quand $r < n / 2$, mais quand $r < n / 4$.
\end{enumerate}

		\section{Barres en triangle}


On note $a(x)$ le côté du triangule équilatéral, et donc $x = a(x) \sqrt{3}  / 2$.

On calcule le flux $\Phi$ magnétique : \[
	\Phi = B \frac{a x}{2} = B x^2 / \sqrt{3}
.\]
Ainsi, d'après la loi de \textsc{Faraday}, on a \[
	e = - \frac{\mathrm{d}\Phi}{\mathrm{d}t} = - \frac{2B}{\sqrt{3}}  \: x\,\dot{x}
.\]
Or, par loi d'\textsc{Ohm}, $i = e / R(x)$.
Et, on connait la resistance du circuit $R(x) = 3 a(x) / \gamma S$.
Alors, 
\begin{align*}
	i(x) &= \frac{-2B\,x\,\dot{x}}{\sqrt{3} \cdot \frac{3a(x)}{\gamma S}} = - \frac{2B \gamma S}{3 \sqrt{3}} \cdot \frac{x\,\dot{x}}{2 \frac{x}{\sqrt{3}}}\\
	&= - B \gamma S \dot{x} / 3. \\
\end{align*}
On calcule donc la force de \textsc{Laplace} :
\begin{align*}
	\vec{F}_{\mathcal{L}} &= i \cdot [\mathrm{CD}] \cdot B \cdot \vec{e}_x\\
	&= i \cdot \frac{2\dot{x}}{\sqrt{3}} B \vec{e}_x \\
	&= -\underbrace{\frac{2 B^2 \gamma S}{3\sqrt{3}}}_\alpha  \cdot x \dot{x} \\
\end{align*}

D'après le \textsc{pfd}, on a donc \[
	m \ddot{x} = - \alpha x \dot{x} \text{ d'où }  \ddot{x} = - \frac{\alpha}{m} x\dot{x} 
.\] où $m = \rho S L$. 

On intègre les deux côtés de l'équation, \[
	[\dot{x}]_0^{t_\mathrm{f}} = -\frac{\alpha}{m} \cdot \left[ \frac{x^2}{2} \right]_0^{t_\mathrm{f}}
.\]
D'où, \[
	0 - v_0 = \frac{-\alpha}{m} \cdot x_\mathrm{f}^2 / 2
.\] On en conclut \[
	x_\mathrm{f} = \sqrt{\frac{2m}{\alpha} v_0} 
.\] 


		\begin{multicols}{2}
	\section{Implications}
	\begin{enumerate}
		\item
			\[
				\begin{prooftree}
					\infer 0[Ax]{p,q \vdash q}
					\infer 1[$\to$i]{q \vdash p \to q}
				\end{prooftree}
			\]
		\item
			\[
				\begin{prooftree}
					\infer 0[Ax]{p, p\land q\vdash p\land q}
					\infer 1[$\land$e,d]{p, p \land q \vdash q}
					\infer 1[$\to$i]{p \land q \vdash p \to q}
				\end{prooftree}
			\]
		\item 
			\[
				\begin{prooftree}
					\infer 0[Ax]{p, p\to q \vdash p}
					\infer 0[Ax]{p, p\to q \vdash p \to q}
					\infer 0[Ax]{p, p\to q \vdash p}
					\infer 2[$\to$e]{p, p \to q \vdash q}
					\infer 2[$\land$i]{p, p \to q \vdash p \land q}
					\infer 1[$\to$i]{p \to q \vdash p\to (p \land q)}
				\end{prooftree}
			\]
		\item
			\[
				\begin{prooftree}
					\infer 0[Ax]{\lnot q, p\to q, p \vdash p \to q}
					\infer 0[Ax]{\lnot q, p\to q, p \vdash p}
					\infer 2[$\to$e]{\lnot q, p\to q, p \vdash q}
					\infer 0[Ax]{\lnot q, p\to q, p \vdash \lnot q}
					\infer 2[$\lnot$e]{\lnot q, p\to q, p \vdash \bot}
					\infer 1[$\lnot$i]{\lnot q, p\to q \vdash\lnot p}
					\infer 1[$\to$i]{p \to q \vdash \lnot q \to \lnot p}
				\end{prooftree}
			\]
		\item 
			\[
				\begin{prooftree}
					\infer 0[Ax]{p \land q, p\to r \vdash p \land q}
					\infer 1[$\land$e,g]{p \land q, p\to r \vdash p}
					\infer 0[Ax]{p \land q, p\to r \vdash p \to r}
					\infer 2[$\to$e]{p \land q, p \to r \vdash r}
					\infer 1[$\to$i]{p \to r \vdash (p \land q) \to r}
				\end{prooftree}
			\] 
		\item
			\[
				\begin{prooftree}
					\infer 0[Ax]{p,q\vdash p}
					\infer 1[$\to$i]{p \vdash q \to p}
					\infer 1[$\to$i]{\vdash p \to (q \to p)}
				\end{prooftree}
			\]
		\item
			\[
				\begin{prooftree}
					\infer 0[Ax]{p \to q, p \vdash p\to q}
					\infer 0[Ax]{p\to q,p \vdash p}
					\infer 2[$\to$e]{p, p\to q \vdash q}
					\infer 1[$\to$i]{p \vdash (p \to q) \to q}
				\end{prooftree}
			\] 
	\end{enumerate}
\end{multicols}

		\section{Langage de \textsc{Dyck}}

\begin{enumerate}
	\item On suppose ce langage reconnaissable par un automate à $n$ états. On considère le mot $w = {\red(}^n \cdot {\red)}^n$, donc $|w| \ge n$.
		Ainsi, il existe $x$, $y$ et $z$ trois mots tels que $w = xyz$, $|xy| \le n$, $y \neq \varepsilon$ et $\forall p \in \N,\: x y^p z \in \mathcal{L}(\mathcal{G})$.
		Soit alors $p \in \llbracket 1,n-1 \rrbracket$ et $q \in \llbracket 1,n -p \rrbracket$ tels que $x = {\red(}^p$, $y = {\red(}^q$ et $z = {\red(}^{n-q-p} \cdot {\red)}^n$.
		Ainsi, $xy \in \mathcal{L}(\mathcal{G})$, ce qui est absurde. On en déduit que $\mathcal{L}(\mathcal{G})$ n'est pas reconnaissable, il n'est donc pas régulier.
	\item On pose $\mathcal{G} = (\Sigma, \{\mathrm{S}\}, \{\mathrm{S} \to \red( \mathrm{S}\red)  \mid \mathrm{SS}  \mid \varepsilon\}, \mathrm{S})$.
	\item
		\begin{itemize}
			\item On le montre par induction.
				\begin{itemize}
					\item \textbf{Cas $\mathrm{S} \to \varepsilon$.}
						On a $|\varepsilon|_{\red(} = 0 = |\varepsilon|_{\red)}$.
					\item \textbf{Cas $\mathrm{S} \to \red( \mathrm{S}\red)$.}
						Soit $u \in \mathcal{L}(\mathcal{G})$ avec $|u|_{\red(} = |u|_{\red)} = n$. Ainsi, $|\red(u\red)|_{\red(} = |\red(u\red)|_{\red)} = n + 1$.
					\item \textbf{Cas $\mathrm{S}\to \mathrm{SS}$.}
						Soient $u$ et $v$ deux mots de $\mathcal{L}(\mathcal{G})$ tels que $|u|_{\red(} = |u|_{\red)} = n$ et $|v|_{\red(} = |v|_{\red)} = m$.
						Alors, $|u\cdot v|_{\red(} = |v\cdot u|_{\red)} = n + m$.
				\end{itemize}
			\item Montrons par induction $\mathcal{P}_{u}$ : \guillemotleft~pour tout $v$ préfixe de $u$, $|v|_{\red(} \ge |v|_{\red)}$.~\guillemotright\@ 
				\begin{itemize}
					\item \textbf{Cas $\mathrm{S} \to \varepsilon$.}
						Le seul préfixe de $\varepsilon$ est $\varepsilon$, et on a bien $|\varepsilon|_{\red(} = 0 \ge 0 = |\varepsilon|_{\red)}$.
					\item \textbf{Cas $\mathrm{S}\to \red( \mathrm{S} \red)$.}
						Soit $u$ un mot de $\mathcal{L}(\mathcal{G})$ vérifiant $\mathcal{P}_u$.
						Soit $v$ un préfixe de $\red(u\red)$.
						On procède par induction sur $v$.
						\begin{itemize}
							\item \textbf{Cas $v = \varepsilon$ ou $\red($.} \textsc{ok}.
							\item \textbf{Cas $\red(\tilde{u}$,} où $\tilde{u}$ est un préfixe de $u$.
								Par hypothèse d'induction, $|\tilde{u}|_{\red(} \ge |\tilde{u}|_{\red)}$ donc $|\red(\tilde{u}|_{\red(} = |\red(\tilde{u}|_{\red)}$.
							\item \textbf{Cas $\red(u\red)$.} Par hypothèse d'induction, $|u|_{\red(} \ge |u|_{\red)}$ donc $|\red(u\red)|_{\red(}\ge |\red(u\red)|_{\red)}$.
						\end{itemize}
				\end{itemize}
		\end{itemize}
	\item On note $\overline w^j= \big| w_{\llbracket 0,j \rrbracket} \big|_{\red(} - \big| w_{\llbracket 0,j \rrbracket} \big|_{\red)}$.
		Alors les deux conditions se traduisent par $\overline w^{|w|} = 0$ et $\forall i \in \llbracket 0,|w|-1 \rrbracket$, $\overline w^i \ge 0$.
\end{enumerate}

		\section{$\mathcal{N}$}

\newcommand{\ofact}{\charfusion[\mathbin]{\bigcirc}{\scriptstyle!}}

\begin{enumerate}
	\item On définit par induction la fonction suivante \begin{align*}
			\oplus: \mathcal{N}^2 &\longrightarrow \mathcal{N} \\
			(\mathbf{S}(x),y) &\longmapsto \oplus(x, \mathbf{S}(y))\\
			(\mathbf{0}, x) &\longmapsto x.
		\end{align*}
	\item Soit $(x,y) \in \mathcal{N}^2$.
		\begin{itemize}
			\item Si $f(x) = 0$, alors $\oplus(x,y) = y$\/ et donc $f(\oplus(x,y)) = f(y) = f(x) + f(y)$.
			\item Si $f(x) \ge 1$, alors  $x = \mathbf{S}(z)$\/ avec $z \in \mathcal{N}$. Ainsi, $\oplus(x,y) = \oplus(z, \mathbf{S}(y))$. Or, $f(z) = f(x) - 1 \le f(x)$. Et donc, par définition de $\oplus$\/ puis par hypothèse d'induction, on a $f(\oplus(x,y)) = f(\oplus(z, \mathbf{S}(y)) = f(z) + f(\mathbf{S}(y))$. On en déduit que $f(\oplus(x,y)) = f(x) - 1 + f(y) + 1 = f(x) + f(y)$.
		\end{itemize}
		Par induction, on a bien $\forall (x,y) \in \mathcal{N}^2,\:f\big({\oplus}(x,y)\big) = f(x) + f(y)$.
	\item On définit par induction la fonction suivante \begin{align*}
			\otimes: \mathcal{N}^2 &\longrightarrow \mathcal{N} \\
			(\mathbf{S}(x), y) &\longmapsto {\oplus}\big(y, {\otimes}(x,y)\big)\\
			(\mathbf{0}, y) &\longmapsto \mathbf{0}.
		\end{align*}
	\item Soit $(x,y) \in \mathcal{N}^2$.
		\begin{itemize}
			\item Si $f(x) = 0$, alors $\otimes(x,y) = \mathbf{0}$, et donc $f(\otimes(x,y)) = 0 = f(x) \times f(y)$.
			\item Si $f(x) \ge 1$, alors $x = \mathbf{S}(z)$\/ avec $z \in \mathcal{N}$. Ainsi, par définition de $\otimes$, on a $\otimes(x,y) = \oplus(y, \otimes(z,y))$. Or, par hypothèse d'induction, $f(\otimes(z,y)) = f(z) \times f(y)$\/ (car $f(z) < f(x)$), et donc $f(\otimes(x,y)) = f(y) + f(\otimes(z,y)) = f(y) + f(z) \times f(y) = f(y) \times (1 + f(z)) = f(y) \times f(x)$.
		\end{itemize}
		Par induction, on a bien $\forall (x,y) \in \mathcal{N}^2,\:f\big({\otimes}(x,y)\big) = f(x) \times f(y)$.
	\item On définit par induction la fonction suivante \begin{align*}
			\ofact : \mathcal{N} &\longrightarrow \mathcal{N} \\
			\mathbf{0} &\longmapsto \mathbf{S}(\mathbf{0})\\
			\mathbf{S}(x) &\longmapsto {\otimes}\big(\mathbf{S}(x), \ofact(x)\big).
		\end{align*}
	\item Soit $x \in \mathcal{N}$.
		\begin{itemize}
			\item Si $f(x)= 0$, alors $\ofact(x) = \mathbf{S}(\mathbf{0})$\/ par définition, et donc $f(\ofact(x)) = 1 = 0! = f(x) !\:$.
			\item Si $f(x) \ge 1$, alors $x = \mathbf{S}(z)$\/ avec $z \in \mathcal{N}$. Ainsi, par définition de $\ofact$, on a $\ofact(x) = \otimes(x, \ofact(z))$, et donc, par hypothèse de récurrence, $f(\ofact(x)) = f(x) \times f(\ofact(z)) = f(x) \times \big(f(z)!\big)$. Or, comme $f(z) = f(x) - 1$, on a donc $f(\ofact(x)) = f(x) \times \big(f(x) - 1\big)! = f(x)!$\:.
		\end{itemize}
		Par induction, on a bien $\forall x \in \mathcal{N},\:f\big({\ofact}(x)\big) = f(x)!\:$.
\end{enumerate}

	}
	\def\addmacros#1{#1}
}
{
	\td[12]{Algorithmes d'approximation}
	\minitoc
	\renewcommand{\cwd}{../td/td12/}
	\addmacros{
		\section{Quelques problèmes décidables}

\begin{enumerate}
	\item Soit $f : \R \to \R$.
		\begin{itemize}
			\item Si $f$\/ admet un zéro, on pose $\mathcal{M} = \texttt{fun}\ \texttt{s}\ \to \texttt{true}$.
			\item Si $f$\/ n'admet pas un zéro, on pose $\mathcal{M} = \texttt{fun}\ \texttt{s}\ \to \texttt{false}$.
		\end{itemize}
		Alors, $\mathcal{M}$\/ décide \textsc{Zero}$_f$.
	\item Soit $\mathcal{M}$\/ une machine, et soit $w \in \Sigma^*$.
		\begin{itemize}
			\item Si $\mathcal{M}$\/ se termine sur l'entrée $w$, alors on pose $\mathcal{M}' = \texttt{fun}\ \texttt{s} \to \texttt{true}$.
			\item Si $\mathcal{M}$\/ ne se termine pas sur l'entrée $w$, alors on pose $\mathcal{M}' = \texttt{fun}\ \texttt{s} \to \texttt{false}$.
		\end{itemize}
		Alors, $\mathcal{M}'$\/ décide \textsc{Arrêt}$_{\mathcal{M},w}$.
	\item Le problème est trivialement vrai. En effet, soit $M \in \mathcal{O}$, de la forme
		\begin{lstlisting}[language=caml]
let m (s: string): string =
	%*$\langle$\textrm{code}$\rangle$*) 
		\end{lstlisting}
		On crée la machine $\mathcal{N}$\/ ci-dessous.
		\begin{lstlisting}[language=caml]
let n (s: string): string =
	if true then
		%*$\langle$\textrm{code}$\rangle$*) 
	else
		%*$\langle$\textrm{code}$\rangle$*) 
		\end{lstlisting}
		On a $\texttt{m} \neq \texttt{n}$, mais $\mathcal{L}(\texttt{m}) = \mathcal{L}(\texttt{n})$, donc le problème est vrai sur toute entrée et la fonction $\texttt{fun}\ \texttt{s} \to \texttt{true}$\/ répond au problème.
\end{enumerate}

		\section{Ensembles définis inductivement}

La correction est disponible sur \textit{cahier-de-prepa}.

\begin{comment}
	\begin{exm}
		Avec $S = \N$, $\mathcal{B} = \{0, 2\} $, $A_1 = \{0\}$\/ et \begin{align*}
			f_1: A_1 \times \N &\longrightarrow \N \\
			(0, x) &\longmapsto x + 4.
		\end{align*}

		On a \[
			X \supseteq \{0, 2, 4, 6, 8, 10, \ldots, 20, \ldots\} = 2\N
		.\]
	\end{exm}
	\begin{exm}
		Avec $S$\/ l'ensemble des langages sur $\Sigma$, $\mathcal{B} = \{\O\} \cup \bigl\{\{a\}\:\big|\: a \in \Sigma \bigr\}$, et
		\begin{multicols}{3}
			\begin{align*}
				f_1: S \times S &\longrightarrow S \\
				(L_1, L_2) &\longmapsto L_1 \cup L_2,
			\end{align*}
			\begin{align*}
				f_2: S \times S &\longrightarrow S \\
				(L_1, L_2) &\longmapsto L_1 \cdot L_2,
			\end{align*}
			\begin{align*}
				f_3: S &\longrightarrow S \\
				L &\longmapsto L^*.
			\end{align*}
		\end{multicols}
	\end{exm}

\begin{enumerate}
	\item Soit $\mathcal{A} = \{X \subseteq S  \mid X \supseteq \mathcal{B} \mathrel{\text{et}} X \text{ est stable par } f_i\}$. On a $S \in \mathcal{A}$\/ et donc $\mathcal{A} \neq \O$. De plus, soit \[
			Y = \{x \in S  \mid \forall X \in \mathcal{A},\,x \in X\} = \bigcap_{X \in \mathcal{A}} X
		.\]
		Soit $b \in \mathcal{B}$, on a $\forall X \in A,\, b \in X$. D'où $b \in Y$\/ par intersection. On en déduit que $\mathcal{B} \subseteq Y$.

		Soit $i \in \left\llbracket 1,m \right\rrbracket$. Soit $(x_1, \ldots, x_{n_i}) \in Y^{n_i}$\/ et soit $a \in A_i$. Montrons que $f_i(a, x_1, \ldots, x_{n_i}) \in Y$.
		Or, soit $X \in \mathcal{A}$, on a $(x_1, \ldots, x_{n_i}) \in X^{n_i}$\/ donc $f_i(a, x_1, \ldots, x_{n_i}) \in X$. Ceci étant vrai pour tout $X \in \mathcal{A}$, on a $f_i(a, x_1, \ldots, x_{n_i}) \in Y$\/ donc $Y$\/ est stable par $f_i$\/ par tout $i \in \left\llbracket 1,m \right\rrbracket$\/ et donc $Y \in \mathcal{A}$.
		On a également $Y \subseteq X$\/ pour tout $X \in \mathcal{A}$. On en déduit que $Y$\/ est le plus petit élément (pour l'inclusion) de $\mathcal{A}$.
	\item On pose $X_0  = \mathcal{B}$\/ et \[
			X_{n+1} = X_n \cup \big\{ f_i(a, x_1, \ldots, x_{n_i})  \mid a \in A_i,\,(x_1, \ldots, x_{n_i}) \in (X_n)^{n_i},\,i \in \left\llbracket 1,m \right\rrbracket\big\}
		.\]
		Soit $X = \bigcup_{n \in \N} X_n$. Soit $Y$\/ l'ensemble défini par induction à partir de $\mathcal{B}$\/ et des $(f_i)_{i\in\left\llbracket 1,n \right\rrbracket}$. Montrons que $X = Y$.
		On montre que $X$\/ est le plus petit élément (pour l'inclusion) de $\mathcal{A}$\/ et on conclut par unicité du minimum (avec la question précédente).
		Par définition de la suite $(X_n)_{n\in\N}$, elle est croissante (au sens de l'inclusion).
		Montrons à présent, par récurrence, la propriété ci-dessous : $P_n : ``X_n \subseteq Y."$
		\begin{itemize}
			\item Par définition de $Y$, on a $X_0 = \mathcal{B} \subseteq Y$.
			\item Soit 
		\end{itemize}
\end{enumerate}

\subsection{Un théorème d'induction}

\begin{enumerate}
	\item[3.] Soit $Z = \{x \in S  \mid P(x) \text{ vraie}\:\}$.
		Montrons que $\mathcal{X}\subseteq Z$.
		On remarque que $\mathcal{X} \supseteq \mathcal{B}$\/ ; $\mathcal{X}$\/ est stable par $f_i$. On en conclut que $Z \supseteq \mathcal{X}$ et donc $\forall x \in \mathcal{X},\,P(x)$\/ est vraie.
\end{enumerate}

\begin{exm}
	Soit $\mathcal{X}$\/ défini par induction par $\mathcal{B} = \{0, 2\}$\/ et \begin{align*}
		f: \N &\longrightarrow \N \\
		n &\longmapsto n + 2.
	\end{align*}
	Montrons que $\forall n \in \mathcal{X}$, $x$\/ est pair.

	On sait que $0$\/ est pair, $2$\/ est pair ; et, \[
		\forall x,y \in \mathcal{X},\, (x \text{ pair} \land y \text{ pair}) \implies f(x, y) \text{ pair}
	.\]

	On en déduit que \[
		\forall n \in \mathcal{X},\,x \text{ est pair}.
	.\]
\end{exm}
\end{comment}

		\section{Tableaux dynamiques}

\begin{enumerate}[start=3]
	\item On trouve une complexité amortie en $n^2$. À rédiger.
	\item Au lieu de diviser quand $r < n / 2$, mais quand $r < n / 4$.
\end{enumerate}

	}
	\def\addmacros#1{#1}
}
{
	\td[13]{Jeux}
	\minitoc
	\renewcommand{\cwd}{../td/td13/}
	\addmacros{
		\section{Quelques problèmes décidables}

\begin{enumerate}
	\item Soit $f : \R \to \R$.
		\begin{itemize}
			\item Si $f$\/ admet un zéro, on pose $\mathcal{M} = \texttt{fun}\ \texttt{s}\ \to \texttt{true}$.
			\item Si $f$\/ n'admet pas un zéro, on pose $\mathcal{M} = \texttt{fun}\ \texttt{s}\ \to \texttt{false}$.
		\end{itemize}
		Alors, $\mathcal{M}$\/ décide \textsc{Zero}$_f$.
	\item Soit $\mathcal{M}$\/ une machine, et soit $w \in \Sigma^*$.
		\begin{itemize}
			\item Si $\mathcal{M}$\/ se termine sur l'entrée $w$, alors on pose $\mathcal{M}' = \texttt{fun}\ \texttt{s} \to \texttt{true}$.
			\item Si $\mathcal{M}$\/ ne se termine pas sur l'entrée $w$, alors on pose $\mathcal{M}' = \texttt{fun}\ \texttt{s} \to \texttt{false}$.
		\end{itemize}
		Alors, $\mathcal{M}'$\/ décide \textsc{Arrêt}$_{\mathcal{M},w}$.
	\item Le problème est trivialement vrai. En effet, soit $M \in \mathcal{O}$, de la forme
		\begin{lstlisting}[language=caml]
let m (s: string): string =
	%*$\langle$\textrm{code}$\rangle$*) 
		\end{lstlisting}
		On crée la machine $\mathcal{N}$\/ ci-dessous.
		\begin{lstlisting}[language=caml]
let n (s: string): string =
	if true then
		%*$\langle$\textrm{code}$\rangle$*) 
	else
		%*$\langle$\textrm{code}$\rangle$*) 
		\end{lstlisting}
		On a $\texttt{m} \neq \texttt{n}$, mais $\mathcal{L}(\texttt{m}) = \mathcal{L}(\texttt{n})$, donc le problème est vrai sur toute entrée et la fonction $\texttt{fun}\ \texttt{s} \to \texttt{true}$\/ répond au problème.
\end{enumerate}

		\section{Ensembles définis inductivement}

La correction est disponible sur \textit{cahier-de-prepa}.

\begin{comment}
	\begin{exm}
		Avec $S = \N$, $\mathcal{B} = \{0, 2\} $, $A_1 = \{0\}$\/ et \begin{align*}
			f_1: A_1 \times \N &\longrightarrow \N \\
			(0, x) &\longmapsto x + 4.
		\end{align*}

		On a \[
			X \supseteq \{0, 2, 4, 6, 8, 10, \ldots, 20, \ldots\} = 2\N
		.\]
	\end{exm}
	\begin{exm}
		Avec $S$\/ l'ensemble des langages sur $\Sigma$, $\mathcal{B} = \{\O\} \cup \bigl\{\{a\}\:\big|\: a \in \Sigma \bigr\}$, et
		\begin{multicols}{3}
			\begin{align*}
				f_1: S \times S &\longrightarrow S \\
				(L_1, L_2) &\longmapsto L_1 \cup L_2,
			\end{align*}
			\begin{align*}
				f_2: S \times S &\longrightarrow S \\
				(L_1, L_2) &\longmapsto L_1 \cdot L_2,
			\end{align*}
			\begin{align*}
				f_3: S &\longrightarrow S \\
				L &\longmapsto L^*.
			\end{align*}
		\end{multicols}
	\end{exm}

\begin{enumerate}
	\item Soit $\mathcal{A} = \{X \subseteq S  \mid X \supseteq \mathcal{B} \mathrel{\text{et}} X \text{ est stable par } f_i\}$. On a $S \in \mathcal{A}$\/ et donc $\mathcal{A} \neq \O$. De plus, soit \[
			Y = \{x \in S  \mid \forall X \in \mathcal{A},\,x \in X\} = \bigcap_{X \in \mathcal{A}} X
		.\]
		Soit $b \in \mathcal{B}$, on a $\forall X \in A,\, b \in X$. D'où $b \in Y$\/ par intersection. On en déduit que $\mathcal{B} \subseteq Y$.

		Soit $i \in \left\llbracket 1,m \right\rrbracket$. Soit $(x_1, \ldots, x_{n_i}) \in Y^{n_i}$\/ et soit $a \in A_i$. Montrons que $f_i(a, x_1, \ldots, x_{n_i}) \in Y$.
		Or, soit $X \in \mathcal{A}$, on a $(x_1, \ldots, x_{n_i}) \in X^{n_i}$\/ donc $f_i(a, x_1, \ldots, x_{n_i}) \in X$. Ceci étant vrai pour tout $X \in \mathcal{A}$, on a $f_i(a, x_1, \ldots, x_{n_i}) \in Y$\/ donc $Y$\/ est stable par $f_i$\/ par tout $i \in \left\llbracket 1,m \right\rrbracket$\/ et donc $Y \in \mathcal{A}$.
		On a également $Y \subseteq X$\/ pour tout $X \in \mathcal{A}$. On en déduit que $Y$\/ est le plus petit élément (pour l'inclusion) de $\mathcal{A}$.
	\item On pose $X_0  = \mathcal{B}$\/ et \[
			X_{n+1} = X_n \cup \big\{ f_i(a, x_1, \ldots, x_{n_i})  \mid a \in A_i,\,(x_1, \ldots, x_{n_i}) \in (X_n)^{n_i},\,i \in \left\llbracket 1,m \right\rrbracket\big\}
		.\]
		Soit $X = \bigcup_{n \in \N} X_n$. Soit $Y$\/ l'ensemble défini par induction à partir de $\mathcal{B}$\/ et des $(f_i)_{i\in\left\llbracket 1,n \right\rrbracket}$. Montrons que $X = Y$.
		On montre que $X$\/ est le plus petit élément (pour l'inclusion) de $\mathcal{A}$\/ et on conclut par unicité du minimum (avec la question précédente).
		Par définition de la suite $(X_n)_{n\in\N}$, elle est croissante (au sens de l'inclusion).
		Montrons à présent, par récurrence, la propriété ci-dessous : $P_n : ``X_n \subseteq Y."$
		\begin{itemize}
			\item Par définition de $Y$, on a $X_0 = \mathcal{B} \subseteq Y$.
			\item Soit 
		\end{itemize}
\end{enumerate}

\subsection{Un théorème d'induction}

\begin{enumerate}
	\item[3.] Soit $Z = \{x \in S  \mid P(x) \text{ vraie}\:\}$.
		Montrons que $\mathcal{X}\subseteq Z$.
		On remarque que $\mathcal{X} \supseteq \mathcal{B}$\/ ; $\mathcal{X}$\/ est stable par $f_i$. On en conclut que $Z \supseteq \mathcal{X}$ et donc $\forall x \in \mathcal{X},\,P(x)$\/ est vraie.
\end{enumerate}

\begin{exm}
	Soit $\mathcal{X}$\/ défini par induction par $\mathcal{B} = \{0, 2\}$\/ et \begin{align*}
		f: \N &\longrightarrow \N \\
		n &\longmapsto n + 2.
	\end{align*}
	Montrons que $\forall n \in \mathcal{X}$, $x$\/ est pair.

	On sait que $0$\/ est pair, $2$\/ est pair ; et, \[
		\forall x,y \in \mathcal{X},\, (x \text{ pair} \land y \text{ pair}) \implies f(x, y) \text{ pair}
	.\]

	On en déduit que \[
		\forall n \in \mathcal{X},\,x \text{ est pair}.
	.\]
\end{exm}
\end{comment}

		\section{Tableaux dynamiques}

\begin{enumerate}[start=3]
	\item On trouve une complexité amortie en $n^2$. À rédiger.
	\item Au lieu de diviser quand $r < n / 2$, mais quand $r < n / 4$.
\end{enumerate}

		\section{Barres en triangle}


On note $a(x)$ le côté du triangule équilatéral, et donc $x = a(x) \sqrt{3}  / 2$.

On calcule le flux $\Phi$ magnétique : \[
	\Phi = B \frac{a x}{2} = B x^2 / \sqrt{3}
.\]
Ainsi, d'après la loi de \textsc{Faraday}, on a \[
	e = - \frac{\mathrm{d}\Phi}{\mathrm{d}t} = - \frac{2B}{\sqrt{3}}  \: x\,\dot{x}
.\]
Or, par loi d'\textsc{Ohm}, $i = e / R(x)$.
Et, on connait la resistance du circuit $R(x) = 3 a(x) / \gamma S$.
Alors, 
\begin{align*}
	i(x) &= \frac{-2B\,x\,\dot{x}}{\sqrt{3} \cdot \frac{3a(x)}{\gamma S}} = - \frac{2B \gamma S}{3 \sqrt{3}} \cdot \frac{x\,\dot{x}}{2 \frac{x}{\sqrt{3}}}\\
	&= - B \gamma S \dot{x} / 3. \\
\end{align*}
On calcule donc la force de \textsc{Laplace} :
\begin{align*}
	\vec{F}_{\mathcal{L}} &= i \cdot [\mathrm{CD}] \cdot B \cdot \vec{e}_x\\
	&= i \cdot \frac{2\dot{x}}{\sqrt{3}} B \vec{e}_x \\
	&= -\underbrace{\frac{2 B^2 \gamma S}{3\sqrt{3}}}_\alpha  \cdot x \dot{x} \\
\end{align*}

D'après le \textsc{pfd}, on a donc \[
	m \ddot{x} = - \alpha x \dot{x} \text{ d'où }  \ddot{x} = - \frac{\alpha}{m} x\dot{x} 
.\] où $m = \rho S L$. 

On intègre les deux côtés de l'équation, \[
	[\dot{x}]_0^{t_\mathrm{f}} = -\frac{\alpha}{m} \cdot \left[ \frac{x^2}{2} \right]_0^{t_\mathrm{f}}
.\]
D'où, \[
	0 - v_0 = \frac{-\alpha}{m} \cdot x_\mathrm{f}^2 / 2
.\] On en conclut \[
	x_\mathrm{f} = \sqrt{\frac{2m}{\alpha} v_0} 
.\] 


		\begin{multicols}{2}
	\section{Implications}
	\begin{enumerate}
		\item
			\[
				\begin{prooftree}
					\infer 0[Ax]{p,q \vdash q}
					\infer 1[$\to$i]{q \vdash p \to q}
				\end{prooftree}
			\]
		\item
			\[
				\begin{prooftree}
					\infer 0[Ax]{p, p\land q\vdash p\land q}
					\infer 1[$\land$e,d]{p, p \land q \vdash q}
					\infer 1[$\to$i]{p \land q \vdash p \to q}
				\end{prooftree}
			\]
		\item 
			\[
				\begin{prooftree}
					\infer 0[Ax]{p, p\to q \vdash p}
					\infer 0[Ax]{p, p\to q \vdash p \to q}
					\infer 0[Ax]{p, p\to q \vdash p}
					\infer 2[$\to$e]{p, p \to q \vdash q}
					\infer 2[$\land$i]{p, p \to q \vdash p \land q}
					\infer 1[$\to$i]{p \to q \vdash p\to (p \land q)}
				\end{prooftree}
			\]
		\item
			\[
				\begin{prooftree}
					\infer 0[Ax]{\lnot q, p\to q, p \vdash p \to q}
					\infer 0[Ax]{\lnot q, p\to q, p \vdash p}
					\infer 2[$\to$e]{\lnot q, p\to q, p \vdash q}
					\infer 0[Ax]{\lnot q, p\to q, p \vdash \lnot q}
					\infer 2[$\lnot$e]{\lnot q, p\to q, p \vdash \bot}
					\infer 1[$\lnot$i]{\lnot q, p\to q \vdash\lnot p}
					\infer 1[$\to$i]{p \to q \vdash \lnot q \to \lnot p}
				\end{prooftree}
			\]
		\item 
			\[
				\begin{prooftree}
					\infer 0[Ax]{p \land q, p\to r \vdash p \land q}
					\infer 1[$\land$e,g]{p \land q, p\to r \vdash p}
					\infer 0[Ax]{p \land q, p\to r \vdash p \to r}
					\infer 2[$\to$e]{p \land q, p \to r \vdash r}
					\infer 1[$\to$i]{p \to r \vdash (p \land q) \to r}
				\end{prooftree}
			\] 
		\item
			\[
				\begin{prooftree}
					\infer 0[Ax]{p,q\vdash p}
					\infer 1[$\to$i]{p \vdash q \to p}
					\infer 1[$\to$i]{\vdash p \to (q \to p)}
				\end{prooftree}
			\]
		\item
			\[
				\begin{prooftree}
					\infer 0[Ax]{p \to q, p \vdash p\to q}
					\infer 0[Ax]{p\to q,p \vdash p}
					\infer 2[$\to$e]{p, p\to q \vdash q}
					\infer 1[$\to$i]{p \vdash (p \to q) \to q}
				\end{prooftree}
			\] 
	\end{enumerate}
\end{multicols}

	}
	\def\addmacros#1{#1}
}
{
	\td[14]{Grammaires non contextuelles (1)}
	\minitoc
	\renewcommand{\cwd}{../td/td14/}
	\addmacros{
		\section{Quelques problèmes décidables}

\begin{enumerate}
	\item Soit $f : \R \to \R$.
		\begin{itemize}
			\item Si $f$\/ admet un zéro, on pose $\mathcal{M} = \texttt{fun}\ \texttt{s}\ \to \texttt{true}$.
			\item Si $f$\/ n'admet pas un zéro, on pose $\mathcal{M} = \texttt{fun}\ \texttt{s}\ \to \texttt{false}$.
		\end{itemize}
		Alors, $\mathcal{M}$\/ décide \textsc{Zero}$_f$.
	\item Soit $\mathcal{M}$\/ une machine, et soit $w \in \Sigma^*$.
		\begin{itemize}
			\item Si $\mathcal{M}$\/ se termine sur l'entrée $w$, alors on pose $\mathcal{M}' = \texttt{fun}\ \texttt{s} \to \texttt{true}$.
			\item Si $\mathcal{M}$\/ ne se termine pas sur l'entrée $w$, alors on pose $\mathcal{M}' = \texttt{fun}\ \texttt{s} \to \texttt{false}$.
		\end{itemize}
		Alors, $\mathcal{M}'$\/ décide \textsc{Arrêt}$_{\mathcal{M},w}$.
	\item Le problème est trivialement vrai. En effet, soit $M \in \mathcal{O}$, de la forme
		\begin{lstlisting}[language=caml]
let m (s: string): string =
	%*$\langle$\textrm{code}$\rangle$*) 
		\end{lstlisting}
		On crée la machine $\mathcal{N}$\/ ci-dessous.
		\begin{lstlisting}[language=caml]
let n (s: string): string =
	if true then
		%*$\langle$\textrm{code}$\rangle$*) 
	else
		%*$\langle$\textrm{code}$\rangle$*) 
		\end{lstlisting}
		On a $\texttt{m} \neq \texttt{n}$, mais $\mathcal{L}(\texttt{m}) = \mathcal{L}(\texttt{n})$, donc le problème est vrai sur toute entrée et la fonction $\texttt{fun}\ \texttt{s} \to \texttt{true}$\/ répond au problème.
\end{enumerate}

		\section{Ensembles définis inductivement}

La correction est disponible sur \textit{cahier-de-prepa}.

\begin{comment}
	\begin{exm}
		Avec $S = \N$, $\mathcal{B} = \{0, 2\} $, $A_1 = \{0\}$\/ et \begin{align*}
			f_1: A_1 \times \N &\longrightarrow \N \\
			(0, x) &\longmapsto x + 4.
		\end{align*}

		On a \[
			X \supseteq \{0, 2, 4, 6, 8, 10, \ldots, 20, \ldots\} = 2\N
		.\]
	\end{exm}
	\begin{exm}
		Avec $S$\/ l'ensemble des langages sur $\Sigma$, $\mathcal{B} = \{\O\} \cup \bigl\{\{a\}\:\big|\: a \in \Sigma \bigr\}$, et
		\begin{multicols}{3}
			\begin{align*}
				f_1: S \times S &\longrightarrow S \\
				(L_1, L_2) &\longmapsto L_1 \cup L_2,
			\end{align*}
			\begin{align*}
				f_2: S \times S &\longrightarrow S \\
				(L_1, L_2) &\longmapsto L_1 \cdot L_2,
			\end{align*}
			\begin{align*}
				f_3: S &\longrightarrow S \\
				L &\longmapsto L^*.
			\end{align*}
		\end{multicols}
	\end{exm}

\begin{enumerate}
	\item Soit $\mathcal{A} = \{X \subseteq S  \mid X \supseteq \mathcal{B} \mathrel{\text{et}} X \text{ est stable par } f_i\}$. On a $S \in \mathcal{A}$\/ et donc $\mathcal{A} \neq \O$. De plus, soit \[
			Y = \{x \in S  \mid \forall X \in \mathcal{A},\,x \in X\} = \bigcap_{X \in \mathcal{A}} X
		.\]
		Soit $b \in \mathcal{B}$, on a $\forall X \in A,\, b \in X$. D'où $b \in Y$\/ par intersection. On en déduit que $\mathcal{B} \subseteq Y$.

		Soit $i \in \left\llbracket 1,m \right\rrbracket$. Soit $(x_1, \ldots, x_{n_i}) \in Y^{n_i}$\/ et soit $a \in A_i$. Montrons que $f_i(a, x_1, \ldots, x_{n_i}) \in Y$.
		Or, soit $X \in \mathcal{A}$, on a $(x_1, \ldots, x_{n_i}) \in X^{n_i}$\/ donc $f_i(a, x_1, \ldots, x_{n_i}) \in X$. Ceci étant vrai pour tout $X \in \mathcal{A}$, on a $f_i(a, x_1, \ldots, x_{n_i}) \in Y$\/ donc $Y$\/ est stable par $f_i$\/ par tout $i \in \left\llbracket 1,m \right\rrbracket$\/ et donc $Y \in \mathcal{A}$.
		On a également $Y \subseteq X$\/ pour tout $X \in \mathcal{A}$. On en déduit que $Y$\/ est le plus petit élément (pour l'inclusion) de $\mathcal{A}$.
	\item On pose $X_0  = \mathcal{B}$\/ et \[
			X_{n+1} = X_n \cup \big\{ f_i(a, x_1, \ldots, x_{n_i})  \mid a \in A_i,\,(x_1, \ldots, x_{n_i}) \in (X_n)^{n_i},\,i \in \left\llbracket 1,m \right\rrbracket\big\}
		.\]
		Soit $X = \bigcup_{n \in \N} X_n$. Soit $Y$\/ l'ensemble défini par induction à partir de $\mathcal{B}$\/ et des $(f_i)_{i\in\left\llbracket 1,n \right\rrbracket}$. Montrons que $X = Y$.
		On montre que $X$\/ est le plus petit élément (pour l'inclusion) de $\mathcal{A}$\/ et on conclut par unicité du minimum (avec la question précédente).
		Par définition de la suite $(X_n)_{n\in\N}$, elle est croissante (au sens de l'inclusion).
		Montrons à présent, par récurrence, la propriété ci-dessous : $P_n : ``X_n \subseteq Y."$
		\begin{itemize}
			\item Par définition de $Y$, on a $X_0 = \mathcal{B} \subseteq Y$.
			\item Soit 
		\end{itemize}
\end{enumerate}

\subsection{Un théorème d'induction}

\begin{enumerate}
	\item[3.] Soit $Z = \{x \in S  \mid P(x) \text{ vraie}\:\}$.
		Montrons que $\mathcal{X}\subseteq Z$.
		On remarque que $\mathcal{X} \supseteq \mathcal{B}$\/ ; $\mathcal{X}$\/ est stable par $f_i$. On en conclut que $Z \supseteq \mathcal{X}$ et donc $\forall x \in \mathcal{X},\,P(x)$\/ est vraie.
\end{enumerate}

\begin{exm}
	Soit $\mathcal{X}$\/ défini par induction par $\mathcal{B} = \{0, 2\}$\/ et \begin{align*}
		f: \N &\longrightarrow \N \\
		n &\longmapsto n + 2.
	\end{align*}
	Montrons que $\forall n \in \mathcal{X}$, $x$\/ est pair.

	On sait que $0$\/ est pair, $2$\/ est pair ; et, \[
		\forall x,y \in \mathcal{X},\, (x \text{ pair} \land y \text{ pair}) \implies f(x, y) \text{ pair}
	.\]

	On en déduit que \[
		\forall n \in \mathcal{X},\,x \text{ est pair}.
	.\]
\end{exm}
\end{comment}

		\section{Tableaux dynamiques}

\begin{enumerate}[start=3]
	\item On trouve une complexité amortie en $n^2$. À rédiger.
	\item Au lieu de diviser quand $r < n / 2$, mais quand $r < n / 4$.
\end{enumerate}

		\section{Barres en triangle}


On note $a(x)$ le côté du triangule équilatéral, et donc $x = a(x) \sqrt{3}  / 2$.

On calcule le flux $\Phi$ magnétique : \[
	\Phi = B \frac{a x}{2} = B x^2 / \sqrt{3}
.\]
Ainsi, d'après la loi de \textsc{Faraday}, on a \[
	e = - \frac{\mathrm{d}\Phi}{\mathrm{d}t} = - \frac{2B}{\sqrt{3}}  \: x\,\dot{x}
.\]
Or, par loi d'\textsc{Ohm}, $i = e / R(x)$.
Et, on connait la resistance du circuit $R(x) = 3 a(x) / \gamma S$.
Alors, 
\begin{align*}
	i(x) &= \frac{-2B\,x\,\dot{x}}{\sqrt{3} \cdot \frac{3a(x)}{\gamma S}} = - \frac{2B \gamma S}{3 \sqrt{3}} \cdot \frac{x\,\dot{x}}{2 \frac{x}{\sqrt{3}}}\\
	&= - B \gamma S \dot{x} / 3. \\
\end{align*}
On calcule donc la force de \textsc{Laplace} :
\begin{align*}
	\vec{F}_{\mathcal{L}} &= i \cdot [\mathrm{CD}] \cdot B \cdot \vec{e}_x\\
	&= i \cdot \frac{2\dot{x}}{\sqrt{3}} B \vec{e}_x \\
	&= -\underbrace{\frac{2 B^2 \gamma S}{3\sqrt{3}}}_\alpha  \cdot x \dot{x} \\
\end{align*}

D'après le \textsc{pfd}, on a donc \[
	m \ddot{x} = - \alpha x \dot{x} \text{ d'où }  \ddot{x} = - \frac{\alpha}{m} x\dot{x} 
.\] où $m = \rho S L$. 

On intègre les deux côtés de l'équation, \[
	[\dot{x}]_0^{t_\mathrm{f}} = -\frac{\alpha}{m} \cdot \left[ \frac{x^2}{2} \right]_0^{t_\mathrm{f}}
.\]
D'où, \[
	0 - v_0 = \frac{-\alpha}{m} \cdot x_\mathrm{f}^2 / 2
.\] On en conclut \[
	x_\mathrm{f} = \sqrt{\frac{2m}{\alpha} v_0} 
.\] 


		\begin{multicols}{2}
	\section{Implications}
	\begin{enumerate}
		\item
			\[
				\begin{prooftree}
					\infer 0[Ax]{p,q \vdash q}
					\infer 1[$\to$i]{q \vdash p \to q}
				\end{prooftree}
			\]
		\item
			\[
				\begin{prooftree}
					\infer 0[Ax]{p, p\land q\vdash p\land q}
					\infer 1[$\land$e,d]{p, p \land q \vdash q}
					\infer 1[$\to$i]{p \land q \vdash p \to q}
				\end{prooftree}
			\]
		\item 
			\[
				\begin{prooftree}
					\infer 0[Ax]{p, p\to q \vdash p}
					\infer 0[Ax]{p, p\to q \vdash p \to q}
					\infer 0[Ax]{p, p\to q \vdash p}
					\infer 2[$\to$e]{p, p \to q \vdash q}
					\infer 2[$\land$i]{p, p \to q \vdash p \land q}
					\infer 1[$\to$i]{p \to q \vdash p\to (p \land q)}
				\end{prooftree}
			\]
		\item
			\[
				\begin{prooftree}
					\infer 0[Ax]{\lnot q, p\to q, p \vdash p \to q}
					\infer 0[Ax]{\lnot q, p\to q, p \vdash p}
					\infer 2[$\to$e]{\lnot q, p\to q, p \vdash q}
					\infer 0[Ax]{\lnot q, p\to q, p \vdash \lnot q}
					\infer 2[$\lnot$e]{\lnot q, p\to q, p \vdash \bot}
					\infer 1[$\lnot$i]{\lnot q, p\to q \vdash\lnot p}
					\infer 1[$\to$i]{p \to q \vdash \lnot q \to \lnot p}
				\end{prooftree}
			\]
		\item 
			\[
				\begin{prooftree}
					\infer 0[Ax]{p \land q, p\to r \vdash p \land q}
					\infer 1[$\land$e,g]{p \land q, p\to r \vdash p}
					\infer 0[Ax]{p \land q, p\to r \vdash p \to r}
					\infer 2[$\to$e]{p \land q, p \to r \vdash r}
					\infer 1[$\to$i]{p \to r \vdash (p \land q) \to r}
				\end{prooftree}
			\] 
		\item
			\[
				\begin{prooftree}
					\infer 0[Ax]{p,q\vdash p}
					\infer 1[$\to$i]{p \vdash q \to p}
					\infer 1[$\to$i]{\vdash p \to (q \to p)}
				\end{prooftree}
			\]
		\item
			\[
				\begin{prooftree}
					\infer 0[Ax]{p \to q, p \vdash p\to q}
					\infer 0[Ax]{p\to q,p \vdash p}
					\infer 2[$\to$e]{p, p\to q \vdash q}
					\infer 1[$\to$i]{p \vdash (p \to q) \to q}
				\end{prooftree}
			\] 
	\end{enumerate}
\end{multicols}

		\section{Langage de \textsc{Dyck}}

\begin{enumerate}
	\item On suppose ce langage reconnaissable par un automate à $n$ états. On considère le mot $w = {\red(}^n \cdot {\red)}^n$, donc $|w| \ge n$.
		Ainsi, il existe $x$, $y$ et $z$ trois mots tels que $w = xyz$, $|xy| \le n$, $y \neq \varepsilon$ et $\forall p \in \N,\: x y^p z \in \mathcal{L}(\mathcal{G})$.
		Soit alors $p \in \llbracket 1,n-1 \rrbracket$ et $q \in \llbracket 1,n -p \rrbracket$ tels que $x = {\red(}^p$, $y = {\red(}^q$ et $z = {\red(}^{n-q-p} \cdot {\red)}^n$.
		Ainsi, $xy \in \mathcal{L}(\mathcal{G})$, ce qui est absurde. On en déduit que $\mathcal{L}(\mathcal{G})$ n'est pas reconnaissable, il n'est donc pas régulier.
	\item On pose $\mathcal{G} = (\Sigma, \{\mathrm{S}\}, \{\mathrm{S} \to \red( \mathrm{S}\red)  \mid \mathrm{SS}  \mid \varepsilon\}, \mathrm{S})$.
	\item
		\begin{itemize}
			\item On le montre par induction.
				\begin{itemize}
					\item \textbf{Cas $\mathrm{S} \to \varepsilon$.}
						On a $|\varepsilon|_{\red(} = 0 = |\varepsilon|_{\red)}$.
					\item \textbf{Cas $\mathrm{S} \to \red( \mathrm{S}\red)$.}
						Soit $u \in \mathcal{L}(\mathcal{G})$ avec $|u|_{\red(} = |u|_{\red)} = n$. Ainsi, $|\red(u\red)|_{\red(} = |\red(u\red)|_{\red)} = n + 1$.
					\item \textbf{Cas $\mathrm{S}\to \mathrm{SS}$.}
						Soient $u$ et $v$ deux mots de $\mathcal{L}(\mathcal{G})$ tels que $|u|_{\red(} = |u|_{\red)} = n$ et $|v|_{\red(} = |v|_{\red)} = m$.
						Alors, $|u\cdot v|_{\red(} = |v\cdot u|_{\red)} = n + m$.
				\end{itemize}
			\item Montrons par induction $\mathcal{P}_{u}$ : \guillemotleft~pour tout $v$ préfixe de $u$, $|v|_{\red(} \ge |v|_{\red)}$.~\guillemotright\@ 
				\begin{itemize}
					\item \textbf{Cas $\mathrm{S} \to \varepsilon$.}
						Le seul préfixe de $\varepsilon$ est $\varepsilon$, et on a bien $|\varepsilon|_{\red(} = 0 \ge 0 = |\varepsilon|_{\red)}$.
					\item \textbf{Cas $\mathrm{S}\to \red( \mathrm{S} \red)$.}
						Soit $u$ un mot de $\mathcal{L}(\mathcal{G})$ vérifiant $\mathcal{P}_u$.
						Soit $v$ un préfixe de $\red(u\red)$.
						On procède par induction sur $v$.
						\begin{itemize}
							\item \textbf{Cas $v = \varepsilon$ ou $\red($.} \textsc{ok}.
							\item \textbf{Cas $\red(\tilde{u}$,} où $\tilde{u}$ est un préfixe de $u$.
								Par hypothèse d'induction, $|\tilde{u}|_{\red(} \ge |\tilde{u}|_{\red)}$ donc $|\red(\tilde{u}|_{\red(} = |\red(\tilde{u}|_{\red)}$.
							\item \textbf{Cas $\red(u\red)$.} Par hypothèse d'induction, $|u|_{\red(} \ge |u|_{\red)}$ donc $|\red(u\red)|_{\red(}\ge |\red(u\red)|_{\red)}$.
						\end{itemize}
				\end{itemize}
		\end{itemize}
	\item On note $\overline w^j= \big| w_{\llbracket 0,j \rrbracket} \big|_{\red(} - \big| w_{\llbracket 0,j \rrbracket} \big|_{\red)}$.
		Alors les deux conditions se traduisent par $\overline w^{|w|} = 0$ et $\forall i \in \llbracket 0,|w|-1 \rrbracket$, $\overline w^i \ge 0$.
\end{enumerate}

		\section{$\mathcal{N}$}

\newcommand{\ofact}{\charfusion[\mathbin]{\bigcirc}{\scriptstyle!}}

\begin{enumerate}
	\item On définit par induction la fonction suivante \begin{align*}
			\oplus: \mathcal{N}^2 &\longrightarrow \mathcal{N} \\
			(\mathbf{S}(x),y) &\longmapsto \oplus(x, \mathbf{S}(y))\\
			(\mathbf{0}, x) &\longmapsto x.
		\end{align*}
	\item Soit $(x,y) \in \mathcal{N}^2$.
		\begin{itemize}
			\item Si $f(x) = 0$, alors $\oplus(x,y) = y$\/ et donc $f(\oplus(x,y)) = f(y) = f(x) + f(y)$.
			\item Si $f(x) \ge 1$, alors  $x = \mathbf{S}(z)$\/ avec $z \in \mathcal{N}$. Ainsi, $\oplus(x,y) = \oplus(z, \mathbf{S}(y))$. Or, $f(z) = f(x) - 1 \le f(x)$. Et donc, par définition de $\oplus$\/ puis par hypothèse d'induction, on a $f(\oplus(x,y)) = f(\oplus(z, \mathbf{S}(y)) = f(z) + f(\mathbf{S}(y))$. On en déduit que $f(\oplus(x,y)) = f(x) - 1 + f(y) + 1 = f(x) + f(y)$.
		\end{itemize}
		Par induction, on a bien $\forall (x,y) \in \mathcal{N}^2,\:f\big({\oplus}(x,y)\big) = f(x) + f(y)$.
	\item On définit par induction la fonction suivante \begin{align*}
			\otimes: \mathcal{N}^2 &\longrightarrow \mathcal{N} \\
			(\mathbf{S}(x), y) &\longmapsto {\oplus}\big(y, {\otimes}(x,y)\big)\\
			(\mathbf{0}, y) &\longmapsto \mathbf{0}.
		\end{align*}
	\item Soit $(x,y) \in \mathcal{N}^2$.
		\begin{itemize}
			\item Si $f(x) = 0$, alors $\otimes(x,y) = \mathbf{0}$, et donc $f(\otimes(x,y)) = 0 = f(x) \times f(y)$.
			\item Si $f(x) \ge 1$, alors $x = \mathbf{S}(z)$\/ avec $z \in \mathcal{N}$. Ainsi, par définition de $\otimes$, on a $\otimes(x,y) = \oplus(y, \otimes(z,y))$. Or, par hypothèse d'induction, $f(\otimes(z,y)) = f(z) \times f(y)$\/ (car $f(z) < f(x)$), et donc $f(\otimes(x,y)) = f(y) + f(\otimes(z,y)) = f(y) + f(z) \times f(y) = f(y) \times (1 + f(z)) = f(y) \times f(x)$.
		\end{itemize}
		Par induction, on a bien $\forall (x,y) \in \mathcal{N}^2,\:f\big({\otimes}(x,y)\big) = f(x) \times f(y)$.
	\item On définit par induction la fonction suivante \begin{align*}
			\ofact : \mathcal{N} &\longrightarrow \mathcal{N} \\
			\mathbf{0} &\longmapsto \mathbf{S}(\mathbf{0})\\
			\mathbf{S}(x) &\longmapsto {\otimes}\big(\mathbf{S}(x), \ofact(x)\big).
		\end{align*}
	\item Soit $x \in \mathcal{N}$.
		\begin{itemize}
			\item Si $f(x)= 0$, alors $\ofact(x) = \mathbf{S}(\mathbf{0})$\/ par définition, et donc $f(\ofact(x)) = 1 = 0! = f(x) !\:$.
			\item Si $f(x) \ge 1$, alors $x = \mathbf{S}(z)$\/ avec $z \in \mathcal{N}$. Ainsi, par définition de $\ofact$, on a $\ofact(x) = \otimes(x, \ofact(z))$, et donc, par hypothèse de récurrence, $f(\ofact(x)) = f(x) \times f(\ofact(z)) = f(x) \times \big(f(z)!\big)$. Or, comme $f(z) = f(x) - 1$, on a donc $f(\ofact(x)) = f(x) \times \big(f(x) - 1\big)! = f(x)!$\:.
		\end{itemize}
		Par induction, on a bien $\forall x \in \mathcal{N},\:f\big({\ofact}(x)\big) = f(x)!\:$.
\end{enumerate}

		\documentclass[a4paper]{article}

\usepackage[margin=1in]{geometry}
\usepackage[utf8]{inputenc}
\usepackage[T1]{fontenc}
\usepackage{mathrsfs}
\usepackage{textcomp}
\usepackage[french]{babel}
\usepackage{amsmath}
\usepackage{amssymb}
\usepackage{cancel}
\usepackage{frcursive}
\usepackage[inline]{asymptote}
\usepackage{tikz}
\usepackage[european,straightvoltages,europeanresistors]{circuitikz}
\usepackage{tikz-cd}
\usepackage{tkz-tab}
\usepackage[b]{esvect}
\usepackage[framemethod=TikZ]{mdframed}
\usepackage{centernot}
\usepackage{diagbox}
\usepackage{dsfont}
\usepackage{fancyhdr}
\usepackage{float}
\usepackage{graphicx}
\usepackage{listings}
\usepackage{multicol}
\usepackage{nicematrix}
\usepackage{pdflscape}
\usepackage{stmaryrd}
\usepackage{xfrac}
\usepackage{hep-math-font}
\usepackage{amsthm}
\usepackage{thmtools}
\usepackage{indentfirst}
\usepackage[framemethod=TikZ]{mdframed}
\usepackage{accents}
\usepackage{soulutf8}
\usepackage{mathtools}
\usepackage{bodegraph}
\usepackage{slashbox}
\usepackage{enumitem}
\usepackage{calligra}
\usepackage{cinzel}
\usepackage{BOONDOX-calo}

% Tikz
\usetikzlibrary{babel}
\usetikzlibrary{positioning}
\usetikzlibrary{calc}

% global settings
\frenchspacing
\reversemarginpar
\setuldepth{a}

%\everymath{\displaystyle}

\frenchbsetup{StandardLists=true}

\def\asydir{asy}

%\sisetup{exponent-product=\cdot,output-decimal-marker={,},separate-uncertainty,range-phrase=\;à\;,locale=FR}

\setlength{\parskip}{1em}

\theoremstyle{definition}

% Changing math
\let\emptyset\varnothing
\let\ge\geqslant
\let\le\leqslant
\let\preceq\preccurlyeq
\let\succeq\succcurlyeq
\let\ds\displaystyle
\let\ts\textstyle

\newcommand{\C}{\mathds{C}}
\newcommand{\R}{\mathds{R}}
\newcommand{\Z}{\mathds{Z}}
\newcommand{\N}{\mathds{N}}
\newcommand{\Q}{\mathds{Q}}

\renewcommand{\O}{\emptyset}

\newcommand\ubar[1]{\underaccent{\bar}{#1}}

\renewcommand\Re{\expandafter\mathfrak{Re}}
\renewcommand\Im{\expandafter\mathfrak{Im}}

\let\slantedpartial\partial
\DeclareRobustCommand{\partial}{\text{\rotatebox[origin=t]{20}{\scalebox{0.95}[1]{$\slantedpartial$}}}\hspace{-1pt}}

% merging two maths characters w/ \charfusion
\makeatletter
\def\moverlay{\mathpalette\mov@rlay}
\def\mov@rlay#1#2{\leavevmode\vtop{%
   \baselineskip\z@skip \lineskiplimit-\maxdimen
   \ialign{\hfil$\m@th#1##$\hfil\cr#2\crcr}}}
\newcommand{\charfusion}[3][\mathord]{
    #1{\ifx#1\mathop\vphantom{#2}\fi
        \mathpalette\mov@rlay{#2\cr#3}
      }
    \ifx#1\mathop\expandafter\displaylimits\fi}
\makeatother

% custom math commands
\newcommand{\T}{{\!\!\,\top}}
\newcommand{\avrt}[1]{\rotatebox{-90}{$#1$}}
\newcommand{\bigcupdot}{\charfusion[\mathop]{\bigcup}{\cdot}}
\newcommand{\cupdot}{\charfusion[\mathbin]{\cup}{\cdot}}
%\newcommand{\danger}{{\large\fontencoding{U}\fontfamily{futs}\selectfont\char 66\relax}\;}
\newcommand{\tendsto}[1]{\xrightarrow[#1]{}}
\newcommand{\vrt}[1]{\rotatebox{90}{$#1$}}
\newcommand{\tsup}[1]{\textsuperscript{\underline{#1}}}
\newcommand{\tsub}[1]{\textsubscript{#1}}

\renewcommand{\mod}[1]{~\left[ #1 \right]}
\renewcommand{\t}{{}^t\!}
\newcommand{\s}{\text{\calligra s}}

% custom units / constants
%\DeclareSIUnit{\litre}{\ell}
\let\hbar\hslash

% header / footer
\pagestyle{fancy}
\fancyhead{} \fancyfoot{}
\fancyfoot[C]{\thepage}

% fonts
\let\sc\scshape
\let\bf\bfseries
\let\it\itshape
\let\sl\slshape

% custom math operators
\let\th\relax
\let\det\relax
\DeclareMathOperator*{\codim}{codim}
\DeclareMathOperator*{\dom}{dom}
\DeclareMathOperator*{\gO}{O}
\DeclareMathOperator*{\po}{\text{\cursive o}}
\DeclareMathOperator*{\sgn}{sgn}
\DeclareMathOperator*{\simi}{\sim}
\DeclareMathOperator{\Arccos}{Arccos}
\DeclareMathOperator{\Arcsin}{Arcsin}
\DeclareMathOperator{\Arctan}{Arctan}
\DeclareMathOperator{\Argsh}{Argsh}
\DeclareMathOperator{\Arg}{Arg}
\DeclareMathOperator{\Aut}{Aut}
\DeclareMathOperator{\Card}{Card}
\DeclareMathOperator{\Cl}{\mathcal{C}\!\ell}
\DeclareMathOperator{\Cov}{Cov}
\DeclareMathOperator{\Ker}{Ker}
\DeclareMathOperator{\Mat}{Mat}
\DeclareMathOperator{\PGCD}{PGCD}
\DeclareMathOperator{\PPCM}{PPCM}
\DeclareMathOperator{\Supp}{Supp}
\DeclareMathOperator{\Vect}{Vect}
\DeclareMathOperator{\argmax}{argmax}
\DeclareMathOperator{\argmin}{argmin}
\DeclareMathOperator{\ch}{ch}
\DeclareMathOperator{\com}{com}
\DeclareMathOperator{\cotan}{cotan}
\DeclareMathOperator{\det}{det}
\DeclareMathOperator{\id}{id}
\DeclareMathOperator{\rg}{rg}
\DeclareMathOperator{\rk}{rk}
\DeclareMathOperator{\sh}{sh}
\DeclareMathOperator{\th}{th}
\DeclareMathOperator{\tr}{tr}

% colors and page style
\definecolor{truewhite}{HTML}{ffffff}
\definecolor{white}{HTML}{faf4ed}
\definecolor{trueblack}{HTML}{000000}
\definecolor{black}{HTML}{575279}
\definecolor{mauve}{HTML}{907aa9}
\definecolor{blue}{HTML}{286983}
\definecolor{red}{HTML}{d7827e}
\definecolor{yellow}{HTML}{ea9d34}
\definecolor{gray}{HTML}{9893a5}
\definecolor{grey}{HTML}{9893a5}
\definecolor{green}{HTML}{a0d971}

\pagecolor{white}
\color{black}

\begin{asydef}
	settings.prc = false;
	settings.render=0;

	white = rgb("faf4ed");
	black = rgb("575279");
	blue = rgb("286983");
	red = rgb("d7827e");
	yellow = rgb("f6c177");
	orange = rgb("ea9d34");
	gray = rgb("9893a5");
	grey = rgb("9893a5");
	deepcyan = rgb("56949f");
	pink = rgb("b4637a");
	magenta = rgb("eb6f92");
	green = rgb("a0d971");
	purple = rgb("907aa9");

	defaultpen(black + fontsize(8pt));

	import three;
	currentlight = nolight;
\end{asydef}

% theorems, proofs, ...

\mdfsetup{skipabove=1em,skipbelow=1em, innertopmargin=6pt, innerbottommargin=6pt,}

\declaretheoremstyle[
	headfont=\normalfont\itshape,
	numbered=no,
	postheadspace=\newline,
	headpunct={:},
	qed=\qedsymbol]{demstyle}

\declaretheorem[style=demstyle, name=Démonstration]{dem}

\newcommand\veczero{\kern-1.2pt\vec{\kern1.2pt 0}} % \vec{0} looks weird since the `0' isn't italicized

\makeatletter
\renewcommand{\title}[2]{
	\AtBeginDocument{
		\begin{titlepage}
			\begin{center}
				\vspace{10cm}
				{\Large \sc Chapitre #1}\\
				\vspace{1cm}
				{\Huge \calligra #2}\\
				\vfill
				Hugo {\sc Salou} MPI${}^{\star}$\\
				{\small Dernière mise à jour le \@date }
			\end{center}
		\end{titlepage}
	}
}

\newcommand{\titletp}[4]{
	\AtBeginDocument{
		\begin{titlepage}
			\begin{center}
				\vspace{10cm}
				{\Large \sc tp #1}\\
				\vspace{1cm}
				{\Huge \textsc{\textit{#2}}}\\
				\vfill
				{#3}\textit{MPI}${}^{\star}$\\
			\end{center}
		\end{titlepage}
	}
	\fancyfoot{}\fancyhead{}
	\fancyfoot[R]{#4 \textit{MPI}${}^{\star}$}
	\fancyhead[C]{{\sc tp #1} : #2}
	\fancyhead[R]{\thepage}
}

\newcommand{\titletd}[2]{
	\AtBeginDocument{
		\begin{titlepage}
			\begin{center}
				\vspace{10cm}
				{\Large \sc td #1}\\
				\vspace{1cm}
				{\Huge \calligra #2}\\
				\vfill
				Hugo {\sc Salou} MPI${}^{\star}$\\
				{\small Dernière mise à jour le \@date }
			\end{center}
		\end{titlepage}
	}
}
\makeatother

\newcommand{\sign}{
	\null
	\vfill
	\begin{center}
		{
			\fontfamily{ccr}\selectfont
			\textit{\textbf{\.{\"i}}}
		}
	\end{center}
	\vfill
	\null
}

\renewcommand{\thefootnote}{\emph{\alph{footnote}}}

% figure support
\usepackage{import}
\usepackage{xifthen}
\pdfminorversion=7
\usepackage{pdfpages}
\usepackage{transparent}
\newcommand{\incfig}[1]{%
	\def\svgwidth{\columnwidth}
	\import{./figures/}{#1.pdf_tex}
}

\pdfsuppresswarningpagegroup=1
\ctikzset{tripoles/european not symbol=circle}

\newcommand{\missingpart}{{\large\color{red} Il manque quelque chose ici\ldots}}


\fancyhead[R]{Hugo {\sc Salou}\/ MPI}
\fancyhead[L]{TD\textsubscript4 -- Exercice 8}

\begin{document}
	\let\thesection\relax
	
\begin{comment}
\section{Exercice 9}

\slshape
Soit la matrice $A = \begin{pmatrix}
	1&1&-1\\
	2&3&-4\\
	4&1&-4
\end{pmatrix}$.
\begin{enumerate}
	\item Déterminer le spectre de la matrice $A$\/ et trouver une matrice $P$\/ inversible telle que $P^{-1} A P$\/ est diagonale.
	\item Soit $B$\/ une matrice de taille $3\times 3$\/ qui commentent avec $A$\/ ($AB = BA$). Montrer que $B$\/ est diagonale.
\end{enumerate}
\upshape

\begin{enumerate}
	\item On sait, tout d'abord, que, pour $x \in \R$,
		\begin{align*}
			\chi_A(x) = \det(x\,I_n - A) &=
			\begin{vmatrix}
				x - 1 &- 1 & 1\\
				-2&x-3&4\\
				-4&-1&x + 4
			\end{vmatrix}\\
			&= \\
		\end{align*}
\end{enumerate}
\end{comment}

\section{Exercice 8}

\begin{enumerate}
	\item Soit un vecteur non nul $\vec{x} \in \Ker(\lambda {\id} - {u \circ v})$. Ainsi, $u(v(\vec{x})) = \lambda \vec{x}$. Et, donc $v(u(v(\vec{x}))) = \lambda v(\vec{x})$. On a donc $v(\vec{x}) \in \Ker(\lambda {\id} - {v  \circ u})$.
		Or, si $\lambda \neq 0$, on a $v(\vec{x}) \neq \vec{0}$\/ ; en effet, si $v(\vec{x}) = \vec{0}$, alors $u \circ v(\vec{x}) = \vec{0} = \lambda \vec{x}$\/ et donc $\vec{x} = \vec{0}$, ce ne serait donc pas un vecteur propre de $u \circ v$\/ : une contradiction. On en déduit que $v(\vec{x})$\/ est un vecteur propre de $u \circ v$\/ associé à la valeur propre $\lambda$.
	\item On pose donc $\lambda = 0$, une valeur propre de $u  \circ v$. L'endomorphisme $u \circ v$\/ n'est donc pas injectif, donc bijectif. On sait donc, comme $E$\/ est de dimension finie, que $\det(u \circ v) = 0$. Or $\det (u \circ v) = \det u \times \det v = \det(v  \circ u)$. Et donc $\det(v  \circ u) = 0$, $v  \circ u$\/ n'est donc pas bijectif, donc injectif. Et donc, on a $0 \in \Sp(v  \circ u)$.
	\item Soit $P \in \R[X]$, et soit $Q$\/ une primitive de $P$.
		\begin{align*}
			P \in \Ker (u  \circ v) \iff& \Big(\int_{0}^X P(t)~\mathrm{d}t\Big)' = 0\\
			\iff& \big(Q(X) - Q(0)\big)' = 0\\
			\iff& Q'(X) = 0\\
			\iff& P(X) = 0
		\end{align*}
		On en déduit que $\Ker (u \circ v) = \{0\}$.

		Également,
		\begin{align*}
			P \in \Ker(v  \circ u) \iff& \int_{0}^{X} P'(t)~\mathrm{d}t = 0\\
			\iff& P(X) - P(0) = 0\\
			\iff& P(X) = P(0)\\
			\iff& \deg P \le 0\\
			\iff& P \in \R_0[X]
		\end{align*}
		On en déduit que $\Ker(v \circ u) = \R_0[X]$.
\end{enumerate}


\end{document}

	}
	\def\addmacros#1{#1}
}
{
	\td[15]{Grammaires non contextuelles (2)}
	\minitoc
	\renewcommand{\cwd}{../td/td15/}
	\addmacros{
		\section{Quelques problèmes décidables}

\begin{enumerate}
	\item Soit $f : \R \to \R$.
		\begin{itemize}
			\item Si $f$\/ admet un zéro, on pose $\mathcal{M} = \texttt{fun}\ \texttt{s}\ \to \texttt{true}$.
			\item Si $f$\/ n'admet pas un zéro, on pose $\mathcal{M} = \texttt{fun}\ \texttt{s}\ \to \texttt{false}$.
		\end{itemize}
		Alors, $\mathcal{M}$\/ décide \textsc{Zero}$_f$.
	\item Soit $\mathcal{M}$\/ une machine, et soit $w \in \Sigma^*$.
		\begin{itemize}
			\item Si $\mathcal{M}$\/ se termine sur l'entrée $w$, alors on pose $\mathcal{M}' = \texttt{fun}\ \texttt{s} \to \texttt{true}$.
			\item Si $\mathcal{M}$\/ ne se termine pas sur l'entrée $w$, alors on pose $\mathcal{M}' = \texttt{fun}\ \texttt{s} \to \texttt{false}$.
		\end{itemize}
		Alors, $\mathcal{M}'$\/ décide \textsc{Arrêt}$_{\mathcal{M},w}$.
	\item Le problème est trivialement vrai. En effet, soit $M \in \mathcal{O}$, de la forme
		\begin{lstlisting}[language=caml]
let m (s: string): string =
	%*$\langle$\textrm{code}$\rangle$*) 
		\end{lstlisting}
		On crée la machine $\mathcal{N}$\/ ci-dessous.
		\begin{lstlisting}[language=caml]
let n (s: string): string =
	if true then
		%*$\langle$\textrm{code}$\rangle$*) 
	else
		%*$\langle$\textrm{code}$\rangle$*) 
		\end{lstlisting}
		On a $\texttt{m} \neq \texttt{n}$, mais $\mathcal{L}(\texttt{m}) = \mathcal{L}(\texttt{n})$, donc le problème est vrai sur toute entrée et la fonction $\texttt{fun}\ \texttt{s} \to \texttt{true}$\/ répond au problème.
\end{enumerate}

		\section{Ensembles définis inductivement}

La correction est disponible sur \textit{cahier-de-prepa}.

\begin{comment}
	\begin{exm}
		Avec $S = \N$, $\mathcal{B} = \{0, 2\} $, $A_1 = \{0\}$\/ et \begin{align*}
			f_1: A_1 \times \N &\longrightarrow \N \\
			(0, x) &\longmapsto x + 4.
		\end{align*}

		On a \[
			X \supseteq \{0, 2, 4, 6, 8, 10, \ldots, 20, \ldots\} = 2\N
		.\]
	\end{exm}
	\begin{exm}
		Avec $S$\/ l'ensemble des langages sur $\Sigma$, $\mathcal{B} = \{\O\} \cup \bigl\{\{a\}\:\big|\: a \in \Sigma \bigr\}$, et
		\begin{multicols}{3}
			\begin{align*}
				f_1: S \times S &\longrightarrow S \\
				(L_1, L_2) &\longmapsto L_1 \cup L_2,
			\end{align*}
			\begin{align*}
				f_2: S \times S &\longrightarrow S \\
				(L_1, L_2) &\longmapsto L_1 \cdot L_2,
			\end{align*}
			\begin{align*}
				f_3: S &\longrightarrow S \\
				L &\longmapsto L^*.
			\end{align*}
		\end{multicols}
	\end{exm}

\begin{enumerate}
	\item Soit $\mathcal{A} = \{X \subseteq S  \mid X \supseteq \mathcal{B} \mathrel{\text{et}} X \text{ est stable par } f_i\}$. On a $S \in \mathcal{A}$\/ et donc $\mathcal{A} \neq \O$. De plus, soit \[
			Y = \{x \in S  \mid \forall X \in \mathcal{A},\,x \in X\} = \bigcap_{X \in \mathcal{A}} X
		.\]
		Soit $b \in \mathcal{B}$, on a $\forall X \in A,\, b \in X$. D'où $b \in Y$\/ par intersection. On en déduit que $\mathcal{B} \subseteq Y$.

		Soit $i \in \left\llbracket 1,m \right\rrbracket$. Soit $(x_1, \ldots, x_{n_i}) \in Y^{n_i}$\/ et soit $a \in A_i$. Montrons que $f_i(a, x_1, \ldots, x_{n_i}) \in Y$.
		Or, soit $X \in \mathcal{A}$, on a $(x_1, \ldots, x_{n_i}) \in X^{n_i}$\/ donc $f_i(a, x_1, \ldots, x_{n_i}) \in X$. Ceci étant vrai pour tout $X \in \mathcal{A}$, on a $f_i(a, x_1, \ldots, x_{n_i}) \in Y$\/ donc $Y$\/ est stable par $f_i$\/ par tout $i \in \left\llbracket 1,m \right\rrbracket$\/ et donc $Y \in \mathcal{A}$.
		On a également $Y \subseteq X$\/ pour tout $X \in \mathcal{A}$. On en déduit que $Y$\/ est le plus petit élément (pour l'inclusion) de $\mathcal{A}$.
	\item On pose $X_0  = \mathcal{B}$\/ et \[
			X_{n+1} = X_n \cup \big\{ f_i(a, x_1, \ldots, x_{n_i})  \mid a \in A_i,\,(x_1, \ldots, x_{n_i}) \in (X_n)^{n_i},\,i \in \left\llbracket 1,m \right\rrbracket\big\}
		.\]
		Soit $X = \bigcup_{n \in \N} X_n$. Soit $Y$\/ l'ensemble défini par induction à partir de $\mathcal{B}$\/ et des $(f_i)_{i\in\left\llbracket 1,n \right\rrbracket}$. Montrons que $X = Y$.
		On montre que $X$\/ est le plus petit élément (pour l'inclusion) de $\mathcal{A}$\/ et on conclut par unicité du minimum (avec la question précédente).
		Par définition de la suite $(X_n)_{n\in\N}$, elle est croissante (au sens de l'inclusion).
		Montrons à présent, par récurrence, la propriété ci-dessous : $P_n : ``X_n \subseteq Y."$
		\begin{itemize}
			\item Par définition de $Y$, on a $X_0 = \mathcal{B} \subseteq Y$.
			\item Soit 
		\end{itemize}
\end{enumerate}

\subsection{Un théorème d'induction}

\begin{enumerate}
	\item[3.] Soit $Z = \{x \in S  \mid P(x) \text{ vraie}\:\}$.
		Montrons que $\mathcal{X}\subseteq Z$.
		On remarque que $\mathcal{X} \supseteq \mathcal{B}$\/ ; $\mathcal{X}$\/ est stable par $f_i$. On en conclut que $Z \supseteq \mathcal{X}$ et donc $\forall x \in \mathcal{X},\,P(x)$\/ est vraie.
\end{enumerate}

\begin{exm}
	Soit $\mathcal{X}$\/ défini par induction par $\mathcal{B} = \{0, 2\}$\/ et \begin{align*}
		f: \N &\longrightarrow \N \\
		n &\longmapsto n + 2.
	\end{align*}
	Montrons que $\forall n \in \mathcal{X}$, $x$\/ est pair.

	On sait que $0$\/ est pair, $2$\/ est pair ; et, \[
		\forall x,y \in \mathcal{X},\, (x \text{ pair} \land y \text{ pair}) \implies f(x, y) \text{ pair}
	.\]

	On en déduit que \[
		\forall n \in \mathcal{X},\,x \text{ est pair}.
	.\]
\end{exm}
\end{comment}

		\section{Tableaux dynamiques}

\begin{enumerate}[start=3]
	\item On trouve une complexité amortie en $n^2$. À rédiger.
	\item Au lieu de diviser quand $r < n / 2$, mais quand $r < n / 4$.
\end{enumerate}

		\section{Barres en triangle}


On note $a(x)$ le côté du triangule équilatéral, et donc $x = a(x) \sqrt{3}  / 2$.

On calcule le flux $\Phi$ magnétique : \[
	\Phi = B \frac{a x}{2} = B x^2 / \sqrt{3}
.\]
Ainsi, d'après la loi de \textsc{Faraday}, on a \[
	e = - \frac{\mathrm{d}\Phi}{\mathrm{d}t} = - \frac{2B}{\sqrt{3}}  \: x\,\dot{x}
.\]
Or, par loi d'\textsc{Ohm}, $i = e / R(x)$.
Et, on connait la resistance du circuit $R(x) = 3 a(x) / \gamma S$.
Alors, 
\begin{align*}
	i(x) &= \frac{-2B\,x\,\dot{x}}{\sqrt{3} \cdot \frac{3a(x)}{\gamma S}} = - \frac{2B \gamma S}{3 \sqrt{3}} \cdot \frac{x\,\dot{x}}{2 \frac{x}{\sqrt{3}}}\\
	&= - B \gamma S \dot{x} / 3. \\
\end{align*}
On calcule donc la force de \textsc{Laplace} :
\begin{align*}
	\vec{F}_{\mathcal{L}} &= i \cdot [\mathrm{CD}] \cdot B \cdot \vec{e}_x\\
	&= i \cdot \frac{2\dot{x}}{\sqrt{3}} B \vec{e}_x \\
	&= -\underbrace{\frac{2 B^2 \gamma S}{3\sqrt{3}}}_\alpha  \cdot x \dot{x} \\
\end{align*}

D'après le \textsc{pfd}, on a donc \[
	m \ddot{x} = - \alpha x \dot{x} \text{ d'où }  \ddot{x} = - \frac{\alpha}{m} x\dot{x} 
.\] où $m = \rho S L$. 

On intègre les deux côtés de l'équation, \[
	[\dot{x}]_0^{t_\mathrm{f}} = -\frac{\alpha}{m} \cdot \left[ \frac{x^2}{2} \right]_0^{t_\mathrm{f}}
.\]
D'où, \[
	0 - v_0 = \frac{-\alpha}{m} \cdot x_\mathrm{f}^2 / 2
.\] On en conclut \[
	x_\mathrm{f} = \sqrt{\frac{2m}{\alpha} v_0} 
.\] 


		\begin{multicols}{2}
	\section{Implications}
	\begin{enumerate}
		\item
			\[
				\begin{prooftree}
					\infer 0[Ax]{p,q \vdash q}
					\infer 1[$\to$i]{q \vdash p \to q}
				\end{prooftree}
			\]
		\item
			\[
				\begin{prooftree}
					\infer 0[Ax]{p, p\land q\vdash p\land q}
					\infer 1[$\land$e,d]{p, p \land q \vdash q}
					\infer 1[$\to$i]{p \land q \vdash p \to q}
				\end{prooftree}
			\]
		\item 
			\[
				\begin{prooftree}
					\infer 0[Ax]{p, p\to q \vdash p}
					\infer 0[Ax]{p, p\to q \vdash p \to q}
					\infer 0[Ax]{p, p\to q \vdash p}
					\infer 2[$\to$e]{p, p \to q \vdash q}
					\infer 2[$\land$i]{p, p \to q \vdash p \land q}
					\infer 1[$\to$i]{p \to q \vdash p\to (p \land q)}
				\end{prooftree}
			\]
		\item
			\[
				\begin{prooftree}
					\infer 0[Ax]{\lnot q, p\to q, p \vdash p \to q}
					\infer 0[Ax]{\lnot q, p\to q, p \vdash p}
					\infer 2[$\to$e]{\lnot q, p\to q, p \vdash q}
					\infer 0[Ax]{\lnot q, p\to q, p \vdash \lnot q}
					\infer 2[$\lnot$e]{\lnot q, p\to q, p \vdash \bot}
					\infer 1[$\lnot$i]{\lnot q, p\to q \vdash\lnot p}
					\infer 1[$\to$i]{p \to q \vdash \lnot q \to \lnot p}
				\end{prooftree}
			\]
		\item 
			\[
				\begin{prooftree}
					\infer 0[Ax]{p \land q, p\to r \vdash p \land q}
					\infer 1[$\land$e,g]{p \land q, p\to r \vdash p}
					\infer 0[Ax]{p \land q, p\to r \vdash p \to r}
					\infer 2[$\to$e]{p \land q, p \to r \vdash r}
					\infer 1[$\to$i]{p \to r \vdash (p \land q) \to r}
				\end{prooftree}
			\] 
		\item
			\[
				\begin{prooftree}
					\infer 0[Ax]{p,q\vdash p}
					\infer 1[$\to$i]{p \vdash q \to p}
					\infer 1[$\to$i]{\vdash p \to (q \to p)}
				\end{prooftree}
			\]
		\item
			\[
				\begin{prooftree}
					\infer 0[Ax]{p \to q, p \vdash p\to q}
					\infer 0[Ax]{p\to q,p \vdash p}
					\infer 2[$\to$e]{p, p\to q \vdash q}
					\infer 1[$\to$i]{p \vdash (p \to q) \to q}
				\end{prooftree}
			\] 
	\end{enumerate}
\end{multicols}

	}
	\def\addmacros#1{#1}
}
{
	\td[16]{Concurrence}
	\minitoc
	\renewcommand{\cwd}{../td/td16/}
	\addmacros{
		\section{Quelques problèmes décidables}

\begin{enumerate}
	\item Soit $f : \R \to \R$.
		\begin{itemize}
			\item Si $f$\/ admet un zéro, on pose $\mathcal{M} = \texttt{fun}\ \texttt{s}\ \to \texttt{true}$.
			\item Si $f$\/ n'admet pas un zéro, on pose $\mathcal{M} = \texttt{fun}\ \texttt{s}\ \to \texttt{false}$.
		\end{itemize}
		Alors, $\mathcal{M}$\/ décide \textsc{Zero}$_f$.
	\item Soit $\mathcal{M}$\/ une machine, et soit $w \in \Sigma^*$.
		\begin{itemize}
			\item Si $\mathcal{M}$\/ se termine sur l'entrée $w$, alors on pose $\mathcal{M}' = \texttt{fun}\ \texttt{s} \to \texttt{true}$.
			\item Si $\mathcal{M}$\/ ne se termine pas sur l'entrée $w$, alors on pose $\mathcal{M}' = \texttt{fun}\ \texttt{s} \to \texttt{false}$.
		\end{itemize}
		Alors, $\mathcal{M}'$\/ décide \textsc{Arrêt}$_{\mathcal{M},w}$.
	\item Le problème est trivialement vrai. En effet, soit $M \in \mathcal{O}$, de la forme
		\begin{lstlisting}[language=caml]
let m (s: string): string =
	%*$\langle$\textrm{code}$\rangle$*) 
		\end{lstlisting}
		On crée la machine $\mathcal{N}$\/ ci-dessous.
		\begin{lstlisting}[language=caml]
let n (s: string): string =
	if true then
		%*$\langle$\textrm{code}$\rangle$*) 
	else
		%*$\langle$\textrm{code}$\rangle$*) 
		\end{lstlisting}
		On a $\texttt{m} \neq \texttt{n}$, mais $\mathcal{L}(\texttt{m}) = \mathcal{L}(\texttt{n})$, donc le problème est vrai sur toute entrée et la fonction $\texttt{fun}\ \texttt{s} \to \texttt{true}$\/ répond au problème.
\end{enumerate}

		\section{Ensembles définis inductivement}

La correction est disponible sur \textit{cahier-de-prepa}.

\begin{comment}
	\begin{exm}
		Avec $S = \N$, $\mathcal{B} = \{0, 2\} $, $A_1 = \{0\}$\/ et \begin{align*}
			f_1: A_1 \times \N &\longrightarrow \N \\
			(0, x) &\longmapsto x + 4.
		\end{align*}

		On a \[
			X \supseteq \{0, 2, 4, 6, 8, 10, \ldots, 20, \ldots\} = 2\N
		.\]
	\end{exm}
	\begin{exm}
		Avec $S$\/ l'ensemble des langages sur $\Sigma$, $\mathcal{B} = \{\O\} \cup \bigl\{\{a\}\:\big|\: a \in \Sigma \bigr\}$, et
		\begin{multicols}{3}
			\begin{align*}
				f_1: S \times S &\longrightarrow S \\
				(L_1, L_2) &\longmapsto L_1 \cup L_2,
			\end{align*}
			\begin{align*}
				f_2: S \times S &\longrightarrow S \\
				(L_1, L_2) &\longmapsto L_1 \cdot L_2,
			\end{align*}
			\begin{align*}
				f_3: S &\longrightarrow S \\
				L &\longmapsto L^*.
			\end{align*}
		\end{multicols}
	\end{exm}

\begin{enumerate}
	\item Soit $\mathcal{A} = \{X \subseteq S  \mid X \supseteq \mathcal{B} \mathrel{\text{et}} X \text{ est stable par } f_i\}$. On a $S \in \mathcal{A}$\/ et donc $\mathcal{A} \neq \O$. De plus, soit \[
			Y = \{x \in S  \mid \forall X \in \mathcal{A},\,x \in X\} = \bigcap_{X \in \mathcal{A}} X
		.\]
		Soit $b \in \mathcal{B}$, on a $\forall X \in A,\, b \in X$. D'où $b \in Y$\/ par intersection. On en déduit que $\mathcal{B} \subseteq Y$.

		Soit $i \in \left\llbracket 1,m \right\rrbracket$. Soit $(x_1, \ldots, x_{n_i}) \in Y^{n_i}$\/ et soit $a \in A_i$. Montrons que $f_i(a, x_1, \ldots, x_{n_i}) \in Y$.
		Or, soit $X \in \mathcal{A}$, on a $(x_1, \ldots, x_{n_i}) \in X^{n_i}$\/ donc $f_i(a, x_1, \ldots, x_{n_i}) \in X$. Ceci étant vrai pour tout $X \in \mathcal{A}$, on a $f_i(a, x_1, \ldots, x_{n_i}) \in Y$\/ donc $Y$\/ est stable par $f_i$\/ par tout $i \in \left\llbracket 1,m \right\rrbracket$\/ et donc $Y \in \mathcal{A}$.
		On a également $Y \subseteq X$\/ pour tout $X \in \mathcal{A}$. On en déduit que $Y$\/ est le plus petit élément (pour l'inclusion) de $\mathcal{A}$.
	\item On pose $X_0  = \mathcal{B}$\/ et \[
			X_{n+1} = X_n \cup \big\{ f_i(a, x_1, \ldots, x_{n_i})  \mid a \in A_i,\,(x_1, \ldots, x_{n_i}) \in (X_n)^{n_i},\,i \in \left\llbracket 1,m \right\rrbracket\big\}
		.\]
		Soit $X = \bigcup_{n \in \N} X_n$. Soit $Y$\/ l'ensemble défini par induction à partir de $\mathcal{B}$\/ et des $(f_i)_{i\in\left\llbracket 1,n \right\rrbracket}$. Montrons que $X = Y$.
		On montre que $X$\/ est le plus petit élément (pour l'inclusion) de $\mathcal{A}$\/ et on conclut par unicité du minimum (avec la question précédente).
		Par définition de la suite $(X_n)_{n\in\N}$, elle est croissante (au sens de l'inclusion).
		Montrons à présent, par récurrence, la propriété ci-dessous : $P_n : ``X_n \subseteq Y."$
		\begin{itemize}
			\item Par définition de $Y$, on a $X_0 = \mathcal{B} \subseteq Y$.
			\item Soit 
		\end{itemize}
\end{enumerate}

\subsection{Un théorème d'induction}

\begin{enumerate}
	\item[3.] Soit $Z = \{x \in S  \mid P(x) \text{ vraie}\:\}$.
		Montrons que $\mathcal{X}\subseteq Z$.
		On remarque que $\mathcal{X} \supseteq \mathcal{B}$\/ ; $\mathcal{X}$\/ est stable par $f_i$. On en conclut que $Z \supseteq \mathcal{X}$ et donc $\forall x \in \mathcal{X},\,P(x)$\/ est vraie.
\end{enumerate}

\begin{exm}
	Soit $\mathcal{X}$\/ défini par induction par $\mathcal{B} = \{0, 2\}$\/ et \begin{align*}
		f: \N &\longrightarrow \N \\
		n &\longmapsto n + 2.
	\end{align*}
	Montrons que $\forall n \in \mathcal{X}$, $x$\/ est pair.

	On sait que $0$\/ est pair, $2$\/ est pair ; et, \[
		\forall x,y \in \mathcal{X},\, (x \text{ pair} \land y \text{ pair}) \implies f(x, y) \text{ pair}
	.\]

	On en déduit que \[
		\forall n \in \mathcal{X},\,x \text{ est pair}.
	.\]
\end{exm}
\end{comment}

		\section{Tableaux dynamiques}

\begin{enumerate}[start=3]
	\item On trouve une complexité amortie en $n^2$. À rédiger.
	\item Au lieu de diviser quand $r < n / 2$, mais quand $r < n / 4$.
\end{enumerate}

		\section{Barres en triangle}


On note $a(x)$ le côté du triangule équilatéral, et donc $x = a(x) \sqrt{3}  / 2$.

On calcule le flux $\Phi$ magnétique : \[
	\Phi = B \frac{a x}{2} = B x^2 / \sqrt{3}
.\]
Ainsi, d'après la loi de \textsc{Faraday}, on a \[
	e = - \frac{\mathrm{d}\Phi}{\mathrm{d}t} = - \frac{2B}{\sqrt{3}}  \: x\,\dot{x}
.\]
Or, par loi d'\textsc{Ohm}, $i = e / R(x)$.
Et, on connait la resistance du circuit $R(x) = 3 a(x) / \gamma S$.
Alors, 
\begin{align*}
	i(x) &= \frac{-2B\,x\,\dot{x}}{\sqrt{3} \cdot \frac{3a(x)}{\gamma S}} = - \frac{2B \gamma S}{3 \sqrt{3}} \cdot \frac{x\,\dot{x}}{2 \frac{x}{\sqrt{3}}}\\
	&= - B \gamma S \dot{x} / 3. \\
\end{align*}
On calcule donc la force de \textsc{Laplace} :
\begin{align*}
	\vec{F}_{\mathcal{L}} &= i \cdot [\mathrm{CD}] \cdot B \cdot \vec{e}_x\\
	&= i \cdot \frac{2\dot{x}}{\sqrt{3}} B \vec{e}_x \\
	&= -\underbrace{\frac{2 B^2 \gamma S}{3\sqrt{3}}}_\alpha  \cdot x \dot{x} \\
\end{align*}

D'après le \textsc{pfd}, on a donc \[
	m \ddot{x} = - \alpha x \dot{x} \text{ d'où }  \ddot{x} = - \frac{\alpha}{m} x\dot{x} 
.\] où $m = \rho S L$. 

On intègre les deux côtés de l'équation, \[
	[\dot{x}]_0^{t_\mathrm{f}} = -\frac{\alpha}{m} \cdot \left[ \frac{x^2}{2} \right]_0^{t_\mathrm{f}}
.\]
D'où, \[
	0 - v_0 = \frac{-\alpha}{m} \cdot x_\mathrm{f}^2 / 2
.\] On en conclut \[
	x_\mathrm{f} = \sqrt{\frac{2m}{\alpha} v_0} 
.\] 


		\begin{multicols}{2}
	\section{Implications}
	\begin{enumerate}
		\item
			\[
				\begin{prooftree}
					\infer 0[Ax]{p,q \vdash q}
					\infer 1[$\to$i]{q \vdash p \to q}
				\end{prooftree}
			\]
		\item
			\[
				\begin{prooftree}
					\infer 0[Ax]{p, p\land q\vdash p\land q}
					\infer 1[$\land$e,d]{p, p \land q \vdash q}
					\infer 1[$\to$i]{p \land q \vdash p \to q}
				\end{prooftree}
			\]
		\item 
			\[
				\begin{prooftree}
					\infer 0[Ax]{p, p\to q \vdash p}
					\infer 0[Ax]{p, p\to q \vdash p \to q}
					\infer 0[Ax]{p, p\to q \vdash p}
					\infer 2[$\to$e]{p, p \to q \vdash q}
					\infer 2[$\land$i]{p, p \to q \vdash p \land q}
					\infer 1[$\to$i]{p \to q \vdash p\to (p \land q)}
				\end{prooftree}
			\]
		\item
			\[
				\begin{prooftree}
					\infer 0[Ax]{\lnot q, p\to q, p \vdash p \to q}
					\infer 0[Ax]{\lnot q, p\to q, p \vdash p}
					\infer 2[$\to$e]{\lnot q, p\to q, p \vdash q}
					\infer 0[Ax]{\lnot q, p\to q, p \vdash \lnot q}
					\infer 2[$\lnot$e]{\lnot q, p\to q, p \vdash \bot}
					\infer 1[$\lnot$i]{\lnot q, p\to q \vdash\lnot p}
					\infer 1[$\to$i]{p \to q \vdash \lnot q \to \lnot p}
				\end{prooftree}
			\]
		\item 
			\[
				\begin{prooftree}
					\infer 0[Ax]{p \land q, p\to r \vdash p \land q}
					\infer 1[$\land$e,g]{p \land q, p\to r \vdash p}
					\infer 0[Ax]{p \land q, p\to r \vdash p \to r}
					\infer 2[$\to$e]{p \land q, p \to r \vdash r}
					\infer 1[$\to$i]{p \to r \vdash (p \land q) \to r}
				\end{prooftree}
			\] 
		\item
			\[
				\begin{prooftree}
					\infer 0[Ax]{p,q\vdash p}
					\infer 1[$\to$i]{p \vdash q \to p}
					\infer 1[$\to$i]{\vdash p \to (q \to p)}
				\end{prooftree}
			\]
		\item
			\[
				\begin{prooftree}
					\infer 0[Ax]{p \to q, p \vdash p\to q}
					\infer 0[Ax]{p\to q,p \vdash p}
					\infer 2[$\to$e]{p, p\to q \vdash q}
					\infer 1[$\to$i]{p \vdash (p \to q) \to q}
				\end{prooftree}
			\] 
	\end{enumerate}
\end{multicols}

	}
	\def\addmacros#1{#1}
}
{
	\td[17]{Concurrence}
	\minitoc
	\renewcommand{\cwd}{../td/td17/}
	\addmacros{
		\section{Quelques problèmes décidables}

\begin{enumerate}
	\item Soit $f : \R \to \R$.
		\begin{itemize}
			\item Si $f$\/ admet un zéro, on pose $\mathcal{M} = \texttt{fun}\ \texttt{s}\ \to \texttt{true}$.
			\item Si $f$\/ n'admet pas un zéro, on pose $\mathcal{M} = \texttt{fun}\ \texttt{s}\ \to \texttt{false}$.
		\end{itemize}
		Alors, $\mathcal{M}$\/ décide \textsc{Zero}$_f$.
	\item Soit $\mathcal{M}$\/ une machine, et soit $w \in \Sigma^*$.
		\begin{itemize}
			\item Si $\mathcal{M}$\/ se termine sur l'entrée $w$, alors on pose $\mathcal{M}' = \texttt{fun}\ \texttt{s} \to \texttt{true}$.
			\item Si $\mathcal{M}$\/ ne se termine pas sur l'entrée $w$, alors on pose $\mathcal{M}' = \texttt{fun}\ \texttt{s} \to \texttt{false}$.
		\end{itemize}
		Alors, $\mathcal{M}'$\/ décide \textsc{Arrêt}$_{\mathcal{M},w}$.
	\item Le problème est trivialement vrai. En effet, soit $M \in \mathcal{O}$, de la forme
		\begin{lstlisting}[language=caml]
let m (s: string): string =
	%*$\langle$\textrm{code}$\rangle$*) 
		\end{lstlisting}
		On crée la machine $\mathcal{N}$\/ ci-dessous.
		\begin{lstlisting}[language=caml]
let n (s: string): string =
	if true then
		%*$\langle$\textrm{code}$\rangle$*) 
	else
		%*$\langle$\textrm{code}$\rangle$*) 
		\end{lstlisting}
		On a $\texttt{m} \neq \texttt{n}$, mais $\mathcal{L}(\texttt{m}) = \mathcal{L}(\texttt{n})$, donc le problème est vrai sur toute entrée et la fonction $\texttt{fun}\ \texttt{s} \to \texttt{true}$\/ répond au problème.
\end{enumerate}

		\section{Ensembles définis inductivement}

La correction est disponible sur \textit{cahier-de-prepa}.

\begin{comment}
	\begin{exm}
		Avec $S = \N$, $\mathcal{B} = \{0, 2\} $, $A_1 = \{0\}$\/ et \begin{align*}
			f_1: A_1 \times \N &\longrightarrow \N \\
			(0, x) &\longmapsto x + 4.
		\end{align*}

		On a \[
			X \supseteq \{0, 2, 4, 6, 8, 10, \ldots, 20, \ldots\} = 2\N
		.\]
	\end{exm}
	\begin{exm}
		Avec $S$\/ l'ensemble des langages sur $\Sigma$, $\mathcal{B} = \{\O\} \cup \bigl\{\{a\}\:\big|\: a \in \Sigma \bigr\}$, et
		\begin{multicols}{3}
			\begin{align*}
				f_1: S \times S &\longrightarrow S \\
				(L_1, L_2) &\longmapsto L_1 \cup L_2,
			\end{align*}
			\begin{align*}
				f_2: S \times S &\longrightarrow S \\
				(L_1, L_2) &\longmapsto L_1 \cdot L_2,
			\end{align*}
			\begin{align*}
				f_3: S &\longrightarrow S \\
				L &\longmapsto L^*.
			\end{align*}
		\end{multicols}
	\end{exm}

\begin{enumerate}
	\item Soit $\mathcal{A} = \{X \subseteq S  \mid X \supseteq \mathcal{B} \mathrel{\text{et}} X \text{ est stable par } f_i\}$. On a $S \in \mathcal{A}$\/ et donc $\mathcal{A} \neq \O$. De plus, soit \[
			Y = \{x \in S  \mid \forall X \in \mathcal{A},\,x \in X\} = \bigcap_{X \in \mathcal{A}} X
		.\]
		Soit $b \in \mathcal{B}$, on a $\forall X \in A,\, b \in X$. D'où $b \in Y$\/ par intersection. On en déduit que $\mathcal{B} \subseteq Y$.

		Soit $i \in \left\llbracket 1,m \right\rrbracket$. Soit $(x_1, \ldots, x_{n_i}) \in Y^{n_i}$\/ et soit $a \in A_i$. Montrons que $f_i(a, x_1, \ldots, x_{n_i}) \in Y$.
		Or, soit $X \in \mathcal{A}$, on a $(x_1, \ldots, x_{n_i}) \in X^{n_i}$\/ donc $f_i(a, x_1, \ldots, x_{n_i}) \in X$. Ceci étant vrai pour tout $X \in \mathcal{A}$, on a $f_i(a, x_1, \ldots, x_{n_i}) \in Y$\/ donc $Y$\/ est stable par $f_i$\/ par tout $i \in \left\llbracket 1,m \right\rrbracket$\/ et donc $Y \in \mathcal{A}$.
		On a également $Y \subseteq X$\/ pour tout $X \in \mathcal{A}$. On en déduit que $Y$\/ est le plus petit élément (pour l'inclusion) de $\mathcal{A}$.
	\item On pose $X_0  = \mathcal{B}$\/ et \[
			X_{n+1} = X_n \cup \big\{ f_i(a, x_1, \ldots, x_{n_i})  \mid a \in A_i,\,(x_1, \ldots, x_{n_i}) \in (X_n)^{n_i},\,i \in \left\llbracket 1,m \right\rrbracket\big\}
		.\]
		Soit $X = \bigcup_{n \in \N} X_n$. Soit $Y$\/ l'ensemble défini par induction à partir de $\mathcal{B}$\/ et des $(f_i)_{i\in\left\llbracket 1,n \right\rrbracket}$. Montrons que $X = Y$.
		On montre que $X$\/ est le plus petit élément (pour l'inclusion) de $\mathcal{A}$\/ et on conclut par unicité du minimum (avec la question précédente).
		Par définition de la suite $(X_n)_{n\in\N}$, elle est croissante (au sens de l'inclusion).
		Montrons à présent, par récurrence, la propriété ci-dessous : $P_n : ``X_n \subseteq Y."$
		\begin{itemize}
			\item Par définition de $Y$, on a $X_0 = \mathcal{B} \subseteq Y$.
			\item Soit 
		\end{itemize}
\end{enumerate}

\subsection{Un théorème d'induction}

\begin{enumerate}
	\item[3.] Soit $Z = \{x \in S  \mid P(x) \text{ vraie}\:\}$.
		Montrons que $\mathcal{X}\subseteq Z$.
		On remarque que $\mathcal{X} \supseteq \mathcal{B}$\/ ; $\mathcal{X}$\/ est stable par $f_i$. On en conclut que $Z \supseteq \mathcal{X}$ et donc $\forall x \in \mathcal{X},\,P(x)$\/ est vraie.
\end{enumerate}

\begin{exm}
	Soit $\mathcal{X}$\/ défini par induction par $\mathcal{B} = \{0, 2\}$\/ et \begin{align*}
		f: \N &\longrightarrow \N \\
		n &\longmapsto n + 2.
	\end{align*}
	Montrons que $\forall n \in \mathcal{X}$, $x$\/ est pair.

	On sait que $0$\/ est pair, $2$\/ est pair ; et, \[
		\forall x,y \in \mathcal{X},\, (x \text{ pair} \land y \text{ pair}) \implies f(x, y) \text{ pair}
	.\]

	On en déduit que \[
		\forall n \in \mathcal{X},\,x \text{ est pair}.
	.\]
\end{exm}
\end{comment}

		\section{Tableaux dynamiques}

\begin{enumerate}[start=3]
	\item On trouve une complexité amortie en $n^2$. À rédiger.
	\item Au lieu de diviser quand $r < n / 2$, mais quand $r < n / 4$.
\end{enumerate}

	}
	\def\addmacros#1{#1}
}
\def\prefix{\textsc{td bonus}}
{
	\td[1]{Complexité amortie}
	\minitoc
	\renewcommand{\cwd}{../td/td_b1/}
	\addmacros{
		\section{Quelques problèmes décidables}

\begin{enumerate}
	\item Soit $f : \R \to \R$.
		\begin{itemize}
			\item Si $f$\/ admet un zéro, on pose $\mathcal{M} = \texttt{fun}\ \texttt{s}\ \to \texttt{true}$.
			\item Si $f$\/ n'admet pas un zéro, on pose $\mathcal{M} = \texttt{fun}\ \texttt{s}\ \to \texttt{false}$.
		\end{itemize}
		Alors, $\mathcal{M}$\/ décide \textsc{Zero}$_f$.
	\item Soit $\mathcal{M}$\/ une machine, et soit $w \in \Sigma^*$.
		\begin{itemize}
			\item Si $\mathcal{M}$\/ se termine sur l'entrée $w$, alors on pose $\mathcal{M}' = \texttt{fun}\ \texttt{s} \to \texttt{true}$.
			\item Si $\mathcal{M}$\/ ne se termine pas sur l'entrée $w$, alors on pose $\mathcal{M}' = \texttt{fun}\ \texttt{s} \to \texttt{false}$.
		\end{itemize}
		Alors, $\mathcal{M}'$\/ décide \textsc{Arrêt}$_{\mathcal{M},w}$.
	\item Le problème est trivialement vrai. En effet, soit $M \in \mathcal{O}$, de la forme
		\begin{lstlisting}[language=caml]
let m (s: string): string =
	%*$\langle$\textrm{code}$\rangle$*) 
		\end{lstlisting}
		On crée la machine $\mathcal{N}$\/ ci-dessous.
		\begin{lstlisting}[language=caml]
let n (s: string): string =
	if true then
		%*$\langle$\textrm{code}$\rangle$*) 
	else
		%*$\langle$\textrm{code}$\rangle$*) 
		\end{lstlisting}
		On a $\texttt{m} \neq \texttt{n}$, mais $\mathcal{L}(\texttt{m}) = \mathcal{L}(\texttt{n})$, donc le problème est vrai sur toute entrée et la fonction $\texttt{fun}\ \texttt{s} \to \texttt{true}$\/ répond au problème.
\end{enumerate}

		\section{Ensembles définis inductivement}

La correction est disponible sur \textit{cahier-de-prepa}.

\begin{comment}
	\begin{exm}
		Avec $S = \N$, $\mathcal{B} = \{0, 2\} $, $A_1 = \{0\}$\/ et \begin{align*}
			f_1: A_1 \times \N &\longrightarrow \N \\
			(0, x) &\longmapsto x + 4.
		\end{align*}

		On a \[
			X \supseteq \{0, 2, 4, 6, 8, 10, \ldots, 20, \ldots\} = 2\N
		.\]
	\end{exm}
	\begin{exm}
		Avec $S$\/ l'ensemble des langages sur $\Sigma$, $\mathcal{B} = \{\O\} \cup \bigl\{\{a\}\:\big|\: a \in \Sigma \bigr\}$, et
		\begin{multicols}{3}
			\begin{align*}
				f_1: S \times S &\longrightarrow S \\
				(L_1, L_2) &\longmapsto L_1 \cup L_2,
			\end{align*}
			\begin{align*}
				f_2: S \times S &\longrightarrow S \\
				(L_1, L_2) &\longmapsto L_1 \cdot L_2,
			\end{align*}
			\begin{align*}
				f_3: S &\longrightarrow S \\
				L &\longmapsto L^*.
			\end{align*}
		\end{multicols}
	\end{exm}

\begin{enumerate}
	\item Soit $\mathcal{A} = \{X \subseteq S  \mid X \supseteq \mathcal{B} \mathrel{\text{et}} X \text{ est stable par } f_i\}$. On a $S \in \mathcal{A}$\/ et donc $\mathcal{A} \neq \O$. De plus, soit \[
			Y = \{x \in S  \mid \forall X \in \mathcal{A},\,x \in X\} = \bigcap_{X \in \mathcal{A}} X
		.\]
		Soit $b \in \mathcal{B}$, on a $\forall X \in A,\, b \in X$. D'où $b \in Y$\/ par intersection. On en déduit que $\mathcal{B} \subseteq Y$.

		Soit $i \in \left\llbracket 1,m \right\rrbracket$. Soit $(x_1, \ldots, x_{n_i}) \in Y^{n_i}$\/ et soit $a \in A_i$. Montrons que $f_i(a, x_1, \ldots, x_{n_i}) \in Y$.
		Or, soit $X \in \mathcal{A}$, on a $(x_1, \ldots, x_{n_i}) \in X^{n_i}$\/ donc $f_i(a, x_1, \ldots, x_{n_i}) \in X$. Ceci étant vrai pour tout $X \in \mathcal{A}$, on a $f_i(a, x_1, \ldots, x_{n_i}) \in Y$\/ donc $Y$\/ est stable par $f_i$\/ par tout $i \in \left\llbracket 1,m \right\rrbracket$\/ et donc $Y \in \mathcal{A}$.
		On a également $Y \subseteq X$\/ pour tout $X \in \mathcal{A}$. On en déduit que $Y$\/ est le plus petit élément (pour l'inclusion) de $\mathcal{A}$.
	\item On pose $X_0  = \mathcal{B}$\/ et \[
			X_{n+1} = X_n \cup \big\{ f_i(a, x_1, \ldots, x_{n_i})  \mid a \in A_i,\,(x_1, \ldots, x_{n_i}) \in (X_n)^{n_i},\,i \in \left\llbracket 1,m \right\rrbracket\big\}
		.\]
		Soit $X = \bigcup_{n \in \N} X_n$. Soit $Y$\/ l'ensemble défini par induction à partir de $\mathcal{B}$\/ et des $(f_i)_{i\in\left\llbracket 1,n \right\rrbracket}$. Montrons que $X = Y$.
		On montre que $X$\/ est le plus petit élément (pour l'inclusion) de $\mathcal{A}$\/ et on conclut par unicité du minimum (avec la question précédente).
		Par définition de la suite $(X_n)_{n\in\N}$, elle est croissante (au sens de l'inclusion).
		Montrons à présent, par récurrence, la propriété ci-dessous : $P_n : ``X_n \subseteq Y."$
		\begin{itemize}
			\item Par définition de $Y$, on a $X_0 = \mathcal{B} \subseteq Y$.
			\item Soit 
		\end{itemize}
\end{enumerate}

\subsection{Un théorème d'induction}

\begin{enumerate}
	\item[3.] Soit $Z = \{x \in S  \mid P(x) \text{ vraie}\:\}$.
		Montrons que $\mathcal{X}\subseteq Z$.
		On remarque que $\mathcal{X} \supseteq \mathcal{B}$\/ ; $\mathcal{X}$\/ est stable par $f_i$. On en conclut que $Z \supseteq \mathcal{X}$ et donc $\forall x \in \mathcal{X},\,P(x)$\/ est vraie.
\end{enumerate}

\begin{exm}
	Soit $\mathcal{X}$\/ défini par induction par $\mathcal{B} = \{0, 2\}$\/ et \begin{align*}
		f: \N &\longrightarrow \N \\
		n &\longmapsto n + 2.
	\end{align*}
	Montrons que $\forall n \in \mathcal{X}$, $x$\/ est pair.

	On sait que $0$\/ est pair, $2$\/ est pair ; et, \[
		\forall x,y \in \mathcal{X},\, (x \text{ pair} \land y \text{ pair}) \implies f(x, y) \text{ pair}
	.\]

	On en déduit que \[
		\forall n \in \mathcal{X},\,x \text{ est pair}.
	.\]
\end{exm}
\end{comment}

		\section{Tableaux dynamiques}

\begin{enumerate}[start=3]
	\item On trouve une complexité amortie en $n^2$. À rédiger.
	\item Au lieu de diviser quand $r < n / 2$, mais quand $r < n / 4$.
\end{enumerate}

		\section{Barres en triangle}


On note $a(x)$ le côté du triangule équilatéral, et donc $x = a(x) \sqrt{3}  / 2$.

On calcule le flux $\Phi$ magnétique : \[
	\Phi = B \frac{a x}{2} = B x^2 / \sqrt{3}
.\]
Ainsi, d'après la loi de \textsc{Faraday}, on a \[
	e = - \frac{\mathrm{d}\Phi}{\mathrm{d}t} = - \frac{2B}{\sqrt{3}}  \: x\,\dot{x}
.\]
Or, par loi d'\textsc{Ohm}, $i = e / R(x)$.
Et, on connait la resistance du circuit $R(x) = 3 a(x) / \gamma S$.
Alors, 
\begin{align*}
	i(x) &= \frac{-2B\,x\,\dot{x}}{\sqrt{3} \cdot \frac{3a(x)}{\gamma S}} = - \frac{2B \gamma S}{3 \sqrt{3}} \cdot \frac{x\,\dot{x}}{2 \frac{x}{\sqrt{3}}}\\
	&= - B \gamma S \dot{x} / 3. \\
\end{align*}
On calcule donc la force de \textsc{Laplace} :
\begin{align*}
	\vec{F}_{\mathcal{L}} &= i \cdot [\mathrm{CD}] \cdot B \cdot \vec{e}_x\\
	&= i \cdot \frac{2\dot{x}}{\sqrt{3}} B \vec{e}_x \\
	&= -\underbrace{\frac{2 B^2 \gamma S}{3\sqrt{3}}}_\alpha  \cdot x \dot{x} \\
\end{align*}

D'après le \textsc{pfd}, on a donc \[
	m \ddot{x} = - \alpha x \dot{x} \text{ d'où }  \ddot{x} = - \frac{\alpha}{m} x\dot{x} 
.\] où $m = \rho S L$. 

On intègre les deux côtés de l'équation, \[
	[\dot{x}]_0^{t_\mathrm{f}} = -\frac{\alpha}{m} \cdot \left[ \frac{x^2}{2} \right]_0^{t_\mathrm{f}}
.\]
D'où, \[
	0 - v_0 = \frac{-\alpha}{m} \cdot x_\mathrm{f}^2 / 2
.\] On en conclut \[
	x_\mathrm{f} = \sqrt{\frac{2m}{\alpha} v_0} 
.\] 


	}
	\def\addmacros#1{#1}
}
\def\prefix{\textsc{td bonus}}
{
	\td[2]{Diviser pour régner}
	\minitoc
	\renewcommand{\cwd}{../td/td_b2/}
	\addmacros{
		\section{Quelques problèmes décidables}

\begin{enumerate}
	\item Soit $f : \R \to \R$.
		\begin{itemize}
			\item Si $f$\/ admet un zéro, on pose $\mathcal{M} = \texttt{fun}\ \texttt{s}\ \to \texttt{true}$.
			\item Si $f$\/ n'admet pas un zéro, on pose $\mathcal{M} = \texttt{fun}\ \texttt{s}\ \to \texttt{false}$.
		\end{itemize}
		Alors, $\mathcal{M}$\/ décide \textsc{Zero}$_f$.
	\item Soit $\mathcal{M}$\/ une machine, et soit $w \in \Sigma^*$.
		\begin{itemize}
			\item Si $\mathcal{M}$\/ se termine sur l'entrée $w$, alors on pose $\mathcal{M}' = \texttt{fun}\ \texttt{s} \to \texttt{true}$.
			\item Si $\mathcal{M}$\/ ne se termine pas sur l'entrée $w$, alors on pose $\mathcal{M}' = \texttt{fun}\ \texttt{s} \to \texttt{false}$.
		\end{itemize}
		Alors, $\mathcal{M}'$\/ décide \textsc{Arrêt}$_{\mathcal{M},w}$.
	\item Le problème est trivialement vrai. En effet, soit $M \in \mathcal{O}$, de la forme
		\begin{lstlisting}[language=caml]
let m (s: string): string =
	%*$\langle$\textrm{code}$\rangle$*) 
		\end{lstlisting}
		On crée la machine $\mathcal{N}$\/ ci-dessous.
		\begin{lstlisting}[language=caml]
let n (s: string): string =
	if true then
		%*$\langle$\textrm{code}$\rangle$*) 
	else
		%*$\langle$\textrm{code}$\rangle$*) 
		\end{lstlisting}
		On a $\texttt{m} \neq \texttt{n}$, mais $\mathcal{L}(\texttt{m}) = \mathcal{L}(\texttt{n})$, donc le problème est vrai sur toute entrée et la fonction $\texttt{fun}\ \texttt{s} \to \texttt{true}$\/ répond au problème.
\end{enumerate}

		\section{Ensembles définis inductivement}

La correction est disponible sur \textit{cahier-de-prepa}.

\begin{comment}
	\begin{exm}
		Avec $S = \N$, $\mathcal{B} = \{0, 2\} $, $A_1 = \{0\}$\/ et \begin{align*}
			f_1: A_1 \times \N &\longrightarrow \N \\
			(0, x) &\longmapsto x + 4.
		\end{align*}

		On a \[
			X \supseteq \{0, 2, 4, 6, 8, 10, \ldots, 20, \ldots\} = 2\N
		.\]
	\end{exm}
	\begin{exm}
		Avec $S$\/ l'ensemble des langages sur $\Sigma$, $\mathcal{B} = \{\O\} \cup \bigl\{\{a\}\:\big|\: a \in \Sigma \bigr\}$, et
		\begin{multicols}{3}
			\begin{align*}
				f_1: S \times S &\longrightarrow S \\
				(L_1, L_2) &\longmapsto L_1 \cup L_2,
			\end{align*}
			\begin{align*}
				f_2: S \times S &\longrightarrow S \\
				(L_1, L_2) &\longmapsto L_1 \cdot L_2,
			\end{align*}
			\begin{align*}
				f_3: S &\longrightarrow S \\
				L &\longmapsto L^*.
			\end{align*}
		\end{multicols}
	\end{exm}

\begin{enumerate}
	\item Soit $\mathcal{A} = \{X \subseteq S  \mid X \supseteq \mathcal{B} \mathrel{\text{et}} X \text{ est stable par } f_i\}$. On a $S \in \mathcal{A}$\/ et donc $\mathcal{A} \neq \O$. De plus, soit \[
			Y = \{x \in S  \mid \forall X \in \mathcal{A},\,x \in X\} = \bigcap_{X \in \mathcal{A}} X
		.\]
		Soit $b \in \mathcal{B}$, on a $\forall X \in A,\, b \in X$. D'où $b \in Y$\/ par intersection. On en déduit que $\mathcal{B} \subseteq Y$.

		Soit $i \in \left\llbracket 1,m \right\rrbracket$. Soit $(x_1, \ldots, x_{n_i}) \in Y^{n_i}$\/ et soit $a \in A_i$. Montrons que $f_i(a, x_1, \ldots, x_{n_i}) \in Y$.
		Or, soit $X \in \mathcal{A}$, on a $(x_1, \ldots, x_{n_i}) \in X^{n_i}$\/ donc $f_i(a, x_1, \ldots, x_{n_i}) \in X$. Ceci étant vrai pour tout $X \in \mathcal{A}$, on a $f_i(a, x_1, \ldots, x_{n_i}) \in Y$\/ donc $Y$\/ est stable par $f_i$\/ par tout $i \in \left\llbracket 1,m \right\rrbracket$\/ et donc $Y \in \mathcal{A}$.
		On a également $Y \subseteq X$\/ pour tout $X \in \mathcal{A}$. On en déduit que $Y$\/ est le plus petit élément (pour l'inclusion) de $\mathcal{A}$.
	\item On pose $X_0  = \mathcal{B}$\/ et \[
			X_{n+1} = X_n \cup \big\{ f_i(a, x_1, \ldots, x_{n_i})  \mid a \in A_i,\,(x_1, \ldots, x_{n_i}) \in (X_n)^{n_i},\,i \in \left\llbracket 1,m \right\rrbracket\big\}
		.\]
		Soit $X = \bigcup_{n \in \N} X_n$. Soit $Y$\/ l'ensemble défini par induction à partir de $\mathcal{B}$\/ et des $(f_i)_{i\in\left\llbracket 1,n \right\rrbracket}$. Montrons que $X = Y$.
		On montre que $X$\/ est le plus petit élément (pour l'inclusion) de $\mathcal{A}$\/ et on conclut par unicité du minimum (avec la question précédente).
		Par définition de la suite $(X_n)_{n\in\N}$, elle est croissante (au sens de l'inclusion).
		Montrons à présent, par récurrence, la propriété ci-dessous : $P_n : ``X_n \subseteq Y."$
		\begin{itemize}
			\item Par définition de $Y$, on a $X_0 = \mathcal{B} \subseteq Y$.
			\item Soit 
		\end{itemize}
\end{enumerate}

\subsection{Un théorème d'induction}

\begin{enumerate}
	\item[3.] Soit $Z = \{x \in S  \mid P(x) \text{ vraie}\:\}$.
		Montrons que $\mathcal{X}\subseteq Z$.
		On remarque que $\mathcal{X} \supseteq \mathcal{B}$\/ ; $\mathcal{X}$\/ est stable par $f_i$. On en conclut que $Z \supseteq \mathcal{X}$ et donc $\forall x \in \mathcal{X},\,P(x)$\/ est vraie.
\end{enumerate}

\begin{exm}
	Soit $\mathcal{X}$\/ défini par induction par $\mathcal{B} = \{0, 2\}$\/ et \begin{align*}
		f: \N &\longrightarrow \N \\
		n &\longmapsto n + 2.
	\end{align*}
	Montrons que $\forall n \in \mathcal{X}$, $x$\/ est pair.

	On sait que $0$\/ est pair, $2$\/ est pair ; et, \[
		\forall x,y \in \mathcal{X},\, (x \text{ pair} \land y \text{ pair}) \implies f(x, y) \text{ pair}
	.\]

	On en déduit que \[
		\forall n \in \mathcal{X},\,x \text{ est pair}.
	.\]
\end{exm}
\end{comment}

	}
	\def\addmacros#1{#1}
}
\def\prefix{\textsc{td bonus}}
{
	\td[3]{Invariants plus complexes}
	\minitoc
	\renewcommand{\cwd}{../td/td_b3/}
	\addmacros{
	}
	\def\addmacros#1{#1}
}


\part{Travaux Pratiques}
\def\prefix{\textsc{tp}}
\renewcommand{\chaptername}{Travaux pratiques}


{
	\tp[1]{Logique propositionnelle}
	\minitoc
	\renewcommand{\cwd}{../tps/tp01/}
	\addmacros{
		\section{Quelques problèmes décidables}

\begin{enumerate}
	\item Soit $f : \R \to \R$.
		\begin{itemize}
			\item Si $f$\/ admet un zéro, on pose $\mathcal{M} = \texttt{fun}\ \texttt{s}\ \to \texttt{true}$.
			\item Si $f$\/ n'admet pas un zéro, on pose $\mathcal{M} = \texttt{fun}\ \texttt{s}\ \to \texttt{false}$.
		\end{itemize}
		Alors, $\mathcal{M}$\/ décide \textsc{Zero}$_f$.
	\item Soit $\mathcal{M}$\/ une machine, et soit $w \in \Sigma^*$.
		\begin{itemize}
			\item Si $\mathcal{M}$\/ se termine sur l'entrée $w$, alors on pose $\mathcal{M}' = \texttt{fun}\ \texttt{s} \to \texttt{true}$.
			\item Si $\mathcal{M}$\/ ne se termine pas sur l'entrée $w$, alors on pose $\mathcal{M}' = \texttt{fun}\ \texttt{s} \to \texttt{false}$.
		\end{itemize}
		Alors, $\mathcal{M}'$\/ décide \textsc{Arrêt}$_{\mathcal{M},w}$.
	\item Le problème est trivialement vrai. En effet, soit $M \in \mathcal{O}$, de la forme
		\begin{lstlisting}[language=caml]
let m (s: string): string =
	%*$\langle$\textrm{code}$\rangle$*) 
		\end{lstlisting}
		On crée la machine $\mathcal{N}$\/ ci-dessous.
		\begin{lstlisting}[language=caml]
let n (s: string): string =
	if true then
		%*$\langle$\textrm{code}$\rangle$*) 
	else
		%*$\langle$\textrm{code}$\rangle$*) 
		\end{lstlisting}
		On a $\texttt{m} \neq \texttt{n}$, mais $\mathcal{L}(\texttt{m}) = \mathcal{L}(\texttt{n})$, donc le problème est vrai sur toute entrée et la fonction $\texttt{fun}\ \texttt{s} \to \texttt{true}$\/ répond au problème.
\end{enumerate}

		\section{Ensembles définis inductivement}

La correction est disponible sur \textit{cahier-de-prepa}.

\begin{comment}
	\begin{exm}
		Avec $S = \N$, $\mathcal{B} = \{0, 2\} $, $A_1 = \{0\}$\/ et \begin{align*}
			f_1: A_1 \times \N &\longrightarrow \N \\
			(0, x) &\longmapsto x + 4.
		\end{align*}

		On a \[
			X \supseteq \{0, 2, 4, 6, 8, 10, \ldots, 20, \ldots\} = 2\N
		.\]
	\end{exm}
	\begin{exm}
		Avec $S$\/ l'ensemble des langages sur $\Sigma$, $\mathcal{B} = \{\O\} \cup \bigl\{\{a\}\:\big|\: a \in \Sigma \bigr\}$, et
		\begin{multicols}{3}
			\begin{align*}
				f_1: S \times S &\longrightarrow S \\
				(L_1, L_2) &\longmapsto L_1 \cup L_2,
			\end{align*}
			\begin{align*}
				f_2: S \times S &\longrightarrow S \\
				(L_1, L_2) &\longmapsto L_1 \cdot L_2,
			\end{align*}
			\begin{align*}
				f_3: S &\longrightarrow S \\
				L &\longmapsto L^*.
			\end{align*}
		\end{multicols}
	\end{exm}

\begin{enumerate}
	\item Soit $\mathcal{A} = \{X \subseteq S  \mid X \supseteq \mathcal{B} \mathrel{\text{et}} X \text{ est stable par } f_i\}$. On a $S \in \mathcal{A}$\/ et donc $\mathcal{A} \neq \O$. De plus, soit \[
			Y = \{x \in S  \mid \forall X \in \mathcal{A},\,x \in X\} = \bigcap_{X \in \mathcal{A}} X
		.\]
		Soit $b \in \mathcal{B}$, on a $\forall X \in A,\, b \in X$. D'où $b \in Y$\/ par intersection. On en déduit que $\mathcal{B} \subseteq Y$.

		Soit $i \in \left\llbracket 1,m \right\rrbracket$. Soit $(x_1, \ldots, x_{n_i}) \in Y^{n_i}$\/ et soit $a \in A_i$. Montrons que $f_i(a, x_1, \ldots, x_{n_i}) \in Y$.
		Or, soit $X \in \mathcal{A}$, on a $(x_1, \ldots, x_{n_i}) \in X^{n_i}$\/ donc $f_i(a, x_1, \ldots, x_{n_i}) \in X$. Ceci étant vrai pour tout $X \in \mathcal{A}$, on a $f_i(a, x_1, \ldots, x_{n_i}) \in Y$\/ donc $Y$\/ est stable par $f_i$\/ par tout $i \in \left\llbracket 1,m \right\rrbracket$\/ et donc $Y \in \mathcal{A}$.
		On a également $Y \subseteq X$\/ pour tout $X \in \mathcal{A}$. On en déduit que $Y$\/ est le plus petit élément (pour l'inclusion) de $\mathcal{A}$.
	\item On pose $X_0  = \mathcal{B}$\/ et \[
			X_{n+1} = X_n \cup \big\{ f_i(a, x_1, \ldots, x_{n_i})  \mid a \in A_i,\,(x_1, \ldots, x_{n_i}) \in (X_n)^{n_i},\,i \in \left\llbracket 1,m \right\rrbracket\big\}
		.\]
		Soit $X = \bigcup_{n \in \N} X_n$. Soit $Y$\/ l'ensemble défini par induction à partir de $\mathcal{B}$\/ et des $(f_i)_{i\in\left\llbracket 1,n \right\rrbracket}$. Montrons que $X = Y$.
		On montre que $X$\/ est le plus petit élément (pour l'inclusion) de $\mathcal{A}$\/ et on conclut par unicité du minimum (avec la question précédente).
		Par définition de la suite $(X_n)_{n\in\N}$, elle est croissante (au sens de l'inclusion).
		Montrons à présent, par récurrence, la propriété ci-dessous : $P_n : ``X_n \subseteq Y."$
		\begin{itemize}
			\item Par définition de $Y$, on a $X_0 = \mathcal{B} \subseteq Y$.
			\item Soit 
		\end{itemize}
\end{enumerate}

\subsection{Un théorème d'induction}

\begin{enumerate}
	\item[3.] Soit $Z = \{x \in S  \mid P(x) \text{ vraie}\:\}$.
		Montrons que $\mathcal{X}\subseteq Z$.
		On remarque que $\mathcal{X} \supseteq \mathcal{B}$\/ ; $\mathcal{X}$\/ est stable par $f_i$. On en conclut que $Z \supseteq \mathcal{X}$ et donc $\forall x \in \mathcal{X},\,P(x)$\/ est vraie.
\end{enumerate}

\begin{exm}
	Soit $\mathcal{X}$\/ défini par induction par $\mathcal{B} = \{0, 2\}$\/ et \begin{align*}
		f: \N &\longrightarrow \N \\
		n &\longmapsto n + 2.
	\end{align*}
	Montrons que $\forall n \in \mathcal{X}$, $x$\/ est pair.

	On sait que $0$\/ est pair, $2$\/ est pair ; et, \[
		\forall x,y \in \mathcal{X},\, (x \text{ pair} \land y \text{ pair}) \implies f(x, y) \text{ pair}
	.\]

	On en déduit que \[
		\forall n \in \mathcal{X},\,x \text{ est pair}.
	.\]
\end{exm}
\end{comment}

		\section{Tableaux dynamiques}

\begin{enumerate}[start=3]
	\item On trouve une complexité amortie en $n^2$. À rédiger.
	\item Au lieu de diviser quand $r < n / 2$, mais quand $r < n / 4$.
\end{enumerate}

	}
	\def\addmacros#1{#1}
}
{
	% No `main.tex' file for tp02
	\def\addmacros#1{#1}
}
{
	\tp[3]{Langages et expressions régulières (2)}
	\minitoc
	\renewcommand{\cwd}{../tps/tp03/}
	\addmacros{
	}
	\def\addmacros#1{#1}
}
{
	% No `main.tex' file for tp04
	\def\addmacros#1{#1}
}
{
	% No `main.tex' file for tp05
	\def\addmacros#1{#1}
}
{
	% No `main.tex' file for tp06
	\def\addmacros#1{#1}
}
{
	\tp[7]{Algorithme de Kosaraju en \textsc{OCaml}}
	\minitoc
	\renewcommand{\cwd}{../tps/tp07/}
	\addmacros{
		\section{Quelques problèmes décidables}

\begin{enumerate}
	\item Soit $f : \R \to \R$.
		\begin{itemize}
			\item Si $f$\/ admet un zéro, on pose $\mathcal{M} = \texttt{fun}\ \texttt{s}\ \to \texttt{true}$.
			\item Si $f$\/ n'admet pas un zéro, on pose $\mathcal{M} = \texttt{fun}\ \texttt{s}\ \to \texttt{false}$.
		\end{itemize}
		Alors, $\mathcal{M}$\/ décide \textsc{Zero}$_f$.
	\item Soit $\mathcal{M}$\/ une machine, et soit $w \in \Sigma^*$.
		\begin{itemize}
			\item Si $\mathcal{M}$\/ se termine sur l'entrée $w$, alors on pose $\mathcal{M}' = \texttt{fun}\ \texttt{s} \to \texttt{true}$.
			\item Si $\mathcal{M}$\/ ne se termine pas sur l'entrée $w$, alors on pose $\mathcal{M}' = \texttt{fun}\ \texttt{s} \to \texttt{false}$.
		\end{itemize}
		Alors, $\mathcal{M}'$\/ décide \textsc{Arrêt}$_{\mathcal{M},w}$.
	\item Le problème est trivialement vrai. En effet, soit $M \in \mathcal{O}$, de la forme
		\begin{lstlisting}[language=caml]
let m (s: string): string =
	%*$\langle$\textrm{code}$\rangle$*) 
		\end{lstlisting}
		On crée la machine $\mathcal{N}$\/ ci-dessous.
		\begin{lstlisting}[language=caml]
let n (s: string): string =
	if true then
		%*$\langle$\textrm{code}$\rangle$*) 
	else
		%*$\langle$\textrm{code}$\rangle$*) 
		\end{lstlisting}
		On a $\texttt{m} \neq \texttt{n}$, mais $\mathcal{L}(\texttt{m}) = \mathcal{L}(\texttt{n})$, donc le problème est vrai sur toute entrée et la fonction $\texttt{fun}\ \texttt{s} \to \texttt{true}$\/ répond au problème.
\end{enumerate}

	}
	\def\addmacros#1{#1}
}
{
	% No `main.tex' file for tp08
	\def\addmacros#1{#1}
}
{
	% No `main.tex' file for tp09
	\def\addmacros#1{#1}
}
{
	% No `main.tex' file for tp10
	\def\addmacros#1{#1}
}
\def\prefix{\textsc{tp bonus}}
{
	% No `main.tex' file for tp_b1
	\def\addmacros#1{#1}
}
\def\prefix{\textsc{tp bonus}}
{
	% No `main.tex' file for tp_b2
	\def\addmacros#1{#1}
}


\part{Annexes}
\def\prefix{Annexe}
\renewcommand{\chaptername}{Annexe}
\useroman


{
	\chap[1]{Complexité amortie}
	\minitoc
	\renewcommand{\cwd}{../cours/annexeA/}
	\addmacros{
		Avec une fonction $\mathrm{PA}(n)$\/ ayant une complexité en $\mathcal{O}(f(n))$, on considère le problème ci-dessous.
		\begin{lstlisting}[language=caml]
			for i = 0 to n - 1 do
				%*$\mathrm{PA}$*)(i)
			done
		\end{lstlisting}
		Cet algorithme a une complexité en $\mathcal{O}(n f(n))$. Mais, parfois, cette complexité est trop approximative : parfois, des sommes mathématiques se compensent : \[
			\sum_{i=0}^{n-1} 2^i = 2^n \neq \cancel{\Theta}(n 2^n)
		.\]
	
		On considère le problème ci-dessous.
		\begin{lstlisting}[language=c]
			for (int i = 0; i < n; i = i + 1) (
				// calcul qui ne coute pas cher
			}
		\end{lstlisting}
		Le calcul \texttt{i = i + 1} est parfois plus coûteux que le calcul dans la boucle.
		Par exemple, un algorithme permettant de faire ce calcul est celui ci-dessous.
		\begin{algorithm}[H]
			\centering
			\begin{algorithmic}[1]
				\Entree un entier $n$, représenté sous la forme d'un tableau de \textit{bit} $T$\/ (où les \textit{bit}s de poids forts sont à droite
				\State $I \gets \mathrm{len}(T) - 1$\/ 
				\While{$T[I] = 1$}
				\State $T[I] \gets 0$
				\State $I \gets I - 1$
				\EndWhile
				\State $T[I] \gets 1$
			\end{algorithmic}
			\caption{Calcul de $n + 1$\/ avec un tableau de \textit{bit}s}
		\end{algorithm}
		Avec un tel algorithme, on a une complexité, dans le pire des cas, en $\Theta(\log_2 n)$.
		Ainsi, en modifiant le code, on peut avoir une complexité importante.
		\begin{lstlisting}[language=c]
			// n est une valeur donnee par l'utilisateur
			for (int i = 0; i < 2%*$^\texttt{n}$*); i = i + 1) (
				// calcul qui ne coute pas cher
			}
		\end{lstlisting}
		La complexité de cet algorithme, que l'on nommera $\mathcal{A}$\/ dans la suite,  est en $\mathcal{O}(\texttt{n}\:2^\texttt{n})$, car le \texttt{i = i + 1} coûte, au pire des cas, $\texttt{n}$.
		En réalisant des mesures, et en graphant le temps de cet algorithme divisé par $2^\texttt{n}$, on remarque que ce le ratio n'est pas une droite de coefficient directeur $\texttt{n}$, mais une constante (à partir d'un certain rang).
		On doit donc faire une étude plus précise de la complexité, et faire le calcul de la somme plus proprement.
		Une étude plus fine nous montre que l'algorithme est beaucoup plus long pour les entiers en plaisances de deux, mais les autres nombres, on n'a pas besoin d'autant de calcul.
		Faisons cette étude plus fine.
	
		On nomme $\mathcal{T}$\/ l'ensemble des tableaux de taille $n$\/ contenant des valeurs dans $\{0,1\}$. On a, \[
			\mathrm{Co\hat ut}_\mathcal{A}(n) = \sum_{t \in \mathcal{T}} \mathrm{Co\hat ut}_\text{Incr}(t)
		.\]Partitionnons $\mathcal{T}$\/ : \[
			\mathcal{T}_i = \Bigg\{
				\begin{array}{|c|c|c|c|c|c|}
					\hline
					\ldots & i + 1 & i & i - 1 & \ldots & 0\\ \hline
					\ldots & \mathbf{0} & \mathbf{1} & \mathbf{1} & \ldots & \mathbf{1}\\ \hline
				\end{array} \in \mathcal{T}
			\Bigg\}
		.\]  Si $t \in \mathcal{T}_i$, on sait que $\mathrm{Co\hat ut}_\text{Incr}(t) = i + 1$.
		Ainsi,
		\begin{align*}
			\sum_{t \in \mathcal{T}} \mathrm{Co\hat ut}_\text{Incr}(t)
			&= \sum_{i=0}^{n-1} \sum_{t \in \mathcal{T}_i} \mathrm{Co\hat ut}_\text{Incr}(t) \\
			&= \sum_{i=0}^{n-1} |\mathcal{T}_i| \cdot (i + 1) \\
			&= \sum_{i=0}^{n-1} 2^{n - 1 - i}\:(i+1) \\
			&= 2^n \sum_{i=1}^n \frac{i}{2^i}\\
			&= 2^n \times \mathcal{O}(1) \\
			&= \mathcal{O}(2^n) \\
		\end{align*}
		ce qui explique les résultats trouvés précédemment.
		L'incrementation \texttt{i = i + 1} est donc en $\mathcal{O}(1)$, et non en $\mathcal{O}(\log_2 \texttt{i})$.
	
		\begin{defn}
			Étant donnée une structure (des éléments d'un type de données abstrait,~\textsc{tda}), $\mathds{F}$, munie d'opérations $\mathds{O}$ opérant sur $\mathds{F}$\/ munis de fonctions de coût \[
				\forall o \in \mathds{O}, \quad C_o : \mathds{F} \to \R^+
			.\]
			Étant donné un élément initial $f_0 \in \mathds{F}$\/ et une suite d'opérations $(o_1, \ldots, o_n) \in \mathds{O}^n$, cela conduit donc à une suite d'éléments \[
				f_0 \overset{o_1}\leadsto f_1 \overset{o_2}\leadsto f_2 \leadsto \cdots \leadsto f_n.
			\] On appelle \textit{complexité} de cette séquence, notée $\tilde c$, \[
				\tilde C\big((o_1, \ldots, o_n), f_0\big)  = \sum_{i=0}^n C_{o_i}(f_{i-1})
			.\]
			On appelle alors \textit{complexité amortie} depuis $f_0 \in \mathds{F}$\/ la suite \[
				C_\mathrm{A}(f_0, n) = \frac{1}{n} \sup_{(o_1, \ldots, o_n) \in \mathds{O}^n} \tilde C\big((o_1, \ldots, o_n), f_0\big)
			.\]
		\end{defn}
	
		\begin{exm}[Tableaux dynamiques]
			On s'intéresse aux tableaux à longueur variable : on alloue un tableau de petite taille, et on alloue plus de mémoire au besoin.
			On a une structure de tableau dynamique :
			\begin{algorithm}[H]
				\centering
				\begin{algorithmic}[1]
					\State Soit $\mathrm{taille}' = f\big(\mathrm{len}(T)\big)$ \Comment{$f$ reste à déterminer}
					\State On alloue $T'$\/ de taille $\mathrm{taille}'$
					\State On recopie $T$\/ dans $T'$\/
					\State $T \gets T'$\/
				\end{algorithmic}
				\caption{$\textsc{Agrandit}(T)$, fonction agrandissant le tableau $t$\/}
			\end{algorithm}
			Cet algorithme a une complexité $\mathrm{Co\hat ut}_\textsc{Agrandit}(n) = n + f(n)$.
			On suppose que le tableau $T$\/ est rempli jusqu'à la $r$-ième case.
			\begin{algorithm}[H]
				\centering
				\begin{algorithmic}[1]
					\If{$\mathrm{len}(T) = r$}
					\State $\textsc{Agrandit}(T)$
					\EndIf
					\State $T[r] \gets x$
					\State $r \gets r + 1$
				\end{algorithmic}
				\caption{$\textsc{Ajout}(T,x)$, ajout d'un élément dans le tableau}
			\end{algorithm}
			On choisit la fonction $f$.
			\begin{description}
				\item[Cas 1] On choisit $f(n) = n + 1$. Soit une suite de $n$\/ opérations \textsc{Ajout} depuis un tableau de taille 1, où $r = 0$. Ainsi, \[
						f_0\overset{\textsc{Ajout}}\leadsto f_1 \overset{\textsc{Ajout}}\leadsto \cdots \leadsto f_i \leadsto \cdots \leadsto f_{n-1} \overset{\textsc{Ajout}}\leadsto f_n
					.\] La complexité de cette suite d'opérations est \[
						\tilde C\big((o_1, \ldots, o_n), f_0) = n + 2 \cdot \frac{n(n+1)}{2},
					\] d'où la complexité amortie est de $C_\mathrm{A}(f_0, n) = \Theta(n)$.
				\item[Cas 2] On choisit $f(n) = 2n$. On somme les complexités : $2n$\/ (clairement par dessin). Ainsi, $C_\mathrm{A}(f_0, n) = \Theta(1)$.
			\end{description}
		\end{exm}
	
		\begin{met}[du potentiel]
			Considérons une fonction $h : \mathds{F} \to \R^+$\/ dite \textit{de potentiel} telle que $h(f_0) = 0$.
			Intéressons nous alors à $\ubar{C}_o(f) = C_o(f) + h(\bar{f}) - h(f)$, où $f \overset o\leadsto \bar{f}$.
			Soit alors \[
				f_0 \overset{o_1}\leadsto f_1 \overset{o_1}\leadsto f_2 \leadsto \cdots \leadsto f_n
			\]une suite d'opérations. Alors,
			\begin{align*}
				\sum_{i=1}^n \ubar{C}_{o_i}(f_{i-1}) &= \sum_{i=1}^n \Big(C_{o_i}(f_{i-1}) + h(f_i) - h(f_{i-1})\Big)\\
				&= \bigg(\sum_{i=1}^n C_{o_i}(f_{i-1})\bigg) + \underbrace{h(f_n) - h(f_0)}_{\ge  0} \\
			\end{align*}
			par télescopage. Ainsi, \[
				\sum_{i=1}^n C_{o_i}(f_{i-1}) \le \sum_{i=1}^n \ubar{C}_{o_i}(f_{i-1})
			.\]
		\end{met}
	
		\begin{exm}
			On applique la méthode du potentiel au cas 2 de l'exemple ci-avant.
			On rappelle que $\mathds{F}$\/ est l'ensemble des tableaux. On pose la fonction \begin{align*}
				h: \mathds{F} &\longrightarrow \R^+ \\
				(T, r) &\longmapsto 6\left(r - \frac{\mathrm{len}(T)}{2}\right)
			\end{align*}
			Inspectons alors \[
				\ubar{C}_\textsc{Ajout}(T, r) = C_\textsc{Ajout}(\ubar{T}, \ubar{r}) + 6\bar{r} - 3\:\mathrm{len}(\bar{T}) - 3\ubar{r} + 3\:\mathrm{len}(\ubar{T})
			.\]
			Si $\mathrm{len}(\ubar{T}) = \ubar{r}$, alors $C_\textsc{Ajout}(\ubar{T}, \ubar{r}) = 3\:\mathrm{len}(\ubar{T})$\/ et $\mathrm{len}(\bar{T}) = 2\: \mathrm{len}(\ubar{T})$\/ et $\bar{r} = \ubar{r} + 1$.
			D'où, \[
				\ubar{C}_\textsc{Ajout}(\ubar{T}, \ubar{r}) = 3\:\mathrm{len}(\ubar{T}) + 6\ubar{r} + 6 - 6\:\mathrm{len}(\ubar{T}) - 6\ubar{r} + 3\:\mathrm{len}(\ubar{T}) = 6
			.\] 
			Sinon, $\mathrm{len}(\ubar{T}) > \ubar{r}$, alors $\mathrm{len}(\bar{T}) = \mathrm{len}(\ubar{T})$\/ et $\bar{r} = \ubar{r} + 1$.
			Ainsi,
			\begin{align*}
				\ubar{C}_\textsc{Ajout}(\ubar{T}, \ubar{r}) = 1 + 6 (\ubar{r} + 1) - 6\:\mathrm{len}(\bar{T}) - 6 \ubar{r} + 6\:\mathrm{len}(\ubar{T}) = 7
			\end{align*}
			D'où \[
				\sum_{i=1}^n C_{o_i}(f_{i-1}) \le \sum_{i=1}^n \ubar{C}_{o_i}(f_{i-1}) \le 7n
			.\] Le coût amorti est en $\mathcal{O}(1)$.
		\end{exm}
		\begin{exm}[Méthode du Banquier]
			On encode une file avec deux piles.
			Au moment de défiler, on doit potentiellement transvaser une pile dans une autre.
			Avec la méthode du Banquier, on a l'\textit{intuition} que le coût amorti est constant.
	
			On pose $\mathds{F}$\/ l'ensemble des couples de piles $(p_1, p_2)$.
			On a \[
				C_\text{défiler}\big((p_1, p_2)\big) = \begin{cases}
					\mathrm{taille}\ p_1 + 1 \quad& \text{si $p_2$ est vide}\\
					1 \quad& \text{ sinon}
				\end{cases}, \text{ et } C_\text{enfiler}\big((p_1, p_2)\big) = 1
			.\] Soit $h$\/ la fonction de potentiel définie comme \begin{align*}
				h: \mathds{F} &\longrightarrow \R^+ \\
				(p_1, p_2) &\longmapsto \mathrm{taille}\ p_1
			\end{align*}
			Étudions alors $\ubar{C}_\text{défiler}\big((p_1, p_2)\big)$.
			\begin{itemize}
				\item Si $p_2$\/ est vide, alors
					\begin{align*}
						\ubar{C}_\text{défiler}\big((p_1, p_2)\big) &= C_\text{défiler}\big((p_1,p_2)\big) + h\big((\bar{p}_1, \bar{p}_2)\big) - h\big((p_1, p_2)\big)\\
						&= \mathrm{taille} \ p_1 + 1 + \overbrace{\mathrm{taille}\ \bar{p}_1}^{=0} - \mathrm{taille}\ p_1 \\
						&= 1.
					\end{align*}
				\item Si $p_2$\/ n'est pas vide, alors \[
						\ubar{C}_\text{défiler}\big((p_1, p_2)\big) = 1 + \underbrace{\mathrm{taille}\  \bar{p}_1}_{\substack{\ds=\\ \ds \mathrm{taille}\ p_1}} - \mathrm{taille}\ p_1
					.\]
			\end{itemize}
			D'où, pour $(p_1, p_2) \in \mathds{F}$, $\ubar{C}_ \text{défiler}\big((p_1, p_2)\big) \le 1$.
			De plus,
			\begin{align*}
				\ubar{C}_ \text{ enfiler}\big((p_1, p_2)\big)
				&= C_ \text{enfiler}\big((p_1, p_2)\big) + h\big((\bar{p}_1, \bar{p}_2)\big) - h\big((p_1, p_2)\big) \\
				&= 1 + \mathrm{taille}\ \bar{p}_1 - \mathrm{taille}\ p_1 \\
				&= 2 \\
			\end{align*}
			Finalement, pour toute séquence d'opérations $o_1, \ldots, o_n$\/ initialisée à la file vide $f_0$, on a
			\begin{align*}
				\frac{1}{n} \tilde C\big((o_1,\ldots,o_n), f_0\big)
				&= \frac{1}{n} \sum_{i=0}^n C_{o_i}(f_{i-1}) \\
				&\le \frac{1}{n} \sum_{i=1}^n \ubar{C}_{o_i}(f_{i-1}) \\
				&\le 2
			\end{align*}
			D'où, un coût amorti constant.
		\end{exm}
	}
	\def\addmacros#1{#1}
}
{
	\chap[2]{Algorithmes \textsc{Dijkstra} et $A^*$}
	\minitoc
	\renewcommand{\cwd}{../cours/annexeB/}
	\addmacros{
		On s'intéresse, dans cette annexe, à l'algorithme $A^*$.
		Cette annexe se situe à l'intersection des chapitres sur les graphes, et sur les jeux.
		L'algorithme $A^*$ est une modification de l'algorithme de \textsc{Dijkstra}.
		Dans cette annexe, on prouvera la correction de l'algorithme $A^*$.
	
		On se place dans le contexte d'exécution d'un algorithme de calcul de plus cours chemin utilisant un tableau de distances $\mu$, et le manipulant en n'effectuant que des opérations \textsc{Relâcher}.
		Notons le graphe $G = (V,E)$, le sommet source $s$.
		Notons également $d(\cdot,\cdot)$ la distance induite par les arêtes du graphe $G$.
		De plus, on notera $c(\cdot,\cdot)$ les coûts (positifs, non nuls) d'une arête de $G$.
		Notons $\ell(\cdot)$ les rongeurs des chemins.
	
		\begin{numlem}
			\[
				\forall (u,v) \in E,\quad d(s,v) \le d(s, u) + c(u,v)
			.\]
		\end{numlem}
	
		\begin{prv}
			Soit $(u,v) \in E$.
			Soit $\gamma_u$ un plus court chemin de $s$ à $u$.
			Alors, $\gamma_u \cdot v$ est un chemin de $s$ à $v$ :
			\[
				\ell(\gamma_u \cdot v) = \ell(\gamma_u) + c(u,v) = d(s,u) + c(u,v) \ge d(s,v).
			\]
		\end{prv}
	
		\begin{numlem}
			Pour tout sommet $u$, la valeur de $\mu[u]$ est décroissant à mesure que l'algorithme s'exécute.
		\end{numlem}
	
		\begin{prv}
			Soit $\ubar{\mu}$ et $\bar\mu$ les valeurs de $\mu$ avant et après une opération $\textsc{Relâcher}(x,y)$.
			Pour tout sommet $v \neq y$, $\bar\mu[v] = \ubar\mu[v]$.
			De plus, par disjonction de cas,
			\begin{itemize}
				\item ou bien $\bar\mu[y] = \ubar\mu[y]$, \textsc{ok}.
				\item ou bien $\bar\mu[y] = \ubar\mu[x] + c(x,y)$ lorsque $\ubar\mu[x] + c(x,y) \le \ubar\mu[y]$, donc $\bar\mu[y] \le \ubar\mu[y]$, \textsc{ok}.
			\end{itemize}
		\end{prv}
	
		\begin{numlem}
			Supposons que l'algorithme ait initialisé $\mu$ de la manière suivante : \[
				\forall u \in V,\quad\quad \mu[u] = \begin{cases}
					+\infty & \text{ si } u \neq s\\
					0 & \text{ sinon}.
				\end{cases}
			\] Alors, tout au long de l'exécution de l'algorithme, pour tout sommet $u$, $\mu[u] \ge d(s,u)$.
		\end{numlem}
	
		\begin{prv}
			\begin{description}
				\item[Initialement] La propriété est vraie par hypothèse.
				\item[Hérédité] Supposons vrai jusqu'à un certain état $\ubar\mu$, pour une opération $\textsc{Relâcher}(x,y)$.
					Pour tout sommet $v \neq y$, $\ubar\mu[v] = \bar\mu[v] \ge d(s,v)$.
					De plus, par disjonction de cas,
					\begin{itemize}
						\item si $\bar\mu[y] = \ubar\mu[y] \ge d(s,y)$ ;
						\item sinon si $\ubar\mu[y] = \ubar\mu[x] + c(x,y) \ge d(s,x) + c(x,y) \ge d(s,y)$ par hypothèse de récurrence, puis par lemme 1.
					\end{itemize}
			\end{description}
		\end{prv}
	
		\begin{crlr}
			Si \guillemotleft~à un moment~\guillemotright\ $\mu[u] = d(s,u)$, alors \guillemotleft~pour toujours après~\guillemotright\ $\mu[u] = d(s,u)$.
			\qed
		\end{crlr}
	
		\begin{numlem}
			Si $(s, \ldots, u, v)$ est un plus court chemin de $s$ à $v$ tel que $\ubar\mu[u] = d(s,u)$ \guillemotleft~à un certain moment de l'exécution de l'algorithme.~\guillemotright\@ Notons $\bar\mu$ obtenu par $\textsc{Relâcher}(u,v)$.
		\end{numlem}
	
		\begin{prv}
			On a \[
				\bar\mu = \begin{cases}
					\ubar\mu[v] & \text{ si } \ubar\mu[v] < \ubar\mu[u] + c(u,v)\\
					\ubar\mu[u] + c(u,v) &\text{ sinon}.
				\end{cases}
			\]Par disjonction de cas,
			\begin{itemize}
				\item si $\ubar\mu[v] < \ubar\mu[u] + c(u,v) = d(s, u) + c(u,v) = d(s,v)$, et donc, en utilisant le lemme 3, $\bar\mu[v] = \ubar\mu[v] = d(s,v)$.
				\item sinon, $\bar\mu[v] = \ubar\mu[u] + c(u,v) = d(u,v) + c(u, v) = d(s,v)$.
			\end{itemize}
		\end{prv}
	
		\begin{numlem}
			Soit $(s=x_0, x_1, x_2, \ldots, x_n)$ un plus court chemin. Si on effectue des opérations \hbox{$\textsc{Relâcher}(x_i, x_{i+1})$} dans l'ordre $0 \to n - 1$, possiblement entremêlés avec d'autres opérations $\textsc{Relâcher}$, alors pour tout $i \in \llbracket 0,n \rrbracket$, $\mu_{\text{final}}[x_i] = d(s, x_i)$.
		\end{numlem}
	
		\begin{prv}[par récurrence]
			\begin{itemize}
				\item Initialement, $\mu[x_0] = d(s, x_0) = d(s,s)$.
				\item Et, pour tout les $i$ inférieurs stricts, $\mu[x_i] = d(s, x_i)$, on conclut par le lemme 4.
			\end{itemize}
		\end{prv}
	
		(De ce lemme découle l'algorithme de \textsc{Bellman-Ford}.)
	
		\begin{crlr}
			L'algorithme \textsc{Dijkstra} est correct.
		\end{crlr}
	
		\begin{prv}
			Soit $t \in V$, un sommet du graphe. Soit $(s = x_0, x_1, \ldots, x_{p-1}, x_p = t)$ un plus court chemin de $s$ à $t$. Montrons que $\mu_{\text{final}}[t] = d(s, t)$.
			En utilisant le lemme 5, il suffit de montrer que \textsc{Dijkstra} relâche les arêtes dans cet ordre.
			Supposons les sommets extraits $\mathrm{todo}$ dans l'ordre $x_0, \ldots, x_i$, pour $i \in \llbracket 0,p-1 \rrbracket$.
			Par l'absurde, supposons que \textsc{Dijkstra} sorte $x_k$ de $\mathrm{todo}$ pour $k \in \llbracket i+2, p \rrbracket$.
			\guillemotleft~À ce moment là,~\guillemotright\ on a \[
				d(s, x_k) \le \mu[x_k] \le \mu[x_{i+1}] \le d(s, x_{i+1}),
			\]d'après le lemme 5, ce qui est absurde ($k > i + 1$).
		\end{prv}
	
		\begin{crlr}
			L'algorithme $A^*$ est correct.
		\end{crlr}
	
		\begin{algorithm}[H]
			\centering
			\begin{algorithmic}[1]
				\Procedure{Relâcher}{$u,v$}
				\If{$\mu[v] > \mu[u] + c(u,v)$}
				\State $\mu[v] \gets \mu[u] + c(u,v)$
				\State $\pi[v] \gets u$
				\State $\eta[v] \gets \mu[v] + h(v)$
				\EndIf
				\EndProcedure
			\end{algorithmic}
			\caption{Algorithme $A^*$ (partiel)}
		\end{algorithm}
	
		\begin{prv}
			Par l'absurde, supposons que non.
			Soit $t \in V$, un sommet du graphe, tel que $\mu_{\text{final}}[t] \neq d(s,t)$.
			Donc $d = \mu_{\text{final}}[t] > d(s,t) = d^*$.
			Soit $(s = x_0, x_1, \ldots, x_{p-1}, x_p = t)$ un plus court chemin de $s$ à $t$ de longueur $d^*$.
			L'algorithme commence par visiter $x_0 = s$ et on relâche les arêtes sortantes.
			Alors, $\mu[x_i] = d(s, x_1)$ et $\eta[x_1] = \mu[x_1] + \mu[x_1] + h(x_1) \le d(s, x_1) + d(x_1, t) = d(s,t) = d^* < d$ par hypothèse.
			\guillemotleft~À ce state,~\guillemotright\ $\eta[t] = \mu[t] + h(t) \ge d + 0$.
			Ainsi, $x_1$ devrait être choisi avant $t$. À un tel moment, $\mu[x_1] = d(s, x_1)$, on relâche alors ses arêtes sortantes ; en particulier $x_1$ et $x_2$. Ceci assure alors que $\mu[x_2] = d(s, x_2)$, et $\eta[x_2] = \mu[x_2] + h(x_2) \le d(sn x_2) + d(x_2, t) \le d(s,t) = d^* < d$.
			\guillemotleft~De proche en proche,~\guillemotright\ alors que l'on choisit $x_{p-1}$ dans $\mathrm{todo}$, on a $\mu[x_{p-1}] = d(s, x_{p-1})$.
			On relâche alors $\mu[x_p] = d(s, x_p) = d^*$.
			Or, $d = \mu_{\text{final}}[t] \le \mu_{\text{à ce moment}}[t]$. Absurde.
		\end{prv}
	
		\begin{exm}[ré-entrée dans $\mathrm{todo}$]
			\begin{comment}
				     b (h = 6)
						/ \
				 1 /   \ 1
					/  3  \     5
				 s - - - a - - - - t
			        (h = 0)   (h = 0)
			\end{comment}
			Exécution de l'algorithme $A^*$ sur l'entrée ci-dessus.
			La pile $\mathrm{todo}$ est vaut donc $\cancel s, \cancel a, \cancel b, \cancel t, \cancel a$.
		\end{exm}
	}
	\def\addmacros#1{#1}
}
{
	\chap[3]{Diviser pour régner}
	\minitoc
	\renewcommand{\cwd}{../cours/annexeC/}
	\addmacros{
		\section{(Ne pas) être diagonalisable}

\begin{defn}
	Soit une matrice carrée $A$. On dit que $A$\/ est {\it diagonalisable}\/ s'il existe une matrice inversible~$P \in \mathrm{GL}_n(\mathds{K})$\/ telle que $P^{-1}\cdot A\cdot P$\/ est diagonale.
\end{defn}

\begin{exo}
	\begin{enumerate}
		\item Montrons que la matrice $B = {7\: 1\choose 0\:7}$\/ n'est pas diagonalisable.
			Par l'absurde : on suppose qu'il existe $P \in \mathrm{GL}_2(\R)$\/ et $(\lambda_1, \lambda_2) \in \R^2$\/ tels que \[
				P^{-1} \cdot B \cdot P = \begin{bmatrix}
					\lambda_1 & 0\\
					0&\lambda_2
				\end{bmatrix}
			.\] On applique la trace $\tr$\/ et le déterminant $\det$\/ :
			\begin{gather*}
				\tr(B) = \tr{\lambda_1\:0\choose 0\:\lambda_2} \quad\text{d'où}\quad \lambda_1 + \lambda_2 = 7 + 7 = 14 = \s\\
				\det(B) = \det{\lambda_1\:0\choose 0\:\lambda_2} \quad\text{d'où}\quad \lambda_1 \times \lambda_2 = 7 \times 7 = 49 = p
			\end{gather*}
			D'où $\lambda_1$\/ et $\lambda_2$\/ sont des solutions de l'équation $X^2 - \s X + p = 0$. Or
			\begin{align*}
				X^2 - \s X + p = 0 \iff& X^2 - 14X + 49 = 0\\
				\iff& (X-7)^2 = 0\\
				\iff& X = 7.
			\end{align*}
			D'où 
			\begin{align*}
				B = P P^{-1} B P P^{-1} = P \begin{pmatrix}
					7&0\\
					0&7
				\end{pmatrix} P^{-1} = P \cdot 7I_2\cdot P^{-1} = 7I_2.
			\end{align*}
			La matrice $B$\/ n'est donc pas diagonalisable.

			De même, montrons que la matrice $A$\/ n'est pas diagonalisable. On remarque que \[
				A \cdot \mat{1\\1\\1} = \begin{pmatrix}
					0&1&2\\
					1&0&2\\
					0&0&3
				\end{pmatrix} \begin{pmatrix}
					1\\1\\1
				\end{pmatrix} = \begin{pmatrix}
					3\\3\\3
				\end{pmatrix} = 3\begin{pmatrix}
					1\\1\\1
				\end{pmatrix} 
			.\] Ainsi, \[
				P^{-1}\cdot A\cdot P = \begin{pmatrix}
					3&0&0\\
					0&?&0\\
					0&0&?
				\end{pmatrix}\qquad\text{où}\qquad P = \begin{pmatrix}
					1&?&?\\
					1&?&?\\
					1&?&?
				\end{pmatrix}
			.\] De même, $A\left( \substack{1\\1\\0} \right) = 1 \times \left( \substack{1\\1\\0} \right)$. D'où \[
				P^{-1}\cdot A\cdot P = \begin{pmatrix}
					3&0&0\\
					0&1&0\\
					0&0&?
				\end{pmatrix}\qquad\text{où}\qquad P = \begin{pmatrix}
					1&1&?\\
					1&1&?\\
					1&0&?
				\end{pmatrix}
			.\] Finalement, on en conclut que \[
				P = \begin{pmatrix}
					3&0&0\\
					0&1&0\\
					0&0&-1
				\end{pmatrix} \qquad \text{et}\qquad P^{-1}\cdot A\cdot P = \begin{pmatrix}
					1&1&1\\
					1&1&-1\\
					1&0&0
				\end{pmatrix} = D
			.\]
			De plus, la matrice $P$\/ est inversible car $\det P \neq 0$.
		\item Pour calculer $A^n$, on pourrait chercher un polynôme annulateur $Q$\/ de $A$, et on exprime $X^n = Q \times T_n + R_n$, et donc $A^n = R_n(A)$.
			Mais, on peut également diagonaliser $A$\/ (si elle est diagonalisable).
			Ainsi,  \[
				D^n = (P^{-1}\cdot A\cdot P)^n = P^{-1}\cdot A\cdot \cancel P\cdot \cancel{P^{-1}} \cdot \ldots\cdot \cancel{P^{-1}} \cdot A \cdot P = P^{-1}\cdot  A^n\cdot P
			.\] D'où $A^n = P \cdot D^n \cdot P^{-1}$. Or, \[
				D^n = \begin{pmatrix}
					3&0&0\\
					0&1&0\\
					0&0&-1
				\end{pmatrix}^n = \begin{pmatrix}
					3^n&0&0\\
					0&1^n&0\\
					0&0&(-1)^n
				\end{pmatrix}
			.\]
			On calcule donc $A^{n}$\/ en calculant l'inverse de $P$\/ : \[
				A^n = \begin{pmatrix}
					1&1&1\\
					1&1&-1\\
					1&0&0
				\end{pmatrix} \begin{pmatrix}
					3^n&0&0\\
					0&1^n&0\\
					0&0&(-1)^n
				\end{pmatrix} \cdot P^{-1}
			.\]
		\item
			\begin{align*}
				\begin{rcases*}
					\hfill u_{n+1} = v_n + 2w_n\\
					\hfill v_{n+1} = u_n + 2w_n\\
					\hfill w_{n+1} = 3w_n
				\end{rcases*} \iff& \begin{pmatrix}
					u_{n+1}\\v_{n+1}\\w_{n+1}
				\end{pmatrix} = \begin{pmatrix}
					0&1&2\\
					1&0&2\\
					0&0&3
				\end{pmatrix} \begin{pmatrix}
					u_n\\ v_n\\ w_n
				\end{pmatrix}\\
				\iff& U_{n+1} = A\cdot U_n\\
				\iff& U'_{n+1} = D \cdot U'_{n}
			\end{align*}
			où $D = P^{-1} \cdot A \cdot P$, $U'_{n+1} = P\cdot U_{n+1}$\/ et $U'_n = P\cdot U_n$.
			\begin{align*}
				\phantom{\begin{rcases*}
					\hfill mm_{n+1} = v_n + 2w_n\\
					\hfill v_{n+1} = u_n + 2w_n\\
					\hfill w_{n+1} = 3w_n
				\end{rcases*}} \iff&
				\begin{pmatrix}
					u'_{n+1}\\v'_{n+1}\\w'_{n+1}
				\end{pmatrix} = \begin{pmatrix}
					3&0&0\\
					0&1&0\\
					0&0&-1
				\end{pmatrix} \cdot \begin{pmatrix}
					u'_n\\
					v'_n\\
					w'_n
				\end{pmatrix}\\
				\iff& \begin{cases}
					u'_{n+1} = 3u'_n\\
					v'_{n+1} = v'_n\\
					w'_{n+1} = -w'_n
				\end{cases}\\
				\iff& \begin{cases}
					u'_n = K\times  3^n\\
					v'_n = L\\
					w'_n = M \times (-1)^n
				\end{cases}
			\end{align*}
			Ainsi, \[
				\begin{pmatrix}
					u_n\\v_n\\w_n
				\end{pmatrix} = \underbrace{\begin{pmatrix}
					1&1&1\\
					1&1&-1\\
					1&0&0
				\end{pmatrix}}_P \cdot \begin{pmatrix}
					K\times 3^n\\
					L\\
					M\times (-1)^n
				\end{pmatrix}
			.\] D'où $u_n = K\cdot 3^n + L + M \cdot (-1)^n$, $v_n = K\times 3^n + L - M \cdot (-1)^n$\/ et $w_n = K\cdot 3^n$, où les constantes $K$, $L$\/ et $M$\/ sont des constantes fixées par les conditions initiales.
		\item
			\begin{align*}
				\begin{rcases*}
					\hfill x'(t) = y(t) + 2z(t)\\
					\hfill y'(t) = x(t) + 2z(t)\\
					\hfill z'(t) = 3z(t)
				\end{rcases*} \iff& \begin{pmatrix}
					x'(t)\\
					y'(t)\\
					z'(t)
				\end{pmatrix} = \begin{pmatrix}
					0&1&2\\
					1&0&2\\
					0&0&3
				\end{pmatrix} \cdot \begin{pmatrix}
					x(t)\\
					y(t)\\
					z(t)
				\end{pmatrix}\\
				\iff& X'(t) = A\cdot X(t)\\
				\iff& U'(t) = D \cdot U(t) \text{ avec } D = P^{-1} \cdot A\cdot P \text{ et } X(t) = P\cdot U(t)\\
				\iff& \begin{pmatrix}
					u'(t)\\
					v'(t)\\
					w'(t)
				\end{pmatrix} = \begin{pmatrix}
					3&0&0\\
					0&1&0\\
					0&0&-1
				\end{pmatrix} \cdot \begin{pmatrix}
					u(t)\\
					v(t)\\
					w(t)
				\end{pmatrix}\\
				\iff& \begin{cases}
					u'(t) = 3u(t)\\
					v'(t) = v(t)\\
					w'(t) = -w(t)
				\end{cases}\\
				\iff& \begin{cases}
					u(t) = K \cdot \mathrm{e}^{3t}\\
					v(t) = L \cdot \mathrm{e}^{t}\\
					w(t) = M \cdot \mathrm{e}^{-t}
				\end{cases}
			\end{align*}
			Ainsi \[
				\begin{pmatrix}
					x(t)\\
					y(t)\\
					z(t)
				\end{pmatrix} = \underbrace{\begin{pmatrix}
					1&1&1\\
					1&1&-1\\
					1&0&0
				\end{pmatrix}}_P \cdot \begin{pmatrix}
					K \times \mathrm{e}^{3t}\\
					L \cdot \mathrm{e}^{t}\\
					M \cdot \mathrm{e}^{-t}
				\end{pmatrix}
			.\] 
			D'où $x(t) = K\cdot \mathrm{e}^{3t} + L \cdot \mathrm{e}^{t} + M \cdot \mathrm{e}^{-t}$, $y(t) = K \cdot \mathrm{e}^{3t} + L \cdot \mathrm{e}^{t} - M \cdot \mathrm{e}^{-t}$\/ et $z(t) = K\cdot \mathrm{e}^{3t}$. Les constantes $K$, $L$\/ et $M$\/ peuvent être déterminées à partir des conditions initiales.
	\end{enumerate}
\end{exo}

\begin{rmkn}[équations différentielles]
	On considère l'équation différentielle $(*)$ : $x'(t) = \lambda \cdot x(t)$.
	Les fonctions $x : t \mapsto K\cdot \mathrm{e}^{\lambda t}$\/ sont des solutions de cette équation. On peut utiliser la méthode de {\sc Lagrange}\/ : la méthode de la~\guillemotleft~variation de la constante.~\guillemotright\@ On cherche des solutions sous la forme $x(t) = k(t) \cdot \mathrm{e}^{\lambda t}$ (vision du~physicien). D'où $k(t) = x(t) / \mathrm{e}^{\lambda t}$\/ (vision du mathématicien). De plus, $x'(t) = k'(t) \mathrm{e}^{\lambda t} + k(t) \lambda \mathrm{e}^{\lambda t}$.
	Ainsi, on injecte ce $k(t)$\/ dans l'équation différentielle :
	\begin{align*}
		(*) \iff& k'(t) \mathrm{e}^{\lambda t} + k(t) \lambda \mathrm{e}^{\lambda t} = \lambda k(t)\mathrm{e}^{\lambda t}\\
		\iff& k'(t) \mathrm{e}^{\lambda t} = 0\\
		\iff& k'(t) = 0\\
		\iff& \exists K \in \R\,\:k(t) = K.
	\end{align*}
	Les solutions trouvées dans l'exercice précédent sont donc les uniques solutions du système d'équations différentielles.

	De même, pour résoudre une équation différentielle avec 2\tsup{nd} membre de la forme \[
		(**) : \qquad x'(t) - \lambda \cdot x(t) = b(t)
	.\]
	La fonction $t \mapsto x(t)$\/ est une solution de l'équation {\sc sans}\/ 2\tsup{nd} membre si et seulement si \[
		\exists K \in \R,\:\forall t \in \R,\quad x(t) = K \cdot \mathrm{e}^{\lambda t}
	.\]
	\begin{center}
		\slshape Comment résoudre l'équation différentielle {\scshape avec}\/ 2\tsup{nd} membre si on connaît la solution générale de l'équation {\scshape sans}\/ 2\tsup{nd} membre ?
	\end{center}
	On utilise la méthode le la variation de la constante.
	Soit $x(t) = k(t) \cdot \mathrm{e}^{\lambda t}$. Ainsi, en injectant cette expression de $x$\/ dans l'équation $(**)$, on trouve
	\begin{align*}
		(**) \iff& k'(t) \mathrm{e}^{\lambda t} + k(t) \cdot \lambda \mathrm{e}^{\lambda t} = \lambda k(t) \mathrm{e}^{\lambda t} + b(t)\\
		\iff& k'(t) \mathrm{e}^{\lambda t} = b(t)\\
		\iff& k'(t) = b(t) \cdot \mathrm{e}^{-\lambda t}\\
		\iff& k(t) = \int_{0}^{t} b(u)\cdot \mathrm{e}^{-\lambda u}~\mathrm{d}u + K\\
		\iff& x(t) = \left( \int_{0}^{t} b(u) \cdot \mathrm{e}^{-\lambda u}~\mathrm{d}u + K \right) \mathrm{e}^{\lambda t}\\
		\iff& x(t) = \underbrace{\int_{0}^{t} b(u) \cdot \mathrm{e}^{\lambda (t-u)}~\mathrm{d}u}_{\text{solution particulière}} + \underbrace{K \cdot \mathrm{e}^{\lambda t}}_{\substack{\text{solution}\\\text{générale}\\\text{de $(*)$}}}.
	\end{align*}
\end{rmkn}

	}
	\def\addmacros#1{#1}
}
{
	\chap[4]{Lemme d'\textsc{Arden} et retour sur le théorème de \textsc{Kleene}}
	\minitoc
	\renewcommand{\cwd}{../cours/annexeD/}
	\addmacros{
		\begin{exm}[Lemme d'\textsc{Arden}]
			Soient $K$ et $L$ deux langages. Résoudre $X = K\cdot X \cup L$ pour $X$ un langage.
			(On trouve~$X = K^* \cdot L$.) On suppose que $\varepsilon \not\in K$.
			On procède par double-inclusion.
			\begin{itemize}
				\item[``$\supseteq$''] Soit $X$ un langage tel que $X = K\cdot X \cup L$.
					Montrons par récurrence \guillemotleft~si $w$ est un mot de $X$ de taille $n$, alors $w \in K^* \cdot L$.
					\begin{itemize}
						\item Si $n= 0$, alors $w \in L$ car $\varepsilon \not\in  K$.
							Ainsi, $w = \varepsilon \cdot w$ et $\varepsilon \in K^*$. On en déduit que $w \in K^* \cdot L$.
						\item Si $|w| = n$, alors
							\begin{itemize}
								\item si $w \in L$, alors $w = \varepsilon \cdot w$ et donc $w \in K^* L$.
								\item si $w = v \cdot w'$ où $v \in K$ et $w' \in X$, alors $|w'| < |w|$. Ainsi, par hypothèse de récurrence, $w' \in K^* \cdot L$. Ainsi, $v \cdot w' \in K^* \cdot L$.
							\end{itemize}
					\end{itemize}
					Ainsi, $X \subseteq K^* \cdot L$.
				\item[``$\subseteq$'']
					Soit $w \in K^* \cdot L$. Il existe donc $n \in \N$, $(v_1, \ldots, v_n) \in K^n$ et $w' \in L$ tels que $w = v_1 \cdot \ldots \cdot v_n \cdot  w'$.
					Alors, $w' \in X$ donc $v_n \cdot w' \in X$ donc \ldots donc $v_1 v_2 \ldots v_n w' \in X$.
					Ainsi, $w \in X$.
			\end{itemize}
		\end{exm}
	
		\begin{exm}
			On considère l'automate ci-dessous.
			\begin{figure}[H]
				\centering
				\tikzfig{auto-ex}
				\caption{Automate exemple ($\mathcal{A}$)}
			\end{figure}
			On pose $X_i = \mathcal{L}\big((\Sigma, \mathcal{Q}, \{i\}, F, \delta)\big)$, où $x_i$ est l'unique point de départ.
			Ainsi, $\mathcal{L}(\mathcal{A}) = \bigcup_{i \in  I} X_i$.
			Déterminons les valeurs de $X_1$, $X_2$ et $X_3$.
			On applique un algorithme similaire au \guillemotleft~pivot de Gau\ss.~\guillemotright\ 
	
			\begin{align*}
				\left.\begin{array}{rl}
					X_1 &= \{a\} \cdot X_2 \cup \{a\} X_1\\
					X_2 &= \{b\} \cdot X_1 \cup \{a\} \cdot X_3 \cup \{\varepsilon\}\\
					X_3 &= \{b\} X_3 \cup \{\varepsilon\}
				\end{array}\right\}
				\iff& 
				\begin{cases}
					X_1&= \{a\}^* \cdot \{a\} \cdot X_2\\
					X_2&= \mathcal{L}(ba^* \cdot a) X_2 \cup \{a\} X_3 \cup \{\varepsilon\}\\
					X_3&= \{b\} \cdot X_3 \cup \{\varepsilon\}
				\end{cases}\\
				\iff& 
				\begin{cases}
					X_1&= \mathcal{L}(a^* \cdot a) X_2\\
					X_2&= \mathcal{L}\big((b\cdot a^* \cdot a)^*\big) \cdot \big(\{a\} X_3 \cup \{\varepsilon\}\big)\\
					X_3&= \{b\} X_3 \cup \{\varepsilon\}
				\end{cases}\\
				\iff& \begin{cases}
					X_1&= \mathcal{L}(a^* \cdot a) X_2\\
					X_2&= \mathcal{L}\big((b\cdot a^* \cdot a)^*\big) \cdot \big(\{a\} X_3 \cup \{\varepsilon\}\big)\\
					X_3&= \mathcal{L}(b^*)
				\end{cases} \\
				\iff& \begin{cases}
					X_1&= \mathcal{L}(a^* \cdot a) X_2\\
					X_2&= \mathcal{L}\big((ba^*a)^* \cdot (ab^*  \mid \varepsilon)\big)\\
					X_3&= \mathcal{L}(b^*)
				\end{cases} \\
				\iff& \begin{cases}
					X_1&= \mathcal{L}\big(a^* \cdot a \cdot (ba^*a)^* \cdot (ab^*  \mid \varepsilon)\big)\\
					X_2&= \mathcal{L}\big((ba^*a)^* \cdot (ab^*  \mid \varepsilon)\big)\\
					X_3&= \mathcal{L}(b^*)
				\end{cases}
			\end{align*}
		\end{exm}
	
		On peut généraliser la méthode employée dans l'exemple précédent pour montrer que tout langage reconnaissable est régulier.
	}
	\def\addmacros#1{#1}
}
{
	\chap[5]{Tas et files de priorités}
	\minitoc
	\renewcommand{\cwd}{../cours/annexeE/}
	\addmacros{
		L'objectif d'une file de priorité est de récupérer l'élément de priorité minimale.
		On organise cette structure de données sous forme d'un arbre tournois.\footnote{Un arbre tournois n'est pas un arbre binaire de recherche.}
		Un arbre tournois est un arbre dont la priorité d'un nœud est supérieur à celle de ses fils.
		On impose une structure supplémentaire, l'arbre doit être parfait : l'arbre est complet jusqu'à l'avant dernier niveau, où il est replis à gauche.
		 On définit plusieurs opérations sur cette file de priorité (de type \texttt{fp}, où les éléments sont de type \texttt{elem}) :
		\begin{itemize}
			\item $\texttt{insérer} : \texttt{fp} \to \texttt{elem} \to \texttt{fp}$ qui insère un élément,
			\item $\texttt{lire\_min} : \texttt{fp} \to \texttt{elem}$ qui récupère l'élément de priorité minimale,
			\item $\texttt{supprimer\_min} : \texttt{fp} \to \texttt{fp}$ qui supprime l'élément de priorité minimale,
			\item ($\texttt{diminuer\_priorité} : \texttt{fp} \to \texttt{elem} \to \texttt{fp}$),\footnote{Cette opération est parfois omise car trop compliquée à implémenter.}
			\item $\texttt{créer} : (\:) \to \texttt{fp}$.
		\end{itemize}
		On définit un type \texttt{btree}, représentant un arbre binaire, et on implémente les opérations ci-dessous en \textsc{OCaml}.
		\begin{lstlisting}[language=caml,caption=Définition du type \texttt{btree}]
	type 'a btree =
	| Node of 'a * 'a btree * 'a btree
	| Empty
		\end{lstlisting}
		Pour l'opération \texttt{créer}, on retourne \texttt{Empty} (cela donne une complexité en $\Theta(1)$). 
		Pour l'opération \texttt{insérer}, on insère l'élément comme feuille (de manière à conserver la propriété de l'arbre parfait), et on inverse le nœud avec son parent jusqu'à ce que la propriété soit vérifiée ($\Theta(\log_2 n)$).
		Pour l'opération \texttt{lire\_min}, on lit la racine ($\Theta(1)$).
		Pour l'opération \texttt{supprimer\_min}, on permute la racine et le dernier nœud (\textit{i.e.} le nœud le plus à droite de hauteur maximale), et on restore la structure d'arbre tournois en permutant un nœud et son fils de valeur minimale, et en répétant ($\Theta(\log_2 n)$).
		Pour trouver le dernier nœud, on garde en mémoire cet emplacement.
		On peut aussi implémenter cet algorithme avec un tableau ($\triangleright$ \textsc{tp}), ou avec une liste triée (mais la complexité est moins bien).
	}
	\def\addmacros#1{#1}
}
{
	\chap[6]{Arithmétique}
	\minitoc
	\renewcommand{\cwd}{../cours/annexeF/}
	\addmacros{
		Un des premiers algorithmes codé est l'algorithme d'Euclide pour calculer le \textsc{pgcd}. Pour $a \neq 0$, on a $a \wedge 0 = a$ et $a \wedge b = b \wedge (a\ \mathrm{mod}\ b)$.
		On peut le coder en \textsc{OCaml} avec la fonction \texttt{euclid} suivante.
	
		\begin{lstlisting}[language=caml,caption=Algorithme d'Euclide calculant le \textsc{pgcd}]
	let rec euclid (a: int) (b: int): int =
		(* Hyp: a >= b et a != 0 *)
		if b = 0 then a
		else euclide b (a mod b)
		\end{lstlisting}
		
		Quelle est la complexité de cet algorithme ?
		On représente le nombre d'appels récursifs à \texttt{euclid}, et on devine une courbe logarithmique.
		En notant $(u_n)$ les divisions euclidiennes réalisées et $(q_n)$ les quotients, ainsi, on $u_n = q_{n-1} \cdot u_{n-1} + u_{n-2}$.
		Alors, $\texttt{euclid}(u_n, u_{n-1}) = \cdots = \texttt{euclid}(u_3, u_2) = \texttt{euclid}(u_2, u_1) = \texttt{euclid}(u_1, u_0)$.
	
		En fixant la complexité, on cherche les valeurs de $(u_n)$ maximisant les appels récursifs.
		On peut montrer par récurrence que si $\texttt{euclid}(a,b)$ conduit à $n$ appels récursifs de \texttt{euclid}, alors $a \ge F_n$ et $b \ge F_{n-1}$, où $(F_n)_{n\in\N}$\/ est la suite de Fibonacci.
	
		En effet, soit un tel couple $(a,b)$. Alors, $(b, a\ \mathrm{mod}\ b)$ conduit à $n - 1$ appels récursifs donc~$b \ge F_{n-1}$ et~$a\ \mathrm{mod}\ b \ge F_{n-2}$ par hypothèse de recurrence.
		Et, $a = bq + (a\ \mathrm{mod}\ b)$ et donc $a \ge F_{n-1} + F_{n-2} = F_n$.
	
		De plus, pour tout $n \in \N \setminus \{0,1\}$, $F_n \ge \varphi^{n-2}$ où $\varphi$ est le nombre d'or.\footnote{C'est la solution positive de $X^2 - X - 1 = 0$.}
		En effet, $F_2 = 1 \ge \varphi^0 = 1$ et $F_3 = 2 \ge \varphi^1 = \varphi = (1 + \sqrt{5}) / 2$. Et, $F_n = F_{n-1} + F_{n-2} \ge \varphi^{n-3} + \varphi^{n-4} \ge \varphi^{n-4}(1 + \varphi) \ge \varphi^{n-2}$.
	
		Soient $(p,q)$, où $p \ge q$, une entrée de l'algorithme d'Euclide. Si l'appel $\texttt{euclid}(p,q)$ conduit à plus de $\left\lceil \log_\varphi p \right\rceil + 4$ appels, alors $p \ge F_{\left\lceil \log_\varphi p \right\rceil + 4} \ge \varphi^{\left\lceil \log_\varphi p \right\rceil + 4 - 2} > \varphi^{\log_\varphi p} = p$, ce qui est absurde.
	
		Ceci conduit à une complexité en $\mathcal{O}(\log p)$.
	
		\bigskip
	
		Soit $n$ un entier premier.
		Pour l'algorithme RSA, on cherche un inverse de $a \in \sfrac{\Z}{n\Z}$ : on cherche $b \in \sfrac\Z{n\Z}$ tel que $ab \equiv 1 \mod 1$. D'après le théorème de Bézout, on a $au + nv = 1$ car $a \wedge n = 1$. L'inverse est $v$. D'où l'importance des coefficients de Bézout.
	
		Comment calculer les coefficients de Bézout ?
		On peut utiliser l'algorithme d'Euclide.
		On pose $r_n$\/ la valeur de \texttt{a} après $n$ appels récursifs.
	
		\begin{table}[H]
			\centering
			\begin{tabular}{c|c|c|c|c}
				$r_i$ & $u_i$ & & $v_i$ &\\ \hline \hline
				$r_0 = a$ & $1$ & $a$ & $0$ & $b$\\
				$r_1 = b$ & $0$ & $b$ & $1$ & $a$\\
				$\vdots$ & $\vdots$ & $\vdots$ & $\vdots$ & $\vdots$ \\
				$r_{i-2}$ & $u_{i-2}$ & $a$ & $v_{i-2}$ & $b$\\
				$r_{i-1}$ & $u_{i-1}$ & $a$ & $v_{i-1}$ & $b$ \\
			\end{tabular}
			\caption{Valeurs de $r_i$ avec invariant $r_i = a u_i + b v_i$}
		\end{table}
	
		Alors,
		\begin{align*}
			r_i &= u_{i-2} a + v_{i-2} b - (r_{i-2} / r_{i-1}) (u_{i-1}a + v_{i-1} b)\\
			&= \big(u_{i-2} - (r_{i-2}/r_{i-1}) u_{i-1}\big) a + \big(v_{i-2} - (r_{i-2}/r_{i-1}) v_{i-1}\big) b \\
		\end{align*}
		Ainsi, on a bien $\mathrm{pgcd}(a,b) = u_{n-1} a + v_{n-1} b$.
	}
	\def\addmacros#1{#1}
}
{
	\chap[7]{Arbres rouges-noirs}
	\minitoc
	\renewcommand{\cwd}{../cours/annexeG/}
	\addmacros{
		Un arbre rouge-noir est un cas particulier des arbres binaires de recherches.
		On l'utilise notamment pour représenter des ensembles, on veut donc réaliser deux opérations simples : l'insertion et le test d'appartenance.
		Initialement, on pense représenter un ensemble par une liste triée.
		Mais, on utilise plutôt un arbre binaire pour représenter des données avec une hauteur logarithmique, contrairement à une hauteur linéaire.
		Les arbres binaires de recherches sont des arbres dans lesquels ont peut réaliser une dichotomie.
		\begin{lstlisting}[language=caml,caption=Arbre binaire de recherche]
	type 'a btree = E | N of 'a * 'a btree * 'a btree
	
	let rec mem (x: 'a) (t: 'a btree): bool =
		match t with
		| E -> false
		| N(y, g, d) ->
				if x < y      then mem x g
				else if x = y then true
				else               mem x d
	
	let rec insere (x: 'a) (t: 'a btree): 'a btree =
		match t with
		| E -> false
		| N(y, g, d) ->
				if x < y      then N(y, insere x g, d)
				else if x = y then t
				else               N(y, g, insere x d)
		\end{lstlisting}
		La fonction \texttt{mem} permet de réaliser ce test d'appartenance et la fonction \texttt{insere} insère l'insertion dans l'arbre.
		Ainsi, on peut représenter un ensemble avec le type \texttt{'a btree}.
	
		Mais, cet arbre peut être déséquilibré, et l'utilisation de la dichotomie ne donne pas de résultats très avantageux.
		On utilise donc un arbre \textit{auto-équilibrant}, comme les \textsc{avl} du 1er \textsc{dm}. Pour les \textsc{avl}, la différence de hauteur est $-1$, $0$ ou $1$.
	
		On introduit donc le concept d'arbre rouge-noir.
		Un arbre rouge-noir est un arbre parfait, qui a une certaine \guillemotleft~élasticité.~\guillemotright\@
		On colorie chaque nœuds pour imposer des contraintes sur cette élasticité.
		Les branches de l'arbre a une longueur de rupture.
		Un arbre contenant uniquement des nœuds noirs est un arbre parfait.
		Et, entre deux nœuds noirs, on peut insérer un nœud rouge.
		Un arbre rouge-noir vérifie donc les trois propriétés suivantes :
		\begin{enumerate}[label=(\textit{\alph*})]
			\item la racine est noire,
			\item le père d'un nœud rouge est noir,
			\item la hauteur noir de chaque feuille externe est constante. \hfill [important]
		\end{enumerate}
		Une \textit{feuille externe} est, dans le code \textsc{OCaml}, l'expression \texttt{E} ; et, la \textit{hauteur noir} d'une feuille externe est le nombre de nœuds noirs depuis la racine.
		On définit la \textit{hauteur noir} d'un arbre comme la hauteur noir de chaque feuille externe (qui est constante).
	
		Le problème est l'insertion d'un nœud.
		Insérer un nœud noir est, en général, plus dangereux car il modifie la hauteur noir de tout l'arbre.
		On préfère donc insérer un nœud rouge, sauf dans le cas de la racine.
	
		On considère donc la propriété (\textit{c}) comme invariante. En effet, corriger un arbre pour valider la propriété (\textit{a}) ou la propriété (\textit{b}) est bien plus simple.
	
		On traite tous les cas dans le diaporama sur \textit{cahier-de-prépa}, et on réalise l'exemple sur les lettres A, L, G, O, R, I, T, H, M, E.
	
		\bigskip
		
		Pour supprimer un nœud dans un arbre binaire classique, on peut le remplacer par le maximal de son sous-arbre droit, ou le minimum de son sous-arbre gauche. Si on supprime un nœud ayant un seul fils, on n'a qu'à re-brancher le sous-arbre.
		Pour les arbres rouges-noirs, c'est supprimer un nœud noir qui pose problème.
		Pour cela, on introduit les nœuds doublement noirs, qui comptent pour deux dans la hauteur noir.
		Ainsi, on supprime le nœud noir et on remplace les autres nœuds par des nœuds doublement noirs.
		L'algorithme n'a donc qu'à faire remonter le nœud doublement noir, jusqu'à la racine, où il sera transformé en nœud simplement noir.
	
		Un arbre de hauteur $h$ a une hauteur noir $\mathrm{bh} \ge h / 2$. Et, la taille, \textit{i.e.} le nombre de nœuds, est supérieure à $2^{\mathrm{bh}} - 1$.
		On conclut que $h \le 2 \log_2(\mathrm{taille} + 1)$.
		De même par la propriété (\textit{c}) permet de conclure que $h = \Theta(\log_2 \mathrm{taille})$.
	}
	\def\addmacros#1{#1}
}
{
	\chap[8]{Complexité moyenne}
	\minitoc
	\renewcommand{\cwd}{../cours/annexeH/}
	\addmacros{
		Dans les annexes et cours précédents, on a vu la complexité \guillemotleft~pire cas~\guillemotright\ et la complexité amortie.
		On considère les nombres d'opérations possibles pour toute entrée de taille $n$.
		La complexité \guillemotleft~pire cas~\guillemotright\ est la complexité obtenue en prenant le $\max$ d'opération possibilité.
		Pour la complexité moyenne, on suppose que chaque ensemble d'entrée de taille $n$ est munit d'une probabilité $P_n$.
		Par exemple, on considère qu'une entrée est une permutation de $n$ éléments, \textit{i.e.}\ un élément de $\mathfrak{S}_n$.
		On suppose que chaque entrée arrive avec équiprobabilité. Ainsi, $\forall \sigma \in \mathfrak{S}_n$, $P_n(\sigma) = 1/n!$.
		Ainsi, on a \[
			C_{\max}(n) = \max_{\sigma \in \mathfrak{S}_n} C(s) \text{ et } C_{\mathrm{moy}} = \sum_{\sigma \in \mathfrak{S}_n} P(\sigma) \cdot C(\sigma)
		.\]
		La complexité moyenne est la moyenne des complexité pondérées par les probabilités.
	
		\begin{exm}
			On considère l'algorithme ci-dessous.
			\begin{algorithm}[H]
				\centering
				\begin{algorithmic}[1]
					\Entree $\sigma \in \mathfrak{S}_n$ et $i \in \llbracket 1,n \rrbracket$
					\Sortie $j \in \llbracket 1,n \rrbracket$ tel que $\sigma(j) = i$.
					\For{$j \in \llbracket 1,n \rrbracket$}
					\If{$\sigma(j) = i$} \Return $j$
					\EndIf
					\EndFor
				\end{algorithmic}
				\caption{Calcul d'inverse d'une permutation}
			\end{algorithm}
			\noindent
			On munit $\mathfrak{S}_n$ de la probabilité uniforme.
			Soit $i \in \llbracket 1,n \rrbracket$.
			Notons, pour tout $j \in \llbracket 1,n \rrbracket$, $\mathfrak{S}_n^j = \{\sigma \in \mathfrak{S}_n  \mid \sigma(j) = i\}$.
			Remarquons que $\mathfrak{S}_n = \bigcupdot_{j=1}^n \mathfrak{S}_n^j$.
			Ainsi,
			\begin{align*}
				C_{\mathrm{moy}} &= \sum_{j=1}^n \sum_{\sigma \in \mathfrak{S}_n^j}P_n(\sigma) C(\sigma)\\
				&= \sum_{j=1}^n j \times \frac{|\mathfrak{S}_n^j|}{n!}\\
				&= \sum_{j=1}^n j \times \frac{(n+1)!}{n!}\\
				&= \frac{1}{n} \cdot \frac{n(n+1)}{2}\\
				&= \frac{n+1}{2}
			\end{align*}
		\end{exm}
	}
	\def\addmacros#1{#1}
}
{
	\chap[9]{Preuves de correction pour les fonctions récursives}
	\minitoc
	\renewcommand{\cwd}{../cours/annexeI/}
	\addmacros{
		On considère l'insertion dans un arbre binaire de recherche.
		Démontrons qu'elle est correcte.
		On adopte les notations de l'annexe G sur les arbres rouges-noirs.
		Montrons que, pour tout ABR $t$, et pour tout étiquette $e \in \mathds{E}$, 
		\[
			\mathrm{\acute{e}tiquettes}(\texttt{insertion}(t, x)) = \mathrm{\acute{e}tiquettes}(t) \cup \{x\}
		.\]
		Montrons le par induction.
		\begin{enumerate}
			\item Si $t = \texttt{E}$. Soit $x \in \mathds{E}$ : \[
					\mathrm{\acute{e}tiquettes}(\texttt{insertion}(\texttt{E},x)) = \mathrm{\acute{e}tiquettes}(N(x, \texttt{E},\texttt{E})) = \{x\} = \mathord{\underbrace{\mathrm{\acute{e}tiquettes}(\texttt{E})}_{\O}} \cup \{x\}
				.\]
			\item Si $t = \texttt{N}(y, g, d)$. Soit $x \in \mathds{E}$.
				\begin{itemize}
					\item si $x < y$, alors
						\begin{align*}
							\mathrm{\acute{e}tiquettes}(\texttt{insertion}(t,x))
							&= \mathrm{\acute{e}tiquettes}(\texttt{N}(y, \texttt{insertion}(g,x), d)) \\
							&= \{x\} \cup \mathrm{\acute{e}tiquettes}(\texttt{insertion}(g,x)) \cup \mathrm{\acute{e}tiquettes}(d) \\
							&= \{y\}  \cup \mathrm{\acute{e}tiquettes}(g) \cup \{x\} \cup \mathrm{\acute{e}tiquettes}(d) \\
							&= \mathrm{\acute{e}tiquettes}(\texttt{N}(y,g,d)) \cup \{x\} \\
							&= \mathrm{\acute{e}tiquettes}(t) \cup \{x\} \\
						\end{align*}
					\item on procède de même pour les autres cas.
				\end{itemize}
		\end{enumerate}
		On procède de même pour les autres propriétés.
	}
	\def\addmacros#1{#1}
}
