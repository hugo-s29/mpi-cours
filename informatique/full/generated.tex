%==============================================================
% Fichier généré automatiquement par `run.py,' ne pas modifier
%==============================================================

\part{Cours}

{
	\chap[-1]{Ordres et induction}
	\minitoc
	\renewcommand{\cwd}{../cours/chap-01/}
	\addmacros{
		\section{Motivation}

\lettrine On place au centre de la classe 40 bonbons. On en distribue un chacun. Si, par exemple, chacun choisit un bonbon et, au \textit{top} départ, prennent celui choisi.
Il est probable que plusieurs choisissent le même. Comme gérer lorsque plusieurs essaient d'accéder à la mémoire ?

Deuxièmement, sur l'ordinateur, plusieurs applications tournent en même temps. Pour le moment, on considérait qu'un seul programme était exécuté, mais, le \textsc{pc} ne s'arrête pas pendant l'exécution du programme.

On s'intéresse à la notion de \guillemotleft~processus~\guillemotright\ qui représente une tâche à réaliser.
On ne peut pas assigner un processus à une unité de calcul, mais on peut \guillemotleft~allumer~\guillemotright\ et \guillemotleft~éteindre~\guillemotright\ un processus.
Le programme allumant et éteignant les processus est \guillemotleft~l'ordonnanceur.~\guillemotright\@ Il doit aussi s'occuper de la mémoire du processus (chaque processus à sa mémoire séparée).

On s'intéresse, dans ce chapitre, à des programmes qui \guillemotleft~partent du même~\guillemotright\ : un programme peut créer un \guillemotleft~fil d'exécution~\guillemotright\ (en anglais, \textit{thread}). Le programme peut gérer les fils d'exécution qu'il a créé, et éventuellement les arrêter.
Les fils d'exécutions partagent la mémoire du programme qui les a créé.

En C, une tâche est représenté par une fonction de type \lstinline[language=c]!void* tache(void* arg)!. Le type \lstinline[language=c]!void*!\ est l'équivalent du type \lstinline[language=caml]!'a! : on peut le \textit{cast} à un autre type (comme \lstinline[language=c]-char*-).

\begin{lstlisting}[language=c,caption=Création de \textit{thread}s en C]
void* tache(void* arg) {
	printf("%s\n", (char*) arg);
	return NULL;
}

int main() {
	pthread_t p1, p2;

	printf("main: begin\n");

	pthread_create(&p1, NULL, tache, "A");
	pthread_create(&p2, NULL, tache, "B");

	pthread_join(p1, NULL);
	pthread_join(p2, NULL);

	printf("main: end\n");

	return 0;
}
\end{lstlisting}

\begin{lstlisting}[language=c,caption=Mémoire dans les \textit{thread}s en C]
int max = 10;
volatile int counter = 0;

void* tache(void* arg) {
	char* letter = arg;
	int i;

	printf("%s begin [addr of i: %p] \n", letter, &i);

	for(i = 0; i < max; i++) {
		counter = counter + 1;
	}

	printf("%s : done\n", letter);
	return NULL;
}

int main() {
	pthread_t p1, p2;

	printf("main: begin\n");

	pthread_create(&p1, NULL, tache, "A");
	pthread_create(&p2, NULL, tache, "B");

	pthread_join(p1, NULL);
	pthread_join(p2, NULL);

	printf("main: end\n");

	return 0;
}
\end{lstlisting}

Dans les \textit{thread}s, les variables locales (comme \texttt{i}) sont séparées en mémoire. Mais, la variable \texttt{counter} est modifiée, mais elle ne correspond pas forcément à $2 \times \texttt{max}$. En effet, si \texttt{p1} et \texttt{p2} essaient d'exécuter au même moment de réaliser l'opération \lstinline[language=c]-counter = counter + 1-, ils peuvent récupérer deux valeurs identiques de \texttt{counter}, ajouter 1, puis réassigner \texttt{counter}.
Ils \guillemotleft~se marchent sur les pieds.~\guillemotright\ 

Parmi les opérations, on distingue certaines dénommées \guillemotleft~atomiques~\guillemotright\ qui ne peuvent pas être séparées. L'opération \lstinline[language=c]-i++- n'est pas atomique, mais la lecture et l'écriture mémoire le sont.

\begin{defn}
	On dit d'une variable qu'elle est \textit{atomique} lorsque l'ordonnanceur ne l'interrompt pas.
\end{defn}

\begin{exm}
	L'opération \lstinline[language=c]-counter = counter + 1- exécutée en série peut être représentée comme ci-dessous. Avec \texttt{counter} valant 40, cette exécution donne 42.
	\begin{table}[H]
		\centering
		\begin{tabular}{l|l}
			Exécution du fil A & Exécution du fil B\\ \hline
			(1)~$\mathrm{reg}_1 \gets \texttt{counter}$ & (4)~$\mathrm{reg}_2 \gets \texttt{counter}$ \\
			(2)~$\mathrm{reg}_1{++}$ & (5)~$\mathrm{reg}_2{++}$ \\
			(3)~$\texttt{counter} \gets \mathrm{reg}_1$ & (6)~$\texttt{counter} \gets \mathrm{reg}_2$
		\end{tabular}
	\end{table}
	\noindent Mais, avec l'exécution en simultanée, la valeur de \texttt{counter} sera 41.
	\begin{table}[H]
		\centering
		\begin{tabular}{l|l}
			Exécution du fil A & Exécution du fil B\\ \hline
			(1)~$\mathrm{reg}_1 \gets \texttt{counter}$ & (2)~$\mathrm{reg}_2 \gets \texttt{counter}$ \\
			(3)~$\mathrm{reg}_1{++}$ & (5)~$\mathrm{reg}_2{++}$ \\
			(4)~$\texttt{counter} \gets \mathrm{reg}_1$ & (6)~$\texttt{counter} \gets \mathrm{reg}_2$
		\end{tabular}
	\end{table}
	\noindent Il y a \textit{entrelacement} des deux fils d'exécution.
\end{exm}

\begin{rmk}[Problèmes de la programmation concurrentielle]
	\begin{itemize}
		\item Problème d'accès en mémoire,
		\item Problème du rendez-vous,\footnote{Lorsque deux programmes terminent, ils doivent s'attendre pour donner leurs valeurs.}
		\item Problème du producteur-consommateur,\footnote{Certains programmes doivent ralentir ou accélérer.}
		\item Problème de l'entreblocage,\footnote{\textit{c.f.} exemple ci-après.}
		\item Problème famine, du dîner des philosophes.\footnote{Les philosophes mangent autour d'une table, et mangent du riz avec des baguettes. Ils décident de n'acheter qu'une seule baguette par personne. Un philosophe peut, ou penser, ou manger. Mais, pour manger, ils ont besoin de deux baguettes. S'ils ne mangent pas, ils meurent.}
	\end{itemize}
\end{rmk}

\begin{exm}[Problème de l'entreblocage]~

	\begin{table}[H]
		\centering
		\begin{tabular}{l|l|l}
			Fil A & Fil B & Fil C\\ \hline
			RDV(C) & RDV(A) & RDV(B)\\
			RDV(B) & RDV(C) & RDV(A)\\
		\end{tabular}
		\caption{Problème de l'entreblocage}
	\end{table}
\end{exm}

Comment résoudre le problème des deux incrementations ? Il suffit de \guillemotleft~mettre un verrou.~\guillemotright\ Le premier fil d'exécution \guillemotleft~s'enferme~\guillemotright\ avec l'expression \lstinline[language=c]!count++!, le second fil d'exécution attend que l'autre sorte pour pouvoir entrer et s'enfermer à son tour.


		\section{Continuité}

\begin{exm}
	Dans l'exercice 2, chaque fonction $f_n : t \mapsto t^n$\/ est continue sur $[0,1]$\/ mais la limite $f$\/ n'est pas continue sur $[0,1]$\/ (car elle n'est pas continue en $1$).
\end{exm}

\begin{thm}
	Soit $a$\/ un réel dans un intervalle $T$\/ de $\R$. Si une suite de fonctions $(f_n)_{n\in\N}$\/ continues en $a$\/ converge uniformément sur $T$\/ vers une fonction $f$, alors $f$\/ est aussi continue en $a$.
\end{thm}

\begin{prv}
	On suppose les fonctions $f_n$\/ continues en $a$\/ ($f_n(x) \longrightarrow f_n(a)$) et que la suite de fonctions $(f_n)_{n\in\N}$\/ converge uniformément vers $f$\/ ($\sup\:|f_n -f| \longrightarrow 0$). On veut montrer que $f$\/ est continue en $a$\/ : $f(x) \tendsto{x \to a} f(a)$, i.e.\ \[
		\forall \varepsilon > 0,\:\exists \delta > 0,\: \forall x \in T,\quad|x-a| \le \delta \implies |f(x) - f(a)| \le \varepsilon
	.\]
	Soit $\varepsilon > 0$. On calcule \[
		\big|f(x) - f(a)\big| \le \big|f(x) - f_n(x)\big| + \big|f_n(x) - f_n(a)\big| + \big|f_n(a) - f(a)\big|
	\] par inégalité triangulaire. Or, par hypothèse, il existe un rang $N \in \N$\/ (qui ne dépend pas de $x$\/ ou de $a$), tel que, $\forall n \ge N$, $\big|f(x) - f_n(x)\big| \le \frac{1}{3} \varepsilon$, et $\big|f_n(a) - f(a)\big| \le \frac{1}{3} \varepsilon$.
	De plus, par hypothèse, il existe $\delta >0$\/ tel que si $|x - a| \le \delta$, alors $|f_n(x) - f_n(a)| \le \frac{1}{3}\varepsilon$.\footnote{C'est là où l'hypothèse de la convergence uniforme est utilisée : on a besoin que le $N$\/ ne dépende pas de $x$\/ car on le fait varier.}
	On en déduit que $\big|f(x) - f(a)\big| \le \varepsilon$.
\end{prv}

\begin{crlr}
	Soit $T$\/ un intervalle de $\R$. Si une suite de fonctions $(f_n)_{n\in\N}$\/ continues sur $T$\/ converge uniformément sur $T$\/ vers une fonction continue sur $T$.
\end{crlr}

\begin{met}[Stratégie de la barrière]
	\begin{enumerate}
		\item La continuité (la dérivabilité aussi) est une propriété {\it locale}. Pour montrer qu'une fonction est continue sur un intervalle $T$, il suffit donc de montrer qu'elle est continue sur tout segment inclus dans $T$.
		\item Mais, la convergence uniforme est une propriété {\it globale}. La convergence sur tout segment inclus dans un intervalle n'implique pas la convergence uniforme sur l'intervalle (voir l'exercice 2).
		\item On n'écrit pas \[
				\substack{\ds\text{convergence uniforme}\\\ds\text{avec barrière}} \mathop{\red\implies} \substack{\ds\text{convergence uniforme}\\\ds\text{sans barrière}} \implies \substack{\ds\text{continuité}\\\ds\text{sans barrière}}
			\] mais plutôt \[
				\substack{\ds\text{convergence uniforme}\\\ds\text{avec barrière}} \implies \substack{\ds\text{continuité}\\\ds\text{avec barrière}} \implies \substack{\ds\text{continuité}\\\ds\text{sans barrière}}
			.\]
		\item Si, pour tous $a$\/ et $b$, $f$\/ est bornée sur $[a,b] \subset T$, mais cela n'implique pas que $f$\/ est bornée. Contre-exemple : la fonction $f : x \mapsto \frac{1}{x}$\/ est bornée sur tout intervalle $[a,b]$\/ avec $a$, $b \in \R^+_*$, \red{\sc mais} $f$\/ n'est pas bornée sur $]0,+\infty[$.
	\end{enumerate}
\end{met}

\begin{thm}[double-limite ou d'interversion des limites]
	Soit une suite de fonctions $(f_n)_{n\in\N}$\/ définies sur un intervalle $T$, et, soit $a$\/ une extrémité (éventuellement infinie)\footnote{autrement dit, $a \in \bar\R = \R \cup \{+\infty,-\infty\}$} de cet intervalle. Si la suite de fonctions $(f_n)_{n\in\N}$\/ converge \underline{uniformément} sur $T$\/ vers $f$\/ et si chaque fonction $f_n$\/ admet une limite finie $b_n$\/ en $a$, alors la suite de réels $b_n$\/ converge vers un réel $b$, et $\lim_{t\to a} f(t) = b$. Autrement dit, \[
		\lim_{t\to a} \Big(\underbrace{\lim_{n\to +\infty} f_n(t)}_{f(x)}\Big) = \lim_{n\to +\infty} \Big(\underbrace{\lim_{t\to a} f_n(t)}_{b_n}\Big)
	.\] \qed
\end{thm}

\begin{rmkn}
	Le théorème de la double-limite \guillemotleft~contient~\guillemotright\ le théorème 6 (théorème de préservation/transmission de la continuité), c'est un cas particulier. En effet, si les fonctions $f_n$\/ sont continues, alors \[
		\lim_{x \to a}f(x) = \underbrace{\lim_{n\to +\infty} f_n(a)}_{f(a)}
	.\]
\end{rmkn}


		\addrecap
	}
	\def\addmacros#1{#1}
}

{
	\chap[0]{Logique}
	\minitoc
	\renewcommand{\cwd}{../cours/chap00/}
	\addmacros{
		\section{Motivation}

\lettrine On place au centre de la classe 40 bonbons. On en distribue un chacun. Si, par exemple, chacun choisit un bonbon et, au \textit{top} départ, prennent celui choisi.
Il est probable que plusieurs choisissent le même. Comme gérer lorsque plusieurs essaient d'accéder à la mémoire ?

Deuxièmement, sur l'ordinateur, plusieurs applications tournent en même temps. Pour le moment, on considérait qu'un seul programme était exécuté, mais, le \textsc{pc} ne s'arrête pas pendant l'exécution du programme.

On s'intéresse à la notion de \guillemotleft~processus~\guillemotright\ qui représente une tâche à réaliser.
On ne peut pas assigner un processus à une unité de calcul, mais on peut \guillemotleft~allumer~\guillemotright\ et \guillemotleft~éteindre~\guillemotright\ un processus.
Le programme allumant et éteignant les processus est \guillemotleft~l'ordonnanceur.~\guillemotright\@ Il doit aussi s'occuper de la mémoire du processus (chaque processus à sa mémoire séparée).

On s'intéresse, dans ce chapitre, à des programmes qui \guillemotleft~partent du même~\guillemotright\ : un programme peut créer un \guillemotleft~fil d'exécution~\guillemotright\ (en anglais, \textit{thread}). Le programme peut gérer les fils d'exécution qu'il a créé, et éventuellement les arrêter.
Les fils d'exécutions partagent la mémoire du programme qui les a créé.

En C, une tâche est représenté par une fonction de type \lstinline[language=c]!void* tache(void* arg)!. Le type \lstinline[language=c]!void*!\ est l'équivalent du type \lstinline[language=caml]!'a! : on peut le \textit{cast} à un autre type (comme \lstinline[language=c]-char*-).

\begin{lstlisting}[language=c,caption=Création de \textit{thread}s en C]
void* tache(void* arg) {
	printf("%s\n", (char*) arg);
	return NULL;
}

int main() {
	pthread_t p1, p2;

	printf("main: begin\n");

	pthread_create(&p1, NULL, tache, "A");
	pthread_create(&p2, NULL, tache, "B");

	pthread_join(p1, NULL);
	pthread_join(p2, NULL);

	printf("main: end\n");

	return 0;
}
\end{lstlisting}

\begin{lstlisting}[language=c,caption=Mémoire dans les \textit{thread}s en C]
int max = 10;
volatile int counter = 0;

void* tache(void* arg) {
	char* letter = arg;
	int i;

	printf("%s begin [addr of i: %p] \n", letter, &i);

	for(i = 0; i < max; i++) {
		counter = counter + 1;
	}

	printf("%s : done\n", letter);
	return NULL;
}

int main() {
	pthread_t p1, p2;

	printf("main: begin\n");

	pthread_create(&p1, NULL, tache, "A");
	pthread_create(&p2, NULL, tache, "B");

	pthread_join(p1, NULL);
	pthread_join(p2, NULL);

	printf("main: end\n");

	return 0;
}
\end{lstlisting}

Dans les \textit{thread}s, les variables locales (comme \texttt{i}) sont séparées en mémoire. Mais, la variable \texttt{counter} est modifiée, mais elle ne correspond pas forcément à $2 \times \texttt{max}$. En effet, si \texttt{p1} et \texttt{p2} essaient d'exécuter au même moment de réaliser l'opération \lstinline[language=c]-counter = counter + 1-, ils peuvent récupérer deux valeurs identiques de \texttt{counter}, ajouter 1, puis réassigner \texttt{counter}.
Ils \guillemotleft~se marchent sur les pieds.~\guillemotright\ 

Parmi les opérations, on distingue certaines dénommées \guillemotleft~atomiques~\guillemotright\ qui ne peuvent pas être séparées. L'opération \lstinline[language=c]-i++- n'est pas atomique, mais la lecture et l'écriture mémoire le sont.

\begin{defn}
	On dit d'une variable qu'elle est \textit{atomique} lorsque l'ordonnanceur ne l'interrompt pas.
\end{defn}

\begin{exm}
	L'opération \lstinline[language=c]-counter = counter + 1- exécutée en série peut être représentée comme ci-dessous. Avec \texttt{counter} valant 40, cette exécution donne 42.
	\begin{table}[H]
		\centering
		\begin{tabular}{l|l}
			Exécution du fil A & Exécution du fil B\\ \hline
			(1)~$\mathrm{reg}_1 \gets \texttt{counter}$ & (4)~$\mathrm{reg}_2 \gets \texttt{counter}$ \\
			(2)~$\mathrm{reg}_1{++}$ & (5)~$\mathrm{reg}_2{++}$ \\
			(3)~$\texttt{counter} \gets \mathrm{reg}_1$ & (6)~$\texttt{counter} \gets \mathrm{reg}_2$
		\end{tabular}
	\end{table}
	\noindent Mais, avec l'exécution en simultanée, la valeur de \texttt{counter} sera 41.
	\begin{table}[H]
		\centering
		\begin{tabular}{l|l}
			Exécution du fil A & Exécution du fil B\\ \hline
			(1)~$\mathrm{reg}_1 \gets \texttt{counter}$ & (2)~$\mathrm{reg}_2 \gets \texttt{counter}$ \\
			(3)~$\mathrm{reg}_1{++}$ & (5)~$\mathrm{reg}_2{++}$ \\
			(4)~$\texttt{counter} \gets \mathrm{reg}_1$ & (6)~$\texttt{counter} \gets \mathrm{reg}_2$
		\end{tabular}
	\end{table}
	\noindent Il y a \textit{entrelacement} des deux fils d'exécution.
\end{exm}

\begin{rmk}[Problèmes de la programmation concurrentielle]
	\begin{itemize}
		\item Problème d'accès en mémoire,
		\item Problème du rendez-vous,\footnote{Lorsque deux programmes terminent, ils doivent s'attendre pour donner leurs valeurs.}
		\item Problème du producteur-consommateur,\footnote{Certains programmes doivent ralentir ou accélérer.}
		\item Problème de l'entreblocage,\footnote{\textit{c.f.} exemple ci-après.}
		\item Problème famine, du dîner des philosophes.\footnote{Les philosophes mangent autour d'une table, et mangent du riz avec des baguettes. Ils décident de n'acheter qu'une seule baguette par personne. Un philosophe peut, ou penser, ou manger. Mais, pour manger, ils ont besoin de deux baguettes. S'ils ne mangent pas, ils meurent.}
	\end{itemize}
\end{rmk}

\begin{exm}[Problème de l'entreblocage]~

	\begin{table}[H]
		\centering
		\begin{tabular}{l|l|l}
			Fil A & Fil B & Fil C\\ \hline
			RDV(C) & RDV(A) & RDV(B)\\
			RDV(B) & RDV(C) & RDV(A)\\
		\end{tabular}
		\caption{Problème de l'entreblocage}
	\end{table}
\end{exm}

Comment résoudre le problème des deux incrementations ? Il suffit de \guillemotleft~mettre un verrou.~\guillemotright\ Le premier fil d'exécution \guillemotleft~s'enferme~\guillemotright\ avec l'expression \lstinline[language=c]!count++!, le second fil d'exécution attend que l'autre sorte pour pouvoir entrer et s'enfermer à son tour.


		\section{Continuité}

\begin{exm}
	Dans l'exercice 2, chaque fonction $f_n : t \mapsto t^n$\/ est continue sur $[0,1]$\/ mais la limite $f$\/ n'est pas continue sur $[0,1]$\/ (car elle n'est pas continue en $1$).
\end{exm}

\begin{thm}
	Soit $a$\/ un réel dans un intervalle $T$\/ de $\R$. Si une suite de fonctions $(f_n)_{n\in\N}$\/ continues en $a$\/ converge uniformément sur $T$\/ vers une fonction $f$, alors $f$\/ est aussi continue en $a$.
\end{thm}

\begin{prv}
	On suppose les fonctions $f_n$\/ continues en $a$\/ ($f_n(x) \longrightarrow f_n(a)$) et que la suite de fonctions $(f_n)_{n\in\N}$\/ converge uniformément vers $f$\/ ($\sup\:|f_n -f| \longrightarrow 0$). On veut montrer que $f$\/ est continue en $a$\/ : $f(x) \tendsto{x \to a} f(a)$, i.e.\ \[
		\forall \varepsilon > 0,\:\exists \delta > 0,\: \forall x \in T,\quad|x-a| \le \delta \implies |f(x) - f(a)| \le \varepsilon
	.\]
	Soit $\varepsilon > 0$. On calcule \[
		\big|f(x) - f(a)\big| \le \big|f(x) - f_n(x)\big| + \big|f_n(x) - f_n(a)\big| + \big|f_n(a) - f(a)\big|
	\] par inégalité triangulaire. Or, par hypothèse, il existe un rang $N \in \N$\/ (qui ne dépend pas de $x$\/ ou de $a$), tel que, $\forall n \ge N$, $\big|f(x) - f_n(x)\big| \le \frac{1}{3} \varepsilon$, et $\big|f_n(a) - f(a)\big| \le \frac{1}{3} \varepsilon$.
	De plus, par hypothèse, il existe $\delta >0$\/ tel que si $|x - a| \le \delta$, alors $|f_n(x) - f_n(a)| \le \frac{1}{3}\varepsilon$.\footnote{C'est là où l'hypothèse de la convergence uniforme est utilisée : on a besoin que le $N$\/ ne dépende pas de $x$\/ car on le fait varier.}
	On en déduit que $\big|f(x) - f(a)\big| \le \varepsilon$.
\end{prv}

\begin{crlr}
	Soit $T$\/ un intervalle de $\R$. Si une suite de fonctions $(f_n)_{n\in\N}$\/ continues sur $T$\/ converge uniformément sur $T$\/ vers une fonction continue sur $T$.
\end{crlr}

\begin{met}[Stratégie de la barrière]
	\begin{enumerate}
		\item La continuité (la dérivabilité aussi) est une propriété {\it locale}. Pour montrer qu'une fonction est continue sur un intervalle $T$, il suffit donc de montrer qu'elle est continue sur tout segment inclus dans $T$.
		\item Mais, la convergence uniforme est une propriété {\it globale}. La convergence sur tout segment inclus dans un intervalle n'implique pas la convergence uniforme sur l'intervalle (voir l'exercice 2).
		\item On n'écrit pas \[
				\substack{\ds\text{convergence uniforme}\\\ds\text{avec barrière}} \mathop{\red\implies} \substack{\ds\text{convergence uniforme}\\\ds\text{sans barrière}} \implies \substack{\ds\text{continuité}\\\ds\text{sans barrière}}
			\] mais plutôt \[
				\substack{\ds\text{convergence uniforme}\\\ds\text{avec barrière}} \implies \substack{\ds\text{continuité}\\\ds\text{avec barrière}} \implies \substack{\ds\text{continuité}\\\ds\text{sans barrière}}
			.\]
		\item Si, pour tous $a$\/ et $b$, $f$\/ est bornée sur $[a,b] \subset T$, mais cela n'implique pas que $f$\/ est bornée. Contre-exemple : la fonction $f : x \mapsto \frac{1}{x}$\/ est bornée sur tout intervalle $[a,b]$\/ avec $a$, $b \in \R^+_*$, \red{\sc mais} $f$\/ n'est pas bornée sur $]0,+\infty[$.
	\end{enumerate}
\end{met}

\begin{thm}[double-limite ou d'interversion des limites]
	Soit une suite de fonctions $(f_n)_{n\in\N}$\/ définies sur un intervalle $T$, et, soit $a$\/ une extrémité (éventuellement infinie)\footnote{autrement dit, $a \in \bar\R = \R \cup \{+\infty,-\infty\}$} de cet intervalle. Si la suite de fonctions $(f_n)_{n\in\N}$\/ converge \underline{uniformément} sur $T$\/ vers $f$\/ et si chaque fonction $f_n$\/ admet une limite finie $b_n$\/ en $a$, alors la suite de réels $b_n$\/ converge vers un réel $b$, et $\lim_{t\to a} f(t) = b$. Autrement dit, \[
		\lim_{t\to a} \Big(\underbrace{\lim_{n\to +\infty} f_n(t)}_{f(x)}\Big) = \lim_{n\to +\infty} \Big(\underbrace{\lim_{t\to a} f_n(t)}_{b_n}\Big)
	.\] \qed
\end{thm}

\begin{rmkn}
	Le théorème de la double-limite \guillemotleft~contient~\guillemotright\ le théorème 6 (théorème de préservation/transmission de la continuité), c'est un cas particulier. En effet, si les fonctions $f_n$\/ sont continues, alors \[
		\lim_{x \to a}f(x) = \underbrace{\lim_{n\to +\infty} f_n(a)}_{f(a)}
	.\]
\end{rmkn}


		\section{Endomorphismes adjoints}

\begin{defn}
	On dit qu'un endomorphisme $f : E \to E$\/ est \textit{autoadjoint} si \[
		\forall (\vec{u}, \vec{v}) \in E^2,\quad \big<f(\vec{u})\:\big|\:\vec{v}\big> = \big<\vec{u}\:\big|\:f(\vec{v})\big>
	.\] 
\end{defn}

Un endomorphisme autoadjoint est aussi appelé endomorphisme \textit{symétrique} (\textit{c.f.}\ proposition suivante). L'ensemble des endomorphismes autoadjoints est noté $\mathcal{S}(E)$.

\begin{prop}
	Un endomorphism est autoadjoint si, et seulement si la matrice de $F$\/ dans une base \ul{orthonormée} $\mathcal{B}$\/ est orthogonale.
	Autrement dit : \[
		f \in \mathcal{S}(E) \iff \big[\:f\:\big]_\mathcal{B} \in \mathcal{S}_n(\R)
	.\]
\end{prop}

\begin{prv}
	\begin{description}
		\item[$\implies$] Soit $\mathcal{B} = (\vec{\varepsilon}_1, \ldots, \vec{\varepsilon}_n)$\/ une base orthonormée de $E$. Ainsi, \[
				\forall i,\:\forall j,\quad \big<f(\vec{\varepsilon}_i)\:\big|\: \vec{\varepsilon}_j\big> = \big<\vec{\varepsilon}_i\:\big|\:f(\vec{\varepsilon}_j)\big>
			.\] On pose $\big[\:f\:\big]_{\mathcal{B}} = (a_{i,j})$\/ : \[
				\begin{pNiceMatrix}[last-col,last-row]
					\quad&\quad&a_{1,j}&\quad&\quad&\vec{\varepsilon}_i\\
					&&&&&\\
					\quad&\quad&a_{i,j}&\quad&\quad&\vec{\varepsilon}_i\\
					&&&&&\\
					\quad&\quad&a_{n,j}&\quad&\quad&\vec{\varepsilon}_n\\
					f(\vec{\varepsilon}_i)&&f(\vec{\varepsilon}_j)&&f(\vec{\varepsilon}_n)\\
				\end{pNiceMatrix}
			.\] Ainsi, $f(\vec{\varepsilon}_j) = a_{1,j} \vec{\varepsilon}_1 + \cdots + a_{i,j} \vec{\varepsilon}_i + \cdots + a_{n,j} \vec{\varepsilon}_n$. D'où, $\left<\vec{\varepsilon}_i  \mid f(\vec{\varepsilon}_j) \right> = a_{i,j}$\/ car la base $\mathcal{B}$\/ est orthonormée.
			De même avec l'autre produit scalaire, $\left< f(\vec{\varepsilon}_i)  \mid \vec{\varepsilon}_j \right>$, d'où $a_{i,j} = a_{j,i}$\/ par symétrie du produit scalaire. On en déduit que $\big[\:f\:\big]_\mathcal{B} \in \mathcal{S}_n(\R)$.
		\item[$\impliedby$]
			Si $\big[\: f\:\big]_\mathcal{B} \in \mathcal{S}_n(\R)$, alors $\left<f(\vec{\varepsilon}_i)  \mid \vec{\varepsilon}_j \right> = \left<\vec{\varepsilon}_i  \mid f(\vec{\varepsilon}_j)\right>$.
			Or, on pose $\vec{u} = x_1 \vec{\varepsilon}_1+ \cdots + x_n \vec{\varepsilon}_n$, et $\vec{v} = y_1 \vec{\varepsilon}_1 + \cdots + y_n \vec{\varepsilon}_n$.
			\begin{align*}
				\left<f(\vec{u})  \mid \vec{v} \right> &= \left<x_1 f(\vec{\varepsilon}_1) + \cdots + x_n f(\vec{\varepsilon}_n)  \mid y_1 f(\vec{\varepsilon}_1) + \cdots + y_n f(\vec{\varepsilon}_n) \right> \\
				&= \Big<\sum_{i=1}^n x_i f(\vec{\varepsilon}_i)\:\Big|\: \sum_{j=1}^n y_j \vec{\varepsilon}_j\Big> \\
				&= \sum_{i,j \in \llbracket 1,n \rrbracket}  x_i y_j \left<f(\vec{\varepsilon}_i) \mid \vec{\varepsilon}_j \right>\\
			\end{align*}
			De même en inversant $\vec{u}$\/ et $\vec{v}$.
			On en déduit donc $\left<f(\vec{u} \mid \vec{v} \right> = \left<\vec{u}  \mid f(\vec{v}) \right>$.
	\end{description}
\end{prv}

\begin{exo}
	\begin{enumerate}
		\item Si $f$\/ est autoadjoint, montrons que $\Ker f \perp \Im f$, et $\Ker f \oplus \Im f$.
			On suppose $\forall \vec{u}$, $\forall \vec{v}$, $\left<f(\vec{u}) \mid \vec{v} \right> = \left<\vec{u}  \mid f(\vec{v}) \right>$.
			Soit $\vec{u} \in \Ker f$, et soit $\vec{v} \in \Im f$.
			On sait que $f(\vec{u}) = \vec{0}$, et qu'il existe $\vec{x} \in E$\/ tel que $\vec{v} = f(\vec{x})$.
			Ainsi, \[
				\left<\vec{u}  \mid \vec{v} \right>
				= \left<\vec{u}  \mid f(\vec{x}) \right>
				= \left<f(\vec{u})  \mid \vec{x} \right>
				= 0
			.\] 
			D'où $\vec{u} \perp \vec{v}$. Ainsi, $\Ker f \perp \Im f$.


			De plus, $E$\/ est de dimension finie, d'où, d'après le théorème du rang, \[
				\dim \Ker f + \dim \Im f = \dim E
			.\] Aussi, $\Ker f \oplus (\Ker f)^\perp = E$, donc $\dim(\Ker f) + \dim(\Ker f)^\perp = \dim E$.
			On en déduit donc que $\dim(\Im f)= \dim(\Ker f)^\perp$.
			Or, $\Im f \subset (\Ker f)^\perp$\/ car $\Im f \perp \Ker f$.
			Ainsi $\Im f = (\Ker f)^\perp$, on en déduit que \[
				\Im f \oplus \Ker f = E
			.\]
			\begin{description}
				\item[$\impliedby$] 
					Soit $p$\/ la projection sur $F$\/ parallèlement à $G$.
					Supposons l'endomorphisme $P$\/ autoadjoint.
					D'après la question 1., le $\Ker p \perp \Im p$.
					Ainsi, $F = \Im p$\/ et $G = \Ker p$.
					D'où, $F \perp G$, $p$\/ est donc une projection orthogonale.
				\item[$\implies$]
					Réciproquement, supposons $p$\/ une projection orthogonale.
					Soit $\mathcal{B} = (\vec{\varepsilon}_1, \ldots, \vec{\varepsilon}_q)$\/ une base orthonormée de $F$.
					Ainsi, pour tout $\vec{x} \in E$, \[
						p(\vec{x}) = \sum_{i = 1}^q \left<\vec{x}  \mid \vec{\varepsilon}_i \right>\,\vec{\varepsilon}_i
					.\] 
					On veut montrer que l'endomorphisme $p$\/ est autoadjoint.
					Soient $\vec{u}$\/ et $\vec{v}$\/ deux vecteurs de $E$.
					\begin{align*}
						\left<p(\vec{u})  \mid \vec{v} \right>
						= \Big<\sum_{i=1}^q \left< \vec{u}\mid \vec{\varepsilon}_i \right>\vec{\varepsilon}_i\:\Big|\; \vec{v}\;\Big>
						&= \sum_{i=1}^q \left<u  \mid \vec{\varepsilon}_{i} \right>\: \left< \vec{\varepsilon}_i  \mid v\right>\\
						&= \sum_{i=1}^q \left<v  \mid \vec{\varepsilon}_i \right>\:\left<\vec{\varepsilon}_i   \mid u\right> \\
						&= \left< \vec{u}  \mid p(\vec{v}) \right> \\
					\end{align*}

					Autre méthode, pour tous vecteurs $\vec{u}$\/ et $\vec{v}$\/ de $E$,
					\begin{align*}
						\left<p(\vec{u})  \mid \vec{v} \right>
						&= \left<p(\vec{u})  \mid p(\vec{v}) + \vec{v} - p(\vec{v}) \right> \\
						&= \left< p(\vec{u})  \mid p(\vec{v}) \right> + \left<p(\vec{u})  \mid  \vec{v} - p(\vec{v}) \right> \\
						&= \left<p(\vec{u}) \mid p(\vec{v}) \right> + \left<u - p(\vec{u})  \mid p(\vec{v}) \right> \\
						&= \left<\vec{u}  \mid p(\vec{v}) \right> \\
					\end{align*}
					car $p$\/ est orthogonale.
			\end{description}
	\end{enumerate}
\end{exo}


\begin{prop-defn}
	Si $f$\/ est un endomorphisme d'un espace euclidien $E$, alors il existe un unique endomorphisme de $E$, noté $f^\star$\/ et appelé l'\textit{adjoint} de $f$, tel que \[
		\forall (\vec{u},\vec{v}) \in E^2,\quad\quad \left<f^\star(\vec{u})  \mid \vec{v} \right> =  \left<\vec{u}  \mid f(\vec{v}) \right>
	.\] 
	Si $A$\/ est la matrice $f$\/ dans une base orthonormée $\mathcal{B}$\/ de $E$, alors $A^\top$\/ est la matrice de $f^\star$\/ dans~$\mathcal{B}$\/ : \[
		\big[\:f\:\big]_\mathcal{B} = \big[\:f\:\big]_\mathcal{B}^\top
	.\]
\end{prop-defn}

\begin{prv}
	Soit $\vec{u} \in E$. L'application \begin{align*}
		\varphi: E &\longrightarrow \R \\
		\vec{v} &\longmapsto \left<\vec{u}  \mid f(\vec{v}) \right>.
	\end{align*}
	La forme $\varphi$\/ est linéaire car $\varphi(\alpha_1 \vec{v}_1 + \alpha_2 \vec{v}_2) = \left<\vec{u} \mid f(\alpha_1 \vec{v}_1 + \alpha_2 \vec{v}_2) \right> = \left<\vec{u}  \mid \alpha_1 f(\vec{v}_1) + \alpha_2 f(\vec{v}_2) \right> = \alpha_1\left<\vec{u}  \mid  f(\vec{v}) \right> + \alpha_2 \left<\vec{u}  \mid f(\vec{v}_2) \right> = \alpha_1 \varphi(\vec{v}_1) + \alpha_2 \varphi(\vec{v}_2)$.
	D'où, d'après le théorème de \textsc{Riesz}, il existe un \ul{unique} vecteur $\vec{a} \in E$\/ tel que $\varphi(\vec{v}) = \left<\vec{a}  \mid \vec{v} \right>$\/ pour tout $\vec{v} \in E$.
	Ainsi, pour tout vecteur $\vec{v} \in E$, $\left<\vec{u}  \mid f(\vec{v}) \right> = \left<\vec{a}  \mid \vec{v} \right>$.
	On note $\vec{a} = f^\star(\vec{u})$.
	Soit l'application \begin{align*}
		f^\star : E &\longrightarrow E \\
		\vec{u} &\longmapsto f^\star(\vec{u}).
	\end{align*}
	La démonstration telle que $f^\star $\/ est linéaire est dans le poly.
	L'application $f^\star$\/ vérifie : $\left< \vec{u} \mid f(\vec{v}) \right> = \left<f^\star (\vec{u})  \mid \vec{v} \right>$, pour tous vecteurs $\vec{u}$\/ et $\vec{v}$.
	\textsl{Quelle est la matrice de $f^\star$, dans une base orthonormée ?}\@
	Soit $\mathcal{B}$\/ une base orthonormée de $E$, et soient $A = \big[\:f\:\big]_\mathcal{B}$, $B = \big[\:f^\star \:\big]_\mathcal{B}$, $U = \big[\:\vec{u}\:\big]_\mathcal{B}$, et $V = \big[\:\vec{v}\:\big]_\mathcal{B}$.
	Les matrices $U$\/ et $V$\/ sont des vecteurs colonnes, et $A$\/ et $B$\/ sont des matrices carrées.
	Ainsi, \[
		U^\top \cdot A \cdot V = \left< \vec{u}  \mid f(\vec{v}) \right> 
		= \left<f^\star (\vec{u})  \mid \vec{v} \right> = (B\cdot U)^\top \cdot V,
	\] ce qui est vrai quelque soit les vecteurs colonnes $U$\/ et $V$.
	D'où, $\forall U$, $\forall V$, $U^\top \cdot \big(A \cdot V\big) = U^\top \cdot \big(B^\top \cdot V\big)$.
	Ainsi, pour tous vecteurs $U$\/ et $V$, \[
		U^\top \cdot \Big[ (AV) - (B^\top V)\Big] = 0
	.\] En particulier, si $U = (AV) - (B^\top V)$, le produit scalaire $\left<\vec{u}  \mid \vec{u} \right>$\/ est nul, donc $U = 0$.
	Ainsi, \[
		\forall V,\quad A\cdot V = B^\top\cdot  V
	.\] De même, on conclut que $A = B^\top$. On en déduit donc que \[
	\big[\:f^\star\:\big]_\mathcal{B} = \big[\:f\:\big]_\mathcal{B}^\top
	.\]
\end{prv}

Les propriétés suivantes sont vrais :
\begin{itemize}
	\item $(f  \circ g)^\star  = g^\star \circ f^\star$, \quad $(f^\star)^\star = f$, \quad et \quad $(\alpha f + \beta g)^\star  = \alpha f^\star + \beta g^\star $\/ ;
	\item $(A\cdot B)^\top  = B^\top \cdot A^\top$, \quad $(A^\top)^\top = A$, \quad et \quad $(\alpha A + \beta B)^\top = \alpha A^\top+ \beta B^\top $.
\end{itemize}
Des deuxièmes et troisièmes points,  il en résulte que les applications $f \mapsto f^\star$, et $A \mapsto A^\top$\/ sont des applications involutives.



		\begin{prop-defn}
  Soit $(\Omega, \mathcal{A}, P)$\/ un espace probabilisé, et soit $X$\/ une \textit{vard}. Si $X^2$\/ est d'espérance finie, alors $X$\/ aussi, et on appelle \textit{variance} le réel positif \[
    \mathrm{V}(X) = \mathrm{E}\Big(\big[X - \mathrm{E}(X)\big]^2\Big) = \underbrace{\mathrm{E}(X^2) - \big(\mathrm{E}(X)\big)^2}_{\mathclap{\text{Relation de \textsc{König \& Huygens}}}} \ge 0
  .\]
  L'\textit{écart-type} $\sigma(X)$\/ est la racine carrée de la variance : \[
    \sigma(X) = \sqrt{\mathrm{V}(X)}
  .\]
\end{prop-defn}

\begin{prv}
  On pose $\mu = \mathrm{E}(X)$, et on a $[X - \mu]^2 = X^2 - 2 \mu X + \mu^2$. D'où, par linéarité de l'espérance,
  \begin{align*}
    \mathrm{E}\big((X-\mu)^2\big)
    &= \mathrm{E}(X^2 - 2\mu X + \mu^2) \\
    &= \mathrm{E}(X^2) - 2\mu \mathrm{E}(X) + \mu^2 \\
    &= \mathrm{E}(X^2) - 2\mu^2 + \mu^2 \\
    &= \mathrm{E}(X^2) - \big(\mathrm{E}(X)\big)^2. \\
  \end{align*}
  De plus, d'après le lemme précédent, si $X^2$\/ est d'espérance finie, alors $X$\/ est d'espérance finie.
\end{prv}

\begin{rmk}
  \begin{enumerate}
    \item La variance mesure la \textit{dispersion}, ou l'\textit{étalement} des valeurs $a_i$\/ autour de l'espérance $\mathrm{E}(X)$. En particulier, s'il existe $a \in \R$\/ tel que $P(X = a) = 1$, alors $\mathrm{E}(X) = a$\/ et $\mathrm{V}(X) = 0$. (C'est même une équivalence.)
    \item Si la variable $X$\/ a une unité ($\mathrm{km}/\mathrm{s}$, $\mathrm{V}/\mathrm{m}$, etc.), alors l'écart type a la même unité (d'où l'intérêt de calculer la racine carrée de la variance).
    \item Soient $\alpha$\/ et $\beta$\/ deux réels. Si  $X^2$\/ est d'espérance finie, alors \[
      \mathrm{V}(\alpha X + \beta) = \alpha^2\cdot  \mathrm{V}(X)
    .\]
    (Une translation ne change pas la dispersion des valeurs, et multiplier par un réel multiplie l'espérance, mais aussi la dispersion, d'où le carré.)
  \end{enumerate}
\end{rmk}

\begin{exo}
  \textsl{Montrer que
  \begin{enumerate}
    \item si $X \sim \mathcal{B}(n, p)$, alors $X^2$\/ est d'espérance finie et $\mathrm{V}(X) = n\cdot p\cdot q$.
    \item si $T \sim \mathcal{G}(p)$, alors $T^2$\/ est d'espérance finie et $\mathrm{V}(T) = \frac{q}{p^2}$.
    \item si $X \sim \mathcal{P}(\lambda)$, alors $X^2$\/ est d'espérance finie et $\mathrm{V}(X) = \lambda$.
  \end{enumerate}
  }

  \begin{enumerate}
    \item
      Si $X \sim \mathcal{B}(n,p)$, alors $X(\Omega) = \llbracket 0,n \rrbracket$\/ et, pour $k \in X(\Omega)$, $P(X = k) = {n\choose k}\,p^k\,q^{n-k}$.
      On a déjà montré que $\mathrm{E}(X) = n\cdot p$.
      On va montrer que $\mathrm{V}(X) = n\,p\,q$.
      La variable aléatoire $X^2$\/ est d'espérance finie car $X(\Omega)$\/ est fini.
      Et,
      \begin{align*}
        \mathrm{E}(X^2) &= \sum_{k=0}^n k^2\: P(X = k)\\
        &= \sum_{k=0}^n k^2 {n\choose p} p^k q^{n-k} \\
        &= \ldots \\
      \end{align*}
      En effet, d'après la ``petite formule,'' on a \[
        \forall k \ge 1,\quad k{n\choose k} = n{n-1 \choose k-1}
      \] d'où, $(k-1) {n-1\choose k-1} = (n-1) \choose {n-2 \choose k-2}$. Ainsi, \[
        \forall k \ge 2,\quad k(k-1){n\choose k} = n(n+1) {n-2\choose k-2}
      .\] 
    \item Si $T \sim \mathcal{G}(p)$, alors $T(\Omega) = \N^*$\/ et $\forall k \in T(\Omega)$, $P(T = k) = p \times q^{k-1}$.
      On a déjà prouvé que $\mathrm{E}(T) = \frac{1}{p}$.
      On veut montrer que $\mathrm{V}(T) = \frac{q}{p^2}$. Montrons que la variable $T^2$\/ possède une espérance : la série $\sum k^2\: P(T = k)$\/ converge absolument car $k^2 \:P(T = k) = k^2 \cdot p \cdot q^{k-1}$.
      Or, pour $k \ge 2$, $\frac{\mathrm{d}^2}{\mathrm{d}x^2} x^k = k(k-1)\,x^{k-2}$. Et, on peut dériver terme à terme une série entière sans changer son rayon de convergence, et la série $\sum x^k$\/ a pour rayon de convergence 1. D'où, $\sum k(k-1)\, x^{k-2}$\/ a pour rayon de convergence 1. Or, $q \in {]0,1[} \subset {]-1,1[}$\/ donc la série $\sum k(k-1)q^{k-2}$\/ converge. De plus, $\sum k (k-1)\, q^{k-2} = \sum k^2\,q^{k-2} - \sum k\,q^{k-2}$.
      D'où,  $\sum k^2 q^{k-2} = \sum k(k-1)\, q^{k-2} + \sum k\,q^{k-2}$, qui converge. Par suite,
      \begin{align*}
        \sum_{k=1}^\infty k^2\, P(T = k) &= \sum_{k=1}^\infty k^2\,p\,q^{k-1}\\
        &= p + pq \sum_{k=2}^\infty k^2 q^{k-2} \\
        &= p + pq \sum_{k=2}^\infty k(k-1)\,q^{k-2} + p\sum_{k=2}^\infty k\,q^{k-1} \\
        &= p + pq\, \frac{2}{(1-q)^3} + p\left(\frac{1}{(1-q)^2} - 1 \right) \rlap{\quad\quad \text{\textit{c.f.} en effet après}}\\
        &= p + pq\, \frac{2}{p^3} + p\left( \frac{1}{p^2} - 1 \right) \\
        &= \frac{2q}{p^2} + \frac{1}{p} \\
        &= \frac{2q + p}{p^2} \\
        &= \frac{2q + (1-q)}{p^2} \\
        &= \frac{q+1}{p^2}. \\
      \end{align*}
      En effet, $\forall x \in {]-1,1[}$, $\sum_{k=0}^\infty x^k = \frac{1}{1-x}$. D'où, pour $x \in {]-1,1[}$, \[
        \sum_{k=1}^\infty k\,x^{k-1} = \frac{1}{(1-x)^2}
        \quad \text{ et }\quad
        \sum_{k=2}^\infty k(k-1)\,x^{k-2} = \frac{2}{(1-x)^3}
      .\]
      Ainsi, $\mathrm{E}(T^2) = \frac{q^{+1}}{p^2}$. D'où
      \begin{align*}
        \mathrm{V}(T) &= \mathrm{E}(T^2) - \big(\mathrm{E}(T)\big)^2 \\
        &= \frac{q+1}{p^2} - \left( \frac{1}{p} \right)^2 \\
        &= \frac{q}{p^2} \\
      \end{align*}
    \item À tenter
  \end{enumerate}
\end{exo}


\section{Les inégalités de \textsc{Markov} et de \textsc{Bienaymé}--\textsc{Tchebychev}, inégalités de concentration}

\begin{lem}[Markov]
  Soit $(\Omega, \mathcal{A}, P)$\/ un espace probabilisé, et soit $X$\/ une variable aléatoire \underline{positive}.
  Si $X$\/ est d'espérance finie, alors \[
    \forall a > 0, \quad P(X \ge a) \le \frac{\mathrm{E}(X)}{a}
  .\]
\end{lem}

\begin{prv}
  On suppose $X$\/ d'espérance finie. Ainsi, on a \[
    \mathrm{E}(X) = \sum_{x \in X(\Omega)} x\:P(X = x)
  .\]
  Soit $I$\/ l'ensemble $I = \{x \in X(\Omega) \mid x \ge a\}$.
  Alors, \[
    \mathrm{E}(X) = \underbrace{\sum_{x \in I} x\:P(X = x)}_{\text{ ici } x \ge a} + \underbrace{\sum_{x \in X(\Omega) \setminus I} x\:P(X = x)}_{\ge 0 \text{ par hypothèse}}
  .\]
  D'où, \[
    \mathrm{E}(X) \ge \sum_{x \in I} x\:P(X = x) \ge \sum_{x \in I} a\: P(X = x) = a \sum_{x \in I} P(X = x) \ge a\:P(x \ge a)
  .\]
\end{prv}

\begin{prop}[\textsc{Bienaymé--Tchebychev}]
  Soit $(\Omega, \mathcal{A}, P)$\/ un espace probabilisé, et soit $X$\/ une \textit{vard}. Si $X^2$\/ est d'espérance finie, alors \[
    \forall a > 0, \quad\quad P\Big(\big|X - \mathrm{E}(X)\big| \ge a\Big) \le \frac{\mathrm{V}(X)}{a^2}
  .\]
\end{prop}

\begin{prv}
  On pose $\mu = \mathrm{E}(X)$.
  L'événement $\big(|X - \mu| \ge a\big) = \big((X - \mu)^2 \ge a^2\big)$, d'où, les probabilités \[
    P\big(|X - \mu| \ge a\big) = P\big(\underbrace{(X - \mu)^2}_{\ge 0} \ge \underbrace{a^2}°{\ge 0}\big).
  \] On valide donc \textit{une} des hypothèses de l'inégalité de \textsc{Markov}.
  De plus, l'autre hypothèse est vérifiée : $X^2$\/ est d'espérance finie, donc $(X - \mu)^2$\/ aussi. On en déduit, d'après le lemme de \textsc{Markov}, que \[
    P\big((X-\mu)^2 \ge a^2\big) \le \frac{\mathrm{E}\big((X-\mu)^2\big)}{a^2} = \frac{\mathrm{V}(X)}{a^2}
  .\]
\end{prv}

\section{Série génératrice}

\begin{defn}
  Soit $X$\/ une \textit{vad} telle que $X(\Omega) \subset \N$. La \textit{série génératrice} de $X$\/ est la série entière $\sum a_n x^n$ de coefficients $a_n = P(X = n)$.
\end{defn}


La série $\sum a_n$\/ converge car sa somme vaut $\sum_{n=0}^\infty a_n = 1$. D'où, 
\begin{itemize}
  \item le rayon de convergence $R$\/ de la série est supérieur ou égal à 1.
  \item la série génératrice converge normalement sur $[-1,1]$, car la série $\sum |a_n|$\/ converge, or, $\forall x \in [-1,1]$, $|p_nt^n| \le |p_n|$, d'où la convergence normale.
    D'où la \textit{fonction génératrice} \[
      \mathrm{G}_X \colon t \longmapsto \sum_{n=0}^\infty p_n t^n
    \] est définie et même continue sur $[-1,1]$, car la convergence est uniforme.
  \item la fonction génératrice $\mathrm{G}_X$\/ est de classe $\mathcal{C}^\infty$\/ sur $]-1,1[$\/ et \[
      \forall k \in \N,\quad P(X = k) = a_n \frac{{\mathrm{G}_X}^{(k)}(0)}{k!}
    .\] La fonction génératrice de $X$\/ permet donc de retrouver la loi de probabilité de $X$.
\end{itemize}


		\begin{exo}
	Soient $\lambda_1, \ldots, \lambda_r \in \R$\/ distincts deux à deux.
	Montrons que, si $\forall x \in \R$, $\alpha_1 \mathrm{e}^{\lambda_1 x} + \cdots + \alpha_r \mathrm{e}^{\lambda_r x} = 0$, alors $\alpha_1 = \cdots = \alpha_r$.
	On peut procéder de différentes manières : le déterminant de {\sc Vandermonde}, par analyse-sythèse, ou, en utilisant \[
		\frac{\mathrm{d}}{\mathrm{d}x}\left( \mathrm{e}^{\lambda_k x} \right) = \lambda_k \mathrm{e}^{\lambda_k x},\quad\text{d'où}\quad\varphi(f_k) = \lambda_k f_k, \text{ avec } f_k : x \mapsto \mathrm{e}^{\lambda_k x}\quad\text{et}\quad\varphi:f\mapsto f'
	.\]
	On doit vérifier que les $f_k$\/ sont des vecteurs et l'application $\varphi$\/ soit un endomorphisme. On se place donc dans l'espace vectoriel $\mathscr{C}^{\infty}$. (On ne peut pas se placer dans l'espace $\mathscr{C}^k$, car sinon l'application $\varphi$\/ est de l'espace $\mathscr{C}^k$\/ à $\mathscr{C}^{k-1}$, ce n'est donc pas un endomorphisme ; ce n'est pas le cas pour l'espace $\mathscr{C}^\infty$.)
	Or, les $\lambda_k$\/ sont distincts deux à deux d'où les vecteurs propres $f_k$\/ sont linéairement indépendants. Et donc si $\alpha_1 f_1 + \alpha_2 f_2 + \cdots + \alpha_r f_r = 0$\/ alors $\alpha_1 = \cdots = \alpha_r=0$.
	Mais, comme $\forall x \in \R$, $\alpha_1 f_1(x) + \alpha_2 f_2(x) + \cdots + \alpha_r f_r(x) = 0$, on en déduit que \[
		\boxed{\alpha_1 = \cdots = \alpha_r = 0.}
	\]
\end{exo}

\section{Critères de diagonalisabilité}

\begin{prop}[une condition \underline{suffisante} pour qu'une matrice soit diagonalisable]
	Soit $A$\/ une matrice carrée de taille $n \ge 2$. {\color{red}Si} $A$\/ possède $n$\/ valeurs propres distinctes deux à deux, {\color{red}alors} $A$\/ est diagonalisable.
\end{prop}

\begin{rmkn}
	La réciproque est fausse : par exemple, pour $n > 1$, $7 I_n$\/ est diagonalisable car elle est diagonale. Mais, elle ne possède pas $n$\/ valeurs propres distinctes deux à deux.
\end{rmkn}

\begin{prv}
	On suppose que la matrice $A \in \mathscr{M}_{n,n}(\mathds{K})$\/ possède $n$\/ valeurs propres distinctes deux à deux (i.e.~$\Card \Sp(A) = n$). D'où, d'après la proposition 16, les $n$\/ vecteurs propres associés $\varepsilon_1,\ldots,\varepsilon_n$\/ sont libres. D'où $(\varepsilon_1, \ldots, \varepsilon_n)$\/ est une base formée de vecteurs propres. Donc, d'après la définition 5, la matrice $A$\/ est diagonalisable.
\end{prv}

\begin{thm}[conditions \underline{nécessaires et suffisantes} pour qu'une matrice soit diagonalisable]
	Soient $E$\/ un espace vectoriel de dimension finie et $u : E \to E$\/ un endomorphisme.
	Alors,
	\begin{align*}
		(1)\quad u \text{ diagonalisable } \iff& E = \bigoplus_{\lambda \in \Sp(u)} \Ker(\lambda\id_E - u) \quad(2)\\
		\iff& \dim E = \sum_{\lambda \in \Sp(u)} \dim(\mathrm{SEP}(\lambda))\quad(3)\\
		\iff& \chi_u \text{ scindé et } \forall \lambda \in \Sp(u),\:\dim(\mathrm{SEP}(\lambda)) = m_\lambda\quad(4)
	\end{align*}
	où $m_\lambda$\/ est la multiplicité de la racine $\lambda$\/ du polynôme $\chi_u$.
\end{thm}

\begin{prv}
	\begin{itemize}
		\item[``$(1)\implies(2)$''] On suppose $u$\/ diagonalisable. Il existe donc une base $(\varepsilon_1$, \ldots, $\varepsilon_n)$\/ de $E$\/ formée de vecteurs propres de $u$. On les regroupes par leurs valeurs propres : $(\varepsilon_i, \ldots, \varepsilon_{i+j})$\/ forme une base de $\mathrm{SEP(\lambda_k)}$. D'où la base $(\varepsilon_1, \ldots, \varepsilon_n)$\/ de l'espace vectoriel $E$\/ est une concaténation des bases des sous-espaces propres de $u$. D'où \[
				E = \bigoplus_{\lambda \in \Sp(u)} \mathrm{SEP}(\lambda)
			.\]
		\item[``$(2)\implies(1)$''] On suppose que $E = \mathrm{SEP}(\lambda_1) \oplus \mathrm{SEP}(\lambda_2) \oplus \cdots \oplus \mathrm{SEP}(\lambda_r)$.
			Soient $(\varepsilon_1, \ldots, \varepsilon_{d_1})$\/ une base de $\mathrm{SEP}(\lambda_1)$, $(\varepsilon_{d_1 + 1}, \ldots, \varepsilon_{d_1 + d_2})$\/ une base de $\mathrm{SEP}(\lambda_2)$, \ldots, $(\varepsilon_{d_1+\cdots + d_{r-1}+1}$, \ldots, $\varepsilon_{d_1+ \cdots + d_r})$\/ une base de $\mathrm{SEP}(\lambda_r)$.
			En concaténant ces base, on obtient une base de $E$, d'après l'hypothèse. Dans cette base, tous les vecteurs sont propres donc $u$\/ est diagonalisable.
		\item[``$(2)\implies(3)$''] On suppose $E = \mathrm{SEP}(\lambda_1) \oplus \mathrm{SEP}(\lambda_2) \oplus \cdots \oplus \mathrm{SEP}(\lambda_r)$. D'où  \[
					\dim E = \dim(\mathrm{SEP}(\lambda_1)) + \dim(\mathrm{SEP}(\lambda_2)) + \cdots + \dim(\mathrm{SEP}(\lambda_r))
			\] car la dimension d'une somme directe est égale à la somme des dimensions.
		\item[``$(3)\implies(1)$''] On suppose $\dim E = \dim(\mathrm{SEP}(\lambda_1)) + \dim(\mathrm{SEP}(\lambda_2)) + \cdots + \dim(\mathrm{SEP}(\lambda_r))$. Or, les sous-espaces propres sont en somme directe, d'après la proposition 16. D'où $\dim\Big(\sum_{\lambda \in \Sp(u)} \mathrm{SEP}(\lambda) \Big)= \sum_{\lambda \in \Sp(u)} \dim(\mathrm{SEP}(\lambda))$. Donc $\sum_{\lambda \in \Sp(u)} \mathrm{SEP}(\lambda) = E$.
		\item[``$(4)\implies(3)$''] On suppose (a) $\chi_u$\/ scindé et (b) $\dim(\mathrm{SEP}(\lambda)) = m_\lambda$. D'où, d'après (a): \[
				\chi_u(x) = (x - \lambda_1)^{m_{\lambda_1}}(x - \lambda_2)^{m_{\lambda_2}} \cdots (x - \lambda_r)^{m_{\lambda_r}} = x^n + \cdots
			\] d'où $m_{\lambda_1} + m_{\lambda_2} + \cdots + m_{\lambda_r} = n$, et d'où \[
				\dim(\mathrm{SEP}(\lambda_1)) + \dim(\mathrm{SEP}(\lambda_2)) + \cdots + \dim(\mathrm{SEP}(\lambda_r)) = n
			\] d'après l'hypothèse (b).
		\item[``$(1)\implies(4)$'']
			On suppose $u$\/ diagonalisable. D'où, dans une certaine base $\mathscr{B}$, la matrice $\big[u\big]_\mathscr{B}$\/ est diagonale. Quitte à changer l'ordre des éléments de $\mathscr{B} = (\varepsilon_1,\ldots,\varepsilon_r)$, on peut supposer que $\big[u\big]_\mathscr{B}$\/ est de la forme \[
				\big[u\big]_\mathscr{B} = 
				\begin{bNiceArray}{c|c|c|c}[last-col]
					\begin{array}{cccc}\lambda_1\\&\lambda_1\\&&\ddots\\&&&\lambda_1\end{array}&0&0&0&\begin{array}{l}\varepsilon_1\\\varepsilon_2\\\vdots\\\varepsilon_{d_1}\\\end{array}\\ \hline
					0&\begin{array}{cccc}\lambda_2\\&\lambda_2\\&&\ddots\\&&&\lambda_2\end{array}&0&0&\begin{array}{l}\varepsilon_{d_1+1}\\\varepsilon_{d_1+2}\\\vdots\\\varepsilon_{d_1+d_2}\\\end{array}\\ \hline
					 &&\ddots&&\vdots\\ \hline
					0&0&0&0\begin{array}{cccc}\lambda_r\\&\lambda_r\\&&\ddots\\&&&\lambda_r\end{array}&\begin{array}{l}\varepsilon_{d_1+\cdots + d_{r-1} + 1}\\\varepsilon_{d_1+\cdots + d_{r-1}+2}\\\vdots\\\varepsilon_{d_1 + \cdots + d_r}\\\end{array}\\
				\end{bNiceArray}
			.\] D'où $\forall k \in \left\llbracket 1,r \right\rrbracket$, $d_k = \dim(\mathrm{SEP}(\lambda_k))$. En outre, $\chi_u(x) = \det(x\id - u) = (x - \lambda_1)^{d_1}\cdot (x-\lambda_2)^{d_2} \cdots (r - \lambda_r)^{d_r}$.
			D'où $\forall k \in \left\llbracket 1,r \right\rrbracket$, $d_k = m_{\lambda_k}$\/ et $\chi_u$\/ est scindé.
	\end{itemize}
\end{prv}

\begin{exo}
	{\slshape On considère la matrice $E$\/ ci-dessous \[
		E = \begin{pmatrix}
			7&0&1\\
			0&3&0\\
			0&0&7
		\end{pmatrix}.
	\] La matrice $E$\/ ci-dessous est-elle diagonalisable ?}

	Soit $\lambda \in \R$. On sait que $\lambda \in \Sp(E)$\/ si et seulement si $\det(\lambda I_3 - E) = 0$. Or \[
		\det(\lambda I_3 - E) =
		\begin{vmatrix}
			\lambda - 7&0&-1\\
			0&\lambda-3&0\\
			0&0&\lambda - 7
		\end{vmatrix} = (\lambda - 7)^2\cdot  (\lambda - 3)^1
	.\] Donc $\Sp(E) = \{3,7\}$, $1 \le \dim\big(\mathrm{SEP}(3)\big) \le 1$, et $1 \le \dim\big(\mathrm{SEP}(7)\big) \le 2$.
	La matrice $E$\/ est diagonalisable si et seulement si $\dim(\mathrm{SEP}(3)) + \dim(\mathrm{SEP}(7)) = 3$, donc si et seulement si $\dim(\mathrm{SEP}(7)) = 2$. On cherche donc la dimension de ce sous-espace propre : soit $X = \left( \substack{x\\y\\z} \right) \in \mathscr{M}_{3,1}(\R)$. On sait que
	\begin{align*}
		X \in \mathrm{SEP}(7) \iff& E\cdot X = 7X\\
		\iff& \begin{pmatrix}
			7&0&1\\
			0&3&0\\
			0&0&7
		\end{pmatrix} \begin{pmatrix}
			x\\y\\z
		\end{pmatrix} = 7 \begin{pmatrix}
			x\\y\\z
		\end{pmatrix}\\
		\iff& \begin{cases}
			7x + 0y + 1z = 7x\\
			3y = 7y\\
			7z = 7z
		\end{cases}\\
		\iff& \begin{cases}
			z = 0\\
			y = 0
		\end{cases}\\
		\iff& X = \begin{pmatrix}
			x\\0\\y
		\end{pmatrix} = x \underbrace{\begin{pmatrix}
			1\\0\\0
		\end{pmatrix}}_{\varepsilon_1}
	\end{align*}
	Donc $\mathrm{SEP}(7) = \Vect(\varepsilon_1)$, d'où $\dim(\mathrm{SEP}(7)) = 1$. Donc la matrice $E$\/ n'est pas diagonalisable.
\end{exo}

\section{Trigonalisation}

Trigonaliser une matrice ne sert que si la matrice n'est pas diagonalisable.

\begin{defn}
	On dit d'une matrice carrée $A \in \mathscr{M}_{n,n}(\mathds{K})$\/ qu'elle est {\it trigonalisable}\/ s'il existe une matrice inversible $P$\/ telle que $P^{-1} \cdot A \cdot P$\/ est triangulaire : \[
		P^{-1} \cdot A \cdot P = \begin{pNiceMatrix}
			\lambda_1&\Block{2-2}*&\\
			\Block{2-2}0&\Ddots&\\
			&&\lambda_r
		\end{pNiceMatrix}
	.\]
\end{defn}

\begin{rmk}
	$\O$\/
\end{rmk}

\begin{thm}
	Une matrice carrée $A \in \mathscr{M}_{n,n}(\mathds{K})$\/ est trigonalisable si et seulement si son polynôme caractéristique $\chi_A \in \mathds{K}[X]$\/ est scindé.
\end{thm}

		\addrecap
	}
	\def\addmacros#1{#1}
}

{
	\chap[1]{Langages réguliers et Automates}
	\minitoc
	\renewcommand{\cwd}{../cours/chap01/}
	\addmacros{
		\section{Motivation}

\lettrine On place au centre de la classe 40 bonbons. On en distribue un chacun. Si, par exemple, chacun choisit un bonbon et, au \textit{top} départ, prennent celui choisi.
Il est probable que plusieurs choisissent le même. Comme gérer lorsque plusieurs essaient d'accéder à la mémoire ?

Deuxièmement, sur l'ordinateur, plusieurs applications tournent en même temps. Pour le moment, on considérait qu'un seul programme était exécuté, mais, le \textsc{pc} ne s'arrête pas pendant l'exécution du programme.

On s'intéresse à la notion de \guillemotleft~processus~\guillemotright\ qui représente une tâche à réaliser.
On ne peut pas assigner un processus à une unité de calcul, mais on peut \guillemotleft~allumer~\guillemotright\ et \guillemotleft~éteindre~\guillemotright\ un processus.
Le programme allumant et éteignant les processus est \guillemotleft~l'ordonnanceur.~\guillemotright\@ Il doit aussi s'occuper de la mémoire du processus (chaque processus à sa mémoire séparée).

On s'intéresse, dans ce chapitre, à des programmes qui \guillemotleft~partent du même~\guillemotright\ : un programme peut créer un \guillemotleft~fil d'exécution~\guillemotright\ (en anglais, \textit{thread}). Le programme peut gérer les fils d'exécution qu'il a créé, et éventuellement les arrêter.
Les fils d'exécutions partagent la mémoire du programme qui les a créé.

En C, une tâche est représenté par une fonction de type \lstinline[language=c]!void* tache(void* arg)!. Le type \lstinline[language=c]!void*!\ est l'équivalent du type \lstinline[language=caml]!'a! : on peut le \textit{cast} à un autre type (comme \lstinline[language=c]-char*-).

\begin{lstlisting}[language=c,caption=Création de \textit{thread}s en C]
void* tache(void* arg) {
	printf("%s\n", (char*) arg);
	return NULL;
}

int main() {
	pthread_t p1, p2;

	printf("main: begin\n");

	pthread_create(&p1, NULL, tache, "A");
	pthread_create(&p2, NULL, tache, "B");

	pthread_join(p1, NULL);
	pthread_join(p2, NULL);

	printf("main: end\n");

	return 0;
}
\end{lstlisting}

\begin{lstlisting}[language=c,caption=Mémoire dans les \textit{thread}s en C]
int max = 10;
volatile int counter = 0;

void* tache(void* arg) {
	char* letter = arg;
	int i;

	printf("%s begin [addr of i: %p] \n", letter, &i);

	for(i = 0; i < max; i++) {
		counter = counter + 1;
	}

	printf("%s : done\n", letter);
	return NULL;
}

int main() {
	pthread_t p1, p2;

	printf("main: begin\n");

	pthread_create(&p1, NULL, tache, "A");
	pthread_create(&p2, NULL, tache, "B");

	pthread_join(p1, NULL);
	pthread_join(p2, NULL);

	printf("main: end\n");

	return 0;
}
\end{lstlisting}

Dans les \textit{thread}s, les variables locales (comme \texttt{i}) sont séparées en mémoire. Mais, la variable \texttt{counter} est modifiée, mais elle ne correspond pas forcément à $2 \times \texttt{max}$. En effet, si \texttt{p1} et \texttt{p2} essaient d'exécuter au même moment de réaliser l'opération \lstinline[language=c]-counter = counter + 1-, ils peuvent récupérer deux valeurs identiques de \texttt{counter}, ajouter 1, puis réassigner \texttt{counter}.
Ils \guillemotleft~se marchent sur les pieds.~\guillemotright\ 

Parmi les opérations, on distingue certaines dénommées \guillemotleft~atomiques~\guillemotright\ qui ne peuvent pas être séparées. L'opération \lstinline[language=c]-i++- n'est pas atomique, mais la lecture et l'écriture mémoire le sont.

\begin{defn}
	On dit d'une variable qu'elle est \textit{atomique} lorsque l'ordonnanceur ne l'interrompt pas.
\end{defn}

\begin{exm}
	L'opération \lstinline[language=c]-counter = counter + 1- exécutée en série peut être représentée comme ci-dessous. Avec \texttt{counter} valant 40, cette exécution donne 42.
	\begin{table}[H]
		\centering
		\begin{tabular}{l|l}
			Exécution du fil A & Exécution du fil B\\ \hline
			(1)~$\mathrm{reg}_1 \gets \texttt{counter}$ & (4)~$\mathrm{reg}_2 \gets \texttt{counter}$ \\
			(2)~$\mathrm{reg}_1{++}$ & (5)~$\mathrm{reg}_2{++}$ \\
			(3)~$\texttt{counter} \gets \mathrm{reg}_1$ & (6)~$\texttt{counter} \gets \mathrm{reg}_2$
		\end{tabular}
	\end{table}
	\noindent Mais, avec l'exécution en simultanée, la valeur de \texttt{counter} sera 41.
	\begin{table}[H]
		\centering
		\begin{tabular}{l|l}
			Exécution du fil A & Exécution du fil B\\ \hline
			(1)~$\mathrm{reg}_1 \gets \texttt{counter}$ & (2)~$\mathrm{reg}_2 \gets \texttt{counter}$ \\
			(3)~$\mathrm{reg}_1{++}$ & (5)~$\mathrm{reg}_2{++}$ \\
			(4)~$\texttt{counter} \gets \mathrm{reg}_1$ & (6)~$\texttt{counter} \gets \mathrm{reg}_2$
		\end{tabular}
	\end{table}
	\noindent Il y a \textit{entrelacement} des deux fils d'exécution.
\end{exm}

\begin{rmk}[Problèmes de la programmation concurrentielle]
	\begin{itemize}
		\item Problème d'accès en mémoire,
		\item Problème du rendez-vous,\footnote{Lorsque deux programmes terminent, ils doivent s'attendre pour donner leurs valeurs.}
		\item Problème du producteur-consommateur,\footnote{Certains programmes doivent ralentir ou accélérer.}
		\item Problème de l'entreblocage,\footnote{\textit{c.f.} exemple ci-après.}
		\item Problème famine, du dîner des philosophes.\footnote{Les philosophes mangent autour d'une table, et mangent du riz avec des baguettes. Ils décident de n'acheter qu'une seule baguette par personne. Un philosophe peut, ou penser, ou manger. Mais, pour manger, ils ont besoin de deux baguettes. S'ils ne mangent pas, ils meurent.}
	\end{itemize}
\end{rmk}

\begin{exm}[Problème de l'entreblocage]~

	\begin{table}[H]
		\centering
		\begin{tabular}{l|l|l}
			Fil A & Fil B & Fil C\\ \hline
			RDV(C) & RDV(A) & RDV(B)\\
			RDV(B) & RDV(C) & RDV(A)\\
		\end{tabular}
		\caption{Problème de l'entreblocage}
	\end{table}
\end{exm}

Comment résoudre le problème des deux incrementations ? Il suffit de \guillemotleft~mettre un verrou.~\guillemotright\ Le premier fil d'exécution \guillemotleft~s'enferme~\guillemotright\ avec l'expression \lstinline[language=c]!count++!, le second fil d'exécution attend que l'autre sorte pour pouvoir entrer et s'enfermer à son tour.


		\section{Continuité}

\begin{exm}
	Dans l'exercice 2, chaque fonction $f_n : t \mapsto t^n$\/ est continue sur $[0,1]$\/ mais la limite $f$\/ n'est pas continue sur $[0,1]$\/ (car elle n'est pas continue en $1$).
\end{exm}

\begin{thm}
	Soit $a$\/ un réel dans un intervalle $T$\/ de $\R$. Si une suite de fonctions $(f_n)_{n\in\N}$\/ continues en $a$\/ converge uniformément sur $T$\/ vers une fonction $f$, alors $f$\/ est aussi continue en $a$.
\end{thm}

\begin{prv}
	On suppose les fonctions $f_n$\/ continues en $a$\/ ($f_n(x) \longrightarrow f_n(a)$) et que la suite de fonctions $(f_n)_{n\in\N}$\/ converge uniformément vers $f$\/ ($\sup\:|f_n -f| \longrightarrow 0$). On veut montrer que $f$\/ est continue en $a$\/ : $f(x) \tendsto{x \to a} f(a)$, i.e.\ \[
		\forall \varepsilon > 0,\:\exists \delta > 0,\: \forall x \in T,\quad|x-a| \le \delta \implies |f(x) - f(a)| \le \varepsilon
	.\]
	Soit $\varepsilon > 0$. On calcule \[
		\big|f(x) - f(a)\big| \le \big|f(x) - f_n(x)\big| + \big|f_n(x) - f_n(a)\big| + \big|f_n(a) - f(a)\big|
	\] par inégalité triangulaire. Or, par hypothèse, il existe un rang $N \in \N$\/ (qui ne dépend pas de $x$\/ ou de $a$), tel que, $\forall n \ge N$, $\big|f(x) - f_n(x)\big| \le \frac{1}{3} \varepsilon$, et $\big|f_n(a) - f(a)\big| \le \frac{1}{3} \varepsilon$.
	De plus, par hypothèse, il existe $\delta >0$\/ tel que si $|x - a| \le \delta$, alors $|f_n(x) - f_n(a)| \le \frac{1}{3}\varepsilon$.\footnote{C'est là où l'hypothèse de la convergence uniforme est utilisée : on a besoin que le $N$\/ ne dépende pas de $x$\/ car on le fait varier.}
	On en déduit que $\big|f(x) - f(a)\big| \le \varepsilon$.
\end{prv}

\begin{crlr}
	Soit $T$\/ un intervalle de $\R$. Si une suite de fonctions $(f_n)_{n\in\N}$\/ continues sur $T$\/ converge uniformément sur $T$\/ vers une fonction continue sur $T$.
\end{crlr}

\begin{met}[Stratégie de la barrière]
	\begin{enumerate}
		\item La continuité (la dérivabilité aussi) est une propriété {\it locale}. Pour montrer qu'une fonction est continue sur un intervalle $T$, il suffit donc de montrer qu'elle est continue sur tout segment inclus dans $T$.
		\item Mais, la convergence uniforme est une propriété {\it globale}. La convergence sur tout segment inclus dans un intervalle n'implique pas la convergence uniforme sur l'intervalle (voir l'exercice 2).
		\item On n'écrit pas \[
				\substack{\ds\text{convergence uniforme}\\\ds\text{avec barrière}} \mathop{\red\implies} \substack{\ds\text{convergence uniforme}\\\ds\text{sans barrière}} \implies \substack{\ds\text{continuité}\\\ds\text{sans barrière}}
			\] mais plutôt \[
				\substack{\ds\text{convergence uniforme}\\\ds\text{avec barrière}} \implies \substack{\ds\text{continuité}\\\ds\text{avec barrière}} \implies \substack{\ds\text{continuité}\\\ds\text{sans barrière}}
			.\]
		\item Si, pour tous $a$\/ et $b$, $f$\/ est bornée sur $[a,b] \subset T$, mais cela n'implique pas que $f$\/ est bornée. Contre-exemple : la fonction $f : x \mapsto \frac{1}{x}$\/ est bornée sur tout intervalle $[a,b]$\/ avec $a$, $b \in \R^+_*$, \red{\sc mais} $f$\/ n'est pas bornée sur $]0,+\infty[$.
	\end{enumerate}
\end{met}

\begin{thm}[double-limite ou d'interversion des limites]
	Soit une suite de fonctions $(f_n)_{n\in\N}$\/ définies sur un intervalle $T$, et, soit $a$\/ une extrémité (éventuellement infinie)\footnote{autrement dit, $a \in \bar\R = \R \cup \{+\infty,-\infty\}$} de cet intervalle. Si la suite de fonctions $(f_n)_{n\in\N}$\/ converge \underline{uniformément} sur $T$\/ vers $f$\/ et si chaque fonction $f_n$\/ admet une limite finie $b_n$\/ en $a$, alors la suite de réels $b_n$\/ converge vers un réel $b$, et $\lim_{t\to a} f(t) = b$. Autrement dit, \[
		\lim_{t\to a} \Big(\underbrace{\lim_{n\to +\infty} f_n(t)}_{f(x)}\Big) = \lim_{n\to +\infty} \Big(\underbrace{\lim_{t\to a} f_n(t)}_{b_n}\Big)
	.\] \qed
\end{thm}

\begin{rmkn}
	Le théorème de la double-limite \guillemotleft~contient~\guillemotright\ le théorème 6 (théorème de préservation/transmission de la continuité), c'est un cas particulier. En effet, si les fonctions $f_n$\/ sont continues, alors \[
		\lim_{x \to a}f(x) = \underbrace{\lim_{n\to +\infty} f_n(a)}_{f(a)}
	.\]
\end{rmkn}


		\section{Endomorphismes adjoints}

\begin{defn}
	On dit qu'un endomorphisme $f : E \to E$\/ est \textit{autoadjoint} si \[
		\forall (\vec{u}, \vec{v}) \in E^2,\quad \big<f(\vec{u})\:\big|\:\vec{v}\big> = \big<\vec{u}\:\big|\:f(\vec{v})\big>
	.\] 
\end{defn}

Un endomorphisme autoadjoint est aussi appelé endomorphisme \textit{symétrique} (\textit{c.f.}\ proposition suivante). L'ensemble des endomorphismes autoadjoints est noté $\mathcal{S}(E)$.

\begin{prop}
	Un endomorphism est autoadjoint si, et seulement si la matrice de $F$\/ dans une base \ul{orthonormée} $\mathcal{B}$\/ est orthogonale.
	Autrement dit : \[
		f \in \mathcal{S}(E) \iff \big[\:f\:\big]_\mathcal{B} \in \mathcal{S}_n(\R)
	.\]
\end{prop}

\begin{prv}
	\begin{description}
		\item[$\implies$] Soit $\mathcal{B} = (\vec{\varepsilon}_1, \ldots, \vec{\varepsilon}_n)$\/ une base orthonormée de $E$. Ainsi, \[
				\forall i,\:\forall j,\quad \big<f(\vec{\varepsilon}_i)\:\big|\: \vec{\varepsilon}_j\big> = \big<\vec{\varepsilon}_i\:\big|\:f(\vec{\varepsilon}_j)\big>
			.\] On pose $\big[\:f\:\big]_{\mathcal{B}} = (a_{i,j})$\/ : \[
				\begin{pNiceMatrix}[last-col,last-row]
					\quad&\quad&a_{1,j}&\quad&\quad&\vec{\varepsilon}_i\\
					&&&&&\\
					\quad&\quad&a_{i,j}&\quad&\quad&\vec{\varepsilon}_i\\
					&&&&&\\
					\quad&\quad&a_{n,j}&\quad&\quad&\vec{\varepsilon}_n\\
					f(\vec{\varepsilon}_i)&&f(\vec{\varepsilon}_j)&&f(\vec{\varepsilon}_n)\\
				\end{pNiceMatrix}
			.\] Ainsi, $f(\vec{\varepsilon}_j) = a_{1,j} \vec{\varepsilon}_1 + \cdots + a_{i,j} \vec{\varepsilon}_i + \cdots + a_{n,j} \vec{\varepsilon}_n$. D'où, $\left<\vec{\varepsilon}_i  \mid f(\vec{\varepsilon}_j) \right> = a_{i,j}$\/ car la base $\mathcal{B}$\/ est orthonormée.
			De même avec l'autre produit scalaire, $\left< f(\vec{\varepsilon}_i)  \mid \vec{\varepsilon}_j \right>$, d'où $a_{i,j} = a_{j,i}$\/ par symétrie du produit scalaire. On en déduit que $\big[\:f\:\big]_\mathcal{B} \in \mathcal{S}_n(\R)$.
		\item[$\impliedby$]
			Si $\big[\: f\:\big]_\mathcal{B} \in \mathcal{S}_n(\R)$, alors $\left<f(\vec{\varepsilon}_i)  \mid \vec{\varepsilon}_j \right> = \left<\vec{\varepsilon}_i  \mid f(\vec{\varepsilon}_j)\right>$.
			Or, on pose $\vec{u} = x_1 \vec{\varepsilon}_1+ \cdots + x_n \vec{\varepsilon}_n$, et $\vec{v} = y_1 \vec{\varepsilon}_1 + \cdots + y_n \vec{\varepsilon}_n$.
			\begin{align*}
				\left<f(\vec{u})  \mid \vec{v} \right> &= \left<x_1 f(\vec{\varepsilon}_1) + \cdots + x_n f(\vec{\varepsilon}_n)  \mid y_1 f(\vec{\varepsilon}_1) + \cdots + y_n f(\vec{\varepsilon}_n) \right> \\
				&= \Big<\sum_{i=1}^n x_i f(\vec{\varepsilon}_i)\:\Big|\: \sum_{j=1}^n y_j \vec{\varepsilon}_j\Big> \\
				&= \sum_{i,j \in \llbracket 1,n \rrbracket}  x_i y_j \left<f(\vec{\varepsilon}_i) \mid \vec{\varepsilon}_j \right>\\
			\end{align*}
			De même en inversant $\vec{u}$\/ et $\vec{v}$.
			On en déduit donc $\left<f(\vec{u} \mid \vec{v} \right> = \left<\vec{u}  \mid f(\vec{v}) \right>$.
	\end{description}
\end{prv}

\begin{exo}
	\begin{enumerate}
		\item Si $f$\/ est autoadjoint, montrons que $\Ker f \perp \Im f$, et $\Ker f \oplus \Im f$.
			On suppose $\forall \vec{u}$, $\forall \vec{v}$, $\left<f(\vec{u}) \mid \vec{v} \right> = \left<\vec{u}  \mid f(\vec{v}) \right>$.
			Soit $\vec{u} \in \Ker f$, et soit $\vec{v} \in \Im f$.
			On sait que $f(\vec{u}) = \vec{0}$, et qu'il existe $\vec{x} \in E$\/ tel que $\vec{v} = f(\vec{x})$.
			Ainsi, \[
				\left<\vec{u}  \mid \vec{v} \right>
				= \left<\vec{u}  \mid f(\vec{x}) \right>
				= \left<f(\vec{u})  \mid \vec{x} \right>
				= 0
			.\] 
			D'où $\vec{u} \perp \vec{v}$. Ainsi, $\Ker f \perp \Im f$.


			De plus, $E$\/ est de dimension finie, d'où, d'après le théorème du rang, \[
				\dim \Ker f + \dim \Im f = \dim E
			.\] Aussi, $\Ker f \oplus (\Ker f)^\perp = E$, donc $\dim(\Ker f) + \dim(\Ker f)^\perp = \dim E$.
			On en déduit donc que $\dim(\Im f)= \dim(\Ker f)^\perp$.
			Or, $\Im f \subset (\Ker f)^\perp$\/ car $\Im f \perp \Ker f$.
			Ainsi $\Im f = (\Ker f)^\perp$, on en déduit que \[
				\Im f \oplus \Ker f = E
			.\]
			\begin{description}
				\item[$\impliedby$] 
					Soit $p$\/ la projection sur $F$\/ parallèlement à $G$.
					Supposons l'endomorphisme $P$\/ autoadjoint.
					D'après la question 1., le $\Ker p \perp \Im p$.
					Ainsi, $F = \Im p$\/ et $G = \Ker p$.
					D'où, $F \perp G$, $p$\/ est donc une projection orthogonale.
				\item[$\implies$]
					Réciproquement, supposons $p$\/ une projection orthogonale.
					Soit $\mathcal{B} = (\vec{\varepsilon}_1, \ldots, \vec{\varepsilon}_q)$\/ une base orthonormée de $F$.
					Ainsi, pour tout $\vec{x} \in E$, \[
						p(\vec{x}) = \sum_{i = 1}^q \left<\vec{x}  \mid \vec{\varepsilon}_i \right>\,\vec{\varepsilon}_i
					.\] 
					On veut montrer que l'endomorphisme $p$\/ est autoadjoint.
					Soient $\vec{u}$\/ et $\vec{v}$\/ deux vecteurs de $E$.
					\begin{align*}
						\left<p(\vec{u})  \mid \vec{v} \right>
						= \Big<\sum_{i=1}^q \left< \vec{u}\mid \vec{\varepsilon}_i \right>\vec{\varepsilon}_i\:\Big|\; \vec{v}\;\Big>
						&= \sum_{i=1}^q \left<u  \mid \vec{\varepsilon}_{i} \right>\: \left< \vec{\varepsilon}_i  \mid v\right>\\
						&= \sum_{i=1}^q \left<v  \mid \vec{\varepsilon}_i \right>\:\left<\vec{\varepsilon}_i   \mid u\right> \\
						&= \left< \vec{u}  \mid p(\vec{v}) \right> \\
					\end{align*}

					Autre méthode, pour tous vecteurs $\vec{u}$\/ et $\vec{v}$\/ de $E$,
					\begin{align*}
						\left<p(\vec{u})  \mid \vec{v} \right>
						&= \left<p(\vec{u})  \mid p(\vec{v}) + \vec{v} - p(\vec{v}) \right> \\
						&= \left< p(\vec{u})  \mid p(\vec{v}) \right> + \left<p(\vec{u})  \mid  \vec{v} - p(\vec{v}) \right> \\
						&= \left<p(\vec{u}) \mid p(\vec{v}) \right> + \left<u - p(\vec{u})  \mid p(\vec{v}) \right> \\
						&= \left<\vec{u}  \mid p(\vec{v}) \right> \\
					\end{align*}
					car $p$\/ est orthogonale.
			\end{description}
	\end{enumerate}
\end{exo}


\begin{prop-defn}
	Si $f$\/ est un endomorphisme d'un espace euclidien $E$, alors il existe un unique endomorphisme de $E$, noté $f^\star$\/ et appelé l'\textit{adjoint} de $f$, tel que \[
		\forall (\vec{u},\vec{v}) \in E^2,\quad\quad \left<f^\star(\vec{u})  \mid \vec{v} \right> =  \left<\vec{u}  \mid f(\vec{v}) \right>
	.\] 
	Si $A$\/ est la matrice $f$\/ dans une base orthonormée $\mathcal{B}$\/ de $E$, alors $A^\top$\/ est la matrice de $f^\star$\/ dans~$\mathcal{B}$\/ : \[
		\big[\:f\:\big]_\mathcal{B} = \big[\:f\:\big]_\mathcal{B}^\top
	.\]
\end{prop-defn}

\begin{prv}
	Soit $\vec{u} \in E$. L'application \begin{align*}
		\varphi: E &\longrightarrow \R \\
		\vec{v} &\longmapsto \left<\vec{u}  \mid f(\vec{v}) \right>.
	\end{align*}
	La forme $\varphi$\/ est linéaire car $\varphi(\alpha_1 \vec{v}_1 + \alpha_2 \vec{v}_2) = \left<\vec{u} \mid f(\alpha_1 \vec{v}_1 + \alpha_2 \vec{v}_2) \right> = \left<\vec{u}  \mid \alpha_1 f(\vec{v}_1) + \alpha_2 f(\vec{v}_2) \right> = \alpha_1\left<\vec{u}  \mid  f(\vec{v}) \right> + \alpha_2 \left<\vec{u}  \mid f(\vec{v}_2) \right> = \alpha_1 \varphi(\vec{v}_1) + \alpha_2 \varphi(\vec{v}_2)$.
	D'où, d'après le théorème de \textsc{Riesz}, il existe un \ul{unique} vecteur $\vec{a} \in E$\/ tel que $\varphi(\vec{v}) = \left<\vec{a}  \mid \vec{v} \right>$\/ pour tout $\vec{v} \in E$.
	Ainsi, pour tout vecteur $\vec{v} \in E$, $\left<\vec{u}  \mid f(\vec{v}) \right> = \left<\vec{a}  \mid \vec{v} \right>$.
	On note $\vec{a} = f^\star(\vec{u})$.
	Soit l'application \begin{align*}
		f^\star : E &\longrightarrow E \\
		\vec{u} &\longmapsto f^\star(\vec{u}).
	\end{align*}
	La démonstration telle que $f^\star $\/ est linéaire est dans le poly.
	L'application $f^\star$\/ vérifie : $\left< \vec{u} \mid f(\vec{v}) \right> = \left<f^\star (\vec{u})  \mid \vec{v} \right>$, pour tous vecteurs $\vec{u}$\/ et $\vec{v}$.
	\textsl{Quelle est la matrice de $f^\star$, dans une base orthonormée ?}\@
	Soit $\mathcal{B}$\/ une base orthonormée de $E$, et soient $A = \big[\:f\:\big]_\mathcal{B}$, $B = \big[\:f^\star \:\big]_\mathcal{B}$, $U = \big[\:\vec{u}\:\big]_\mathcal{B}$, et $V = \big[\:\vec{v}\:\big]_\mathcal{B}$.
	Les matrices $U$\/ et $V$\/ sont des vecteurs colonnes, et $A$\/ et $B$\/ sont des matrices carrées.
	Ainsi, \[
		U^\top \cdot A \cdot V = \left< \vec{u}  \mid f(\vec{v}) \right> 
		= \left<f^\star (\vec{u})  \mid \vec{v} \right> = (B\cdot U)^\top \cdot V,
	\] ce qui est vrai quelque soit les vecteurs colonnes $U$\/ et $V$.
	D'où, $\forall U$, $\forall V$, $U^\top \cdot \big(A \cdot V\big) = U^\top \cdot \big(B^\top \cdot V\big)$.
	Ainsi, pour tous vecteurs $U$\/ et $V$, \[
		U^\top \cdot \Big[ (AV) - (B^\top V)\Big] = 0
	.\] En particulier, si $U = (AV) - (B^\top V)$, le produit scalaire $\left<\vec{u}  \mid \vec{u} \right>$\/ est nul, donc $U = 0$.
	Ainsi, \[
		\forall V,\quad A\cdot V = B^\top\cdot  V
	.\] De même, on conclut que $A = B^\top$. On en déduit donc que \[
	\big[\:f^\star\:\big]_\mathcal{B} = \big[\:f\:\big]_\mathcal{B}^\top
	.\]
\end{prv}

Les propriétés suivantes sont vrais :
\begin{itemize}
	\item $(f  \circ g)^\star  = g^\star \circ f^\star$, \quad $(f^\star)^\star = f$, \quad et \quad $(\alpha f + \beta g)^\star  = \alpha f^\star + \beta g^\star $\/ ;
	\item $(A\cdot B)^\top  = B^\top \cdot A^\top$, \quad $(A^\top)^\top = A$, \quad et \quad $(\alpha A + \beta B)^\top = \alpha A^\top+ \beta B^\top $.
\end{itemize}
Des deuxièmes et troisièmes points,  il en résulte que les applications $f \mapsto f^\star$, et $A \mapsto A^\top$\/ sont des applications involutives.



		\begin{prop-defn}
  Soit $(\Omega, \mathcal{A}, P)$\/ un espace probabilisé, et soit $X$\/ une \textit{vard}. Si $X^2$\/ est d'espérance finie, alors $X$\/ aussi, et on appelle \textit{variance} le réel positif \[
    \mathrm{V}(X) = \mathrm{E}\Big(\big[X - \mathrm{E}(X)\big]^2\Big) = \underbrace{\mathrm{E}(X^2) - \big(\mathrm{E}(X)\big)^2}_{\mathclap{\text{Relation de \textsc{König \& Huygens}}}} \ge 0
  .\]
  L'\textit{écart-type} $\sigma(X)$\/ est la racine carrée de la variance : \[
    \sigma(X) = \sqrt{\mathrm{V}(X)}
  .\]
\end{prop-defn}

\begin{prv}
  On pose $\mu = \mathrm{E}(X)$, et on a $[X - \mu]^2 = X^2 - 2 \mu X + \mu^2$. D'où, par linéarité de l'espérance,
  \begin{align*}
    \mathrm{E}\big((X-\mu)^2\big)
    &= \mathrm{E}(X^2 - 2\mu X + \mu^2) \\
    &= \mathrm{E}(X^2) - 2\mu \mathrm{E}(X) + \mu^2 \\
    &= \mathrm{E}(X^2) - 2\mu^2 + \mu^2 \\
    &= \mathrm{E}(X^2) - \big(\mathrm{E}(X)\big)^2. \\
  \end{align*}
  De plus, d'après le lemme précédent, si $X^2$\/ est d'espérance finie, alors $X$\/ est d'espérance finie.
\end{prv}

\begin{rmk}
  \begin{enumerate}
    \item La variance mesure la \textit{dispersion}, ou l'\textit{étalement} des valeurs $a_i$\/ autour de l'espérance $\mathrm{E}(X)$. En particulier, s'il existe $a \in \R$\/ tel que $P(X = a) = 1$, alors $\mathrm{E}(X) = a$\/ et $\mathrm{V}(X) = 0$. (C'est même une équivalence.)
    \item Si la variable $X$\/ a une unité ($\mathrm{km}/\mathrm{s}$, $\mathrm{V}/\mathrm{m}$, etc.), alors l'écart type a la même unité (d'où l'intérêt de calculer la racine carrée de la variance).
    \item Soient $\alpha$\/ et $\beta$\/ deux réels. Si  $X^2$\/ est d'espérance finie, alors \[
      \mathrm{V}(\alpha X + \beta) = \alpha^2\cdot  \mathrm{V}(X)
    .\]
    (Une translation ne change pas la dispersion des valeurs, et multiplier par un réel multiplie l'espérance, mais aussi la dispersion, d'où le carré.)
  \end{enumerate}
\end{rmk}

\begin{exo}
  \textsl{Montrer que
  \begin{enumerate}
    \item si $X \sim \mathcal{B}(n, p)$, alors $X^2$\/ est d'espérance finie et $\mathrm{V}(X) = n\cdot p\cdot q$.
    \item si $T \sim \mathcal{G}(p)$, alors $T^2$\/ est d'espérance finie et $\mathrm{V}(T) = \frac{q}{p^2}$.
    \item si $X \sim \mathcal{P}(\lambda)$, alors $X^2$\/ est d'espérance finie et $\mathrm{V}(X) = \lambda$.
  \end{enumerate}
  }

  \begin{enumerate}
    \item
      Si $X \sim \mathcal{B}(n,p)$, alors $X(\Omega) = \llbracket 0,n \rrbracket$\/ et, pour $k \in X(\Omega)$, $P(X = k) = {n\choose k}\,p^k\,q^{n-k}$.
      On a déjà montré que $\mathrm{E}(X) = n\cdot p$.
      On va montrer que $\mathrm{V}(X) = n\,p\,q$.
      La variable aléatoire $X^2$\/ est d'espérance finie car $X(\Omega)$\/ est fini.
      Et,
      \begin{align*}
        \mathrm{E}(X^2) &= \sum_{k=0}^n k^2\: P(X = k)\\
        &= \sum_{k=0}^n k^2 {n\choose p} p^k q^{n-k} \\
        &= \ldots \\
      \end{align*}
      En effet, d'après la ``petite formule,'' on a \[
        \forall k \ge 1,\quad k{n\choose k} = n{n-1 \choose k-1}
      \] d'où, $(k-1) {n-1\choose k-1} = (n-1) \choose {n-2 \choose k-2}$. Ainsi, \[
        \forall k \ge 2,\quad k(k-1){n\choose k} = n(n+1) {n-2\choose k-2}
      .\] 
    \item Si $T \sim \mathcal{G}(p)$, alors $T(\Omega) = \N^*$\/ et $\forall k \in T(\Omega)$, $P(T = k) = p \times q^{k-1}$.
      On a déjà prouvé que $\mathrm{E}(T) = \frac{1}{p}$.
      On veut montrer que $\mathrm{V}(T) = \frac{q}{p^2}$. Montrons que la variable $T^2$\/ possède une espérance : la série $\sum k^2\: P(T = k)$\/ converge absolument car $k^2 \:P(T = k) = k^2 \cdot p \cdot q^{k-1}$.
      Or, pour $k \ge 2$, $\frac{\mathrm{d}^2}{\mathrm{d}x^2} x^k = k(k-1)\,x^{k-2}$. Et, on peut dériver terme à terme une série entière sans changer son rayon de convergence, et la série $\sum x^k$\/ a pour rayon de convergence 1. D'où, $\sum k(k-1)\, x^{k-2}$\/ a pour rayon de convergence 1. Or, $q \in {]0,1[} \subset {]-1,1[}$\/ donc la série $\sum k(k-1)q^{k-2}$\/ converge. De plus, $\sum k (k-1)\, q^{k-2} = \sum k^2\,q^{k-2} - \sum k\,q^{k-2}$.
      D'où,  $\sum k^2 q^{k-2} = \sum k(k-1)\, q^{k-2} + \sum k\,q^{k-2}$, qui converge. Par suite,
      \begin{align*}
        \sum_{k=1}^\infty k^2\, P(T = k) &= \sum_{k=1}^\infty k^2\,p\,q^{k-1}\\
        &= p + pq \sum_{k=2}^\infty k^2 q^{k-2} \\
        &= p + pq \sum_{k=2}^\infty k(k-1)\,q^{k-2} + p\sum_{k=2}^\infty k\,q^{k-1} \\
        &= p + pq\, \frac{2}{(1-q)^3} + p\left(\frac{1}{(1-q)^2} - 1 \right) \rlap{\quad\quad \text{\textit{c.f.} en effet après}}\\
        &= p + pq\, \frac{2}{p^3} + p\left( \frac{1}{p^2} - 1 \right) \\
        &= \frac{2q}{p^2} + \frac{1}{p} \\
        &= \frac{2q + p}{p^2} \\
        &= \frac{2q + (1-q)}{p^2} \\
        &= \frac{q+1}{p^2}. \\
      \end{align*}
      En effet, $\forall x \in {]-1,1[}$, $\sum_{k=0}^\infty x^k = \frac{1}{1-x}$. D'où, pour $x \in {]-1,1[}$, \[
        \sum_{k=1}^\infty k\,x^{k-1} = \frac{1}{(1-x)^2}
        \quad \text{ et }\quad
        \sum_{k=2}^\infty k(k-1)\,x^{k-2} = \frac{2}{(1-x)^3}
      .\]
      Ainsi, $\mathrm{E}(T^2) = \frac{q^{+1}}{p^2}$. D'où
      \begin{align*}
        \mathrm{V}(T) &= \mathrm{E}(T^2) - \big(\mathrm{E}(T)\big)^2 \\
        &= \frac{q+1}{p^2} - \left( \frac{1}{p} \right)^2 \\
        &= \frac{q}{p^2} \\
      \end{align*}
    \item À tenter
  \end{enumerate}
\end{exo}


\section{Les inégalités de \textsc{Markov} et de \textsc{Bienaymé}--\textsc{Tchebychev}, inégalités de concentration}

\begin{lem}[Markov]
  Soit $(\Omega, \mathcal{A}, P)$\/ un espace probabilisé, et soit $X$\/ une variable aléatoire \underline{positive}.
  Si $X$\/ est d'espérance finie, alors \[
    \forall a > 0, \quad P(X \ge a) \le \frac{\mathrm{E}(X)}{a}
  .\]
\end{lem}

\begin{prv}
  On suppose $X$\/ d'espérance finie. Ainsi, on a \[
    \mathrm{E}(X) = \sum_{x \in X(\Omega)} x\:P(X = x)
  .\]
  Soit $I$\/ l'ensemble $I = \{x \in X(\Omega) \mid x \ge a\}$.
  Alors, \[
    \mathrm{E}(X) = \underbrace{\sum_{x \in I} x\:P(X = x)}_{\text{ ici } x \ge a} + \underbrace{\sum_{x \in X(\Omega) \setminus I} x\:P(X = x)}_{\ge 0 \text{ par hypothèse}}
  .\]
  D'où, \[
    \mathrm{E}(X) \ge \sum_{x \in I} x\:P(X = x) \ge \sum_{x \in I} a\: P(X = x) = a \sum_{x \in I} P(X = x) \ge a\:P(x \ge a)
  .\]
\end{prv}

\begin{prop}[\textsc{Bienaymé--Tchebychev}]
  Soit $(\Omega, \mathcal{A}, P)$\/ un espace probabilisé, et soit $X$\/ une \textit{vard}. Si $X^2$\/ est d'espérance finie, alors \[
    \forall a > 0, \quad\quad P\Big(\big|X - \mathrm{E}(X)\big| \ge a\Big) \le \frac{\mathrm{V}(X)}{a^2}
  .\]
\end{prop}

\begin{prv}
  On pose $\mu = \mathrm{E}(X)$.
  L'événement $\big(|X - \mu| \ge a\big) = \big((X - \mu)^2 \ge a^2\big)$, d'où, les probabilités \[
    P\big(|X - \mu| \ge a\big) = P\big(\underbrace{(X - \mu)^2}_{\ge 0} \ge \underbrace{a^2}°{\ge 0}\big).
  \] On valide donc \textit{une} des hypothèses de l'inégalité de \textsc{Markov}.
  De plus, l'autre hypothèse est vérifiée : $X^2$\/ est d'espérance finie, donc $(X - \mu)^2$\/ aussi. On en déduit, d'après le lemme de \textsc{Markov}, que \[
    P\big((X-\mu)^2 \ge a^2\big) \le \frac{\mathrm{E}\big((X-\mu)^2\big)}{a^2} = \frac{\mathrm{V}(X)}{a^2}
  .\]
\end{prv}

\section{Série génératrice}

\begin{defn}
  Soit $X$\/ une \textit{vad} telle que $X(\Omega) \subset \N$. La \textit{série génératrice} de $X$\/ est la série entière $\sum a_n x^n$ de coefficients $a_n = P(X = n)$.
\end{defn}


La série $\sum a_n$\/ converge car sa somme vaut $\sum_{n=0}^\infty a_n = 1$. D'où, 
\begin{itemize}
  \item le rayon de convergence $R$\/ de la série est supérieur ou égal à 1.
  \item la série génératrice converge normalement sur $[-1,1]$, car la série $\sum |a_n|$\/ converge, or, $\forall x \in [-1,1]$, $|p_nt^n| \le |p_n|$, d'où la convergence normale.
    D'où la \textit{fonction génératrice} \[
      \mathrm{G}_X \colon t \longmapsto \sum_{n=0}^\infty p_n t^n
    \] est définie et même continue sur $[-1,1]$, car la convergence est uniforme.
  \item la fonction génératrice $\mathrm{G}_X$\/ est de classe $\mathcal{C}^\infty$\/ sur $]-1,1[$\/ et \[
      \forall k \in \N,\quad P(X = k) = a_n \frac{{\mathrm{G}_X}^{(k)}(0)}{k!}
    .\] La fonction génératrice de $X$\/ permet donc de retrouver la loi de probabilité de $X$.
\end{itemize}


		\begin{exo}
	Soient $\lambda_1, \ldots, \lambda_r \in \R$\/ distincts deux à deux.
	Montrons que, si $\forall x \in \R$, $\alpha_1 \mathrm{e}^{\lambda_1 x} + \cdots + \alpha_r \mathrm{e}^{\lambda_r x} = 0$, alors $\alpha_1 = \cdots = \alpha_r$.
	On peut procéder de différentes manières : le déterminant de {\sc Vandermonde}, par analyse-sythèse, ou, en utilisant \[
		\frac{\mathrm{d}}{\mathrm{d}x}\left( \mathrm{e}^{\lambda_k x} \right) = \lambda_k \mathrm{e}^{\lambda_k x},\quad\text{d'où}\quad\varphi(f_k) = \lambda_k f_k, \text{ avec } f_k : x \mapsto \mathrm{e}^{\lambda_k x}\quad\text{et}\quad\varphi:f\mapsto f'
	.\]
	On doit vérifier que les $f_k$\/ sont des vecteurs et l'application $\varphi$\/ soit un endomorphisme. On se place donc dans l'espace vectoriel $\mathscr{C}^{\infty}$. (On ne peut pas se placer dans l'espace $\mathscr{C}^k$, car sinon l'application $\varphi$\/ est de l'espace $\mathscr{C}^k$\/ à $\mathscr{C}^{k-1}$, ce n'est donc pas un endomorphisme ; ce n'est pas le cas pour l'espace $\mathscr{C}^\infty$.)
	Or, les $\lambda_k$\/ sont distincts deux à deux d'où les vecteurs propres $f_k$\/ sont linéairement indépendants. Et donc si $\alpha_1 f_1 + \alpha_2 f_2 + \cdots + \alpha_r f_r = 0$\/ alors $\alpha_1 = \cdots = \alpha_r=0$.
	Mais, comme $\forall x \in \R$, $\alpha_1 f_1(x) + \alpha_2 f_2(x) + \cdots + \alpha_r f_r(x) = 0$, on en déduit que \[
		\boxed{\alpha_1 = \cdots = \alpha_r = 0.}
	\]
\end{exo}

\section{Critères de diagonalisabilité}

\begin{prop}[une condition \underline{suffisante} pour qu'une matrice soit diagonalisable]
	Soit $A$\/ une matrice carrée de taille $n \ge 2$. {\color{red}Si} $A$\/ possède $n$\/ valeurs propres distinctes deux à deux, {\color{red}alors} $A$\/ est diagonalisable.
\end{prop}

\begin{rmkn}
	La réciproque est fausse : par exemple, pour $n > 1$, $7 I_n$\/ est diagonalisable car elle est diagonale. Mais, elle ne possède pas $n$\/ valeurs propres distinctes deux à deux.
\end{rmkn}

\begin{prv}
	On suppose que la matrice $A \in \mathscr{M}_{n,n}(\mathds{K})$\/ possède $n$\/ valeurs propres distinctes deux à deux (i.e.~$\Card \Sp(A) = n$). D'où, d'après la proposition 16, les $n$\/ vecteurs propres associés $\varepsilon_1,\ldots,\varepsilon_n$\/ sont libres. D'où $(\varepsilon_1, \ldots, \varepsilon_n)$\/ est une base formée de vecteurs propres. Donc, d'après la définition 5, la matrice $A$\/ est diagonalisable.
\end{prv}

\begin{thm}[conditions \underline{nécessaires et suffisantes} pour qu'une matrice soit diagonalisable]
	Soient $E$\/ un espace vectoriel de dimension finie et $u : E \to E$\/ un endomorphisme.
	Alors,
	\begin{align*}
		(1)\quad u \text{ diagonalisable } \iff& E = \bigoplus_{\lambda \in \Sp(u)} \Ker(\lambda\id_E - u) \quad(2)\\
		\iff& \dim E = \sum_{\lambda \in \Sp(u)} \dim(\mathrm{SEP}(\lambda))\quad(3)\\
		\iff& \chi_u \text{ scindé et } \forall \lambda \in \Sp(u),\:\dim(\mathrm{SEP}(\lambda)) = m_\lambda\quad(4)
	\end{align*}
	où $m_\lambda$\/ est la multiplicité de la racine $\lambda$\/ du polynôme $\chi_u$.
\end{thm}

\begin{prv}
	\begin{itemize}
		\item[``$(1)\implies(2)$''] On suppose $u$\/ diagonalisable. Il existe donc une base $(\varepsilon_1$, \ldots, $\varepsilon_n)$\/ de $E$\/ formée de vecteurs propres de $u$. On les regroupes par leurs valeurs propres : $(\varepsilon_i, \ldots, \varepsilon_{i+j})$\/ forme une base de $\mathrm{SEP(\lambda_k)}$. D'où la base $(\varepsilon_1, \ldots, \varepsilon_n)$\/ de l'espace vectoriel $E$\/ est une concaténation des bases des sous-espaces propres de $u$. D'où \[
				E = \bigoplus_{\lambda \in \Sp(u)} \mathrm{SEP}(\lambda)
			.\]
		\item[``$(2)\implies(1)$''] On suppose que $E = \mathrm{SEP}(\lambda_1) \oplus \mathrm{SEP}(\lambda_2) \oplus \cdots \oplus \mathrm{SEP}(\lambda_r)$.
			Soient $(\varepsilon_1, \ldots, \varepsilon_{d_1})$\/ une base de $\mathrm{SEP}(\lambda_1)$, $(\varepsilon_{d_1 + 1}, \ldots, \varepsilon_{d_1 + d_2})$\/ une base de $\mathrm{SEP}(\lambda_2)$, \ldots, $(\varepsilon_{d_1+\cdots + d_{r-1}+1}$, \ldots, $\varepsilon_{d_1+ \cdots + d_r})$\/ une base de $\mathrm{SEP}(\lambda_r)$.
			En concaténant ces base, on obtient une base de $E$, d'après l'hypothèse. Dans cette base, tous les vecteurs sont propres donc $u$\/ est diagonalisable.
		\item[``$(2)\implies(3)$''] On suppose $E = \mathrm{SEP}(\lambda_1) \oplus \mathrm{SEP}(\lambda_2) \oplus \cdots \oplus \mathrm{SEP}(\lambda_r)$. D'où  \[
					\dim E = \dim(\mathrm{SEP}(\lambda_1)) + \dim(\mathrm{SEP}(\lambda_2)) + \cdots + \dim(\mathrm{SEP}(\lambda_r))
			\] car la dimension d'une somme directe est égale à la somme des dimensions.
		\item[``$(3)\implies(1)$''] On suppose $\dim E = \dim(\mathrm{SEP}(\lambda_1)) + \dim(\mathrm{SEP}(\lambda_2)) + \cdots + \dim(\mathrm{SEP}(\lambda_r))$. Or, les sous-espaces propres sont en somme directe, d'après la proposition 16. D'où $\dim\Big(\sum_{\lambda \in \Sp(u)} \mathrm{SEP}(\lambda) \Big)= \sum_{\lambda \in \Sp(u)} \dim(\mathrm{SEP}(\lambda))$. Donc $\sum_{\lambda \in \Sp(u)} \mathrm{SEP}(\lambda) = E$.
		\item[``$(4)\implies(3)$''] On suppose (a) $\chi_u$\/ scindé et (b) $\dim(\mathrm{SEP}(\lambda)) = m_\lambda$. D'où, d'après (a): \[
				\chi_u(x) = (x - \lambda_1)^{m_{\lambda_1}}(x - \lambda_2)^{m_{\lambda_2}} \cdots (x - \lambda_r)^{m_{\lambda_r}} = x^n + \cdots
			\] d'où $m_{\lambda_1} + m_{\lambda_2} + \cdots + m_{\lambda_r} = n$, et d'où \[
				\dim(\mathrm{SEP}(\lambda_1)) + \dim(\mathrm{SEP}(\lambda_2)) + \cdots + \dim(\mathrm{SEP}(\lambda_r)) = n
			\] d'après l'hypothèse (b).
		\item[``$(1)\implies(4)$'']
			On suppose $u$\/ diagonalisable. D'où, dans une certaine base $\mathscr{B}$, la matrice $\big[u\big]_\mathscr{B}$\/ est diagonale. Quitte à changer l'ordre des éléments de $\mathscr{B} = (\varepsilon_1,\ldots,\varepsilon_r)$, on peut supposer que $\big[u\big]_\mathscr{B}$\/ est de la forme \[
				\big[u\big]_\mathscr{B} = 
				\begin{bNiceArray}{c|c|c|c}[last-col]
					\begin{array}{cccc}\lambda_1\\&\lambda_1\\&&\ddots\\&&&\lambda_1\end{array}&0&0&0&\begin{array}{l}\varepsilon_1\\\varepsilon_2\\\vdots\\\varepsilon_{d_1}\\\end{array}\\ \hline
					0&\begin{array}{cccc}\lambda_2\\&\lambda_2\\&&\ddots\\&&&\lambda_2\end{array}&0&0&\begin{array}{l}\varepsilon_{d_1+1}\\\varepsilon_{d_1+2}\\\vdots\\\varepsilon_{d_1+d_2}\\\end{array}\\ \hline
					 &&\ddots&&\vdots\\ \hline
					0&0&0&0\begin{array}{cccc}\lambda_r\\&\lambda_r\\&&\ddots\\&&&\lambda_r\end{array}&\begin{array}{l}\varepsilon_{d_1+\cdots + d_{r-1} + 1}\\\varepsilon_{d_1+\cdots + d_{r-1}+2}\\\vdots\\\varepsilon_{d_1 + \cdots + d_r}\\\end{array}\\
				\end{bNiceArray}
			.\] D'où $\forall k \in \left\llbracket 1,r \right\rrbracket$, $d_k = \dim(\mathrm{SEP}(\lambda_k))$. En outre, $\chi_u(x) = \det(x\id - u) = (x - \lambda_1)^{d_1}\cdot (x-\lambda_2)^{d_2} \cdots (r - \lambda_r)^{d_r}$.
			D'où $\forall k \in \left\llbracket 1,r \right\rrbracket$, $d_k = m_{\lambda_k}$\/ et $\chi_u$\/ est scindé.
	\end{itemize}
\end{prv}

\begin{exo}
	{\slshape On considère la matrice $E$\/ ci-dessous \[
		E = \begin{pmatrix}
			7&0&1\\
			0&3&0\\
			0&0&7
		\end{pmatrix}.
	\] La matrice $E$\/ ci-dessous est-elle diagonalisable ?}

	Soit $\lambda \in \R$. On sait que $\lambda \in \Sp(E)$\/ si et seulement si $\det(\lambda I_3 - E) = 0$. Or \[
		\det(\lambda I_3 - E) =
		\begin{vmatrix}
			\lambda - 7&0&-1\\
			0&\lambda-3&0\\
			0&0&\lambda - 7
		\end{vmatrix} = (\lambda - 7)^2\cdot  (\lambda - 3)^1
	.\] Donc $\Sp(E) = \{3,7\}$, $1 \le \dim\big(\mathrm{SEP}(3)\big) \le 1$, et $1 \le \dim\big(\mathrm{SEP}(7)\big) \le 2$.
	La matrice $E$\/ est diagonalisable si et seulement si $\dim(\mathrm{SEP}(3)) + \dim(\mathrm{SEP}(7)) = 3$, donc si et seulement si $\dim(\mathrm{SEP}(7)) = 2$. On cherche donc la dimension de ce sous-espace propre : soit $X = \left( \substack{x\\y\\z} \right) \in \mathscr{M}_{3,1}(\R)$. On sait que
	\begin{align*}
		X \in \mathrm{SEP}(7) \iff& E\cdot X = 7X\\
		\iff& \begin{pmatrix}
			7&0&1\\
			0&3&0\\
			0&0&7
		\end{pmatrix} \begin{pmatrix}
			x\\y\\z
		\end{pmatrix} = 7 \begin{pmatrix}
			x\\y\\z
		\end{pmatrix}\\
		\iff& \begin{cases}
			7x + 0y + 1z = 7x\\
			3y = 7y\\
			7z = 7z
		\end{cases}\\
		\iff& \begin{cases}
			z = 0\\
			y = 0
		\end{cases}\\
		\iff& X = \begin{pmatrix}
			x\\0\\y
		\end{pmatrix} = x \underbrace{\begin{pmatrix}
			1\\0\\0
		\end{pmatrix}}_{\varepsilon_1}
	\end{align*}
	Donc $\mathrm{SEP}(7) = \Vect(\varepsilon_1)$, d'où $\dim(\mathrm{SEP}(7)) = 1$. Donc la matrice $E$\/ n'est pas diagonalisable.
\end{exo}

\section{Trigonalisation}

Trigonaliser une matrice ne sert que si la matrice n'est pas diagonalisable.

\begin{defn}
	On dit d'une matrice carrée $A \in \mathscr{M}_{n,n}(\mathds{K})$\/ qu'elle est {\it trigonalisable}\/ s'il existe une matrice inversible $P$\/ telle que $P^{-1} \cdot A \cdot P$\/ est triangulaire : \[
		P^{-1} \cdot A \cdot P = \begin{pNiceMatrix}
			\lambda_1&\Block{2-2}*&\\
			\Block{2-2}0&\Ddots&\\
			&&\lambda_r
		\end{pNiceMatrix}
	.\]
\end{defn}

\begin{rmk}
	$\O$\/
\end{rmk}

\begin{thm}
	Une matrice carrée $A \in \mathscr{M}_{n,n}(\mathds{K})$\/ est trigonalisable si et seulement si son polynôme caractéristique $\chi_A \in \mathds{K}[X]$\/ est scindé.
\end{thm}

		\begin{prv}[par récurrence sur $n$, la largeur de la matrice]
	\begin{itemize}
		\item Si $n = 1$, alors la matrice $A = (a_{11})$\/ est déjà triangulaire.
		\item On suppose le polynôme caractéristique $\chi_A$\/ de la matrice scindé dans $\mathds{K}[X]$, d'où il a au moins une racine dans $\mathds{K}$. D'où, la matrice $A$\/ a au moins une valeur propre $\lambda_1 \in \mathds{K}$. Il existe donc un vecteur non nul $\vec{\varepsilon}_1$\/ tel que $A \cdot \vec{\varepsilon}_1 = \lambda_1\,\vec{\varepsilon}_1$. On complète $(\vec{\varepsilon}_1)$\/ en une base de $\mathds{K}^n$\/ : $(\vec{\varepsilon}_1, \vec{e}_2, \ldots, \vec{e}_n)$. En changent de base, il existe une matrice inversible $P$\/ telle que \[
			A' = P^{-1}\cdot A\cdot P = 
			\begin{pNiceArray}[last-row,last-col]{c|ccc}
				\lambda_1 & *&\Ldots&*&\varepsilon_1\\ \hline
				0 & \Block{3-3}{B}&&&e_1\\
				\Vdots&&&&\Vdots\\
				0&&&&e_n\\
				f(\vec{\varepsilon}_1)&f(\vec{e}_1)&\ldots&f(\vec{e}_n)
			\end{pNiceArray}
		.\]
		Comme le polynôme caractéristique est invariant par changement de base, on en déduit que \[
			\chi_A(x) = \chi_{A'}(x) = \left|
			\begin{array}{c|c}
				x-\lambda_1 &*\\ \hline
				0&xI_{n-1} - B\\
			\end{array} \right| = (x-\lambda_1) \cdot \Pi(x)
		.\]
		Or, comme $\chi_A$\/ est scindé, $\Pi(x)$\/ est aussi scindé.
		Or, $\Pi(x) = \det(xI_{n-1} - B)$ d'où $B$\/ est trigonalisable.
	\end{itemize}
\end{prv}

\begin{crlr}
	Toute matrice de $\mathscr{M}_{n,n}(\C)$\/ est trigonalisable.
\end{crlr}

\begin{exo}\relax
	{\slshape Soit une matrice carrée $A \in \mathscr{M}_{n,n}(\mathds{K})$ (où $\mathds{K}$\/ est $\R$\/ ou $\C$). Montrer que
		\begin{align*}
			(1)\quad\text{la matrice } A \text{ est nilpotente}
			\iff& \text{ le polynôme caractéristique de } A \text{ est } \chi_A(X) = X^n\quad(2)\\
			\iff& \text{ la matrice } A \text{ est trigonalisable avec des zéros}\\
			&\text{ sur sa diagonale} \quad(3)
		\end{align*}
	}

	On montre $``\,(1) \implies (2),"$ $``\,(2) \implies (3)\,"$\/ puis $``\,(3) \implies (1)."$

	\begin{itemize}
		\item[$``\,(3) \implies(1)\,"$] Il existe donc une matrice inversible $P$\/ telle que $T = P^{-1}\cdot A\cdot P$\/ et $T$\/ est une matrice trigonalisable. Or, à chaque produit $A^n \cdot A$, une \guillemotleft~sur-diagonalse~\guillemotright\  de zéros supplémentaires. D'où, à partir d'un certain rang $p$, on a $A^p = 0$. La matrice $A$\/ est donc nilpotente.
		\item[$``\,(2) \implies(3)\,"$] On sait que $\chi_A = X^n = (X-0)^n$\/ est scindé, d'où $A$\/ est trigonalisable.
			Il existe donc une matrice inversible $P$\/ telle que \[
				P^{-1}\cdot A\cdot P = A' = \begin{pmatrix}
					\lambda_1 & * & \ldots & *\\
					0 & \ddots&\ddots&\vdots\\
					\vdots&\ddots&\ddots&*\\
					0&\ldots&0&\lambda_n
				\end{pmatrix}
			.\]
			Et donc $\chi_{A'}(x) = (x-\lambda_1)(x-\lambda_2) \cdots (x-\lambda_n)$.
			Or, le polynôme caractéristique est invariant par changement de base, d'où $\lambda_1 = \lambda_2 = \cdots = \lambda_n$.
		\item[$``\,(1)\implies(2)\,"$] On passe dans $\C$\/ alors $\chi_A$\/ est scindé dans $\C$. D'où, il existe $(\lambda_1, \lambda_2, \ldots, \lambda_n) \in \C^n$\/ tels que \[
			\chi_A(X) = (X - \lambda_1) (X - \lambda_2) \cdots (X-\lambda_n)
		.\]
		D'où, chaque $\lambda_i$\/ est une valeur propre \ul{complexe} de la matrice $A$. Or $A$\/ est nilpotente, d'où, par définition, il existe $p \in \N$\/ tel que $A^p = 0$. Les scalaires $\lambda_i$\/ sont dans le spectre de $A$\/ : en effet, il existe un vecteur colonne $X$\/ non nul tel que $A\cdot X = \lambda_i\,X$, d'où $A^2 \cdot X = A\cdot AX = A\cdot \lambda_iX = \lambda_i^2 X$. De même, $A^3 \cdot X = A \cdot A^2 \cdot X = A \cdot \lambda_i^2 X = \lambda_i^2 (A\cdot X) = \lambda_i^3 X$.
		Et, de \guillemotleft~proche en proche~\guillemotright, on a donc \[
			\forall k \in \N,\:A^k\cdot X = \lambda_i^k X
		.\]
		En particulier, si $k=p$, on a $0 = 0\cdot X = A^p\cdot X = \lambda_i^pX$. D'où $\lambda_i^p X = 0$. Or, $X \neq 0$, d'où $\lambda_i^p = 0$\/ et donc $\lambda_i = 0$.
		Finalement, $\chi_A(X) = (X-\lambda_1) \cdots (X-\lambda_n) = (X-0)\cdots(X-0) = X^n  \in \C[X]$. On a donc $\chi_A(X) \in \R[X]$.
	\end{itemize}
\end{exo}

		\begin{rmk}[Notation]
	Si $\Sigma_1$\/ et $\Sigma_2$\/ sont deux alphabets et $\varphi: \Sigma_1 \to \Sigma_2$, alors on note $\tilde\varphi$\/ l'extension de $\varphi$\/ aux mots de $\Sigma_1^*$\/ : \[
		\tilde\varphi(w_1\ldots w_n) = \varphi(w_1)\ldots\varphi(w_n)
	\] et, de plus, on note \[
		\tilde\varphi(L) = \{\tilde\varphi(w)  \mid w \in L\}
	.\]
\end{rmk}

\begin{rmk}
	On a $\tilde\varphi(L \cup M) =\tilde{\varphi}(L \cup M) =\tilde{\varphi}L \cup \tilde\varphi(M)$.
\end{rmk}

\begin{prop}
	\[\tilde\varphi(L \cdot M) = \tilde\varphi(L)\cdot \tilde\varphi(M)\]
\end{prop}

\begin{prv}
	\begin{align*}
		w \in \tilde\varphi(L\cdot M) \iff& \exists u \in L \cdot M,\,w = \tilde\varphi(u)\\
		\iff& \exists (v,t) \in L \times M,\,w = \tilde\varphi(v\cdot t)\\
		\iff& \exists (v,t) \in L \times M,\,w = \tilde\varphi(v) \cdot\tilde\varphi(t)\\
		\iff& w \in \tilde\varphi(L) \cdot \tilde\varphi(M).
	\end{align*}
\end{prv}

\begin{defn}
	Soient $e \in \Reg(\Sigma_1)$, $\varphi : \Sigma_1 \to \Sigma_2$. On définit alors inductivement $e_\varphi$\/ comme étant
	\begin{multicols}{3}
		\[
			\O_\varphi = \O
		\] \[
			\varepsilon_\varphi = \varepsilon
		\] \[
			a_\varphi = \varphi(a) \text{ si } a \in \Sigma_1
		\] \[
			(e_1 \cdot e_2)_\varphi = (e_1)_\varphi \cdot (e_2)_\varphi
		\] \[
			(e_1  \mid e_2)_\varphi = (e_1)_\varphi  \mid (e_2)_\varphi
		\] \[
			(e_1^*)_\varphi = ((e_1)_\varphi)^*.
		\]
	\end{multicols}
\end{defn}

\begin{prop}
	Si $\varphi : \Sigma_1 \to \Sigma_2$\/ et $e \in \Reg(\Sigma_1)$, alors \[
		\mathcal{L}(e_\varphi) = \tilde{\varphi}(\mathcal{L}(e))
	.\]
\end{prop}

\begin{prv}[par incuction sur $e \in \Reg(\Sigma_1)$]
	\begin{itemize}
		\item[cas $\O$] $\mathcal{L}(\O_\varphi) = \mathcal{L}(\O) = \O =\tilde{\varphi}(\O) = \tilde{\varphi}(\mathcal{L}(\O))$\/
		\item[cas $\varepsilon$] $\mathcal{L}(\varepsilon_\varphi) = \mathcal{L}(\varepsilon) = \{\varepsilon\} = \tilde\varphi($\/
			\todo{recopier ici}
		\item[cas $e_1\cdot e_2$] $\mathcal{L}((e_1\cdot e_2)_\varphi) = \mathcal{L}((e_1)_\varphi \cdot (e_2)_\varphi)) = \mathcal{L}((e_1)_\varphi) \cdot \mathcal{L}((e_2)_\varphi) = \tilde{\varphi}(\mathcal{L}(e_1))\cdot\tilde{\varphi}(\mathcal{L}(e_2)) = \tilde{\varphi}(\mathcal{L}(e_1)\cdot \mathcal{L}(e_2)) = \tilde{\varphi}(\mathcal{L}(e_1\cdot e_2))$.
		\item[De même pour les autres cas]
	\end{itemize}
\end{prv}

\begin{prop}
	Soit $e \in \Reg(\Sigma_1)$. Il existe $f \in \Reg(\Sigma)$\/ et $\varphi: \Sigma \to \Sigma_1$\/ tel que $f$\/ est linéaire et $e = f_\varphi$.
\end{prop}

\begin{prv}
	Il suffit de numéroter les lettres (c.f.\ exemple ci-dessous).
\end{prv}

\begin{exm}
	Avec $e = c^* ((a\cdot a) \mid \varepsilon)\cdot ((a \mid c \mid \varepsilon)^*))^* \cdot b \cdot a \cdot a^*$, on a \[
		f = c_1^*((a_1\cdot a_2) \mid \varepsilon)\cdot (b_1((a_3 \mid c_2 \mid \varepsilon)^*))^* \cdot b_2\cdot a_4\cdot a_5
	\] et \[
		\varphi :
		\left(\begin{array}{rcl}
			a_1&\mapsto &a\\
			a_2&\mapsto &a\\
			a_3&\mapsto &a\\
			a_4&\mapsto &a\\
			a_5&\mapsto &a\\
			b_1&\mapsto&b\\
			b_2&\mapsto&b\\
			c_1&\mapsto&c\\
			c_2&\mapsto&c\\
		\end{array}\right)
	.\] 
\end{exm}

\subsection{Automates locaux}

\begin{defn}[automate local, local standard]
	Un automate $\mathcal{A} = (\Sigma, \mathcal{Q}, I, F, \delta)$\/ est dit local dès lors que pour out $\forall (q_1,q_2,\ell,q_3,q_4) \in \mathcal{Q}\times \mathcal{Q}\times \Sigma\times \mathcal{Q}\times \mathcal{Q}$, \[
		(q_1,\ell,q_3) \in \delta \quad\text{et}\quad(q_2,\ell,q_4) \in \delta \qquad\implies\qquad q_3 = q_4
	.\]
	L'automate $\mathcal{A}$\/ est dit, de plus, standard lorsque $\Card(I) = 1$\/ et qu'il n'existe pas de transitions entrante en l'unique état initial $q_0$.
\end{defn}

\begin{prop}
	Un langage est local si et seulement s'il est reconnu par un automate local standard.
\end{prop}

\begin{exm}~\\
	\begin{figure}[H]
		\centering
		\tikzfig{automate-local-1}
		\caption{Automate local reconnaissant le langage $(ab)^*$}
	\end{figure}
\end{exm}

\begin{prv}
	\begin{itemize}
		\item[``$\implies$''] Soit $L$\/ un langage local. Soit $(\Lambda,S,P,F,N)$\/ tels que \[
				L = \Lambda \cup (P\Sigma^*\cap \Sigma^*) \setminus (\Sigma^* N\Sigma^*)
			.\]
			Soit alors l'automate $\mathcal{Q} = \Sigma \cupdot \{\varepsilon\}$, $I = \{\varepsilon\}$, $F_\mathcal{A} = S \cup \Lambda$, et
			\begin{align*}
				\delta &= \phantom{\cup}\{(q,\ell,q') \in \mathcal{Q} \times \Sigma \times \mathcal{Q}  \mid qq' \in F \text{ et } q' = \ell\}\\
				&\mathrel{\phantom{=}} \cup \{(\varepsilon,\ell,q) \in \mathcal{Q}\times \Sigma \times \mathcal{Q}  \mid  \ell = q \text{ et } q \in P\}.
			\end{align*}
			On pose $\mathcal{A} = (\Sigma, \mathcal{Q}, I, F_\mathcal{A}, \delta)$. Montrons que $\mathcal{L}(\mathcal{A}) = L$.
			\begin{itemize}
				\item[``$\subseteq$''] Soit $w \in \mathcal{L}(\mathcal{A})$. Soit donc \[q_1 \xrightarrow{w_1} q_2 \to \cdots \to q_{n-1} \xrightarrow{w_n} q_n\] une exécution acceptante dans $\mathcal{A}$. Montrons que $w_1\ldots w_n \in L$.
					\begin{itemize}
						\item[Cas 1] $w = \varepsilon$\/ et $n = 0$. Ainsi $q_0 = q_n = \varepsilon$\/ et $F \cap I = \O$. Or $F_\mathcal{A} = S \cup \Lambda$, et donc $\Lambda = \{\varepsilon\}$, d'où $\varepsilon \in L$.
						\item[Cas 2] $w \neq \varepsilon$. On sait que $w_1 \in P$\/ ; en effet, $(\varepsilon, w_1, q_1) \in \delta$\/ donc $w_1 = q_1 \in P$.
							De même, $(q_{n-1}, w_n,q_n) \in \delta$, d'où $S \ni w_n = q_n$.
							De plus, $\forall i \in \left\llbracket 1,n-1 \right\rrbracket$, $(q_{i-1}, w_i, q_i) \in \delta$\/ et $(q_i, w_{i+1}, q_{i+1}) \in \delta$.
							Ainsi, $w_i = q_i$\/ et $w_{i+1} = q_{i+1}$\/ avec $q_iq_{i+1} \in F$, d'où $w_iw_{i+1} \in F$. Donc $w \in L$.
					\end{itemize}
				\item[``$\supseteq$''] Soit $w = w_1\ldots w_n \in L$.
					\begin{itemize}
						\item[Cas 1] $w = \varepsilon$. On a $\Lambda = \{\varepsilon\}$, et donc $\varepsilon$\/ est final (ou initial). On en déduit que $\varepsilon \in \mathcal{L}(\mathcal{A})$.
						\item[Cas 2] $w \neq \varepsilon$.
							Montrons, par récurrence finie sur $p \le n$, qu'il existe une exécution \[
								q_0 \xrightarrow{w_1}q_1\to \cdots \to q_{n-1}\xrightarrow{w_p} q_p
							\] dans $\mathcal{A}$.
							\begin{itemize}
								\item Avec $p = 1$, on a $w_1 \in P$\/ donc $(\varepsilon, w_1, w_1) \in \delta$. Ainsi, $\varepsilon \xrightarrow{w_1}w_1$\/ est une exécution dans $\mathcal{A}$.
								\item Supposons construit $\varepsilon \xrightarrow{w_1} q_1 \to \cdots \xrightarrow{w_p} q_p = w_p$\/ avec $p < n$.
									Or, $w_pw_{p+1} \in F$\/ donc $(w_p, w_{p+1}, w_{p+1}) \in \delta$. Ainsi, \[
										\varepsilon \xrightarrow{w_1} q_1 \to \cdots \xrightarrow{w_p} q_p \xrightarrow{w_{p+1}} w_{p+1}
									\] est une exécution acceptante de $\mathcal{A}$.
							\end{itemize}
							De proche en proche, on a \[
								\varepsilon \xrightarrow{w_1} q_1 \to \cdots \to q_{n-1} \xrightarrow{w_{n}} w_{n}
							\] une exécution dans $\mathcal{A}$. Or, $w_n \in S = F_\mathcal{A}$\/ et donc l'exécution est acceptante dans $\mathcal{A}$, et $w \in \mathcal{L}(\mathcal{A})$.
					\end{itemize}
			\end{itemize}
		\item[``$\impliedby$'']
			Soit $\mathcal{A} = (\Sigma, \mathcal{Q}, I, F_\mathcal{A}, \delta)$\/ un automate localement standard.
			Montrons que $\mathcal{L}(\mathcal{A})$\/ est local. Il suffit de montrer que $\rho(\mathcal{L}(\mathcal{A})) = \mathcal{L}(\mathcal{A})$. Or $\mathcal{L}(\mathcal{A}) \subseteq \rho(\mathcal{L}(\mathcal{A}))$. On montre donc $\rho(\mathcal{L}(\mathcal{A}))\subseteq \mathcal{L}(\mathcal{A})$.

			Soit $w \in \rho(\mathcal{L}(\mathcal{A}))$. Ainsi, \[
				w \in \Lambda(\mathcal{L}(\mathcal{A})) \cup \Big(P(\mathcal{L}(\mathcal{A})) \Sigma^* \cap \Sigma^* S(\mathcal{L}(\mathcal{A}))\Big) \setminus \Big(\Sigma^* N(\mathcal{L}(\mathcal{A}))\Sigma^*\Big)
			.\]
			Montrons que $w \in \mathcal{L}(\mathcal{A})$.
			\begin{itemize}
				\item Si $w \in \Lambda(\mathcal{L}(\mathcal{A}))$, alors $w = \varepsilon$. Or, $\Lambda(\mathcal{L}(\mathcal{A})) = \mathcal{L}(\mathcal{A}) \cap \{\varepsilon\}$. Ainsi $w \in \mathcal{L}(\mathcal{A})$.
				\item Sinon, $w = w_1\ldots w_n$\/ avec $w_1 \in P(\mathcal{L}(\mathcal{A}))$, donc il existe $u \in \Sigma^*$\/ tel que $w_1\cdot u \in \mathcal{L}(\mathcal{A})$. Il existe donc une exécution acceptante \[ % todo dashed arrow
						I \ni q_0 \xrightarrow{w_1} q_1 \overset{u}{-\ -\ \to} q_s \in F_\mathcal{A}
					.\]
					Il existe donc une exécution $q_0\xrightarrow{w_1}q_1$.

					Supposons construit $q_1 \xrightarrow{w_1} q_1 \to \cdots \xrightarrow{w_p} q_p$\/ avec $p < n$.
					Or, $w_pw_{p+1} \in F(\mathcal{L}(\mathcal{A}))$, donc il existe $w \in \Sigma^*$\/ et $y \in \Sigma^*$\/ tels que $x \cdot w_p \cdot w_{p+1}\cdot y \in \mathcal{L}(\mathcal{A})$. Il existe donc une exécution acceptante  \[
						r_0 \overset x{-\ -\ \to} r_{p-1} \xrightarrow{w_p} r_p \xrightarrow{w_{p+1}} r_{p+1} \overset y{-\ -\ \to} r_s.
					\] Or, par localité de l'automate, $q_p = r_p$. Il existe donc $q_{p+1}$\/ ($= r_{p+1}$) tel que $(q_p, w_{p+1}, q_{p+1}) \in \delta$.
					On a donc une exécution \[
						q_0 \xrightarrow{w_1} q_1 \to \cdots \to q_p \xrightarrow{w_{p+1}} q_{p+1}
					.\]
			\end{itemize}
			De proche en proche, il existe une exécution \[
				q_0 \xrightarrow{w_1} q_1 \to \cdots \to q_n
			.\] Or, $w_n \in S(\mathcal{L}(\mathcal{A}))$, il existe donc $v \in \Sigma^*$\/ tel que $v \cdot w_n \in \mathcal{L}(\mathcal{A})$, donc il existe un exécution acceptante \[
				I\ni r_0 \overset v{-\ -\ \to} r_{s-1} \xrightarrow{w_n} r_s  \in F_\mathcal{A}
			.\]
			Par localité, $r_s = q_n \in F_\mathcal{A}$.

			Donc $\rho(\mathcal{L}(\mathcal{A})) \subseteq \mathcal{L}(\mathcal{A})$\/ et donc $\rho(\mathcal{L}(\mathcal{A})) = \mathcal{L}(\mathcal{A})$. On en déduit que $\mathcal{L}(\mathcal{A})$\/ est local.
	\end{itemize}
\end{prv}

\begin{exm}
	Dans le langage local $(ab)^*  \mid c^*$, on a $\Lambda = \{\varepsilon\}$, $S = \{c,b\}$, $P = \{a,c\}$\/ et $F = \{ab,ba,cc\}$.
	L'automate local reconnaissant ce langage est celui ci-dessous.
	\begin{figure}[H]
		\centering
		\tikzfig{automate-local-2}
		\caption{Automate local reconnaissant $(ab)^*  \mid c^*$}
	\end{figure}
\end{exm}

\begin{prop}
	Soit $\mathcal{A} = (\Sigma, \mathcal{Q}, I, F, \delta)$\/ un automate et $\varphi : \Sigma \to \Sigma_1$. On pose \[
		\delta' = \{(a,\varphi(\ell),q')  \mid (q, \ell, q') \in \delta\}
	.\] On pose $\mathcal{A}' = (\Sigma, \mathcal{Q}, I, F, \delta')$. On a $\mathcal{L}(\mathcal{A}') = \tilde\varphi(\mathcal{L}(\mathcal{A}))$.
\end{prop}

\begin{exo}
	Montrons que $\mathrm{LR} \subsetneq \wp(\Sigma^*)$\/ i.e.\ il existe des langages non reconnaissables.

	$\R$\/ n'est pas dénombrable. On écrit un nombre réel comme une suite infinie \[0{,}10110010101\ldots1010110\ldots\] On pose $\Sigma = \{a\}$, on crée le langage $L$, associé au nombre ci-dessus comme l'ensemble contenant $a$, $aaa$, $aaaa$,\ldots
\end{exo}

\begin{rmk}[Notation]
	On note $A_\varphi$, l'automate $\big(\varphi(E), \mathcal{Q}, I, F, \delta')$\/ où $\delta' = \{(q,\varphi(\ell),q')  \mid (q,\ell,q') \in \delta\}$.
\end{rmk}


		\subsection{Algorithme de {\scshape Berry-Sethi}\/ : les langages réguliers sont reconnaissables}

\begin{exm}
	On considère l'expression régulière $aab(a \mid b)^*$. On numérote les lettres : $a_1a_2b_1(a_3 \mid b_2)^*$, avec  \[
		\varphi : \left(
			\begin{array}{ccc}
				a_1&\mapsto&a\\
				a_2&\mapsto&a\\
				a_3&\mapsto&a\\
				b_1&\mapsto&b\\
				b_2&\mapsto&b\\
			\end{array}
		\right)
	.\]

	\begin{table}[H]
		\centering
		\[
			\begin{array}{c|c|c|c|c}
				&\Lambda&S&P&F\\\hline
				a_1&\O&a_1&a_1&\O\\
				a_2&\O&a_2&a_2&\O\\
				a_1\cdot a_2&\O&a_2&a_1&a_1a_2\\
				b_1&\O&b_1&b_1&\O\\
				a_1a_2b_1&\O&b_1&a_1&a_1a_2,a_2b_1\\
				a_3&\O&a_3&a_3&\O\\
				b_2&\O&b_2&b_2&\O\\
				a_3 \mid b_2&\O&a_3,b_2&a_3,b_2&\O\\
				(a_3 \mid b_2)^*&\varepsilon&a_3,b_2&a_3,b_2&a_3b_2,b_2a_3,a_3a_3,b_2b_2\\
				a_1a_2b_1(a_3 \mid b_2)^*&\O&a_3,b_2ab_1&a_1&a_3b_2,b_2a_3,a_3a_3,b_2b_2,a_1a_2,a_2b_1,b_1a_3,b_1b_2
			\end{array}
		\]
		\caption{$\Lambda$, $S$, $P$\/ et $F$\/ pour les différents mots reconnus}
		\label{tab:l,s,p,f}
	\end{table}
	On crée donc l'automate ci-dessous.
	\begin{figure}[H]
		\centering
		\tikzfig{automate-numerote}
		\caption{Automate déduit de la table \ref{tab:l,s,p,f}}
		\label{aut:num}
	\end{figure}
	On applique la fonction $\varphi$\/ a tous les états et transitions pour obtenir l'automate ci-dessous. Cet algorithme reconnaît le langage $aab(a \mid b)^*$.
	\begin{figure}[H]
		\centering
		\tikzfig{automate-non-numerote}
		\caption{Application de $\varphi$\/ à l'automate de la figure \ref{aut:num}}
	\end{figure}
\end{exm}

\begin{thm}
	Tout langage régulier est reconnaissable.
	De plus, on a un algorithme qui calcule un automate le reconnaissant, à partir de sa représentation sous forme d'expression régulière.
\end{thm}

\begin{algo}[\scshape Berry-Sethi]
	Entrée : Une expression régulière $e$\/ \\
	Sortie : Un automate reconnaissant $\mathcal{L}(e)$\/ \\
	\begin{enumerate}
		\item On linéarise $e$\/ en $f$\/ avec une fonction $\varphi$\/ telle que $f_\varphi = e$.
		\item On calcule inductivement $\Lambda(f)$, $S(f)$, $P(f)$, et $F(f)$.
		\item On fabrique $\mathcal{A} = (\Sigma, \mathcal{Q}, I, F, \delta)$\/ un automate reconnaissant $\mathcal{L}(f)$.
		\item On retourne $\mathcal{A}_\varphi$.
	\end{enumerate}
	\todo{refaire la mise en page pour les algorithmes}
\end{algo}

\subsection{Les langages reconnaissables sont réguliers}

On fait le \guillemotleft~sens inverse~\guillemotright\ : à partir d'un automate, comment en déduire le langage reconnu par cet automate ?

L'idée est de supprimer les états un à un.
Premièrement, on rassemble les états initiaux en les reliant à un état {\raisebox{-5pt}{\tikz\node [style=new style 0] at (0, 0) {\clap{$i$}};}}, et de même, on relie les états finaux à {\raisebox{-5pt}{\tikz\node [style=new style 0] at (0, 0) {\clap{$f$}};}}.
Pour une suite d'états, on concatène les lettres reconnus sur chaque transition :
\begin{figure}[H]
	\centering
	\tikzfig{automate-succession}
	\caption{Succession d'états}
\end{figure}
De même, lors de \guillemotleft~branches~\guillemotright\ en parallèles, on les concatène avec un $ \mid $.
En appliquant cet algorithme à l'automate précédent, on a \[
	(aab)  \cdot \Big((\varepsilon \mid aa^*)  \mid (b \mid aa^*b)\cdot (b \mid aa^*b)^* (aa^* \mid \varepsilon)\Big)
.\]

\begin{defn}
	Un automate généralisé est un quintuplet $(\Sigma, \mathcal{Q}, I, F, \delta)$\/ où
	\begin{itemize}
		\item $\Sigma$\/ est un alphabet ;
		\item $\mathcal{Q}$\/ est un ensemble fini ;
		\item $I \subseteq  \mathcal{Q}$\/ ;
		\item $F \subseteq \mathcal{Q}$\/ ;
		\item $\delta \subseteq \mathcal{Q} \times \Reg(\Sigma) \times \mathcal{Q}$, avec \[
			\forall r \in \Reg(\Sigma),\:\forall (q,q') \in \mathcal{Q}^2,\:\Card(\{(q,r,q') \in \delta\}) \le 1
		.\]
	\end{itemize}
\end{defn}

\begin{defn}[Langage reconnu par un automate généralisé]
	Soit $(\Sigma, \mathcal{Q}, I, F, \delta)$\/ un automate généralisé.
	On dit qu'un mot $w$\/ est reconnu par l'automate s'il existe une suite \[
		q_0 \xrightarrow{r_1} q_1 \xrightarrow{r_2} q_2 \to \cdots \to q_{n-1}\xrightarrow{r_n} q_n
	\] et $(u_i)_{i \in \left\llbracket 1,n \right\rrbracket}$\/ tels que $\forall i \in \left\llbracket 1,n \right\rrbracket$, $u_i \in \mathcal{L}(r_i)$\/ et $w = u_1 \cdot u_2 \cdot \ldots \cdot u_n$.
\end{defn}

\begin{defn}
	Un automate généralisé $(\Sigma, \mathcal{Q}, I, F, \delta)$\/ est dit \guillemotleft~bien détouré\footnotemark~\guillemotright\ si $I = \{i\}$ et $F = \{f\}$, avec $i \neq f$, tels que $i$\/ n'a pas de transitions entrantes et $f$\/ n'a pas de transitions sortantes.
\end{defn}
\footnotetext{Cette notation n'est pas officielle.}

\begin{lem}
	Tout automate généralisé est équivalent à un automate généralisé \guillemotleft~bien détouré.~\guillemotright\@ En effet, soit $\mathcal{A} = (\Sigma, \mathcal{Q}, I, F, \delta)$\/ un automate généralisé.
	Soit $i \not\in \mathcal{Q}$\/ et $f \not\in \mathcal{Q}$.
	On pose $\Sigma' = \Sigma$, $I' = \{i\}$, $F' = \{f\}$, $\mathcal{Q}' = \mathcal{Q} \cupdot \{i,f\}$\/ et \[
		\delta' = \delta \cupdot \{(i,\varepsilon,q)  \mid q \in I\} \cupdot \{(q,\varepsilon,f)  \mid q \in F\}
	.\] Alors, l'automate $\mathcal{A}' = (\Sigma', \mathcal{Q}', I', F', \delta')$\/ est équivalent à $\mathcal{A}$\/ et \guillemotleft~bien détouré.~\guillemotright
\end{lem}

\begin{lem}
	Soit $\mathcal{A} = (\Sigma, \mathcal{Q}, I, F, \delta)$\/ un automate généralisé \guillemotleft~bien détouré~\guillemotright\ tel que $|\mathcal{Q}| \ge 3$. Alors il existe un automate généralisé \guillemotleft~bien détouré~\guillemotright\ $\mathcal{A}'  =(\Sigma, \mathcal{Q}', I, F, \delta')$\/ avec $\mathcal{Q}' \subsetneq \mathcal{Q}$\/ et $\mathcal{L}(\mathcal{A}) = \mathcal{L}(\mathcal{A}')$.
\end{lem}

\begin{prv}
	Étant donné qu'il existe au plus une transition entre chaque pair d'état $(q, q') \in \mathcal{Q}^2$, il est possible de le représenter au moyen d'une fonction de transition \[
		T : \mathcal{Q} \times \mathcal{Q} \longrightarrow \Reg(\Sigma)
	.\]
	\todo{Recopier la def de $T$}
	Soit $q \in \mathcal{Q} \setminus \{i,f\}$. Soit alors $T'$\/ défini, pour $(q_a, q_b) \in \mathcal{Q} \setminus \{q\}$, par \[
		T'(q_a, q_b) = T(q_a, q_b)  \mid T(q_a, q) \cdot T(q,q)^* \cdot T(q,q_b)
	.\] On considère l'automate $\mathcal{Q}' = \mathcal{Q} \setminus \{q\}$\/ et $\delta'$\/ construit à partir de $T'$.
\end{prv}

\begin{exm}
	On considère l'automate ci-dessous.
	\begin{figure}[H]
		\centering
		\tikzfig{automate-11}
		\caption{Automate exemple}
		\label{aut:a11}
	\end{figure}
	La fonction $T$\/ peut être représentée dans la table ci-dessous.
	\begin{table}[H]
		\centering
		\begin{tabular}{c|ccc}
			&0&1&2\\\hline
			0&$\O$&$a \mid b$&$\varepsilon$ \\
			1&$\O$\/ &$\O$\/ &$\O$\/ \\
			2&$\O$\/ &$\O$\/ &$a^*$\/\\
		\end{tabular}
		\caption{Fonction $T$\/ équivalente à l'automate de la figure \ref{aut:a11}}
	\end{table}
\end{exm}

\begin{exm}
	On applique l'algorithme à l'automate suivant.
	\begin{figure}[H]
		\centering
		\tikzfig{automate-12}
		\tikzfig{automate-12b}
		\tikzfig{automate-12c}
		\tikzfig{automate-12d}
		\tikzfig{automate-12e}
		\tikzfig{automate-12f}
		\caption{Application de l'algorithme à un exemple}
	\end{figure}
	On a donc que le langage de l'automate initial est \[
		\mathcal{L}(d^* (aa) \mid (ac \mid b)(cc)^*(b \mid ca)(d \mid e)^*)
	.\]
\end{exm}

\begin{thm}
	Un langage reconnaissable est régulier.
\end{thm}

\begin{prv}
	On itère le lemme précédent depuis un automate généralisé $\mathcal{A}$\/ jusqu'à obtention d'un automate comme celui ci-dessous.
	\begin{figure}[H]
		\centering
		\tikzfig{automate-13}
		\caption{Automate résultat de l'application du lemme}
	\end{figure}
	On a alors $\mathcal{L}(\mathcal{A}) = \mathcal{L}(r)$.
\end{prv}


		\begin{thm}[{\scshape Kleene}]
	Un langage est régulier si et seulement s'il est reconnaissable. Et, on a donné un algorithme effectuant ce calcul dans les deux sens.
\end{thm}

\section{La classe des langages réguliers}

\begin{figure}[H]
	\centering
	\tikzfig{ensembles-de-langages}
	\caption{Ensembles de langages}
\end{figure}

\begin{prop}
	La classe des langages réguliers/reconnaissables est stable par passage au complémentaire.
\end{prop}

\begin{prv}
	Soit $L \in \mathrm{LR}$. Soit $\mathcal{A} = (\Sigma, \mathcal{Q}, I, F, \delta)$\/ un automate reconnaissant le langage $L$. Soit $\mathcal{A}'= (\Sigma, \mathcal{Q}', I', F', \delta')$\/ un automate déterministe et complet équivalent à $\mathcal{A}$. Soit $\mathcal{A}'' = (\Sigma, \mathcal{Q}', I', \mathcal{Q}'\setminus F', \delta')$.
	Alors (à prouver à la maison) $\mathcal{L}(\mathcal{A}'') = \Sigma^* \setminus \mathcal{L}(\mathcal{A}) = \Sigma^* \setminus L$\/ et donc $\Sigma^* \setminus L$\/ est reconnaissable/régulier.
\end{prv}

\begin{crlr}
	On a la stabilité par intersection. En effet, \[
		L \cap L' = \Big(L^{\mathrm{c}} \cup (L')^{\mathrm{c}}\Big)^{\mathrm{c}}
	\] où $L^{\mathrm{c}}$\/ est le complémentaire de $L$.
\end{crlr}

\begin{crlr}
	Si $L$\/ et $L'$\/ sont deux langages réguliers (quelconques), alors $L \setminus L'$\/ est un langage régulier. En effet, \[
		L \setminus L' = L \cap (L')^\mathrm{c}
	.\]
\end{crlr}

\begin{crlr}
	Si $L$\/ et $L'$\/ sont deux langages réguliers. Alors $L \mathbin\triangle L'$\/ \footnotemark\ est un langage régulier. En effet \[
		L \mathbin\triangle L' \mathrel{\mathop=^{\mathrm{(def)}}} (L \cup L') \setminus (L \cap L')
	.\]
\end{crlr}
\footnotetext{$\triangle$ est la différence symétrique}

\subsection{Limite de la classe/Lemme de l'étoile}

\begin{thm}[Lemme de l'étoile]
	Soit $L$\/ un langage reconnu par un automate à $n$\/ états.
	Pour tout mot $u \in L$\/ de longueur supérieure ou égale à $n$, il existe trois mots $x$, $y$\/ et $z$\/ tels que \[
		u = x \cdot y \cdot z,\qquad|x\cdot y| \le n,\qquad y \neq \varepsilon, \qquad\text{et}\qquad \forall p \in \N,\:x\cdot y^p \cdot z \in L
	.\]
\end{thm}

\begin{prv}
	Soit $L$\/ un langage reconnu par un automate $\mathcal{A} = (\Sigma, \mathcal{Q}, I, F, \delta)$\/ à $n$\/ états. Soit $u$\/ un mot d'un alphabet $\Sigma$\/ de longueur supérieure ou égale à $n$\/ ($u \in \Sigma^{\ge n}$) tel que $u \in L$.
	Alors, il existe un exécution acceptante \[
		q_0 \xrightarrow{u_1} q_1 \xrightarrow{u_2} q_2 \to \cdots \to q_{m-1} \xrightarrow{u_m} q_m
	\] avec $m \ge n$.
	Par principe des tiroirs, l'ensemble $\{(i,j) \in \left\llbracket 0,m \right\rrbracket^2  \mid i < j \text{ et } q_i = q_j\}$\/ est non vide.
	Et donc $A = \{j \in \left\llbracket 0,m \right\rrbracket  \mid \exists i \in \left\llbracket 0,j-1 \right\rrbracket,\:q_i = q_j\}$\/ est non vide.
	Soit alors $j_0 = \min A$\/ bien défini. Alors, par définition de $A$, il existe $i_0 \in \left\llbracket 0,j_0-1 \right\rrbracket$\/ tel que $q_{i_0} = q_{j_0}$\/ et $j_0 \le n$. On pose donc \[
		\underbrace{q_0 \xrightarrow{u_1} q_1 \xrightarrow{u_2} \cdots \xrightarrow{u_{i_0}}}_x q_{i_0} \underbrace{\xrightarrow{u_{i_0+1}} q_{i_0 + 1} \to \cdots \xrightarrow{u_{j_0}}}_y q_{j_0} \underbrace{\xrightarrow{u_{j_0+1}} q_{j_0 + 1} \to \cdots \xrightarrow{u_m}}_z q_m\::
	\] $x = u_1u_2\ldots u_{i_0}$, $y = u_{i_0+1}\ldots u_{j_0}$\/ et $z_{j_0 + 1}\ldots u_m$.
	On a donc $y \neq \varepsilon$\/ : en effet $i_0 < j_0$.
	Également, on a $|x\cdot y| = j_0 \le n$\/ et $u = x\cdot y\cdot z$.
	Montrons alors que $\forall p \in \N,\,x \cdot y^p \cdot z \in L$.
	La suite de transitions \[
		q_0 \xrightarrow{u_1} q_1 \to \cdots \xrightarrow{u_{i_0}} q_{i_0} \xrightarrow{u_{j_0+1}}  q_{j_0+1} \to \cdots \xrightarrow{u_m} q_m
	\] est une exécution acceptante donc $x \cdot z \in L$. De proche en proche, on en déduit que $x \cdot y^p \cdot z \in L$\/ pour tout $p \in \N$.
\end{prv}

\begin{crlr}
	Il y a des langages non réguliers/reconnaissables.
\end{crlr}

\begin{prv}
	Soit $L = \{a^n \cdot b^n  \mid n \in \N\}$. Montrons que $L$\/ n'est pas régulier par l'absurde. Supposons~$L$\/ reconnaissable par un automate $\mathcal{A}$\/ à $n$\/ états, et soit $u = a^n \cdot b^n$. Alors $|u| \ge n$. D'où, d'après le lemme de l'étoile, il existe un triplet $(x,y,z) \in  (\Sigma^*)^3$\/ tel que $y \neq \varepsilon$, $u = x \cdot y \cdot z$, $|x\cdot y| \le n$\/ et $x\cdot y^*\cdot z \subseteq L$ $(*)$.
	Il existe donc $p \in \left\llbracket 1,n \right\rrbracket$\/ tel que $y = a^p$. De même, il existe $q \in \left\llbracket 0,n-p \right\rrbracket$\/ tel que $x = a^q$\/ et $z = a^{n-p-q}\cdot b^n$. Donc, d'après $(*)$, $x\cdot y\cdot y\cdot z \in L$\/ et donc $a^q\cdot a^p\cdot a^p\cdot a^{n-p-q}\cdot b^n \in L$, d'où $a^{n+p} \cdot b^n \in L$. Or, comme $p \neq 0$, $n + p \neq n$\/ : une contradiction.
\end{prv}

\begin{exo}
	On considère le langage $L_2 = \{w \in \Sigma^*  \mid  |w|_a = |w|_b\}$.
	{\slshape Le langage $L_2$\/ est-il régulier ?}\/ La même démonstration fonction en remplaçant $L$\/ par $L_2$. Mais, nous allons procéder autrement, par l'absurde : on suppose $L_2$\/ régulier. Or, on sait que, d'après la preuve précédente, $L = L_2 \cap a^* \cdot b^*$, et $a^* \cdot b^*$\/ est régulier. D'où $L$\/ régulier, ce qui est absurde.
\end{exo}

\begin{exo}
	On considère le langage $L = \{w \in \Sigma^*  \mid |w|_a \equiv |w|_b\mod 3\}$.
	{\slshape Le langage $L$\/ est-il régulier ?}\/ Oui, l'automate de la figure suivante reconnait le langage $L$ (les états représentent la différence $|w|_a - |w|_b\ \mathrm{mod}\ 3$).

	Montrons à présent qu'un automate à moins de trois états n'est pas possible : si $\delta^*(i_0, a^x) = \delta^*(i_0, a^y)$\/ avec $\left\llbracket 0,2 \right\rrbracket \ni x < y \in \left\llbracket 0,2 \right\rrbracket$, alors pour tout $z \in \N$, $\delta^*(i_0, a^{x+z}) = \delta^*(i_0, a^{y+z})$. On pose $z = 3 -y$. Alors \[
		\delta^*(i_0, \underset{\substack{\vrt{\not\in}\\F}}{a^{x+3-y}}) = 
		\delta^*(i_0, \underset{\substack{\vrt{\in}\\F}}{a^{3}})
	.\] 

	\begin{figure}[H]
		\centering
		\tikzfig{automate-14}
		\caption{Automate reconnaissant le langage $\{w \in \Sigma^*  \mid |w|_a \equiv |w|_b \mod 3\}$}
	\end{figure}
\end{exo}

\begin{exo}
	Soit $\Sigma = \{0,1,`(\text{'},`)\text{'}, `\{\text{'}, `\}\text{'},`,\text{'}\}$.
	On écrit en OCaml la fonction {\tt to\_string}\/ définie telle que si $({\tt affiche}\/\ \mathcal{A})$ et $({\tt affiche}\/\ \mathcal{A}')$ donnent le même affichage, alors $\mathcal{A} = \mathcal{A}'$.

	\begin{figure}[H]
		\centering
		\tikzfig{automate-15}
		\caption{Codage d'un automate par une chaîne de caractères}
	\end{figure}

	Par exemple, on représente l'automate ci-dessus par \[
		``\big(\{0,1,10\},\{0\}, \{10\}, \{(0,0,1),(1,0,10),(10,0,1),(1,1,0)\}\big)."
	\]

	\begin{lstlisting}[language=caml,caption=Fonction {\tt affiche}\/ affichant un automate]
let affiche (%*$Q$*),%*$I$*),%*$F$*),%*$\delta$*)) = 
	\end{lstlisting}
	\todo{Recopier le code}
\end{exo}

\begin{exo}
	Supposons que tout langage est reconnaissable.
	Soit $L = \{w \in \Sigma^*  \mid \exists \mathcal{A},\:w \leftarrow {\tt affiche}\/\ \mathcal{A}\:\text{et}\:w \not\in \mathcal{L}(\mathcal{A})\}$.
	Soit $B$\/ un automate tel que $L = \mathcal{L}(B)$.
	Soit $w \in {\tt affiche}\/\ B$. Si $w \in L$, alors il existe un automate tel que $w = {\tt affiche}\/\ \mathcal{A}$\/ et $w \not\in  \mathcal{L}(\mathcal{A})$. D'où $\mathcal{A} = B$\/ par injectivité et donc $w \not\in  \mathcal{L}(B) = L$, ce qui est absurde.
	Sinon, si $w \not\in L$, alors $w = {\tt affiche}\/\ B$\/ avec $w \not\in  \mathcal{L}(B)$\/ et $w \in L$, ce qui est absurde.
\end{exo}



		\clearpage
		\setcounter{section}{0}		\renewcommand{\thesection}{\llap{Annexe }\thechapter.\Alph{section}}
		\renewcommand{\thesectionnum}{\Alph{section}}
		\section{Programmation dynamique}

On rappelle le problème \textsc{Knapsack} : \[
	\begin{cases}
		\text{\textbf{Entrée}} &: n \in \N,\: w \in (\N^*)^n,\: v \in (\N^*)^n,\: P \in \N\\
		\text{\textbf{Sortie}} &:\max_{x \in \{0,1\}^n} \big\{\left<x,v \right> \:\big|\: \left<x,w \right> \le P \}.
	\end{cases}
\]

On pose \[\textsc{sad}(n, w, v, P) = \max_{x \in \{0,1\}^n} \big\{\left<x,v \right> \:\big|\: \left<x,w \right> \le P \},\] et \[\mathrm{sol}(n, w, v, P) = \{\left<x,v \right>  \mid \left<x,w \right> \le P,\: x \in \{0,1\}^n \}.\] Lorsque $y \in \R^n$, avec $y = (y_1, \ldots, y_n)$, on note $\R^{n-1} \owns \tilde y = (y_2, y_3, \ldots, 0)$. Ainsi, si $n > 0$,
\begin{align*}
	\mathrm{sol}(n, w, v, P) &= \{\left<x, v \right>  \mid \left< x,w \right> \le P,\: x \in \{0,1\}^n \text{ et } x_1 = 0\}\\
	&\quad\cup \{\left<x, v \right>  \mid \left< x,w \right> \le P,\: x \in \{0,1\}^n \text{ et } x_1 = 1\}\\
	&= \{\left<\tilde{x}, \tilde{v} \right>  \mid \left<\tilde{x}, \tilde{w} \right> \le P,\: x u\in \{0,1\}^n \text{ et } x_1 = 0\} \\
	&\quad\cup \{v_1 + \left<\tilde{x}, \tilde{v} \right>  \mid \left<\tilde{x},\tilde{w} \right> \le P - w_1,\: x \in \{0,1\}^n \text{ et } x_1 = 1\}\\
	&= \{\left<y, \tilde{v} \right>  \mid \left<y, \tilde{w} \right> \le P \text{ et } y \in \{0,1\}^{n-1}\}  \\
	&\quad\cup \{v_1 + \left<y, \tilde{v} \right>  \mid \left<y, \tilde{w} \right> \le P - w_1 \text{ et } y \in \{0,1\}^{n-1}\}
\end{align*}
D'où, par passage au $\max$, si $n > 0$,
\begin{align*}
	\textsc{sad}(n, w, v, P) &= \max( \\
	&\quad\quad \max \{\left<y  \mid \tilde{v} \right>  \mid \left<y, \tilde{w} \right> \le P \text{ et } y \in \{0,1\}^{n-1}\}\\
	&\quad\quad v_1 + \max \{\left<y  \mid \tilde{v} \right>  \mid \left<y, \tilde{w} \right> \le P - w_1 \text{ et } y \in \{0,1\}^{n-1}\} \\
	&) = \max\!\big(\textsc{sad}(n-1, \tilde{w}, \tilde{v}, P), v_1 + \textsc{sad}(n-1, \tilde{w}, \tilde{v}, P - w_1)\big).
\end{align*}
Si $n = 0$, alors $\textsc{sad}(0, v, w, P) = 0$.

\begin{rmk}
	Si on le code \textit{tel quel}, il y aura $\mathcal{O}(2^n)$ appels récursifs. Mais, on a $(n+1)(P+1)$\/ sous-problèmes.
\end{rmk}

Notons alors, pour $n$, $v$, $w$, $P$ fixés, $(s_{i,j})_{\substack{i \in \llbracket 1,n \rrbracket\\ j \in \llbracket 0,P \rrbracket}}$\/ tel que \[
	s_{i,j} = \textsc{sad}\Big(n-i, v_{\big|\llbracket i+1,n \rrbracket}, w_{\big|\llbracket i+1,n \rrbracket}, j\Big)
.\]
On a alors $\textsc{sad}(n, v, w, P) = s_{0,P}$. Ainsi, pour $j \in \llbracket 0,P \rrbracket$, $s_{n,j} = 0$ ; pour $i \in \llbracket 0,n \rrbracket$, $s_{i,0} = 0$ ; \[
	s_{i,j} = \max(s_{i+1,j}, v_{i+1} + s_{i+1,j- w_{i+1}})
;\] et, si $w_{i+1} > j$, alors $s_{i,j} = s_{i+1,j}$.

La complexité de remplissage de la matrice est en $\mathcal{O}(n\,P)$\/ en temps et en espace.
On n'a pas prouvé $\text{\textbf{P}} = \text{\textbf{NP}}$, la taille de l'entrée est
\begin{itemize}
	\item pour un entier $n$\/ : $\log_2(n)$,
	\item pour un tableau de $n$\/ entiers : $n \log_2(n)$,
	\item pour un tableau de $n$\/ entiers : $n \log_2(n)$,
	\item pour un entier $P$ : {\color{red}$\log_2(P)$}.
\end{itemize}
Vis à vis de la taille de l'entrée, la complexité de remplissage est {\color{red}exponentielle}.


		\section{\scshape Hors-programme}

\begin{defn}
	On appelle monoïde un ensemble $M$\/ muni d'une loi ``$\cdot$'' interne associative admettant un élément neutre $1_M$.
\end{defn}

\begin{defn}
	Étant donné deux monoïdes $M$\/ et $N$, on appelle morphisme de monoïdes une fonction $\mu : M \to N$\/ telle que
	\begin{enumerate}
		\item $\mu(1_M) = 1_N$\/ ;
		\item $\mu(x \cdot_M y) = \mu(x) \cdot_N \mu(y)$.
	\end{enumerate}
\end{defn}

\begin{exm}
	$|\:\cdot\:|\:: (\Sigma^*, \cdot) \to (\N, +)$\/ est un morphisme de monoïdes.
\end{exm}

\begin{defn}
	Un langage $L$\/ est dit reconnu par un monoïde $M$, un morphisme $\mu : \Sigma^* \to M$\/ et un ensemble $P \subseteq M$\/ si $L = \mu^{-1}(P)$.
\end{defn}

\begin{exm}
	L'ensemble $\{a^{n^3} \mid n \in \N\}$\/ est reconnu par le morphisme $|\:\cdot\:|$\/ et l'ensemble $P = \{n^3  \mid n \in \N\}$.
\end{exm}

\begin{thm}
	Un langage est régulier si et seulement s'il est reconnu par un monoïde fini.
\end{thm}

\begin{exm}
	L'ensemble $\{a^{2n}  \mid n \in \N\}$\/ est un langage régulier. En effet, on a $M = \ZdZ$, $P = \{0\}$\/ et \begin{align*}
		\mu: \Sigma^* &\longrightarrow \ZdZ \\
		w &\longmapsto |w|\ \mathrm{mod}\ 2.
	\end{align*}

	\begin{figure}[H]
		\centering
		\tikzfig{automate-monoide-1}
		\caption{Automate reconnaissant $\mu^{-1}(P) = L$}
	\end{figure}
\end{exm}

\begin{prv}
	\begin{itemize}
		\item[``$\implies$''] Soit $L \in \wp(\Sigma^*)$\/ reconnu par un monoïde $M$ fini, un morphisme $\mu$\/ et un ensemble $P$ : $L = \mu^{-1}(P)$. Posons $\mathcal{A} = (\Sigma', \mathcal{Q}, I, F, \delta)$\/ avec

			\vspace{-5mm}
			\begin{multicols}{4}
				\[\Sigma' = \Sigma\]
				\[\mathcal{Q} = M\]
				\[I = \{1_M\}\]
				\[F = P\]
			\end{multicols}
			\vspace{-7mm}\[
				\delta = \{(q,\ell, q') \in \mathcal{Q} \times \Sigma \times \mathcal{Q}  \mid q \cdot \mu(\ell) = q'\}.
			\]
			Montrons que $\mathcal{L}(\mathcal{A}) = L$.
			Soit $w \in \mathcal{L}(\mathcal{A})$. Il existe une exécution acceptante \[
				1_M = q_0 \xrightarrow{w_1} q_1 \to \cdots \xrightarrow{w_n} q_n \in P
			.\] Or, $\mu(w_1\ldots w_n) = \prod_{i=1}^n \mu(w_i) = q_0 \prod_{i=1}^n \mu(w_i) = q_0 \mu(w_1) \cdot \prod_{i=1}^n \mu(w_i) = q_1 \prod_{i=1}^n \mu(w_i) = q_n \in P$.
	\end{itemize}
\end{prv}


	}
	\def\addmacros#1{#1}
}

{
	\chap[2]{Algorithmes probabilistes}
	\minitoc
	\renewcommand{\cwd}{../cours/chap02/}
	\addmacros{
		\section{Motivation}

\lettrine On place au centre de la classe 40 bonbons. On en distribue un chacun. Si, par exemple, chacun choisit un bonbon et, au \textit{top} départ, prennent celui choisi.
Il est probable que plusieurs choisissent le même. Comme gérer lorsque plusieurs essaient d'accéder à la mémoire ?

Deuxièmement, sur l'ordinateur, plusieurs applications tournent en même temps. Pour le moment, on considérait qu'un seul programme était exécuté, mais, le \textsc{pc} ne s'arrête pas pendant l'exécution du programme.

On s'intéresse à la notion de \guillemotleft~processus~\guillemotright\ qui représente une tâche à réaliser.
On ne peut pas assigner un processus à une unité de calcul, mais on peut \guillemotleft~allumer~\guillemotright\ et \guillemotleft~éteindre~\guillemotright\ un processus.
Le programme allumant et éteignant les processus est \guillemotleft~l'ordonnanceur.~\guillemotright\@ Il doit aussi s'occuper de la mémoire du processus (chaque processus à sa mémoire séparée).

On s'intéresse, dans ce chapitre, à des programmes qui \guillemotleft~partent du même~\guillemotright\ : un programme peut créer un \guillemotleft~fil d'exécution~\guillemotright\ (en anglais, \textit{thread}). Le programme peut gérer les fils d'exécution qu'il a créé, et éventuellement les arrêter.
Les fils d'exécutions partagent la mémoire du programme qui les a créé.

En C, une tâche est représenté par une fonction de type \lstinline[language=c]!void* tache(void* arg)!. Le type \lstinline[language=c]!void*!\ est l'équivalent du type \lstinline[language=caml]!'a! : on peut le \textit{cast} à un autre type (comme \lstinline[language=c]-char*-).

\begin{lstlisting}[language=c,caption=Création de \textit{thread}s en C]
void* tache(void* arg) {
	printf("%s\n", (char*) arg);
	return NULL;
}

int main() {
	pthread_t p1, p2;

	printf("main: begin\n");

	pthread_create(&p1, NULL, tache, "A");
	pthread_create(&p2, NULL, tache, "B");

	pthread_join(p1, NULL);
	pthread_join(p2, NULL);

	printf("main: end\n");

	return 0;
}
\end{lstlisting}

\begin{lstlisting}[language=c,caption=Mémoire dans les \textit{thread}s en C]
int max = 10;
volatile int counter = 0;

void* tache(void* arg) {
	char* letter = arg;
	int i;

	printf("%s begin [addr of i: %p] \n", letter, &i);

	for(i = 0; i < max; i++) {
		counter = counter + 1;
	}

	printf("%s : done\n", letter);
	return NULL;
}

int main() {
	pthread_t p1, p2;

	printf("main: begin\n");

	pthread_create(&p1, NULL, tache, "A");
	pthread_create(&p2, NULL, tache, "B");

	pthread_join(p1, NULL);
	pthread_join(p2, NULL);

	printf("main: end\n");

	return 0;
}
\end{lstlisting}

Dans les \textit{thread}s, les variables locales (comme \texttt{i}) sont séparées en mémoire. Mais, la variable \texttt{counter} est modifiée, mais elle ne correspond pas forcément à $2 \times \texttt{max}$. En effet, si \texttt{p1} et \texttt{p2} essaient d'exécuter au même moment de réaliser l'opération \lstinline[language=c]-counter = counter + 1-, ils peuvent récupérer deux valeurs identiques de \texttt{counter}, ajouter 1, puis réassigner \texttt{counter}.
Ils \guillemotleft~se marchent sur les pieds.~\guillemotright\ 

Parmi les opérations, on distingue certaines dénommées \guillemotleft~atomiques~\guillemotright\ qui ne peuvent pas être séparées. L'opération \lstinline[language=c]-i++- n'est pas atomique, mais la lecture et l'écriture mémoire le sont.

\begin{defn}
	On dit d'une variable qu'elle est \textit{atomique} lorsque l'ordonnanceur ne l'interrompt pas.
\end{defn}

\begin{exm}
	L'opération \lstinline[language=c]-counter = counter + 1- exécutée en série peut être représentée comme ci-dessous. Avec \texttt{counter} valant 40, cette exécution donne 42.
	\begin{table}[H]
		\centering
		\begin{tabular}{l|l}
			Exécution du fil A & Exécution du fil B\\ \hline
			(1)~$\mathrm{reg}_1 \gets \texttt{counter}$ & (4)~$\mathrm{reg}_2 \gets \texttt{counter}$ \\
			(2)~$\mathrm{reg}_1{++}$ & (5)~$\mathrm{reg}_2{++}$ \\
			(3)~$\texttt{counter} \gets \mathrm{reg}_1$ & (6)~$\texttt{counter} \gets \mathrm{reg}_2$
		\end{tabular}
	\end{table}
	\noindent Mais, avec l'exécution en simultanée, la valeur de \texttt{counter} sera 41.
	\begin{table}[H]
		\centering
		\begin{tabular}{l|l}
			Exécution du fil A & Exécution du fil B\\ \hline
			(1)~$\mathrm{reg}_1 \gets \texttt{counter}$ & (2)~$\mathrm{reg}_2 \gets \texttt{counter}$ \\
			(3)~$\mathrm{reg}_1{++}$ & (5)~$\mathrm{reg}_2{++}$ \\
			(4)~$\texttt{counter} \gets \mathrm{reg}_1$ & (6)~$\texttt{counter} \gets \mathrm{reg}_2$
		\end{tabular}
	\end{table}
	\noindent Il y a \textit{entrelacement} des deux fils d'exécution.
\end{exm}

\begin{rmk}[Problèmes de la programmation concurrentielle]
	\begin{itemize}
		\item Problème d'accès en mémoire,
		\item Problème du rendez-vous,\footnote{Lorsque deux programmes terminent, ils doivent s'attendre pour donner leurs valeurs.}
		\item Problème du producteur-consommateur,\footnote{Certains programmes doivent ralentir ou accélérer.}
		\item Problème de l'entreblocage,\footnote{\textit{c.f.} exemple ci-après.}
		\item Problème famine, du dîner des philosophes.\footnote{Les philosophes mangent autour d'une table, et mangent du riz avec des baguettes. Ils décident de n'acheter qu'une seule baguette par personne. Un philosophe peut, ou penser, ou manger. Mais, pour manger, ils ont besoin de deux baguettes. S'ils ne mangent pas, ils meurent.}
	\end{itemize}
\end{rmk}

\begin{exm}[Problème de l'entreblocage]~

	\begin{table}[H]
		\centering
		\begin{tabular}{l|l|l}
			Fil A & Fil B & Fil C\\ \hline
			RDV(C) & RDV(A) & RDV(B)\\
			RDV(B) & RDV(C) & RDV(A)\\
		\end{tabular}
		\caption{Problème de l'entreblocage}
	\end{table}
\end{exm}

Comment résoudre le problème des deux incrementations ? Il suffit de \guillemotleft~mettre un verrou.~\guillemotright\ Le premier fil d'exécution \guillemotleft~s'enferme~\guillemotright\ avec l'expression \lstinline[language=c]!count++!, le second fil d'exécution attend que l'autre sorte pour pouvoir entrer et s'enfermer à son tour.


		\section{Continuité}

\begin{exm}
	Dans l'exercice 2, chaque fonction $f_n : t \mapsto t^n$\/ est continue sur $[0,1]$\/ mais la limite $f$\/ n'est pas continue sur $[0,1]$\/ (car elle n'est pas continue en $1$).
\end{exm}

\begin{thm}
	Soit $a$\/ un réel dans un intervalle $T$\/ de $\R$. Si une suite de fonctions $(f_n)_{n\in\N}$\/ continues en $a$\/ converge uniformément sur $T$\/ vers une fonction $f$, alors $f$\/ est aussi continue en $a$.
\end{thm}

\begin{prv}
	On suppose les fonctions $f_n$\/ continues en $a$\/ ($f_n(x) \longrightarrow f_n(a)$) et que la suite de fonctions $(f_n)_{n\in\N}$\/ converge uniformément vers $f$\/ ($\sup\:|f_n -f| \longrightarrow 0$). On veut montrer que $f$\/ est continue en $a$\/ : $f(x) \tendsto{x \to a} f(a)$, i.e.\ \[
		\forall \varepsilon > 0,\:\exists \delta > 0,\: \forall x \in T,\quad|x-a| \le \delta \implies |f(x) - f(a)| \le \varepsilon
	.\]
	Soit $\varepsilon > 0$. On calcule \[
		\big|f(x) - f(a)\big| \le \big|f(x) - f_n(x)\big| + \big|f_n(x) - f_n(a)\big| + \big|f_n(a) - f(a)\big|
	\] par inégalité triangulaire. Or, par hypothèse, il existe un rang $N \in \N$\/ (qui ne dépend pas de $x$\/ ou de $a$), tel que, $\forall n \ge N$, $\big|f(x) - f_n(x)\big| \le \frac{1}{3} \varepsilon$, et $\big|f_n(a) - f(a)\big| \le \frac{1}{3} \varepsilon$.
	De plus, par hypothèse, il existe $\delta >0$\/ tel que si $|x - a| \le \delta$, alors $|f_n(x) - f_n(a)| \le \frac{1}{3}\varepsilon$.\footnote{C'est là où l'hypothèse de la convergence uniforme est utilisée : on a besoin que le $N$\/ ne dépende pas de $x$\/ car on le fait varier.}
	On en déduit que $\big|f(x) - f(a)\big| \le \varepsilon$.
\end{prv}

\begin{crlr}
	Soit $T$\/ un intervalle de $\R$. Si une suite de fonctions $(f_n)_{n\in\N}$\/ continues sur $T$\/ converge uniformément sur $T$\/ vers une fonction continue sur $T$.
\end{crlr}

\begin{met}[Stratégie de la barrière]
	\begin{enumerate}
		\item La continuité (la dérivabilité aussi) est une propriété {\it locale}. Pour montrer qu'une fonction est continue sur un intervalle $T$, il suffit donc de montrer qu'elle est continue sur tout segment inclus dans $T$.
		\item Mais, la convergence uniforme est une propriété {\it globale}. La convergence sur tout segment inclus dans un intervalle n'implique pas la convergence uniforme sur l'intervalle (voir l'exercice 2).
		\item On n'écrit pas \[
				\substack{\ds\text{convergence uniforme}\\\ds\text{avec barrière}} \mathop{\red\implies} \substack{\ds\text{convergence uniforme}\\\ds\text{sans barrière}} \implies \substack{\ds\text{continuité}\\\ds\text{sans barrière}}
			\] mais plutôt \[
				\substack{\ds\text{convergence uniforme}\\\ds\text{avec barrière}} \implies \substack{\ds\text{continuité}\\\ds\text{avec barrière}} \implies \substack{\ds\text{continuité}\\\ds\text{sans barrière}}
			.\]
		\item Si, pour tous $a$\/ et $b$, $f$\/ est bornée sur $[a,b] \subset T$, mais cela n'implique pas que $f$\/ est bornée. Contre-exemple : la fonction $f : x \mapsto \frac{1}{x}$\/ est bornée sur tout intervalle $[a,b]$\/ avec $a$, $b \in \R^+_*$, \red{\sc mais} $f$\/ n'est pas bornée sur $]0,+\infty[$.
	\end{enumerate}
\end{met}

\begin{thm}[double-limite ou d'interversion des limites]
	Soit une suite de fonctions $(f_n)_{n\in\N}$\/ définies sur un intervalle $T$, et, soit $a$\/ une extrémité (éventuellement infinie)\footnote{autrement dit, $a \in \bar\R = \R \cup \{+\infty,-\infty\}$} de cet intervalle. Si la suite de fonctions $(f_n)_{n\in\N}$\/ converge \underline{uniformément} sur $T$\/ vers $f$\/ et si chaque fonction $f_n$\/ admet une limite finie $b_n$\/ en $a$, alors la suite de réels $b_n$\/ converge vers un réel $b$, et $\lim_{t\to a} f(t) = b$. Autrement dit, \[
		\lim_{t\to a} \Big(\underbrace{\lim_{n\to +\infty} f_n(t)}_{f(x)}\Big) = \lim_{n\to +\infty} \Big(\underbrace{\lim_{t\to a} f_n(t)}_{b_n}\Big)
	.\] \qed
\end{thm}

\begin{rmkn}
	Le théorème de la double-limite \guillemotleft~contient~\guillemotright\ le théorème 6 (théorème de préservation/transmission de la continuité), c'est un cas particulier. En effet, si les fonctions $f_n$\/ sont continues, alors \[
		\lim_{x \to a}f(x) = \underbrace{\lim_{n\to +\infty} f_n(a)}_{f(a)}
	.\]
\end{rmkn}


		\section{Endomorphismes adjoints}

\begin{defn}
	On dit qu'un endomorphisme $f : E \to E$\/ est \textit{autoadjoint} si \[
		\forall (\vec{u}, \vec{v}) \in E^2,\quad \big<f(\vec{u})\:\big|\:\vec{v}\big> = \big<\vec{u}\:\big|\:f(\vec{v})\big>
	.\] 
\end{defn}

Un endomorphisme autoadjoint est aussi appelé endomorphisme \textit{symétrique} (\textit{c.f.}\ proposition suivante). L'ensemble des endomorphismes autoadjoints est noté $\mathcal{S}(E)$.

\begin{prop}
	Un endomorphism est autoadjoint si, et seulement si la matrice de $F$\/ dans une base \ul{orthonormée} $\mathcal{B}$\/ est orthogonale.
	Autrement dit : \[
		f \in \mathcal{S}(E) \iff \big[\:f\:\big]_\mathcal{B} \in \mathcal{S}_n(\R)
	.\]
\end{prop}

\begin{prv}
	\begin{description}
		\item[$\implies$] Soit $\mathcal{B} = (\vec{\varepsilon}_1, \ldots, \vec{\varepsilon}_n)$\/ une base orthonormée de $E$. Ainsi, \[
				\forall i,\:\forall j,\quad \big<f(\vec{\varepsilon}_i)\:\big|\: \vec{\varepsilon}_j\big> = \big<\vec{\varepsilon}_i\:\big|\:f(\vec{\varepsilon}_j)\big>
			.\] On pose $\big[\:f\:\big]_{\mathcal{B}} = (a_{i,j})$\/ : \[
				\begin{pNiceMatrix}[last-col,last-row]
					\quad&\quad&a_{1,j}&\quad&\quad&\vec{\varepsilon}_i\\
					&&&&&\\
					\quad&\quad&a_{i,j}&\quad&\quad&\vec{\varepsilon}_i\\
					&&&&&\\
					\quad&\quad&a_{n,j}&\quad&\quad&\vec{\varepsilon}_n\\
					f(\vec{\varepsilon}_i)&&f(\vec{\varepsilon}_j)&&f(\vec{\varepsilon}_n)\\
				\end{pNiceMatrix}
			.\] Ainsi, $f(\vec{\varepsilon}_j) = a_{1,j} \vec{\varepsilon}_1 + \cdots + a_{i,j} \vec{\varepsilon}_i + \cdots + a_{n,j} \vec{\varepsilon}_n$. D'où, $\left<\vec{\varepsilon}_i  \mid f(\vec{\varepsilon}_j) \right> = a_{i,j}$\/ car la base $\mathcal{B}$\/ est orthonormée.
			De même avec l'autre produit scalaire, $\left< f(\vec{\varepsilon}_i)  \mid \vec{\varepsilon}_j \right>$, d'où $a_{i,j} = a_{j,i}$\/ par symétrie du produit scalaire. On en déduit que $\big[\:f\:\big]_\mathcal{B} \in \mathcal{S}_n(\R)$.
		\item[$\impliedby$]
			Si $\big[\: f\:\big]_\mathcal{B} \in \mathcal{S}_n(\R)$, alors $\left<f(\vec{\varepsilon}_i)  \mid \vec{\varepsilon}_j \right> = \left<\vec{\varepsilon}_i  \mid f(\vec{\varepsilon}_j)\right>$.
			Or, on pose $\vec{u} = x_1 \vec{\varepsilon}_1+ \cdots + x_n \vec{\varepsilon}_n$, et $\vec{v} = y_1 \vec{\varepsilon}_1 + \cdots + y_n \vec{\varepsilon}_n$.
			\begin{align*}
				\left<f(\vec{u})  \mid \vec{v} \right> &= \left<x_1 f(\vec{\varepsilon}_1) + \cdots + x_n f(\vec{\varepsilon}_n)  \mid y_1 f(\vec{\varepsilon}_1) + \cdots + y_n f(\vec{\varepsilon}_n) \right> \\
				&= \Big<\sum_{i=1}^n x_i f(\vec{\varepsilon}_i)\:\Big|\: \sum_{j=1}^n y_j \vec{\varepsilon}_j\Big> \\
				&= \sum_{i,j \in \llbracket 1,n \rrbracket}  x_i y_j \left<f(\vec{\varepsilon}_i) \mid \vec{\varepsilon}_j \right>\\
			\end{align*}
			De même en inversant $\vec{u}$\/ et $\vec{v}$.
			On en déduit donc $\left<f(\vec{u} \mid \vec{v} \right> = \left<\vec{u}  \mid f(\vec{v}) \right>$.
	\end{description}
\end{prv}

\begin{exo}
	\begin{enumerate}
		\item Si $f$\/ est autoadjoint, montrons que $\Ker f \perp \Im f$, et $\Ker f \oplus \Im f$.
			On suppose $\forall \vec{u}$, $\forall \vec{v}$, $\left<f(\vec{u}) \mid \vec{v} \right> = \left<\vec{u}  \mid f(\vec{v}) \right>$.
			Soit $\vec{u} \in \Ker f$, et soit $\vec{v} \in \Im f$.
			On sait que $f(\vec{u}) = \vec{0}$, et qu'il existe $\vec{x} \in E$\/ tel que $\vec{v} = f(\vec{x})$.
			Ainsi, \[
				\left<\vec{u}  \mid \vec{v} \right>
				= \left<\vec{u}  \mid f(\vec{x}) \right>
				= \left<f(\vec{u})  \mid \vec{x} \right>
				= 0
			.\] 
			D'où $\vec{u} \perp \vec{v}$. Ainsi, $\Ker f \perp \Im f$.


			De plus, $E$\/ est de dimension finie, d'où, d'après le théorème du rang, \[
				\dim \Ker f + \dim \Im f = \dim E
			.\] Aussi, $\Ker f \oplus (\Ker f)^\perp = E$, donc $\dim(\Ker f) + \dim(\Ker f)^\perp = \dim E$.
			On en déduit donc que $\dim(\Im f)= \dim(\Ker f)^\perp$.
			Or, $\Im f \subset (\Ker f)^\perp$\/ car $\Im f \perp \Ker f$.
			Ainsi $\Im f = (\Ker f)^\perp$, on en déduit que \[
				\Im f \oplus \Ker f = E
			.\]
			\begin{description}
				\item[$\impliedby$] 
					Soit $p$\/ la projection sur $F$\/ parallèlement à $G$.
					Supposons l'endomorphisme $P$\/ autoadjoint.
					D'après la question 1., le $\Ker p \perp \Im p$.
					Ainsi, $F = \Im p$\/ et $G = \Ker p$.
					D'où, $F \perp G$, $p$\/ est donc une projection orthogonale.
				\item[$\implies$]
					Réciproquement, supposons $p$\/ une projection orthogonale.
					Soit $\mathcal{B} = (\vec{\varepsilon}_1, \ldots, \vec{\varepsilon}_q)$\/ une base orthonormée de $F$.
					Ainsi, pour tout $\vec{x} \in E$, \[
						p(\vec{x}) = \sum_{i = 1}^q \left<\vec{x}  \mid \vec{\varepsilon}_i \right>\,\vec{\varepsilon}_i
					.\] 
					On veut montrer que l'endomorphisme $p$\/ est autoadjoint.
					Soient $\vec{u}$\/ et $\vec{v}$\/ deux vecteurs de $E$.
					\begin{align*}
						\left<p(\vec{u})  \mid \vec{v} \right>
						= \Big<\sum_{i=1}^q \left< \vec{u}\mid \vec{\varepsilon}_i \right>\vec{\varepsilon}_i\:\Big|\; \vec{v}\;\Big>
						&= \sum_{i=1}^q \left<u  \mid \vec{\varepsilon}_{i} \right>\: \left< \vec{\varepsilon}_i  \mid v\right>\\
						&= \sum_{i=1}^q \left<v  \mid \vec{\varepsilon}_i \right>\:\left<\vec{\varepsilon}_i   \mid u\right> \\
						&= \left< \vec{u}  \mid p(\vec{v}) \right> \\
					\end{align*}

					Autre méthode, pour tous vecteurs $\vec{u}$\/ et $\vec{v}$\/ de $E$,
					\begin{align*}
						\left<p(\vec{u})  \mid \vec{v} \right>
						&= \left<p(\vec{u})  \mid p(\vec{v}) + \vec{v} - p(\vec{v}) \right> \\
						&= \left< p(\vec{u})  \mid p(\vec{v}) \right> + \left<p(\vec{u})  \mid  \vec{v} - p(\vec{v}) \right> \\
						&= \left<p(\vec{u}) \mid p(\vec{v}) \right> + \left<u - p(\vec{u})  \mid p(\vec{v}) \right> \\
						&= \left<\vec{u}  \mid p(\vec{v}) \right> \\
					\end{align*}
					car $p$\/ est orthogonale.
			\end{description}
	\end{enumerate}
\end{exo}


\begin{prop-defn}
	Si $f$\/ est un endomorphisme d'un espace euclidien $E$, alors il existe un unique endomorphisme de $E$, noté $f^\star$\/ et appelé l'\textit{adjoint} de $f$, tel que \[
		\forall (\vec{u},\vec{v}) \in E^2,\quad\quad \left<f^\star(\vec{u})  \mid \vec{v} \right> =  \left<\vec{u}  \mid f(\vec{v}) \right>
	.\] 
	Si $A$\/ est la matrice $f$\/ dans une base orthonormée $\mathcal{B}$\/ de $E$, alors $A^\top$\/ est la matrice de $f^\star$\/ dans~$\mathcal{B}$\/ : \[
		\big[\:f\:\big]_\mathcal{B} = \big[\:f\:\big]_\mathcal{B}^\top
	.\]
\end{prop-defn}

\begin{prv}
	Soit $\vec{u} \in E$. L'application \begin{align*}
		\varphi: E &\longrightarrow \R \\
		\vec{v} &\longmapsto \left<\vec{u}  \mid f(\vec{v}) \right>.
	\end{align*}
	La forme $\varphi$\/ est linéaire car $\varphi(\alpha_1 \vec{v}_1 + \alpha_2 \vec{v}_2) = \left<\vec{u} \mid f(\alpha_1 \vec{v}_1 + \alpha_2 \vec{v}_2) \right> = \left<\vec{u}  \mid \alpha_1 f(\vec{v}_1) + \alpha_2 f(\vec{v}_2) \right> = \alpha_1\left<\vec{u}  \mid  f(\vec{v}) \right> + \alpha_2 \left<\vec{u}  \mid f(\vec{v}_2) \right> = \alpha_1 \varphi(\vec{v}_1) + \alpha_2 \varphi(\vec{v}_2)$.
	D'où, d'après le théorème de \textsc{Riesz}, il existe un \ul{unique} vecteur $\vec{a} \in E$\/ tel que $\varphi(\vec{v}) = \left<\vec{a}  \mid \vec{v} \right>$\/ pour tout $\vec{v} \in E$.
	Ainsi, pour tout vecteur $\vec{v} \in E$, $\left<\vec{u}  \mid f(\vec{v}) \right> = \left<\vec{a}  \mid \vec{v} \right>$.
	On note $\vec{a} = f^\star(\vec{u})$.
	Soit l'application \begin{align*}
		f^\star : E &\longrightarrow E \\
		\vec{u} &\longmapsto f^\star(\vec{u}).
	\end{align*}
	La démonstration telle que $f^\star $\/ est linéaire est dans le poly.
	L'application $f^\star$\/ vérifie : $\left< \vec{u} \mid f(\vec{v}) \right> = \left<f^\star (\vec{u})  \mid \vec{v} \right>$, pour tous vecteurs $\vec{u}$\/ et $\vec{v}$.
	\textsl{Quelle est la matrice de $f^\star$, dans une base orthonormée ?}\@
	Soit $\mathcal{B}$\/ une base orthonormée de $E$, et soient $A = \big[\:f\:\big]_\mathcal{B}$, $B = \big[\:f^\star \:\big]_\mathcal{B}$, $U = \big[\:\vec{u}\:\big]_\mathcal{B}$, et $V = \big[\:\vec{v}\:\big]_\mathcal{B}$.
	Les matrices $U$\/ et $V$\/ sont des vecteurs colonnes, et $A$\/ et $B$\/ sont des matrices carrées.
	Ainsi, \[
		U^\top \cdot A \cdot V = \left< \vec{u}  \mid f(\vec{v}) \right> 
		= \left<f^\star (\vec{u})  \mid \vec{v} \right> = (B\cdot U)^\top \cdot V,
	\] ce qui est vrai quelque soit les vecteurs colonnes $U$\/ et $V$.
	D'où, $\forall U$, $\forall V$, $U^\top \cdot \big(A \cdot V\big) = U^\top \cdot \big(B^\top \cdot V\big)$.
	Ainsi, pour tous vecteurs $U$\/ et $V$, \[
		U^\top \cdot \Big[ (AV) - (B^\top V)\Big] = 0
	.\] En particulier, si $U = (AV) - (B^\top V)$, le produit scalaire $\left<\vec{u}  \mid \vec{u} \right>$\/ est nul, donc $U = 0$.
	Ainsi, \[
		\forall V,\quad A\cdot V = B^\top\cdot  V
	.\] De même, on conclut que $A = B^\top$. On en déduit donc que \[
	\big[\:f^\star\:\big]_\mathcal{B} = \big[\:f\:\big]_\mathcal{B}^\top
	.\]
\end{prv}

Les propriétés suivantes sont vrais :
\begin{itemize}
	\item $(f  \circ g)^\star  = g^\star \circ f^\star$, \quad $(f^\star)^\star = f$, \quad et \quad $(\alpha f + \beta g)^\star  = \alpha f^\star + \beta g^\star $\/ ;
	\item $(A\cdot B)^\top  = B^\top \cdot A^\top$, \quad $(A^\top)^\top = A$, \quad et \quad $(\alpha A + \beta B)^\top = \alpha A^\top+ \beta B^\top $.
\end{itemize}
Des deuxièmes et troisièmes points,  il en résulte que les applications $f \mapsto f^\star$, et $A \mapsto A^\top$\/ sont des applications involutives.



		\clearpage
		\setcounter{section}{0}		\renewcommand{\thesection}{\llap{Annexe }\thechapter.\Alph{section}}
		\renewcommand{\thesectionnum}{\Alph{section}}
		\section{Programmation dynamique}

On rappelle le problème \textsc{Knapsack} : \[
	\begin{cases}
		\text{\textbf{Entrée}} &: n \in \N,\: w \in (\N^*)^n,\: v \in (\N^*)^n,\: P \in \N\\
		\text{\textbf{Sortie}} &:\max_{x \in \{0,1\}^n} \big\{\left<x,v \right> \:\big|\: \left<x,w \right> \le P \}.
	\end{cases}
\]

On pose \[\textsc{sad}(n, w, v, P) = \max_{x \in \{0,1\}^n} \big\{\left<x,v \right> \:\big|\: \left<x,w \right> \le P \},\] et \[\mathrm{sol}(n, w, v, P) = \{\left<x,v \right>  \mid \left<x,w \right> \le P,\: x \in \{0,1\}^n \}.\] Lorsque $y \in \R^n$, avec $y = (y_1, \ldots, y_n)$, on note $\R^{n-1} \owns \tilde y = (y_2, y_3, \ldots, 0)$. Ainsi, si $n > 0$,
\begin{align*}
	\mathrm{sol}(n, w, v, P) &= \{\left<x, v \right>  \mid \left< x,w \right> \le P,\: x \in \{0,1\}^n \text{ et } x_1 = 0\}\\
	&\quad\cup \{\left<x, v \right>  \mid \left< x,w \right> \le P,\: x \in \{0,1\}^n \text{ et } x_1 = 1\}\\
	&= \{\left<\tilde{x}, \tilde{v} \right>  \mid \left<\tilde{x}, \tilde{w} \right> \le P,\: x u\in \{0,1\}^n \text{ et } x_1 = 0\} \\
	&\quad\cup \{v_1 + \left<\tilde{x}, \tilde{v} \right>  \mid \left<\tilde{x},\tilde{w} \right> \le P - w_1,\: x \in \{0,1\}^n \text{ et } x_1 = 1\}\\
	&= \{\left<y, \tilde{v} \right>  \mid \left<y, \tilde{w} \right> \le P \text{ et } y \in \{0,1\}^{n-1}\}  \\
	&\quad\cup \{v_1 + \left<y, \tilde{v} \right>  \mid \left<y, \tilde{w} \right> \le P - w_1 \text{ et } y \in \{0,1\}^{n-1}\}
\end{align*}
D'où, par passage au $\max$, si $n > 0$,
\begin{align*}
	\textsc{sad}(n, w, v, P) &= \max( \\
	&\quad\quad \max \{\left<y  \mid \tilde{v} \right>  \mid \left<y, \tilde{w} \right> \le P \text{ et } y \in \{0,1\}^{n-1}\}\\
	&\quad\quad v_1 + \max \{\left<y  \mid \tilde{v} \right>  \mid \left<y, \tilde{w} \right> \le P - w_1 \text{ et } y \in \{0,1\}^{n-1}\} \\
	&) = \max\!\big(\textsc{sad}(n-1, \tilde{w}, \tilde{v}, P), v_1 + \textsc{sad}(n-1, \tilde{w}, \tilde{v}, P - w_1)\big).
\end{align*}
Si $n = 0$, alors $\textsc{sad}(0, v, w, P) = 0$.

\begin{rmk}
	Si on le code \textit{tel quel}, il y aura $\mathcal{O}(2^n)$ appels récursifs. Mais, on a $(n+1)(P+1)$\/ sous-problèmes.
\end{rmk}

Notons alors, pour $n$, $v$, $w$, $P$ fixés, $(s_{i,j})_{\substack{i \in \llbracket 1,n \rrbracket\\ j \in \llbracket 0,P \rrbracket}}$\/ tel que \[
	s_{i,j} = \textsc{sad}\Big(n-i, v_{\big|\llbracket i+1,n \rrbracket}, w_{\big|\llbracket i+1,n \rrbracket}, j\Big)
.\]
On a alors $\textsc{sad}(n, v, w, P) = s_{0,P}$. Ainsi, pour $j \in \llbracket 0,P \rrbracket$, $s_{n,j} = 0$ ; pour $i \in \llbracket 0,n \rrbracket$, $s_{i,0} = 0$ ; \[
	s_{i,j} = \max(s_{i+1,j}, v_{i+1} + s_{i+1,j- w_{i+1}})
;\] et, si $w_{i+1} > j$, alors $s_{i,j} = s_{i+1,j}$.

La complexité de remplissage de la matrice est en $\mathcal{O}(n\,P)$\/ en temps et en espace.
On n'a pas prouvé $\text{\textbf{P}} = \text{\textbf{NP}}$, la taille de l'entrée est
\begin{itemize}
	\item pour un entier $n$\/ : $\log_2(n)$,
	\item pour un tableau de $n$\/ entiers : $n \log_2(n)$,
	\item pour un tableau de $n$\/ entiers : $n \log_2(n)$,
	\item pour un entier $P$ : {\color{red}$\log_2(P)$}.
\end{itemize}
Vis à vis de la taille de l'entrée, la complexité de remplissage est {\color{red}exponentielle}.


	}
	\def\addmacros#1{#1}
}

{
	\chap[3]{Apprentissage}
	\minitoc
	\renewcommand{\cwd}{../cours/chap03/}
	\addmacros{
		\section{Motivation}

\lettrine On place au centre de la classe 40 bonbons. On en distribue un chacun. Si, par exemple, chacun choisit un bonbon et, au \textit{top} départ, prennent celui choisi.
Il est probable que plusieurs choisissent le même. Comme gérer lorsque plusieurs essaient d'accéder à la mémoire ?

Deuxièmement, sur l'ordinateur, plusieurs applications tournent en même temps. Pour le moment, on considérait qu'un seul programme était exécuté, mais, le \textsc{pc} ne s'arrête pas pendant l'exécution du programme.

On s'intéresse à la notion de \guillemotleft~processus~\guillemotright\ qui représente une tâche à réaliser.
On ne peut pas assigner un processus à une unité de calcul, mais on peut \guillemotleft~allumer~\guillemotright\ et \guillemotleft~éteindre~\guillemotright\ un processus.
Le programme allumant et éteignant les processus est \guillemotleft~l'ordonnanceur.~\guillemotright\@ Il doit aussi s'occuper de la mémoire du processus (chaque processus à sa mémoire séparée).

On s'intéresse, dans ce chapitre, à des programmes qui \guillemotleft~partent du même~\guillemotright\ : un programme peut créer un \guillemotleft~fil d'exécution~\guillemotright\ (en anglais, \textit{thread}). Le programme peut gérer les fils d'exécution qu'il a créé, et éventuellement les arrêter.
Les fils d'exécutions partagent la mémoire du programme qui les a créé.

En C, une tâche est représenté par une fonction de type \lstinline[language=c]!void* tache(void* arg)!. Le type \lstinline[language=c]!void*!\ est l'équivalent du type \lstinline[language=caml]!'a! : on peut le \textit{cast} à un autre type (comme \lstinline[language=c]-char*-).

\begin{lstlisting}[language=c,caption=Création de \textit{thread}s en C]
void* tache(void* arg) {
	printf("%s\n", (char*) arg);
	return NULL;
}

int main() {
	pthread_t p1, p2;

	printf("main: begin\n");

	pthread_create(&p1, NULL, tache, "A");
	pthread_create(&p2, NULL, tache, "B");

	pthread_join(p1, NULL);
	pthread_join(p2, NULL);

	printf("main: end\n");

	return 0;
}
\end{lstlisting}

\begin{lstlisting}[language=c,caption=Mémoire dans les \textit{thread}s en C]
int max = 10;
volatile int counter = 0;

void* tache(void* arg) {
	char* letter = arg;
	int i;

	printf("%s begin [addr of i: %p] \n", letter, &i);

	for(i = 0; i < max; i++) {
		counter = counter + 1;
	}

	printf("%s : done\n", letter);
	return NULL;
}

int main() {
	pthread_t p1, p2;

	printf("main: begin\n");

	pthread_create(&p1, NULL, tache, "A");
	pthread_create(&p2, NULL, tache, "B");

	pthread_join(p1, NULL);
	pthread_join(p2, NULL);

	printf("main: end\n");

	return 0;
}
\end{lstlisting}

Dans les \textit{thread}s, les variables locales (comme \texttt{i}) sont séparées en mémoire. Mais, la variable \texttt{counter} est modifiée, mais elle ne correspond pas forcément à $2 \times \texttt{max}$. En effet, si \texttt{p1} et \texttt{p2} essaient d'exécuter au même moment de réaliser l'opération \lstinline[language=c]-counter = counter + 1-, ils peuvent récupérer deux valeurs identiques de \texttt{counter}, ajouter 1, puis réassigner \texttt{counter}.
Ils \guillemotleft~se marchent sur les pieds.~\guillemotright\ 

Parmi les opérations, on distingue certaines dénommées \guillemotleft~atomiques~\guillemotright\ qui ne peuvent pas être séparées. L'opération \lstinline[language=c]-i++- n'est pas atomique, mais la lecture et l'écriture mémoire le sont.

\begin{defn}
	On dit d'une variable qu'elle est \textit{atomique} lorsque l'ordonnanceur ne l'interrompt pas.
\end{defn}

\begin{exm}
	L'opération \lstinline[language=c]-counter = counter + 1- exécutée en série peut être représentée comme ci-dessous. Avec \texttt{counter} valant 40, cette exécution donne 42.
	\begin{table}[H]
		\centering
		\begin{tabular}{l|l}
			Exécution du fil A & Exécution du fil B\\ \hline
			(1)~$\mathrm{reg}_1 \gets \texttt{counter}$ & (4)~$\mathrm{reg}_2 \gets \texttt{counter}$ \\
			(2)~$\mathrm{reg}_1{++}$ & (5)~$\mathrm{reg}_2{++}$ \\
			(3)~$\texttt{counter} \gets \mathrm{reg}_1$ & (6)~$\texttt{counter} \gets \mathrm{reg}_2$
		\end{tabular}
	\end{table}
	\noindent Mais, avec l'exécution en simultanée, la valeur de \texttt{counter} sera 41.
	\begin{table}[H]
		\centering
		\begin{tabular}{l|l}
			Exécution du fil A & Exécution du fil B\\ \hline
			(1)~$\mathrm{reg}_1 \gets \texttt{counter}$ & (2)~$\mathrm{reg}_2 \gets \texttt{counter}$ \\
			(3)~$\mathrm{reg}_1{++}$ & (5)~$\mathrm{reg}_2{++}$ \\
			(4)~$\texttt{counter} \gets \mathrm{reg}_1$ & (6)~$\texttt{counter} \gets \mathrm{reg}_2$
		\end{tabular}
	\end{table}
	\noindent Il y a \textit{entrelacement} des deux fils d'exécution.
\end{exm}

\begin{rmk}[Problèmes de la programmation concurrentielle]
	\begin{itemize}
		\item Problème d'accès en mémoire,
		\item Problème du rendez-vous,\footnote{Lorsque deux programmes terminent, ils doivent s'attendre pour donner leurs valeurs.}
		\item Problème du producteur-consommateur,\footnote{Certains programmes doivent ralentir ou accélérer.}
		\item Problème de l'entreblocage,\footnote{\textit{c.f.} exemple ci-après.}
		\item Problème famine, du dîner des philosophes.\footnote{Les philosophes mangent autour d'une table, et mangent du riz avec des baguettes. Ils décident de n'acheter qu'une seule baguette par personne. Un philosophe peut, ou penser, ou manger. Mais, pour manger, ils ont besoin de deux baguettes. S'ils ne mangent pas, ils meurent.}
	\end{itemize}
\end{rmk}

\begin{exm}[Problème de l'entreblocage]~

	\begin{table}[H]
		\centering
		\begin{tabular}{l|l|l}
			Fil A & Fil B & Fil C\\ \hline
			RDV(C) & RDV(A) & RDV(B)\\
			RDV(B) & RDV(C) & RDV(A)\\
		\end{tabular}
		\caption{Problème de l'entreblocage}
	\end{table}
\end{exm}

Comment résoudre le problème des deux incrementations ? Il suffit de \guillemotleft~mettre un verrou.~\guillemotright\ Le premier fil d'exécution \guillemotleft~s'enferme~\guillemotright\ avec l'expression \lstinline[language=c]!count++!, le second fil d'exécution attend que l'autre sorte pour pouvoir entrer et s'enfermer à son tour.


		\section{Continuité}

\begin{exm}
	Dans l'exercice 2, chaque fonction $f_n : t \mapsto t^n$\/ est continue sur $[0,1]$\/ mais la limite $f$\/ n'est pas continue sur $[0,1]$\/ (car elle n'est pas continue en $1$).
\end{exm}

\begin{thm}
	Soit $a$\/ un réel dans un intervalle $T$\/ de $\R$. Si une suite de fonctions $(f_n)_{n\in\N}$\/ continues en $a$\/ converge uniformément sur $T$\/ vers une fonction $f$, alors $f$\/ est aussi continue en $a$.
\end{thm}

\begin{prv}
	On suppose les fonctions $f_n$\/ continues en $a$\/ ($f_n(x) \longrightarrow f_n(a)$) et que la suite de fonctions $(f_n)_{n\in\N}$\/ converge uniformément vers $f$\/ ($\sup\:|f_n -f| \longrightarrow 0$). On veut montrer que $f$\/ est continue en $a$\/ : $f(x) \tendsto{x \to a} f(a)$, i.e.\ \[
		\forall \varepsilon > 0,\:\exists \delta > 0,\: \forall x \in T,\quad|x-a| \le \delta \implies |f(x) - f(a)| \le \varepsilon
	.\]
	Soit $\varepsilon > 0$. On calcule \[
		\big|f(x) - f(a)\big| \le \big|f(x) - f_n(x)\big| + \big|f_n(x) - f_n(a)\big| + \big|f_n(a) - f(a)\big|
	\] par inégalité triangulaire. Or, par hypothèse, il existe un rang $N \in \N$\/ (qui ne dépend pas de $x$\/ ou de $a$), tel que, $\forall n \ge N$, $\big|f(x) - f_n(x)\big| \le \frac{1}{3} \varepsilon$, et $\big|f_n(a) - f(a)\big| \le \frac{1}{3} \varepsilon$.
	De plus, par hypothèse, il existe $\delta >0$\/ tel que si $|x - a| \le \delta$, alors $|f_n(x) - f_n(a)| \le \frac{1}{3}\varepsilon$.\footnote{C'est là où l'hypothèse de la convergence uniforme est utilisée : on a besoin que le $N$\/ ne dépende pas de $x$\/ car on le fait varier.}
	On en déduit que $\big|f(x) - f(a)\big| \le \varepsilon$.
\end{prv}

\begin{crlr}
	Soit $T$\/ un intervalle de $\R$. Si une suite de fonctions $(f_n)_{n\in\N}$\/ continues sur $T$\/ converge uniformément sur $T$\/ vers une fonction continue sur $T$.
\end{crlr}

\begin{met}[Stratégie de la barrière]
	\begin{enumerate}
		\item La continuité (la dérivabilité aussi) est une propriété {\it locale}. Pour montrer qu'une fonction est continue sur un intervalle $T$, il suffit donc de montrer qu'elle est continue sur tout segment inclus dans $T$.
		\item Mais, la convergence uniforme est une propriété {\it globale}. La convergence sur tout segment inclus dans un intervalle n'implique pas la convergence uniforme sur l'intervalle (voir l'exercice 2).
		\item On n'écrit pas \[
				\substack{\ds\text{convergence uniforme}\\\ds\text{avec barrière}} \mathop{\red\implies} \substack{\ds\text{convergence uniforme}\\\ds\text{sans barrière}} \implies \substack{\ds\text{continuité}\\\ds\text{sans barrière}}
			\] mais plutôt \[
				\substack{\ds\text{convergence uniforme}\\\ds\text{avec barrière}} \implies \substack{\ds\text{continuité}\\\ds\text{avec barrière}} \implies \substack{\ds\text{continuité}\\\ds\text{sans barrière}}
			.\]
		\item Si, pour tous $a$\/ et $b$, $f$\/ est bornée sur $[a,b] \subset T$, mais cela n'implique pas que $f$\/ est bornée. Contre-exemple : la fonction $f : x \mapsto \frac{1}{x}$\/ est bornée sur tout intervalle $[a,b]$\/ avec $a$, $b \in \R^+_*$, \red{\sc mais} $f$\/ n'est pas bornée sur $]0,+\infty[$.
	\end{enumerate}
\end{met}

\begin{thm}[double-limite ou d'interversion des limites]
	Soit une suite de fonctions $(f_n)_{n\in\N}$\/ définies sur un intervalle $T$, et, soit $a$\/ une extrémité (éventuellement infinie)\footnote{autrement dit, $a \in \bar\R = \R \cup \{+\infty,-\infty\}$} de cet intervalle. Si la suite de fonctions $(f_n)_{n\in\N}$\/ converge \underline{uniformément} sur $T$\/ vers $f$\/ et si chaque fonction $f_n$\/ admet une limite finie $b_n$\/ en $a$, alors la suite de réels $b_n$\/ converge vers un réel $b$, et $\lim_{t\to a} f(t) = b$. Autrement dit, \[
		\lim_{t\to a} \Big(\underbrace{\lim_{n\to +\infty} f_n(t)}_{f(x)}\Big) = \lim_{n\to +\infty} \Big(\underbrace{\lim_{t\to a} f_n(t)}_{b_n}\Big)
	.\] \qed
\end{thm}

\begin{rmkn}
	Le théorème de la double-limite \guillemotleft~contient~\guillemotright\ le théorème 6 (théorème de préservation/transmission de la continuité), c'est un cas particulier. En effet, si les fonctions $f_n$\/ sont continues, alors \[
		\lim_{x \to a}f(x) = \underbrace{\lim_{n\to +\infty} f_n(a)}_{f(a)}
	.\]
\end{rmkn}


		\section{Endomorphismes adjoints}

\begin{defn}
	On dit qu'un endomorphisme $f : E \to E$\/ est \textit{autoadjoint} si \[
		\forall (\vec{u}, \vec{v}) \in E^2,\quad \big<f(\vec{u})\:\big|\:\vec{v}\big> = \big<\vec{u}\:\big|\:f(\vec{v})\big>
	.\] 
\end{defn}

Un endomorphisme autoadjoint est aussi appelé endomorphisme \textit{symétrique} (\textit{c.f.}\ proposition suivante). L'ensemble des endomorphismes autoadjoints est noté $\mathcal{S}(E)$.

\begin{prop}
	Un endomorphism est autoadjoint si, et seulement si la matrice de $F$\/ dans une base \ul{orthonormée} $\mathcal{B}$\/ est orthogonale.
	Autrement dit : \[
		f \in \mathcal{S}(E) \iff \big[\:f\:\big]_\mathcal{B} \in \mathcal{S}_n(\R)
	.\]
\end{prop}

\begin{prv}
	\begin{description}
		\item[$\implies$] Soit $\mathcal{B} = (\vec{\varepsilon}_1, \ldots, \vec{\varepsilon}_n)$\/ une base orthonormée de $E$. Ainsi, \[
				\forall i,\:\forall j,\quad \big<f(\vec{\varepsilon}_i)\:\big|\: \vec{\varepsilon}_j\big> = \big<\vec{\varepsilon}_i\:\big|\:f(\vec{\varepsilon}_j)\big>
			.\] On pose $\big[\:f\:\big]_{\mathcal{B}} = (a_{i,j})$\/ : \[
				\begin{pNiceMatrix}[last-col,last-row]
					\quad&\quad&a_{1,j}&\quad&\quad&\vec{\varepsilon}_i\\
					&&&&&\\
					\quad&\quad&a_{i,j}&\quad&\quad&\vec{\varepsilon}_i\\
					&&&&&\\
					\quad&\quad&a_{n,j}&\quad&\quad&\vec{\varepsilon}_n\\
					f(\vec{\varepsilon}_i)&&f(\vec{\varepsilon}_j)&&f(\vec{\varepsilon}_n)\\
				\end{pNiceMatrix}
			.\] Ainsi, $f(\vec{\varepsilon}_j) = a_{1,j} \vec{\varepsilon}_1 + \cdots + a_{i,j} \vec{\varepsilon}_i + \cdots + a_{n,j} \vec{\varepsilon}_n$. D'où, $\left<\vec{\varepsilon}_i  \mid f(\vec{\varepsilon}_j) \right> = a_{i,j}$\/ car la base $\mathcal{B}$\/ est orthonormée.
			De même avec l'autre produit scalaire, $\left< f(\vec{\varepsilon}_i)  \mid \vec{\varepsilon}_j \right>$, d'où $a_{i,j} = a_{j,i}$\/ par symétrie du produit scalaire. On en déduit que $\big[\:f\:\big]_\mathcal{B} \in \mathcal{S}_n(\R)$.
		\item[$\impliedby$]
			Si $\big[\: f\:\big]_\mathcal{B} \in \mathcal{S}_n(\R)$, alors $\left<f(\vec{\varepsilon}_i)  \mid \vec{\varepsilon}_j \right> = \left<\vec{\varepsilon}_i  \mid f(\vec{\varepsilon}_j)\right>$.
			Or, on pose $\vec{u} = x_1 \vec{\varepsilon}_1+ \cdots + x_n \vec{\varepsilon}_n$, et $\vec{v} = y_1 \vec{\varepsilon}_1 + \cdots + y_n \vec{\varepsilon}_n$.
			\begin{align*}
				\left<f(\vec{u})  \mid \vec{v} \right> &= \left<x_1 f(\vec{\varepsilon}_1) + \cdots + x_n f(\vec{\varepsilon}_n)  \mid y_1 f(\vec{\varepsilon}_1) + \cdots + y_n f(\vec{\varepsilon}_n) \right> \\
				&= \Big<\sum_{i=1}^n x_i f(\vec{\varepsilon}_i)\:\Big|\: \sum_{j=1}^n y_j \vec{\varepsilon}_j\Big> \\
				&= \sum_{i,j \in \llbracket 1,n \rrbracket}  x_i y_j \left<f(\vec{\varepsilon}_i) \mid \vec{\varepsilon}_j \right>\\
			\end{align*}
			De même en inversant $\vec{u}$\/ et $\vec{v}$.
			On en déduit donc $\left<f(\vec{u} \mid \vec{v} \right> = \left<\vec{u}  \mid f(\vec{v}) \right>$.
	\end{description}
\end{prv}

\begin{exo}
	\begin{enumerate}
		\item Si $f$\/ est autoadjoint, montrons que $\Ker f \perp \Im f$, et $\Ker f \oplus \Im f$.
			On suppose $\forall \vec{u}$, $\forall \vec{v}$, $\left<f(\vec{u}) \mid \vec{v} \right> = \left<\vec{u}  \mid f(\vec{v}) \right>$.
			Soit $\vec{u} \in \Ker f$, et soit $\vec{v} \in \Im f$.
			On sait que $f(\vec{u}) = \vec{0}$, et qu'il existe $\vec{x} \in E$\/ tel que $\vec{v} = f(\vec{x})$.
			Ainsi, \[
				\left<\vec{u}  \mid \vec{v} \right>
				= \left<\vec{u}  \mid f(\vec{x}) \right>
				= \left<f(\vec{u})  \mid \vec{x} \right>
				= 0
			.\] 
			D'où $\vec{u} \perp \vec{v}$. Ainsi, $\Ker f \perp \Im f$.


			De plus, $E$\/ est de dimension finie, d'où, d'après le théorème du rang, \[
				\dim \Ker f + \dim \Im f = \dim E
			.\] Aussi, $\Ker f \oplus (\Ker f)^\perp = E$, donc $\dim(\Ker f) + \dim(\Ker f)^\perp = \dim E$.
			On en déduit donc que $\dim(\Im f)= \dim(\Ker f)^\perp$.
			Or, $\Im f \subset (\Ker f)^\perp$\/ car $\Im f \perp \Ker f$.
			Ainsi $\Im f = (\Ker f)^\perp$, on en déduit que \[
				\Im f \oplus \Ker f = E
			.\]
			\begin{description}
				\item[$\impliedby$] 
					Soit $p$\/ la projection sur $F$\/ parallèlement à $G$.
					Supposons l'endomorphisme $P$\/ autoadjoint.
					D'après la question 1., le $\Ker p \perp \Im p$.
					Ainsi, $F = \Im p$\/ et $G = \Ker p$.
					D'où, $F \perp G$, $p$\/ est donc une projection orthogonale.
				\item[$\implies$]
					Réciproquement, supposons $p$\/ une projection orthogonale.
					Soit $\mathcal{B} = (\vec{\varepsilon}_1, \ldots, \vec{\varepsilon}_q)$\/ une base orthonormée de $F$.
					Ainsi, pour tout $\vec{x} \in E$, \[
						p(\vec{x}) = \sum_{i = 1}^q \left<\vec{x}  \mid \vec{\varepsilon}_i \right>\,\vec{\varepsilon}_i
					.\] 
					On veut montrer que l'endomorphisme $p$\/ est autoadjoint.
					Soient $\vec{u}$\/ et $\vec{v}$\/ deux vecteurs de $E$.
					\begin{align*}
						\left<p(\vec{u})  \mid \vec{v} \right>
						= \Big<\sum_{i=1}^q \left< \vec{u}\mid \vec{\varepsilon}_i \right>\vec{\varepsilon}_i\:\Big|\; \vec{v}\;\Big>
						&= \sum_{i=1}^q \left<u  \mid \vec{\varepsilon}_{i} \right>\: \left< \vec{\varepsilon}_i  \mid v\right>\\
						&= \sum_{i=1}^q \left<v  \mid \vec{\varepsilon}_i \right>\:\left<\vec{\varepsilon}_i   \mid u\right> \\
						&= \left< \vec{u}  \mid p(\vec{v}) \right> \\
					\end{align*}

					Autre méthode, pour tous vecteurs $\vec{u}$\/ et $\vec{v}$\/ de $E$,
					\begin{align*}
						\left<p(\vec{u})  \mid \vec{v} \right>
						&= \left<p(\vec{u})  \mid p(\vec{v}) + \vec{v} - p(\vec{v}) \right> \\
						&= \left< p(\vec{u})  \mid p(\vec{v}) \right> + \left<p(\vec{u})  \mid  \vec{v} - p(\vec{v}) \right> \\
						&= \left<p(\vec{u}) \mid p(\vec{v}) \right> + \left<u - p(\vec{u})  \mid p(\vec{v}) \right> \\
						&= \left<\vec{u}  \mid p(\vec{v}) \right> \\
					\end{align*}
					car $p$\/ est orthogonale.
			\end{description}
	\end{enumerate}
\end{exo}


\begin{prop-defn}
	Si $f$\/ est un endomorphisme d'un espace euclidien $E$, alors il existe un unique endomorphisme de $E$, noté $f^\star$\/ et appelé l'\textit{adjoint} de $f$, tel que \[
		\forall (\vec{u},\vec{v}) \in E^2,\quad\quad \left<f^\star(\vec{u})  \mid \vec{v} \right> =  \left<\vec{u}  \mid f(\vec{v}) \right>
	.\] 
	Si $A$\/ est la matrice $f$\/ dans une base orthonormée $\mathcal{B}$\/ de $E$, alors $A^\top$\/ est la matrice de $f^\star$\/ dans~$\mathcal{B}$\/ : \[
		\big[\:f\:\big]_\mathcal{B} = \big[\:f\:\big]_\mathcal{B}^\top
	.\]
\end{prop-defn}

\begin{prv}
	Soit $\vec{u} \in E$. L'application \begin{align*}
		\varphi: E &\longrightarrow \R \\
		\vec{v} &\longmapsto \left<\vec{u}  \mid f(\vec{v}) \right>.
	\end{align*}
	La forme $\varphi$\/ est linéaire car $\varphi(\alpha_1 \vec{v}_1 + \alpha_2 \vec{v}_2) = \left<\vec{u} \mid f(\alpha_1 \vec{v}_1 + \alpha_2 \vec{v}_2) \right> = \left<\vec{u}  \mid \alpha_1 f(\vec{v}_1) + \alpha_2 f(\vec{v}_2) \right> = \alpha_1\left<\vec{u}  \mid  f(\vec{v}) \right> + \alpha_2 \left<\vec{u}  \mid f(\vec{v}_2) \right> = \alpha_1 \varphi(\vec{v}_1) + \alpha_2 \varphi(\vec{v}_2)$.
	D'où, d'après le théorème de \textsc{Riesz}, il existe un \ul{unique} vecteur $\vec{a} \in E$\/ tel que $\varphi(\vec{v}) = \left<\vec{a}  \mid \vec{v} \right>$\/ pour tout $\vec{v} \in E$.
	Ainsi, pour tout vecteur $\vec{v} \in E$, $\left<\vec{u}  \mid f(\vec{v}) \right> = \left<\vec{a}  \mid \vec{v} \right>$.
	On note $\vec{a} = f^\star(\vec{u})$.
	Soit l'application \begin{align*}
		f^\star : E &\longrightarrow E \\
		\vec{u} &\longmapsto f^\star(\vec{u}).
	\end{align*}
	La démonstration telle que $f^\star $\/ est linéaire est dans le poly.
	L'application $f^\star$\/ vérifie : $\left< \vec{u} \mid f(\vec{v}) \right> = \left<f^\star (\vec{u})  \mid \vec{v} \right>$, pour tous vecteurs $\vec{u}$\/ et $\vec{v}$.
	\textsl{Quelle est la matrice de $f^\star$, dans une base orthonormée ?}\@
	Soit $\mathcal{B}$\/ une base orthonormée de $E$, et soient $A = \big[\:f\:\big]_\mathcal{B}$, $B = \big[\:f^\star \:\big]_\mathcal{B}$, $U = \big[\:\vec{u}\:\big]_\mathcal{B}$, et $V = \big[\:\vec{v}\:\big]_\mathcal{B}$.
	Les matrices $U$\/ et $V$\/ sont des vecteurs colonnes, et $A$\/ et $B$\/ sont des matrices carrées.
	Ainsi, \[
		U^\top \cdot A \cdot V = \left< \vec{u}  \mid f(\vec{v}) \right> 
		= \left<f^\star (\vec{u})  \mid \vec{v} \right> = (B\cdot U)^\top \cdot V,
	\] ce qui est vrai quelque soit les vecteurs colonnes $U$\/ et $V$.
	D'où, $\forall U$, $\forall V$, $U^\top \cdot \big(A \cdot V\big) = U^\top \cdot \big(B^\top \cdot V\big)$.
	Ainsi, pour tous vecteurs $U$\/ et $V$, \[
		U^\top \cdot \Big[ (AV) - (B^\top V)\Big] = 0
	.\] En particulier, si $U = (AV) - (B^\top V)$, le produit scalaire $\left<\vec{u}  \mid \vec{u} \right>$\/ est nul, donc $U = 0$.
	Ainsi, \[
		\forall V,\quad A\cdot V = B^\top\cdot  V
	.\] De même, on conclut que $A = B^\top$. On en déduit donc que \[
	\big[\:f^\star\:\big]_\mathcal{B} = \big[\:f\:\big]_\mathcal{B}^\top
	.\]
\end{prv}

Les propriétés suivantes sont vrais :
\begin{itemize}
	\item $(f  \circ g)^\star  = g^\star \circ f^\star$, \quad $(f^\star)^\star = f$, \quad et \quad $(\alpha f + \beta g)^\star  = \alpha f^\star + \beta g^\star $\/ ;
	\item $(A\cdot B)^\top  = B^\top \cdot A^\top$, \quad $(A^\top)^\top = A$, \quad et \quad $(\alpha A + \beta B)^\top = \alpha A^\top+ \beta B^\top $.
\end{itemize}
Des deuxièmes et troisièmes points,  il en résulte que les applications $f \mapsto f^\star$, et $A \mapsto A^\top$\/ sont des applications involutives.



	}
	\def\addmacros#1{#1}
}

{
	\chap[4]{Calculabilité, Décidabilité, Complexité}
	\minitoc
	\renewcommand{\cwd}{../cours/chap04/}
	\addmacros{
		\section{Motivation}

\lettrine On place au centre de la classe 40 bonbons. On en distribue un chacun. Si, par exemple, chacun choisit un bonbon et, au \textit{top} départ, prennent celui choisi.
Il est probable que plusieurs choisissent le même. Comme gérer lorsque plusieurs essaient d'accéder à la mémoire ?

Deuxièmement, sur l'ordinateur, plusieurs applications tournent en même temps. Pour le moment, on considérait qu'un seul programme était exécuté, mais, le \textsc{pc} ne s'arrête pas pendant l'exécution du programme.

On s'intéresse à la notion de \guillemotleft~processus~\guillemotright\ qui représente une tâche à réaliser.
On ne peut pas assigner un processus à une unité de calcul, mais on peut \guillemotleft~allumer~\guillemotright\ et \guillemotleft~éteindre~\guillemotright\ un processus.
Le programme allumant et éteignant les processus est \guillemotleft~l'ordonnanceur.~\guillemotright\@ Il doit aussi s'occuper de la mémoire du processus (chaque processus à sa mémoire séparée).

On s'intéresse, dans ce chapitre, à des programmes qui \guillemotleft~partent du même~\guillemotright\ : un programme peut créer un \guillemotleft~fil d'exécution~\guillemotright\ (en anglais, \textit{thread}). Le programme peut gérer les fils d'exécution qu'il a créé, et éventuellement les arrêter.
Les fils d'exécutions partagent la mémoire du programme qui les a créé.

En C, une tâche est représenté par une fonction de type \lstinline[language=c]!void* tache(void* arg)!. Le type \lstinline[language=c]!void*!\ est l'équivalent du type \lstinline[language=caml]!'a! : on peut le \textit{cast} à un autre type (comme \lstinline[language=c]-char*-).

\begin{lstlisting}[language=c,caption=Création de \textit{thread}s en C]
void* tache(void* arg) {
	printf("%s\n", (char*) arg);
	return NULL;
}

int main() {
	pthread_t p1, p2;

	printf("main: begin\n");

	pthread_create(&p1, NULL, tache, "A");
	pthread_create(&p2, NULL, tache, "B");

	pthread_join(p1, NULL);
	pthread_join(p2, NULL);

	printf("main: end\n");

	return 0;
}
\end{lstlisting}

\begin{lstlisting}[language=c,caption=Mémoire dans les \textit{thread}s en C]
int max = 10;
volatile int counter = 0;

void* tache(void* arg) {
	char* letter = arg;
	int i;

	printf("%s begin [addr of i: %p] \n", letter, &i);

	for(i = 0; i < max; i++) {
		counter = counter + 1;
	}

	printf("%s : done\n", letter);
	return NULL;
}

int main() {
	pthread_t p1, p2;

	printf("main: begin\n");

	pthread_create(&p1, NULL, tache, "A");
	pthread_create(&p2, NULL, tache, "B");

	pthread_join(p1, NULL);
	pthread_join(p2, NULL);

	printf("main: end\n");

	return 0;
}
\end{lstlisting}

Dans les \textit{thread}s, les variables locales (comme \texttt{i}) sont séparées en mémoire. Mais, la variable \texttt{counter} est modifiée, mais elle ne correspond pas forcément à $2 \times \texttt{max}$. En effet, si \texttt{p1} et \texttt{p2} essaient d'exécuter au même moment de réaliser l'opération \lstinline[language=c]-counter = counter + 1-, ils peuvent récupérer deux valeurs identiques de \texttt{counter}, ajouter 1, puis réassigner \texttt{counter}.
Ils \guillemotleft~se marchent sur les pieds.~\guillemotright\ 

Parmi les opérations, on distingue certaines dénommées \guillemotleft~atomiques~\guillemotright\ qui ne peuvent pas être séparées. L'opération \lstinline[language=c]-i++- n'est pas atomique, mais la lecture et l'écriture mémoire le sont.

\begin{defn}
	On dit d'une variable qu'elle est \textit{atomique} lorsque l'ordonnanceur ne l'interrompt pas.
\end{defn}

\begin{exm}
	L'opération \lstinline[language=c]-counter = counter + 1- exécutée en série peut être représentée comme ci-dessous. Avec \texttt{counter} valant 40, cette exécution donne 42.
	\begin{table}[H]
		\centering
		\begin{tabular}{l|l}
			Exécution du fil A & Exécution du fil B\\ \hline
			(1)~$\mathrm{reg}_1 \gets \texttt{counter}$ & (4)~$\mathrm{reg}_2 \gets \texttt{counter}$ \\
			(2)~$\mathrm{reg}_1{++}$ & (5)~$\mathrm{reg}_2{++}$ \\
			(3)~$\texttt{counter} \gets \mathrm{reg}_1$ & (6)~$\texttt{counter} \gets \mathrm{reg}_2$
		\end{tabular}
	\end{table}
	\noindent Mais, avec l'exécution en simultanée, la valeur de \texttt{counter} sera 41.
	\begin{table}[H]
		\centering
		\begin{tabular}{l|l}
			Exécution du fil A & Exécution du fil B\\ \hline
			(1)~$\mathrm{reg}_1 \gets \texttt{counter}$ & (2)~$\mathrm{reg}_2 \gets \texttt{counter}$ \\
			(3)~$\mathrm{reg}_1{++}$ & (5)~$\mathrm{reg}_2{++}$ \\
			(4)~$\texttt{counter} \gets \mathrm{reg}_1$ & (6)~$\texttt{counter} \gets \mathrm{reg}_2$
		\end{tabular}
	\end{table}
	\noindent Il y a \textit{entrelacement} des deux fils d'exécution.
\end{exm}

\begin{rmk}[Problèmes de la programmation concurrentielle]
	\begin{itemize}
		\item Problème d'accès en mémoire,
		\item Problème du rendez-vous,\footnote{Lorsque deux programmes terminent, ils doivent s'attendre pour donner leurs valeurs.}
		\item Problème du producteur-consommateur,\footnote{Certains programmes doivent ralentir ou accélérer.}
		\item Problème de l'entreblocage,\footnote{\textit{c.f.} exemple ci-après.}
		\item Problème famine, du dîner des philosophes.\footnote{Les philosophes mangent autour d'une table, et mangent du riz avec des baguettes. Ils décident de n'acheter qu'une seule baguette par personne. Un philosophe peut, ou penser, ou manger. Mais, pour manger, ils ont besoin de deux baguettes. S'ils ne mangent pas, ils meurent.}
	\end{itemize}
\end{rmk}

\begin{exm}[Problème de l'entreblocage]~

	\begin{table}[H]
		\centering
		\begin{tabular}{l|l|l}
			Fil A & Fil B & Fil C\\ \hline
			RDV(C) & RDV(A) & RDV(B)\\
			RDV(B) & RDV(C) & RDV(A)\\
		\end{tabular}
		\caption{Problème de l'entreblocage}
	\end{table}
\end{exm}

Comment résoudre le problème des deux incrementations ? Il suffit de \guillemotleft~mettre un verrou.~\guillemotright\ Le premier fil d'exécution \guillemotleft~s'enferme~\guillemotright\ avec l'expression \lstinline[language=c]!count++!, le second fil d'exécution attend que l'autre sorte pour pouvoir entrer et s'enfermer à son tour.


		\section{Continuité}

\begin{exm}
	Dans l'exercice 2, chaque fonction $f_n : t \mapsto t^n$\/ est continue sur $[0,1]$\/ mais la limite $f$\/ n'est pas continue sur $[0,1]$\/ (car elle n'est pas continue en $1$).
\end{exm}

\begin{thm}
	Soit $a$\/ un réel dans un intervalle $T$\/ de $\R$. Si une suite de fonctions $(f_n)_{n\in\N}$\/ continues en $a$\/ converge uniformément sur $T$\/ vers une fonction $f$, alors $f$\/ est aussi continue en $a$.
\end{thm}

\begin{prv}
	On suppose les fonctions $f_n$\/ continues en $a$\/ ($f_n(x) \longrightarrow f_n(a)$) et que la suite de fonctions $(f_n)_{n\in\N}$\/ converge uniformément vers $f$\/ ($\sup\:|f_n -f| \longrightarrow 0$). On veut montrer que $f$\/ est continue en $a$\/ : $f(x) \tendsto{x \to a} f(a)$, i.e.\ \[
		\forall \varepsilon > 0,\:\exists \delta > 0,\: \forall x \in T,\quad|x-a| \le \delta \implies |f(x) - f(a)| \le \varepsilon
	.\]
	Soit $\varepsilon > 0$. On calcule \[
		\big|f(x) - f(a)\big| \le \big|f(x) - f_n(x)\big| + \big|f_n(x) - f_n(a)\big| + \big|f_n(a) - f(a)\big|
	\] par inégalité triangulaire. Or, par hypothèse, il existe un rang $N \in \N$\/ (qui ne dépend pas de $x$\/ ou de $a$), tel que, $\forall n \ge N$, $\big|f(x) - f_n(x)\big| \le \frac{1}{3} \varepsilon$, et $\big|f_n(a) - f(a)\big| \le \frac{1}{3} \varepsilon$.
	De plus, par hypothèse, il existe $\delta >0$\/ tel que si $|x - a| \le \delta$, alors $|f_n(x) - f_n(a)| \le \frac{1}{3}\varepsilon$.\footnote{C'est là où l'hypothèse de la convergence uniforme est utilisée : on a besoin que le $N$\/ ne dépende pas de $x$\/ car on le fait varier.}
	On en déduit que $\big|f(x) - f(a)\big| \le \varepsilon$.
\end{prv}

\begin{crlr}
	Soit $T$\/ un intervalle de $\R$. Si une suite de fonctions $(f_n)_{n\in\N}$\/ continues sur $T$\/ converge uniformément sur $T$\/ vers une fonction continue sur $T$.
\end{crlr}

\begin{met}[Stratégie de la barrière]
	\begin{enumerate}
		\item La continuité (la dérivabilité aussi) est une propriété {\it locale}. Pour montrer qu'une fonction est continue sur un intervalle $T$, il suffit donc de montrer qu'elle est continue sur tout segment inclus dans $T$.
		\item Mais, la convergence uniforme est une propriété {\it globale}. La convergence sur tout segment inclus dans un intervalle n'implique pas la convergence uniforme sur l'intervalle (voir l'exercice 2).
		\item On n'écrit pas \[
				\substack{\ds\text{convergence uniforme}\\\ds\text{avec barrière}} \mathop{\red\implies} \substack{\ds\text{convergence uniforme}\\\ds\text{sans barrière}} \implies \substack{\ds\text{continuité}\\\ds\text{sans barrière}}
			\] mais plutôt \[
				\substack{\ds\text{convergence uniforme}\\\ds\text{avec barrière}} \implies \substack{\ds\text{continuité}\\\ds\text{avec barrière}} \implies \substack{\ds\text{continuité}\\\ds\text{sans barrière}}
			.\]
		\item Si, pour tous $a$\/ et $b$, $f$\/ est bornée sur $[a,b] \subset T$, mais cela n'implique pas que $f$\/ est bornée. Contre-exemple : la fonction $f : x \mapsto \frac{1}{x}$\/ est bornée sur tout intervalle $[a,b]$\/ avec $a$, $b \in \R^+_*$, \red{\sc mais} $f$\/ n'est pas bornée sur $]0,+\infty[$.
	\end{enumerate}
\end{met}

\begin{thm}[double-limite ou d'interversion des limites]
	Soit une suite de fonctions $(f_n)_{n\in\N}$\/ définies sur un intervalle $T$, et, soit $a$\/ une extrémité (éventuellement infinie)\footnote{autrement dit, $a \in \bar\R = \R \cup \{+\infty,-\infty\}$} de cet intervalle. Si la suite de fonctions $(f_n)_{n\in\N}$\/ converge \underline{uniformément} sur $T$\/ vers $f$\/ et si chaque fonction $f_n$\/ admet une limite finie $b_n$\/ en $a$, alors la suite de réels $b_n$\/ converge vers un réel $b$, et $\lim_{t\to a} f(t) = b$. Autrement dit, \[
		\lim_{t\to a} \Big(\underbrace{\lim_{n\to +\infty} f_n(t)}_{f(x)}\Big) = \lim_{n\to +\infty} \Big(\underbrace{\lim_{t\to a} f_n(t)}_{b_n}\Big)
	.\] \qed
\end{thm}

\begin{rmkn}
	Le théorème de la double-limite \guillemotleft~contient~\guillemotright\ le théorème 6 (théorème de préservation/transmission de la continuité), c'est un cas particulier. En effet, si les fonctions $f_n$\/ sont continues, alors \[
		\lim_{x \to a}f(x) = \underbrace{\lim_{n\to +\infty} f_n(a)}_{f(a)}
	.\]
\end{rmkn}


		\section{Endomorphismes adjoints}

\begin{defn}
	On dit qu'un endomorphisme $f : E \to E$\/ est \textit{autoadjoint} si \[
		\forall (\vec{u}, \vec{v}) \in E^2,\quad \big<f(\vec{u})\:\big|\:\vec{v}\big> = \big<\vec{u}\:\big|\:f(\vec{v})\big>
	.\] 
\end{defn}

Un endomorphisme autoadjoint est aussi appelé endomorphisme \textit{symétrique} (\textit{c.f.}\ proposition suivante). L'ensemble des endomorphismes autoadjoints est noté $\mathcal{S}(E)$.

\begin{prop}
	Un endomorphism est autoadjoint si, et seulement si la matrice de $F$\/ dans une base \ul{orthonormée} $\mathcal{B}$\/ est orthogonale.
	Autrement dit : \[
		f \in \mathcal{S}(E) \iff \big[\:f\:\big]_\mathcal{B} \in \mathcal{S}_n(\R)
	.\]
\end{prop}

\begin{prv}
	\begin{description}
		\item[$\implies$] Soit $\mathcal{B} = (\vec{\varepsilon}_1, \ldots, \vec{\varepsilon}_n)$\/ une base orthonormée de $E$. Ainsi, \[
				\forall i,\:\forall j,\quad \big<f(\vec{\varepsilon}_i)\:\big|\: \vec{\varepsilon}_j\big> = \big<\vec{\varepsilon}_i\:\big|\:f(\vec{\varepsilon}_j)\big>
			.\] On pose $\big[\:f\:\big]_{\mathcal{B}} = (a_{i,j})$\/ : \[
				\begin{pNiceMatrix}[last-col,last-row]
					\quad&\quad&a_{1,j}&\quad&\quad&\vec{\varepsilon}_i\\
					&&&&&\\
					\quad&\quad&a_{i,j}&\quad&\quad&\vec{\varepsilon}_i\\
					&&&&&\\
					\quad&\quad&a_{n,j}&\quad&\quad&\vec{\varepsilon}_n\\
					f(\vec{\varepsilon}_i)&&f(\vec{\varepsilon}_j)&&f(\vec{\varepsilon}_n)\\
				\end{pNiceMatrix}
			.\] Ainsi, $f(\vec{\varepsilon}_j) = a_{1,j} \vec{\varepsilon}_1 + \cdots + a_{i,j} \vec{\varepsilon}_i + \cdots + a_{n,j} \vec{\varepsilon}_n$. D'où, $\left<\vec{\varepsilon}_i  \mid f(\vec{\varepsilon}_j) \right> = a_{i,j}$\/ car la base $\mathcal{B}$\/ est orthonormée.
			De même avec l'autre produit scalaire, $\left< f(\vec{\varepsilon}_i)  \mid \vec{\varepsilon}_j \right>$, d'où $a_{i,j} = a_{j,i}$\/ par symétrie du produit scalaire. On en déduit que $\big[\:f\:\big]_\mathcal{B} \in \mathcal{S}_n(\R)$.
		\item[$\impliedby$]
			Si $\big[\: f\:\big]_\mathcal{B} \in \mathcal{S}_n(\R)$, alors $\left<f(\vec{\varepsilon}_i)  \mid \vec{\varepsilon}_j \right> = \left<\vec{\varepsilon}_i  \mid f(\vec{\varepsilon}_j)\right>$.
			Or, on pose $\vec{u} = x_1 \vec{\varepsilon}_1+ \cdots + x_n \vec{\varepsilon}_n$, et $\vec{v} = y_1 \vec{\varepsilon}_1 + \cdots + y_n \vec{\varepsilon}_n$.
			\begin{align*}
				\left<f(\vec{u})  \mid \vec{v} \right> &= \left<x_1 f(\vec{\varepsilon}_1) + \cdots + x_n f(\vec{\varepsilon}_n)  \mid y_1 f(\vec{\varepsilon}_1) + \cdots + y_n f(\vec{\varepsilon}_n) \right> \\
				&= \Big<\sum_{i=1}^n x_i f(\vec{\varepsilon}_i)\:\Big|\: \sum_{j=1}^n y_j \vec{\varepsilon}_j\Big> \\
				&= \sum_{i,j \in \llbracket 1,n \rrbracket}  x_i y_j \left<f(\vec{\varepsilon}_i) \mid \vec{\varepsilon}_j \right>\\
			\end{align*}
			De même en inversant $\vec{u}$\/ et $\vec{v}$.
			On en déduit donc $\left<f(\vec{u} \mid \vec{v} \right> = \left<\vec{u}  \mid f(\vec{v}) \right>$.
	\end{description}
\end{prv}

\begin{exo}
	\begin{enumerate}
		\item Si $f$\/ est autoadjoint, montrons que $\Ker f \perp \Im f$, et $\Ker f \oplus \Im f$.
			On suppose $\forall \vec{u}$, $\forall \vec{v}$, $\left<f(\vec{u}) \mid \vec{v} \right> = \left<\vec{u}  \mid f(\vec{v}) \right>$.
			Soit $\vec{u} \in \Ker f$, et soit $\vec{v} \in \Im f$.
			On sait que $f(\vec{u}) = \vec{0}$, et qu'il existe $\vec{x} \in E$\/ tel que $\vec{v} = f(\vec{x})$.
			Ainsi, \[
				\left<\vec{u}  \mid \vec{v} \right>
				= \left<\vec{u}  \mid f(\vec{x}) \right>
				= \left<f(\vec{u})  \mid \vec{x} \right>
				= 0
			.\] 
			D'où $\vec{u} \perp \vec{v}$. Ainsi, $\Ker f \perp \Im f$.


			De plus, $E$\/ est de dimension finie, d'où, d'après le théorème du rang, \[
				\dim \Ker f + \dim \Im f = \dim E
			.\] Aussi, $\Ker f \oplus (\Ker f)^\perp = E$, donc $\dim(\Ker f) + \dim(\Ker f)^\perp = \dim E$.
			On en déduit donc que $\dim(\Im f)= \dim(\Ker f)^\perp$.
			Or, $\Im f \subset (\Ker f)^\perp$\/ car $\Im f \perp \Ker f$.
			Ainsi $\Im f = (\Ker f)^\perp$, on en déduit que \[
				\Im f \oplus \Ker f = E
			.\]
			\begin{description}
				\item[$\impliedby$] 
					Soit $p$\/ la projection sur $F$\/ parallèlement à $G$.
					Supposons l'endomorphisme $P$\/ autoadjoint.
					D'après la question 1., le $\Ker p \perp \Im p$.
					Ainsi, $F = \Im p$\/ et $G = \Ker p$.
					D'où, $F \perp G$, $p$\/ est donc une projection orthogonale.
				\item[$\implies$]
					Réciproquement, supposons $p$\/ une projection orthogonale.
					Soit $\mathcal{B} = (\vec{\varepsilon}_1, \ldots, \vec{\varepsilon}_q)$\/ une base orthonormée de $F$.
					Ainsi, pour tout $\vec{x} \in E$, \[
						p(\vec{x}) = \sum_{i = 1}^q \left<\vec{x}  \mid \vec{\varepsilon}_i \right>\,\vec{\varepsilon}_i
					.\] 
					On veut montrer que l'endomorphisme $p$\/ est autoadjoint.
					Soient $\vec{u}$\/ et $\vec{v}$\/ deux vecteurs de $E$.
					\begin{align*}
						\left<p(\vec{u})  \mid \vec{v} \right>
						= \Big<\sum_{i=1}^q \left< \vec{u}\mid \vec{\varepsilon}_i \right>\vec{\varepsilon}_i\:\Big|\; \vec{v}\;\Big>
						&= \sum_{i=1}^q \left<u  \mid \vec{\varepsilon}_{i} \right>\: \left< \vec{\varepsilon}_i  \mid v\right>\\
						&= \sum_{i=1}^q \left<v  \mid \vec{\varepsilon}_i \right>\:\left<\vec{\varepsilon}_i   \mid u\right> \\
						&= \left< \vec{u}  \mid p(\vec{v}) \right> \\
					\end{align*}

					Autre méthode, pour tous vecteurs $\vec{u}$\/ et $\vec{v}$\/ de $E$,
					\begin{align*}
						\left<p(\vec{u})  \mid \vec{v} \right>
						&= \left<p(\vec{u})  \mid p(\vec{v}) + \vec{v} - p(\vec{v}) \right> \\
						&= \left< p(\vec{u})  \mid p(\vec{v}) \right> + \left<p(\vec{u})  \mid  \vec{v} - p(\vec{v}) \right> \\
						&= \left<p(\vec{u}) \mid p(\vec{v}) \right> + \left<u - p(\vec{u})  \mid p(\vec{v}) \right> \\
						&= \left<\vec{u}  \mid p(\vec{v}) \right> \\
					\end{align*}
					car $p$\/ est orthogonale.
			\end{description}
	\end{enumerate}
\end{exo}


\begin{prop-defn}
	Si $f$\/ est un endomorphisme d'un espace euclidien $E$, alors il existe un unique endomorphisme de $E$, noté $f^\star$\/ et appelé l'\textit{adjoint} de $f$, tel que \[
		\forall (\vec{u},\vec{v}) \in E^2,\quad\quad \left<f^\star(\vec{u})  \mid \vec{v} \right> =  \left<\vec{u}  \mid f(\vec{v}) \right>
	.\] 
	Si $A$\/ est la matrice $f$\/ dans une base orthonormée $\mathcal{B}$\/ de $E$, alors $A^\top$\/ est la matrice de $f^\star$\/ dans~$\mathcal{B}$\/ : \[
		\big[\:f\:\big]_\mathcal{B} = \big[\:f\:\big]_\mathcal{B}^\top
	.\]
\end{prop-defn}

\begin{prv}
	Soit $\vec{u} \in E$. L'application \begin{align*}
		\varphi: E &\longrightarrow \R \\
		\vec{v} &\longmapsto \left<\vec{u}  \mid f(\vec{v}) \right>.
	\end{align*}
	La forme $\varphi$\/ est linéaire car $\varphi(\alpha_1 \vec{v}_1 + \alpha_2 \vec{v}_2) = \left<\vec{u} \mid f(\alpha_1 \vec{v}_1 + \alpha_2 \vec{v}_2) \right> = \left<\vec{u}  \mid \alpha_1 f(\vec{v}_1) + \alpha_2 f(\vec{v}_2) \right> = \alpha_1\left<\vec{u}  \mid  f(\vec{v}) \right> + \alpha_2 \left<\vec{u}  \mid f(\vec{v}_2) \right> = \alpha_1 \varphi(\vec{v}_1) + \alpha_2 \varphi(\vec{v}_2)$.
	D'où, d'après le théorème de \textsc{Riesz}, il existe un \ul{unique} vecteur $\vec{a} \in E$\/ tel que $\varphi(\vec{v}) = \left<\vec{a}  \mid \vec{v} \right>$\/ pour tout $\vec{v} \in E$.
	Ainsi, pour tout vecteur $\vec{v} \in E$, $\left<\vec{u}  \mid f(\vec{v}) \right> = \left<\vec{a}  \mid \vec{v} \right>$.
	On note $\vec{a} = f^\star(\vec{u})$.
	Soit l'application \begin{align*}
		f^\star : E &\longrightarrow E \\
		\vec{u} &\longmapsto f^\star(\vec{u}).
	\end{align*}
	La démonstration telle que $f^\star $\/ est linéaire est dans le poly.
	L'application $f^\star$\/ vérifie : $\left< \vec{u} \mid f(\vec{v}) \right> = \left<f^\star (\vec{u})  \mid \vec{v} \right>$, pour tous vecteurs $\vec{u}$\/ et $\vec{v}$.
	\textsl{Quelle est la matrice de $f^\star$, dans une base orthonormée ?}\@
	Soit $\mathcal{B}$\/ une base orthonormée de $E$, et soient $A = \big[\:f\:\big]_\mathcal{B}$, $B = \big[\:f^\star \:\big]_\mathcal{B}$, $U = \big[\:\vec{u}\:\big]_\mathcal{B}$, et $V = \big[\:\vec{v}\:\big]_\mathcal{B}$.
	Les matrices $U$\/ et $V$\/ sont des vecteurs colonnes, et $A$\/ et $B$\/ sont des matrices carrées.
	Ainsi, \[
		U^\top \cdot A \cdot V = \left< \vec{u}  \mid f(\vec{v}) \right> 
		= \left<f^\star (\vec{u})  \mid \vec{v} \right> = (B\cdot U)^\top \cdot V,
	\] ce qui est vrai quelque soit les vecteurs colonnes $U$\/ et $V$.
	D'où, $\forall U$, $\forall V$, $U^\top \cdot \big(A \cdot V\big) = U^\top \cdot \big(B^\top \cdot V\big)$.
	Ainsi, pour tous vecteurs $U$\/ et $V$, \[
		U^\top \cdot \Big[ (AV) - (B^\top V)\Big] = 0
	.\] En particulier, si $U = (AV) - (B^\top V)$, le produit scalaire $\left<\vec{u}  \mid \vec{u} \right>$\/ est nul, donc $U = 0$.
	Ainsi, \[
		\forall V,\quad A\cdot V = B^\top\cdot  V
	.\] De même, on conclut que $A = B^\top$. On en déduit donc que \[
	\big[\:f^\star\:\big]_\mathcal{B} = \big[\:f\:\big]_\mathcal{B}^\top
	.\]
\end{prv}

Les propriétés suivantes sont vrais :
\begin{itemize}
	\item $(f  \circ g)^\star  = g^\star \circ f^\star$, \quad $(f^\star)^\star = f$, \quad et \quad $(\alpha f + \beta g)^\star  = \alpha f^\star + \beta g^\star $\/ ;
	\item $(A\cdot B)^\top  = B^\top \cdot A^\top$, \quad $(A^\top)^\top = A$, \quad et \quad $(\alpha A + \beta B)^\top = \alpha A^\top+ \beta B^\top $.
\end{itemize}
Des deuxièmes et troisièmes points,  il en résulte que les applications $f \mapsto f^\star$, et $A \mapsto A^\top$\/ sont des applications involutives.



		\begin{prop-defn}
  Soit $(\Omega, \mathcal{A}, P)$\/ un espace probabilisé, et soit $X$\/ une \textit{vard}. Si $X^2$\/ est d'espérance finie, alors $X$\/ aussi, et on appelle \textit{variance} le réel positif \[
    \mathrm{V}(X) = \mathrm{E}\Big(\big[X - \mathrm{E}(X)\big]^2\Big) = \underbrace{\mathrm{E}(X^2) - \big(\mathrm{E}(X)\big)^2}_{\mathclap{\text{Relation de \textsc{König \& Huygens}}}} \ge 0
  .\]
  L'\textit{écart-type} $\sigma(X)$\/ est la racine carrée de la variance : \[
    \sigma(X) = \sqrt{\mathrm{V}(X)}
  .\]
\end{prop-defn}

\begin{prv}
  On pose $\mu = \mathrm{E}(X)$, et on a $[X - \mu]^2 = X^2 - 2 \mu X + \mu^2$. D'où, par linéarité de l'espérance,
  \begin{align*}
    \mathrm{E}\big((X-\mu)^2\big)
    &= \mathrm{E}(X^2 - 2\mu X + \mu^2) \\
    &= \mathrm{E}(X^2) - 2\mu \mathrm{E}(X) + \mu^2 \\
    &= \mathrm{E}(X^2) - 2\mu^2 + \mu^2 \\
    &= \mathrm{E}(X^2) - \big(\mathrm{E}(X)\big)^2. \\
  \end{align*}
  De plus, d'après le lemme précédent, si $X^2$\/ est d'espérance finie, alors $X$\/ est d'espérance finie.
\end{prv}

\begin{rmk}
  \begin{enumerate}
    \item La variance mesure la \textit{dispersion}, ou l'\textit{étalement} des valeurs $a_i$\/ autour de l'espérance $\mathrm{E}(X)$. En particulier, s'il existe $a \in \R$\/ tel que $P(X = a) = 1$, alors $\mathrm{E}(X) = a$\/ et $\mathrm{V}(X) = 0$. (C'est même une équivalence.)
    \item Si la variable $X$\/ a une unité ($\mathrm{km}/\mathrm{s}$, $\mathrm{V}/\mathrm{m}$, etc.), alors l'écart type a la même unité (d'où l'intérêt de calculer la racine carrée de la variance).
    \item Soient $\alpha$\/ et $\beta$\/ deux réels. Si  $X^2$\/ est d'espérance finie, alors \[
      \mathrm{V}(\alpha X + \beta) = \alpha^2\cdot  \mathrm{V}(X)
    .\]
    (Une translation ne change pas la dispersion des valeurs, et multiplier par un réel multiplie l'espérance, mais aussi la dispersion, d'où le carré.)
  \end{enumerate}
\end{rmk}

\begin{exo}
  \textsl{Montrer que
  \begin{enumerate}
    \item si $X \sim \mathcal{B}(n, p)$, alors $X^2$\/ est d'espérance finie et $\mathrm{V}(X) = n\cdot p\cdot q$.
    \item si $T \sim \mathcal{G}(p)$, alors $T^2$\/ est d'espérance finie et $\mathrm{V}(T) = \frac{q}{p^2}$.
    \item si $X \sim \mathcal{P}(\lambda)$, alors $X^2$\/ est d'espérance finie et $\mathrm{V}(X) = \lambda$.
  \end{enumerate}
  }

  \begin{enumerate}
    \item
      Si $X \sim \mathcal{B}(n,p)$, alors $X(\Omega) = \llbracket 0,n \rrbracket$\/ et, pour $k \in X(\Omega)$, $P(X = k) = {n\choose k}\,p^k\,q^{n-k}$.
      On a déjà montré que $\mathrm{E}(X) = n\cdot p$.
      On va montrer que $\mathrm{V}(X) = n\,p\,q$.
      La variable aléatoire $X^2$\/ est d'espérance finie car $X(\Omega)$\/ est fini.
      Et,
      \begin{align*}
        \mathrm{E}(X^2) &= \sum_{k=0}^n k^2\: P(X = k)\\
        &= \sum_{k=0}^n k^2 {n\choose p} p^k q^{n-k} \\
        &= \ldots \\
      \end{align*}
      En effet, d'après la ``petite formule,'' on a \[
        \forall k \ge 1,\quad k{n\choose k} = n{n-1 \choose k-1}
      \] d'où, $(k-1) {n-1\choose k-1} = (n-1) \choose {n-2 \choose k-2}$. Ainsi, \[
        \forall k \ge 2,\quad k(k-1){n\choose k} = n(n+1) {n-2\choose k-2}
      .\] 
    \item Si $T \sim \mathcal{G}(p)$, alors $T(\Omega) = \N^*$\/ et $\forall k \in T(\Omega)$, $P(T = k) = p \times q^{k-1}$.
      On a déjà prouvé que $\mathrm{E}(T) = \frac{1}{p}$.
      On veut montrer que $\mathrm{V}(T) = \frac{q}{p^2}$. Montrons que la variable $T^2$\/ possède une espérance : la série $\sum k^2\: P(T = k)$\/ converge absolument car $k^2 \:P(T = k) = k^2 \cdot p \cdot q^{k-1}$.
      Or, pour $k \ge 2$, $\frac{\mathrm{d}^2}{\mathrm{d}x^2} x^k = k(k-1)\,x^{k-2}$. Et, on peut dériver terme à terme une série entière sans changer son rayon de convergence, et la série $\sum x^k$\/ a pour rayon de convergence 1. D'où, $\sum k(k-1)\, x^{k-2}$\/ a pour rayon de convergence 1. Or, $q \in {]0,1[} \subset {]-1,1[}$\/ donc la série $\sum k(k-1)q^{k-2}$\/ converge. De plus, $\sum k (k-1)\, q^{k-2} = \sum k^2\,q^{k-2} - \sum k\,q^{k-2}$.
      D'où,  $\sum k^2 q^{k-2} = \sum k(k-1)\, q^{k-2} + \sum k\,q^{k-2}$, qui converge. Par suite,
      \begin{align*}
        \sum_{k=1}^\infty k^2\, P(T = k) &= \sum_{k=1}^\infty k^2\,p\,q^{k-1}\\
        &= p + pq \sum_{k=2}^\infty k^2 q^{k-2} \\
        &= p + pq \sum_{k=2}^\infty k(k-1)\,q^{k-2} + p\sum_{k=2}^\infty k\,q^{k-1} \\
        &= p + pq\, \frac{2}{(1-q)^3} + p\left(\frac{1}{(1-q)^2} - 1 \right) \rlap{\quad\quad \text{\textit{c.f.} en effet après}}\\
        &= p + pq\, \frac{2}{p^3} + p\left( \frac{1}{p^2} - 1 \right) \\
        &= \frac{2q}{p^2} + \frac{1}{p} \\
        &= \frac{2q + p}{p^2} \\
        &= \frac{2q + (1-q)}{p^2} \\
        &= \frac{q+1}{p^2}. \\
      \end{align*}
      En effet, $\forall x \in {]-1,1[}$, $\sum_{k=0}^\infty x^k = \frac{1}{1-x}$. D'où, pour $x \in {]-1,1[}$, \[
        \sum_{k=1}^\infty k\,x^{k-1} = \frac{1}{(1-x)^2}
        \quad \text{ et }\quad
        \sum_{k=2}^\infty k(k-1)\,x^{k-2} = \frac{2}{(1-x)^3}
      .\]
      Ainsi, $\mathrm{E}(T^2) = \frac{q^{+1}}{p^2}$. D'où
      \begin{align*}
        \mathrm{V}(T) &= \mathrm{E}(T^2) - \big(\mathrm{E}(T)\big)^2 \\
        &= \frac{q+1}{p^2} - \left( \frac{1}{p} \right)^2 \\
        &= \frac{q}{p^2} \\
      \end{align*}
    \item À tenter
  \end{enumerate}
\end{exo}


\section{Les inégalités de \textsc{Markov} et de \textsc{Bienaymé}--\textsc{Tchebychev}, inégalités de concentration}

\begin{lem}[Markov]
  Soit $(\Omega, \mathcal{A}, P)$\/ un espace probabilisé, et soit $X$\/ une variable aléatoire \underline{positive}.
  Si $X$\/ est d'espérance finie, alors \[
    \forall a > 0, \quad P(X \ge a) \le \frac{\mathrm{E}(X)}{a}
  .\]
\end{lem}

\begin{prv}
  On suppose $X$\/ d'espérance finie. Ainsi, on a \[
    \mathrm{E}(X) = \sum_{x \in X(\Omega)} x\:P(X = x)
  .\]
  Soit $I$\/ l'ensemble $I = \{x \in X(\Omega) \mid x \ge a\}$.
  Alors, \[
    \mathrm{E}(X) = \underbrace{\sum_{x \in I} x\:P(X = x)}_{\text{ ici } x \ge a} + \underbrace{\sum_{x \in X(\Omega) \setminus I} x\:P(X = x)}_{\ge 0 \text{ par hypothèse}}
  .\]
  D'où, \[
    \mathrm{E}(X) \ge \sum_{x \in I} x\:P(X = x) \ge \sum_{x \in I} a\: P(X = x) = a \sum_{x \in I} P(X = x) \ge a\:P(x \ge a)
  .\]
\end{prv}

\begin{prop}[\textsc{Bienaymé--Tchebychev}]
  Soit $(\Omega, \mathcal{A}, P)$\/ un espace probabilisé, et soit $X$\/ une \textit{vard}. Si $X^2$\/ est d'espérance finie, alors \[
    \forall a > 0, \quad\quad P\Big(\big|X - \mathrm{E}(X)\big| \ge a\Big) \le \frac{\mathrm{V}(X)}{a^2}
  .\]
\end{prop}

\begin{prv}
  On pose $\mu = \mathrm{E}(X)$.
  L'événement $\big(|X - \mu| \ge a\big) = \big((X - \mu)^2 \ge a^2\big)$, d'où, les probabilités \[
    P\big(|X - \mu| \ge a\big) = P\big(\underbrace{(X - \mu)^2}_{\ge 0} \ge \underbrace{a^2}°{\ge 0}\big).
  \] On valide donc \textit{une} des hypothèses de l'inégalité de \textsc{Markov}.
  De plus, l'autre hypothèse est vérifiée : $X^2$\/ est d'espérance finie, donc $(X - \mu)^2$\/ aussi. On en déduit, d'après le lemme de \textsc{Markov}, que \[
    P\big((X-\mu)^2 \ge a^2\big) \le \frac{\mathrm{E}\big((X-\mu)^2\big)}{a^2} = \frac{\mathrm{V}(X)}{a^2}
  .\]
\end{prv}

\section{Série génératrice}

\begin{defn}
  Soit $X$\/ une \textit{vad} telle que $X(\Omega) \subset \N$. La \textit{série génératrice} de $X$\/ est la série entière $\sum a_n x^n$ de coefficients $a_n = P(X = n)$.
\end{defn}


La série $\sum a_n$\/ converge car sa somme vaut $\sum_{n=0}^\infty a_n = 1$. D'où, 
\begin{itemize}
  \item le rayon de convergence $R$\/ de la série est supérieur ou égal à 1.
  \item la série génératrice converge normalement sur $[-1,1]$, car la série $\sum |a_n|$\/ converge, or, $\forall x \in [-1,1]$, $|p_nt^n| \le |p_n|$, d'où la convergence normale.
    D'où la \textit{fonction génératrice} \[
      \mathrm{G}_X \colon t \longmapsto \sum_{n=0}^\infty p_n t^n
    \] est définie et même continue sur $[-1,1]$, car la convergence est uniforme.
  \item la fonction génératrice $\mathrm{G}_X$\/ est de classe $\mathcal{C}^\infty$\/ sur $]-1,1[$\/ et \[
      \forall k \in \N,\quad P(X = k) = a_n \frac{{\mathrm{G}_X}^{(k)}(0)}{k!}
    .\] La fonction génératrice de $X$\/ permet donc de retrouver la loi de probabilité de $X$.
\end{itemize}


		\begin{exo}
	Soient $\lambda_1, \ldots, \lambda_r \in \R$\/ distincts deux à deux.
	Montrons que, si $\forall x \in \R$, $\alpha_1 \mathrm{e}^{\lambda_1 x} + \cdots + \alpha_r \mathrm{e}^{\lambda_r x} = 0$, alors $\alpha_1 = \cdots = \alpha_r$.
	On peut procéder de différentes manières : le déterminant de {\sc Vandermonde}, par analyse-sythèse, ou, en utilisant \[
		\frac{\mathrm{d}}{\mathrm{d}x}\left( \mathrm{e}^{\lambda_k x} \right) = \lambda_k \mathrm{e}^{\lambda_k x},\quad\text{d'où}\quad\varphi(f_k) = \lambda_k f_k, \text{ avec } f_k : x \mapsto \mathrm{e}^{\lambda_k x}\quad\text{et}\quad\varphi:f\mapsto f'
	.\]
	On doit vérifier que les $f_k$\/ sont des vecteurs et l'application $\varphi$\/ soit un endomorphisme. On se place donc dans l'espace vectoriel $\mathscr{C}^{\infty}$. (On ne peut pas se placer dans l'espace $\mathscr{C}^k$, car sinon l'application $\varphi$\/ est de l'espace $\mathscr{C}^k$\/ à $\mathscr{C}^{k-1}$, ce n'est donc pas un endomorphisme ; ce n'est pas le cas pour l'espace $\mathscr{C}^\infty$.)
	Or, les $\lambda_k$\/ sont distincts deux à deux d'où les vecteurs propres $f_k$\/ sont linéairement indépendants. Et donc si $\alpha_1 f_1 + \alpha_2 f_2 + \cdots + \alpha_r f_r = 0$\/ alors $\alpha_1 = \cdots = \alpha_r=0$.
	Mais, comme $\forall x \in \R$, $\alpha_1 f_1(x) + \alpha_2 f_2(x) + \cdots + \alpha_r f_r(x) = 0$, on en déduit que \[
		\boxed{\alpha_1 = \cdots = \alpha_r = 0.}
	\]
\end{exo}

\section{Critères de diagonalisabilité}

\begin{prop}[une condition \underline{suffisante} pour qu'une matrice soit diagonalisable]
	Soit $A$\/ une matrice carrée de taille $n \ge 2$. {\color{red}Si} $A$\/ possède $n$\/ valeurs propres distinctes deux à deux, {\color{red}alors} $A$\/ est diagonalisable.
\end{prop}

\begin{rmkn}
	La réciproque est fausse : par exemple, pour $n > 1$, $7 I_n$\/ est diagonalisable car elle est diagonale. Mais, elle ne possède pas $n$\/ valeurs propres distinctes deux à deux.
\end{rmkn}

\begin{prv}
	On suppose que la matrice $A \in \mathscr{M}_{n,n}(\mathds{K})$\/ possède $n$\/ valeurs propres distinctes deux à deux (i.e.~$\Card \Sp(A) = n$). D'où, d'après la proposition 16, les $n$\/ vecteurs propres associés $\varepsilon_1,\ldots,\varepsilon_n$\/ sont libres. D'où $(\varepsilon_1, \ldots, \varepsilon_n)$\/ est une base formée de vecteurs propres. Donc, d'après la définition 5, la matrice $A$\/ est diagonalisable.
\end{prv}

\begin{thm}[conditions \underline{nécessaires et suffisantes} pour qu'une matrice soit diagonalisable]
	Soient $E$\/ un espace vectoriel de dimension finie et $u : E \to E$\/ un endomorphisme.
	Alors,
	\begin{align*}
		(1)\quad u \text{ diagonalisable } \iff& E = \bigoplus_{\lambda \in \Sp(u)} \Ker(\lambda\id_E - u) \quad(2)\\
		\iff& \dim E = \sum_{\lambda \in \Sp(u)} \dim(\mathrm{SEP}(\lambda))\quad(3)\\
		\iff& \chi_u \text{ scindé et } \forall \lambda \in \Sp(u),\:\dim(\mathrm{SEP}(\lambda)) = m_\lambda\quad(4)
	\end{align*}
	où $m_\lambda$\/ est la multiplicité de la racine $\lambda$\/ du polynôme $\chi_u$.
\end{thm}

\begin{prv}
	\begin{itemize}
		\item[``$(1)\implies(2)$''] On suppose $u$\/ diagonalisable. Il existe donc une base $(\varepsilon_1$, \ldots, $\varepsilon_n)$\/ de $E$\/ formée de vecteurs propres de $u$. On les regroupes par leurs valeurs propres : $(\varepsilon_i, \ldots, \varepsilon_{i+j})$\/ forme une base de $\mathrm{SEP(\lambda_k)}$. D'où la base $(\varepsilon_1, \ldots, \varepsilon_n)$\/ de l'espace vectoriel $E$\/ est une concaténation des bases des sous-espaces propres de $u$. D'où \[
				E = \bigoplus_{\lambda \in \Sp(u)} \mathrm{SEP}(\lambda)
			.\]
		\item[``$(2)\implies(1)$''] On suppose que $E = \mathrm{SEP}(\lambda_1) \oplus \mathrm{SEP}(\lambda_2) \oplus \cdots \oplus \mathrm{SEP}(\lambda_r)$.
			Soient $(\varepsilon_1, \ldots, \varepsilon_{d_1})$\/ une base de $\mathrm{SEP}(\lambda_1)$, $(\varepsilon_{d_1 + 1}, \ldots, \varepsilon_{d_1 + d_2})$\/ une base de $\mathrm{SEP}(\lambda_2)$, \ldots, $(\varepsilon_{d_1+\cdots + d_{r-1}+1}$, \ldots, $\varepsilon_{d_1+ \cdots + d_r})$\/ une base de $\mathrm{SEP}(\lambda_r)$.
			En concaténant ces base, on obtient une base de $E$, d'après l'hypothèse. Dans cette base, tous les vecteurs sont propres donc $u$\/ est diagonalisable.
		\item[``$(2)\implies(3)$''] On suppose $E = \mathrm{SEP}(\lambda_1) \oplus \mathrm{SEP}(\lambda_2) \oplus \cdots \oplus \mathrm{SEP}(\lambda_r)$. D'où  \[
					\dim E = \dim(\mathrm{SEP}(\lambda_1)) + \dim(\mathrm{SEP}(\lambda_2)) + \cdots + \dim(\mathrm{SEP}(\lambda_r))
			\] car la dimension d'une somme directe est égale à la somme des dimensions.
		\item[``$(3)\implies(1)$''] On suppose $\dim E = \dim(\mathrm{SEP}(\lambda_1)) + \dim(\mathrm{SEP}(\lambda_2)) + \cdots + \dim(\mathrm{SEP}(\lambda_r))$. Or, les sous-espaces propres sont en somme directe, d'après la proposition 16. D'où $\dim\Big(\sum_{\lambda \in \Sp(u)} \mathrm{SEP}(\lambda) \Big)= \sum_{\lambda \in \Sp(u)} \dim(\mathrm{SEP}(\lambda))$. Donc $\sum_{\lambda \in \Sp(u)} \mathrm{SEP}(\lambda) = E$.
		\item[``$(4)\implies(3)$''] On suppose (a) $\chi_u$\/ scindé et (b) $\dim(\mathrm{SEP}(\lambda)) = m_\lambda$. D'où, d'après (a): \[
				\chi_u(x) = (x - \lambda_1)^{m_{\lambda_1}}(x - \lambda_2)^{m_{\lambda_2}} \cdots (x - \lambda_r)^{m_{\lambda_r}} = x^n + \cdots
			\] d'où $m_{\lambda_1} + m_{\lambda_2} + \cdots + m_{\lambda_r} = n$, et d'où \[
				\dim(\mathrm{SEP}(\lambda_1)) + \dim(\mathrm{SEP}(\lambda_2)) + \cdots + \dim(\mathrm{SEP}(\lambda_r)) = n
			\] d'après l'hypothèse (b).
		\item[``$(1)\implies(4)$'']
			On suppose $u$\/ diagonalisable. D'où, dans une certaine base $\mathscr{B}$, la matrice $\big[u\big]_\mathscr{B}$\/ est diagonale. Quitte à changer l'ordre des éléments de $\mathscr{B} = (\varepsilon_1,\ldots,\varepsilon_r)$, on peut supposer que $\big[u\big]_\mathscr{B}$\/ est de la forme \[
				\big[u\big]_\mathscr{B} = 
				\begin{bNiceArray}{c|c|c|c}[last-col]
					\begin{array}{cccc}\lambda_1\\&\lambda_1\\&&\ddots\\&&&\lambda_1\end{array}&0&0&0&\begin{array}{l}\varepsilon_1\\\varepsilon_2\\\vdots\\\varepsilon_{d_1}\\\end{array}\\ \hline
					0&\begin{array}{cccc}\lambda_2\\&\lambda_2\\&&\ddots\\&&&\lambda_2\end{array}&0&0&\begin{array}{l}\varepsilon_{d_1+1}\\\varepsilon_{d_1+2}\\\vdots\\\varepsilon_{d_1+d_2}\\\end{array}\\ \hline
					 &&\ddots&&\vdots\\ \hline
					0&0&0&0\begin{array}{cccc}\lambda_r\\&\lambda_r\\&&\ddots\\&&&\lambda_r\end{array}&\begin{array}{l}\varepsilon_{d_1+\cdots + d_{r-1} + 1}\\\varepsilon_{d_1+\cdots + d_{r-1}+2}\\\vdots\\\varepsilon_{d_1 + \cdots + d_r}\\\end{array}\\
				\end{bNiceArray}
			.\] D'où $\forall k \in \left\llbracket 1,r \right\rrbracket$, $d_k = \dim(\mathrm{SEP}(\lambda_k))$. En outre, $\chi_u(x) = \det(x\id - u) = (x - \lambda_1)^{d_1}\cdot (x-\lambda_2)^{d_2} \cdots (r - \lambda_r)^{d_r}$.
			D'où $\forall k \in \left\llbracket 1,r \right\rrbracket$, $d_k = m_{\lambda_k}$\/ et $\chi_u$\/ est scindé.
	\end{itemize}
\end{prv}

\begin{exo}
	{\slshape On considère la matrice $E$\/ ci-dessous \[
		E = \begin{pmatrix}
			7&0&1\\
			0&3&0\\
			0&0&7
		\end{pmatrix}.
	\] La matrice $E$\/ ci-dessous est-elle diagonalisable ?}

	Soit $\lambda \in \R$. On sait que $\lambda \in \Sp(E)$\/ si et seulement si $\det(\lambda I_3 - E) = 0$. Or \[
		\det(\lambda I_3 - E) =
		\begin{vmatrix}
			\lambda - 7&0&-1\\
			0&\lambda-3&0\\
			0&0&\lambda - 7
		\end{vmatrix} = (\lambda - 7)^2\cdot  (\lambda - 3)^1
	.\] Donc $\Sp(E) = \{3,7\}$, $1 \le \dim\big(\mathrm{SEP}(3)\big) \le 1$, et $1 \le \dim\big(\mathrm{SEP}(7)\big) \le 2$.
	La matrice $E$\/ est diagonalisable si et seulement si $\dim(\mathrm{SEP}(3)) + \dim(\mathrm{SEP}(7)) = 3$, donc si et seulement si $\dim(\mathrm{SEP}(7)) = 2$. On cherche donc la dimension de ce sous-espace propre : soit $X = \left( \substack{x\\y\\z} \right) \in \mathscr{M}_{3,1}(\R)$. On sait que
	\begin{align*}
		X \in \mathrm{SEP}(7) \iff& E\cdot X = 7X\\
		\iff& \begin{pmatrix}
			7&0&1\\
			0&3&0\\
			0&0&7
		\end{pmatrix} \begin{pmatrix}
			x\\y\\z
		\end{pmatrix} = 7 \begin{pmatrix}
			x\\y\\z
		\end{pmatrix}\\
		\iff& \begin{cases}
			7x + 0y + 1z = 7x\\
			3y = 7y\\
			7z = 7z
		\end{cases}\\
		\iff& \begin{cases}
			z = 0\\
			y = 0
		\end{cases}\\
		\iff& X = \begin{pmatrix}
			x\\0\\y
		\end{pmatrix} = x \underbrace{\begin{pmatrix}
			1\\0\\0
		\end{pmatrix}}_{\varepsilon_1}
	\end{align*}
	Donc $\mathrm{SEP}(7) = \Vect(\varepsilon_1)$, d'où $\dim(\mathrm{SEP}(7)) = 1$. Donc la matrice $E$\/ n'est pas diagonalisable.
\end{exo}

\section{Trigonalisation}

Trigonaliser une matrice ne sert que si la matrice n'est pas diagonalisable.

\begin{defn}
	On dit d'une matrice carrée $A \in \mathscr{M}_{n,n}(\mathds{K})$\/ qu'elle est {\it trigonalisable}\/ s'il existe une matrice inversible $P$\/ telle que $P^{-1} \cdot A \cdot P$\/ est triangulaire : \[
		P^{-1} \cdot A \cdot P = \begin{pNiceMatrix}
			\lambda_1&\Block{2-2}*&\\
			\Block{2-2}0&\Ddots&\\
			&&\lambda_r
		\end{pNiceMatrix}
	.\]
\end{defn}

\begin{rmk}
	$\O$\/
\end{rmk}

\begin{thm}
	Une matrice carrée $A \in \mathscr{M}_{n,n}(\mathds{K})$\/ est trigonalisable si et seulement si son polynôme caractéristique $\chi_A \in \mathds{K}[X]$\/ est scindé.
\end{thm}

	}
	\def\addmacros#1{#1}
}

{
	\chap[5]{Trois exemples d'algorithmes de graphes}
	\minitoc
	\renewcommand{\cwd}{../cours/chap05/}
	\addmacros{
		\section{Motivation}

\lettrine On place au centre de la classe 40 bonbons. On en distribue un chacun. Si, par exemple, chacun choisit un bonbon et, au \textit{top} départ, prennent celui choisi.
Il est probable que plusieurs choisissent le même. Comme gérer lorsque plusieurs essaient d'accéder à la mémoire ?

Deuxièmement, sur l'ordinateur, plusieurs applications tournent en même temps. Pour le moment, on considérait qu'un seul programme était exécuté, mais, le \textsc{pc} ne s'arrête pas pendant l'exécution du programme.

On s'intéresse à la notion de \guillemotleft~processus~\guillemotright\ qui représente une tâche à réaliser.
On ne peut pas assigner un processus à une unité de calcul, mais on peut \guillemotleft~allumer~\guillemotright\ et \guillemotleft~éteindre~\guillemotright\ un processus.
Le programme allumant et éteignant les processus est \guillemotleft~l'ordonnanceur.~\guillemotright\@ Il doit aussi s'occuper de la mémoire du processus (chaque processus à sa mémoire séparée).

On s'intéresse, dans ce chapitre, à des programmes qui \guillemotleft~partent du même~\guillemotright\ : un programme peut créer un \guillemotleft~fil d'exécution~\guillemotright\ (en anglais, \textit{thread}). Le programme peut gérer les fils d'exécution qu'il a créé, et éventuellement les arrêter.
Les fils d'exécutions partagent la mémoire du programme qui les a créé.

En C, une tâche est représenté par une fonction de type \lstinline[language=c]!void* tache(void* arg)!. Le type \lstinline[language=c]!void*!\ est l'équivalent du type \lstinline[language=caml]!'a! : on peut le \textit{cast} à un autre type (comme \lstinline[language=c]-char*-).

\begin{lstlisting}[language=c,caption=Création de \textit{thread}s en C]
void* tache(void* arg) {
	printf("%s\n", (char*) arg);
	return NULL;
}

int main() {
	pthread_t p1, p2;

	printf("main: begin\n");

	pthread_create(&p1, NULL, tache, "A");
	pthread_create(&p2, NULL, tache, "B");

	pthread_join(p1, NULL);
	pthread_join(p2, NULL);

	printf("main: end\n");

	return 0;
}
\end{lstlisting}

\begin{lstlisting}[language=c,caption=Mémoire dans les \textit{thread}s en C]
int max = 10;
volatile int counter = 0;

void* tache(void* arg) {
	char* letter = arg;
	int i;

	printf("%s begin [addr of i: %p] \n", letter, &i);

	for(i = 0; i < max; i++) {
		counter = counter + 1;
	}

	printf("%s : done\n", letter);
	return NULL;
}

int main() {
	pthread_t p1, p2;

	printf("main: begin\n");

	pthread_create(&p1, NULL, tache, "A");
	pthread_create(&p2, NULL, tache, "B");

	pthread_join(p1, NULL);
	pthread_join(p2, NULL);

	printf("main: end\n");

	return 0;
}
\end{lstlisting}

Dans les \textit{thread}s, les variables locales (comme \texttt{i}) sont séparées en mémoire. Mais, la variable \texttt{counter} est modifiée, mais elle ne correspond pas forcément à $2 \times \texttt{max}$. En effet, si \texttt{p1} et \texttt{p2} essaient d'exécuter au même moment de réaliser l'opération \lstinline[language=c]-counter = counter + 1-, ils peuvent récupérer deux valeurs identiques de \texttt{counter}, ajouter 1, puis réassigner \texttt{counter}.
Ils \guillemotleft~se marchent sur les pieds.~\guillemotright\ 

Parmi les opérations, on distingue certaines dénommées \guillemotleft~atomiques~\guillemotright\ qui ne peuvent pas être séparées. L'opération \lstinline[language=c]-i++- n'est pas atomique, mais la lecture et l'écriture mémoire le sont.

\begin{defn}
	On dit d'une variable qu'elle est \textit{atomique} lorsque l'ordonnanceur ne l'interrompt pas.
\end{defn}

\begin{exm}
	L'opération \lstinline[language=c]-counter = counter + 1- exécutée en série peut être représentée comme ci-dessous. Avec \texttt{counter} valant 40, cette exécution donne 42.
	\begin{table}[H]
		\centering
		\begin{tabular}{l|l}
			Exécution du fil A & Exécution du fil B\\ \hline
			(1)~$\mathrm{reg}_1 \gets \texttt{counter}$ & (4)~$\mathrm{reg}_2 \gets \texttt{counter}$ \\
			(2)~$\mathrm{reg}_1{++}$ & (5)~$\mathrm{reg}_2{++}$ \\
			(3)~$\texttt{counter} \gets \mathrm{reg}_1$ & (6)~$\texttt{counter} \gets \mathrm{reg}_2$
		\end{tabular}
	\end{table}
	\noindent Mais, avec l'exécution en simultanée, la valeur de \texttt{counter} sera 41.
	\begin{table}[H]
		\centering
		\begin{tabular}{l|l}
			Exécution du fil A & Exécution du fil B\\ \hline
			(1)~$\mathrm{reg}_1 \gets \texttt{counter}$ & (2)~$\mathrm{reg}_2 \gets \texttt{counter}$ \\
			(3)~$\mathrm{reg}_1{++}$ & (5)~$\mathrm{reg}_2{++}$ \\
			(4)~$\texttt{counter} \gets \mathrm{reg}_1$ & (6)~$\texttt{counter} \gets \mathrm{reg}_2$
		\end{tabular}
	\end{table}
	\noindent Il y a \textit{entrelacement} des deux fils d'exécution.
\end{exm}

\begin{rmk}[Problèmes de la programmation concurrentielle]
	\begin{itemize}
		\item Problème d'accès en mémoire,
		\item Problème du rendez-vous,\footnote{Lorsque deux programmes terminent, ils doivent s'attendre pour donner leurs valeurs.}
		\item Problème du producteur-consommateur,\footnote{Certains programmes doivent ralentir ou accélérer.}
		\item Problème de l'entreblocage,\footnote{\textit{c.f.} exemple ci-après.}
		\item Problème famine, du dîner des philosophes.\footnote{Les philosophes mangent autour d'une table, et mangent du riz avec des baguettes. Ils décident de n'acheter qu'une seule baguette par personne. Un philosophe peut, ou penser, ou manger. Mais, pour manger, ils ont besoin de deux baguettes. S'ils ne mangent pas, ils meurent.}
	\end{itemize}
\end{rmk}

\begin{exm}[Problème de l'entreblocage]~

	\begin{table}[H]
		\centering
		\begin{tabular}{l|l|l}
			Fil A & Fil B & Fil C\\ \hline
			RDV(C) & RDV(A) & RDV(B)\\
			RDV(B) & RDV(C) & RDV(A)\\
		\end{tabular}
		\caption{Problème de l'entreblocage}
	\end{table}
\end{exm}

Comment résoudre le problème des deux incrementations ? Il suffit de \guillemotleft~mettre un verrou.~\guillemotright\ Le premier fil d'exécution \guillemotleft~s'enferme~\guillemotright\ avec l'expression \lstinline[language=c]!count++!, le second fil d'exécution attend que l'autre sorte pour pouvoir entrer et s'enfermer à son tour.


		\section{Continuité}

\begin{exm}
	Dans l'exercice 2, chaque fonction $f_n : t \mapsto t^n$\/ est continue sur $[0,1]$\/ mais la limite $f$\/ n'est pas continue sur $[0,1]$\/ (car elle n'est pas continue en $1$).
\end{exm}

\begin{thm}
	Soit $a$\/ un réel dans un intervalle $T$\/ de $\R$. Si une suite de fonctions $(f_n)_{n\in\N}$\/ continues en $a$\/ converge uniformément sur $T$\/ vers une fonction $f$, alors $f$\/ est aussi continue en $a$.
\end{thm}

\begin{prv}
	On suppose les fonctions $f_n$\/ continues en $a$\/ ($f_n(x) \longrightarrow f_n(a)$) et que la suite de fonctions $(f_n)_{n\in\N}$\/ converge uniformément vers $f$\/ ($\sup\:|f_n -f| \longrightarrow 0$). On veut montrer que $f$\/ est continue en $a$\/ : $f(x) \tendsto{x \to a} f(a)$, i.e.\ \[
		\forall \varepsilon > 0,\:\exists \delta > 0,\: \forall x \in T,\quad|x-a| \le \delta \implies |f(x) - f(a)| \le \varepsilon
	.\]
	Soit $\varepsilon > 0$. On calcule \[
		\big|f(x) - f(a)\big| \le \big|f(x) - f_n(x)\big| + \big|f_n(x) - f_n(a)\big| + \big|f_n(a) - f(a)\big|
	\] par inégalité triangulaire. Or, par hypothèse, il existe un rang $N \in \N$\/ (qui ne dépend pas de $x$\/ ou de $a$), tel que, $\forall n \ge N$, $\big|f(x) - f_n(x)\big| \le \frac{1}{3} \varepsilon$, et $\big|f_n(a) - f(a)\big| \le \frac{1}{3} \varepsilon$.
	De plus, par hypothèse, il existe $\delta >0$\/ tel que si $|x - a| \le \delta$, alors $|f_n(x) - f_n(a)| \le \frac{1}{3}\varepsilon$.\footnote{C'est là où l'hypothèse de la convergence uniforme est utilisée : on a besoin que le $N$\/ ne dépende pas de $x$\/ car on le fait varier.}
	On en déduit que $\big|f(x) - f(a)\big| \le \varepsilon$.
\end{prv}

\begin{crlr}
	Soit $T$\/ un intervalle de $\R$. Si une suite de fonctions $(f_n)_{n\in\N}$\/ continues sur $T$\/ converge uniformément sur $T$\/ vers une fonction continue sur $T$.
\end{crlr}

\begin{met}[Stratégie de la barrière]
	\begin{enumerate}
		\item La continuité (la dérivabilité aussi) est une propriété {\it locale}. Pour montrer qu'une fonction est continue sur un intervalle $T$, il suffit donc de montrer qu'elle est continue sur tout segment inclus dans $T$.
		\item Mais, la convergence uniforme est une propriété {\it globale}. La convergence sur tout segment inclus dans un intervalle n'implique pas la convergence uniforme sur l'intervalle (voir l'exercice 2).
		\item On n'écrit pas \[
				\substack{\ds\text{convergence uniforme}\\\ds\text{avec barrière}} \mathop{\red\implies} \substack{\ds\text{convergence uniforme}\\\ds\text{sans barrière}} \implies \substack{\ds\text{continuité}\\\ds\text{sans barrière}}
			\] mais plutôt \[
				\substack{\ds\text{convergence uniforme}\\\ds\text{avec barrière}} \implies \substack{\ds\text{continuité}\\\ds\text{avec barrière}} \implies \substack{\ds\text{continuité}\\\ds\text{sans barrière}}
			.\]
		\item Si, pour tous $a$\/ et $b$, $f$\/ est bornée sur $[a,b] \subset T$, mais cela n'implique pas que $f$\/ est bornée. Contre-exemple : la fonction $f : x \mapsto \frac{1}{x}$\/ est bornée sur tout intervalle $[a,b]$\/ avec $a$, $b \in \R^+_*$, \red{\sc mais} $f$\/ n'est pas bornée sur $]0,+\infty[$.
	\end{enumerate}
\end{met}

\begin{thm}[double-limite ou d'interversion des limites]
	Soit une suite de fonctions $(f_n)_{n\in\N}$\/ définies sur un intervalle $T$, et, soit $a$\/ une extrémité (éventuellement infinie)\footnote{autrement dit, $a \in \bar\R = \R \cup \{+\infty,-\infty\}$} de cet intervalle. Si la suite de fonctions $(f_n)_{n\in\N}$\/ converge \underline{uniformément} sur $T$\/ vers $f$\/ et si chaque fonction $f_n$\/ admet une limite finie $b_n$\/ en $a$, alors la suite de réels $b_n$\/ converge vers un réel $b$, et $\lim_{t\to a} f(t) = b$. Autrement dit, \[
		\lim_{t\to a} \Big(\underbrace{\lim_{n\to +\infty} f_n(t)}_{f(x)}\Big) = \lim_{n\to +\infty} \Big(\underbrace{\lim_{t\to a} f_n(t)}_{b_n}\Big)
	.\] \qed
\end{thm}

\begin{rmkn}
	Le théorème de la double-limite \guillemotleft~contient~\guillemotright\ le théorème 6 (théorème de préservation/transmission de la continuité), c'est un cas particulier. En effet, si les fonctions $f_n$\/ sont continues, alors \[
		\lim_{x \to a}f(x) = \underbrace{\lim_{n\to +\infty} f_n(a)}_{f(a)}
	.\]
\end{rmkn}


		\section{Endomorphismes adjoints}

\begin{defn}
	On dit qu'un endomorphisme $f : E \to E$\/ est \textit{autoadjoint} si \[
		\forall (\vec{u}, \vec{v}) \in E^2,\quad \big<f(\vec{u})\:\big|\:\vec{v}\big> = \big<\vec{u}\:\big|\:f(\vec{v})\big>
	.\] 
\end{defn}

Un endomorphisme autoadjoint est aussi appelé endomorphisme \textit{symétrique} (\textit{c.f.}\ proposition suivante). L'ensemble des endomorphismes autoadjoints est noté $\mathcal{S}(E)$.

\begin{prop}
	Un endomorphism est autoadjoint si, et seulement si la matrice de $F$\/ dans une base \ul{orthonormée} $\mathcal{B}$\/ est orthogonale.
	Autrement dit : \[
		f \in \mathcal{S}(E) \iff \big[\:f\:\big]_\mathcal{B} \in \mathcal{S}_n(\R)
	.\]
\end{prop}

\begin{prv}
	\begin{description}
		\item[$\implies$] Soit $\mathcal{B} = (\vec{\varepsilon}_1, \ldots, \vec{\varepsilon}_n)$\/ une base orthonormée de $E$. Ainsi, \[
				\forall i,\:\forall j,\quad \big<f(\vec{\varepsilon}_i)\:\big|\: \vec{\varepsilon}_j\big> = \big<\vec{\varepsilon}_i\:\big|\:f(\vec{\varepsilon}_j)\big>
			.\] On pose $\big[\:f\:\big]_{\mathcal{B}} = (a_{i,j})$\/ : \[
				\begin{pNiceMatrix}[last-col,last-row]
					\quad&\quad&a_{1,j}&\quad&\quad&\vec{\varepsilon}_i\\
					&&&&&\\
					\quad&\quad&a_{i,j}&\quad&\quad&\vec{\varepsilon}_i\\
					&&&&&\\
					\quad&\quad&a_{n,j}&\quad&\quad&\vec{\varepsilon}_n\\
					f(\vec{\varepsilon}_i)&&f(\vec{\varepsilon}_j)&&f(\vec{\varepsilon}_n)\\
				\end{pNiceMatrix}
			.\] Ainsi, $f(\vec{\varepsilon}_j) = a_{1,j} \vec{\varepsilon}_1 + \cdots + a_{i,j} \vec{\varepsilon}_i + \cdots + a_{n,j} \vec{\varepsilon}_n$. D'où, $\left<\vec{\varepsilon}_i  \mid f(\vec{\varepsilon}_j) \right> = a_{i,j}$\/ car la base $\mathcal{B}$\/ est orthonormée.
			De même avec l'autre produit scalaire, $\left< f(\vec{\varepsilon}_i)  \mid \vec{\varepsilon}_j \right>$, d'où $a_{i,j} = a_{j,i}$\/ par symétrie du produit scalaire. On en déduit que $\big[\:f\:\big]_\mathcal{B} \in \mathcal{S}_n(\R)$.
		\item[$\impliedby$]
			Si $\big[\: f\:\big]_\mathcal{B} \in \mathcal{S}_n(\R)$, alors $\left<f(\vec{\varepsilon}_i)  \mid \vec{\varepsilon}_j \right> = \left<\vec{\varepsilon}_i  \mid f(\vec{\varepsilon}_j)\right>$.
			Or, on pose $\vec{u} = x_1 \vec{\varepsilon}_1+ \cdots + x_n \vec{\varepsilon}_n$, et $\vec{v} = y_1 \vec{\varepsilon}_1 + \cdots + y_n \vec{\varepsilon}_n$.
			\begin{align*}
				\left<f(\vec{u})  \mid \vec{v} \right> &= \left<x_1 f(\vec{\varepsilon}_1) + \cdots + x_n f(\vec{\varepsilon}_n)  \mid y_1 f(\vec{\varepsilon}_1) + \cdots + y_n f(\vec{\varepsilon}_n) \right> \\
				&= \Big<\sum_{i=1}^n x_i f(\vec{\varepsilon}_i)\:\Big|\: \sum_{j=1}^n y_j \vec{\varepsilon}_j\Big> \\
				&= \sum_{i,j \in \llbracket 1,n \rrbracket}  x_i y_j \left<f(\vec{\varepsilon}_i) \mid \vec{\varepsilon}_j \right>\\
			\end{align*}
			De même en inversant $\vec{u}$\/ et $\vec{v}$.
			On en déduit donc $\left<f(\vec{u} \mid \vec{v} \right> = \left<\vec{u}  \mid f(\vec{v}) \right>$.
	\end{description}
\end{prv}

\begin{exo}
	\begin{enumerate}
		\item Si $f$\/ est autoadjoint, montrons que $\Ker f \perp \Im f$, et $\Ker f \oplus \Im f$.
			On suppose $\forall \vec{u}$, $\forall \vec{v}$, $\left<f(\vec{u}) \mid \vec{v} \right> = \left<\vec{u}  \mid f(\vec{v}) \right>$.
			Soit $\vec{u} \in \Ker f$, et soit $\vec{v} \in \Im f$.
			On sait que $f(\vec{u}) = \vec{0}$, et qu'il existe $\vec{x} \in E$\/ tel que $\vec{v} = f(\vec{x})$.
			Ainsi, \[
				\left<\vec{u}  \mid \vec{v} \right>
				= \left<\vec{u}  \mid f(\vec{x}) \right>
				= \left<f(\vec{u})  \mid \vec{x} \right>
				= 0
			.\] 
			D'où $\vec{u} \perp \vec{v}$. Ainsi, $\Ker f \perp \Im f$.


			De plus, $E$\/ est de dimension finie, d'où, d'après le théorème du rang, \[
				\dim \Ker f + \dim \Im f = \dim E
			.\] Aussi, $\Ker f \oplus (\Ker f)^\perp = E$, donc $\dim(\Ker f) + \dim(\Ker f)^\perp = \dim E$.
			On en déduit donc que $\dim(\Im f)= \dim(\Ker f)^\perp$.
			Or, $\Im f \subset (\Ker f)^\perp$\/ car $\Im f \perp \Ker f$.
			Ainsi $\Im f = (\Ker f)^\perp$, on en déduit que \[
				\Im f \oplus \Ker f = E
			.\]
			\begin{description}
				\item[$\impliedby$] 
					Soit $p$\/ la projection sur $F$\/ parallèlement à $G$.
					Supposons l'endomorphisme $P$\/ autoadjoint.
					D'après la question 1., le $\Ker p \perp \Im p$.
					Ainsi, $F = \Im p$\/ et $G = \Ker p$.
					D'où, $F \perp G$, $p$\/ est donc une projection orthogonale.
				\item[$\implies$]
					Réciproquement, supposons $p$\/ une projection orthogonale.
					Soit $\mathcal{B} = (\vec{\varepsilon}_1, \ldots, \vec{\varepsilon}_q)$\/ une base orthonormée de $F$.
					Ainsi, pour tout $\vec{x} \in E$, \[
						p(\vec{x}) = \sum_{i = 1}^q \left<\vec{x}  \mid \vec{\varepsilon}_i \right>\,\vec{\varepsilon}_i
					.\] 
					On veut montrer que l'endomorphisme $p$\/ est autoadjoint.
					Soient $\vec{u}$\/ et $\vec{v}$\/ deux vecteurs de $E$.
					\begin{align*}
						\left<p(\vec{u})  \mid \vec{v} \right>
						= \Big<\sum_{i=1}^q \left< \vec{u}\mid \vec{\varepsilon}_i \right>\vec{\varepsilon}_i\:\Big|\; \vec{v}\;\Big>
						&= \sum_{i=1}^q \left<u  \mid \vec{\varepsilon}_{i} \right>\: \left< \vec{\varepsilon}_i  \mid v\right>\\
						&= \sum_{i=1}^q \left<v  \mid \vec{\varepsilon}_i \right>\:\left<\vec{\varepsilon}_i   \mid u\right> \\
						&= \left< \vec{u}  \mid p(\vec{v}) \right> \\
					\end{align*}

					Autre méthode, pour tous vecteurs $\vec{u}$\/ et $\vec{v}$\/ de $E$,
					\begin{align*}
						\left<p(\vec{u})  \mid \vec{v} \right>
						&= \left<p(\vec{u})  \mid p(\vec{v}) + \vec{v} - p(\vec{v}) \right> \\
						&= \left< p(\vec{u})  \mid p(\vec{v}) \right> + \left<p(\vec{u})  \mid  \vec{v} - p(\vec{v}) \right> \\
						&= \left<p(\vec{u}) \mid p(\vec{v}) \right> + \left<u - p(\vec{u})  \mid p(\vec{v}) \right> \\
						&= \left<\vec{u}  \mid p(\vec{v}) \right> \\
					\end{align*}
					car $p$\/ est orthogonale.
			\end{description}
	\end{enumerate}
\end{exo}


\begin{prop-defn}
	Si $f$\/ est un endomorphisme d'un espace euclidien $E$, alors il existe un unique endomorphisme de $E$, noté $f^\star$\/ et appelé l'\textit{adjoint} de $f$, tel que \[
		\forall (\vec{u},\vec{v}) \in E^2,\quad\quad \left<f^\star(\vec{u})  \mid \vec{v} \right> =  \left<\vec{u}  \mid f(\vec{v}) \right>
	.\] 
	Si $A$\/ est la matrice $f$\/ dans une base orthonormée $\mathcal{B}$\/ de $E$, alors $A^\top$\/ est la matrice de $f^\star$\/ dans~$\mathcal{B}$\/ : \[
		\big[\:f\:\big]_\mathcal{B} = \big[\:f\:\big]_\mathcal{B}^\top
	.\]
\end{prop-defn}

\begin{prv}
	Soit $\vec{u} \in E$. L'application \begin{align*}
		\varphi: E &\longrightarrow \R \\
		\vec{v} &\longmapsto \left<\vec{u}  \mid f(\vec{v}) \right>.
	\end{align*}
	La forme $\varphi$\/ est linéaire car $\varphi(\alpha_1 \vec{v}_1 + \alpha_2 \vec{v}_2) = \left<\vec{u} \mid f(\alpha_1 \vec{v}_1 + \alpha_2 \vec{v}_2) \right> = \left<\vec{u}  \mid \alpha_1 f(\vec{v}_1) + \alpha_2 f(\vec{v}_2) \right> = \alpha_1\left<\vec{u}  \mid  f(\vec{v}) \right> + \alpha_2 \left<\vec{u}  \mid f(\vec{v}_2) \right> = \alpha_1 \varphi(\vec{v}_1) + \alpha_2 \varphi(\vec{v}_2)$.
	D'où, d'après le théorème de \textsc{Riesz}, il existe un \ul{unique} vecteur $\vec{a} \in E$\/ tel que $\varphi(\vec{v}) = \left<\vec{a}  \mid \vec{v} \right>$\/ pour tout $\vec{v} \in E$.
	Ainsi, pour tout vecteur $\vec{v} \in E$, $\left<\vec{u}  \mid f(\vec{v}) \right> = \left<\vec{a}  \mid \vec{v} \right>$.
	On note $\vec{a} = f^\star(\vec{u})$.
	Soit l'application \begin{align*}
		f^\star : E &\longrightarrow E \\
		\vec{u} &\longmapsto f^\star(\vec{u}).
	\end{align*}
	La démonstration telle que $f^\star $\/ est linéaire est dans le poly.
	L'application $f^\star$\/ vérifie : $\left< \vec{u} \mid f(\vec{v}) \right> = \left<f^\star (\vec{u})  \mid \vec{v} \right>$, pour tous vecteurs $\vec{u}$\/ et $\vec{v}$.
	\textsl{Quelle est la matrice de $f^\star$, dans une base orthonormée ?}\@
	Soit $\mathcal{B}$\/ une base orthonormée de $E$, et soient $A = \big[\:f\:\big]_\mathcal{B}$, $B = \big[\:f^\star \:\big]_\mathcal{B}$, $U = \big[\:\vec{u}\:\big]_\mathcal{B}$, et $V = \big[\:\vec{v}\:\big]_\mathcal{B}$.
	Les matrices $U$\/ et $V$\/ sont des vecteurs colonnes, et $A$\/ et $B$\/ sont des matrices carrées.
	Ainsi, \[
		U^\top \cdot A \cdot V = \left< \vec{u}  \mid f(\vec{v}) \right> 
		= \left<f^\star (\vec{u})  \mid \vec{v} \right> = (B\cdot U)^\top \cdot V,
	\] ce qui est vrai quelque soit les vecteurs colonnes $U$\/ et $V$.
	D'où, $\forall U$, $\forall V$, $U^\top \cdot \big(A \cdot V\big) = U^\top \cdot \big(B^\top \cdot V\big)$.
	Ainsi, pour tous vecteurs $U$\/ et $V$, \[
		U^\top \cdot \Big[ (AV) - (B^\top V)\Big] = 0
	.\] En particulier, si $U = (AV) - (B^\top V)$, le produit scalaire $\left<\vec{u}  \mid \vec{u} \right>$\/ est nul, donc $U = 0$.
	Ainsi, \[
		\forall V,\quad A\cdot V = B^\top\cdot  V
	.\] De même, on conclut que $A = B^\top$. On en déduit donc que \[
	\big[\:f^\star\:\big]_\mathcal{B} = \big[\:f\:\big]_\mathcal{B}^\top
	.\]
\end{prv}

Les propriétés suivantes sont vrais :
\begin{itemize}
	\item $(f  \circ g)^\star  = g^\star \circ f^\star$, \quad $(f^\star)^\star = f$, \quad et \quad $(\alpha f + \beta g)^\star  = \alpha f^\star + \beta g^\star $\/ ;
	\item $(A\cdot B)^\top  = B^\top \cdot A^\top$, \quad $(A^\top)^\top = A$, \quad et \quad $(\alpha A + \beta B)^\top = \alpha A^\top+ \beta B^\top $.
\end{itemize}
Des deuxièmes et troisièmes points,  il en résulte que les applications $f \mapsto f^\star$, et $A \mapsto A^\top$\/ sont des applications involutives.



		\begin{prop-defn}
  Soit $(\Omega, \mathcal{A}, P)$\/ un espace probabilisé, et soit $X$\/ une \textit{vard}. Si $X^2$\/ est d'espérance finie, alors $X$\/ aussi, et on appelle \textit{variance} le réel positif \[
    \mathrm{V}(X) = \mathrm{E}\Big(\big[X - \mathrm{E}(X)\big]^2\Big) = \underbrace{\mathrm{E}(X^2) - \big(\mathrm{E}(X)\big)^2}_{\mathclap{\text{Relation de \textsc{König \& Huygens}}}} \ge 0
  .\]
  L'\textit{écart-type} $\sigma(X)$\/ est la racine carrée de la variance : \[
    \sigma(X) = \sqrt{\mathrm{V}(X)}
  .\]
\end{prop-defn}

\begin{prv}
  On pose $\mu = \mathrm{E}(X)$, et on a $[X - \mu]^2 = X^2 - 2 \mu X + \mu^2$. D'où, par linéarité de l'espérance,
  \begin{align*}
    \mathrm{E}\big((X-\mu)^2\big)
    &= \mathrm{E}(X^2 - 2\mu X + \mu^2) \\
    &= \mathrm{E}(X^2) - 2\mu \mathrm{E}(X) + \mu^2 \\
    &= \mathrm{E}(X^2) - 2\mu^2 + \mu^2 \\
    &= \mathrm{E}(X^2) - \big(\mathrm{E}(X)\big)^2. \\
  \end{align*}
  De plus, d'après le lemme précédent, si $X^2$\/ est d'espérance finie, alors $X$\/ est d'espérance finie.
\end{prv}

\begin{rmk}
  \begin{enumerate}
    \item La variance mesure la \textit{dispersion}, ou l'\textit{étalement} des valeurs $a_i$\/ autour de l'espérance $\mathrm{E}(X)$. En particulier, s'il existe $a \in \R$\/ tel que $P(X = a) = 1$, alors $\mathrm{E}(X) = a$\/ et $\mathrm{V}(X) = 0$. (C'est même une équivalence.)
    \item Si la variable $X$\/ a une unité ($\mathrm{km}/\mathrm{s}$, $\mathrm{V}/\mathrm{m}$, etc.), alors l'écart type a la même unité (d'où l'intérêt de calculer la racine carrée de la variance).
    \item Soient $\alpha$\/ et $\beta$\/ deux réels. Si  $X^2$\/ est d'espérance finie, alors \[
      \mathrm{V}(\alpha X + \beta) = \alpha^2\cdot  \mathrm{V}(X)
    .\]
    (Une translation ne change pas la dispersion des valeurs, et multiplier par un réel multiplie l'espérance, mais aussi la dispersion, d'où le carré.)
  \end{enumerate}
\end{rmk}

\begin{exo}
  \textsl{Montrer que
  \begin{enumerate}
    \item si $X \sim \mathcal{B}(n, p)$, alors $X^2$\/ est d'espérance finie et $\mathrm{V}(X) = n\cdot p\cdot q$.
    \item si $T \sim \mathcal{G}(p)$, alors $T^2$\/ est d'espérance finie et $\mathrm{V}(T) = \frac{q}{p^2}$.
    \item si $X \sim \mathcal{P}(\lambda)$, alors $X^2$\/ est d'espérance finie et $\mathrm{V}(X) = \lambda$.
  \end{enumerate}
  }

  \begin{enumerate}
    \item
      Si $X \sim \mathcal{B}(n,p)$, alors $X(\Omega) = \llbracket 0,n \rrbracket$\/ et, pour $k \in X(\Omega)$, $P(X = k) = {n\choose k}\,p^k\,q^{n-k}$.
      On a déjà montré que $\mathrm{E}(X) = n\cdot p$.
      On va montrer que $\mathrm{V}(X) = n\,p\,q$.
      La variable aléatoire $X^2$\/ est d'espérance finie car $X(\Omega)$\/ est fini.
      Et,
      \begin{align*}
        \mathrm{E}(X^2) &= \sum_{k=0}^n k^2\: P(X = k)\\
        &= \sum_{k=0}^n k^2 {n\choose p} p^k q^{n-k} \\
        &= \ldots \\
      \end{align*}
      En effet, d'après la ``petite formule,'' on a \[
        \forall k \ge 1,\quad k{n\choose k} = n{n-1 \choose k-1}
      \] d'où, $(k-1) {n-1\choose k-1} = (n-1) \choose {n-2 \choose k-2}$. Ainsi, \[
        \forall k \ge 2,\quad k(k-1){n\choose k} = n(n+1) {n-2\choose k-2}
      .\] 
    \item Si $T \sim \mathcal{G}(p)$, alors $T(\Omega) = \N^*$\/ et $\forall k \in T(\Omega)$, $P(T = k) = p \times q^{k-1}$.
      On a déjà prouvé que $\mathrm{E}(T) = \frac{1}{p}$.
      On veut montrer que $\mathrm{V}(T) = \frac{q}{p^2}$. Montrons que la variable $T^2$\/ possède une espérance : la série $\sum k^2\: P(T = k)$\/ converge absolument car $k^2 \:P(T = k) = k^2 \cdot p \cdot q^{k-1}$.
      Or, pour $k \ge 2$, $\frac{\mathrm{d}^2}{\mathrm{d}x^2} x^k = k(k-1)\,x^{k-2}$. Et, on peut dériver terme à terme une série entière sans changer son rayon de convergence, et la série $\sum x^k$\/ a pour rayon de convergence 1. D'où, $\sum k(k-1)\, x^{k-2}$\/ a pour rayon de convergence 1. Or, $q \in {]0,1[} \subset {]-1,1[}$\/ donc la série $\sum k(k-1)q^{k-2}$\/ converge. De plus, $\sum k (k-1)\, q^{k-2} = \sum k^2\,q^{k-2} - \sum k\,q^{k-2}$.
      D'où,  $\sum k^2 q^{k-2} = \sum k(k-1)\, q^{k-2} + \sum k\,q^{k-2}$, qui converge. Par suite,
      \begin{align*}
        \sum_{k=1}^\infty k^2\, P(T = k) &= \sum_{k=1}^\infty k^2\,p\,q^{k-1}\\
        &= p + pq \sum_{k=2}^\infty k^2 q^{k-2} \\
        &= p + pq \sum_{k=2}^\infty k(k-1)\,q^{k-2} + p\sum_{k=2}^\infty k\,q^{k-1} \\
        &= p + pq\, \frac{2}{(1-q)^3} + p\left(\frac{1}{(1-q)^2} - 1 \right) \rlap{\quad\quad \text{\textit{c.f.} en effet après}}\\
        &= p + pq\, \frac{2}{p^3} + p\left( \frac{1}{p^2} - 1 \right) \\
        &= \frac{2q}{p^2} + \frac{1}{p} \\
        &= \frac{2q + p}{p^2} \\
        &= \frac{2q + (1-q)}{p^2} \\
        &= \frac{q+1}{p^2}. \\
      \end{align*}
      En effet, $\forall x \in {]-1,1[}$, $\sum_{k=0}^\infty x^k = \frac{1}{1-x}$. D'où, pour $x \in {]-1,1[}$, \[
        \sum_{k=1}^\infty k\,x^{k-1} = \frac{1}{(1-x)^2}
        \quad \text{ et }\quad
        \sum_{k=2}^\infty k(k-1)\,x^{k-2} = \frac{2}{(1-x)^3}
      .\]
      Ainsi, $\mathrm{E}(T^2) = \frac{q^{+1}}{p^2}$. D'où
      \begin{align*}
        \mathrm{V}(T) &= \mathrm{E}(T^2) - \big(\mathrm{E}(T)\big)^2 \\
        &= \frac{q+1}{p^2} - \left( \frac{1}{p} \right)^2 \\
        &= \frac{q}{p^2} \\
      \end{align*}
    \item À tenter
  \end{enumerate}
\end{exo}


\section{Les inégalités de \textsc{Markov} et de \textsc{Bienaymé}--\textsc{Tchebychev}, inégalités de concentration}

\begin{lem}[Markov]
  Soit $(\Omega, \mathcal{A}, P)$\/ un espace probabilisé, et soit $X$\/ une variable aléatoire \underline{positive}.
  Si $X$\/ est d'espérance finie, alors \[
    \forall a > 0, \quad P(X \ge a) \le \frac{\mathrm{E}(X)}{a}
  .\]
\end{lem}

\begin{prv}
  On suppose $X$\/ d'espérance finie. Ainsi, on a \[
    \mathrm{E}(X) = \sum_{x \in X(\Omega)} x\:P(X = x)
  .\]
  Soit $I$\/ l'ensemble $I = \{x \in X(\Omega) \mid x \ge a\}$.
  Alors, \[
    \mathrm{E}(X) = \underbrace{\sum_{x \in I} x\:P(X = x)}_{\text{ ici } x \ge a} + \underbrace{\sum_{x \in X(\Omega) \setminus I} x\:P(X = x)}_{\ge 0 \text{ par hypothèse}}
  .\]
  D'où, \[
    \mathrm{E}(X) \ge \sum_{x \in I} x\:P(X = x) \ge \sum_{x \in I} a\: P(X = x) = a \sum_{x \in I} P(X = x) \ge a\:P(x \ge a)
  .\]
\end{prv}

\begin{prop}[\textsc{Bienaymé--Tchebychev}]
  Soit $(\Omega, \mathcal{A}, P)$\/ un espace probabilisé, et soit $X$\/ une \textit{vard}. Si $X^2$\/ est d'espérance finie, alors \[
    \forall a > 0, \quad\quad P\Big(\big|X - \mathrm{E}(X)\big| \ge a\Big) \le \frac{\mathrm{V}(X)}{a^2}
  .\]
\end{prop}

\begin{prv}
  On pose $\mu = \mathrm{E}(X)$.
  L'événement $\big(|X - \mu| \ge a\big) = \big((X - \mu)^2 \ge a^2\big)$, d'où, les probabilités \[
    P\big(|X - \mu| \ge a\big) = P\big(\underbrace{(X - \mu)^2}_{\ge 0} \ge \underbrace{a^2}°{\ge 0}\big).
  \] On valide donc \textit{une} des hypothèses de l'inégalité de \textsc{Markov}.
  De plus, l'autre hypothèse est vérifiée : $X^2$\/ est d'espérance finie, donc $(X - \mu)^2$\/ aussi. On en déduit, d'après le lemme de \textsc{Markov}, que \[
    P\big((X-\mu)^2 \ge a^2\big) \le \frac{\mathrm{E}\big((X-\mu)^2\big)}{a^2} = \frac{\mathrm{V}(X)}{a^2}
  .\]
\end{prv}

\section{Série génératrice}

\begin{defn}
  Soit $X$\/ une \textit{vad} telle que $X(\Omega) \subset \N$. La \textit{série génératrice} de $X$\/ est la série entière $\sum a_n x^n$ de coefficients $a_n = P(X = n)$.
\end{defn}


La série $\sum a_n$\/ converge car sa somme vaut $\sum_{n=0}^\infty a_n = 1$. D'où, 
\begin{itemize}
  \item le rayon de convergence $R$\/ de la série est supérieur ou égal à 1.
  \item la série génératrice converge normalement sur $[-1,1]$, car la série $\sum |a_n|$\/ converge, or, $\forall x \in [-1,1]$, $|p_nt^n| \le |p_n|$, d'où la convergence normale.
    D'où la \textit{fonction génératrice} \[
      \mathrm{G}_X \colon t \longmapsto \sum_{n=0}^\infty p_n t^n
    \] est définie et même continue sur $[-1,1]$, car la convergence est uniforme.
  \item la fonction génératrice $\mathrm{G}_X$\/ est de classe $\mathcal{C}^\infty$\/ sur $]-1,1[$\/ et \[
      \forall k \in \N,\quad P(X = k) = a_n \frac{{\mathrm{G}_X}^{(k)}(0)}{k!}
    .\] La fonction génératrice de $X$\/ permet donc de retrouver la loi de probabilité de $X$.
\end{itemize}


		\begin{exo}
	Soient $\lambda_1, \ldots, \lambda_r \in \R$\/ distincts deux à deux.
	Montrons que, si $\forall x \in \R$, $\alpha_1 \mathrm{e}^{\lambda_1 x} + \cdots + \alpha_r \mathrm{e}^{\lambda_r x} = 0$, alors $\alpha_1 = \cdots = \alpha_r$.
	On peut procéder de différentes manières : le déterminant de {\sc Vandermonde}, par analyse-sythèse, ou, en utilisant \[
		\frac{\mathrm{d}}{\mathrm{d}x}\left( \mathrm{e}^{\lambda_k x} \right) = \lambda_k \mathrm{e}^{\lambda_k x},\quad\text{d'où}\quad\varphi(f_k) = \lambda_k f_k, \text{ avec } f_k : x \mapsto \mathrm{e}^{\lambda_k x}\quad\text{et}\quad\varphi:f\mapsto f'
	.\]
	On doit vérifier que les $f_k$\/ sont des vecteurs et l'application $\varphi$\/ soit un endomorphisme. On se place donc dans l'espace vectoriel $\mathscr{C}^{\infty}$. (On ne peut pas se placer dans l'espace $\mathscr{C}^k$, car sinon l'application $\varphi$\/ est de l'espace $\mathscr{C}^k$\/ à $\mathscr{C}^{k-1}$, ce n'est donc pas un endomorphisme ; ce n'est pas le cas pour l'espace $\mathscr{C}^\infty$.)
	Or, les $\lambda_k$\/ sont distincts deux à deux d'où les vecteurs propres $f_k$\/ sont linéairement indépendants. Et donc si $\alpha_1 f_1 + \alpha_2 f_2 + \cdots + \alpha_r f_r = 0$\/ alors $\alpha_1 = \cdots = \alpha_r=0$.
	Mais, comme $\forall x \in \R$, $\alpha_1 f_1(x) + \alpha_2 f_2(x) + \cdots + \alpha_r f_r(x) = 0$, on en déduit que \[
		\boxed{\alpha_1 = \cdots = \alpha_r = 0.}
	\]
\end{exo}

\section{Critères de diagonalisabilité}

\begin{prop}[une condition \underline{suffisante} pour qu'une matrice soit diagonalisable]
	Soit $A$\/ une matrice carrée de taille $n \ge 2$. {\color{red}Si} $A$\/ possède $n$\/ valeurs propres distinctes deux à deux, {\color{red}alors} $A$\/ est diagonalisable.
\end{prop}

\begin{rmkn}
	La réciproque est fausse : par exemple, pour $n > 1$, $7 I_n$\/ est diagonalisable car elle est diagonale. Mais, elle ne possède pas $n$\/ valeurs propres distinctes deux à deux.
\end{rmkn}

\begin{prv}
	On suppose que la matrice $A \in \mathscr{M}_{n,n}(\mathds{K})$\/ possède $n$\/ valeurs propres distinctes deux à deux (i.e.~$\Card \Sp(A) = n$). D'où, d'après la proposition 16, les $n$\/ vecteurs propres associés $\varepsilon_1,\ldots,\varepsilon_n$\/ sont libres. D'où $(\varepsilon_1, \ldots, \varepsilon_n)$\/ est une base formée de vecteurs propres. Donc, d'après la définition 5, la matrice $A$\/ est diagonalisable.
\end{prv}

\begin{thm}[conditions \underline{nécessaires et suffisantes} pour qu'une matrice soit diagonalisable]
	Soient $E$\/ un espace vectoriel de dimension finie et $u : E \to E$\/ un endomorphisme.
	Alors,
	\begin{align*}
		(1)\quad u \text{ diagonalisable } \iff& E = \bigoplus_{\lambda \in \Sp(u)} \Ker(\lambda\id_E - u) \quad(2)\\
		\iff& \dim E = \sum_{\lambda \in \Sp(u)} \dim(\mathrm{SEP}(\lambda))\quad(3)\\
		\iff& \chi_u \text{ scindé et } \forall \lambda \in \Sp(u),\:\dim(\mathrm{SEP}(\lambda)) = m_\lambda\quad(4)
	\end{align*}
	où $m_\lambda$\/ est la multiplicité de la racine $\lambda$\/ du polynôme $\chi_u$.
\end{thm}

\begin{prv}
	\begin{itemize}
		\item[``$(1)\implies(2)$''] On suppose $u$\/ diagonalisable. Il existe donc une base $(\varepsilon_1$, \ldots, $\varepsilon_n)$\/ de $E$\/ formée de vecteurs propres de $u$. On les regroupes par leurs valeurs propres : $(\varepsilon_i, \ldots, \varepsilon_{i+j})$\/ forme une base de $\mathrm{SEP(\lambda_k)}$. D'où la base $(\varepsilon_1, \ldots, \varepsilon_n)$\/ de l'espace vectoriel $E$\/ est une concaténation des bases des sous-espaces propres de $u$. D'où \[
				E = \bigoplus_{\lambda \in \Sp(u)} \mathrm{SEP}(\lambda)
			.\]
		\item[``$(2)\implies(1)$''] On suppose que $E = \mathrm{SEP}(\lambda_1) \oplus \mathrm{SEP}(\lambda_2) \oplus \cdots \oplus \mathrm{SEP}(\lambda_r)$.
			Soient $(\varepsilon_1, \ldots, \varepsilon_{d_1})$\/ une base de $\mathrm{SEP}(\lambda_1)$, $(\varepsilon_{d_1 + 1}, \ldots, \varepsilon_{d_1 + d_2})$\/ une base de $\mathrm{SEP}(\lambda_2)$, \ldots, $(\varepsilon_{d_1+\cdots + d_{r-1}+1}$, \ldots, $\varepsilon_{d_1+ \cdots + d_r})$\/ une base de $\mathrm{SEP}(\lambda_r)$.
			En concaténant ces base, on obtient une base de $E$, d'après l'hypothèse. Dans cette base, tous les vecteurs sont propres donc $u$\/ est diagonalisable.
		\item[``$(2)\implies(3)$''] On suppose $E = \mathrm{SEP}(\lambda_1) \oplus \mathrm{SEP}(\lambda_2) \oplus \cdots \oplus \mathrm{SEP}(\lambda_r)$. D'où  \[
					\dim E = \dim(\mathrm{SEP}(\lambda_1)) + \dim(\mathrm{SEP}(\lambda_2)) + \cdots + \dim(\mathrm{SEP}(\lambda_r))
			\] car la dimension d'une somme directe est égale à la somme des dimensions.
		\item[``$(3)\implies(1)$''] On suppose $\dim E = \dim(\mathrm{SEP}(\lambda_1)) + \dim(\mathrm{SEP}(\lambda_2)) + \cdots + \dim(\mathrm{SEP}(\lambda_r))$. Or, les sous-espaces propres sont en somme directe, d'après la proposition 16. D'où $\dim\Big(\sum_{\lambda \in \Sp(u)} \mathrm{SEP}(\lambda) \Big)= \sum_{\lambda \in \Sp(u)} \dim(\mathrm{SEP}(\lambda))$. Donc $\sum_{\lambda \in \Sp(u)} \mathrm{SEP}(\lambda) = E$.
		\item[``$(4)\implies(3)$''] On suppose (a) $\chi_u$\/ scindé et (b) $\dim(\mathrm{SEP}(\lambda)) = m_\lambda$. D'où, d'après (a): \[
				\chi_u(x) = (x - \lambda_1)^{m_{\lambda_1}}(x - \lambda_2)^{m_{\lambda_2}} \cdots (x - \lambda_r)^{m_{\lambda_r}} = x^n + \cdots
			\] d'où $m_{\lambda_1} + m_{\lambda_2} + \cdots + m_{\lambda_r} = n$, et d'où \[
				\dim(\mathrm{SEP}(\lambda_1)) + \dim(\mathrm{SEP}(\lambda_2)) + \cdots + \dim(\mathrm{SEP}(\lambda_r)) = n
			\] d'après l'hypothèse (b).
		\item[``$(1)\implies(4)$'']
			On suppose $u$\/ diagonalisable. D'où, dans une certaine base $\mathscr{B}$, la matrice $\big[u\big]_\mathscr{B}$\/ est diagonale. Quitte à changer l'ordre des éléments de $\mathscr{B} = (\varepsilon_1,\ldots,\varepsilon_r)$, on peut supposer que $\big[u\big]_\mathscr{B}$\/ est de la forme \[
				\big[u\big]_\mathscr{B} = 
				\begin{bNiceArray}{c|c|c|c}[last-col]
					\begin{array}{cccc}\lambda_1\\&\lambda_1\\&&\ddots\\&&&\lambda_1\end{array}&0&0&0&\begin{array}{l}\varepsilon_1\\\varepsilon_2\\\vdots\\\varepsilon_{d_1}\\\end{array}\\ \hline
					0&\begin{array}{cccc}\lambda_2\\&\lambda_2\\&&\ddots\\&&&\lambda_2\end{array}&0&0&\begin{array}{l}\varepsilon_{d_1+1}\\\varepsilon_{d_1+2}\\\vdots\\\varepsilon_{d_1+d_2}\\\end{array}\\ \hline
					 &&\ddots&&\vdots\\ \hline
					0&0&0&0\begin{array}{cccc}\lambda_r\\&\lambda_r\\&&\ddots\\&&&\lambda_r\end{array}&\begin{array}{l}\varepsilon_{d_1+\cdots + d_{r-1} + 1}\\\varepsilon_{d_1+\cdots + d_{r-1}+2}\\\vdots\\\varepsilon_{d_1 + \cdots + d_r}\\\end{array}\\
				\end{bNiceArray}
			.\] D'où $\forall k \in \left\llbracket 1,r \right\rrbracket$, $d_k = \dim(\mathrm{SEP}(\lambda_k))$. En outre, $\chi_u(x) = \det(x\id - u) = (x - \lambda_1)^{d_1}\cdot (x-\lambda_2)^{d_2} \cdots (r - \lambda_r)^{d_r}$.
			D'où $\forall k \in \left\llbracket 1,r \right\rrbracket$, $d_k = m_{\lambda_k}$\/ et $\chi_u$\/ est scindé.
	\end{itemize}
\end{prv}

\begin{exo}
	{\slshape On considère la matrice $E$\/ ci-dessous \[
		E = \begin{pmatrix}
			7&0&1\\
			0&3&0\\
			0&0&7
		\end{pmatrix}.
	\] La matrice $E$\/ ci-dessous est-elle diagonalisable ?}

	Soit $\lambda \in \R$. On sait que $\lambda \in \Sp(E)$\/ si et seulement si $\det(\lambda I_3 - E) = 0$. Or \[
		\det(\lambda I_3 - E) =
		\begin{vmatrix}
			\lambda - 7&0&-1\\
			0&\lambda-3&0\\
			0&0&\lambda - 7
		\end{vmatrix} = (\lambda - 7)^2\cdot  (\lambda - 3)^1
	.\] Donc $\Sp(E) = \{3,7\}$, $1 \le \dim\big(\mathrm{SEP}(3)\big) \le 1$, et $1 \le \dim\big(\mathrm{SEP}(7)\big) \le 2$.
	La matrice $E$\/ est diagonalisable si et seulement si $\dim(\mathrm{SEP}(3)) + \dim(\mathrm{SEP}(7)) = 3$, donc si et seulement si $\dim(\mathrm{SEP}(7)) = 2$. On cherche donc la dimension de ce sous-espace propre : soit $X = \left( \substack{x\\y\\z} \right) \in \mathscr{M}_{3,1}(\R)$. On sait que
	\begin{align*}
		X \in \mathrm{SEP}(7) \iff& E\cdot X = 7X\\
		\iff& \begin{pmatrix}
			7&0&1\\
			0&3&0\\
			0&0&7
		\end{pmatrix} \begin{pmatrix}
			x\\y\\z
		\end{pmatrix} = 7 \begin{pmatrix}
			x\\y\\z
		\end{pmatrix}\\
		\iff& \begin{cases}
			7x + 0y + 1z = 7x\\
			3y = 7y\\
			7z = 7z
		\end{cases}\\
		\iff& \begin{cases}
			z = 0\\
			y = 0
		\end{cases}\\
		\iff& X = \begin{pmatrix}
			x\\0\\y
		\end{pmatrix} = x \underbrace{\begin{pmatrix}
			1\\0\\0
		\end{pmatrix}}_{\varepsilon_1}
	\end{align*}
	Donc $\mathrm{SEP}(7) = \Vect(\varepsilon_1)$, d'où $\dim(\mathrm{SEP}(7)) = 1$. Donc la matrice $E$\/ n'est pas diagonalisable.
\end{exo}

\section{Trigonalisation}

Trigonaliser une matrice ne sert que si la matrice n'est pas diagonalisable.

\begin{defn}
	On dit d'une matrice carrée $A \in \mathscr{M}_{n,n}(\mathds{K})$\/ qu'elle est {\it trigonalisable}\/ s'il existe une matrice inversible $P$\/ telle que $P^{-1} \cdot A \cdot P$\/ est triangulaire : \[
		P^{-1} \cdot A \cdot P = \begin{pNiceMatrix}
			\lambda_1&\Block{2-2}*&\\
			\Block{2-2}0&\Ddots&\\
			&&\lambda_r
		\end{pNiceMatrix}
	.\]
\end{defn}

\begin{rmk}
	$\O$\/
\end{rmk}

\begin{thm}
	Une matrice carrée $A \in \mathscr{M}_{n,n}(\mathds{K})$\/ est trigonalisable si et seulement si son polynôme caractéristique $\chi_A \in \mathds{K}[X]$\/ est scindé.
\end{thm}

		\begin{prv}[par récurrence sur $n$, la largeur de la matrice]
	\begin{itemize}
		\item Si $n = 1$, alors la matrice $A = (a_{11})$\/ est déjà triangulaire.
		\item On suppose le polynôme caractéristique $\chi_A$\/ de la matrice scindé dans $\mathds{K}[X]$, d'où il a au moins une racine dans $\mathds{K}$. D'où, la matrice $A$\/ a au moins une valeur propre $\lambda_1 \in \mathds{K}$. Il existe donc un vecteur non nul $\vec{\varepsilon}_1$\/ tel que $A \cdot \vec{\varepsilon}_1 = \lambda_1\,\vec{\varepsilon}_1$. On complète $(\vec{\varepsilon}_1)$\/ en une base de $\mathds{K}^n$\/ : $(\vec{\varepsilon}_1, \vec{e}_2, \ldots, \vec{e}_n)$. En changent de base, il existe une matrice inversible $P$\/ telle que \[
			A' = P^{-1}\cdot A\cdot P = 
			\begin{pNiceArray}[last-row,last-col]{c|ccc}
				\lambda_1 & *&\Ldots&*&\varepsilon_1\\ \hline
				0 & \Block{3-3}{B}&&&e_1\\
				\Vdots&&&&\Vdots\\
				0&&&&e_n\\
				f(\vec{\varepsilon}_1)&f(\vec{e}_1)&\ldots&f(\vec{e}_n)
			\end{pNiceArray}
		.\]
		Comme le polynôme caractéristique est invariant par changement de base, on en déduit que \[
			\chi_A(x) = \chi_{A'}(x) = \left|
			\begin{array}{c|c}
				x-\lambda_1 &*\\ \hline
				0&xI_{n-1} - B\\
			\end{array} \right| = (x-\lambda_1) \cdot \Pi(x)
		.\]
		Or, comme $\chi_A$\/ est scindé, $\Pi(x)$\/ est aussi scindé.
		Or, $\Pi(x) = \det(xI_{n-1} - B)$ d'où $B$\/ est trigonalisable.
	\end{itemize}
\end{prv}

\begin{crlr}
	Toute matrice de $\mathscr{M}_{n,n}(\C)$\/ est trigonalisable.
\end{crlr}

\begin{exo}\relax
	{\slshape Soit une matrice carrée $A \in \mathscr{M}_{n,n}(\mathds{K})$ (où $\mathds{K}$\/ est $\R$\/ ou $\C$). Montrer que
		\begin{align*}
			(1)\quad\text{la matrice } A \text{ est nilpotente}
			\iff& \text{ le polynôme caractéristique de } A \text{ est } \chi_A(X) = X^n\quad(2)\\
			\iff& \text{ la matrice } A \text{ est trigonalisable avec des zéros}\\
			&\text{ sur sa diagonale} \quad(3)
		\end{align*}
	}

	On montre $``\,(1) \implies (2),"$ $``\,(2) \implies (3)\,"$\/ puis $``\,(3) \implies (1)."$

	\begin{itemize}
		\item[$``\,(3) \implies(1)\,"$] Il existe donc une matrice inversible $P$\/ telle que $T = P^{-1}\cdot A\cdot P$\/ et $T$\/ est une matrice trigonalisable. Or, à chaque produit $A^n \cdot A$, une \guillemotleft~sur-diagonalse~\guillemotright\  de zéros supplémentaires. D'où, à partir d'un certain rang $p$, on a $A^p = 0$. La matrice $A$\/ est donc nilpotente.
		\item[$``\,(2) \implies(3)\,"$] On sait que $\chi_A = X^n = (X-0)^n$\/ est scindé, d'où $A$\/ est trigonalisable.
			Il existe donc une matrice inversible $P$\/ telle que \[
				P^{-1}\cdot A\cdot P = A' = \begin{pmatrix}
					\lambda_1 & * & \ldots & *\\
					0 & \ddots&\ddots&\vdots\\
					\vdots&\ddots&\ddots&*\\
					0&\ldots&0&\lambda_n
				\end{pmatrix}
			.\]
			Et donc $\chi_{A'}(x) = (x-\lambda_1)(x-\lambda_2) \cdots (x-\lambda_n)$.
			Or, le polynôme caractéristique est invariant par changement de base, d'où $\lambda_1 = \lambda_2 = \cdots = \lambda_n$.
		\item[$``\,(1)\implies(2)\,"$] On passe dans $\C$\/ alors $\chi_A$\/ est scindé dans $\C$. D'où, il existe $(\lambda_1, \lambda_2, \ldots, \lambda_n) \in \C^n$\/ tels que \[
			\chi_A(X) = (X - \lambda_1) (X - \lambda_2) \cdots (X-\lambda_n)
		.\]
		D'où, chaque $\lambda_i$\/ est une valeur propre \ul{complexe} de la matrice $A$. Or $A$\/ est nilpotente, d'où, par définition, il existe $p \in \N$\/ tel que $A^p = 0$. Les scalaires $\lambda_i$\/ sont dans le spectre de $A$\/ : en effet, il existe un vecteur colonne $X$\/ non nul tel que $A\cdot X = \lambda_i\,X$, d'où $A^2 \cdot X = A\cdot AX = A\cdot \lambda_iX = \lambda_i^2 X$. De même, $A^3 \cdot X = A \cdot A^2 \cdot X = A \cdot \lambda_i^2 X = \lambda_i^2 (A\cdot X) = \lambda_i^3 X$.
		Et, de \guillemotleft~proche en proche~\guillemotright, on a donc \[
			\forall k \in \N,\:A^k\cdot X = \lambda_i^k X
		.\]
		En particulier, si $k=p$, on a $0 = 0\cdot X = A^p\cdot X = \lambda_i^pX$. D'où $\lambda_i^p X = 0$. Or, $X \neq 0$, d'où $\lambda_i^p = 0$\/ et donc $\lambda_i = 0$.
		Finalement, $\chi_A(X) = (X-\lambda_1) \cdots (X-\lambda_n) = (X-0)\cdots(X-0) = X^n  \in \C[X]$. On a donc $\chi_A(X) \in \R[X]$.
	\end{itemize}
\end{exo}

		\clearpage
		\setcounter{section}{0}		\renewcommand{\thesection}{\llap{Annexe }\thechapter.\Alph{section}}
		\renewcommand{\thesectionnum}{\Alph{section}}
		\section{Programmation dynamique}

On rappelle le problème \textsc{Knapsack} : \[
	\begin{cases}
		\text{\textbf{Entrée}} &: n \in \N,\: w \in (\N^*)^n,\: v \in (\N^*)^n,\: P \in \N\\
		\text{\textbf{Sortie}} &:\max_{x \in \{0,1\}^n} \big\{\left<x,v \right> \:\big|\: \left<x,w \right> \le P \}.
	\end{cases}
\]

On pose \[\textsc{sad}(n, w, v, P) = \max_{x \in \{0,1\}^n} \big\{\left<x,v \right> \:\big|\: \left<x,w \right> \le P \},\] et \[\mathrm{sol}(n, w, v, P) = \{\left<x,v \right>  \mid \left<x,w \right> \le P,\: x \in \{0,1\}^n \}.\] Lorsque $y \in \R^n$, avec $y = (y_1, \ldots, y_n)$, on note $\R^{n-1} \owns \tilde y = (y_2, y_3, \ldots, 0)$. Ainsi, si $n > 0$,
\begin{align*}
	\mathrm{sol}(n, w, v, P) &= \{\left<x, v \right>  \mid \left< x,w \right> \le P,\: x \in \{0,1\}^n \text{ et } x_1 = 0\}\\
	&\quad\cup \{\left<x, v \right>  \mid \left< x,w \right> \le P,\: x \in \{0,1\}^n \text{ et } x_1 = 1\}\\
	&= \{\left<\tilde{x}, \tilde{v} \right>  \mid \left<\tilde{x}, \tilde{w} \right> \le P,\: x u\in \{0,1\}^n \text{ et } x_1 = 0\} \\
	&\quad\cup \{v_1 + \left<\tilde{x}, \tilde{v} \right>  \mid \left<\tilde{x},\tilde{w} \right> \le P - w_1,\: x \in \{0,1\}^n \text{ et } x_1 = 1\}\\
	&= \{\left<y, \tilde{v} \right>  \mid \left<y, \tilde{w} \right> \le P \text{ et } y \in \{0,1\}^{n-1}\}  \\
	&\quad\cup \{v_1 + \left<y, \tilde{v} \right>  \mid \left<y, \tilde{w} \right> \le P - w_1 \text{ et } y \in \{0,1\}^{n-1}\}
\end{align*}
D'où, par passage au $\max$, si $n > 0$,
\begin{align*}
	\textsc{sad}(n, w, v, P) &= \max( \\
	&\quad\quad \max \{\left<y  \mid \tilde{v} \right>  \mid \left<y, \tilde{w} \right> \le P \text{ et } y \in \{0,1\}^{n-1}\}\\
	&\quad\quad v_1 + \max \{\left<y  \mid \tilde{v} \right>  \mid \left<y, \tilde{w} \right> \le P - w_1 \text{ et } y \in \{0,1\}^{n-1}\} \\
	&) = \max\!\big(\textsc{sad}(n-1, \tilde{w}, \tilde{v}, P), v_1 + \textsc{sad}(n-1, \tilde{w}, \tilde{v}, P - w_1)\big).
\end{align*}
Si $n = 0$, alors $\textsc{sad}(0, v, w, P) = 0$.

\begin{rmk}
	Si on le code \textit{tel quel}, il y aura $\mathcal{O}(2^n)$ appels récursifs. Mais, on a $(n+1)(P+1)$\/ sous-problèmes.
\end{rmk}

Notons alors, pour $n$, $v$, $w$, $P$ fixés, $(s_{i,j})_{\substack{i \in \llbracket 1,n \rrbracket\\ j \in \llbracket 0,P \rrbracket}}$\/ tel que \[
	s_{i,j} = \textsc{sad}\Big(n-i, v_{\big|\llbracket i+1,n \rrbracket}, w_{\big|\llbracket i+1,n \rrbracket}, j\Big)
.\]
On a alors $\textsc{sad}(n, v, w, P) = s_{0,P}$. Ainsi, pour $j \in \llbracket 0,P \rrbracket$, $s_{n,j} = 0$ ; pour $i \in \llbracket 0,n \rrbracket$, $s_{i,0} = 0$ ; \[
	s_{i,j} = \max(s_{i+1,j}, v_{i+1} + s_{i+1,j- w_{i+1}})
;\] et, si $w_{i+1} > j$, alors $s_{i,j} = s_{i+1,j}$.

La complexité de remplissage de la matrice est en $\mathcal{O}(n\,P)$\/ en temps et en espace.
On n'a pas prouvé $\text{\textbf{P}} = \text{\textbf{NP}}$, la taille de l'entrée est
\begin{itemize}
	\item pour un entier $n$\/ : $\log_2(n)$,
	\item pour un tableau de $n$\/ entiers : $n \log_2(n)$,
	\item pour un tableau de $n$\/ entiers : $n \log_2(n)$,
	\item pour un entier $P$ : {\color{red}$\log_2(P)$}.
\end{itemize}
Vis à vis de la taille de l'entrée, la complexité de remplissage est {\color{red}exponentielle}.


	}
	\def\addmacros#1{#1}
}

{
	\chap[6]{Preuves}
	\minitoc
	\renewcommand{\cwd}{../cours/chap06/}
	\addmacros{
		\section{Motivation}

\lettrine On place au centre de la classe 40 bonbons. On en distribue un chacun. Si, par exemple, chacun choisit un bonbon et, au \textit{top} départ, prennent celui choisi.
Il est probable que plusieurs choisissent le même. Comme gérer lorsque plusieurs essaient d'accéder à la mémoire ?

Deuxièmement, sur l'ordinateur, plusieurs applications tournent en même temps. Pour le moment, on considérait qu'un seul programme était exécuté, mais, le \textsc{pc} ne s'arrête pas pendant l'exécution du programme.

On s'intéresse à la notion de \guillemotleft~processus~\guillemotright\ qui représente une tâche à réaliser.
On ne peut pas assigner un processus à une unité de calcul, mais on peut \guillemotleft~allumer~\guillemotright\ et \guillemotleft~éteindre~\guillemotright\ un processus.
Le programme allumant et éteignant les processus est \guillemotleft~l'ordonnanceur.~\guillemotright\@ Il doit aussi s'occuper de la mémoire du processus (chaque processus à sa mémoire séparée).

On s'intéresse, dans ce chapitre, à des programmes qui \guillemotleft~partent du même~\guillemotright\ : un programme peut créer un \guillemotleft~fil d'exécution~\guillemotright\ (en anglais, \textit{thread}). Le programme peut gérer les fils d'exécution qu'il a créé, et éventuellement les arrêter.
Les fils d'exécutions partagent la mémoire du programme qui les a créé.

En C, une tâche est représenté par une fonction de type \lstinline[language=c]!void* tache(void* arg)!. Le type \lstinline[language=c]!void*!\ est l'équivalent du type \lstinline[language=caml]!'a! : on peut le \textit{cast} à un autre type (comme \lstinline[language=c]-char*-).

\begin{lstlisting}[language=c,caption=Création de \textit{thread}s en C]
void* tache(void* arg) {
	printf("%s\n", (char*) arg);
	return NULL;
}

int main() {
	pthread_t p1, p2;

	printf("main: begin\n");

	pthread_create(&p1, NULL, tache, "A");
	pthread_create(&p2, NULL, tache, "B");

	pthread_join(p1, NULL);
	pthread_join(p2, NULL);

	printf("main: end\n");

	return 0;
}
\end{lstlisting}

\begin{lstlisting}[language=c,caption=Mémoire dans les \textit{thread}s en C]
int max = 10;
volatile int counter = 0;

void* tache(void* arg) {
	char* letter = arg;
	int i;

	printf("%s begin [addr of i: %p] \n", letter, &i);

	for(i = 0; i < max; i++) {
		counter = counter + 1;
	}

	printf("%s : done\n", letter);
	return NULL;
}

int main() {
	pthread_t p1, p2;

	printf("main: begin\n");

	pthread_create(&p1, NULL, tache, "A");
	pthread_create(&p2, NULL, tache, "B");

	pthread_join(p1, NULL);
	pthread_join(p2, NULL);

	printf("main: end\n");

	return 0;
}
\end{lstlisting}

Dans les \textit{thread}s, les variables locales (comme \texttt{i}) sont séparées en mémoire. Mais, la variable \texttt{counter} est modifiée, mais elle ne correspond pas forcément à $2 \times \texttt{max}$. En effet, si \texttt{p1} et \texttt{p2} essaient d'exécuter au même moment de réaliser l'opération \lstinline[language=c]-counter = counter + 1-, ils peuvent récupérer deux valeurs identiques de \texttt{counter}, ajouter 1, puis réassigner \texttt{counter}.
Ils \guillemotleft~se marchent sur les pieds.~\guillemotright\ 

Parmi les opérations, on distingue certaines dénommées \guillemotleft~atomiques~\guillemotright\ qui ne peuvent pas être séparées. L'opération \lstinline[language=c]-i++- n'est pas atomique, mais la lecture et l'écriture mémoire le sont.

\begin{defn}
	On dit d'une variable qu'elle est \textit{atomique} lorsque l'ordonnanceur ne l'interrompt pas.
\end{defn}

\begin{exm}
	L'opération \lstinline[language=c]-counter = counter + 1- exécutée en série peut être représentée comme ci-dessous. Avec \texttt{counter} valant 40, cette exécution donne 42.
	\begin{table}[H]
		\centering
		\begin{tabular}{l|l}
			Exécution du fil A & Exécution du fil B\\ \hline
			(1)~$\mathrm{reg}_1 \gets \texttt{counter}$ & (4)~$\mathrm{reg}_2 \gets \texttt{counter}$ \\
			(2)~$\mathrm{reg}_1{++}$ & (5)~$\mathrm{reg}_2{++}$ \\
			(3)~$\texttt{counter} \gets \mathrm{reg}_1$ & (6)~$\texttt{counter} \gets \mathrm{reg}_2$
		\end{tabular}
	\end{table}
	\noindent Mais, avec l'exécution en simultanée, la valeur de \texttt{counter} sera 41.
	\begin{table}[H]
		\centering
		\begin{tabular}{l|l}
			Exécution du fil A & Exécution du fil B\\ \hline
			(1)~$\mathrm{reg}_1 \gets \texttt{counter}$ & (2)~$\mathrm{reg}_2 \gets \texttt{counter}$ \\
			(3)~$\mathrm{reg}_1{++}$ & (5)~$\mathrm{reg}_2{++}$ \\
			(4)~$\texttt{counter} \gets \mathrm{reg}_1$ & (6)~$\texttt{counter} \gets \mathrm{reg}_2$
		\end{tabular}
	\end{table}
	\noindent Il y a \textit{entrelacement} des deux fils d'exécution.
\end{exm}

\begin{rmk}[Problèmes de la programmation concurrentielle]
	\begin{itemize}
		\item Problème d'accès en mémoire,
		\item Problème du rendez-vous,\footnote{Lorsque deux programmes terminent, ils doivent s'attendre pour donner leurs valeurs.}
		\item Problème du producteur-consommateur,\footnote{Certains programmes doivent ralentir ou accélérer.}
		\item Problème de l'entreblocage,\footnote{\textit{c.f.} exemple ci-après.}
		\item Problème famine, du dîner des philosophes.\footnote{Les philosophes mangent autour d'une table, et mangent du riz avec des baguettes. Ils décident de n'acheter qu'une seule baguette par personne. Un philosophe peut, ou penser, ou manger. Mais, pour manger, ils ont besoin de deux baguettes. S'ils ne mangent pas, ils meurent.}
	\end{itemize}
\end{rmk}

\begin{exm}[Problème de l'entreblocage]~

	\begin{table}[H]
		\centering
		\begin{tabular}{l|l|l}
			Fil A & Fil B & Fil C\\ \hline
			RDV(C) & RDV(A) & RDV(B)\\
			RDV(B) & RDV(C) & RDV(A)\\
		\end{tabular}
		\caption{Problème de l'entreblocage}
	\end{table}
\end{exm}

Comment résoudre le problème des deux incrementations ? Il suffit de \guillemotleft~mettre un verrou.~\guillemotright\ Le premier fil d'exécution \guillemotleft~s'enferme~\guillemotright\ avec l'expression \lstinline[language=c]!count++!, le second fil d'exécution attend que l'autre sorte pour pouvoir entrer et s'enfermer à son tour.


		\section{Continuité}

\begin{exm}
	Dans l'exercice 2, chaque fonction $f_n : t \mapsto t^n$\/ est continue sur $[0,1]$\/ mais la limite $f$\/ n'est pas continue sur $[0,1]$\/ (car elle n'est pas continue en $1$).
\end{exm}

\begin{thm}
	Soit $a$\/ un réel dans un intervalle $T$\/ de $\R$. Si une suite de fonctions $(f_n)_{n\in\N}$\/ continues en $a$\/ converge uniformément sur $T$\/ vers une fonction $f$, alors $f$\/ est aussi continue en $a$.
\end{thm}

\begin{prv}
	On suppose les fonctions $f_n$\/ continues en $a$\/ ($f_n(x) \longrightarrow f_n(a)$) et que la suite de fonctions $(f_n)_{n\in\N}$\/ converge uniformément vers $f$\/ ($\sup\:|f_n -f| \longrightarrow 0$). On veut montrer que $f$\/ est continue en $a$\/ : $f(x) \tendsto{x \to a} f(a)$, i.e.\ \[
		\forall \varepsilon > 0,\:\exists \delta > 0,\: \forall x \in T,\quad|x-a| \le \delta \implies |f(x) - f(a)| \le \varepsilon
	.\]
	Soit $\varepsilon > 0$. On calcule \[
		\big|f(x) - f(a)\big| \le \big|f(x) - f_n(x)\big| + \big|f_n(x) - f_n(a)\big| + \big|f_n(a) - f(a)\big|
	\] par inégalité triangulaire. Or, par hypothèse, il existe un rang $N \in \N$\/ (qui ne dépend pas de $x$\/ ou de $a$), tel que, $\forall n \ge N$, $\big|f(x) - f_n(x)\big| \le \frac{1}{3} \varepsilon$, et $\big|f_n(a) - f(a)\big| \le \frac{1}{3} \varepsilon$.
	De plus, par hypothèse, il existe $\delta >0$\/ tel que si $|x - a| \le \delta$, alors $|f_n(x) - f_n(a)| \le \frac{1}{3}\varepsilon$.\footnote{C'est là où l'hypothèse de la convergence uniforme est utilisée : on a besoin que le $N$\/ ne dépende pas de $x$\/ car on le fait varier.}
	On en déduit que $\big|f(x) - f(a)\big| \le \varepsilon$.
\end{prv}

\begin{crlr}
	Soit $T$\/ un intervalle de $\R$. Si une suite de fonctions $(f_n)_{n\in\N}$\/ continues sur $T$\/ converge uniformément sur $T$\/ vers une fonction continue sur $T$.
\end{crlr}

\begin{met}[Stratégie de la barrière]
	\begin{enumerate}
		\item La continuité (la dérivabilité aussi) est une propriété {\it locale}. Pour montrer qu'une fonction est continue sur un intervalle $T$, il suffit donc de montrer qu'elle est continue sur tout segment inclus dans $T$.
		\item Mais, la convergence uniforme est une propriété {\it globale}. La convergence sur tout segment inclus dans un intervalle n'implique pas la convergence uniforme sur l'intervalle (voir l'exercice 2).
		\item On n'écrit pas \[
				\substack{\ds\text{convergence uniforme}\\\ds\text{avec barrière}} \mathop{\red\implies} \substack{\ds\text{convergence uniforme}\\\ds\text{sans barrière}} \implies \substack{\ds\text{continuité}\\\ds\text{sans barrière}}
			\] mais plutôt \[
				\substack{\ds\text{convergence uniforme}\\\ds\text{avec barrière}} \implies \substack{\ds\text{continuité}\\\ds\text{avec barrière}} \implies \substack{\ds\text{continuité}\\\ds\text{sans barrière}}
			.\]
		\item Si, pour tous $a$\/ et $b$, $f$\/ est bornée sur $[a,b] \subset T$, mais cela n'implique pas que $f$\/ est bornée. Contre-exemple : la fonction $f : x \mapsto \frac{1}{x}$\/ est bornée sur tout intervalle $[a,b]$\/ avec $a$, $b \in \R^+_*$, \red{\sc mais} $f$\/ n'est pas bornée sur $]0,+\infty[$.
	\end{enumerate}
\end{met}

\begin{thm}[double-limite ou d'interversion des limites]
	Soit une suite de fonctions $(f_n)_{n\in\N}$\/ définies sur un intervalle $T$, et, soit $a$\/ une extrémité (éventuellement infinie)\footnote{autrement dit, $a \in \bar\R = \R \cup \{+\infty,-\infty\}$} de cet intervalle. Si la suite de fonctions $(f_n)_{n\in\N}$\/ converge \underline{uniformément} sur $T$\/ vers $f$\/ et si chaque fonction $f_n$\/ admet une limite finie $b_n$\/ en $a$, alors la suite de réels $b_n$\/ converge vers un réel $b$, et $\lim_{t\to a} f(t) = b$. Autrement dit, \[
		\lim_{t\to a} \Big(\underbrace{\lim_{n\to +\infty} f_n(t)}_{f(x)}\Big) = \lim_{n\to +\infty} \Big(\underbrace{\lim_{t\to a} f_n(t)}_{b_n}\Big)
	.\] \qed
\end{thm}

\begin{rmkn}
	Le théorème de la double-limite \guillemotleft~contient~\guillemotright\ le théorème 6 (théorème de préservation/transmission de la continuité), c'est un cas particulier. En effet, si les fonctions $f_n$\/ sont continues, alors \[
		\lim_{x \to a}f(x) = \underbrace{\lim_{n\to +\infty} f_n(a)}_{f(a)}
	.\]
\end{rmkn}


		\section{Endomorphismes adjoints}

\begin{defn}
	On dit qu'un endomorphisme $f : E \to E$\/ est \textit{autoadjoint} si \[
		\forall (\vec{u}, \vec{v}) \in E^2,\quad \big<f(\vec{u})\:\big|\:\vec{v}\big> = \big<\vec{u}\:\big|\:f(\vec{v})\big>
	.\] 
\end{defn}

Un endomorphisme autoadjoint est aussi appelé endomorphisme \textit{symétrique} (\textit{c.f.}\ proposition suivante). L'ensemble des endomorphismes autoadjoints est noté $\mathcal{S}(E)$.

\begin{prop}
	Un endomorphism est autoadjoint si, et seulement si la matrice de $F$\/ dans une base \ul{orthonormée} $\mathcal{B}$\/ est orthogonale.
	Autrement dit : \[
		f \in \mathcal{S}(E) \iff \big[\:f\:\big]_\mathcal{B} \in \mathcal{S}_n(\R)
	.\]
\end{prop}

\begin{prv}
	\begin{description}
		\item[$\implies$] Soit $\mathcal{B} = (\vec{\varepsilon}_1, \ldots, \vec{\varepsilon}_n)$\/ une base orthonormée de $E$. Ainsi, \[
				\forall i,\:\forall j,\quad \big<f(\vec{\varepsilon}_i)\:\big|\: \vec{\varepsilon}_j\big> = \big<\vec{\varepsilon}_i\:\big|\:f(\vec{\varepsilon}_j)\big>
			.\] On pose $\big[\:f\:\big]_{\mathcal{B}} = (a_{i,j})$\/ : \[
				\begin{pNiceMatrix}[last-col,last-row]
					\quad&\quad&a_{1,j}&\quad&\quad&\vec{\varepsilon}_i\\
					&&&&&\\
					\quad&\quad&a_{i,j}&\quad&\quad&\vec{\varepsilon}_i\\
					&&&&&\\
					\quad&\quad&a_{n,j}&\quad&\quad&\vec{\varepsilon}_n\\
					f(\vec{\varepsilon}_i)&&f(\vec{\varepsilon}_j)&&f(\vec{\varepsilon}_n)\\
				\end{pNiceMatrix}
			.\] Ainsi, $f(\vec{\varepsilon}_j) = a_{1,j} \vec{\varepsilon}_1 + \cdots + a_{i,j} \vec{\varepsilon}_i + \cdots + a_{n,j} \vec{\varepsilon}_n$. D'où, $\left<\vec{\varepsilon}_i  \mid f(\vec{\varepsilon}_j) \right> = a_{i,j}$\/ car la base $\mathcal{B}$\/ est orthonormée.
			De même avec l'autre produit scalaire, $\left< f(\vec{\varepsilon}_i)  \mid \vec{\varepsilon}_j \right>$, d'où $a_{i,j} = a_{j,i}$\/ par symétrie du produit scalaire. On en déduit que $\big[\:f\:\big]_\mathcal{B} \in \mathcal{S}_n(\R)$.
		\item[$\impliedby$]
			Si $\big[\: f\:\big]_\mathcal{B} \in \mathcal{S}_n(\R)$, alors $\left<f(\vec{\varepsilon}_i)  \mid \vec{\varepsilon}_j \right> = \left<\vec{\varepsilon}_i  \mid f(\vec{\varepsilon}_j)\right>$.
			Or, on pose $\vec{u} = x_1 \vec{\varepsilon}_1+ \cdots + x_n \vec{\varepsilon}_n$, et $\vec{v} = y_1 \vec{\varepsilon}_1 + \cdots + y_n \vec{\varepsilon}_n$.
			\begin{align*}
				\left<f(\vec{u})  \mid \vec{v} \right> &= \left<x_1 f(\vec{\varepsilon}_1) + \cdots + x_n f(\vec{\varepsilon}_n)  \mid y_1 f(\vec{\varepsilon}_1) + \cdots + y_n f(\vec{\varepsilon}_n) \right> \\
				&= \Big<\sum_{i=1}^n x_i f(\vec{\varepsilon}_i)\:\Big|\: \sum_{j=1}^n y_j \vec{\varepsilon}_j\Big> \\
				&= \sum_{i,j \in \llbracket 1,n \rrbracket}  x_i y_j \left<f(\vec{\varepsilon}_i) \mid \vec{\varepsilon}_j \right>\\
			\end{align*}
			De même en inversant $\vec{u}$\/ et $\vec{v}$.
			On en déduit donc $\left<f(\vec{u} \mid \vec{v} \right> = \left<\vec{u}  \mid f(\vec{v}) \right>$.
	\end{description}
\end{prv}

\begin{exo}
	\begin{enumerate}
		\item Si $f$\/ est autoadjoint, montrons que $\Ker f \perp \Im f$, et $\Ker f \oplus \Im f$.
			On suppose $\forall \vec{u}$, $\forall \vec{v}$, $\left<f(\vec{u}) \mid \vec{v} \right> = \left<\vec{u}  \mid f(\vec{v}) \right>$.
			Soit $\vec{u} \in \Ker f$, et soit $\vec{v} \in \Im f$.
			On sait que $f(\vec{u}) = \vec{0}$, et qu'il existe $\vec{x} \in E$\/ tel que $\vec{v} = f(\vec{x})$.
			Ainsi, \[
				\left<\vec{u}  \mid \vec{v} \right>
				= \left<\vec{u}  \mid f(\vec{x}) \right>
				= \left<f(\vec{u})  \mid \vec{x} \right>
				= 0
			.\] 
			D'où $\vec{u} \perp \vec{v}$. Ainsi, $\Ker f \perp \Im f$.


			De plus, $E$\/ est de dimension finie, d'où, d'après le théorème du rang, \[
				\dim \Ker f + \dim \Im f = \dim E
			.\] Aussi, $\Ker f \oplus (\Ker f)^\perp = E$, donc $\dim(\Ker f) + \dim(\Ker f)^\perp = \dim E$.
			On en déduit donc que $\dim(\Im f)= \dim(\Ker f)^\perp$.
			Or, $\Im f \subset (\Ker f)^\perp$\/ car $\Im f \perp \Ker f$.
			Ainsi $\Im f = (\Ker f)^\perp$, on en déduit que \[
				\Im f \oplus \Ker f = E
			.\]
			\begin{description}
				\item[$\impliedby$] 
					Soit $p$\/ la projection sur $F$\/ parallèlement à $G$.
					Supposons l'endomorphisme $P$\/ autoadjoint.
					D'après la question 1., le $\Ker p \perp \Im p$.
					Ainsi, $F = \Im p$\/ et $G = \Ker p$.
					D'où, $F \perp G$, $p$\/ est donc une projection orthogonale.
				\item[$\implies$]
					Réciproquement, supposons $p$\/ une projection orthogonale.
					Soit $\mathcal{B} = (\vec{\varepsilon}_1, \ldots, \vec{\varepsilon}_q)$\/ une base orthonormée de $F$.
					Ainsi, pour tout $\vec{x} \in E$, \[
						p(\vec{x}) = \sum_{i = 1}^q \left<\vec{x}  \mid \vec{\varepsilon}_i \right>\,\vec{\varepsilon}_i
					.\] 
					On veut montrer que l'endomorphisme $p$\/ est autoadjoint.
					Soient $\vec{u}$\/ et $\vec{v}$\/ deux vecteurs de $E$.
					\begin{align*}
						\left<p(\vec{u})  \mid \vec{v} \right>
						= \Big<\sum_{i=1}^q \left< \vec{u}\mid \vec{\varepsilon}_i \right>\vec{\varepsilon}_i\:\Big|\; \vec{v}\;\Big>
						&= \sum_{i=1}^q \left<u  \mid \vec{\varepsilon}_{i} \right>\: \left< \vec{\varepsilon}_i  \mid v\right>\\
						&= \sum_{i=1}^q \left<v  \mid \vec{\varepsilon}_i \right>\:\left<\vec{\varepsilon}_i   \mid u\right> \\
						&= \left< \vec{u}  \mid p(\vec{v}) \right> \\
					\end{align*}

					Autre méthode, pour tous vecteurs $\vec{u}$\/ et $\vec{v}$\/ de $E$,
					\begin{align*}
						\left<p(\vec{u})  \mid \vec{v} \right>
						&= \left<p(\vec{u})  \mid p(\vec{v}) + \vec{v} - p(\vec{v}) \right> \\
						&= \left< p(\vec{u})  \mid p(\vec{v}) \right> + \left<p(\vec{u})  \mid  \vec{v} - p(\vec{v}) \right> \\
						&= \left<p(\vec{u}) \mid p(\vec{v}) \right> + \left<u - p(\vec{u})  \mid p(\vec{v}) \right> \\
						&= \left<\vec{u}  \mid p(\vec{v}) \right> \\
					\end{align*}
					car $p$\/ est orthogonale.
			\end{description}
	\end{enumerate}
\end{exo}


\begin{prop-defn}
	Si $f$\/ est un endomorphisme d'un espace euclidien $E$, alors il existe un unique endomorphisme de $E$, noté $f^\star$\/ et appelé l'\textit{adjoint} de $f$, tel que \[
		\forall (\vec{u},\vec{v}) \in E^2,\quad\quad \left<f^\star(\vec{u})  \mid \vec{v} \right> =  \left<\vec{u}  \mid f(\vec{v}) \right>
	.\] 
	Si $A$\/ est la matrice $f$\/ dans une base orthonormée $\mathcal{B}$\/ de $E$, alors $A^\top$\/ est la matrice de $f^\star$\/ dans~$\mathcal{B}$\/ : \[
		\big[\:f\:\big]_\mathcal{B} = \big[\:f\:\big]_\mathcal{B}^\top
	.\]
\end{prop-defn}

\begin{prv}
	Soit $\vec{u} \in E$. L'application \begin{align*}
		\varphi: E &\longrightarrow \R \\
		\vec{v} &\longmapsto \left<\vec{u}  \mid f(\vec{v}) \right>.
	\end{align*}
	La forme $\varphi$\/ est linéaire car $\varphi(\alpha_1 \vec{v}_1 + \alpha_2 \vec{v}_2) = \left<\vec{u} \mid f(\alpha_1 \vec{v}_1 + \alpha_2 \vec{v}_2) \right> = \left<\vec{u}  \mid \alpha_1 f(\vec{v}_1) + \alpha_2 f(\vec{v}_2) \right> = \alpha_1\left<\vec{u}  \mid  f(\vec{v}) \right> + \alpha_2 \left<\vec{u}  \mid f(\vec{v}_2) \right> = \alpha_1 \varphi(\vec{v}_1) + \alpha_2 \varphi(\vec{v}_2)$.
	D'où, d'après le théorème de \textsc{Riesz}, il existe un \ul{unique} vecteur $\vec{a} \in E$\/ tel que $\varphi(\vec{v}) = \left<\vec{a}  \mid \vec{v} \right>$\/ pour tout $\vec{v} \in E$.
	Ainsi, pour tout vecteur $\vec{v} \in E$, $\left<\vec{u}  \mid f(\vec{v}) \right> = \left<\vec{a}  \mid \vec{v} \right>$.
	On note $\vec{a} = f^\star(\vec{u})$.
	Soit l'application \begin{align*}
		f^\star : E &\longrightarrow E \\
		\vec{u} &\longmapsto f^\star(\vec{u}).
	\end{align*}
	La démonstration telle que $f^\star $\/ est linéaire est dans le poly.
	L'application $f^\star$\/ vérifie : $\left< \vec{u} \mid f(\vec{v}) \right> = \left<f^\star (\vec{u})  \mid \vec{v} \right>$, pour tous vecteurs $\vec{u}$\/ et $\vec{v}$.
	\textsl{Quelle est la matrice de $f^\star$, dans une base orthonormée ?}\@
	Soit $\mathcal{B}$\/ une base orthonormée de $E$, et soient $A = \big[\:f\:\big]_\mathcal{B}$, $B = \big[\:f^\star \:\big]_\mathcal{B}$, $U = \big[\:\vec{u}\:\big]_\mathcal{B}$, et $V = \big[\:\vec{v}\:\big]_\mathcal{B}$.
	Les matrices $U$\/ et $V$\/ sont des vecteurs colonnes, et $A$\/ et $B$\/ sont des matrices carrées.
	Ainsi, \[
		U^\top \cdot A \cdot V = \left< \vec{u}  \mid f(\vec{v}) \right> 
		= \left<f^\star (\vec{u})  \mid \vec{v} \right> = (B\cdot U)^\top \cdot V,
	\] ce qui est vrai quelque soit les vecteurs colonnes $U$\/ et $V$.
	D'où, $\forall U$, $\forall V$, $U^\top \cdot \big(A \cdot V\big) = U^\top \cdot \big(B^\top \cdot V\big)$.
	Ainsi, pour tous vecteurs $U$\/ et $V$, \[
		U^\top \cdot \Big[ (AV) - (B^\top V)\Big] = 0
	.\] En particulier, si $U = (AV) - (B^\top V)$, le produit scalaire $\left<\vec{u}  \mid \vec{u} \right>$\/ est nul, donc $U = 0$.
	Ainsi, \[
		\forall V,\quad A\cdot V = B^\top\cdot  V
	.\] De même, on conclut que $A = B^\top$. On en déduit donc que \[
	\big[\:f^\star\:\big]_\mathcal{B} = \big[\:f\:\big]_\mathcal{B}^\top
	.\]
\end{prv}

Les propriétés suivantes sont vrais :
\begin{itemize}
	\item $(f  \circ g)^\star  = g^\star \circ f^\star$, \quad $(f^\star)^\star = f$, \quad et \quad $(\alpha f + \beta g)^\star  = \alpha f^\star + \beta g^\star $\/ ;
	\item $(A\cdot B)^\top  = B^\top \cdot A^\top$, \quad $(A^\top)^\top = A$, \quad et \quad $(\alpha A + \beta B)^\top = \alpha A^\top+ \beta B^\top $.
\end{itemize}
Des deuxièmes et troisièmes points,  il en résulte que les applications $f \mapsto f^\star$, et $A \mapsto A^\top$\/ sont des applications involutives.



		\begin{prop-defn}
  Soit $(\Omega, \mathcal{A}, P)$\/ un espace probabilisé, et soit $X$\/ une \textit{vard}. Si $X^2$\/ est d'espérance finie, alors $X$\/ aussi, et on appelle \textit{variance} le réel positif \[
    \mathrm{V}(X) = \mathrm{E}\Big(\big[X - \mathrm{E}(X)\big]^2\Big) = \underbrace{\mathrm{E}(X^2) - \big(\mathrm{E}(X)\big)^2}_{\mathclap{\text{Relation de \textsc{König \& Huygens}}}} \ge 0
  .\]
  L'\textit{écart-type} $\sigma(X)$\/ est la racine carrée de la variance : \[
    \sigma(X) = \sqrt{\mathrm{V}(X)}
  .\]
\end{prop-defn}

\begin{prv}
  On pose $\mu = \mathrm{E}(X)$, et on a $[X - \mu]^2 = X^2 - 2 \mu X + \mu^2$. D'où, par linéarité de l'espérance,
  \begin{align*}
    \mathrm{E}\big((X-\mu)^2\big)
    &= \mathrm{E}(X^2 - 2\mu X + \mu^2) \\
    &= \mathrm{E}(X^2) - 2\mu \mathrm{E}(X) + \mu^2 \\
    &= \mathrm{E}(X^2) - 2\mu^2 + \mu^2 \\
    &= \mathrm{E}(X^2) - \big(\mathrm{E}(X)\big)^2. \\
  \end{align*}
  De plus, d'après le lemme précédent, si $X^2$\/ est d'espérance finie, alors $X$\/ est d'espérance finie.
\end{prv}

\begin{rmk}
  \begin{enumerate}
    \item La variance mesure la \textit{dispersion}, ou l'\textit{étalement} des valeurs $a_i$\/ autour de l'espérance $\mathrm{E}(X)$. En particulier, s'il existe $a \in \R$\/ tel que $P(X = a) = 1$, alors $\mathrm{E}(X) = a$\/ et $\mathrm{V}(X) = 0$. (C'est même une équivalence.)
    \item Si la variable $X$\/ a une unité ($\mathrm{km}/\mathrm{s}$, $\mathrm{V}/\mathrm{m}$, etc.), alors l'écart type a la même unité (d'où l'intérêt de calculer la racine carrée de la variance).
    \item Soient $\alpha$\/ et $\beta$\/ deux réels. Si  $X^2$\/ est d'espérance finie, alors \[
      \mathrm{V}(\alpha X + \beta) = \alpha^2\cdot  \mathrm{V}(X)
    .\]
    (Une translation ne change pas la dispersion des valeurs, et multiplier par un réel multiplie l'espérance, mais aussi la dispersion, d'où le carré.)
  \end{enumerate}
\end{rmk}

\begin{exo}
  \textsl{Montrer que
  \begin{enumerate}
    \item si $X \sim \mathcal{B}(n, p)$, alors $X^2$\/ est d'espérance finie et $\mathrm{V}(X) = n\cdot p\cdot q$.
    \item si $T \sim \mathcal{G}(p)$, alors $T^2$\/ est d'espérance finie et $\mathrm{V}(T) = \frac{q}{p^2}$.
    \item si $X \sim \mathcal{P}(\lambda)$, alors $X^2$\/ est d'espérance finie et $\mathrm{V}(X) = \lambda$.
  \end{enumerate}
  }

  \begin{enumerate}
    \item
      Si $X \sim \mathcal{B}(n,p)$, alors $X(\Omega) = \llbracket 0,n \rrbracket$\/ et, pour $k \in X(\Omega)$, $P(X = k) = {n\choose k}\,p^k\,q^{n-k}$.
      On a déjà montré que $\mathrm{E}(X) = n\cdot p$.
      On va montrer que $\mathrm{V}(X) = n\,p\,q$.
      La variable aléatoire $X^2$\/ est d'espérance finie car $X(\Omega)$\/ est fini.
      Et,
      \begin{align*}
        \mathrm{E}(X^2) &= \sum_{k=0}^n k^2\: P(X = k)\\
        &= \sum_{k=0}^n k^2 {n\choose p} p^k q^{n-k} \\
        &= \ldots \\
      \end{align*}
      En effet, d'après la ``petite formule,'' on a \[
        \forall k \ge 1,\quad k{n\choose k} = n{n-1 \choose k-1}
      \] d'où, $(k-1) {n-1\choose k-1} = (n-1) \choose {n-2 \choose k-2}$. Ainsi, \[
        \forall k \ge 2,\quad k(k-1){n\choose k} = n(n+1) {n-2\choose k-2}
      .\] 
    \item Si $T \sim \mathcal{G}(p)$, alors $T(\Omega) = \N^*$\/ et $\forall k \in T(\Omega)$, $P(T = k) = p \times q^{k-1}$.
      On a déjà prouvé que $\mathrm{E}(T) = \frac{1}{p}$.
      On veut montrer que $\mathrm{V}(T) = \frac{q}{p^2}$. Montrons que la variable $T^2$\/ possède une espérance : la série $\sum k^2\: P(T = k)$\/ converge absolument car $k^2 \:P(T = k) = k^2 \cdot p \cdot q^{k-1}$.
      Or, pour $k \ge 2$, $\frac{\mathrm{d}^2}{\mathrm{d}x^2} x^k = k(k-1)\,x^{k-2}$. Et, on peut dériver terme à terme une série entière sans changer son rayon de convergence, et la série $\sum x^k$\/ a pour rayon de convergence 1. D'où, $\sum k(k-1)\, x^{k-2}$\/ a pour rayon de convergence 1. Or, $q \in {]0,1[} \subset {]-1,1[}$\/ donc la série $\sum k(k-1)q^{k-2}$\/ converge. De plus, $\sum k (k-1)\, q^{k-2} = \sum k^2\,q^{k-2} - \sum k\,q^{k-2}$.
      D'où,  $\sum k^2 q^{k-2} = \sum k(k-1)\, q^{k-2} + \sum k\,q^{k-2}$, qui converge. Par suite,
      \begin{align*}
        \sum_{k=1}^\infty k^2\, P(T = k) &= \sum_{k=1}^\infty k^2\,p\,q^{k-1}\\
        &= p + pq \sum_{k=2}^\infty k^2 q^{k-2} \\
        &= p + pq \sum_{k=2}^\infty k(k-1)\,q^{k-2} + p\sum_{k=2}^\infty k\,q^{k-1} \\
        &= p + pq\, \frac{2}{(1-q)^3} + p\left(\frac{1}{(1-q)^2} - 1 \right) \rlap{\quad\quad \text{\textit{c.f.} en effet après}}\\
        &= p + pq\, \frac{2}{p^3} + p\left( \frac{1}{p^2} - 1 \right) \\
        &= \frac{2q}{p^2} + \frac{1}{p} \\
        &= \frac{2q + p}{p^2} \\
        &= \frac{2q + (1-q)}{p^2} \\
        &= \frac{q+1}{p^2}. \\
      \end{align*}
      En effet, $\forall x \in {]-1,1[}$, $\sum_{k=0}^\infty x^k = \frac{1}{1-x}$. D'où, pour $x \in {]-1,1[}$, \[
        \sum_{k=1}^\infty k\,x^{k-1} = \frac{1}{(1-x)^2}
        \quad \text{ et }\quad
        \sum_{k=2}^\infty k(k-1)\,x^{k-2} = \frac{2}{(1-x)^3}
      .\]
      Ainsi, $\mathrm{E}(T^2) = \frac{q^{+1}}{p^2}$. D'où
      \begin{align*}
        \mathrm{V}(T) &= \mathrm{E}(T^2) - \big(\mathrm{E}(T)\big)^2 \\
        &= \frac{q+1}{p^2} - \left( \frac{1}{p} \right)^2 \\
        &= \frac{q}{p^2} \\
      \end{align*}
    \item À tenter
  \end{enumerate}
\end{exo}


\section{Les inégalités de \textsc{Markov} et de \textsc{Bienaymé}--\textsc{Tchebychev}, inégalités de concentration}

\begin{lem}[Markov]
  Soit $(\Omega, \mathcal{A}, P)$\/ un espace probabilisé, et soit $X$\/ une variable aléatoire \underline{positive}.
  Si $X$\/ est d'espérance finie, alors \[
    \forall a > 0, \quad P(X \ge a) \le \frac{\mathrm{E}(X)}{a}
  .\]
\end{lem}

\begin{prv}
  On suppose $X$\/ d'espérance finie. Ainsi, on a \[
    \mathrm{E}(X) = \sum_{x \in X(\Omega)} x\:P(X = x)
  .\]
  Soit $I$\/ l'ensemble $I = \{x \in X(\Omega) \mid x \ge a\}$.
  Alors, \[
    \mathrm{E}(X) = \underbrace{\sum_{x \in I} x\:P(X = x)}_{\text{ ici } x \ge a} + \underbrace{\sum_{x \in X(\Omega) \setminus I} x\:P(X = x)}_{\ge 0 \text{ par hypothèse}}
  .\]
  D'où, \[
    \mathrm{E}(X) \ge \sum_{x \in I} x\:P(X = x) \ge \sum_{x \in I} a\: P(X = x) = a \sum_{x \in I} P(X = x) \ge a\:P(x \ge a)
  .\]
\end{prv}

\begin{prop}[\textsc{Bienaymé--Tchebychev}]
  Soit $(\Omega, \mathcal{A}, P)$\/ un espace probabilisé, et soit $X$\/ une \textit{vard}. Si $X^2$\/ est d'espérance finie, alors \[
    \forall a > 0, \quad\quad P\Big(\big|X - \mathrm{E}(X)\big| \ge a\Big) \le \frac{\mathrm{V}(X)}{a^2}
  .\]
\end{prop}

\begin{prv}
  On pose $\mu = \mathrm{E}(X)$.
  L'événement $\big(|X - \mu| \ge a\big) = \big((X - \mu)^2 \ge a^2\big)$, d'où, les probabilités \[
    P\big(|X - \mu| \ge a\big) = P\big(\underbrace{(X - \mu)^2}_{\ge 0} \ge \underbrace{a^2}°{\ge 0}\big).
  \] On valide donc \textit{une} des hypothèses de l'inégalité de \textsc{Markov}.
  De plus, l'autre hypothèse est vérifiée : $X^2$\/ est d'espérance finie, donc $(X - \mu)^2$\/ aussi. On en déduit, d'après le lemme de \textsc{Markov}, que \[
    P\big((X-\mu)^2 \ge a^2\big) \le \frac{\mathrm{E}\big((X-\mu)^2\big)}{a^2} = \frac{\mathrm{V}(X)}{a^2}
  .\]
\end{prv}

\section{Série génératrice}

\begin{defn}
  Soit $X$\/ une \textit{vad} telle que $X(\Omega) \subset \N$. La \textit{série génératrice} de $X$\/ est la série entière $\sum a_n x^n$ de coefficients $a_n = P(X = n)$.
\end{defn}


La série $\sum a_n$\/ converge car sa somme vaut $\sum_{n=0}^\infty a_n = 1$. D'où, 
\begin{itemize}
  \item le rayon de convergence $R$\/ de la série est supérieur ou égal à 1.
  \item la série génératrice converge normalement sur $[-1,1]$, car la série $\sum |a_n|$\/ converge, or, $\forall x \in [-1,1]$, $|p_nt^n| \le |p_n|$, d'où la convergence normale.
    D'où la \textit{fonction génératrice} \[
      \mathrm{G}_X \colon t \longmapsto \sum_{n=0}^\infty p_n t^n
    \] est définie et même continue sur $[-1,1]$, car la convergence est uniforme.
  \item la fonction génératrice $\mathrm{G}_X$\/ est de classe $\mathcal{C}^\infty$\/ sur $]-1,1[$\/ et \[
      \forall k \in \N,\quad P(X = k) = a_n \frac{{\mathrm{G}_X}^{(k)}(0)}{k!}
    .\] La fonction génératrice de $X$\/ permet donc de retrouver la loi de probabilité de $X$.
\end{itemize}


		\begin{exo}
	Soient $\lambda_1, \ldots, \lambda_r \in \R$\/ distincts deux à deux.
	Montrons que, si $\forall x \in \R$, $\alpha_1 \mathrm{e}^{\lambda_1 x} + \cdots + \alpha_r \mathrm{e}^{\lambda_r x} = 0$, alors $\alpha_1 = \cdots = \alpha_r$.
	On peut procéder de différentes manières : le déterminant de {\sc Vandermonde}, par analyse-sythèse, ou, en utilisant \[
		\frac{\mathrm{d}}{\mathrm{d}x}\left( \mathrm{e}^{\lambda_k x} \right) = \lambda_k \mathrm{e}^{\lambda_k x},\quad\text{d'où}\quad\varphi(f_k) = \lambda_k f_k, \text{ avec } f_k : x \mapsto \mathrm{e}^{\lambda_k x}\quad\text{et}\quad\varphi:f\mapsto f'
	.\]
	On doit vérifier que les $f_k$\/ sont des vecteurs et l'application $\varphi$\/ soit un endomorphisme. On se place donc dans l'espace vectoriel $\mathscr{C}^{\infty}$. (On ne peut pas se placer dans l'espace $\mathscr{C}^k$, car sinon l'application $\varphi$\/ est de l'espace $\mathscr{C}^k$\/ à $\mathscr{C}^{k-1}$, ce n'est donc pas un endomorphisme ; ce n'est pas le cas pour l'espace $\mathscr{C}^\infty$.)
	Or, les $\lambda_k$\/ sont distincts deux à deux d'où les vecteurs propres $f_k$\/ sont linéairement indépendants. Et donc si $\alpha_1 f_1 + \alpha_2 f_2 + \cdots + \alpha_r f_r = 0$\/ alors $\alpha_1 = \cdots = \alpha_r=0$.
	Mais, comme $\forall x \in \R$, $\alpha_1 f_1(x) + \alpha_2 f_2(x) + \cdots + \alpha_r f_r(x) = 0$, on en déduit que \[
		\boxed{\alpha_1 = \cdots = \alpha_r = 0.}
	\]
\end{exo}

\section{Critères de diagonalisabilité}

\begin{prop}[une condition \underline{suffisante} pour qu'une matrice soit diagonalisable]
	Soit $A$\/ une matrice carrée de taille $n \ge 2$. {\color{red}Si} $A$\/ possède $n$\/ valeurs propres distinctes deux à deux, {\color{red}alors} $A$\/ est diagonalisable.
\end{prop}

\begin{rmkn}
	La réciproque est fausse : par exemple, pour $n > 1$, $7 I_n$\/ est diagonalisable car elle est diagonale. Mais, elle ne possède pas $n$\/ valeurs propres distinctes deux à deux.
\end{rmkn}

\begin{prv}
	On suppose que la matrice $A \in \mathscr{M}_{n,n}(\mathds{K})$\/ possède $n$\/ valeurs propres distinctes deux à deux (i.e.~$\Card \Sp(A) = n$). D'où, d'après la proposition 16, les $n$\/ vecteurs propres associés $\varepsilon_1,\ldots,\varepsilon_n$\/ sont libres. D'où $(\varepsilon_1, \ldots, \varepsilon_n)$\/ est une base formée de vecteurs propres. Donc, d'après la définition 5, la matrice $A$\/ est diagonalisable.
\end{prv}

\begin{thm}[conditions \underline{nécessaires et suffisantes} pour qu'une matrice soit diagonalisable]
	Soient $E$\/ un espace vectoriel de dimension finie et $u : E \to E$\/ un endomorphisme.
	Alors,
	\begin{align*}
		(1)\quad u \text{ diagonalisable } \iff& E = \bigoplus_{\lambda \in \Sp(u)} \Ker(\lambda\id_E - u) \quad(2)\\
		\iff& \dim E = \sum_{\lambda \in \Sp(u)} \dim(\mathrm{SEP}(\lambda))\quad(3)\\
		\iff& \chi_u \text{ scindé et } \forall \lambda \in \Sp(u),\:\dim(\mathrm{SEP}(\lambda)) = m_\lambda\quad(4)
	\end{align*}
	où $m_\lambda$\/ est la multiplicité de la racine $\lambda$\/ du polynôme $\chi_u$.
\end{thm}

\begin{prv}
	\begin{itemize}
		\item[``$(1)\implies(2)$''] On suppose $u$\/ diagonalisable. Il existe donc une base $(\varepsilon_1$, \ldots, $\varepsilon_n)$\/ de $E$\/ formée de vecteurs propres de $u$. On les regroupes par leurs valeurs propres : $(\varepsilon_i, \ldots, \varepsilon_{i+j})$\/ forme une base de $\mathrm{SEP(\lambda_k)}$. D'où la base $(\varepsilon_1, \ldots, \varepsilon_n)$\/ de l'espace vectoriel $E$\/ est une concaténation des bases des sous-espaces propres de $u$. D'où \[
				E = \bigoplus_{\lambda \in \Sp(u)} \mathrm{SEP}(\lambda)
			.\]
		\item[``$(2)\implies(1)$''] On suppose que $E = \mathrm{SEP}(\lambda_1) \oplus \mathrm{SEP}(\lambda_2) \oplus \cdots \oplus \mathrm{SEP}(\lambda_r)$.
			Soient $(\varepsilon_1, \ldots, \varepsilon_{d_1})$\/ une base de $\mathrm{SEP}(\lambda_1)$, $(\varepsilon_{d_1 + 1}, \ldots, \varepsilon_{d_1 + d_2})$\/ une base de $\mathrm{SEP}(\lambda_2)$, \ldots, $(\varepsilon_{d_1+\cdots + d_{r-1}+1}$, \ldots, $\varepsilon_{d_1+ \cdots + d_r})$\/ une base de $\mathrm{SEP}(\lambda_r)$.
			En concaténant ces base, on obtient une base de $E$, d'après l'hypothèse. Dans cette base, tous les vecteurs sont propres donc $u$\/ est diagonalisable.
		\item[``$(2)\implies(3)$''] On suppose $E = \mathrm{SEP}(\lambda_1) \oplus \mathrm{SEP}(\lambda_2) \oplus \cdots \oplus \mathrm{SEP}(\lambda_r)$. D'où  \[
					\dim E = \dim(\mathrm{SEP}(\lambda_1)) + \dim(\mathrm{SEP}(\lambda_2)) + \cdots + \dim(\mathrm{SEP}(\lambda_r))
			\] car la dimension d'une somme directe est égale à la somme des dimensions.
		\item[``$(3)\implies(1)$''] On suppose $\dim E = \dim(\mathrm{SEP}(\lambda_1)) + \dim(\mathrm{SEP}(\lambda_2)) + \cdots + \dim(\mathrm{SEP}(\lambda_r))$. Or, les sous-espaces propres sont en somme directe, d'après la proposition 16. D'où $\dim\Big(\sum_{\lambda \in \Sp(u)} \mathrm{SEP}(\lambda) \Big)= \sum_{\lambda \in \Sp(u)} \dim(\mathrm{SEP}(\lambda))$. Donc $\sum_{\lambda \in \Sp(u)} \mathrm{SEP}(\lambda) = E$.
		\item[``$(4)\implies(3)$''] On suppose (a) $\chi_u$\/ scindé et (b) $\dim(\mathrm{SEP}(\lambda)) = m_\lambda$. D'où, d'après (a): \[
				\chi_u(x) = (x - \lambda_1)^{m_{\lambda_1}}(x - \lambda_2)^{m_{\lambda_2}} \cdots (x - \lambda_r)^{m_{\lambda_r}} = x^n + \cdots
			\] d'où $m_{\lambda_1} + m_{\lambda_2} + \cdots + m_{\lambda_r} = n$, et d'où \[
				\dim(\mathrm{SEP}(\lambda_1)) + \dim(\mathrm{SEP}(\lambda_2)) + \cdots + \dim(\mathrm{SEP}(\lambda_r)) = n
			\] d'après l'hypothèse (b).
		\item[``$(1)\implies(4)$'']
			On suppose $u$\/ diagonalisable. D'où, dans une certaine base $\mathscr{B}$, la matrice $\big[u\big]_\mathscr{B}$\/ est diagonale. Quitte à changer l'ordre des éléments de $\mathscr{B} = (\varepsilon_1,\ldots,\varepsilon_r)$, on peut supposer que $\big[u\big]_\mathscr{B}$\/ est de la forme \[
				\big[u\big]_\mathscr{B} = 
				\begin{bNiceArray}{c|c|c|c}[last-col]
					\begin{array}{cccc}\lambda_1\\&\lambda_1\\&&\ddots\\&&&\lambda_1\end{array}&0&0&0&\begin{array}{l}\varepsilon_1\\\varepsilon_2\\\vdots\\\varepsilon_{d_1}\\\end{array}\\ \hline
					0&\begin{array}{cccc}\lambda_2\\&\lambda_2\\&&\ddots\\&&&\lambda_2\end{array}&0&0&\begin{array}{l}\varepsilon_{d_1+1}\\\varepsilon_{d_1+2}\\\vdots\\\varepsilon_{d_1+d_2}\\\end{array}\\ \hline
					 &&\ddots&&\vdots\\ \hline
					0&0&0&0\begin{array}{cccc}\lambda_r\\&\lambda_r\\&&\ddots\\&&&\lambda_r\end{array}&\begin{array}{l}\varepsilon_{d_1+\cdots + d_{r-1} + 1}\\\varepsilon_{d_1+\cdots + d_{r-1}+2}\\\vdots\\\varepsilon_{d_1 + \cdots + d_r}\\\end{array}\\
				\end{bNiceArray}
			.\] D'où $\forall k \in \left\llbracket 1,r \right\rrbracket$, $d_k = \dim(\mathrm{SEP}(\lambda_k))$. En outre, $\chi_u(x) = \det(x\id - u) = (x - \lambda_1)^{d_1}\cdot (x-\lambda_2)^{d_2} \cdots (r - \lambda_r)^{d_r}$.
			D'où $\forall k \in \left\llbracket 1,r \right\rrbracket$, $d_k = m_{\lambda_k}$\/ et $\chi_u$\/ est scindé.
	\end{itemize}
\end{prv}

\begin{exo}
	{\slshape On considère la matrice $E$\/ ci-dessous \[
		E = \begin{pmatrix}
			7&0&1\\
			0&3&0\\
			0&0&7
		\end{pmatrix}.
	\] La matrice $E$\/ ci-dessous est-elle diagonalisable ?}

	Soit $\lambda \in \R$. On sait que $\lambda \in \Sp(E)$\/ si et seulement si $\det(\lambda I_3 - E) = 0$. Or \[
		\det(\lambda I_3 - E) =
		\begin{vmatrix}
			\lambda - 7&0&-1\\
			0&\lambda-3&0\\
			0&0&\lambda - 7
		\end{vmatrix} = (\lambda - 7)^2\cdot  (\lambda - 3)^1
	.\] Donc $\Sp(E) = \{3,7\}$, $1 \le \dim\big(\mathrm{SEP}(3)\big) \le 1$, et $1 \le \dim\big(\mathrm{SEP}(7)\big) \le 2$.
	La matrice $E$\/ est diagonalisable si et seulement si $\dim(\mathrm{SEP}(3)) + \dim(\mathrm{SEP}(7)) = 3$, donc si et seulement si $\dim(\mathrm{SEP}(7)) = 2$. On cherche donc la dimension de ce sous-espace propre : soit $X = \left( \substack{x\\y\\z} \right) \in \mathscr{M}_{3,1}(\R)$. On sait que
	\begin{align*}
		X \in \mathrm{SEP}(7) \iff& E\cdot X = 7X\\
		\iff& \begin{pmatrix}
			7&0&1\\
			0&3&0\\
			0&0&7
		\end{pmatrix} \begin{pmatrix}
			x\\y\\z
		\end{pmatrix} = 7 \begin{pmatrix}
			x\\y\\z
		\end{pmatrix}\\
		\iff& \begin{cases}
			7x + 0y + 1z = 7x\\
			3y = 7y\\
			7z = 7z
		\end{cases}\\
		\iff& \begin{cases}
			z = 0\\
			y = 0
		\end{cases}\\
		\iff& X = \begin{pmatrix}
			x\\0\\y
		\end{pmatrix} = x \underbrace{\begin{pmatrix}
			1\\0\\0
		\end{pmatrix}}_{\varepsilon_1}
	\end{align*}
	Donc $\mathrm{SEP}(7) = \Vect(\varepsilon_1)$, d'où $\dim(\mathrm{SEP}(7)) = 1$. Donc la matrice $E$\/ n'est pas diagonalisable.
\end{exo}

\section{Trigonalisation}

Trigonaliser une matrice ne sert que si la matrice n'est pas diagonalisable.

\begin{defn}
	On dit d'une matrice carrée $A \in \mathscr{M}_{n,n}(\mathds{K})$\/ qu'elle est {\it trigonalisable}\/ s'il existe une matrice inversible $P$\/ telle que $P^{-1} \cdot A \cdot P$\/ est triangulaire : \[
		P^{-1} \cdot A \cdot P = \begin{pNiceMatrix}
			\lambda_1&\Block{2-2}*&\\
			\Block{2-2}0&\Ddots&\\
			&&\lambda_r
		\end{pNiceMatrix}
	.\]
\end{defn}

\begin{rmk}
	$\O$\/
\end{rmk}

\begin{thm}
	Une matrice carrée $A \in \mathscr{M}_{n,n}(\mathds{K})$\/ est trigonalisable si et seulement si son polynôme caractéristique $\chi_A \in \mathds{K}[X]$\/ est scindé.
\end{thm}

		\addrecap
	}
	\def\addmacros#1{#1}
}

{
	\chap[7]{Tentative de réponse à la \textbf{NP}-complétude}
	\minitoc
	\renewcommand{\cwd}{../cours/chap07/}
	\addmacros{
		\section{Motivation}

\lettrine On place au centre de la classe 40 bonbons. On en distribue un chacun. Si, par exemple, chacun choisit un bonbon et, au \textit{top} départ, prennent celui choisi.
Il est probable que plusieurs choisissent le même. Comme gérer lorsque plusieurs essaient d'accéder à la mémoire ?

Deuxièmement, sur l'ordinateur, plusieurs applications tournent en même temps. Pour le moment, on considérait qu'un seul programme était exécuté, mais, le \textsc{pc} ne s'arrête pas pendant l'exécution du programme.

On s'intéresse à la notion de \guillemotleft~processus~\guillemotright\ qui représente une tâche à réaliser.
On ne peut pas assigner un processus à une unité de calcul, mais on peut \guillemotleft~allumer~\guillemotright\ et \guillemotleft~éteindre~\guillemotright\ un processus.
Le programme allumant et éteignant les processus est \guillemotleft~l'ordonnanceur.~\guillemotright\@ Il doit aussi s'occuper de la mémoire du processus (chaque processus à sa mémoire séparée).

On s'intéresse, dans ce chapitre, à des programmes qui \guillemotleft~partent du même~\guillemotright\ : un programme peut créer un \guillemotleft~fil d'exécution~\guillemotright\ (en anglais, \textit{thread}). Le programme peut gérer les fils d'exécution qu'il a créé, et éventuellement les arrêter.
Les fils d'exécutions partagent la mémoire du programme qui les a créé.

En C, une tâche est représenté par une fonction de type \lstinline[language=c]!void* tache(void* arg)!. Le type \lstinline[language=c]!void*!\ est l'équivalent du type \lstinline[language=caml]!'a! : on peut le \textit{cast} à un autre type (comme \lstinline[language=c]-char*-).

\begin{lstlisting}[language=c,caption=Création de \textit{thread}s en C]
void* tache(void* arg) {
	printf("%s\n", (char*) arg);
	return NULL;
}

int main() {
	pthread_t p1, p2;

	printf("main: begin\n");

	pthread_create(&p1, NULL, tache, "A");
	pthread_create(&p2, NULL, tache, "B");

	pthread_join(p1, NULL);
	pthread_join(p2, NULL);

	printf("main: end\n");

	return 0;
}
\end{lstlisting}

\begin{lstlisting}[language=c,caption=Mémoire dans les \textit{thread}s en C]
int max = 10;
volatile int counter = 0;

void* tache(void* arg) {
	char* letter = arg;
	int i;

	printf("%s begin [addr of i: %p] \n", letter, &i);

	for(i = 0; i < max; i++) {
		counter = counter + 1;
	}

	printf("%s : done\n", letter);
	return NULL;
}

int main() {
	pthread_t p1, p2;

	printf("main: begin\n");

	pthread_create(&p1, NULL, tache, "A");
	pthread_create(&p2, NULL, tache, "B");

	pthread_join(p1, NULL);
	pthread_join(p2, NULL);

	printf("main: end\n");

	return 0;
}
\end{lstlisting}

Dans les \textit{thread}s, les variables locales (comme \texttt{i}) sont séparées en mémoire. Mais, la variable \texttt{counter} est modifiée, mais elle ne correspond pas forcément à $2 \times \texttt{max}$. En effet, si \texttt{p1} et \texttt{p2} essaient d'exécuter au même moment de réaliser l'opération \lstinline[language=c]-counter = counter + 1-, ils peuvent récupérer deux valeurs identiques de \texttt{counter}, ajouter 1, puis réassigner \texttt{counter}.
Ils \guillemotleft~se marchent sur les pieds.~\guillemotright\ 

Parmi les opérations, on distingue certaines dénommées \guillemotleft~atomiques~\guillemotright\ qui ne peuvent pas être séparées. L'opération \lstinline[language=c]-i++- n'est pas atomique, mais la lecture et l'écriture mémoire le sont.

\begin{defn}
	On dit d'une variable qu'elle est \textit{atomique} lorsque l'ordonnanceur ne l'interrompt pas.
\end{defn}

\begin{exm}
	L'opération \lstinline[language=c]-counter = counter + 1- exécutée en série peut être représentée comme ci-dessous. Avec \texttt{counter} valant 40, cette exécution donne 42.
	\begin{table}[H]
		\centering
		\begin{tabular}{l|l}
			Exécution du fil A & Exécution du fil B\\ \hline
			(1)~$\mathrm{reg}_1 \gets \texttt{counter}$ & (4)~$\mathrm{reg}_2 \gets \texttt{counter}$ \\
			(2)~$\mathrm{reg}_1{++}$ & (5)~$\mathrm{reg}_2{++}$ \\
			(3)~$\texttt{counter} \gets \mathrm{reg}_1$ & (6)~$\texttt{counter} \gets \mathrm{reg}_2$
		\end{tabular}
	\end{table}
	\noindent Mais, avec l'exécution en simultanée, la valeur de \texttt{counter} sera 41.
	\begin{table}[H]
		\centering
		\begin{tabular}{l|l}
			Exécution du fil A & Exécution du fil B\\ \hline
			(1)~$\mathrm{reg}_1 \gets \texttt{counter}$ & (2)~$\mathrm{reg}_2 \gets \texttt{counter}$ \\
			(3)~$\mathrm{reg}_1{++}$ & (5)~$\mathrm{reg}_2{++}$ \\
			(4)~$\texttt{counter} \gets \mathrm{reg}_1$ & (6)~$\texttt{counter} \gets \mathrm{reg}_2$
		\end{tabular}
	\end{table}
	\noindent Il y a \textit{entrelacement} des deux fils d'exécution.
\end{exm}

\begin{rmk}[Problèmes de la programmation concurrentielle]
	\begin{itemize}
		\item Problème d'accès en mémoire,
		\item Problème du rendez-vous,\footnote{Lorsque deux programmes terminent, ils doivent s'attendre pour donner leurs valeurs.}
		\item Problème du producteur-consommateur,\footnote{Certains programmes doivent ralentir ou accélérer.}
		\item Problème de l'entreblocage,\footnote{\textit{c.f.} exemple ci-après.}
		\item Problème famine, du dîner des philosophes.\footnote{Les philosophes mangent autour d'une table, et mangent du riz avec des baguettes. Ils décident de n'acheter qu'une seule baguette par personne. Un philosophe peut, ou penser, ou manger. Mais, pour manger, ils ont besoin de deux baguettes. S'ils ne mangent pas, ils meurent.}
	\end{itemize}
\end{rmk}

\begin{exm}[Problème de l'entreblocage]~

	\begin{table}[H]
		\centering
		\begin{tabular}{l|l|l}
			Fil A & Fil B & Fil C\\ \hline
			RDV(C) & RDV(A) & RDV(B)\\
			RDV(B) & RDV(C) & RDV(A)\\
		\end{tabular}
		\caption{Problème de l'entreblocage}
	\end{table}
\end{exm}

Comment résoudre le problème des deux incrementations ? Il suffit de \guillemotleft~mettre un verrou.~\guillemotright\ Le premier fil d'exécution \guillemotleft~s'enferme~\guillemotright\ avec l'expression \lstinline[language=c]!count++!, le second fil d'exécution attend que l'autre sorte pour pouvoir entrer et s'enfermer à son tour.


		\section{Continuité}

\begin{exm}
	Dans l'exercice 2, chaque fonction $f_n : t \mapsto t^n$\/ est continue sur $[0,1]$\/ mais la limite $f$\/ n'est pas continue sur $[0,1]$\/ (car elle n'est pas continue en $1$).
\end{exm}

\begin{thm}
	Soit $a$\/ un réel dans un intervalle $T$\/ de $\R$. Si une suite de fonctions $(f_n)_{n\in\N}$\/ continues en $a$\/ converge uniformément sur $T$\/ vers une fonction $f$, alors $f$\/ est aussi continue en $a$.
\end{thm}

\begin{prv}
	On suppose les fonctions $f_n$\/ continues en $a$\/ ($f_n(x) \longrightarrow f_n(a)$) et que la suite de fonctions $(f_n)_{n\in\N}$\/ converge uniformément vers $f$\/ ($\sup\:|f_n -f| \longrightarrow 0$). On veut montrer que $f$\/ est continue en $a$\/ : $f(x) \tendsto{x \to a} f(a)$, i.e.\ \[
		\forall \varepsilon > 0,\:\exists \delta > 0,\: \forall x \in T,\quad|x-a| \le \delta \implies |f(x) - f(a)| \le \varepsilon
	.\]
	Soit $\varepsilon > 0$. On calcule \[
		\big|f(x) - f(a)\big| \le \big|f(x) - f_n(x)\big| + \big|f_n(x) - f_n(a)\big| + \big|f_n(a) - f(a)\big|
	\] par inégalité triangulaire. Or, par hypothèse, il existe un rang $N \in \N$\/ (qui ne dépend pas de $x$\/ ou de $a$), tel que, $\forall n \ge N$, $\big|f(x) - f_n(x)\big| \le \frac{1}{3} \varepsilon$, et $\big|f_n(a) - f(a)\big| \le \frac{1}{3} \varepsilon$.
	De plus, par hypothèse, il existe $\delta >0$\/ tel que si $|x - a| \le \delta$, alors $|f_n(x) - f_n(a)| \le \frac{1}{3}\varepsilon$.\footnote{C'est là où l'hypothèse de la convergence uniforme est utilisée : on a besoin que le $N$\/ ne dépende pas de $x$\/ car on le fait varier.}
	On en déduit que $\big|f(x) - f(a)\big| \le \varepsilon$.
\end{prv}

\begin{crlr}
	Soit $T$\/ un intervalle de $\R$. Si une suite de fonctions $(f_n)_{n\in\N}$\/ continues sur $T$\/ converge uniformément sur $T$\/ vers une fonction continue sur $T$.
\end{crlr}

\begin{met}[Stratégie de la barrière]
	\begin{enumerate}
		\item La continuité (la dérivabilité aussi) est une propriété {\it locale}. Pour montrer qu'une fonction est continue sur un intervalle $T$, il suffit donc de montrer qu'elle est continue sur tout segment inclus dans $T$.
		\item Mais, la convergence uniforme est une propriété {\it globale}. La convergence sur tout segment inclus dans un intervalle n'implique pas la convergence uniforme sur l'intervalle (voir l'exercice 2).
		\item On n'écrit pas \[
				\substack{\ds\text{convergence uniforme}\\\ds\text{avec barrière}} \mathop{\red\implies} \substack{\ds\text{convergence uniforme}\\\ds\text{sans barrière}} \implies \substack{\ds\text{continuité}\\\ds\text{sans barrière}}
			\] mais plutôt \[
				\substack{\ds\text{convergence uniforme}\\\ds\text{avec barrière}} \implies \substack{\ds\text{continuité}\\\ds\text{avec barrière}} \implies \substack{\ds\text{continuité}\\\ds\text{sans barrière}}
			.\]
		\item Si, pour tous $a$\/ et $b$, $f$\/ est bornée sur $[a,b] \subset T$, mais cela n'implique pas que $f$\/ est bornée. Contre-exemple : la fonction $f : x \mapsto \frac{1}{x}$\/ est bornée sur tout intervalle $[a,b]$\/ avec $a$, $b \in \R^+_*$, \red{\sc mais} $f$\/ n'est pas bornée sur $]0,+\infty[$.
	\end{enumerate}
\end{met}

\begin{thm}[double-limite ou d'interversion des limites]
	Soit une suite de fonctions $(f_n)_{n\in\N}$\/ définies sur un intervalle $T$, et, soit $a$\/ une extrémité (éventuellement infinie)\footnote{autrement dit, $a \in \bar\R = \R \cup \{+\infty,-\infty\}$} de cet intervalle. Si la suite de fonctions $(f_n)_{n\in\N}$\/ converge \underline{uniformément} sur $T$\/ vers $f$\/ et si chaque fonction $f_n$\/ admet une limite finie $b_n$\/ en $a$, alors la suite de réels $b_n$\/ converge vers un réel $b$, et $\lim_{t\to a} f(t) = b$. Autrement dit, \[
		\lim_{t\to a} \Big(\underbrace{\lim_{n\to +\infty} f_n(t)}_{f(x)}\Big) = \lim_{n\to +\infty} \Big(\underbrace{\lim_{t\to a} f_n(t)}_{b_n}\Big)
	.\] \qed
\end{thm}

\begin{rmkn}
	Le théorème de la double-limite \guillemotleft~contient~\guillemotright\ le théorème 6 (théorème de préservation/transmission de la continuité), c'est un cas particulier. En effet, si les fonctions $f_n$\/ sont continues, alors \[
		\lim_{x \to a}f(x) = \underbrace{\lim_{n\to +\infty} f_n(a)}_{f(a)}
	.\]
\end{rmkn}


		\section{Endomorphismes adjoints}

\begin{defn}
	On dit qu'un endomorphisme $f : E \to E$\/ est \textit{autoadjoint} si \[
		\forall (\vec{u}, \vec{v}) \in E^2,\quad \big<f(\vec{u})\:\big|\:\vec{v}\big> = \big<\vec{u}\:\big|\:f(\vec{v})\big>
	.\] 
\end{defn}

Un endomorphisme autoadjoint est aussi appelé endomorphisme \textit{symétrique} (\textit{c.f.}\ proposition suivante). L'ensemble des endomorphismes autoadjoints est noté $\mathcal{S}(E)$.

\begin{prop}
	Un endomorphism est autoadjoint si, et seulement si la matrice de $F$\/ dans une base \ul{orthonormée} $\mathcal{B}$\/ est orthogonale.
	Autrement dit : \[
		f \in \mathcal{S}(E) \iff \big[\:f\:\big]_\mathcal{B} \in \mathcal{S}_n(\R)
	.\]
\end{prop}

\begin{prv}
	\begin{description}
		\item[$\implies$] Soit $\mathcal{B} = (\vec{\varepsilon}_1, \ldots, \vec{\varepsilon}_n)$\/ une base orthonormée de $E$. Ainsi, \[
				\forall i,\:\forall j,\quad \big<f(\vec{\varepsilon}_i)\:\big|\: \vec{\varepsilon}_j\big> = \big<\vec{\varepsilon}_i\:\big|\:f(\vec{\varepsilon}_j)\big>
			.\] On pose $\big[\:f\:\big]_{\mathcal{B}} = (a_{i,j})$\/ : \[
				\begin{pNiceMatrix}[last-col,last-row]
					\quad&\quad&a_{1,j}&\quad&\quad&\vec{\varepsilon}_i\\
					&&&&&\\
					\quad&\quad&a_{i,j}&\quad&\quad&\vec{\varepsilon}_i\\
					&&&&&\\
					\quad&\quad&a_{n,j}&\quad&\quad&\vec{\varepsilon}_n\\
					f(\vec{\varepsilon}_i)&&f(\vec{\varepsilon}_j)&&f(\vec{\varepsilon}_n)\\
				\end{pNiceMatrix}
			.\] Ainsi, $f(\vec{\varepsilon}_j) = a_{1,j} \vec{\varepsilon}_1 + \cdots + a_{i,j} \vec{\varepsilon}_i + \cdots + a_{n,j} \vec{\varepsilon}_n$. D'où, $\left<\vec{\varepsilon}_i  \mid f(\vec{\varepsilon}_j) \right> = a_{i,j}$\/ car la base $\mathcal{B}$\/ est orthonormée.
			De même avec l'autre produit scalaire, $\left< f(\vec{\varepsilon}_i)  \mid \vec{\varepsilon}_j \right>$, d'où $a_{i,j} = a_{j,i}$\/ par symétrie du produit scalaire. On en déduit que $\big[\:f\:\big]_\mathcal{B} \in \mathcal{S}_n(\R)$.
		\item[$\impliedby$]
			Si $\big[\: f\:\big]_\mathcal{B} \in \mathcal{S}_n(\R)$, alors $\left<f(\vec{\varepsilon}_i)  \mid \vec{\varepsilon}_j \right> = \left<\vec{\varepsilon}_i  \mid f(\vec{\varepsilon}_j)\right>$.
			Or, on pose $\vec{u} = x_1 \vec{\varepsilon}_1+ \cdots + x_n \vec{\varepsilon}_n$, et $\vec{v} = y_1 \vec{\varepsilon}_1 + \cdots + y_n \vec{\varepsilon}_n$.
			\begin{align*}
				\left<f(\vec{u})  \mid \vec{v} \right> &= \left<x_1 f(\vec{\varepsilon}_1) + \cdots + x_n f(\vec{\varepsilon}_n)  \mid y_1 f(\vec{\varepsilon}_1) + \cdots + y_n f(\vec{\varepsilon}_n) \right> \\
				&= \Big<\sum_{i=1}^n x_i f(\vec{\varepsilon}_i)\:\Big|\: \sum_{j=1}^n y_j \vec{\varepsilon}_j\Big> \\
				&= \sum_{i,j \in \llbracket 1,n \rrbracket}  x_i y_j \left<f(\vec{\varepsilon}_i) \mid \vec{\varepsilon}_j \right>\\
			\end{align*}
			De même en inversant $\vec{u}$\/ et $\vec{v}$.
			On en déduit donc $\left<f(\vec{u} \mid \vec{v} \right> = \left<\vec{u}  \mid f(\vec{v}) \right>$.
	\end{description}
\end{prv}

\begin{exo}
	\begin{enumerate}
		\item Si $f$\/ est autoadjoint, montrons que $\Ker f \perp \Im f$, et $\Ker f \oplus \Im f$.
			On suppose $\forall \vec{u}$, $\forall \vec{v}$, $\left<f(\vec{u}) \mid \vec{v} \right> = \left<\vec{u}  \mid f(\vec{v}) \right>$.
			Soit $\vec{u} \in \Ker f$, et soit $\vec{v} \in \Im f$.
			On sait que $f(\vec{u}) = \vec{0}$, et qu'il existe $\vec{x} \in E$\/ tel que $\vec{v} = f(\vec{x})$.
			Ainsi, \[
				\left<\vec{u}  \mid \vec{v} \right>
				= \left<\vec{u}  \mid f(\vec{x}) \right>
				= \left<f(\vec{u})  \mid \vec{x} \right>
				= 0
			.\] 
			D'où $\vec{u} \perp \vec{v}$. Ainsi, $\Ker f \perp \Im f$.


			De plus, $E$\/ est de dimension finie, d'où, d'après le théorème du rang, \[
				\dim \Ker f + \dim \Im f = \dim E
			.\] Aussi, $\Ker f \oplus (\Ker f)^\perp = E$, donc $\dim(\Ker f) + \dim(\Ker f)^\perp = \dim E$.
			On en déduit donc que $\dim(\Im f)= \dim(\Ker f)^\perp$.
			Or, $\Im f \subset (\Ker f)^\perp$\/ car $\Im f \perp \Ker f$.
			Ainsi $\Im f = (\Ker f)^\perp$, on en déduit que \[
				\Im f \oplus \Ker f = E
			.\]
			\begin{description}
				\item[$\impliedby$] 
					Soit $p$\/ la projection sur $F$\/ parallèlement à $G$.
					Supposons l'endomorphisme $P$\/ autoadjoint.
					D'après la question 1., le $\Ker p \perp \Im p$.
					Ainsi, $F = \Im p$\/ et $G = \Ker p$.
					D'où, $F \perp G$, $p$\/ est donc une projection orthogonale.
				\item[$\implies$]
					Réciproquement, supposons $p$\/ une projection orthogonale.
					Soit $\mathcal{B} = (\vec{\varepsilon}_1, \ldots, \vec{\varepsilon}_q)$\/ une base orthonormée de $F$.
					Ainsi, pour tout $\vec{x} \in E$, \[
						p(\vec{x}) = \sum_{i = 1}^q \left<\vec{x}  \mid \vec{\varepsilon}_i \right>\,\vec{\varepsilon}_i
					.\] 
					On veut montrer que l'endomorphisme $p$\/ est autoadjoint.
					Soient $\vec{u}$\/ et $\vec{v}$\/ deux vecteurs de $E$.
					\begin{align*}
						\left<p(\vec{u})  \mid \vec{v} \right>
						= \Big<\sum_{i=1}^q \left< \vec{u}\mid \vec{\varepsilon}_i \right>\vec{\varepsilon}_i\:\Big|\; \vec{v}\;\Big>
						&= \sum_{i=1}^q \left<u  \mid \vec{\varepsilon}_{i} \right>\: \left< \vec{\varepsilon}_i  \mid v\right>\\
						&= \sum_{i=1}^q \left<v  \mid \vec{\varepsilon}_i \right>\:\left<\vec{\varepsilon}_i   \mid u\right> \\
						&= \left< \vec{u}  \mid p(\vec{v}) \right> \\
					\end{align*}

					Autre méthode, pour tous vecteurs $\vec{u}$\/ et $\vec{v}$\/ de $E$,
					\begin{align*}
						\left<p(\vec{u})  \mid \vec{v} \right>
						&= \left<p(\vec{u})  \mid p(\vec{v}) + \vec{v} - p(\vec{v}) \right> \\
						&= \left< p(\vec{u})  \mid p(\vec{v}) \right> + \left<p(\vec{u})  \mid  \vec{v} - p(\vec{v}) \right> \\
						&= \left<p(\vec{u}) \mid p(\vec{v}) \right> + \left<u - p(\vec{u})  \mid p(\vec{v}) \right> \\
						&= \left<\vec{u}  \mid p(\vec{v}) \right> \\
					\end{align*}
					car $p$\/ est orthogonale.
			\end{description}
	\end{enumerate}
\end{exo}


\begin{prop-defn}
	Si $f$\/ est un endomorphisme d'un espace euclidien $E$, alors il existe un unique endomorphisme de $E$, noté $f^\star$\/ et appelé l'\textit{adjoint} de $f$, tel que \[
		\forall (\vec{u},\vec{v}) \in E^2,\quad\quad \left<f^\star(\vec{u})  \mid \vec{v} \right> =  \left<\vec{u}  \mid f(\vec{v}) \right>
	.\] 
	Si $A$\/ est la matrice $f$\/ dans une base orthonormée $\mathcal{B}$\/ de $E$, alors $A^\top$\/ est la matrice de $f^\star$\/ dans~$\mathcal{B}$\/ : \[
		\big[\:f\:\big]_\mathcal{B} = \big[\:f\:\big]_\mathcal{B}^\top
	.\]
\end{prop-defn}

\begin{prv}
	Soit $\vec{u} \in E$. L'application \begin{align*}
		\varphi: E &\longrightarrow \R \\
		\vec{v} &\longmapsto \left<\vec{u}  \mid f(\vec{v}) \right>.
	\end{align*}
	La forme $\varphi$\/ est linéaire car $\varphi(\alpha_1 \vec{v}_1 + \alpha_2 \vec{v}_2) = \left<\vec{u} \mid f(\alpha_1 \vec{v}_1 + \alpha_2 \vec{v}_2) \right> = \left<\vec{u}  \mid \alpha_1 f(\vec{v}_1) + \alpha_2 f(\vec{v}_2) \right> = \alpha_1\left<\vec{u}  \mid  f(\vec{v}) \right> + \alpha_2 \left<\vec{u}  \mid f(\vec{v}_2) \right> = \alpha_1 \varphi(\vec{v}_1) + \alpha_2 \varphi(\vec{v}_2)$.
	D'où, d'après le théorème de \textsc{Riesz}, il existe un \ul{unique} vecteur $\vec{a} \in E$\/ tel que $\varphi(\vec{v}) = \left<\vec{a}  \mid \vec{v} \right>$\/ pour tout $\vec{v} \in E$.
	Ainsi, pour tout vecteur $\vec{v} \in E$, $\left<\vec{u}  \mid f(\vec{v}) \right> = \left<\vec{a}  \mid \vec{v} \right>$.
	On note $\vec{a} = f^\star(\vec{u})$.
	Soit l'application \begin{align*}
		f^\star : E &\longrightarrow E \\
		\vec{u} &\longmapsto f^\star(\vec{u}).
	\end{align*}
	La démonstration telle que $f^\star $\/ est linéaire est dans le poly.
	L'application $f^\star$\/ vérifie : $\left< \vec{u} \mid f(\vec{v}) \right> = \left<f^\star (\vec{u})  \mid \vec{v} \right>$, pour tous vecteurs $\vec{u}$\/ et $\vec{v}$.
	\textsl{Quelle est la matrice de $f^\star$, dans une base orthonormée ?}\@
	Soit $\mathcal{B}$\/ une base orthonormée de $E$, et soient $A = \big[\:f\:\big]_\mathcal{B}$, $B = \big[\:f^\star \:\big]_\mathcal{B}$, $U = \big[\:\vec{u}\:\big]_\mathcal{B}$, et $V = \big[\:\vec{v}\:\big]_\mathcal{B}$.
	Les matrices $U$\/ et $V$\/ sont des vecteurs colonnes, et $A$\/ et $B$\/ sont des matrices carrées.
	Ainsi, \[
		U^\top \cdot A \cdot V = \left< \vec{u}  \mid f(\vec{v}) \right> 
		= \left<f^\star (\vec{u})  \mid \vec{v} \right> = (B\cdot U)^\top \cdot V,
	\] ce qui est vrai quelque soit les vecteurs colonnes $U$\/ et $V$.
	D'où, $\forall U$, $\forall V$, $U^\top \cdot \big(A \cdot V\big) = U^\top \cdot \big(B^\top \cdot V\big)$.
	Ainsi, pour tous vecteurs $U$\/ et $V$, \[
		U^\top \cdot \Big[ (AV) - (B^\top V)\Big] = 0
	.\] En particulier, si $U = (AV) - (B^\top V)$, le produit scalaire $\left<\vec{u}  \mid \vec{u} \right>$\/ est nul, donc $U = 0$.
	Ainsi, \[
		\forall V,\quad A\cdot V = B^\top\cdot  V
	.\] De même, on conclut que $A = B^\top$. On en déduit donc que \[
	\big[\:f^\star\:\big]_\mathcal{B} = \big[\:f\:\big]_\mathcal{B}^\top
	.\]
\end{prv}

Les propriétés suivantes sont vrais :
\begin{itemize}
	\item $(f  \circ g)^\star  = g^\star \circ f^\star$, \quad $(f^\star)^\star = f$, \quad et \quad $(\alpha f + \beta g)^\star  = \alpha f^\star + \beta g^\star $\/ ;
	\item $(A\cdot B)^\top  = B^\top \cdot A^\top$, \quad $(A^\top)^\top = A$, \quad et \quad $(\alpha A + \beta B)^\top = \alpha A^\top+ \beta B^\top $.
\end{itemize}
Des deuxièmes et troisièmes points,  il en résulte que les applications $f \mapsto f^\star$, et $A \mapsto A^\top$\/ sont des applications involutives.



		\clearpage
		\setcounter{section}{0}		\renewcommand{\thesection}{\llap{Annexe }\thechapter.\Alph{section}}
		\renewcommand{\thesectionnum}{\Alph{section}}
		\section{Programmation dynamique}

On rappelle le problème \textsc{Knapsack} : \[
	\begin{cases}
		\text{\textbf{Entrée}} &: n \in \N,\: w \in (\N^*)^n,\: v \in (\N^*)^n,\: P \in \N\\
		\text{\textbf{Sortie}} &:\max_{x \in \{0,1\}^n} \big\{\left<x,v \right> \:\big|\: \left<x,w \right> \le P \}.
	\end{cases}
\]

On pose \[\textsc{sad}(n, w, v, P) = \max_{x \in \{0,1\}^n} \big\{\left<x,v \right> \:\big|\: \left<x,w \right> \le P \},\] et \[\mathrm{sol}(n, w, v, P) = \{\left<x,v \right>  \mid \left<x,w \right> \le P,\: x \in \{0,1\}^n \}.\] Lorsque $y \in \R^n$, avec $y = (y_1, \ldots, y_n)$, on note $\R^{n-1} \owns \tilde y = (y_2, y_3, \ldots, 0)$. Ainsi, si $n > 0$,
\begin{align*}
	\mathrm{sol}(n, w, v, P) &= \{\left<x, v \right>  \mid \left< x,w \right> \le P,\: x \in \{0,1\}^n \text{ et } x_1 = 0\}\\
	&\quad\cup \{\left<x, v \right>  \mid \left< x,w \right> \le P,\: x \in \{0,1\}^n \text{ et } x_1 = 1\}\\
	&= \{\left<\tilde{x}, \tilde{v} \right>  \mid \left<\tilde{x}, \tilde{w} \right> \le P,\: x u\in \{0,1\}^n \text{ et } x_1 = 0\} \\
	&\quad\cup \{v_1 + \left<\tilde{x}, \tilde{v} \right>  \mid \left<\tilde{x},\tilde{w} \right> \le P - w_1,\: x \in \{0,1\}^n \text{ et } x_1 = 1\}\\
	&= \{\left<y, \tilde{v} \right>  \mid \left<y, \tilde{w} \right> \le P \text{ et } y \in \{0,1\}^{n-1}\}  \\
	&\quad\cup \{v_1 + \left<y, \tilde{v} \right>  \mid \left<y, \tilde{w} \right> \le P - w_1 \text{ et } y \in \{0,1\}^{n-1}\}
\end{align*}
D'où, par passage au $\max$, si $n > 0$,
\begin{align*}
	\textsc{sad}(n, w, v, P) &= \max( \\
	&\quad\quad \max \{\left<y  \mid \tilde{v} \right>  \mid \left<y, \tilde{w} \right> \le P \text{ et } y \in \{0,1\}^{n-1}\}\\
	&\quad\quad v_1 + \max \{\left<y  \mid \tilde{v} \right>  \mid \left<y, \tilde{w} \right> \le P - w_1 \text{ et } y \in \{0,1\}^{n-1}\} \\
	&) = \max\!\big(\textsc{sad}(n-1, \tilde{w}, \tilde{v}, P), v_1 + \textsc{sad}(n-1, \tilde{w}, \tilde{v}, P - w_1)\big).
\end{align*}
Si $n = 0$, alors $\textsc{sad}(0, v, w, P) = 0$.

\begin{rmk}
	Si on le code \textit{tel quel}, il y aura $\mathcal{O}(2^n)$ appels récursifs. Mais, on a $(n+1)(P+1)$\/ sous-problèmes.
\end{rmk}

Notons alors, pour $n$, $v$, $w$, $P$ fixés, $(s_{i,j})_{\substack{i \in \llbracket 1,n \rrbracket\\ j \in \llbracket 0,P \rrbracket}}$\/ tel que \[
	s_{i,j} = \textsc{sad}\Big(n-i, v_{\big|\llbracket i+1,n \rrbracket}, w_{\big|\llbracket i+1,n \rrbracket}, j\Big)
.\]
On a alors $\textsc{sad}(n, v, w, P) = s_{0,P}$. Ainsi, pour $j \in \llbracket 0,P \rrbracket$, $s_{n,j} = 0$ ; pour $i \in \llbracket 0,n \rrbracket$, $s_{i,0} = 0$ ; \[
	s_{i,j} = \max(s_{i+1,j}, v_{i+1} + s_{i+1,j- w_{i+1}})
;\] et, si $w_{i+1} > j$, alors $s_{i,j} = s_{i+1,j}$.

La complexité de remplissage de la matrice est en $\mathcal{O}(n\,P)$\/ en temps et en espace.
On n'a pas prouvé $\text{\textbf{P}} = \text{\textbf{NP}}$, la taille de l'entrée est
\begin{itemize}
	\item pour un entier $n$\/ : $\log_2(n)$,
	\item pour un tableau de $n$\/ entiers : $n \log_2(n)$,
	\item pour un tableau de $n$\/ entiers : $n \log_2(n)$,
	\item pour un entier $P$ : {\color{red}$\log_2(P)$}.
\end{itemize}
Vis à vis de la taille de l'entrée, la complexité de remplissage est {\color{red}exponentielle}.


	}
	\def\addmacros#1{#1}
}

{
	\chap[8]{Jeux}
	\minitoc
	\renewcommand{\cwd}{../cours/chap08/}
	\addmacros{
		\section{Motivation}

\lettrine On place au centre de la classe 40 bonbons. On en distribue un chacun. Si, par exemple, chacun choisit un bonbon et, au \textit{top} départ, prennent celui choisi.
Il est probable que plusieurs choisissent le même. Comme gérer lorsque plusieurs essaient d'accéder à la mémoire ?

Deuxièmement, sur l'ordinateur, plusieurs applications tournent en même temps. Pour le moment, on considérait qu'un seul programme était exécuté, mais, le \textsc{pc} ne s'arrête pas pendant l'exécution du programme.

On s'intéresse à la notion de \guillemotleft~processus~\guillemotright\ qui représente une tâche à réaliser.
On ne peut pas assigner un processus à une unité de calcul, mais on peut \guillemotleft~allumer~\guillemotright\ et \guillemotleft~éteindre~\guillemotright\ un processus.
Le programme allumant et éteignant les processus est \guillemotleft~l'ordonnanceur.~\guillemotright\@ Il doit aussi s'occuper de la mémoire du processus (chaque processus à sa mémoire séparée).

On s'intéresse, dans ce chapitre, à des programmes qui \guillemotleft~partent du même~\guillemotright\ : un programme peut créer un \guillemotleft~fil d'exécution~\guillemotright\ (en anglais, \textit{thread}). Le programme peut gérer les fils d'exécution qu'il a créé, et éventuellement les arrêter.
Les fils d'exécutions partagent la mémoire du programme qui les a créé.

En C, une tâche est représenté par une fonction de type \lstinline[language=c]!void* tache(void* arg)!. Le type \lstinline[language=c]!void*!\ est l'équivalent du type \lstinline[language=caml]!'a! : on peut le \textit{cast} à un autre type (comme \lstinline[language=c]-char*-).

\begin{lstlisting}[language=c,caption=Création de \textit{thread}s en C]
void* tache(void* arg) {
	printf("%s\n", (char*) arg);
	return NULL;
}

int main() {
	pthread_t p1, p2;

	printf("main: begin\n");

	pthread_create(&p1, NULL, tache, "A");
	pthread_create(&p2, NULL, tache, "B");

	pthread_join(p1, NULL);
	pthread_join(p2, NULL);

	printf("main: end\n");

	return 0;
}
\end{lstlisting}

\begin{lstlisting}[language=c,caption=Mémoire dans les \textit{thread}s en C]
int max = 10;
volatile int counter = 0;

void* tache(void* arg) {
	char* letter = arg;
	int i;

	printf("%s begin [addr of i: %p] \n", letter, &i);

	for(i = 0; i < max; i++) {
		counter = counter + 1;
	}

	printf("%s : done\n", letter);
	return NULL;
}

int main() {
	pthread_t p1, p2;

	printf("main: begin\n");

	pthread_create(&p1, NULL, tache, "A");
	pthread_create(&p2, NULL, tache, "B");

	pthread_join(p1, NULL);
	pthread_join(p2, NULL);

	printf("main: end\n");

	return 0;
}
\end{lstlisting}

Dans les \textit{thread}s, les variables locales (comme \texttt{i}) sont séparées en mémoire. Mais, la variable \texttt{counter} est modifiée, mais elle ne correspond pas forcément à $2 \times \texttt{max}$. En effet, si \texttt{p1} et \texttt{p2} essaient d'exécuter au même moment de réaliser l'opération \lstinline[language=c]-counter = counter + 1-, ils peuvent récupérer deux valeurs identiques de \texttt{counter}, ajouter 1, puis réassigner \texttt{counter}.
Ils \guillemotleft~se marchent sur les pieds.~\guillemotright\ 

Parmi les opérations, on distingue certaines dénommées \guillemotleft~atomiques~\guillemotright\ qui ne peuvent pas être séparées. L'opération \lstinline[language=c]-i++- n'est pas atomique, mais la lecture et l'écriture mémoire le sont.

\begin{defn}
	On dit d'une variable qu'elle est \textit{atomique} lorsque l'ordonnanceur ne l'interrompt pas.
\end{defn}

\begin{exm}
	L'opération \lstinline[language=c]-counter = counter + 1- exécutée en série peut être représentée comme ci-dessous. Avec \texttt{counter} valant 40, cette exécution donne 42.
	\begin{table}[H]
		\centering
		\begin{tabular}{l|l}
			Exécution du fil A & Exécution du fil B\\ \hline
			(1)~$\mathrm{reg}_1 \gets \texttt{counter}$ & (4)~$\mathrm{reg}_2 \gets \texttt{counter}$ \\
			(2)~$\mathrm{reg}_1{++}$ & (5)~$\mathrm{reg}_2{++}$ \\
			(3)~$\texttt{counter} \gets \mathrm{reg}_1$ & (6)~$\texttt{counter} \gets \mathrm{reg}_2$
		\end{tabular}
	\end{table}
	\noindent Mais, avec l'exécution en simultanée, la valeur de \texttt{counter} sera 41.
	\begin{table}[H]
		\centering
		\begin{tabular}{l|l}
			Exécution du fil A & Exécution du fil B\\ \hline
			(1)~$\mathrm{reg}_1 \gets \texttt{counter}$ & (2)~$\mathrm{reg}_2 \gets \texttt{counter}$ \\
			(3)~$\mathrm{reg}_1{++}$ & (5)~$\mathrm{reg}_2{++}$ \\
			(4)~$\texttt{counter} \gets \mathrm{reg}_1$ & (6)~$\texttt{counter} \gets \mathrm{reg}_2$
		\end{tabular}
	\end{table}
	\noindent Il y a \textit{entrelacement} des deux fils d'exécution.
\end{exm}

\begin{rmk}[Problèmes de la programmation concurrentielle]
	\begin{itemize}
		\item Problème d'accès en mémoire,
		\item Problème du rendez-vous,\footnote{Lorsque deux programmes terminent, ils doivent s'attendre pour donner leurs valeurs.}
		\item Problème du producteur-consommateur,\footnote{Certains programmes doivent ralentir ou accélérer.}
		\item Problème de l'entreblocage,\footnote{\textit{c.f.} exemple ci-après.}
		\item Problème famine, du dîner des philosophes.\footnote{Les philosophes mangent autour d'une table, et mangent du riz avec des baguettes. Ils décident de n'acheter qu'une seule baguette par personne. Un philosophe peut, ou penser, ou manger. Mais, pour manger, ils ont besoin de deux baguettes. S'ils ne mangent pas, ils meurent.}
	\end{itemize}
\end{rmk}

\begin{exm}[Problème de l'entreblocage]~

	\begin{table}[H]
		\centering
		\begin{tabular}{l|l|l}
			Fil A & Fil B & Fil C\\ \hline
			RDV(C) & RDV(A) & RDV(B)\\
			RDV(B) & RDV(C) & RDV(A)\\
		\end{tabular}
		\caption{Problème de l'entreblocage}
	\end{table}
\end{exm}

Comment résoudre le problème des deux incrementations ? Il suffit de \guillemotleft~mettre un verrou.~\guillemotright\ Le premier fil d'exécution \guillemotleft~s'enferme~\guillemotright\ avec l'expression \lstinline[language=c]!count++!, le second fil d'exécution attend que l'autre sorte pour pouvoir entrer et s'enfermer à son tour.


		\section{Continuité}

\begin{exm}
	Dans l'exercice 2, chaque fonction $f_n : t \mapsto t^n$\/ est continue sur $[0,1]$\/ mais la limite $f$\/ n'est pas continue sur $[0,1]$\/ (car elle n'est pas continue en $1$).
\end{exm}

\begin{thm}
	Soit $a$\/ un réel dans un intervalle $T$\/ de $\R$. Si une suite de fonctions $(f_n)_{n\in\N}$\/ continues en $a$\/ converge uniformément sur $T$\/ vers une fonction $f$, alors $f$\/ est aussi continue en $a$.
\end{thm}

\begin{prv}
	On suppose les fonctions $f_n$\/ continues en $a$\/ ($f_n(x) \longrightarrow f_n(a)$) et que la suite de fonctions $(f_n)_{n\in\N}$\/ converge uniformément vers $f$\/ ($\sup\:|f_n -f| \longrightarrow 0$). On veut montrer que $f$\/ est continue en $a$\/ : $f(x) \tendsto{x \to a} f(a)$, i.e.\ \[
		\forall \varepsilon > 0,\:\exists \delta > 0,\: \forall x \in T,\quad|x-a| \le \delta \implies |f(x) - f(a)| \le \varepsilon
	.\]
	Soit $\varepsilon > 0$. On calcule \[
		\big|f(x) - f(a)\big| \le \big|f(x) - f_n(x)\big| + \big|f_n(x) - f_n(a)\big| + \big|f_n(a) - f(a)\big|
	\] par inégalité triangulaire. Or, par hypothèse, il existe un rang $N \in \N$\/ (qui ne dépend pas de $x$\/ ou de $a$), tel que, $\forall n \ge N$, $\big|f(x) - f_n(x)\big| \le \frac{1}{3} \varepsilon$, et $\big|f_n(a) - f(a)\big| \le \frac{1}{3} \varepsilon$.
	De plus, par hypothèse, il existe $\delta >0$\/ tel que si $|x - a| \le \delta$, alors $|f_n(x) - f_n(a)| \le \frac{1}{3}\varepsilon$.\footnote{C'est là où l'hypothèse de la convergence uniforme est utilisée : on a besoin que le $N$\/ ne dépende pas de $x$\/ car on le fait varier.}
	On en déduit que $\big|f(x) - f(a)\big| \le \varepsilon$.
\end{prv}

\begin{crlr}
	Soit $T$\/ un intervalle de $\R$. Si une suite de fonctions $(f_n)_{n\in\N}$\/ continues sur $T$\/ converge uniformément sur $T$\/ vers une fonction continue sur $T$.
\end{crlr}

\begin{met}[Stratégie de la barrière]
	\begin{enumerate}
		\item La continuité (la dérivabilité aussi) est une propriété {\it locale}. Pour montrer qu'une fonction est continue sur un intervalle $T$, il suffit donc de montrer qu'elle est continue sur tout segment inclus dans $T$.
		\item Mais, la convergence uniforme est une propriété {\it globale}. La convergence sur tout segment inclus dans un intervalle n'implique pas la convergence uniforme sur l'intervalle (voir l'exercice 2).
		\item On n'écrit pas \[
				\substack{\ds\text{convergence uniforme}\\\ds\text{avec barrière}} \mathop{\red\implies} \substack{\ds\text{convergence uniforme}\\\ds\text{sans barrière}} \implies \substack{\ds\text{continuité}\\\ds\text{sans barrière}}
			\] mais plutôt \[
				\substack{\ds\text{convergence uniforme}\\\ds\text{avec barrière}} \implies \substack{\ds\text{continuité}\\\ds\text{avec barrière}} \implies \substack{\ds\text{continuité}\\\ds\text{sans barrière}}
			.\]
		\item Si, pour tous $a$\/ et $b$, $f$\/ est bornée sur $[a,b] \subset T$, mais cela n'implique pas que $f$\/ est bornée. Contre-exemple : la fonction $f : x \mapsto \frac{1}{x}$\/ est bornée sur tout intervalle $[a,b]$\/ avec $a$, $b \in \R^+_*$, \red{\sc mais} $f$\/ n'est pas bornée sur $]0,+\infty[$.
	\end{enumerate}
\end{met}

\begin{thm}[double-limite ou d'interversion des limites]
	Soit une suite de fonctions $(f_n)_{n\in\N}$\/ définies sur un intervalle $T$, et, soit $a$\/ une extrémité (éventuellement infinie)\footnote{autrement dit, $a \in \bar\R = \R \cup \{+\infty,-\infty\}$} de cet intervalle. Si la suite de fonctions $(f_n)_{n\in\N}$\/ converge \underline{uniformément} sur $T$\/ vers $f$\/ et si chaque fonction $f_n$\/ admet une limite finie $b_n$\/ en $a$, alors la suite de réels $b_n$\/ converge vers un réel $b$, et $\lim_{t\to a} f(t) = b$. Autrement dit, \[
		\lim_{t\to a} \Big(\underbrace{\lim_{n\to +\infty} f_n(t)}_{f(x)}\Big) = \lim_{n\to +\infty} \Big(\underbrace{\lim_{t\to a} f_n(t)}_{b_n}\Big)
	.\] \qed
\end{thm}

\begin{rmkn}
	Le théorème de la double-limite \guillemotleft~contient~\guillemotright\ le théorème 6 (théorème de préservation/transmission de la continuité), c'est un cas particulier. En effet, si les fonctions $f_n$\/ sont continues, alors \[
		\lim_{x \to a}f(x) = \underbrace{\lim_{n\to +\infty} f_n(a)}_{f(a)}
	.\]
\end{rmkn}


		\section{Endomorphismes adjoints}

\begin{defn}
	On dit qu'un endomorphisme $f : E \to E$\/ est \textit{autoadjoint} si \[
		\forall (\vec{u}, \vec{v}) \in E^2,\quad \big<f(\vec{u})\:\big|\:\vec{v}\big> = \big<\vec{u}\:\big|\:f(\vec{v})\big>
	.\] 
\end{defn}

Un endomorphisme autoadjoint est aussi appelé endomorphisme \textit{symétrique} (\textit{c.f.}\ proposition suivante). L'ensemble des endomorphismes autoadjoints est noté $\mathcal{S}(E)$.

\begin{prop}
	Un endomorphism est autoadjoint si, et seulement si la matrice de $F$\/ dans une base \ul{orthonormée} $\mathcal{B}$\/ est orthogonale.
	Autrement dit : \[
		f \in \mathcal{S}(E) \iff \big[\:f\:\big]_\mathcal{B} \in \mathcal{S}_n(\R)
	.\]
\end{prop}

\begin{prv}
	\begin{description}
		\item[$\implies$] Soit $\mathcal{B} = (\vec{\varepsilon}_1, \ldots, \vec{\varepsilon}_n)$\/ une base orthonormée de $E$. Ainsi, \[
				\forall i,\:\forall j,\quad \big<f(\vec{\varepsilon}_i)\:\big|\: \vec{\varepsilon}_j\big> = \big<\vec{\varepsilon}_i\:\big|\:f(\vec{\varepsilon}_j)\big>
			.\] On pose $\big[\:f\:\big]_{\mathcal{B}} = (a_{i,j})$\/ : \[
				\begin{pNiceMatrix}[last-col,last-row]
					\quad&\quad&a_{1,j}&\quad&\quad&\vec{\varepsilon}_i\\
					&&&&&\\
					\quad&\quad&a_{i,j}&\quad&\quad&\vec{\varepsilon}_i\\
					&&&&&\\
					\quad&\quad&a_{n,j}&\quad&\quad&\vec{\varepsilon}_n\\
					f(\vec{\varepsilon}_i)&&f(\vec{\varepsilon}_j)&&f(\vec{\varepsilon}_n)\\
				\end{pNiceMatrix}
			.\] Ainsi, $f(\vec{\varepsilon}_j) = a_{1,j} \vec{\varepsilon}_1 + \cdots + a_{i,j} \vec{\varepsilon}_i + \cdots + a_{n,j} \vec{\varepsilon}_n$. D'où, $\left<\vec{\varepsilon}_i  \mid f(\vec{\varepsilon}_j) \right> = a_{i,j}$\/ car la base $\mathcal{B}$\/ est orthonormée.
			De même avec l'autre produit scalaire, $\left< f(\vec{\varepsilon}_i)  \mid \vec{\varepsilon}_j \right>$, d'où $a_{i,j} = a_{j,i}$\/ par symétrie du produit scalaire. On en déduit que $\big[\:f\:\big]_\mathcal{B} \in \mathcal{S}_n(\R)$.
		\item[$\impliedby$]
			Si $\big[\: f\:\big]_\mathcal{B} \in \mathcal{S}_n(\R)$, alors $\left<f(\vec{\varepsilon}_i)  \mid \vec{\varepsilon}_j \right> = \left<\vec{\varepsilon}_i  \mid f(\vec{\varepsilon}_j)\right>$.
			Or, on pose $\vec{u} = x_1 \vec{\varepsilon}_1+ \cdots + x_n \vec{\varepsilon}_n$, et $\vec{v} = y_1 \vec{\varepsilon}_1 + \cdots + y_n \vec{\varepsilon}_n$.
			\begin{align*}
				\left<f(\vec{u})  \mid \vec{v} \right> &= \left<x_1 f(\vec{\varepsilon}_1) + \cdots + x_n f(\vec{\varepsilon}_n)  \mid y_1 f(\vec{\varepsilon}_1) + \cdots + y_n f(\vec{\varepsilon}_n) \right> \\
				&= \Big<\sum_{i=1}^n x_i f(\vec{\varepsilon}_i)\:\Big|\: \sum_{j=1}^n y_j \vec{\varepsilon}_j\Big> \\
				&= \sum_{i,j \in \llbracket 1,n \rrbracket}  x_i y_j \left<f(\vec{\varepsilon}_i) \mid \vec{\varepsilon}_j \right>\\
			\end{align*}
			De même en inversant $\vec{u}$\/ et $\vec{v}$.
			On en déduit donc $\left<f(\vec{u} \mid \vec{v} \right> = \left<\vec{u}  \mid f(\vec{v}) \right>$.
	\end{description}
\end{prv}

\begin{exo}
	\begin{enumerate}
		\item Si $f$\/ est autoadjoint, montrons que $\Ker f \perp \Im f$, et $\Ker f \oplus \Im f$.
			On suppose $\forall \vec{u}$, $\forall \vec{v}$, $\left<f(\vec{u}) \mid \vec{v} \right> = \left<\vec{u}  \mid f(\vec{v}) \right>$.
			Soit $\vec{u} \in \Ker f$, et soit $\vec{v} \in \Im f$.
			On sait que $f(\vec{u}) = \vec{0}$, et qu'il existe $\vec{x} \in E$\/ tel que $\vec{v} = f(\vec{x})$.
			Ainsi, \[
				\left<\vec{u}  \mid \vec{v} \right>
				= \left<\vec{u}  \mid f(\vec{x}) \right>
				= \left<f(\vec{u})  \mid \vec{x} \right>
				= 0
			.\] 
			D'où $\vec{u} \perp \vec{v}$. Ainsi, $\Ker f \perp \Im f$.


			De plus, $E$\/ est de dimension finie, d'où, d'après le théorème du rang, \[
				\dim \Ker f + \dim \Im f = \dim E
			.\] Aussi, $\Ker f \oplus (\Ker f)^\perp = E$, donc $\dim(\Ker f) + \dim(\Ker f)^\perp = \dim E$.
			On en déduit donc que $\dim(\Im f)= \dim(\Ker f)^\perp$.
			Or, $\Im f \subset (\Ker f)^\perp$\/ car $\Im f \perp \Ker f$.
			Ainsi $\Im f = (\Ker f)^\perp$, on en déduit que \[
				\Im f \oplus \Ker f = E
			.\]
			\begin{description}
				\item[$\impliedby$] 
					Soit $p$\/ la projection sur $F$\/ parallèlement à $G$.
					Supposons l'endomorphisme $P$\/ autoadjoint.
					D'après la question 1., le $\Ker p \perp \Im p$.
					Ainsi, $F = \Im p$\/ et $G = \Ker p$.
					D'où, $F \perp G$, $p$\/ est donc une projection orthogonale.
				\item[$\implies$]
					Réciproquement, supposons $p$\/ une projection orthogonale.
					Soit $\mathcal{B} = (\vec{\varepsilon}_1, \ldots, \vec{\varepsilon}_q)$\/ une base orthonormée de $F$.
					Ainsi, pour tout $\vec{x} \in E$, \[
						p(\vec{x}) = \sum_{i = 1}^q \left<\vec{x}  \mid \vec{\varepsilon}_i \right>\,\vec{\varepsilon}_i
					.\] 
					On veut montrer que l'endomorphisme $p$\/ est autoadjoint.
					Soient $\vec{u}$\/ et $\vec{v}$\/ deux vecteurs de $E$.
					\begin{align*}
						\left<p(\vec{u})  \mid \vec{v} \right>
						= \Big<\sum_{i=1}^q \left< \vec{u}\mid \vec{\varepsilon}_i \right>\vec{\varepsilon}_i\:\Big|\; \vec{v}\;\Big>
						&= \sum_{i=1}^q \left<u  \mid \vec{\varepsilon}_{i} \right>\: \left< \vec{\varepsilon}_i  \mid v\right>\\
						&= \sum_{i=1}^q \left<v  \mid \vec{\varepsilon}_i \right>\:\left<\vec{\varepsilon}_i   \mid u\right> \\
						&= \left< \vec{u}  \mid p(\vec{v}) \right> \\
					\end{align*}

					Autre méthode, pour tous vecteurs $\vec{u}$\/ et $\vec{v}$\/ de $E$,
					\begin{align*}
						\left<p(\vec{u})  \mid \vec{v} \right>
						&= \left<p(\vec{u})  \mid p(\vec{v}) + \vec{v} - p(\vec{v}) \right> \\
						&= \left< p(\vec{u})  \mid p(\vec{v}) \right> + \left<p(\vec{u})  \mid  \vec{v} - p(\vec{v}) \right> \\
						&= \left<p(\vec{u}) \mid p(\vec{v}) \right> + \left<u - p(\vec{u})  \mid p(\vec{v}) \right> \\
						&= \left<\vec{u}  \mid p(\vec{v}) \right> \\
					\end{align*}
					car $p$\/ est orthogonale.
			\end{description}
	\end{enumerate}
\end{exo}


\begin{prop-defn}
	Si $f$\/ est un endomorphisme d'un espace euclidien $E$, alors il existe un unique endomorphisme de $E$, noté $f^\star$\/ et appelé l'\textit{adjoint} de $f$, tel que \[
		\forall (\vec{u},\vec{v}) \in E^2,\quad\quad \left<f^\star(\vec{u})  \mid \vec{v} \right> =  \left<\vec{u}  \mid f(\vec{v}) \right>
	.\] 
	Si $A$\/ est la matrice $f$\/ dans une base orthonormée $\mathcal{B}$\/ de $E$, alors $A^\top$\/ est la matrice de $f^\star$\/ dans~$\mathcal{B}$\/ : \[
		\big[\:f\:\big]_\mathcal{B} = \big[\:f\:\big]_\mathcal{B}^\top
	.\]
\end{prop-defn}

\begin{prv}
	Soit $\vec{u} \in E$. L'application \begin{align*}
		\varphi: E &\longrightarrow \R \\
		\vec{v} &\longmapsto \left<\vec{u}  \mid f(\vec{v}) \right>.
	\end{align*}
	La forme $\varphi$\/ est linéaire car $\varphi(\alpha_1 \vec{v}_1 + \alpha_2 \vec{v}_2) = \left<\vec{u} \mid f(\alpha_1 \vec{v}_1 + \alpha_2 \vec{v}_2) \right> = \left<\vec{u}  \mid \alpha_1 f(\vec{v}_1) + \alpha_2 f(\vec{v}_2) \right> = \alpha_1\left<\vec{u}  \mid  f(\vec{v}) \right> + \alpha_2 \left<\vec{u}  \mid f(\vec{v}_2) \right> = \alpha_1 \varphi(\vec{v}_1) + \alpha_2 \varphi(\vec{v}_2)$.
	D'où, d'après le théorème de \textsc{Riesz}, il existe un \ul{unique} vecteur $\vec{a} \in E$\/ tel que $\varphi(\vec{v}) = \left<\vec{a}  \mid \vec{v} \right>$\/ pour tout $\vec{v} \in E$.
	Ainsi, pour tout vecteur $\vec{v} \in E$, $\left<\vec{u}  \mid f(\vec{v}) \right> = \left<\vec{a}  \mid \vec{v} \right>$.
	On note $\vec{a} = f^\star(\vec{u})$.
	Soit l'application \begin{align*}
		f^\star : E &\longrightarrow E \\
		\vec{u} &\longmapsto f^\star(\vec{u}).
	\end{align*}
	La démonstration telle que $f^\star $\/ est linéaire est dans le poly.
	L'application $f^\star$\/ vérifie : $\left< \vec{u} \mid f(\vec{v}) \right> = \left<f^\star (\vec{u})  \mid \vec{v} \right>$, pour tous vecteurs $\vec{u}$\/ et $\vec{v}$.
	\textsl{Quelle est la matrice de $f^\star$, dans une base orthonormée ?}\@
	Soit $\mathcal{B}$\/ une base orthonormée de $E$, et soient $A = \big[\:f\:\big]_\mathcal{B}$, $B = \big[\:f^\star \:\big]_\mathcal{B}$, $U = \big[\:\vec{u}\:\big]_\mathcal{B}$, et $V = \big[\:\vec{v}\:\big]_\mathcal{B}$.
	Les matrices $U$\/ et $V$\/ sont des vecteurs colonnes, et $A$\/ et $B$\/ sont des matrices carrées.
	Ainsi, \[
		U^\top \cdot A \cdot V = \left< \vec{u}  \mid f(\vec{v}) \right> 
		= \left<f^\star (\vec{u})  \mid \vec{v} \right> = (B\cdot U)^\top \cdot V,
	\] ce qui est vrai quelque soit les vecteurs colonnes $U$\/ et $V$.
	D'où, $\forall U$, $\forall V$, $U^\top \cdot \big(A \cdot V\big) = U^\top \cdot \big(B^\top \cdot V\big)$.
	Ainsi, pour tous vecteurs $U$\/ et $V$, \[
		U^\top \cdot \Big[ (AV) - (B^\top V)\Big] = 0
	.\] En particulier, si $U = (AV) - (B^\top V)$, le produit scalaire $\left<\vec{u}  \mid \vec{u} \right>$\/ est nul, donc $U = 0$.
	Ainsi, \[
		\forall V,\quad A\cdot V = B^\top\cdot  V
	.\] De même, on conclut que $A = B^\top$. On en déduit donc que \[
	\big[\:f^\star\:\big]_\mathcal{B} = \big[\:f\:\big]_\mathcal{B}^\top
	.\]
\end{prv}

Les propriétés suivantes sont vrais :
\begin{itemize}
	\item $(f  \circ g)^\star  = g^\star \circ f^\star$, \quad $(f^\star)^\star = f$, \quad et \quad $(\alpha f + \beta g)^\star  = \alpha f^\star + \beta g^\star $\/ ;
	\item $(A\cdot B)^\top  = B^\top \cdot A^\top$, \quad $(A^\top)^\top = A$, \quad et \quad $(\alpha A + \beta B)^\top = \alpha A^\top+ \beta B^\top $.
\end{itemize}
Des deuxièmes et troisièmes points,  il en résulte que les applications $f \mapsto f^\star$, et $A \mapsto A^\top$\/ sont des applications involutives.



	}
	\def\addmacros#1{#1}
}

{
	\chap[9]{Grammaires non contextuelles}
	\minitoc
	\renewcommand{\cwd}{../cours/chap09/}
	\addmacros{
		\section{Motivation}

\lettrine On place au centre de la classe 40 bonbons. On en distribue un chacun. Si, par exemple, chacun choisit un bonbon et, au \textit{top} départ, prennent celui choisi.
Il est probable que plusieurs choisissent le même. Comme gérer lorsque plusieurs essaient d'accéder à la mémoire ?

Deuxièmement, sur l'ordinateur, plusieurs applications tournent en même temps. Pour le moment, on considérait qu'un seul programme était exécuté, mais, le \textsc{pc} ne s'arrête pas pendant l'exécution du programme.

On s'intéresse à la notion de \guillemotleft~processus~\guillemotright\ qui représente une tâche à réaliser.
On ne peut pas assigner un processus à une unité de calcul, mais on peut \guillemotleft~allumer~\guillemotright\ et \guillemotleft~éteindre~\guillemotright\ un processus.
Le programme allumant et éteignant les processus est \guillemotleft~l'ordonnanceur.~\guillemotright\@ Il doit aussi s'occuper de la mémoire du processus (chaque processus à sa mémoire séparée).

On s'intéresse, dans ce chapitre, à des programmes qui \guillemotleft~partent du même~\guillemotright\ : un programme peut créer un \guillemotleft~fil d'exécution~\guillemotright\ (en anglais, \textit{thread}). Le programme peut gérer les fils d'exécution qu'il a créé, et éventuellement les arrêter.
Les fils d'exécutions partagent la mémoire du programme qui les a créé.

En C, une tâche est représenté par une fonction de type \lstinline[language=c]!void* tache(void* arg)!. Le type \lstinline[language=c]!void*!\ est l'équivalent du type \lstinline[language=caml]!'a! : on peut le \textit{cast} à un autre type (comme \lstinline[language=c]-char*-).

\begin{lstlisting}[language=c,caption=Création de \textit{thread}s en C]
void* tache(void* arg) {
	printf("%s\n", (char*) arg);
	return NULL;
}

int main() {
	pthread_t p1, p2;

	printf("main: begin\n");

	pthread_create(&p1, NULL, tache, "A");
	pthread_create(&p2, NULL, tache, "B");

	pthread_join(p1, NULL);
	pthread_join(p2, NULL);

	printf("main: end\n");

	return 0;
}
\end{lstlisting}

\begin{lstlisting}[language=c,caption=Mémoire dans les \textit{thread}s en C]
int max = 10;
volatile int counter = 0;

void* tache(void* arg) {
	char* letter = arg;
	int i;

	printf("%s begin [addr of i: %p] \n", letter, &i);

	for(i = 0; i < max; i++) {
		counter = counter + 1;
	}

	printf("%s : done\n", letter);
	return NULL;
}

int main() {
	pthread_t p1, p2;

	printf("main: begin\n");

	pthread_create(&p1, NULL, tache, "A");
	pthread_create(&p2, NULL, tache, "B");

	pthread_join(p1, NULL);
	pthread_join(p2, NULL);

	printf("main: end\n");

	return 0;
}
\end{lstlisting}

Dans les \textit{thread}s, les variables locales (comme \texttt{i}) sont séparées en mémoire. Mais, la variable \texttt{counter} est modifiée, mais elle ne correspond pas forcément à $2 \times \texttt{max}$. En effet, si \texttt{p1} et \texttt{p2} essaient d'exécuter au même moment de réaliser l'opération \lstinline[language=c]-counter = counter + 1-, ils peuvent récupérer deux valeurs identiques de \texttt{counter}, ajouter 1, puis réassigner \texttt{counter}.
Ils \guillemotleft~se marchent sur les pieds.~\guillemotright\ 

Parmi les opérations, on distingue certaines dénommées \guillemotleft~atomiques~\guillemotright\ qui ne peuvent pas être séparées. L'opération \lstinline[language=c]-i++- n'est pas atomique, mais la lecture et l'écriture mémoire le sont.

\begin{defn}
	On dit d'une variable qu'elle est \textit{atomique} lorsque l'ordonnanceur ne l'interrompt pas.
\end{defn}

\begin{exm}
	L'opération \lstinline[language=c]-counter = counter + 1- exécutée en série peut être représentée comme ci-dessous. Avec \texttt{counter} valant 40, cette exécution donne 42.
	\begin{table}[H]
		\centering
		\begin{tabular}{l|l}
			Exécution du fil A & Exécution du fil B\\ \hline
			(1)~$\mathrm{reg}_1 \gets \texttt{counter}$ & (4)~$\mathrm{reg}_2 \gets \texttt{counter}$ \\
			(2)~$\mathrm{reg}_1{++}$ & (5)~$\mathrm{reg}_2{++}$ \\
			(3)~$\texttt{counter} \gets \mathrm{reg}_1$ & (6)~$\texttt{counter} \gets \mathrm{reg}_2$
		\end{tabular}
	\end{table}
	\noindent Mais, avec l'exécution en simultanée, la valeur de \texttt{counter} sera 41.
	\begin{table}[H]
		\centering
		\begin{tabular}{l|l}
			Exécution du fil A & Exécution du fil B\\ \hline
			(1)~$\mathrm{reg}_1 \gets \texttt{counter}$ & (2)~$\mathrm{reg}_2 \gets \texttt{counter}$ \\
			(3)~$\mathrm{reg}_1{++}$ & (5)~$\mathrm{reg}_2{++}$ \\
			(4)~$\texttt{counter} \gets \mathrm{reg}_1$ & (6)~$\texttt{counter} \gets \mathrm{reg}_2$
		\end{tabular}
	\end{table}
	\noindent Il y a \textit{entrelacement} des deux fils d'exécution.
\end{exm}

\begin{rmk}[Problèmes de la programmation concurrentielle]
	\begin{itemize}
		\item Problème d'accès en mémoire,
		\item Problème du rendez-vous,\footnote{Lorsque deux programmes terminent, ils doivent s'attendre pour donner leurs valeurs.}
		\item Problème du producteur-consommateur,\footnote{Certains programmes doivent ralentir ou accélérer.}
		\item Problème de l'entreblocage,\footnote{\textit{c.f.} exemple ci-après.}
		\item Problème famine, du dîner des philosophes.\footnote{Les philosophes mangent autour d'une table, et mangent du riz avec des baguettes. Ils décident de n'acheter qu'une seule baguette par personne. Un philosophe peut, ou penser, ou manger. Mais, pour manger, ils ont besoin de deux baguettes. S'ils ne mangent pas, ils meurent.}
	\end{itemize}
\end{rmk}

\begin{exm}[Problème de l'entreblocage]~

	\begin{table}[H]
		\centering
		\begin{tabular}{l|l|l}
			Fil A & Fil B & Fil C\\ \hline
			RDV(C) & RDV(A) & RDV(B)\\
			RDV(B) & RDV(C) & RDV(A)\\
		\end{tabular}
		\caption{Problème de l'entreblocage}
	\end{table}
\end{exm}

Comment résoudre le problème des deux incrementations ? Il suffit de \guillemotleft~mettre un verrou.~\guillemotright\ Le premier fil d'exécution \guillemotleft~s'enferme~\guillemotright\ avec l'expression \lstinline[language=c]!count++!, le second fil d'exécution attend que l'autre sorte pour pouvoir entrer et s'enfermer à son tour.


		\section{Continuité}

\begin{exm}
	Dans l'exercice 2, chaque fonction $f_n : t \mapsto t^n$\/ est continue sur $[0,1]$\/ mais la limite $f$\/ n'est pas continue sur $[0,1]$\/ (car elle n'est pas continue en $1$).
\end{exm}

\begin{thm}
	Soit $a$\/ un réel dans un intervalle $T$\/ de $\R$. Si une suite de fonctions $(f_n)_{n\in\N}$\/ continues en $a$\/ converge uniformément sur $T$\/ vers une fonction $f$, alors $f$\/ est aussi continue en $a$.
\end{thm}

\begin{prv}
	On suppose les fonctions $f_n$\/ continues en $a$\/ ($f_n(x) \longrightarrow f_n(a)$) et que la suite de fonctions $(f_n)_{n\in\N}$\/ converge uniformément vers $f$\/ ($\sup\:|f_n -f| \longrightarrow 0$). On veut montrer que $f$\/ est continue en $a$\/ : $f(x) \tendsto{x \to a} f(a)$, i.e.\ \[
		\forall \varepsilon > 0,\:\exists \delta > 0,\: \forall x \in T,\quad|x-a| \le \delta \implies |f(x) - f(a)| \le \varepsilon
	.\]
	Soit $\varepsilon > 0$. On calcule \[
		\big|f(x) - f(a)\big| \le \big|f(x) - f_n(x)\big| + \big|f_n(x) - f_n(a)\big| + \big|f_n(a) - f(a)\big|
	\] par inégalité triangulaire. Or, par hypothèse, il existe un rang $N \in \N$\/ (qui ne dépend pas de $x$\/ ou de $a$), tel que, $\forall n \ge N$, $\big|f(x) - f_n(x)\big| \le \frac{1}{3} \varepsilon$, et $\big|f_n(a) - f(a)\big| \le \frac{1}{3} \varepsilon$.
	De plus, par hypothèse, il existe $\delta >0$\/ tel que si $|x - a| \le \delta$, alors $|f_n(x) - f_n(a)| \le \frac{1}{3}\varepsilon$.\footnote{C'est là où l'hypothèse de la convergence uniforme est utilisée : on a besoin que le $N$\/ ne dépende pas de $x$\/ car on le fait varier.}
	On en déduit que $\big|f(x) - f(a)\big| \le \varepsilon$.
\end{prv}

\begin{crlr}
	Soit $T$\/ un intervalle de $\R$. Si une suite de fonctions $(f_n)_{n\in\N}$\/ continues sur $T$\/ converge uniformément sur $T$\/ vers une fonction continue sur $T$.
\end{crlr}

\begin{met}[Stratégie de la barrière]
	\begin{enumerate}
		\item La continuité (la dérivabilité aussi) est une propriété {\it locale}. Pour montrer qu'une fonction est continue sur un intervalle $T$, il suffit donc de montrer qu'elle est continue sur tout segment inclus dans $T$.
		\item Mais, la convergence uniforme est une propriété {\it globale}. La convergence sur tout segment inclus dans un intervalle n'implique pas la convergence uniforme sur l'intervalle (voir l'exercice 2).
		\item On n'écrit pas \[
				\substack{\ds\text{convergence uniforme}\\\ds\text{avec barrière}} \mathop{\red\implies} \substack{\ds\text{convergence uniforme}\\\ds\text{sans barrière}} \implies \substack{\ds\text{continuité}\\\ds\text{sans barrière}}
			\] mais plutôt \[
				\substack{\ds\text{convergence uniforme}\\\ds\text{avec barrière}} \implies \substack{\ds\text{continuité}\\\ds\text{avec barrière}} \implies \substack{\ds\text{continuité}\\\ds\text{sans barrière}}
			.\]
		\item Si, pour tous $a$\/ et $b$, $f$\/ est bornée sur $[a,b] \subset T$, mais cela n'implique pas que $f$\/ est bornée. Contre-exemple : la fonction $f : x \mapsto \frac{1}{x}$\/ est bornée sur tout intervalle $[a,b]$\/ avec $a$, $b \in \R^+_*$, \red{\sc mais} $f$\/ n'est pas bornée sur $]0,+\infty[$.
	\end{enumerate}
\end{met}

\begin{thm}[double-limite ou d'interversion des limites]
	Soit une suite de fonctions $(f_n)_{n\in\N}$\/ définies sur un intervalle $T$, et, soit $a$\/ une extrémité (éventuellement infinie)\footnote{autrement dit, $a \in \bar\R = \R \cup \{+\infty,-\infty\}$} de cet intervalle. Si la suite de fonctions $(f_n)_{n\in\N}$\/ converge \underline{uniformément} sur $T$\/ vers $f$\/ et si chaque fonction $f_n$\/ admet une limite finie $b_n$\/ en $a$, alors la suite de réels $b_n$\/ converge vers un réel $b$, et $\lim_{t\to a} f(t) = b$. Autrement dit, \[
		\lim_{t\to a} \Big(\underbrace{\lim_{n\to +\infty} f_n(t)}_{f(x)}\Big) = \lim_{n\to +\infty} \Big(\underbrace{\lim_{t\to a} f_n(t)}_{b_n}\Big)
	.\] \qed
\end{thm}

\begin{rmkn}
	Le théorème de la double-limite \guillemotleft~contient~\guillemotright\ le théorème 6 (théorème de préservation/transmission de la continuité), c'est un cas particulier. En effet, si les fonctions $f_n$\/ sont continues, alors \[
		\lim_{x \to a}f(x) = \underbrace{\lim_{n\to +\infty} f_n(a)}_{f(a)}
	.\]
\end{rmkn}


		\section{Endomorphismes adjoints}

\begin{defn}
	On dit qu'un endomorphisme $f : E \to E$\/ est \textit{autoadjoint} si \[
		\forall (\vec{u}, \vec{v}) \in E^2,\quad \big<f(\vec{u})\:\big|\:\vec{v}\big> = \big<\vec{u}\:\big|\:f(\vec{v})\big>
	.\] 
\end{defn}

Un endomorphisme autoadjoint est aussi appelé endomorphisme \textit{symétrique} (\textit{c.f.}\ proposition suivante). L'ensemble des endomorphismes autoadjoints est noté $\mathcal{S}(E)$.

\begin{prop}
	Un endomorphism est autoadjoint si, et seulement si la matrice de $F$\/ dans une base \ul{orthonormée} $\mathcal{B}$\/ est orthogonale.
	Autrement dit : \[
		f \in \mathcal{S}(E) \iff \big[\:f\:\big]_\mathcal{B} \in \mathcal{S}_n(\R)
	.\]
\end{prop}

\begin{prv}
	\begin{description}
		\item[$\implies$] Soit $\mathcal{B} = (\vec{\varepsilon}_1, \ldots, \vec{\varepsilon}_n)$\/ une base orthonormée de $E$. Ainsi, \[
				\forall i,\:\forall j,\quad \big<f(\vec{\varepsilon}_i)\:\big|\: \vec{\varepsilon}_j\big> = \big<\vec{\varepsilon}_i\:\big|\:f(\vec{\varepsilon}_j)\big>
			.\] On pose $\big[\:f\:\big]_{\mathcal{B}} = (a_{i,j})$\/ : \[
				\begin{pNiceMatrix}[last-col,last-row]
					\quad&\quad&a_{1,j}&\quad&\quad&\vec{\varepsilon}_i\\
					&&&&&\\
					\quad&\quad&a_{i,j}&\quad&\quad&\vec{\varepsilon}_i\\
					&&&&&\\
					\quad&\quad&a_{n,j}&\quad&\quad&\vec{\varepsilon}_n\\
					f(\vec{\varepsilon}_i)&&f(\vec{\varepsilon}_j)&&f(\vec{\varepsilon}_n)\\
				\end{pNiceMatrix}
			.\] Ainsi, $f(\vec{\varepsilon}_j) = a_{1,j} \vec{\varepsilon}_1 + \cdots + a_{i,j} \vec{\varepsilon}_i + \cdots + a_{n,j} \vec{\varepsilon}_n$. D'où, $\left<\vec{\varepsilon}_i  \mid f(\vec{\varepsilon}_j) \right> = a_{i,j}$\/ car la base $\mathcal{B}$\/ est orthonormée.
			De même avec l'autre produit scalaire, $\left< f(\vec{\varepsilon}_i)  \mid \vec{\varepsilon}_j \right>$, d'où $a_{i,j} = a_{j,i}$\/ par symétrie du produit scalaire. On en déduit que $\big[\:f\:\big]_\mathcal{B} \in \mathcal{S}_n(\R)$.
		\item[$\impliedby$]
			Si $\big[\: f\:\big]_\mathcal{B} \in \mathcal{S}_n(\R)$, alors $\left<f(\vec{\varepsilon}_i)  \mid \vec{\varepsilon}_j \right> = \left<\vec{\varepsilon}_i  \mid f(\vec{\varepsilon}_j)\right>$.
			Or, on pose $\vec{u} = x_1 \vec{\varepsilon}_1+ \cdots + x_n \vec{\varepsilon}_n$, et $\vec{v} = y_1 \vec{\varepsilon}_1 + \cdots + y_n \vec{\varepsilon}_n$.
			\begin{align*}
				\left<f(\vec{u})  \mid \vec{v} \right> &= \left<x_1 f(\vec{\varepsilon}_1) + \cdots + x_n f(\vec{\varepsilon}_n)  \mid y_1 f(\vec{\varepsilon}_1) + \cdots + y_n f(\vec{\varepsilon}_n) \right> \\
				&= \Big<\sum_{i=1}^n x_i f(\vec{\varepsilon}_i)\:\Big|\: \sum_{j=1}^n y_j \vec{\varepsilon}_j\Big> \\
				&= \sum_{i,j \in \llbracket 1,n \rrbracket}  x_i y_j \left<f(\vec{\varepsilon}_i) \mid \vec{\varepsilon}_j \right>\\
			\end{align*}
			De même en inversant $\vec{u}$\/ et $\vec{v}$.
			On en déduit donc $\left<f(\vec{u} \mid \vec{v} \right> = \left<\vec{u}  \mid f(\vec{v}) \right>$.
	\end{description}
\end{prv}

\begin{exo}
	\begin{enumerate}
		\item Si $f$\/ est autoadjoint, montrons que $\Ker f \perp \Im f$, et $\Ker f \oplus \Im f$.
			On suppose $\forall \vec{u}$, $\forall \vec{v}$, $\left<f(\vec{u}) \mid \vec{v} \right> = \left<\vec{u}  \mid f(\vec{v}) \right>$.
			Soit $\vec{u} \in \Ker f$, et soit $\vec{v} \in \Im f$.
			On sait que $f(\vec{u}) = \vec{0}$, et qu'il existe $\vec{x} \in E$\/ tel que $\vec{v} = f(\vec{x})$.
			Ainsi, \[
				\left<\vec{u}  \mid \vec{v} \right>
				= \left<\vec{u}  \mid f(\vec{x}) \right>
				= \left<f(\vec{u})  \mid \vec{x} \right>
				= 0
			.\] 
			D'où $\vec{u} \perp \vec{v}$. Ainsi, $\Ker f \perp \Im f$.


			De plus, $E$\/ est de dimension finie, d'où, d'après le théorème du rang, \[
				\dim \Ker f + \dim \Im f = \dim E
			.\] Aussi, $\Ker f \oplus (\Ker f)^\perp = E$, donc $\dim(\Ker f) + \dim(\Ker f)^\perp = \dim E$.
			On en déduit donc que $\dim(\Im f)= \dim(\Ker f)^\perp$.
			Or, $\Im f \subset (\Ker f)^\perp$\/ car $\Im f \perp \Ker f$.
			Ainsi $\Im f = (\Ker f)^\perp$, on en déduit que \[
				\Im f \oplus \Ker f = E
			.\]
			\begin{description}
				\item[$\impliedby$] 
					Soit $p$\/ la projection sur $F$\/ parallèlement à $G$.
					Supposons l'endomorphisme $P$\/ autoadjoint.
					D'après la question 1., le $\Ker p \perp \Im p$.
					Ainsi, $F = \Im p$\/ et $G = \Ker p$.
					D'où, $F \perp G$, $p$\/ est donc une projection orthogonale.
				\item[$\implies$]
					Réciproquement, supposons $p$\/ une projection orthogonale.
					Soit $\mathcal{B} = (\vec{\varepsilon}_1, \ldots, \vec{\varepsilon}_q)$\/ une base orthonormée de $F$.
					Ainsi, pour tout $\vec{x} \in E$, \[
						p(\vec{x}) = \sum_{i = 1}^q \left<\vec{x}  \mid \vec{\varepsilon}_i \right>\,\vec{\varepsilon}_i
					.\] 
					On veut montrer que l'endomorphisme $p$\/ est autoadjoint.
					Soient $\vec{u}$\/ et $\vec{v}$\/ deux vecteurs de $E$.
					\begin{align*}
						\left<p(\vec{u})  \mid \vec{v} \right>
						= \Big<\sum_{i=1}^q \left< \vec{u}\mid \vec{\varepsilon}_i \right>\vec{\varepsilon}_i\:\Big|\; \vec{v}\;\Big>
						&= \sum_{i=1}^q \left<u  \mid \vec{\varepsilon}_{i} \right>\: \left< \vec{\varepsilon}_i  \mid v\right>\\
						&= \sum_{i=1}^q \left<v  \mid \vec{\varepsilon}_i \right>\:\left<\vec{\varepsilon}_i   \mid u\right> \\
						&= \left< \vec{u}  \mid p(\vec{v}) \right> \\
					\end{align*}

					Autre méthode, pour tous vecteurs $\vec{u}$\/ et $\vec{v}$\/ de $E$,
					\begin{align*}
						\left<p(\vec{u})  \mid \vec{v} \right>
						&= \left<p(\vec{u})  \mid p(\vec{v}) + \vec{v} - p(\vec{v}) \right> \\
						&= \left< p(\vec{u})  \mid p(\vec{v}) \right> + \left<p(\vec{u})  \mid  \vec{v} - p(\vec{v}) \right> \\
						&= \left<p(\vec{u}) \mid p(\vec{v}) \right> + \left<u - p(\vec{u})  \mid p(\vec{v}) \right> \\
						&= \left<\vec{u}  \mid p(\vec{v}) \right> \\
					\end{align*}
					car $p$\/ est orthogonale.
			\end{description}
	\end{enumerate}
\end{exo}


\begin{prop-defn}
	Si $f$\/ est un endomorphisme d'un espace euclidien $E$, alors il existe un unique endomorphisme de $E$, noté $f^\star$\/ et appelé l'\textit{adjoint} de $f$, tel que \[
		\forall (\vec{u},\vec{v}) \in E^2,\quad\quad \left<f^\star(\vec{u})  \mid \vec{v} \right> =  \left<\vec{u}  \mid f(\vec{v}) \right>
	.\] 
	Si $A$\/ est la matrice $f$\/ dans une base orthonormée $\mathcal{B}$\/ de $E$, alors $A^\top$\/ est la matrice de $f^\star$\/ dans~$\mathcal{B}$\/ : \[
		\big[\:f\:\big]_\mathcal{B} = \big[\:f\:\big]_\mathcal{B}^\top
	.\]
\end{prop-defn}

\begin{prv}
	Soit $\vec{u} \in E$. L'application \begin{align*}
		\varphi: E &\longrightarrow \R \\
		\vec{v} &\longmapsto \left<\vec{u}  \mid f(\vec{v}) \right>.
	\end{align*}
	La forme $\varphi$\/ est linéaire car $\varphi(\alpha_1 \vec{v}_1 + \alpha_2 \vec{v}_2) = \left<\vec{u} \mid f(\alpha_1 \vec{v}_1 + \alpha_2 \vec{v}_2) \right> = \left<\vec{u}  \mid \alpha_1 f(\vec{v}_1) + \alpha_2 f(\vec{v}_2) \right> = \alpha_1\left<\vec{u}  \mid  f(\vec{v}) \right> + \alpha_2 \left<\vec{u}  \mid f(\vec{v}_2) \right> = \alpha_1 \varphi(\vec{v}_1) + \alpha_2 \varphi(\vec{v}_2)$.
	D'où, d'après le théorème de \textsc{Riesz}, il existe un \ul{unique} vecteur $\vec{a} \in E$\/ tel que $\varphi(\vec{v}) = \left<\vec{a}  \mid \vec{v} \right>$\/ pour tout $\vec{v} \in E$.
	Ainsi, pour tout vecteur $\vec{v} \in E$, $\left<\vec{u}  \mid f(\vec{v}) \right> = \left<\vec{a}  \mid \vec{v} \right>$.
	On note $\vec{a} = f^\star(\vec{u})$.
	Soit l'application \begin{align*}
		f^\star : E &\longrightarrow E \\
		\vec{u} &\longmapsto f^\star(\vec{u}).
	\end{align*}
	La démonstration telle que $f^\star $\/ est linéaire est dans le poly.
	L'application $f^\star$\/ vérifie : $\left< \vec{u} \mid f(\vec{v}) \right> = \left<f^\star (\vec{u})  \mid \vec{v} \right>$, pour tous vecteurs $\vec{u}$\/ et $\vec{v}$.
	\textsl{Quelle est la matrice de $f^\star$, dans une base orthonormée ?}\@
	Soit $\mathcal{B}$\/ une base orthonormée de $E$, et soient $A = \big[\:f\:\big]_\mathcal{B}$, $B = \big[\:f^\star \:\big]_\mathcal{B}$, $U = \big[\:\vec{u}\:\big]_\mathcal{B}$, et $V = \big[\:\vec{v}\:\big]_\mathcal{B}$.
	Les matrices $U$\/ et $V$\/ sont des vecteurs colonnes, et $A$\/ et $B$\/ sont des matrices carrées.
	Ainsi, \[
		U^\top \cdot A \cdot V = \left< \vec{u}  \mid f(\vec{v}) \right> 
		= \left<f^\star (\vec{u})  \mid \vec{v} \right> = (B\cdot U)^\top \cdot V,
	\] ce qui est vrai quelque soit les vecteurs colonnes $U$\/ et $V$.
	D'où, $\forall U$, $\forall V$, $U^\top \cdot \big(A \cdot V\big) = U^\top \cdot \big(B^\top \cdot V\big)$.
	Ainsi, pour tous vecteurs $U$\/ et $V$, \[
		U^\top \cdot \Big[ (AV) - (B^\top V)\Big] = 0
	.\] En particulier, si $U = (AV) - (B^\top V)$, le produit scalaire $\left<\vec{u}  \mid \vec{u} \right>$\/ est nul, donc $U = 0$.
	Ainsi, \[
		\forall V,\quad A\cdot V = B^\top\cdot  V
	.\] De même, on conclut que $A = B^\top$. On en déduit donc que \[
	\big[\:f^\star\:\big]_\mathcal{B} = \big[\:f\:\big]_\mathcal{B}^\top
	.\]
\end{prv}

Les propriétés suivantes sont vrais :
\begin{itemize}
	\item $(f  \circ g)^\star  = g^\star \circ f^\star$, \quad $(f^\star)^\star = f$, \quad et \quad $(\alpha f + \beta g)^\star  = \alpha f^\star + \beta g^\star $\/ ;
	\item $(A\cdot B)^\top  = B^\top \cdot A^\top$, \quad $(A^\top)^\top = A$, \quad et \quad $(\alpha A + \beta B)^\top = \alpha A^\top+ \beta B^\top $.
\end{itemize}
Des deuxièmes et troisièmes points,  il en résulte que les applications $f \mapsto f^\star$, et $A \mapsto A^\top$\/ sont des applications involutives.



		\begin{prop-defn}
  Soit $(\Omega, \mathcal{A}, P)$\/ un espace probabilisé, et soit $X$\/ une \textit{vard}. Si $X^2$\/ est d'espérance finie, alors $X$\/ aussi, et on appelle \textit{variance} le réel positif \[
    \mathrm{V}(X) = \mathrm{E}\Big(\big[X - \mathrm{E}(X)\big]^2\Big) = \underbrace{\mathrm{E}(X^2) - \big(\mathrm{E}(X)\big)^2}_{\mathclap{\text{Relation de \textsc{König \& Huygens}}}} \ge 0
  .\]
  L'\textit{écart-type} $\sigma(X)$\/ est la racine carrée de la variance : \[
    \sigma(X) = \sqrt{\mathrm{V}(X)}
  .\]
\end{prop-defn}

\begin{prv}
  On pose $\mu = \mathrm{E}(X)$, et on a $[X - \mu]^2 = X^2 - 2 \mu X + \mu^2$. D'où, par linéarité de l'espérance,
  \begin{align*}
    \mathrm{E}\big((X-\mu)^2\big)
    &= \mathrm{E}(X^2 - 2\mu X + \mu^2) \\
    &= \mathrm{E}(X^2) - 2\mu \mathrm{E}(X) + \mu^2 \\
    &= \mathrm{E}(X^2) - 2\mu^2 + \mu^2 \\
    &= \mathrm{E}(X^2) - \big(\mathrm{E}(X)\big)^2. \\
  \end{align*}
  De plus, d'après le lemme précédent, si $X^2$\/ est d'espérance finie, alors $X$\/ est d'espérance finie.
\end{prv}

\begin{rmk}
  \begin{enumerate}
    \item La variance mesure la \textit{dispersion}, ou l'\textit{étalement} des valeurs $a_i$\/ autour de l'espérance $\mathrm{E}(X)$. En particulier, s'il existe $a \in \R$\/ tel que $P(X = a) = 1$, alors $\mathrm{E}(X) = a$\/ et $\mathrm{V}(X) = 0$. (C'est même une équivalence.)
    \item Si la variable $X$\/ a une unité ($\mathrm{km}/\mathrm{s}$, $\mathrm{V}/\mathrm{m}$, etc.), alors l'écart type a la même unité (d'où l'intérêt de calculer la racine carrée de la variance).
    \item Soient $\alpha$\/ et $\beta$\/ deux réels. Si  $X^2$\/ est d'espérance finie, alors \[
      \mathrm{V}(\alpha X + \beta) = \alpha^2\cdot  \mathrm{V}(X)
    .\]
    (Une translation ne change pas la dispersion des valeurs, et multiplier par un réel multiplie l'espérance, mais aussi la dispersion, d'où le carré.)
  \end{enumerate}
\end{rmk}

\begin{exo}
  \textsl{Montrer que
  \begin{enumerate}
    \item si $X \sim \mathcal{B}(n, p)$, alors $X^2$\/ est d'espérance finie et $\mathrm{V}(X) = n\cdot p\cdot q$.
    \item si $T \sim \mathcal{G}(p)$, alors $T^2$\/ est d'espérance finie et $\mathrm{V}(T) = \frac{q}{p^2}$.
    \item si $X \sim \mathcal{P}(\lambda)$, alors $X^2$\/ est d'espérance finie et $\mathrm{V}(X) = \lambda$.
  \end{enumerate}
  }

  \begin{enumerate}
    \item
      Si $X \sim \mathcal{B}(n,p)$, alors $X(\Omega) = \llbracket 0,n \rrbracket$\/ et, pour $k \in X(\Omega)$, $P(X = k) = {n\choose k}\,p^k\,q^{n-k}$.
      On a déjà montré que $\mathrm{E}(X) = n\cdot p$.
      On va montrer que $\mathrm{V}(X) = n\,p\,q$.
      La variable aléatoire $X^2$\/ est d'espérance finie car $X(\Omega)$\/ est fini.
      Et,
      \begin{align*}
        \mathrm{E}(X^2) &= \sum_{k=0}^n k^2\: P(X = k)\\
        &= \sum_{k=0}^n k^2 {n\choose p} p^k q^{n-k} \\
        &= \ldots \\
      \end{align*}
      En effet, d'après la ``petite formule,'' on a \[
        \forall k \ge 1,\quad k{n\choose k} = n{n-1 \choose k-1}
      \] d'où, $(k-1) {n-1\choose k-1} = (n-1) \choose {n-2 \choose k-2}$. Ainsi, \[
        \forall k \ge 2,\quad k(k-1){n\choose k} = n(n+1) {n-2\choose k-2}
      .\] 
    \item Si $T \sim \mathcal{G}(p)$, alors $T(\Omega) = \N^*$\/ et $\forall k \in T(\Omega)$, $P(T = k) = p \times q^{k-1}$.
      On a déjà prouvé que $\mathrm{E}(T) = \frac{1}{p}$.
      On veut montrer que $\mathrm{V}(T) = \frac{q}{p^2}$. Montrons que la variable $T^2$\/ possède une espérance : la série $\sum k^2\: P(T = k)$\/ converge absolument car $k^2 \:P(T = k) = k^2 \cdot p \cdot q^{k-1}$.
      Or, pour $k \ge 2$, $\frac{\mathrm{d}^2}{\mathrm{d}x^2} x^k = k(k-1)\,x^{k-2}$. Et, on peut dériver terme à terme une série entière sans changer son rayon de convergence, et la série $\sum x^k$\/ a pour rayon de convergence 1. D'où, $\sum k(k-1)\, x^{k-2}$\/ a pour rayon de convergence 1. Or, $q \in {]0,1[} \subset {]-1,1[}$\/ donc la série $\sum k(k-1)q^{k-2}$\/ converge. De plus, $\sum k (k-1)\, q^{k-2} = \sum k^2\,q^{k-2} - \sum k\,q^{k-2}$.
      D'où,  $\sum k^2 q^{k-2} = \sum k(k-1)\, q^{k-2} + \sum k\,q^{k-2}$, qui converge. Par suite,
      \begin{align*}
        \sum_{k=1}^\infty k^2\, P(T = k) &= \sum_{k=1}^\infty k^2\,p\,q^{k-1}\\
        &= p + pq \sum_{k=2}^\infty k^2 q^{k-2} \\
        &= p + pq \sum_{k=2}^\infty k(k-1)\,q^{k-2} + p\sum_{k=2}^\infty k\,q^{k-1} \\
        &= p + pq\, \frac{2}{(1-q)^3} + p\left(\frac{1}{(1-q)^2} - 1 \right) \rlap{\quad\quad \text{\textit{c.f.} en effet après}}\\
        &= p + pq\, \frac{2}{p^3} + p\left( \frac{1}{p^2} - 1 \right) \\
        &= \frac{2q}{p^2} + \frac{1}{p} \\
        &= \frac{2q + p}{p^2} \\
        &= \frac{2q + (1-q)}{p^2} \\
        &= \frac{q+1}{p^2}. \\
      \end{align*}
      En effet, $\forall x \in {]-1,1[}$, $\sum_{k=0}^\infty x^k = \frac{1}{1-x}$. D'où, pour $x \in {]-1,1[}$, \[
        \sum_{k=1}^\infty k\,x^{k-1} = \frac{1}{(1-x)^2}
        \quad \text{ et }\quad
        \sum_{k=2}^\infty k(k-1)\,x^{k-2} = \frac{2}{(1-x)^3}
      .\]
      Ainsi, $\mathrm{E}(T^2) = \frac{q^{+1}}{p^2}$. D'où
      \begin{align*}
        \mathrm{V}(T) &= \mathrm{E}(T^2) - \big(\mathrm{E}(T)\big)^2 \\
        &= \frac{q+1}{p^2} - \left( \frac{1}{p} \right)^2 \\
        &= \frac{q}{p^2} \\
      \end{align*}
    \item À tenter
  \end{enumerate}
\end{exo}


\section{Les inégalités de \textsc{Markov} et de \textsc{Bienaymé}--\textsc{Tchebychev}, inégalités de concentration}

\begin{lem}[Markov]
  Soit $(\Omega, \mathcal{A}, P)$\/ un espace probabilisé, et soit $X$\/ une variable aléatoire \underline{positive}.
  Si $X$\/ est d'espérance finie, alors \[
    \forall a > 0, \quad P(X \ge a) \le \frac{\mathrm{E}(X)}{a}
  .\]
\end{lem}

\begin{prv}
  On suppose $X$\/ d'espérance finie. Ainsi, on a \[
    \mathrm{E}(X) = \sum_{x \in X(\Omega)} x\:P(X = x)
  .\]
  Soit $I$\/ l'ensemble $I = \{x \in X(\Omega) \mid x \ge a\}$.
  Alors, \[
    \mathrm{E}(X) = \underbrace{\sum_{x \in I} x\:P(X = x)}_{\text{ ici } x \ge a} + \underbrace{\sum_{x \in X(\Omega) \setminus I} x\:P(X = x)}_{\ge 0 \text{ par hypothèse}}
  .\]
  D'où, \[
    \mathrm{E}(X) \ge \sum_{x \in I} x\:P(X = x) \ge \sum_{x \in I} a\: P(X = x) = a \sum_{x \in I} P(X = x) \ge a\:P(x \ge a)
  .\]
\end{prv}

\begin{prop}[\textsc{Bienaymé--Tchebychev}]
  Soit $(\Omega, \mathcal{A}, P)$\/ un espace probabilisé, et soit $X$\/ une \textit{vard}. Si $X^2$\/ est d'espérance finie, alors \[
    \forall a > 0, \quad\quad P\Big(\big|X - \mathrm{E}(X)\big| \ge a\Big) \le \frac{\mathrm{V}(X)}{a^2}
  .\]
\end{prop}

\begin{prv}
  On pose $\mu = \mathrm{E}(X)$.
  L'événement $\big(|X - \mu| \ge a\big) = \big((X - \mu)^2 \ge a^2\big)$, d'où, les probabilités \[
    P\big(|X - \mu| \ge a\big) = P\big(\underbrace{(X - \mu)^2}_{\ge 0} \ge \underbrace{a^2}°{\ge 0}\big).
  \] On valide donc \textit{une} des hypothèses de l'inégalité de \textsc{Markov}.
  De plus, l'autre hypothèse est vérifiée : $X^2$\/ est d'espérance finie, donc $(X - \mu)^2$\/ aussi. On en déduit, d'après le lemme de \textsc{Markov}, que \[
    P\big((X-\mu)^2 \ge a^2\big) \le \frac{\mathrm{E}\big((X-\mu)^2\big)}{a^2} = \frac{\mathrm{V}(X)}{a^2}
  .\]
\end{prv}

\section{Série génératrice}

\begin{defn}
  Soit $X$\/ une \textit{vad} telle que $X(\Omega) \subset \N$. La \textit{série génératrice} de $X$\/ est la série entière $\sum a_n x^n$ de coefficients $a_n = P(X = n)$.
\end{defn}


La série $\sum a_n$\/ converge car sa somme vaut $\sum_{n=0}^\infty a_n = 1$. D'où, 
\begin{itemize}
  \item le rayon de convergence $R$\/ de la série est supérieur ou égal à 1.
  \item la série génératrice converge normalement sur $[-1,1]$, car la série $\sum |a_n|$\/ converge, or, $\forall x \in [-1,1]$, $|p_nt^n| \le |p_n|$, d'où la convergence normale.
    D'où la \textit{fonction génératrice} \[
      \mathrm{G}_X \colon t \longmapsto \sum_{n=0}^\infty p_n t^n
    \] est définie et même continue sur $[-1,1]$, car la convergence est uniforme.
  \item la fonction génératrice $\mathrm{G}_X$\/ est de classe $\mathcal{C}^\infty$\/ sur $]-1,1[$\/ et \[
      \forall k \in \N,\quad P(X = k) = a_n \frac{{\mathrm{G}_X}^{(k)}(0)}{k!}
    .\] La fonction génératrice de $X$\/ permet donc de retrouver la loi de probabilité de $X$.
\end{itemize}


		\begin{exo}
	Soient $\lambda_1, \ldots, \lambda_r \in \R$\/ distincts deux à deux.
	Montrons que, si $\forall x \in \R$, $\alpha_1 \mathrm{e}^{\lambda_1 x} + \cdots + \alpha_r \mathrm{e}^{\lambda_r x} = 0$, alors $\alpha_1 = \cdots = \alpha_r$.
	On peut procéder de différentes manières : le déterminant de {\sc Vandermonde}, par analyse-sythèse, ou, en utilisant \[
		\frac{\mathrm{d}}{\mathrm{d}x}\left( \mathrm{e}^{\lambda_k x} \right) = \lambda_k \mathrm{e}^{\lambda_k x},\quad\text{d'où}\quad\varphi(f_k) = \lambda_k f_k, \text{ avec } f_k : x \mapsto \mathrm{e}^{\lambda_k x}\quad\text{et}\quad\varphi:f\mapsto f'
	.\]
	On doit vérifier que les $f_k$\/ sont des vecteurs et l'application $\varphi$\/ soit un endomorphisme. On se place donc dans l'espace vectoriel $\mathscr{C}^{\infty}$. (On ne peut pas se placer dans l'espace $\mathscr{C}^k$, car sinon l'application $\varphi$\/ est de l'espace $\mathscr{C}^k$\/ à $\mathscr{C}^{k-1}$, ce n'est donc pas un endomorphisme ; ce n'est pas le cas pour l'espace $\mathscr{C}^\infty$.)
	Or, les $\lambda_k$\/ sont distincts deux à deux d'où les vecteurs propres $f_k$\/ sont linéairement indépendants. Et donc si $\alpha_1 f_1 + \alpha_2 f_2 + \cdots + \alpha_r f_r = 0$\/ alors $\alpha_1 = \cdots = \alpha_r=0$.
	Mais, comme $\forall x \in \R$, $\alpha_1 f_1(x) + \alpha_2 f_2(x) + \cdots + \alpha_r f_r(x) = 0$, on en déduit que \[
		\boxed{\alpha_1 = \cdots = \alpha_r = 0.}
	\]
\end{exo}

\section{Critères de diagonalisabilité}

\begin{prop}[une condition \underline{suffisante} pour qu'une matrice soit diagonalisable]
	Soit $A$\/ une matrice carrée de taille $n \ge 2$. {\color{red}Si} $A$\/ possède $n$\/ valeurs propres distinctes deux à deux, {\color{red}alors} $A$\/ est diagonalisable.
\end{prop}

\begin{rmkn}
	La réciproque est fausse : par exemple, pour $n > 1$, $7 I_n$\/ est diagonalisable car elle est diagonale. Mais, elle ne possède pas $n$\/ valeurs propres distinctes deux à deux.
\end{rmkn}

\begin{prv}
	On suppose que la matrice $A \in \mathscr{M}_{n,n}(\mathds{K})$\/ possède $n$\/ valeurs propres distinctes deux à deux (i.e.~$\Card \Sp(A) = n$). D'où, d'après la proposition 16, les $n$\/ vecteurs propres associés $\varepsilon_1,\ldots,\varepsilon_n$\/ sont libres. D'où $(\varepsilon_1, \ldots, \varepsilon_n)$\/ est une base formée de vecteurs propres. Donc, d'après la définition 5, la matrice $A$\/ est diagonalisable.
\end{prv}

\begin{thm}[conditions \underline{nécessaires et suffisantes} pour qu'une matrice soit diagonalisable]
	Soient $E$\/ un espace vectoriel de dimension finie et $u : E \to E$\/ un endomorphisme.
	Alors,
	\begin{align*}
		(1)\quad u \text{ diagonalisable } \iff& E = \bigoplus_{\lambda \in \Sp(u)} \Ker(\lambda\id_E - u) \quad(2)\\
		\iff& \dim E = \sum_{\lambda \in \Sp(u)} \dim(\mathrm{SEP}(\lambda))\quad(3)\\
		\iff& \chi_u \text{ scindé et } \forall \lambda \in \Sp(u),\:\dim(\mathrm{SEP}(\lambda)) = m_\lambda\quad(4)
	\end{align*}
	où $m_\lambda$\/ est la multiplicité de la racine $\lambda$\/ du polynôme $\chi_u$.
\end{thm}

\begin{prv}
	\begin{itemize}
		\item[``$(1)\implies(2)$''] On suppose $u$\/ diagonalisable. Il existe donc une base $(\varepsilon_1$, \ldots, $\varepsilon_n)$\/ de $E$\/ formée de vecteurs propres de $u$. On les regroupes par leurs valeurs propres : $(\varepsilon_i, \ldots, \varepsilon_{i+j})$\/ forme une base de $\mathrm{SEP(\lambda_k)}$. D'où la base $(\varepsilon_1, \ldots, \varepsilon_n)$\/ de l'espace vectoriel $E$\/ est une concaténation des bases des sous-espaces propres de $u$. D'où \[
				E = \bigoplus_{\lambda \in \Sp(u)} \mathrm{SEP}(\lambda)
			.\]
		\item[``$(2)\implies(1)$''] On suppose que $E = \mathrm{SEP}(\lambda_1) \oplus \mathrm{SEP}(\lambda_2) \oplus \cdots \oplus \mathrm{SEP}(\lambda_r)$.
			Soient $(\varepsilon_1, \ldots, \varepsilon_{d_1})$\/ une base de $\mathrm{SEP}(\lambda_1)$, $(\varepsilon_{d_1 + 1}, \ldots, \varepsilon_{d_1 + d_2})$\/ une base de $\mathrm{SEP}(\lambda_2)$, \ldots, $(\varepsilon_{d_1+\cdots + d_{r-1}+1}$, \ldots, $\varepsilon_{d_1+ \cdots + d_r})$\/ une base de $\mathrm{SEP}(\lambda_r)$.
			En concaténant ces base, on obtient une base de $E$, d'après l'hypothèse. Dans cette base, tous les vecteurs sont propres donc $u$\/ est diagonalisable.
		\item[``$(2)\implies(3)$''] On suppose $E = \mathrm{SEP}(\lambda_1) \oplus \mathrm{SEP}(\lambda_2) \oplus \cdots \oplus \mathrm{SEP}(\lambda_r)$. D'où  \[
					\dim E = \dim(\mathrm{SEP}(\lambda_1)) + \dim(\mathrm{SEP}(\lambda_2)) + \cdots + \dim(\mathrm{SEP}(\lambda_r))
			\] car la dimension d'une somme directe est égale à la somme des dimensions.
		\item[``$(3)\implies(1)$''] On suppose $\dim E = \dim(\mathrm{SEP}(\lambda_1)) + \dim(\mathrm{SEP}(\lambda_2)) + \cdots + \dim(\mathrm{SEP}(\lambda_r))$. Or, les sous-espaces propres sont en somme directe, d'après la proposition 16. D'où $\dim\Big(\sum_{\lambda \in \Sp(u)} \mathrm{SEP}(\lambda) \Big)= \sum_{\lambda \in \Sp(u)} \dim(\mathrm{SEP}(\lambda))$. Donc $\sum_{\lambda \in \Sp(u)} \mathrm{SEP}(\lambda) = E$.
		\item[``$(4)\implies(3)$''] On suppose (a) $\chi_u$\/ scindé et (b) $\dim(\mathrm{SEP}(\lambda)) = m_\lambda$. D'où, d'après (a): \[
				\chi_u(x) = (x - \lambda_1)^{m_{\lambda_1}}(x - \lambda_2)^{m_{\lambda_2}} \cdots (x - \lambda_r)^{m_{\lambda_r}} = x^n + \cdots
			\] d'où $m_{\lambda_1} + m_{\lambda_2} + \cdots + m_{\lambda_r} = n$, et d'où \[
				\dim(\mathrm{SEP}(\lambda_1)) + \dim(\mathrm{SEP}(\lambda_2)) + \cdots + \dim(\mathrm{SEP}(\lambda_r)) = n
			\] d'après l'hypothèse (b).
		\item[``$(1)\implies(4)$'']
			On suppose $u$\/ diagonalisable. D'où, dans une certaine base $\mathscr{B}$, la matrice $\big[u\big]_\mathscr{B}$\/ est diagonale. Quitte à changer l'ordre des éléments de $\mathscr{B} = (\varepsilon_1,\ldots,\varepsilon_r)$, on peut supposer que $\big[u\big]_\mathscr{B}$\/ est de la forme \[
				\big[u\big]_\mathscr{B} = 
				\begin{bNiceArray}{c|c|c|c}[last-col]
					\begin{array}{cccc}\lambda_1\\&\lambda_1\\&&\ddots\\&&&\lambda_1\end{array}&0&0&0&\begin{array}{l}\varepsilon_1\\\varepsilon_2\\\vdots\\\varepsilon_{d_1}\\\end{array}\\ \hline
					0&\begin{array}{cccc}\lambda_2\\&\lambda_2\\&&\ddots\\&&&\lambda_2\end{array}&0&0&\begin{array}{l}\varepsilon_{d_1+1}\\\varepsilon_{d_1+2}\\\vdots\\\varepsilon_{d_1+d_2}\\\end{array}\\ \hline
					 &&\ddots&&\vdots\\ \hline
					0&0&0&0\begin{array}{cccc}\lambda_r\\&\lambda_r\\&&\ddots\\&&&\lambda_r\end{array}&\begin{array}{l}\varepsilon_{d_1+\cdots + d_{r-1} + 1}\\\varepsilon_{d_1+\cdots + d_{r-1}+2}\\\vdots\\\varepsilon_{d_1 + \cdots + d_r}\\\end{array}\\
				\end{bNiceArray}
			.\] D'où $\forall k \in \left\llbracket 1,r \right\rrbracket$, $d_k = \dim(\mathrm{SEP}(\lambda_k))$. En outre, $\chi_u(x) = \det(x\id - u) = (x - \lambda_1)^{d_1}\cdot (x-\lambda_2)^{d_2} \cdots (r - \lambda_r)^{d_r}$.
			D'où $\forall k \in \left\llbracket 1,r \right\rrbracket$, $d_k = m_{\lambda_k}$\/ et $\chi_u$\/ est scindé.
	\end{itemize}
\end{prv}

\begin{exo}
	{\slshape On considère la matrice $E$\/ ci-dessous \[
		E = \begin{pmatrix}
			7&0&1\\
			0&3&0\\
			0&0&7
		\end{pmatrix}.
	\] La matrice $E$\/ ci-dessous est-elle diagonalisable ?}

	Soit $\lambda \in \R$. On sait que $\lambda \in \Sp(E)$\/ si et seulement si $\det(\lambda I_3 - E) = 0$. Or \[
		\det(\lambda I_3 - E) =
		\begin{vmatrix}
			\lambda - 7&0&-1\\
			0&\lambda-3&0\\
			0&0&\lambda - 7
		\end{vmatrix} = (\lambda - 7)^2\cdot  (\lambda - 3)^1
	.\] Donc $\Sp(E) = \{3,7\}$, $1 \le \dim\big(\mathrm{SEP}(3)\big) \le 1$, et $1 \le \dim\big(\mathrm{SEP}(7)\big) \le 2$.
	La matrice $E$\/ est diagonalisable si et seulement si $\dim(\mathrm{SEP}(3)) + \dim(\mathrm{SEP}(7)) = 3$, donc si et seulement si $\dim(\mathrm{SEP}(7)) = 2$. On cherche donc la dimension de ce sous-espace propre : soit $X = \left( \substack{x\\y\\z} \right) \in \mathscr{M}_{3,1}(\R)$. On sait que
	\begin{align*}
		X \in \mathrm{SEP}(7) \iff& E\cdot X = 7X\\
		\iff& \begin{pmatrix}
			7&0&1\\
			0&3&0\\
			0&0&7
		\end{pmatrix} \begin{pmatrix}
			x\\y\\z
		\end{pmatrix} = 7 \begin{pmatrix}
			x\\y\\z
		\end{pmatrix}\\
		\iff& \begin{cases}
			7x + 0y + 1z = 7x\\
			3y = 7y\\
			7z = 7z
		\end{cases}\\
		\iff& \begin{cases}
			z = 0\\
			y = 0
		\end{cases}\\
		\iff& X = \begin{pmatrix}
			x\\0\\y
		\end{pmatrix} = x \underbrace{\begin{pmatrix}
			1\\0\\0
		\end{pmatrix}}_{\varepsilon_1}
	\end{align*}
	Donc $\mathrm{SEP}(7) = \Vect(\varepsilon_1)$, d'où $\dim(\mathrm{SEP}(7)) = 1$. Donc la matrice $E$\/ n'est pas diagonalisable.
\end{exo}

\section{Trigonalisation}

Trigonaliser une matrice ne sert que si la matrice n'est pas diagonalisable.

\begin{defn}
	On dit d'une matrice carrée $A \in \mathscr{M}_{n,n}(\mathds{K})$\/ qu'elle est {\it trigonalisable}\/ s'il existe une matrice inversible $P$\/ telle que $P^{-1} \cdot A \cdot P$\/ est triangulaire : \[
		P^{-1} \cdot A \cdot P = \begin{pNiceMatrix}
			\lambda_1&\Block{2-2}*&\\
			\Block{2-2}0&\Ddots&\\
			&&\lambda_r
		\end{pNiceMatrix}
	.\]
\end{defn}

\begin{rmk}
	$\O$\/
\end{rmk}

\begin{thm}
	Une matrice carrée $A \in \mathscr{M}_{n,n}(\mathds{K})$\/ est trigonalisable si et seulement si son polynôme caractéristique $\chi_A \in \mathds{K}[X]$\/ est scindé.
\end{thm}

	}
	\def\addmacros#1{#1}
}

{
	\chap[10]{Concurrence}
	\minitoc
	\renewcommand{\cwd}{../cours/chap10/}
	\addmacros{
		\section{Motivation}

\lettrine On place au centre de la classe 40 bonbons. On en distribue un chacun. Si, par exemple, chacun choisit un bonbon et, au \textit{top} départ, prennent celui choisi.
Il est probable que plusieurs choisissent le même. Comme gérer lorsque plusieurs essaient d'accéder à la mémoire ?

Deuxièmement, sur l'ordinateur, plusieurs applications tournent en même temps. Pour le moment, on considérait qu'un seul programme était exécuté, mais, le \textsc{pc} ne s'arrête pas pendant l'exécution du programme.

On s'intéresse à la notion de \guillemotleft~processus~\guillemotright\ qui représente une tâche à réaliser.
On ne peut pas assigner un processus à une unité de calcul, mais on peut \guillemotleft~allumer~\guillemotright\ et \guillemotleft~éteindre~\guillemotright\ un processus.
Le programme allumant et éteignant les processus est \guillemotleft~l'ordonnanceur.~\guillemotright\@ Il doit aussi s'occuper de la mémoire du processus (chaque processus à sa mémoire séparée).

On s'intéresse, dans ce chapitre, à des programmes qui \guillemotleft~partent du même~\guillemotright\ : un programme peut créer un \guillemotleft~fil d'exécution~\guillemotright\ (en anglais, \textit{thread}). Le programme peut gérer les fils d'exécution qu'il a créé, et éventuellement les arrêter.
Les fils d'exécutions partagent la mémoire du programme qui les a créé.

En C, une tâche est représenté par une fonction de type \lstinline[language=c]!void* tache(void* arg)!. Le type \lstinline[language=c]!void*!\ est l'équivalent du type \lstinline[language=caml]!'a! : on peut le \textit{cast} à un autre type (comme \lstinline[language=c]-char*-).

\begin{lstlisting}[language=c,caption=Création de \textit{thread}s en C]
void* tache(void* arg) {
	printf("%s\n", (char*) arg);
	return NULL;
}

int main() {
	pthread_t p1, p2;

	printf("main: begin\n");

	pthread_create(&p1, NULL, tache, "A");
	pthread_create(&p2, NULL, tache, "B");

	pthread_join(p1, NULL);
	pthread_join(p2, NULL);

	printf("main: end\n");

	return 0;
}
\end{lstlisting}

\begin{lstlisting}[language=c,caption=Mémoire dans les \textit{thread}s en C]
int max = 10;
volatile int counter = 0;

void* tache(void* arg) {
	char* letter = arg;
	int i;

	printf("%s begin [addr of i: %p] \n", letter, &i);

	for(i = 0; i < max; i++) {
		counter = counter + 1;
	}

	printf("%s : done\n", letter);
	return NULL;
}

int main() {
	pthread_t p1, p2;

	printf("main: begin\n");

	pthread_create(&p1, NULL, tache, "A");
	pthread_create(&p2, NULL, tache, "B");

	pthread_join(p1, NULL);
	pthread_join(p2, NULL);

	printf("main: end\n");

	return 0;
}
\end{lstlisting}

Dans les \textit{thread}s, les variables locales (comme \texttt{i}) sont séparées en mémoire. Mais, la variable \texttt{counter} est modifiée, mais elle ne correspond pas forcément à $2 \times \texttt{max}$. En effet, si \texttt{p1} et \texttt{p2} essaient d'exécuter au même moment de réaliser l'opération \lstinline[language=c]-counter = counter + 1-, ils peuvent récupérer deux valeurs identiques de \texttt{counter}, ajouter 1, puis réassigner \texttt{counter}.
Ils \guillemotleft~se marchent sur les pieds.~\guillemotright\ 

Parmi les opérations, on distingue certaines dénommées \guillemotleft~atomiques~\guillemotright\ qui ne peuvent pas être séparées. L'opération \lstinline[language=c]-i++- n'est pas atomique, mais la lecture et l'écriture mémoire le sont.

\begin{defn}
	On dit d'une variable qu'elle est \textit{atomique} lorsque l'ordonnanceur ne l'interrompt pas.
\end{defn}

\begin{exm}
	L'opération \lstinline[language=c]-counter = counter + 1- exécutée en série peut être représentée comme ci-dessous. Avec \texttt{counter} valant 40, cette exécution donne 42.
	\begin{table}[H]
		\centering
		\begin{tabular}{l|l}
			Exécution du fil A & Exécution du fil B\\ \hline
			(1)~$\mathrm{reg}_1 \gets \texttt{counter}$ & (4)~$\mathrm{reg}_2 \gets \texttt{counter}$ \\
			(2)~$\mathrm{reg}_1{++}$ & (5)~$\mathrm{reg}_2{++}$ \\
			(3)~$\texttt{counter} \gets \mathrm{reg}_1$ & (6)~$\texttt{counter} \gets \mathrm{reg}_2$
		\end{tabular}
	\end{table}
	\noindent Mais, avec l'exécution en simultanée, la valeur de \texttt{counter} sera 41.
	\begin{table}[H]
		\centering
		\begin{tabular}{l|l}
			Exécution du fil A & Exécution du fil B\\ \hline
			(1)~$\mathrm{reg}_1 \gets \texttt{counter}$ & (2)~$\mathrm{reg}_2 \gets \texttt{counter}$ \\
			(3)~$\mathrm{reg}_1{++}$ & (5)~$\mathrm{reg}_2{++}$ \\
			(4)~$\texttt{counter} \gets \mathrm{reg}_1$ & (6)~$\texttt{counter} \gets \mathrm{reg}_2$
		\end{tabular}
	\end{table}
	\noindent Il y a \textit{entrelacement} des deux fils d'exécution.
\end{exm}

\begin{rmk}[Problèmes de la programmation concurrentielle]
	\begin{itemize}
		\item Problème d'accès en mémoire,
		\item Problème du rendez-vous,\footnote{Lorsque deux programmes terminent, ils doivent s'attendre pour donner leurs valeurs.}
		\item Problème du producteur-consommateur,\footnote{Certains programmes doivent ralentir ou accélérer.}
		\item Problème de l'entreblocage,\footnote{\textit{c.f.} exemple ci-après.}
		\item Problème famine, du dîner des philosophes.\footnote{Les philosophes mangent autour d'une table, et mangent du riz avec des baguettes. Ils décident de n'acheter qu'une seule baguette par personne. Un philosophe peut, ou penser, ou manger. Mais, pour manger, ils ont besoin de deux baguettes. S'ils ne mangent pas, ils meurent.}
	\end{itemize}
\end{rmk}

\begin{exm}[Problème de l'entreblocage]~

	\begin{table}[H]
		\centering
		\begin{tabular}{l|l|l}
			Fil A & Fil B & Fil C\\ \hline
			RDV(C) & RDV(A) & RDV(B)\\
			RDV(B) & RDV(C) & RDV(A)\\
		\end{tabular}
		\caption{Problème de l'entreblocage}
	\end{table}
\end{exm}

Comment résoudre le problème des deux incrementations ? Il suffit de \guillemotleft~mettre un verrou.~\guillemotright\ Le premier fil d'exécution \guillemotleft~s'enferme~\guillemotright\ avec l'expression \lstinline[language=c]!count++!, le second fil d'exécution attend que l'autre sorte pour pouvoir entrer et s'enfermer à son tour.


		\section{Continuité}

\begin{exm}
	Dans l'exercice 2, chaque fonction $f_n : t \mapsto t^n$\/ est continue sur $[0,1]$\/ mais la limite $f$\/ n'est pas continue sur $[0,1]$\/ (car elle n'est pas continue en $1$).
\end{exm}

\begin{thm}
	Soit $a$\/ un réel dans un intervalle $T$\/ de $\R$. Si une suite de fonctions $(f_n)_{n\in\N}$\/ continues en $a$\/ converge uniformément sur $T$\/ vers une fonction $f$, alors $f$\/ est aussi continue en $a$.
\end{thm}

\begin{prv}
	On suppose les fonctions $f_n$\/ continues en $a$\/ ($f_n(x) \longrightarrow f_n(a)$) et que la suite de fonctions $(f_n)_{n\in\N}$\/ converge uniformément vers $f$\/ ($\sup\:|f_n -f| \longrightarrow 0$). On veut montrer que $f$\/ est continue en $a$\/ : $f(x) \tendsto{x \to a} f(a)$, i.e.\ \[
		\forall \varepsilon > 0,\:\exists \delta > 0,\: \forall x \in T,\quad|x-a| \le \delta \implies |f(x) - f(a)| \le \varepsilon
	.\]
	Soit $\varepsilon > 0$. On calcule \[
		\big|f(x) - f(a)\big| \le \big|f(x) - f_n(x)\big| + \big|f_n(x) - f_n(a)\big| + \big|f_n(a) - f(a)\big|
	\] par inégalité triangulaire. Or, par hypothèse, il existe un rang $N \in \N$\/ (qui ne dépend pas de $x$\/ ou de $a$), tel que, $\forall n \ge N$, $\big|f(x) - f_n(x)\big| \le \frac{1}{3} \varepsilon$, et $\big|f_n(a) - f(a)\big| \le \frac{1}{3} \varepsilon$.
	De plus, par hypothèse, il existe $\delta >0$\/ tel que si $|x - a| \le \delta$, alors $|f_n(x) - f_n(a)| \le \frac{1}{3}\varepsilon$.\footnote{C'est là où l'hypothèse de la convergence uniforme est utilisée : on a besoin que le $N$\/ ne dépende pas de $x$\/ car on le fait varier.}
	On en déduit que $\big|f(x) - f(a)\big| \le \varepsilon$.
\end{prv}

\begin{crlr}
	Soit $T$\/ un intervalle de $\R$. Si une suite de fonctions $(f_n)_{n\in\N}$\/ continues sur $T$\/ converge uniformément sur $T$\/ vers une fonction continue sur $T$.
\end{crlr}

\begin{met}[Stratégie de la barrière]
	\begin{enumerate}
		\item La continuité (la dérivabilité aussi) est une propriété {\it locale}. Pour montrer qu'une fonction est continue sur un intervalle $T$, il suffit donc de montrer qu'elle est continue sur tout segment inclus dans $T$.
		\item Mais, la convergence uniforme est une propriété {\it globale}. La convergence sur tout segment inclus dans un intervalle n'implique pas la convergence uniforme sur l'intervalle (voir l'exercice 2).
		\item On n'écrit pas \[
				\substack{\ds\text{convergence uniforme}\\\ds\text{avec barrière}} \mathop{\red\implies} \substack{\ds\text{convergence uniforme}\\\ds\text{sans barrière}} \implies \substack{\ds\text{continuité}\\\ds\text{sans barrière}}
			\] mais plutôt \[
				\substack{\ds\text{convergence uniforme}\\\ds\text{avec barrière}} \implies \substack{\ds\text{continuité}\\\ds\text{avec barrière}} \implies \substack{\ds\text{continuité}\\\ds\text{sans barrière}}
			.\]
		\item Si, pour tous $a$\/ et $b$, $f$\/ est bornée sur $[a,b] \subset T$, mais cela n'implique pas que $f$\/ est bornée. Contre-exemple : la fonction $f : x \mapsto \frac{1}{x}$\/ est bornée sur tout intervalle $[a,b]$\/ avec $a$, $b \in \R^+_*$, \red{\sc mais} $f$\/ n'est pas bornée sur $]0,+\infty[$.
	\end{enumerate}
\end{met}

\begin{thm}[double-limite ou d'interversion des limites]
	Soit une suite de fonctions $(f_n)_{n\in\N}$\/ définies sur un intervalle $T$, et, soit $a$\/ une extrémité (éventuellement infinie)\footnote{autrement dit, $a \in \bar\R = \R \cup \{+\infty,-\infty\}$} de cet intervalle. Si la suite de fonctions $(f_n)_{n\in\N}$\/ converge \underline{uniformément} sur $T$\/ vers $f$\/ et si chaque fonction $f_n$\/ admet une limite finie $b_n$\/ en $a$, alors la suite de réels $b_n$\/ converge vers un réel $b$, et $\lim_{t\to a} f(t) = b$. Autrement dit, \[
		\lim_{t\to a} \Big(\underbrace{\lim_{n\to +\infty} f_n(t)}_{f(x)}\Big) = \lim_{n\to +\infty} \Big(\underbrace{\lim_{t\to a} f_n(t)}_{b_n}\Big)
	.\] \qed
\end{thm}

\begin{rmkn}
	Le théorème de la double-limite \guillemotleft~contient~\guillemotright\ le théorème 6 (théorème de préservation/transmission de la continuité), c'est un cas particulier. En effet, si les fonctions $f_n$\/ sont continues, alors \[
		\lim_{x \to a}f(x) = \underbrace{\lim_{n\to +\infty} f_n(a)}_{f(a)}
	.\]
\end{rmkn}


		\section{Endomorphismes adjoints}

\begin{defn}
	On dit qu'un endomorphisme $f : E \to E$\/ est \textit{autoadjoint} si \[
		\forall (\vec{u}, \vec{v}) \in E^2,\quad \big<f(\vec{u})\:\big|\:\vec{v}\big> = \big<\vec{u}\:\big|\:f(\vec{v})\big>
	.\] 
\end{defn}

Un endomorphisme autoadjoint est aussi appelé endomorphisme \textit{symétrique} (\textit{c.f.}\ proposition suivante). L'ensemble des endomorphismes autoadjoints est noté $\mathcal{S}(E)$.

\begin{prop}
	Un endomorphism est autoadjoint si, et seulement si la matrice de $F$\/ dans une base \ul{orthonormée} $\mathcal{B}$\/ est orthogonale.
	Autrement dit : \[
		f \in \mathcal{S}(E) \iff \big[\:f\:\big]_\mathcal{B} \in \mathcal{S}_n(\R)
	.\]
\end{prop}

\begin{prv}
	\begin{description}
		\item[$\implies$] Soit $\mathcal{B} = (\vec{\varepsilon}_1, \ldots, \vec{\varepsilon}_n)$\/ une base orthonormée de $E$. Ainsi, \[
				\forall i,\:\forall j,\quad \big<f(\vec{\varepsilon}_i)\:\big|\: \vec{\varepsilon}_j\big> = \big<\vec{\varepsilon}_i\:\big|\:f(\vec{\varepsilon}_j)\big>
			.\] On pose $\big[\:f\:\big]_{\mathcal{B}} = (a_{i,j})$\/ : \[
				\begin{pNiceMatrix}[last-col,last-row]
					\quad&\quad&a_{1,j}&\quad&\quad&\vec{\varepsilon}_i\\
					&&&&&\\
					\quad&\quad&a_{i,j}&\quad&\quad&\vec{\varepsilon}_i\\
					&&&&&\\
					\quad&\quad&a_{n,j}&\quad&\quad&\vec{\varepsilon}_n\\
					f(\vec{\varepsilon}_i)&&f(\vec{\varepsilon}_j)&&f(\vec{\varepsilon}_n)\\
				\end{pNiceMatrix}
			.\] Ainsi, $f(\vec{\varepsilon}_j) = a_{1,j} \vec{\varepsilon}_1 + \cdots + a_{i,j} \vec{\varepsilon}_i + \cdots + a_{n,j} \vec{\varepsilon}_n$. D'où, $\left<\vec{\varepsilon}_i  \mid f(\vec{\varepsilon}_j) \right> = a_{i,j}$\/ car la base $\mathcal{B}$\/ est orthonormée.
			De même avec l'autre produit scalaire, $\left< f(\vec{\varepsilon}_i)  \mid \vec{\varepsilon}_j \right>$, d'où $a_{i,j} = a_{j,i}$\/ par symétrie du produit scalaire. On en déduit que $\big[\:f\:\big]_\mathcal{B} \in \mathcal{S}_n(\R)$.
		\item[$\impliedby$]
			Si $\big[\: f\:\big]_\mathcal{B} \in \mathcal{S}_n(\R)$, alors $\left<f(\vec{\varepsilon}_i)  \mid \vec{\varepsilon}_j \right> = \left<\vec{\varepsilon}_i  \mid f(\vec{\varepsilon}_j)\right>$.
			Or, on pose $\vec{u} = x_1 \vec{\varepsilon}_1+ \cdots + x_n \vec{\varepsilon}_n$, et $\vec{v} = y_1 \vec{\varepsilon}_1 + \cdots + y_n \vec{\varepsilon}_n$.
			\begin{align*}
				\left<f(\vec{u})  \mid \vec{v} \right> &= \left<x_1 f(\vec{\varepsilon}_1) + \cdots + x_n f(\vec{\varepsilon}_n)  \mid y_1 f(\vec{\varepsilon}_1) + \cdots + y_n f(\vec{\varepsilon}_n) \right> \\
				&= \Big<\sum_{i=1}^n x_i f(\vec{\varepsilon}_i)\:\Big|\: \sum_{j=1}^n y_j \vec{\varepsilon}_j\Big> \\
				&= \sum_{i,j \in \llbracket 1,n \rrbracket}  x_i y_j \left<f(\vec{\varepsilon}_i) \mid \vec{\varepsilon}_j \right>\\
			\end{align*}
			De même en inversant $\vec{u}$\/ et $\vec{v}$.
			On en déduit donc $\left<f(\vec{u} \mid \vec{v} \right> = \left<\vec{u}  \mid f(\vec{v}) \right>$.
	\end{description}
\end{prv}

\begin{exo}
	\begin{enumerate}
		\item Si $f$\/ est autoadjoint, montrons que $\Ker f \perp \Im f$, et $\Ker f \oplus \Im f$.
			On suppose $\forall \vec{u}$, $\forall \vec{v}$, $\left<f(\vec{u}) \mid \vec{v} \right> = \left<\vec{u}  \mid f(\vec{v}) \right>$.
			Soit $\vec{u} \in \Ker f$, et soit $\vec{v} \in \Im f$.
			On sait que $f(\vec{u}) = \vec{0}$, et qu'il existe $\vec{x} \in E$\/ tel que $\vec{v} = f(\vec{x})$.
			Ainsi, \[
				\left<\vec{u}  \mid \vec{v} \right>
				= \left<\vec{u}  \mid f(\vec{x}) \right>
				= \left<f(\vec{u})  \mid \vec{x} \right>
				= 0
			.\] 
			D'où $\vec{u} \perp \vec{v}$. Ainsi, $\Ker f \perp \Im f$.


			De plus, $E$\/ est de dimension finie, d'où, d'après le théorème du rang, \[
				\dim \Ker f + \dim \Im f = \dim E
			.\] Aussi, $\Ker f \oplus (\Ker f)^\perp = E$, donc $\dim(\Ker f) + \dim(\Ker f)^\perp = \dim E$.
			On en déduit donc que $\dim(\Im f)= \dim(\Ker f)^\perp$.
			Or, $\Im f \subset (\Ker f)^\perp$\/ car $\Im f \perp \Ker f$.
			Ainsi $\Im f = (\Ker f)^\perp$, on en déduit que \[
				\Im f \oplus \Ker f = E
			.\]
			\begin{description}
				\item[$\impliedby$] 
					Soit $p$\/ la projection sur $F$\/ parallèlement à $G$.
					Supposons l'endomorphisme $P$\/ autoadjoint.
					D'après la question 1., le $\Ker p \perp \Im p$.
					Ainsi, $F = \Im p$\/ et $G = \Ker p$.
					D'où, $F \perp G$, $p$\/ est donc une projection orthogonale.
				\item[$\implies$]
					Réciproquement, supposons $p$\/ une projection orthogonale.
					Soit $\mathcal{B} = (\vec{\varepsilon}_1, \ldots, \vec{\varepsilon}_q)$\/ une base orthonormée de $F$.
					Ainsi, pour tout $\vec{x} \in E$, \[
						p(\vec{x}) = \sum_{i = 1}^q \left<\vec{x}  \mid \vec{\varepsilon}_i \right>\,\vec{\varepsilon}_i
					.\] 
					On veut montrer que l'endomorphisme $p$\/ est autoadjoint.
					Soient $\vec{u}$\/ et $\vec{v}$\/ deux vecteurs de $E$.
					\begin{align*}
						\left<p(\vec{u})  \mid \vec{v} \right>
						= \Big<\sum_{i=1}^q \left< \vec{u}\mid \vec{\varepsilon}_i \right>\vec{\varepsilon}_i\:\Big|\; \vec{v}\;\Big>
						&= \sum_{i=1}^q \left<u  \mid \vec{\varepsilon}_{i} \right>\: \left< \vec{\varepsilon}_i  \mid v\right>\\
						&= \sum_{i=1}^q \left<v  \mid \vec{\varepsilon}_i \right>\:\left<\vec{\varepsilon}_i   \mid u\right> \\
						&= \left< \vec{u}  \mid p(\vec{v}) \right> \\
					\end{align*}

					Autre méthode, pour tous vecteurs $\vec{u}$\/ et $\vec{v}$\/ de $E$,
					\begin{align*}
						\left<p(\vec{u})  \mid \vec{v} \right>
						&= \left<p(\vec{u})  \mid p(\vec{v}) + \vec{v} - p(\vec{v}) \right> \\
						&= \left< p(\vec{u})  \mid p(\vec{v}) \right> + \left<p(\vec{u})  \mid  \vec{v} - p(\vec{v}) \right> \\
						&= \left<p(\vec{u}) \mid p(\vec{v}) \right> + \left<u - p(\vec{u})  \mid p(\vec{v}) \right> \\
						&= \left<\vec{u}  \mid p(\vec{v}) \right> \\
					\end{align*}
					car $p$\/ est orthogonale.
			\end{description}
	\end{enumerate}
\end{exo}


\begin{prop-defn}
	Si $f$\/ est un endomorphisme d'un espace euclidien $E$, alors il existe un unique endomorphisme de $E$, noté $f^\star$\/ et appelé l'\textit{adjoint} de $f$, tel que \[
		\forall (\vec{u},\vec{v}) \in E^2,\quad\quad \left<f^\star(\vec{u})  \mid \vec{v} \right> =  \left<\vec{u}  \mid f(\vec{v}) \right>
	.\] 
	Si $A$\/ est la matrice $f$\/ dans une base orthonormée $\mathcal{B}$\/ de $E$, alors $A^\top$\/ est la matrice de $f^\star$\/ dans~$\mathcal{B}$\/ : \[
		\big[\:f\:\big]_\mathcal{B} = \big[\:f\:\big]_\mathcal{B}^\top
	.\]
\end{prop-defn}

\begin{prv}
	Soit $\vec{u} \in E$. L'application \begin{align*}
		\varphi: E &\longrightarrow \R \\
		\vec{v} &\longmapsto \left<\vec{u}  \mid f(\vec{v}) \right>.
	\end{align*}
	La forme $\varphi$\/ est linéaire car $\varphi(\alpha_1 \vec{v}_1 + \alpha_2 \vec{v}_2) = \left<\vec{u} \mid f(\alpha_1 \vec{v}_1 + \alpha_2 \vec{v}_2) \right> = \left<\vec{u}  \mid \alpha_1 f(\vec{v}_1) + \alpha_2 f(\vec{v}_2) \right> = \alpha_1\left<\vec{u}  \mid  f(\vec{v}) \right> + \alpha_2 \left<\vec{u}  \mid f(\vec{v}_2) \right> = \alpha_1 \varphi(\vec{v}_1) + \alpha_2 \varphi(\vec{v}_2)$.
	D'où, d'après le théorème de \textsc{Riesz}, il existe un \ul{unique} vecteur $\vec{a} \in E$\/ tel que $\varphi(\vec{v}) = \left<\vec{a}  \mid \vec{v} \right>$\/ pour tout $\vec{v} \in E$.
	Ainsi, pour tout vecteur $\vec{v} \in E$, $\left<\vec{u}  \mid f(\vec{v}) \right> = \left<\vec{a}  \mid \vec{v} \right>$.
	On note $\vec{a} = f^\star(\vec{u})$.
	Soit l'application \begin{align*}
		f^\star : E &\longrightarrow E \\
		\vec{u} &\longmapsto f^\star(\vec{u}).
	\end{align*}
	La démonstration telle que $f^\star $\/ est linéaire est dans le poly.
	L'application $f^\star$\/ vérifie : $\left< \vec{u} \mid f(\vec{v}) \right> = \left<f^\star (\vec{u})  \mid \vec{v} \right>$, pour tous vecteurs $\vec{u}$\/ et $\vec{v}$.
	\textsl{Quelle est la matrice de $f^\star$, dans une base orthonormée ?}\@
	Soit $\mathcal{B}$\/ une base orthonormée de $E$, et soient $A = \big[\:f\:\big]_\mathcal{B}$, $B = \big[\:f^\star \:\big]_\mathcal{B}$, $U = \big[\:\vec{u}\:\big]_\mathcal{B}$, et $V = \big[\:\vec{v}\:\big]_\mathcal{B}$.
	Les matrices $U$\/ et $V$\/ sont des vecteurs colonnes, et $A$\/ et $B$\/ sont des matrices carrées.
	Ainsi, \[
		U^\top \cdot A \cdot V = \left< \vec{u}  \mid f(\vec{v}) \right> 
		= \left<f^\star (\vec{u})  \mid \vec{v} \right> = (B\cdot U)^\top \cdot V,
	\] ce qui est vrai quelque soit les vecteurs colonnes $U$\/ et $V$.
	D'où, $\forall U$, $\forall V$, $U^\top \cdot \big(A \cdot V\big) = U^\top \cdot \big(B^\top \cdot V\big)$.
	Ainsi, pour tous vecteurs $U$\/ et $V$, \[
		U^\top \cdot \Big[ (AV) - (B^\top V)\Big] = 0
	.\] En particulier, si $U = (AV) - (B^\top V)$, le produit scalaire $\left<\vec{u}  \mid \vec{u} \right>$\/ est nul, donc $U = 0$.
	Ainsi, \[
		\forall V,\quad A\cdot V = B^\top\cdot  V
	.\] De même, on conclut que $A = B^\top$. On en déduit donc que \[
	\big[\:f^\star\:\big]_\mathcal{B} = \big[\:f\:\big]_\mathcal{B}^\top
	.\]
\end{prv}

Les propriétés suivantes sont vrais :
\begin{itemize}
	\item $(f  \circ g)^\star  = g^\star \circ f^\star$, \quad $(f^\star)^\star = f$, \quad et \quad $(\alpha f + \beta g)^\star  = \alpha f^\star + \beta g^\star $\/ ;
	\item $(A\cdot B)^\top  = B^\top \cdot A^\top$, \quad $(A^\top)^\top = A$, \quad et \quad $(\alpha A + \beta B)^\top = \alpha A^\top+ \beta B^\top $.
\end{itemize}
Des deuxièmes et troisièmes points,  il en résulte que les applications $f \mapsto f^\star$, et $A \mapsto A^\top$\/ sont des applications involutives.



		\begin{prop-defn}
  Soit $(\Omega, \mathcal{A}, P)$\/ un espace probabilisé, et soit $X$\/ une \textit{vard}. Si $X^2$\/ est d'espérance finie, alors $X$\/ aussi, et on appelle \textit{variance} le réel positif \[
    \mathrm{V}(X) = \mathrm{E}\Big(\big[X - \mathrm{E}(X)\big]^2\Big) = \underbrace{\mathrm{E}(X^2) - \big(\mathrm{E}(X)\big)^2}_{\mathclap{\text{Relation de \textsc{König \& Huygens}}}} \ge 0
  .\]
  L'\textit{écart-type} $\sigma(X)$\/ est la racine carrée de la variance : \[
    \sigma(X) = \sqrt{\mathrm{V}(X)}
  .\]
\end{prop-defn}

\begin{prv}
  On pose $\mu = \mathrm{E}(X)$, et on a $[X - \mu]^2 = X^2 - 2 \mu X + \mu^2$. D'où, par linéarité de l'espérance,
  \begin{align*}
    \mathrm{E}\big((X-\mu)^2\big)
    &= \mathrm{E}(X^2 - 2\mu X + \mu^2) \\
    &= \mathrm{E}(X^2) - 2\mu \mathrm{E}(X) + \mu^2 \\
    &= \mathrm{E}(X^2) - 2\mu^2 + \mu^2 \\
    &= \mathrm{E}(X^2) - \big(\mathrm{E}(X)\big)^2. \\
  \end{align*}
  De plus, d'après le lemme précédent, si $X^2$\/ est d'espérance finie, alors $X$\/ est d'espérance finie.
\end{prv}

\begin{rmk}
  \begin{enumerate}
    \item La variance mesure la \textit{dispersion}, ou l'\textit{étalement} des valeurs $a_i$\/ autour de l'espérance $\mathrm{E}(X)$. En particulier, s'il existe $a \in \R$\/ tel que $P(X = a) = 1$, alors $\mathrm{E}(X) = a$\/ et $\mathrm{V}(X) = 0$. (C'est même une équivalence.)
    \item Si la variable $X$\/ a une unité ($\mathrm{km}/\mathrm{s}$, $\mathrm{V}/\mathrm{m}$, etc.), alors l'écart type a la même unité (d'où l'intérêt de calculer la racine carrée de la variance).
    \item Soient $\alpha$\/ et $\beta$\/ deux réels. Si  $X^2$\/ est d'espérance finie, alors \[
      \mathrm{V}(\alpha X + \beta) = \alpha^2\cdot  \mathrm{V}(X)
    .\]
    (Une translation ne change pas la dispersion des valeurs, et multiplier par un réel multiplie l'espérance, mais aussi la dispersion, d'où le carré.)
  \end{enumerate}
\end{rmk}

\begin{exo}
  \textsl{Montrer que
  \begin{enumerate}
    \item si $X \sim \mathcal{B}(n, p)$, alors $X^2$\/ est d'espérance finie et $\mathrm{V}(X) = n\cdot p\cdot q$.
    \item si $T \sim \mathcal{G}(p)$, alors $T^2$\/ est d'espérance finie et $\mathrm{V}(T) = \frac{q}{p^2}$.
    \item si $X \sim \mathcal{P}(\lambda)$, alors $X^2$\/ est d'espérance finie et $\mathrm{V}(X) = \lambda$.
  \end{enumerate}
  }

  \begin{enumerate}
    \item
      Si $X \sim \mathcal{B}(n,p)$, alors $X(\Omega) = \llbracket 0,n \rrbracket$\/ et, pour $k \in X(\Omega)$, $P(X = k) = {n\choose k}\,p^k\,q^{n-k}$.
      On a déjà montré que $\mathrm{E}(X) = n\cdot p$.
      On va montrer que $\mathrm{V}(X) = n\,p\,q$.
      La variable aléatoire $X^2$\/ est d'espérance finie car $X(\Omega)$\/ est fini.
      Et,
      \begin{align*}
        \mathrm{E}(X^2) &= \sum_{k=0}^n k^2\: P(X = k)\\
        &= \sum_{k=0}^n k^2 {n\choose p} p^k q^{n-k} \\
        &= \ldots \\
      \end{align*}
      En effet, d'après la ``petite formule,'' on a \[
        \forall k \ge 1,\quad k{n\choose k} = n{n-1 \choose k-1}
      \] d'où, $(k-1) {n-1\choose k-1} = (n-1) \choose {n-2 \choose k-2}$. Ainsi, \[
        \forall k \ge 2,\quad k(k-1){n\choose k} = n(n+1) {n-2\choose k-2}
      .\] 
    \item Si $T \sim \mathcal{G}(p)$, alors $T(\Omega) = \N^*$\/ et $\forall k \in T(\Omega)$, $P(T = k) = p \times q^{k-1}$.
      On a déjà prouvé que $\mathrm{E}(T) = \frac{1}{p}$.
      On veut montrer que $\mathrm{V}(T) = \frac{q}{p^2}$. Montrons que la variable $T^2$\/ possède une espérance : la série $\sum k^2\: P(T = k)$\/ converge absolument car $k^2 \:P(T = k) = k^2 \cdot p \cdot q^{k-1}$.
      Or, pour $k \ge 2$, $\frac{\mathrm{d}^2}{\mathrm{d}x^2} x^k = k(k-1)\,x^{k-2}$. Et, on peut dériver terme à terme une série entière sans changer son rayon de convergence, et la série $\sum x^k$\/ a pour rayon de convergence 1. D'où, $\sum k(k-1)\, x^{k-2}$\/ a pour rayon de convergence 1. Or, $q \in {]0,1[} \subset {]-1,1[}$\/ donc la série $\sum k(k-1)q^{k-2}$\/ converge. De plus, $\sum k (k-1)\, q^{k-2} = \sum k^2\,q^{k-2} - \sum k\,q^{k-2}$.
      D'où,  $\sum k^2 q^{k-2} = \sum k(k-1)\, q^{k-2} + \sum k\,q^{k-2}$, qui converge. Par suite,
      \begin{align*}
        \sum_{k=1}^\infty k^2\, P(T = k) &= \sum_{k=1}^\infty k^2\,p\,q^{k-1}\\
        &= p + pq \sum_{k=2}^\infty k^2 q^{k-2} \\
        &= p + pq \sum_{k=2}^\infty k(k-1)\,q^{k-2} + p\sum_{k=2}^\infty k\,q^{k-1} \\
        &= p + pq\, \frac{2}{(1-q)^3} + p\left(\frac{1}{(1-q)^2} - 1 \right) \rlap{\quad\quad \text{\textit{c.f.} en effet après}}\\
        &= p + pq\, \frac{2}{p^3} + p\left( \frac{1}{p^2} - 1 \right) \\
        &= \frac{2q}{p^2} + \frac{1}{p} \\
        &= \frac{2q + p}{p^2} \\
        &= \frac{2q + (1-q)}{p^2} \\
        &= \frac{q+1}{p^2}. \\
      \end{align*}
      En effet, $\forall x \in {]-1,1[}$, $\sum_{k=0}^\infty x^k = \frac{1}{1-x}$. D'où, pour $x \in {]-1,1[}$, \[
        \sum_{k=1}^\infty k\,x^{k-1} = \frac{1}{(1-x)^2}
        \quad \text{ et }\quad
        \sum_{k=2}^\infty k(k-1)\,x^{k-2} = \frac{2}{(1-x)^3}
      .\]
      Ainsi, $\mathrm{E}(T^2) = \frac{q^{+1}}{p^2}$. D'où
      \begin{align*}
        \mathrm{V}(T) &= \mathrm{E}(T^2) - \big(\mathrm{E}(T)\big)^2 \\
        &= \frac{q+1}{p^2} - \left( \frac{1}{p} \right)^2 \\
        &= \frac{q}{p^2} \\
      \end{align*}
    \item À tenter
  \end{enumerate}
\end{exo}


\section{Les inégalités de \textsc{Markov} et de \textsc{Bienaymé}--\textsc{Tchebychev}, inégalités de concentration}

\begin{lem}[Markov]
  Soit $(\Omega, \mathcal{A}, P)$\/ un espace probabilisé, et soit $X$\/ une variable aléatoire \underline{positive}.
  Si $X$\/ est d'espérance finie, alors \[
    \forall a > 0, \quad P(X \ge a) \le \frac{\mathrm{E}(X)}{a}
  .\]
\end{lem}

\begin{prv}
  On suppose $X$\/ d'espérance finie. Ainsi, on a \[
    \mathrm{E}(X) = \sum_{x \in X(\Omega)} x\:P(X = x)
  .\]
  Soit $I$\/ l'ensemble $I = \{x \in X(\Omega) \mid x \ge a\}$.
  Alors, \[
    \mathrm{E}(X) = \underbrace{\sum_{x \in I} x\:P(X = x)}_{\text{ ici } x \ge a} + \underbrace{\sum_{x \in X(\Omega) \setminus I} x\:P(X = x)}_{\ge 0 \text{ par hypothèse}}
  .\]
  D'où, \[
    \mathrm{E}(X) \ge \sum_{x \in I} x\:P(X = x) \ge \sum_{x \in I} a\: P(X = x) = a \sum_{x \in I} P(X = x) \ge a\:P(x \ge a)
  .\]
\end{prv}

\begin{prop}[\textsc{Bienaymé--Tchebychev}]
  Soit $(\Omega, \mathcal{A}, P)$\/ un espace probabilisé, et soit $X$\/ une \textit{vard}. Si $X^2$\/ est d'espérance finie, alors \[
    \forall a > 0, \quad\quad P\Big(\big|X - \mathrm{E}(X)\big| \ge a\Big) \le \frac{\mathrm{V}(X)}{a^2}
  .\]
\end{prop}

\begin{prv}
  On pose $\mu = \mathrm{E}(X)$.
  L'événement $\big(|X - \mu| \ge a\big) = \big((X - \mu)^2 \ge a^2\big)$, d'où, les probabilités \[
    P\big(|X - \mu| \ge a\big) = P\big(\underbrace{(X - \mu)^2}_{\ge 0} \ge \underbrace{a^2}°{\ge 0}\big).
  \] On valide donc \textit{une} des hypothèses de l'inégalité de \textsc{Markov}.
  De plus, l'autre hypothèse est vérifiée : $X^2$\/ est d'espérance finie, donc $(X - \mu)^2$\/ aussi. On en déduit, d'après le lemme de \textsc{Markov}, que \[
    P\big((X-\mu)^2 \ge a^2\big) \le \frac{\mathrm{E}\big((X-\mu)^2\big)}{a^2} = \frac{\mathrm{V}(X)}{a^2}
  .\]
\end{prv}

\section{Série génératrice}

\begin{defn}
  Soit $X$\/ une \textit{vad} telle que $X(\Omega) \subset \N$. La \textit{série génératrice} de $X$\/ est la série entière $\sum a_n x^n$ de coefficients $a_n = P(X = n)$.
\end{defn}


La série $\sum a_n$\/ converge car sa somme vaut $\sum_{n=0}^\infty a_n = 1$. D'où, 
\begin{itemize}
  \item le rayon de convergence $R$\/ de la série est supérieur ou égal à 1.
  \item la série génératrice converge normalement sur $[-1,1]$, car la série $\sum |a_n|$\/ converge, or, $\forall x \in [-1,1]$, $|p_nt^n| \le |p_n|$, d'où la convergence normale.
    D'où la \textit{fonction génératrice} \[
      \mathrm{G}_X \colon t \longmapsto \sum_{n=0}^\infty p_n t^n
    \] est définie et même continue sur $[-1,1]$, car la convergence est uniforme.
  \item la fonction génératrice $\mathrm{G}_X$\/ est de classe $\mathcal{C}^\infty$\/ sur $]-1,1[$\/ et \[
      \forall k \in \N,\quad P(X = k) = a_n \frac{{\mathrm{G}_X}^{(k)}(0)}{k!}
    .\] La fonction génératrice de $X$\/ permet donc de retrouver la loi de probabilité de $X$.
\end{itemize}


	}
	\def\addmacros#1{#1}
}



\part{Travaux Dirigés}
\def\prefix{\textsc{td}}
\renewcommand{\chaptername}{Travaux Dirigés}


{
	\td[1]{Ordre \& Induction}
	\minitoc
	\renewcommand{\cwd}{../td/td01/}
	\addmacros{
		\section{Filtre RC double}

\begin{enumerate}
	\item En basse fréquence, un condensateur est équivalent à un interrupteur ouvert. En haute fréquence, un condensateur est équivalent à un interrupteur fermé. D'où, le circuit est un filtre passe-bas.
	\item Par une loi des nœuds, et une loi des mailles, on trouve que
		\[
			\ubar{H}(\mathrm{j}\omega) = \dfrac{1}{1 - \left( \dfrac{\omega}{\omega_0} \right)^2 + \mathrm{j} \dfrac{\omega}{Q\cdot \omega_0}}
		\] en notant $\omega_0 = 1 / {RC}$ et $Q = 1 / 3$
	\item On représente le diagramme de \textsc{Bode} du filtre dans la figure ci-dessous.
		\begin{figure}[H]
			\centering
			\includesvg[width=\linewidth]{figures/bode-1.svg}
			\caption{Diagramme de \textsc{Bode} du filtre (échelle logarithmique)}
		\end{figure}
	\item On calcule $\omega_0 \simeq 6\:\mathrm{rad/s}$, ce qui correspond à une fréquence de coupure de $1\:\mathrm{kHz}$. Le signal de sortie est donc \[
			s(t) = \frac{2E}{3\pi}\cdot \sin(\omega t)
		,\] on le représente sur la figure ci-dessous. En effet, on a un déphasage de $-\pi / 2$, et un gain valant $1 / 3$ à $\omega \simeq \omega_0$.
		\begin{figure}[H]
			\centering
			\includesvg[width=\linewidth]{figures/signal-1.svg}
			\caption{Signal résultant}
		\end{figure}
\end{enumerate}

		\section{Tri topologique}

\begin{figure}[H]
	\centering
	\tikzfig{ex2-q1}
	\caption{Exemple de graphe}
\end{figure}

\begin{enumerate}
	\item Dans le graphe ci-dessus, $a \to c \to b$\/ est un tri topologique mais pas un parcours.
	\item Dans le même graphe, $b \to a \to c$\/ est un parcours mais pas un tri topologique.
	\item Supposons que $L_1$\/ possède un prédécesseur, on le note $L_i$\/ où $i > 1$. Ainsi, $(L_i, L_1) \in A$\/ et donc $i < 1$, ce qui est absurde. De même pour le dernier.
	\item Il existe un tri topologique si, et seulement si le graphe est acyclique.
		\begin{itemize}
			\item[``$\implies$'']
				Soit $L_1,\ldots,L_n$\/ un tri topologique. Montrons que le graphe est acyclique.
				Par l'absurde, on suppose le graphe non acyclique : il existe $(i,j) \in \llbracket 1,n \rrbracket^2$\/ avec $i \neq j$\/ tels que $T_i \to \cdots \to T_j$\/ et $T_j \to \cdots \to T_i$ soient deux chemins valides. Ainsi, comme le tri est topologique et par récurrence, $i \le j$\/ et $j \le i$\/ et donc $i = j$, ce qui est absurde car $i$\/ et $j$\/ sont supposés différents. Le graphe est donc acyclique.
			\item[``$\impliedby$'']
				Soit $G$\/ un graphe tel que tous les sommets possèdent une arrête entrante. On suppose par l'absurde ce graphe acyclique.
				Soit $x_0$\/ un sommet du graphe.
				On construit par récurrence $x_0,x_1,\ldots,x_n,x_{n+1},\ldots$ les successeurs successifs. Il y a un nombre fini de sommets donc deux sommets sont identiques. Donc, il y a nécessairement un cycle, ce qui est absurde.
				\begin{algorithm}[H]
					\centering
					\begin{algorithmic}[1]
						\Entree $G = (S, A)$\/ un graphe acyclique
						\Sortie $\mathrm{Res}$\/ un tri topologique.
						\State $\mathrm{Res} \gets [\quad]$\/
						\While{$G \neq \O$}
							\State Soit $x$\/ un sommet de $G$\/ sans prédécesseur
							\State $G \gets \big(S \setminus \{x\}, A \cap (S \setminus \{x\})^2\big)$\/ 
							\State $\mathrm{Res} \gets \mathrm{Res} \cdot [x]$\/
						\EndWhile
						\State\Return $\mathrm{Res}$\/
					\end{algorithmic}
					\caption{Génération d'un tri topologique d'un graphe acyclique}
				\end{algorithm}
		\end{itemize}
	\item~
		\begin{algorithm}[H]
			\centering
			\begin{algorithmic}[1]
				\Entree $G = (S, A)$\/ un graphe
				\Sortie $\mathrm{Res}$\/ un tri topologique, ou un cycle
				\State $\mathrm{Res} \gets [\quad]$\/
				\While{$G \neq \O$}
					\If{il existe $x$ sans prédécesseurs}
						\State Soit $x$\/ un sommet de $G$\/ sans prédécesseur
						\State $G \gets \big(S \setminus \{x\}, A \cap (S \setminus \{x\})^2\big)$\/ 
						\State $\mathrm{Res} \gets \mathrm{Res} \cdot [x]$\/
					\Else
						\State Soit $x \in S$\/ 
						\State Soit $x \gets x_1 \gets x_2 \gets \cdots \gets x_i$\/ la suite des prédécesseurs
						\State\Return $x_i,x_{i+1},\ldots,x_i$, un cycle
					\EndIf
				\EndWhile
				\State\Return $\mathrm{Res}$\/
			\end{algorithmic}
			\caption{Génération d'un tri topologique d'un graphe}
		\end{algorithm}
	\item On utilise la représentation par liste d'adjacence, et on stocke le nombre de prédécesseurs que l'on décroit à chaque choix de sommet.
	\item On essaie de trouver un tri topologique, et on voit si l'on trouve un cycle.
\end{enumerate}

		\section{Formules duales}

\begin{enumerate}
	\item On définit par induction $(\cdot)^\star$\/ comme
		\begin{multicols}{3}
			\begin{itemize}
				\item $\top^\star = \bot$\/ ;
				\item $\bot^\star = \top$\/ ;
				\item $(G \lor H)^\star = G^\star \land H^\star$\/ ;
				\item $(G \land H)^\star = G^\star \lor H^\star$\/ ;
				\item $(\lnot G)^\star = \lnot G^\star$\/ ;
				\item $p^\star = p$.
			\end{itemize}
		\end{multicols}
	\item Soit $\rho \in \mathds{B}^{\mathcal{P}}$. Montrons, par induction, $P(H) : ``\left\llbracket H^\star \right\rrbracket^\rho = \left\llbracket \lnot H \right\rrbracket^{\bar \rho}"$\/ où $\bar{\rho} : p \mapsto \overline{\rho(p)}$.
		\begin{itemize}
			\item On a $\left\llbracket \bot^\star \right\rrbracket^\rho = \left\llbracket \top \right\rrbracket^\rho = \mathbf{V}$, et $\left\llbracket \lnot \bot \right\rrbracket^{\bar\rho} = \left\llbracket \top \right\rrbracket^{\bar\rho} = \mathbf{V}$, d'où $P(\bot)$.
			\item On a $\left\llbracket \top^\star \right\rrbracket^\rho = \left\llbracket \bot \right\rrbracket^\rho = \mathbf{F}$, et $\left\llbracket \lnot \top \right\rrbracket^{\bar\rho} = \left\llbracket \bot \right\rrbracket^{\bar\rho} = \mathbf{F}$, d'où $P(\top)$.
			\item Soit $p \in \mathcal{P}$. On a $\left\llbracket p^\star  \right\rrbracket^\rho = \left\llbracket p \right\rrbracket^\rho = \rho(p)$, et $\left\llbracket \lnot p \right\rrbracket^{\bar\rho} = \overline{\left\llbracket p \right\rrbracket^{\bar\rho}} = \overline{\bar\rho(p)} = \overline{\overline{\rho(p)}} = \rho(p)$, d'où~$P(p)$.
		\end{itemize}
		Soient $F$\/ et $G$\/ deux formules.
		\begin{itemize}
			\item On a
				\begin{align*}
					\left\llbracket (F \land G)^\star  \right\rrbracket^\rho &= \left\llbracket F^\star  \lor G^\star \right\rrbracket^\rho\\
					&= \left\llbracket F^\star \right\rrbracket^\rho + \left\llbracket G^\star \right\rrbracket^\rho \\
					&= \left\llbracket \lnot F \right\rrbracket^{\bar\rho} + \left\llbracket \lnot G \right\rrbracket^{\bar\rho} \\
					&= \left\llbracket \lnot F \lor \lnot G \right\rrbracket^{\bar\rho} \\
					&= \left\llbracket \lnot (F \land G) \right\rrbracket^{\bar\rho} \\
				\end{align*}
				d'où $P(F \land G)$.
			\item On a
				\begin{align*}
					\left\llbracket (F \lor G)^\star  \right\rrbracket^\rho &= \left\llbracket F^\star  \land G^\star \right\rrbracket^\rho\\
					&= \left\llbracket F^\star \right\rrbracket^\rho \cdot \left\llbracket G^\star \right\rrbracket^\rho \\
					&= \left\llbracket \lnot F \right\rrbracket^{\bar\rho} \cdot \left\llbracket \lnot G \right\rrbracket^{\bar\rho} \\
					&= \left\llbracket \lnot F \land \lnot G \right\rrbracket^{\bar\rho} \\
					&= \left\llbracket \lnot (F \lor G) \right\rrbracket^{\bar\rho} \\
				\end{align*}
				d'où $P(F \lor G)$.
			\item On a \[
					\left\llbracket (\lnot F)^\star  \right\rrbracket^\rho = \left\llbracket \lnot (F^\star) \right\rrbracket^\rho = \overline{\left\llbracket F^\star \right\rrbracket^\rho} = \overline{\left\llbracket \lnot F \right\rrbracket^{\bar\rho}} = \left\llbracket \lnot (\lnot F) \right\rrbracket^{\bar\rho}.
					\] d'où $P(\lnot F)$.
		\end{itemize}
		Par induction, on en conclut que $P(F)$\/ est vraie pour toute formule $F$.
	\item Soit $G$\/ une formule valide. Alors, par définition, $G \equiv \top$. Or, d'après la question précédente, $G^\star \equiv (\top)^\star = \bot$. Ainsi, $G^\star $\/ n'est pas satisfiable.
\end{enumerate}



		\section{Un lemme d'itération}

		\section{Ambigüité}

		\section{Regexp Crossword}
\begin{center}
	\url{https://regexcrossword.com/}
\end{center}

		\documentclass[a4paper]{article}

\usepackage[margin=1in]{geometry}
\usepackage[utf8]{inputenc}
\usepackage[T1]{fontenc}
\usepackage{mathrsfs}
\usepackage{textcomp}
\usepackage[french]{babel}
\usepackage{amsmath}
\usepackage{amssymb}
\usepackage{cancel}
\usepackage{frcursive}
\usepackage[inline]{asymptote}
\usepackage{tikz}
\usepackage[european,straightvoltages,europeanresistors]{circuitikz}
\usepackage{tikz-cd}
\usepackage{tkz-tab}
\usepackage[b]{esvect}
\usepackage[framemethod=TikZ]{mdframed}
\usepackage{centernot}
\usepackage{diagbox}
\usepackage{dsfont}
\usepackage{fancyhdr}
\usepackage{float}
\usepackage{graphicx}
\usepackage{listings}
\usepackage{multicol}
\usepackage{nicematrix}
\usepackage{pdflscape}
\usepackage{stmaryrd}
\usepackage{xfrac}
\usepackage{hep-math-font}
\usepackage{amsthm}
\usepackage{thmtools}
\usepackage{indentfirst}
\usepackage[framemethod=TikZ]{mdframed}
\usepackage{accents}
\usepackage{soulutf8}
\usepackage{mathtools}
\usepackage{bodegraph}
\usepackage{slashbox}
\usepackage{enumitem}
\usepackage{calligra}
\usepackage{cinzel}
\usepackage{BOONDOX-calo}

% Tikz
\usetikzlibrary{babel}
\usetikzlibrary{positioning}
\usetikzlibrary{calc}

% global settings
\frenchspacing
\reversemarginpar
\setuldepth{a}

%\everymath{\displaystyle}

\frenchbsetup{StandardLists=true}

\def\asydir{asy}

%\sisetup{exponent-product=\cdot,output-decimal-marker={,},separate-uncertainty,range-phrase=\;à\;,locale=FR}

\setlength{\parskip}{1em}

\theoremstyle{definition}

% Changing math
\let\emptyset\varnothing
\let\ge\geqslant
\let\le\leqslant
\let\preceq\preccurlyeq
\let\succeq\succcurlyeq
\let\ds\displaystyle
\let\ts\textstyle

\newcommand{\C}{\mathds{C}}
\newcommand{\R}{\mathds{R}}
\newcommand{\Z}{\mathds{Z}}
\newcommand{\N}{\mathds{N}}
\newcommand{\Q}{\mathds{Q}}

\renewcommand{\O}{\emptyset}

\newcommand\ubar[1]{\underaccent{\bar}{#1}}

\renewcommand\Re{\expandafter\mathfrak{Re}}
\renewcommand\Im{\expandafter\mathfrak{Im}}

\let\slantedpartial\partial
\DeclareRobustCommand{\partial}{\text{\rotatebox[origin=t]{20}{\scalebox{0.95}[1]{$\slantedpartial$}}}\hspace{-1pt}}

% merging two maths characters w/ \charfusion
\makeatletter
\def\moverlay{\mathpalette\mov@rlay}
\def\mov@rlay#1#2{\leavevmode\vtop{%
   \baselineskip\z@skip \lineskiplimit-\maxdimen
   \ialign{\hfil$\m@th#1##$\hfil\cr#2\crcr}}}
\newcommand{\charfusion}[3][\mathord]{
    #1{\ifx#1\mathop\vphantom{#2}\fi
        \mathpalette\mov@rlay{#2\cr#3}
      }
    \ifx#1\mathop\expandafter\displaylimits\fi}
\makeatother

% custom math commands
\newcommand{\T}{{\!\!\,\top}}
\newcommand{\avrt}[1]{\rotatebox{-90}{$#1$}}
\newcommand{\bigcupdot}{\charfusion[\mathop]{\bigcup}{\cdot}}
\newcommand{\cupdot}{\charfusion[\mathbin]{\cup}{\cdot}}
%\newcommand{\danger}{{\large\fontencoding{U}\fontfamily{futs}\selectfont\char 66\relax}\;}
\newcommand{\tendsto}[1]{\xrightarrow[#1]{}}
\newcommand{\vrt}[1]{\rotatebox{90}{$#1$}}
\newcommand{\tsup}[1]{\textsuperscript{\underline{#1}}}
\newcommand{\tsub}[1]{\textsubscript{#1}}

\renewcommand{\mod}[1]{~\left[ #1 \right]}
\renewcommand{\t}{{}^t\!}
\newcommand{\s}{\text{\calligra s}}

% custom units / constants
%\DeclareSIUnit{\litre}{\ell}
\let\hbar\hslash

% header / footer
\pagestyle{fancy}
\fancyhead{} \fancyfoot{}
\fancyfoot[C]{\thepage}

% fonts
\let\sc\scshape
\let\bf\bfseries
\let\it\itshape
\let\sl\slshape

% custom math operators
\let\th\relax
\let\det\relax
\DeclareMathOperator*{\codim}{codim}
\DeclareMathOperator*{\dom}{dom}
\DeclareMathOperator*{\gO}{O}
\DeclareMathOperator*{\po}{\text{\cursive o}}
\DeclareMathOperator*{\sgn}{sgn}
\DeclareMathOperator*{\simi}{\sim}
\DeclareMathOperator{\Arccos}{Arccos}
\DeclareMathOperator{\Arcsin}{Arcsin}
\DeclareMathOperator{\Arctan}{Arctan}
\DeclareMathOperator{\Argsh}{Argsh}
\DeclareMathOperator{\Arg}{Arg}
\DeclareMathOperator{\Aut}{Aut}
\DeclareMathOperator{\Card}{Card}
\DeclareMathOperator{\Cl}{\mathcal{C}\!\ell}
\DeclareMathOperator{\Cov}{Cov}
\DeclareMathOperator{\Ker}{Ker}
\DeclareMathOperator{\Mat}{Mat}
\DeclareMathOperator{\PGCD}{PGCD}
\DeclareMathOperator{\PPCM}{PPCM}
\DeclareMathOperator{\Supp}{Supp}
\DeclareMathOperator{\Vect}{Vect}
\DeclareMathOperator{\argmax}{argmax}
\DeclareMathOperator{\argmin}{argmin}
\DeclareMathOperator{\ch}{ch}
\DeclareMathOperator{\com}{com}
\DeclareMathOperator{\cotan}{cotan}
\DeclareMathOperator{\det}{det}
\DeclareMathOperator{\id}{id}
\DeclareMathOperator{\rg}{rg}
\DeclareMathOperator{\rk}{rk}
\DeclareMathOperator{\sh}{sh}
\DeclareMathOperator{\th}{th}
\DeclareMathOperator{\tr}{tr}

% colors and page style
\definecolor{truewhite}{HTML}{ffffff}
\definecolor{white}{HTML}{faf4ed}
\definecolor{trueblack}{HTML}{000000}
\definecolor{black}{HTML}{575279}
\definecolor{mauve}{HTML}{907aa9}
\definecolor{blue}{HTML}{286983}
\definecolor{red}{HTML}{d7827e}
\definecolor{yellow}{HTML}{ea9d34}
\definecolor{gray}{HTML}{9893a5}
\definecolor{grey}{HTML}{9893a5}
\definecolor{green}{HTML}{a0d971}

\pagecolor{white}
\color{black}

\begin{asydef}
	settings.prc = false;
	settings.render=0;

	white = rgb("faf4ed");
	black = rgb("575279");
	blue = rgb("286983");
	red = rgb("d7827e");
	yellow = rgb("f6c177");
	orange = rgb("ea9d34");
	gray = rgb("9893a5");
	grey = rgb("9893a5");
	deepcyan = rgb("56949f");
	pink = rgb("b4637a");
	magenta = rgb("eb6f92");
	green = rgb("a0d971");
	purple = rgb("907aa9");

	defaultpen(black + fontsize(8pt));

	import three;
	currentlight = nolight;
\end{asydef}

% theorems, proofs, ...

\mdfsetup{skipabove=1em,skipbelow=1em, innertopmargin=6pt, innerbottommargin=6pt,}

\declaretheoremstyle[
	headfont=\normalfont\itshape,
	numbered=no,
	postheadspace=\newline,
	headpunct={:},
	qed=\qedsymbol]{demstyle}

\declaretheorem[style=demstyle, name=Démonstration]{dem}

\newcommand\veczero{\kern-1.2pt\vec{\kern1.2pt 0}} % \vec{0} looks weird since the `0' isn't italicized

\makeatletter
\renewcommand{\title}[2]{
	\AtBeginDocument{
		\begin{titlepage}
			\begin{center}
				\vspace{10cm}
				{\Large \sc Chapitre #1}\\
				\vspace{1cm}
				{\Huge \calligra #2}\\
				\vfill
				Hugo {\sc Salou} MPI${}^{\star}$\\
				{\small Dernière mise à jour le \@date }
			\end{center}
		\end{titlepage}
	}
}

\newcommand{\titletp}[4]{
	\AtBeginDocument{
		\begin{titlepage}
			\begin{center}
				\vspace{10cm}
				{\Large \sc tp #1}\\
				\vspace{1cm}
				{\Huge \textsc{\textit{#2}}}\\
				\vfill
				{#3}\textit{MPI}${}^{\star}$\\
			\end{center}
		\end{titlepage}
	}
	\fancyfoot{}\fancyhead{}
	\fancyfoot[R]{#4 \textit{MPI}${}^{\star}$}
	\fancyhead[C]{{\sc tp #1} : #2}
	\fancyhead[R]{\thepage}
}

\newcommand{\titletd}[2]{
	\AtBeginDocument{
		\begin{titlepage}
			\begin{center}
				\vspace{10cm}
				{\Large \sc td #1}\\
				\vspace{1cm}
				{\Huge \calligra #2}\\
				\vfill
				Hugo {\sc Salou} MPI${}^{\star}$\\
				{\small Dernière mise à jour le \@date }
			\end{center}
		\end{titlepage}
	}
}
\makeatother

\newcommand{\sign}{
	\null
	\vfill
	\begin{center}
		{
			\fontfamily{ccr}\selectfont
			\textit{\textbf{\.{\"i}}}
		}
	\end{center}
	\vfill
	\null
}

\renewcommand{\thefootnote}{\emph{\alph{footnote}}}

% figure support
\usepackage{import}
\usepackage{xifthen}
\pdfminorversion=7
\usepackage{pdfpages}
\usepackage{transparent}
\newcommand{\incfig}[1]{%
	\def\svgwidth{\columnwidth}
	\import{./figures/}{#1.pdf_tex}
}

\pdfsuppresswarningpagegroup=1
\ctikzset{tripoles/european not symbol=circle}

\newcommand{\missingpart}{{\large\color{red} Il manque quelque chose ici\ldots}}


\fancyhead[R]{Hugo {\sc Salou}\/ MPI}
\fancyhead[L]{TD\textsubscript2 -- Exercice 7}

\begin{document}
	\let\thesection\relax
	\section{Exercice 3}

{\bf Indication}\/ : pour la $G$, on applique la relation de {\sc Chasles}\/ : l'intégrale $\int_0^7 \mathrm{e}^{-x}\ln x~\mathrm{d}x$\/ converge si et seulement si $\int_0^1 \mathrm{e}^{-x}\ln x\mathrm{d}x$\/ converge et $\int_1^7\mathrm{e}^{-x}\ln x~\mathrm{d}x$\/ converge (qui n'est même pas impropre).

L'intégrale $H = \int_0^1 \frac{\mathrm{e}^{\sin t}}{t}~\mathrm{d}t$\/ est impropre en 0. On sait que $\sin t \tendsto{t\to 0} 0$, et donc, par continuité de la fonction $\exp$\/ en $0$, $\mathrm{e}^{\sin t}\tendsto{t\to 0}e^0 = 1$.
Ainsi, $\frac{\mathrm{e}^{\sin t}}{t} = \mathrm{e}^{\sin t} \times \frac{1}{t} \simi_{t\to 0} \frac{1}{t}$\/ qui ne change pas de signe. Or, $\int_0^1 \frac{1}{t}~\mathrm{d}t$\/ diverge, donc l'intégrale $H$\/ diverge.

L'intégrale $I = \int_{1}^{+\infty} \frac{\mathrm{e}^{\sin t}}{t}~\mathrm{d}t$\/ est impropre en $+\infty$. Par croissance de la fonction exponentielle, on a $\frac{\mathrm{e}^{\sin t}}{t} \ge \frac{\mathrm{e}^{-1}}{t} \ge 0$. Or, l'intégrale $\int_{1}^{+\infty} \frac{1}{t}~\mathrm{d}t$\/ diverge, donc l'intégrale diverge aussi.

L'intégrale $K$, est l'intégrale d'une fonction Gau\ss ienne, et elle est impropre en $+\infty$. On la \guillemotleft~découpe~\guillemotright\ : \[
	\int_{0}^{+\infty} \mathrm{e}^{-x^2}~\mathrm{d}x \text{ converge si et seulement si } \int_{0}^{1} \mathrm{e}^{-x^2}~\mathrm{d}x \text{ converge et } \int_{1}^{+\infty} \mathrm{e}^{-x^2}~\mathrm{d}x \text{ converge.}
\] L'intégrale $\int_{0}^{1} \mathrm{e}^{-x^2}~\mathrm{d}x$\/ n'est même pas impropre, elle converge donc. Et, pour $x \in [1,+\infty[$, on sait, comme $x^2 \ge x$, $0 \le \mathrm{e}^{-x^2} \le \mathrm{e}^{-x}$. Or, $\int_{1}^{+\infty} \mathrm{e}^{-x}~\mathrm{d}x$\/ converge donc $\int_{0}^{+\infty} \mathrm{e}^{-x^2}~\mathrm{d}x$\/ aussi.
On calculera la valeur de cette intégrale dans le {\sc td}\/ \guillemotleft~Intégrales paramétrées.~\guillemotright

Autre méthode pour déterminer la nature de $K$\/ : 
$\mathrm{e}^{-x^2} = \po(\mathrm{e}^{-x})$\/ car $\mathrm{e}^{-x^2} = \underbrace{\mathrm{e}^{-x^2 + x}}_{\to 0} \times \mathrm{e}^{-x}$, car $\mathrm{e}^{-x^2 + x} = \mathrm{e}^{-x^2 \left( 1 - \frac{1}{x} \right)}$\/ et $-x^2\left( 1 - \frac{1}{x} \right) \to -\infty \times 1$.
Et $\int_0^{+\infty} \mathrm{e}^{-x}~\mathrm{d}x$\/ converge donc $\int_0^{+\infty} \mathrm{e}^{-x^2}~\mathrm{d}x$\/ converge.

\begin{figure}[H]
	\centering
	\begin{asy}
		import graph;
		size(10cm);
		draw((-10, 0) -- (10, 0), Arrow(TeXHead));
		draw((0, -3) -- (0, 5), Arrow(TeXHead));
		real f(real x) {
			return 4*exp(-(x/4)^2);
		}
		draw(graph(f, -10, 10), magenta);
	\end{asy}
	\caption{Courbe Gau\ss ienne}
\end{figure}

L'intégrale $F = \int_{7}^{+\infty} \mathrm{e}^{-x}\ln x~\mathrm{d}x$\/ est impropre en $+\infty$. Attention : la fonction n'est pas \guillemotleft~{\color{red} faussement impropre en $+\infty$}.~\guillemotright\ Mais, on peut remarquer que \[
	\mathrm{e}^{-x} \ln x = \mathrm{e}^{-\frac{x}{2}} \underbrace{\mathrm{e}^{-\frac{x}{2}} \ln x}_{\tendsto{x\to +\infty} 0} = \po(\mathrm{e}^{-\frac{x}{2}})
.\] Or, $\int_{7}^{+\infty} \mathrm{e}^{-x}~\mathrm{d}x$\/ converge donc l'intégrale $F$\/ converge aussi.

L'intégrale $G = \int_{0}^{7} \mathrm{e}^{-x}\ln x~\mathrm{d}x$\/ est impropre en 0.
Or, $\mathrm{e}^{-x}\ln x \simi_{x\to 0} \ln x$\/ qui ne change pas de signe au voisinage de 0. Or, $\int_{0}^{7}  \ln x~\mathrm{d}x$\/ converge donc l'intégrale $G$\/ converge également.

L'intégrale $E = \int_{1}^{+\infty} \frac{\ln x}{\sqrt{x}}~\mathrm{d}x$\/ est impropre en $+\infty$. Or, $\forall x \ge \mathrm{e}$, $\frac{\ln(x)}{\sqrt{x}} \ge \frac{1}{\sqrt{x}} \ge 0$\/ converge.
Or, $\int_{1}^{+\infty}  \frac{1}{x^{\sfrac{1}{2}}}~\mathrm{d}x$\/ diverge d'après le critère de {\sc Riemann}\/ en $+\infty$\/ car $\frac{1}{2} < 1$.
D'où l'intégrale $D$\/ diverge.

Autre méthode : intégration par parties. On peut même arriver à calculer une primitive de ${\ln x}\:/{\sqrt{x}}$.

L'intégrale $D = \int_{0}^{1} \frac{\ln x}{\sqrt{x}}~\mathrm{d}x$\/ est impropre en 0. On peut remarque que \[
	0 \le -\frac{\ln x}{\sqrt{x}} = -\frac{x^{0{,}1} \ln x}{x^{0{,}6}} = \po\left( \frac{1}{x^{0{,}6}} \right) \quad\text{car}\quad x^{0{,}1} \ln x \tendsto{x\to 0} 0
\] par croissances comparées.
Or, $\int_{0}^{1} \frac{1}{x^{0{,}6}}~\mathrm{d}x$\/ converge d'après le critère de {\sc Riemann}. D'où $-D$\/ converge et donc $D$\/ converge.

L'intégrale $J = \int_{1}^{+\infty} \frac{\sin t}{\sqrt{t} + \sin t}~\mathrm{d}t$\/ est impropre en $+\infty$. On calcule
\[
	f(t) = \frac{\sin t}{\sqrt{t} + \sin t} = \frac{\sin t}{\sqrt{t}} \times \frac{1}{1+\frac{\sin t}{\sqrt{t}}}
\] et $\frac{\sin t}{\sqrt{t}} \tendsto{t\to +\infty} 0$. D'où \[
	\frac{1}{1+\frac{\sin t}{\sqrt{t}}} = 1  - \frac{\sin t}{\sqrt{t}} + \frac{\sin^2 t}{t} + \po\left( \frac{\sin^2 t}{t} \right)
\] et donc \[
	f(t) = \frac{\cos t}{\sqrt{t}} + \frac{\sin^2 t}{t} + \po\left( \frac{\sin^2 t}{t} \right)
.\]
L'intégrale $\int_{1}^{+\infty} \frac{\sin t}{\sqrt{t}}~\mathrm{d}t$\/ est impropre en $+\infty$.
Soit $x \ge 1$. On calcule avec une intégration par parties,
\[
	\int_{1}^{x} \sin t \times \frac{1}{\sqrt{t}}~\mathrm{d}t = \int_{1}^{x} u'(t)\cdot v(t)~\mathrm{d}t
\] où $u(t) = - \cos t$\/ et $v(t) = \frac{1}{\sqrt{t}} = t^{-\frac{1}{2}}$. Donc
\begin{align*}
	\int_{1}^{x} \frac{\sin t}{\sqrt{t}}~\mathrm{d}t &= \Big[f(t)g(t)\Big]_1^x - \int_{1}^{x} f(t)\cdot g'(t)~\mathrm{d}t\\
	&= \left[ - \frac{\cos t}{\sqrt{t}} \right]_1^x - \int_{1}^{x} (-\cos t)\left( -\frac{1}{2}t^{-\frac{3}{2}} \right)~\mathrm{d}t \\
\end{align*}
D'où \[
	\int_{1}^{x} \frac{\sin t}{\sqrt{t}}~\mathrm{d}t = \cos 1 - \frac{\cos x}{\sqrt{x}} - \frac{1}{2} \int_{1}^{x} \frac{\cos t}{t^{\sfrac{3}{2}}}~\mathrm{d}t
.\]
Or, d'une part $\cos x \times \frac{1}{\sqrt{x}} \tendsto{x\to +\infty} 0$\/ car $\cos$\/ est bornée et $\frac{1}{\sqrt{x}}\tendsto{x\to +\infty} 0$.
Et, d'autre part $\int_{1}^{+\infty} \frac{\cos t}{t^{\sfrac{3}{2}}}~\mathrm{d}t$\/ converge car $\forall t \in [1,+\infty[$, $\left| \frac{\cos t}{t^{\sfrac{3}{2}}} \right| \le \frac{1}{t^{\sfrac{3}{2}}}$\/ et $\int_{1}^{+\infty} \frac{1}{t^{\sfrac{3}{2}}}~\mathrm{d}t$\/ converge.
Pour le 2\tsup{nd} terme du développement limité, on fait une {\sc ipp}, on trouve un terme en $\frac{1}{t^2}$\/ et donc son intégrale converge par critère de {\sc Riemann}. S'il y a des problèmes, voir en {\sc td}.
On étudie maintenant le 3\tsup{ème} terme : \[
	\int_{1}^{+\infty}  \po\left( \frac{\sin t}{t} \right) ~\mathrm{d}t \text{ converge car } \int_{1}^{+\infty} \frac{\sin^2 t}{t} ~\mathrm{d}t \text{ converge et } t\mapsto \frac{\sin^2 t}{t} \text{ est positive}.
\]
Autre méthode : on a \[
	\frac{\sin^2 t}{t} + \po\left( \frac{\sin^2 t}{t} \right) \simi_{t\to +\infty} \frac{\sin^2 t}{t} \text{ qui ne change pas de signe}
.\] Or, $\int_{1}^{+\infty} \frac{\sin^2 t}{t}~\mathrm{d}t$\/ converge et donc \[
	\int_{1}^{+\infty} \left( \frac{\sin^2 t}{t} + \po\left( \frac{\sin^2 t}{t} \right) \right) ~\mathrm{d}t
.\]

\end{document}

		\section{Vocabulaire des automates}

On représente, ci-dessous, l'automate $\mathcal{A}$\/ décrit dans l'énoncé.
\begin{figure}[H]
	\centering
	\tikzfig{automate-ex8}
	\caption{Automate décrit dans l'énoncé de l'exercice 8}
\end{figure}

\begin{enumerate}
	\item Cet automate n'est pas complet : à l'état 0, la lecture d'un $a$\/ peut conduire à l'état 0 ou bien à l'état 1.
	\item Le mot $baba$\/ est reconnu par $\mathcal{A}$\/ mais pas le mot $cabcb$.
	\item L'automate reconnaît les mots dont la 3\tsup{ème} lettre du mot, en partant de la fin, est un $a$.
\end{enumerate}



	}
	\def\addmacros#1{#1}
}
{
	\td[2]{Logique propositionnelle}
	\minitoc
	\renewcommand{\cwd}{../td/td02/}
	\addmacros{
		\section{Filtre RC double}

\begin{enumerate}
	\item En basse fréquence, un condensateur est équivalent à un interrupteur ouvert. En haute fréquence, un condensateur est équivalent à un interrupteur fermé. D'où, le circuit est un filtre passe-bas.
	\item Par une loi des nœuds, et une loi des mailles, on trouve que
		\[
			\ubar{H}(\mathrm{j}\omega) = \dfrac{1}{1 - \left( \dfrac{\omega}{\omega_0} \right)^2 + \mathrm{j} \dfrac{\omega}{Q\cdot \omega_0}}
		\] en notant $\omega_0 = 1 / {RC}$ et $Q = 1 / 3$
	\item On représente le diagramme de \textsc{Bode} du filtre dans la figure ci-dessous.
		\begin{figure}[H]
			\centering
			\includesvg[width=\linewidth]{figures/bode-1.svg}
			\caption{Diagramme de \textsc{Bode} du filtre (échelle logarithmique)}
		\end{figure}
	\item On calcule $\omega_0 \simeq 6\:\mathrm{rad/s}$, ce qui correspond à une fréquence de coupure de $1\:\mathrm{kHz}$. Le signal de sortie est donc \[
			s(t) = \frac{2E}{3\pi}\cdot \sin(\omega t)
		,\] on le représente sur la figure ci-dessous. En effet, on a un déphasage de $-\pi / 2$, et un gain valant $1 / 3$ à $\omega \simeq \omega_0$.
		\begin{figure}[H]
			\centering
			\includesvg[width=\linewidth]{figures/signal-1.svg}
			\caption{Signal résultant}
		\end{figure}
\end{enumerate}

		\section{Tri topologique}

\begin{figure}[H]
	\centering
	\tikzfig{ex2-q1}
	\caption{Exemple de graphe}
\end{figure}

\begin{enumerate}
	\item Dans le graphe ci-dessus, $a \to c \to b$\/ est un tri topologique mais pas un parcours.
	\item Dans le même graphe, $b \to a \to c$\/ est un parcours mais pas un tri topologique.
	\item Supposons que $L_1$\/ possède un prédécesseur, on le note $L_i$\/ où $i > 1$. Ainsi, $(L_i, L_1) \in A$\/ et donc $i < 1$, ce qui est absurde. De même pour le dernier.
	\item Il existe un tri topologique si, et seulement si le graphe est acyclique.
		\begin{itemize}
			\item[``$\implies$'']
				Soit $L_1,\ldots,L_n$\/ un tri topologique. Montrons que le graphe est acyclique.
				Par l'absurde, on suppose le graphe non acyclique : il existe $(i,j) \in \llbracket 1,n \rrbracket^2$\/ avec $i \neq j$\/ tels que $T_i \to \cdots \to T_j$\/ et $T_j \to \cdots \to T_i$ soient deux chemins valides. Ainsi, comme le tri est topologique et par récurrence, $i \le j$\/ et $j \le i$\/ et donc $i = j$, ce qui est absurde car $i$\/ et $j$\/ sont supposés différents. Le graphe est donc acyclique.
			\item[``$\impliedby$'']
				Soit $G$\/ un graphe tel que tous les sommets possèdent une arrête entrante. On suppose par l'absurde ce graphe acyclique.
				Soit $x_0$\/ un sommet du graphe.
				On construit par récurrence $x_0,x_1,\ldots,x_n,x_{n+1},\ldots$ les successeurs successifs. Il y a un nombre fini de sommets donc deux sommets sont identiques. Donc, il y a nécessairement un cycle, ce qui est absurde.
				\begin{algorithm}[H]
					\centering
					\begin{algorithmic}[1]
						\Entree $G = (S, A)$\/ un graphe acyclique
						\Sortie $\mathrm{Res}$\/ un tri topologique.
						\State $\mathrm{Res} \gets [\quad]$\/
						\While{$G \neq \O$}
							\State Soit $x$\/ un sommet de $G$\/ sans prédécesseur
							\State $G \gets \big(S \setminus \{x\}, A \cap (S \setminus \{x\})^2\big)$\/ 
							\State $\mathrm{Res} \gets \mathrm{Res} \cdot [x]$\/
						\EndWhile
						\State\Return $\mathrm{Res}$\/
					\end{algorithmic}
					\caption{Génération d'un tri topologique d'un graphe acyclique}
				\end{algorithm}
		\end{itemize}
	\item~
		\begin{algorithm}[H]
			\centering
			\begin{algorithmic}[1]
				\Entree $G = (S, A)$\/ un graphe
				\Sortie $\mathrm{Res}$\/ un tri topologique, ou un cycle
				\State $\mathrm{Res} \gets [\quad]$\/
				\While{$G \neq \O$}
					\If{il existe $x$ sans prédécesseurs}
						\State Soit $x$\/ un sommet de $G$\/ sans prédécesseur
						\State $G \gets \big(S \setminus \{x\}, A \cap (S \setminus \{x\})^2\big)$\/ 
						\State $\mathrm{Res} \gets \mathrm{Res} \cdot [x]$\/
					\Else
						\State Soit $x \in S$\/ 
						\State Soit $x \gets x_1 \gets x_2 \gets \cdots \gets x_i$\/ la suite des prédécesseurs
						\State\Return $x_i,x_{i+1},\ldots,x_i$, un cycle
					\EndIf
				\EndWhile
				\State\Return $\mathrm{Res}$\/
			\end{algorithmic}
			\caption{Génération d'un tri topologique d'un graphe}
		\end{algorithm}
	\item On utilise la représentation par liste d'adjacence, et on stocke le nombre de prédécesseurs que l'on décroit à chaque choix de sommet.
	\item On essaie de trouver un tri topologique, et on voit si l'on trouve un cycle.
\end{enumerate}

		\section{Formules duales}

\begin{enumerate}
	\item On définit par induction $(\cdot)^\star$\/ comme
		\begin{multicols}{3}
			\begin{itemize}
				\item $\top^\star = \bot$\/ ;
				\item $\bot^\star = \top$\/ ;
				\item $(G \lor H)^\star = G^\star \land H^\star$\/ ;
				\item $(G \land H)^\star = G^\star \lor H^\star$\/ ;
				\item $(\lnot G)^\star = \lnot G^\star$\/ ;
				\item $p^\star = p$.
			\end{itemize}
		\end{multicols}
	\item Soit $\rho \in \mathds{B}^{\mathcal{P}}$. Montrons, par induction, $P(H) : ``\left\llbracket H^\star \right\rrbracket^\rho = \left\llbracket \lnot H \right\rrbracket^{\bar \rho}"$\/ où $\bar{\rho} : p \mapsto \overline{\rho(p)}$.
		\begin{itemize}
			\item On a $\left\llbracket \bot^\star \right\rrbracket^\rho = \left\llbracket \top \right\rrbracket^\rho = \mathbf{V}$, et $\left\llbracket \lnot \bot \right\rrbracket^{\bar\rho} = \left\llbracket \top \right\rrbracket^{\bar\rho} = \mathbf{V}$, d'où $P(\bot)$.
			\item On a $\left\llbracket \top^\star \right\rrbracket^\rho = \left\llbracket \bot \right\rrbracket^\rho = \mathbf{F}$, et $\left\llbracket \lnot \top \right\rrbracket^{\bar\rho} = \left\llbracket \bot \right\rrbracket^{\bar\rho} = \mathbf{F}$, d'où $P(\top)$.
			\item Soit $p \in \mathcal{P}$. On a $\left\llbracket p^\star  \right\rrbracket^\rho = \left\llbracket p \right\rrbracket^\rho = \rho(p)$, et $\left\llbracket \lnot p \right\rrbracket^{\bar\rho} = \overline{\left\llbracket p \right\rrbracket^{\bar\rho}} = \overline{\bar\rho(p)} = \overline{\overline{\rho(p)}} = \rho(p)$, d'où~$P(p)$.
		\end{itemize}
		Soient $F$\/ et $G$\/ deux formules.
		\begin{itemize}
			\item On a
				\begin{align*}
					\left\llbracket (F \land G)^\star  \right\rrbracket^\rho &= \left\llbracket F^\star  \lor G^\star \right\rrbracket^\rho\\
					&= \left\llbracket F^\star \right\rrbracket^\rho + \left\llbracket G^\star \right\rrbracket^\rho \\
					&= \left\llbracket \lnot F \right\rrbracket^{\bar\rho} + \left\llbracket \lnot G \right\rrbracket^{\bar\rho} \\
					&= \left\llbracket \lnot F \lor \lnot G \right\rrbracket^{\bar\rho} \\
					&= \left\llbracket \lnot (F \land G) \right\rrbracket^{\bar\rho} \\
				\end{align*}
				d'où $P(F \land G)$.
			\item On a
				\begin{align*}
					\left\llbracket (F \lor G)^\star  \right\rrbracket^\rho &= \left\llbracket F^\star  \land G^\star \right\rrbracket^\rho\\
					&= \left\llbracket F^\star \right\rrbracket^\rho \cdot \left\llbracket G^\star \right\rrbracket^\rho \\
					&= \left\llbracket \lnot F \right\rrbracket^{\bar\rho} \cdot \left\llbracket \lnot G \right\rrbracket^{\bar\rho} \\
					&= \left\llbracket \lnot F \land \lnot G \right\rrbracket^{\bar\rho} \\
					&= \left\llbracket \lnot (F \lor G) \right\rrbracket^{\bar\rho} \\
				\end{align*}
				d'où $P(F \lor G)$.
			\item On a \[
					\left\llbracket (\lnot F)^\star  \right\rrbracket^\rho = \left\llbracket \lnot (F^\star) \right\rrbracket^\rho = \overline{\left\llbracket F^\star \right\rrbracket^\rho} = \overline{\left\llbracket \lnot F \right\rrbracket^{\bar\rho}} = \left\llbracket \lnot (\lnot F) \right\rrbracket^{\bar\rho}.
					\] d'où $P(\lnot F)$.
		\end{itemize}
		Par induction, on en conclut que $P(F)$\/ est vraie pour toute formule $F$.
	\item Soit $G$\/ une formule valide. Alors, par définition, $G \equiv \top$. Or, d'après la question précédente, $G^\star \equiv (\top)^\star = \bot$. Ainsi, $G^\star $\/ n'est pas satisfiable.
\end{enumerate}



		\section{Un lemme d'itération}

		\section{Ambigüité}

		\section{Regexp Crossword}
\begin{center}
	\url{https://regexcrossword.com/}
\end{center}

		\documentclass[a4paper]{article}

\usepackage[margin=1in]{geometry}
\usepackage[utf8]{inputenc}
\usepackage[T1]{fontenc}
\usepackage{mathrsfs}
\usepackage{textcomp}
\usepackage[french]{babel}
\usepackage{amsmath}
\usepackage{amssymb}
\usepackage{cancel}
\usepackage{frcursive}
\usepackage[inline]{asymptote}
\usepackage{tikz}
\usepackage[european,straightvoltages,europeanresistors]{circuitikz}
\usepackage{tikz-cd}
\usepackage{tkz-tab}
\usepackage[b]{esvect}
\usepackage[framemethod=TikZ]{mdframed}
\usepackage{centernot}
\usepackage{diagbox}
\usepackage{dsfont}
\usepackage{fancyhdr}
\usepackage{float}
\usepackage{graphicx}
\usepackage{listings}
\usepackage{multicol}
\usepackage{nicematrix}
\usepackage{pdflscape}
\usepackage{stmaryrd}
\usepackage{xfrac}
\usepackage{hep-math-font}
\usepackage{amsthm}
\usepackage{thmtools}
\usepackage{indentfirst}
\usepackage[framemethod=TikZ]{mdframed}
\usepackage{accents}
\usepackage{soulutf8}
\usepackage{mathtools}
\usepackage{bodegraph}
\usepackage{slashbox}
\usepackage{enumitem}
\usepackage{calligra}
\usepackage{cinzel}
\usepackage{BOONDOX-calo}

% Tikz
\usetikzlibrary{babel}
\usetikzlibrary{positioning}
\usetikzlibrary{calc}

% global settings
\frenchspacing
\reversemarginpar
\setuldepth{a}

%\everymath{\displaystyle}

\frenchbsetup{StandardLists=true}

\def\asydir{asy}

%\sisetup{exponent-product=\cdot,output-decimal-marker={,},separate-uncertainty,range-phrase=\;à\;,locale=FR}

\setlength{\parskip}{1em}

\theoremstyle{definition}

% Changing math
\let\emptyset\varnothing
\let\ge\geqslant
\let\le\leqslant
\let\preceq\preccurlyeq
\let\succeq\succcurlyeq
\let\ds\displaystyle
\let\ts\textstyle

\newcommand{\C}{\mathds{C}}
\newcommand{\R}{\mathds{R}}
\newcommand{\Z}{\mathds{Z}}
\newcommand{\N}{\mathds{N}}
\newcommand{\Q}{\mathds{Q}}

\renewcommand{\O}{\emptyset}

\newcommand\ubar[1]{\underaccent{\bar}{#1}}

\renewcommand\Re{\expandafter\mathfrak{Re}}
\renewcommand\Im{\expandafter\mathfrak{Im}}

\let\slantedpartial\partial
\DeclareRobustCommand{\partial}{\text{\rotatebox[origin=t]{20}{\scalebox{0.95}[1]{$\slantedpartial$}}}\hspace{-1pt}}

% merging two maths characters w/ \charfusion
\makeatletter
\def\moverlay{\mathpalette\mov@rlay}
\def\mov@rlay#1#2{\leavevmode\vtop{%
   \baselineskip\z@skip \lineskiplimit-\maxdimen
   \ialign{\hfil$\m@th#1##$\hfil\cr#2\crcr}}}
\newcommand{\charfusion}[3][\mathord]{
    #1{\ifx#1\mathop\vphantom{#2}\fi
        \mathpalette\mov@rlay{#2\cr#3}
      }
    \ifx#1\mathop\expandafter\displaylimits\fi}
\makeatother

% custom math commands
\newcommand{\T}{{\!\!\,\top}}
\newcommand{\avrt}[1]{\rotatebox{-90}{$#1$}}
\newcommand{\bigcupdot}{\charfusion[\mathop]{\bigcup}{\cdot}}
\newcommand{\cupdot}{\charfusion[\mathbin]{\cup}{\cdot}}
%\newcommand{\danger}{{\large\fontencoding{U}\fontfamily{futs}\selectfont\char 66\relax}\;}
\newcommand{\tendsto}[1]{\xrightarrow[#1]{}}
\newcommand{\vrt}[1]{\rotatebox{90}{$#1$}}
\newcommand{\tsup}[1]{\textsuperscript{\underline{#1}}}
\newcommand{\tsub}[1]{\textsubscript{#1}}

\renewcommand{\mod}[1]{~\left[ #1 \right]}
\renewcommand{\t}{{}^t\!}
\newcommand{\s}{\text{\calligra s}}

% custom units / constants
%\DeclareSIUnit{\litre}{\ell}
\let\hbar\hslash

% header / footer
\pagestyle{fancy}
\fancyhead{} \fancyfoot{}
\fancyfoot[C]{\thepage}

% fonts
\let\sc\scshape
\let\bf\bfseries
\let\it\itshape
\let\sl\slshape

% custom math operators
\let\th\relax
\let\det\relax
\DeclareMathOperator*{\codim}{codim}
\DeclareMathOperator*{\dom}{dom}
\DeclareMathOperator*{\gO}{O}
\DeclareMathOperator*{\po}{\text{\cursive o}}
\DeclareMathOperator*{\sgn}{sgn}
\DeclareMathOperator*{\simi}{\sim}
\DeclareMathOperator{\Arccos}{Arccos}
\DeclareMathOperator{\Arcsin}{Arcsin}
\DeclareMathOperator{\Arctan}{Arctan}
\DeclareMathOperator{\Argsh}{Argsh}
\DeclareMathOperator{\Arg}{Arg}
\DeclareMathOperator{\Aut}{Aut}
\DeclareMathOperator{\Card}{Card}
\DeclareMathOperator{\Cl}{\mathcal{C}\!\ell}
\DeclareMathOperator{\Cov}{Cov}
\DeclareMathOperator{\Ker}{Ker}
\DeclareMathOperator{\Mat}{Mat}
\DeclareMathOperator{\PGCD}{PGCD}
\DeclareMathOperator{\PPCM}{PPCM}
\DeclareMathOperator{\Supp}{Supp}
\DeclareMathOperator{\Vect}{Vect}
\DeclareMathOperator{\argmax}{argmax}
\DeclareMathOperator{\argmin}{argmin}
\DeclareMathOperator{\ch}{ch}
\DeclareMathOperator{\com}{com}
\DeclareMathOperator{\cotan}{cotan}
\DeclareMathOperator{\det}{det}
\DeclareMathOperator{\id}{id}
\DeclareMathOperator{\rg}{rg}
\DeclareMathOperator{\rk}{rk}
\DeclareMathOperator{\sh}{sh}
\DeclareMathOperator{\th}{th}
\DeclareMathOperator{\tr}{tr}

% colors and page style
\definecolor{truewhite}{HTML}{ffffff}
\definecolor{white}{HTML}{faf4ed}
\definecolor{trueblack}{HTML}{000000}
\definecolor{black}{HTML}{575279}
\definecolor{mauve}{HTML}{907aa9}
\definecolor{blue}{HTML}{286983}
\definecolor{red}{HTML}{d7827e}
\definecolor{yellow}{HTML}{ea9d34}
\definecolor{gray}{HTML}{9893a5}
\definecolor{grey}{HTML}{9893a5}
\definecolor{green}{HTML}{a0d971}

\pagecolor{white}
\color{black}

\begin{asydef}
	settings.prc = false;
	settings.render=0;

	white = rgb("faf4ed");
	black = rgb("575279");
	blue = rgb("286983");
	red = rgb("d7827e");
	yellow = rgb("f6c177");
	orange = rgb("ea9d34");
	gray = rgb("9893a5");
	grey = rgb("9893a5");
	deepcyan = rgb("56949f");
	pink = rgb("b4637a");
	magenta = rgb("eb6f92");
	green = rgb("a0d971");
	purple = rgb("907aa9");

	defaultpen(black + fontsize(8pt));

	import three;
	currentlight = nolight;
\end{asydef}

% theorems, proofs, ...

\mdfsetup{skipabove=1em,skipbelow=1em, innertopmargin=6pt, innerbottommargin=6pt,}

\declaretheoremstyle[
	headfont=\normalfont\itshape,
	numbered=no,
	postheadspace=\newline,
	headpunct={:},
	qed=\qedsymbol]{demstyle}

\declaretheorem[style=demstyle, name=Démonstration]{dem}

\newcommand\veczero{\kern-1.2pt\vec{\kern1.2pt 0}} % \vec{0} looks weird since the `0' isn't italicized

\makeatletter
\renewcommand{\title}[2]{
	\AtBeginDocument{
		\begin{titlepage}
			\begin{center}
				\vspace{10cm}
				{\Large \sc Chapitre #1}\\
				\vspace{1cm}
				{\Huge \calligra #2}\\
				\vfill
				Hugo {\sc Salou} MPI${}^{\star}$\\
				{\small Dernière mise à jour le \@date }
			\end{center}
		\end{titlepage}
	}
}

\newcommand{\titletp}[4]{
	\AtBeginDocument{
		\begin{titlepage}
			\begin{center}
				\vspace{10cm}
				{\Large \sc tp #1}\\
				\vspace{1cm}
				{\Huge \textsc{\textit{#2}}}\\
				\vfill
				{#3}\textit{MPI}${}^{\star}$\\
			\end{center}
		\end{titlepage}
	}
	\fancyfoot{}\fancyhead{}
	\fancyfoot[R]{#4 \textit{MPI}${}^{\star}$}
	\fancyhead[C]{{\sc tp #1} : #2}
	\fancyhead[R]{\thepage}
}

\newcommand{\titletd}[2]{
	\AtBeginDocument{
		\begin{titlepage}
			\begin{center}
				\vspace{10cm}
				{\Large \sc td #1}\\
				\vspace{1cm}
				{\Huge \calligra #2}\\
				\vfill
				Hugo {\sc Salou} MPI${}^{\star}$\\
				{\small Dernière mise à jour le \@date }
			\end{center}
		\end{titlepage}
	}
}
\makeatother

\newcommand{\sign}{
	\null
	\vfill
	\begin{center}
		{
			\fontfamily{ccr}\selectfont
			\textit{\textbf{\.{\"i}}}
		}
	\end{center}
	\vfill
	\null
}

\renewcommand{\thefootnote}{\emph{\alph{footnote}}}

% figure support
\usepackage{import}
\usepackage{xifthen}
\pdfminorversion=7
\usepackage{pdfpages}
\usepackage{transparent}
\newcommand{\incfig}[1]{%
	\def\svgwidth{\columnwidth}
	\import{./figures/}{#1.pdf_tex}
}

\pdfsuppresswarningpagegroup=1
\ctikzset{tripoles/european not symbol=circle}

\newcommand{\missingpart}{{\large\color{red} Il manque quelque chose ici\ldots}}


\fancyhead[R]{Hugo {\sc Salou}\/ MPI}
\fancyhead[L]{TD\textsubscript2 -- Exercice 7}

\begin{document}
	\let\thesection\relax
	\section{Exercice 3}

{\bf Indication}\/ : pour la $G$, on applique la relation de {\sc Chasles}\/ : l'intégrale $\int_0^7 \mathrm{e}^{-x}\ln x~\mathrm{d}x$\/ converge si et seulement si $\int_0^1 \mathrm{e}^{-x}\ln x\mathrm{d}x$\/ converge et $\int_1^7\mathrm{e}^{-x}\ln x~\mathrm{d}x$\/ converge (qui n'est même pas impropre).

L'intégrale $H = \int_0^1 \frac{\mathrm{e}^{\sin t}}{t}~\mathrm{d}t$\/ est impropre en 0. On sait que $\sin t \tendsto{t\to 0} 0$, et donc, par continuité de la fonction $\exp$\/ en $0$, $\mathrm{e}^{\sin t}\tendsto{t\to 0}e^0 = 1$.
Ainsi, $\frac{\mathrm{e}^{\sin t}}{t} = \mathrm{e}^{\sin t} \times \frac{1}{t} \simi_{t\to 0} \frac{1}{t}$\/ qui ne change pas de signe. Or, $\int_0^1 \frac{1}{t}~\mathrm{d}t$\/ diverge, donc l'intégrale $H$\/ diverge.

L'intégrale $I = \int_{1}^{+\infty} \frac{\mathrm{e}^{\sin t}}{t}~\mathrm{d}t$\/ est impropre en $+\infty$. Par croissance de la fonction exponentielle, on a $\frac{\mathrm{e}^{\sin t}}{t} \ge \frac{\mathrm{e}^{-1}}{t} \ge 0$. Or, l'intégrale $\int_{1}^{+\infty} \frac{1}{t}~\mathrm{d}t$\/ diverge, donc l'intégrale diverge aussi.

L'intégrale $K$, est l'intégrale d'une fonction Gau\ss ienne, et elle est impropre en $+\infty$. On la \guillemotleft~découpe~\guillemotright\ : \[
	\int_{0}^{+\infty} \mathrm{e}^{-x^2}~\mathrm{d}x \text{ converge si et seulement si } \int_{0}^{1} \mathrm{e}^{-x^2}~\mathrm{d}x \text{ converge et } \int_{1}^{+\infty} \mathrm{e}^{-x^2}~\mathrm{d}x \text{ converge.}
\] L'intégrale $\int_{0}^{1} \mathrm{e}^{-x^2}~\mathrm{d}x$\/ n'est même pas impropre, elle converge donc. Et, pour $x \in [1,+\infty[$, on sait, comme $x^2 \ge x$, $0 \le \mathrm{e}^{-x^2} \le \mathrm{e}^{-x}$. Or, $\int_{1}^{+\infty} \mathrm{e}^{-x}~\mathrm{d}x$\/ converge donc $\int_{0}^{+\infty} \mathrm{e}^{-x^2}~\mathrm{d}x$\/ aussi.
On calculera la valeur de cette intégrale dans le {\sc td}\/ \guillemotleft~Intégrales paramétrées.~\guillemotright

Autre méthode pour déterminer la nature de $K$\/ : 
$\mathrm{e}^{-x^2} = \po(\mathrm{e}^{-x})$\/ car $\mathrm{e}^{-x^2} = \underbrace{\mathrm{e}^{-x^2 + x}}_{\to 0} \times \mathrm{e}^{-x}$, car $\mathrm{e}^{-x^2 + x} = \mathrm{e}^{-x^2 \left( 1 - \frac{1}{x} \right)}$\/ et $-x^2\left( 1 - \frac{1}{x} \right) \to -\infty \times 1$.
Et $\int_0^{+\infty} \mathrm{e}^{-x}~\mathrm{d}x$\/ converge donc $\int_0^{+\infty} \mathrm{e}^{-x^2}~\mathrm{d}x$\/ converge.

\begin{figure}[H]
	\centering
	\begin{asy}
		import graph;
		size(10cm);
		draw((-10, 0) -- (10, 0), Arrow(TeXHead));
		draw((0, -3) -- (0, 5), Arrow(TeXHead));
		real f(real x) {
			return 4*exp(-(x/4)^2);
		}
		draw(graph(f, -10, 10), magenta);
	\end{asy}
	\caption{Courbe Gau\ss ienne}
\end{figure}

L'intégrale $F = \int_{7}^{+\infty} \mathrm{e}^{-x}\ln x~\mathrm{d}x$\/ est impropre en $+\infty$. Attention : la fonction n'est pas \guillemotleft~{\color{red} faussement impropre en $+\infty$}.~\guillemotright\ Mais, on peut remarquer que \[
	\mathrm{e}^{-x} \ln x = \mathrm{e}^{-\frac{x}{2}} \underbrace{\mathrm{e}^{-\frac{x}{2}} \ln x}_{\tendsto{x\to +\infty} 0} = \po(\mathrm{e}^{-\frac{x}{2}})
.\] Or, $\int_{7}^{+\infty} \mathrm{e}^{-x}~\mathrm{d}x$\/ converge donc l'intégrale $F$\/ converge aussi.

L'intégrale $G = \int_{0}^{7} \mathrm{e}^{-x}\ln x~\mathrm{d}x$\/ est impropre en 0.
Or, $\mathrm{e}^{-x}\ln x \simi_{x\to 0} \ln x$\/ qui ne change pas de signe au voisinage de 0. Or, $\int_{0}^{7}  \ln x~\mathrm{d}x$\/ converge donc l'intégrale $G$\/ converge également.

L'intégrale $E = \int_{1}^{+\infty} \frac{\ln x}{\sqrt{x}}~\mathrm{d}x$\/ est impropre en $+\infty$. Or, $\forall x \ge \mathrm{e}$, $\frac{\ln(x)}{\sqrt{x}} \ge \frac{1}{\sqrt{x}} \ge 0$\/ converge.
Or, $\int_{1}^{+\infty}  \frac{1}{x^{\sfrac{1}{2}}}~\mathrm{d}x$\/ diverge d'après le critère de {\sc Riemann}\/ en $+\infty$\/ car $\frac{1}{2} < 1$.
D'où l'intégrale $D$\/ diverge.

Autre méthode : intégration par parties. On peut même arriver à calculer une primitive de ${\ln x}\:/{\sqrt{x}}$.

L'intégrale $D = \int_{0}^{1} \frac{\ln x}{\sqrt{x}}~\mathrm{d}x$\/ est impropre en 0. On peut remarque que \[
	0 \le -\frac{\ln x}{\sqrt{x}} = -\frac{x^{0{,}1} \ln x}{x^{0{,}6}} = \po\left( \frac{1}{x^{0{,}6}} \right) \quad\text{car}\quad x^{0{,}1} \ln x \tendsto{x\to 0} 0
\] par croissances comparées.
Or, $\int_{0}^{1} \frac{1}{x^{0{,}6}}~\mathrm{d}x$\/ converge d'après le critère de {\sc Riemann}. D'où $-D$\/ converge et donc $D$\/ converge.

L'intégrale $J = \int_{1}^{+\infty} \frac{\sin t}{\sqrt{t} + \sin t}~\mathrm{d}t$\/ est impropre en $+\infty$. On calcule
\[
	f(t) = \frac{\sin t}{\sqrt{t} + \sin t} = \frac{\sin t}{\sqrt{t}} \times \frac{1}{1+\frac{\sin t}{\sqrt{t}}}
\] et $\frac{\sin t}{\sqrt{t}} \tendsto{t\to +\infty} 0$. D'où \[
	\frac{1}{1+\frac{\sin t}{\sqrt{t}}} = 1  - \frac{\sin t}{\sqrt{t}} + \frac{\sin^2 t}{t} + \po\left( \frac{\sin^2 t}{t} \right)
\] et donc \[
	f(t) = \frac{\cos t}{\sqrt{t}} + \frac{\sin^2 t}{t} + \po\left( \frac{\sin^2 t}{t} \right)
.\]
L'intégrale $\int_{1}^{+\infty} \frac{\sin t}{\sqrt{t}}~\mathrm{d}t$\/ est impropre en $+\infty$.
Soit $x \ge 1$. On calcule avec une intégration par parties,
\[
	\int_{1}^{x} \sin t \times \frac{1}{\sqrt{t}}~\mathrm{d}t = \int_{1}^{x} u'(t)\cdot v(t)~\mathrm{d}t
\] où $u(t) = - \cos t$\/ et $v(t) = \frac{1}{\sqrt{t}} = t^{-\frac{1}{2}}$. Donc
\begin{align*}
	\int_{1}^{x} \frac{\sin t}{\sqrt{t}}~\mathrm{d}t &= \Big[f(t)g(t)\Big]_1^x - \int_{1}^{x} f(t)\cdot g'(t)~\mathrm{d}t\\
	&= \left[ - \frac{\cos t}{\sqrt{t}} \right]_1^x - \int_{1}^{x} (-\cos t)\left( -\frac{1}{2}t^{-\frac{3}{2}} \right)~\mathrm{d}t \\
\end{align*}
D'où \[
	\int_{1}^{x} \frac{\sin t}{\sqrt{t}}~\mathrm{d}t = \cos 1 - \frac{\cos x}{\sqrt{x}} - \frac{1}{2} \int_{1}^{x} \frac{\cos t}{t^{\sfrac{3}{2}}}~\mathrm{d}t
.\]
Or, d'une part $\cos x \times \frac{1}{\sqrt{x}} \tendsto{x\to +\infty} 0$\/ car $\cos$\/ est bornée et $\frac{1}{\sqrt{x}}\tendsto{x\to +\infty} 0$.
Et, d'autre part $\int_{1}^{+\infty} \frac{\cos t}{t^{\sfrac{3}{2}}}~\mathrm{d}t$\/ converge car $\forall t \in [1,+\infty[$, $\left| \frac{\cos t}{t^{\sfrac{3}{2}}} \right| \le \frac{1}{t^{\sfrac{3}{2}}}$\/ et $\int_{1}^{+\infty} \frac{1}{t^{\sfrac{3}{2}}}~\mathrm{d}t$\/ converge.
Pour le 2\tsup{nd} terme du développement limité, on fait une {\sc ipp}, on trouve un terme en $\frac{1}{t^2}$\/ et donc son intégrale converge par critère de {\sc Riemann}. S'il y a des problèmes, voir en {\sc td}.
On étudie maintenant le 3\tsup{ème} terme : \[
	\int_{1}^{+\infty}  \po\left( \frac{\sin t}{t} \right) ~\mathrm{d}t \text{ converge car } \int_{1}^{+\infty} \frac{\sin^2 t}{t} ~\mathrm{d}t \text{ converge et } t\mapsto \frac{\sin^2 t}{t} \text{ est positive}.
\]
Autre méthode : on a \[
	\frac{\sin^2 t}{t} + \po\left( \frac{\sin^2 t}{t} \right) \simi_{t\to +\infty} \frac{\sin^2 t}{t} \text{ qui ne change pas de signe}
.\] Or, $\int_{1}^{+\infty} \frac{\sin^2 t}{t}~\mathrm{d}t$\/ converge et donc \[
	\int_{1}^{+\infty} \left( \frac{\sin^2 t}{t} + \po\left( \frac{\sin^2 t}{t} \right) \right) ~\mathrm{d}t
.\]

\end{document}

		\section{Vocabulaire des automates}

On représente, ci-dessous, l'automate $\mathcal{A}$\/ décrit dans l'énoncé.
\begin{figure}[H]
	\centering
	\tikzfig{automate-ex8}
	\caption{Automate décrit dans l'énoncé de l'exercice 8}
\end{figure}

\begin{enumerate}
	\item Cet automate n'est pas complet : à l'état 0, la lecture d'un $a$\/ peut conduire à l'état 0 ou bien à l'état 1.
	\item Le mot $baba$\/ est reconnu par $\mathcal{A}$\/ mais pas le mot $cabcb$.
	\item L'automate reconnaît les mots dont la 3\tsup{ème} lettre du mot, en partant de la fin, est un $a$.
\end{enumerate}



	}
	\def\addmacros#1{#1}
}
{
	\td[3]{Langages et expressions régulières}
	\minitoc
	\renewcommand{\cwd}{../td/td03/}
	\addmacros{
		\section{Filtre RC double}

\begin{enumerate}
	\item En basse fréquence, un condensateur est équivalent à un interrupteur ouvert. En haute fréquence, un condensateur est équivalent à un interrupteur fermé. D'où, le circuit est un filtre passe-bas.
	\item Par une loi des nœuds, et une loi des mailles, on trouve que
		\[
			\ubar{H}(\mathrm{j}\omega) = \dfrac{1}{1 - \left( \dfrac{\omega}{\omega_0} \right)^2 + \mathrm{j} \dfrac{\omega}{Q\cdot \omega_0}}
		\] en notant $\omega_0 = 1 / {RC}$ et $Q = 1 / 3$
	\item On représente le diagramme de \textsc{Bode} du filtre dans la figure ci-dessous.
		\begin{figure}[H]
			\centering
			\includesvg[width=\linewidth]{figures/bode-1.svg}
			\caption{Diagramme de \textsc{Bode} du filtre (échelle logarithmique)}
		\end{figure}
	\item On calcule $\omega_0 \simeq 6\:\mathrm{rad/s}$, ce qui correspond à une fréquence de coupure de $1\:\mathrm{kHz}$. Le signal de sortie est donc \[
			s(t) = \frac{2E}{3\pi}\cdot \sin(\omega t)
		,\] on le représente sur la figure ci-dessous. En effet, on a un déphasage de $-\pi / 2$, et un gain valant $1 / 3$ à $\omega \simeq \omega_0$.
		\begin{figure}[H]
			\centering
			\includesvg[width=\linewidth]{figures/signal-1.svg}
			\caption{Signal résultant}
		\end{figure}
\end{enumerate}

		\section{Exercice 10}

\begin{enumerate}
	\item On a $u_6 \leadsto \hat{v}_\mathrm{e}$, $u_3 \leadsto \hat{v}_\mathrm{c}$\/ (composante continue), $u_2 \leadsto \hat{v}_\mathrm{d}$\/ (discontinuités), $u_4 \leadsto \hat{v}_\mathrm{a}$\/ (composante continue), $u_5 \leadsto \hat{v}_\mathrm{f}$\/ (nombre de fréquences) et $u_1 \leadsto \hat{v}_\mathrm{b}$.
	\item
		Le filtre donnant le signal $\hat{v}_\mathrm{g}$\/ est un passe-bandes (les hautes et basses fréquences sont éliminées) et c'est un filtre non linéaire (de nouvelles fréquences apparaissent).

		Le filtre donnant le signal $\hat{v}_\mathrm{h}$\/ est un filtre passe-bas.

		Le filtre donnant le signal $\hat{v}_\mathrm{i}$\/ est un filtre passe-haut dont sa fréquence de coupure est inférieure à $1\:\mathrm{kHz}$.
\end{enumerate}


		\section{Déterminisation 1}

\begin{enumerate}
	\item \tikzfig{ex11-1}
	\item \tikzfig{ex11-2}
\end{enumerate}

		\section{Déterminisation 2}

\begin{enumerate}
	\item \tikzfig{ex12-1}
	\item \tikzfig{ex12-2}
	\item \tikzfig{ex12-3}
\end{enumerate}

		\section{Tri topologique}

\begin{figure}[H]
	\centering
	\tikzfig{ex2-q1}
	\caption{Exemple de graphe}
\end{figure}

\begin{enumerate}
	\item Dans le graphe ci-dessus, $a \to c \to b$\/ est un tri topologique mais pas un parcours.
	\item Dans le même graphe, $b \to a \to c$\/ est un parcours mais pas un tri topologique.
	\item Supposons que $L_1$\/ possède un prédécesseur, on le note $L_i$\/ où $i > 1$. Ainsi, $(L_i, L_1) \in A$\/ et donc $i < 1$, ce qui est absurde. De même pour le dernier.
	\item Il existe un tri topologique si, et seulement si le graphe est acyclique.
		\begin{itemize}
			\item[``$\implies$'']
				Soit $L_1,\ldots,L_n$\/ un tri topologique. Montrons que le graphe est acyclique.
				Par l'absurde, on suppose le graphe non acyclique : il existe $(i,j) \in \llbracket 1,n \rrbracket^2$\/ avec $i \neq j$\/ tels que $T_i \to \cdots \to T_j$\/ et $T_j \to \cdots \to T_i$ soient deux chemins valides. Ainsi, comme le tri est topologique et par récurrence, $i \le j$\/ et $j \le i$\/ et donc $i = j$, ce qui est absurde car $i$\/ et $j$\/ sont supposés différents. Le graphe est donc acyclique.
			\item[``$\impliedby$'']
				Soit $G$\/ un graphe tel que tous les sommets possèdent une arrête entrante. On suppose par l'absurde ce graphe acyclique.
				Soit $x_0$\/ un sommet du graphe.
				On construit par récurrence $x_0,x_1,\ldots,x_n,x_{n+1},\ldots$ les successeurs successifs. Il y a un nombre fini de sommets donc deux sommets sont identiques. Donc, il y a nécessairement un cycle, ce qui est absurde.
				\begin{algorithm}[H]
					\centering
					\begin{algorithmic}[1]
						\Entree $G = (S, A)$\/ un graphe acyclique
						\Sortie $\mathrm{Res}$\/ un tri topologique.
						\State $\mathrm{Res} \gets [\quad]$\/
						\While{$G \neq \O$}
							\State Soit $x$\/ un sommet de $G$\/ sans prédécesseur
							\State $G \gets \big(S \setminus \{x\}, A \cap (S \setminus \{x\})^2\big)$\/ 
							\State $\mathrm{Res} \gets \mathrm{Res} \cdot [x]$\/
						\EndWhile
						\State\Return $\mathrm{Res}$\/
					\end{algorithmic}
					\caption{Génération d'un tri topologique d'un graphe acyclique}
				\end{algorithm}
		\end{itemize}
	\item~
		\begin{algorithm}[H]
			\centering
			\begin{algorithmic}[1]
				\Entree $G = (S, A)$\/ un graphe
				\Sortie $\mathrm{Res}$\/ un tri topologique, ou un cycle
				\State $\mathrm{Res} \gets [\quad]$\/
				\While{$G \neq \O$}
					\If{il existe $x$ sans prédécesseurs}
						\State Soit $x$\/ un sommet de $G$\/ sans prédécesseur
						\State $G \gets \big(S \setminus \{x\}, A \cap (S \setminus \{x\})^2\big)$\/ 
						\State $\mathrm{Res} \gets \mathrm{Res} \cdot [x]$\/
					\Else
						\State Soit $x \in S$\/ 
						\State Soit $x \gets x_1 \gets x_2 \gets \cdots \gets x_i$\/ la suite des prédécesseurs
						\State\Return $x_i,x_{i+1},\ldots,x_i$, un cycle
					\EndIf
				\EndWhile
				\State\Return $\mathrm{Res}$\/
			\end{algorithmic}
			\caption{Génération d'un tri topologique d'un graphe}
		\end{algorithm}
	\item On utilise la représentation par liste d'adjacence, et on stocke le nombre de prédécesseurs que l'on décroit à chaque choix de sommet.
	\item On essaie de trouver un tri topologique, et on voit si l'on trouve un cycle.
\end{enumerate}

		\section{Formules duales}

\begin{enumerate}
	\item On définit par induction $(\cdot)^\star$\/ comme
		\begin{multicols}{3}
			\begin{itemize}
				\item $\top^\star = \bot$\/ ;
				\item $\bot^\star = \top$\/ ;
				\item $(G \lor H)^\star = G^\star \land H^\star$\/ ;
				\item $(G \land H)^\star = G^\star \lor H^\star$\/ ;
				\item $(\lnot G)^\star = \lnot G^\star$\/ ;
				\item $p^\star = p$.
			\end{itemize}
		\end{multicols}
	\item Soit $\rho \in \mathds{B}^{\mathcal{P}}$. Montrons, par induction, $P(H) : ``\left\llbracket H^\star \right\rrbracket^\rho = \left\llbracket \lnot H \right\rrbracket^{\bar \rho}"$\/ où $\bar{\rho} : p \mapsto \overline{\rho(p)}$.
		\begin{itemize}
			\item On a $\left\llbracket \bot^\star \right\rrbracket^\rho = \left\llbracket \top \right\rrbracket^\rho = \mathbf{V}$, et $\left\llbracket \lnot \bot \right\rrbracket^{\bar\rho} = \left\llbracket \top \right\rrbracket^{\bar\rho} = \mathbf{V}$, d'où $P(\bot)$.
			\item On a $\left\llbracket \top^\star \right\rrbracket^\rho = \left\llbracket \bot \right\rrbracket^\rho = \mathbf{F}$, et $\left\llbracket \lnot \top \right\rrbracket^{\bar\rho} = \left\llbracket \bot \right\rrbracket^{\bar\rho} = \mathbf{F}$, d'où $P(\top)$.
			\item Soit $p \in \mathcal{P}$. On a $\left\llbracket p^\star  \right\rrbracket^\rho = \left\llbracket p \right\rrbracket^\rho = \rho(p)$, et $\left\llbracket \lnot p \right\rrbracket^{\bar\rho} = \overline{\left\llbracket p \right\rrbracket^{\bar\rho}} = \overline{\bar\rho(p)} = \overline{\overline{\rho(p)}} = \rho(p)$, d'où~$P(p)$.
		\end{itemize}
		Soient $F$\/ et $G$\/ deux formules.
		\begin{itemize}
			\item On a
				\begin{align*}
					\left\llbracket (F \land G)^\star  \right\rrbracket^\rho &= \left\llbracket F^\star  \lor G^\star \right\rrbracket^\rho\\
					&= \left\llbracket F^\star \right\rrbracket^\rho + \left\llbracket G^\star \right\rrbracket^\rho \\
					&= \left\llbracket \lnot F \right\rrbracket^{\bar\rho} + \left\llbracket \lnot G \right\rrbracket^{\bar\rho} \\
					&= \left\llbracket \lnot F \lor \lnot G \right\rrbracket^{\bar\rho} \\
					&= \left\llbracket \lnot (F \land G) \right\rrbracket^{\bar\rho} \\
				\end{align*}
				d'où $P(F \land G)$.
			\item On a
				\begin{align*}
					\left\llbracket (F \lor G)^\star  \right\rrbracket^\rho &= \left\llbracket F^\star  \land G^\star \right\rrbracket^\rho\\
					&= \left\llbracket F^\star \right\rrbracket^\rho \cdot \left\llbracket G^\star \right\rrbracket^\rho \\
					&= \left\llbracket \lnot F \right\rrbracket^{\bar\rho} \cdot \left\llbracket \lnot G \right\rrbracket^{\bar\rho} \\
					&= \left\llbracket \lnot F \land \lnot G \right\rrbracket^{\bar\rho} \\
					&= \left\llbracket \lnot (F \lor G) \right\rrbracket^{\bar\rho} \\
				\end{align*}
				d'où $P(F \lor G)$.
			\item On a \[
					\left\llbracket (\lnot F)^\star  \right\rrbracket^\rho = \left\llbracket \lnot (F^\star) \right\rrbracket^\rho = \overline{\left\llbracket F^\star \right\rrbracket^\rho} = \overline{\left\llbracket \lnot F \right\rrbracket^{\bar\rho}} = \left\llbracket \lnot (\lnot F) \right\rrbracket^{\bar\rho}.
					\] d'où $P(\lnot F)$.
		\end{itemize}
		Par induction, on en conclut que $P(F)$\/ est vraie pour toute formule $F$.
	\item Soit $G$\/ une formule valide. Alors, par définition, $G \equiv \top$. Or, d'après la question précédente, $G^\star \equiv (\top)^\star = \bot$. Ainsi, $G^\star $\/ n'est pas satisfiable.
\end{enumerate}



		\section{Un lemme d'itération}

		\section{Ambigüité}

		\section{Regexp Crossword}
\begin{center}
	\url{https://regexcrossword.com/}
\end{center}

		\documentclass[a4paper]{article}

\usepackage[margin=1in]{geometry}
\usepackage[utf8]{inputenc}
\usepackage[T1]{fontenc}
\usepackage{mathrsfs}
\usepackage{textcomp}
\usepackage[french]{babel}
\usepackage{amsmath}
\usepackage{amssymb}
\usepackage{cancel}
\usepackage{frcursive}
\usepackage[inline]{asymptote}
\usepackage{tikz}
\usepackage[european,straightvoltages,europeanresistors]{circuitikz}
\usepackage{tikz-cd}
\usepackage{tkz-tab}
\usepackage[b]{esvect}
\usepackage[framemethod=TikZ]{mdframed}
\usepackage{centernot}
\usepackage{diagbox}
\usepackage{dsfont}
\usepackage{fancyhdr}
\usepackage{float}
\usepackage{graphicx}
\usepackage{listings}
\usepackage{multicol}
\usepackage{nicematrix}
\usepackage{pdflscape}
\usepackage{stmaryrd}
\usepackage{xfrac}
\usepackage{hep-math-font}
\usepackage{amsthm}
\usepackage{thmtools}
\usepackage{indentfirst}
\usepackage[framemethod=TikZ]{mdframed}
\usepackage{accents}
\usepackage{soulutf8}
\usepackage{mathtools}
\usepackage{bodegraph}
\usepackage{slashbox}
\usepackage{enumitem}
\usepackage{calligra}
\usepackage{cinzel}
\usepackage{BOONDOX-calo}

% Tikz
\usetikzlibrary{babel}
\usetikzlibrary{positioning}
\usetikzlibrary{calc}

% global settings
\frenchspacing
\reversemarginpar
\setuldepth{a}

%\everymath{\displaystyle}

\frenchbsetup{StandardLists=true}

\def\asydir{asy}

%\sisetup{exponent-product=\cdot,output-decimal-marker={,},separate-uncertainty,range-phrase=\;à\;,locale=FR}

\setlength{\parskip}{1em}

\theoremstyle{definition}

% Changing math
\let\emptyset\varnothing
\let\ge\geqslant
\let\le\leqslant
\let\preceq\preccurlyeq
\let\succeq\succcurlyeq
\let\ds\displaystyle
\let\ts\textstyle

\newcommand{\C}{\mathds{C}}
\newcommand{\R}{\mathds{R}}
\newcommand{\Z}{\mathds{Z}}
\newcommand{\N}{\mathds{N}}
\newcommand{\Q}{\mathds{Q}}

\renewcommand{\O}{\emptyset}

\newcommand\ubar[1]{\underaccent{\bar}{#1}}

\renewcommand\Re{\expandafter\mathfrak{Re}}
\renewcommand\Im{\expandafter\mathfrak{Im}}

\let\slantedpartial\partial
\DeclareRobustCommand{\partial}{\text{\rotatebox[origin=t]{20}{\scalebox{0.95}[1]{$\slantedpartial$}}}\hspace{-1pt}}

% merging two maths characters w/ \charfusion
\makeatletter
\def\moverlay{\mathpalette\mov@rlay}
\def\mov@rlay#1#2{\leavevmode\vtop{%
   \baselineskip\z@skip \lineskiplimit-\maxdimen
   \ialign{\hfil$\m@th#1##$\hfil\cr#2\crcr}}}
\newcommand{\charfusion}[3][\mathord]{
    #1{\ifx#1\mathop\vphantom{#2}\fi
        \mathpalette\mov@rlay{#2\cr#3}
      }
    \ifx#1\mathop\expandafter\displaylimits\fi}
\makeatother

% custom math commands
\newcommand{\T}{{\!\!\,\top}}
\newcommand{\avrt}[1]{\rotatebox{-90}{$#1$}}
\newcommand{\bigcupdot}{\charfusion[\mathop]{\bigcup}{\cdot}}
\newcommand{\cupdot}{\charfusion[\mathbin]{\cup}{\cdot}}
%\newcommand{\danger}{{\large\fontencoding{U}\fontfamily{futs}\selectfont\char 66\relax}\;}
\newcommand{\tendsto}[1]{\xrightarrow[#1]{}}
\newcommand{\vrt}[1]{\rotatebox{90}{$#1$}}
\newcommand{\tsup}[1]{\textsuperscript{\underline{#1}}}
\newcommand{\tsub}[1]{\textsubscript{#1}}

\renewcommand{\mod}[1]{~\left[ #1 \right]}
\renewcommand{\t}{{}^t\!}
\newcommand{\s}{\text{\calligra s}}

% custom units / constants
%\DeclareSIUnit{\litre}{\ell}
\let\hbar\hslash

% header / footer
\pagestyle{fancy}
\fancyhead{} \fancyfoot{}
\fancyfoot[C]{\thepage}

% fonts
\let\sc\scshape
\let\bf\bfseries
\let\it\itshape
\let\sl\slshape

% custom math operators
\let\th\relax
\let\det\relax
\DeclareMathOperator*{\codim}{codim}
\DeclareMathOperator*{\dom}{dom}
\DeclareMathOperator*{\gO}{O}
\DeclareMathOperator*{\po}{\text{\cursive o}}
\DeclareMathOperator*{\sgn}{sgn}
\DeclareMathOperator*{\simi}{\sim}
\DeclareMathOperator{\Arccos}{Arccos}
\DeclareMathOperator{\Arcsin}{Arcsin}
\DeclareMathOperator{\Arctan}{Arctan}
\DeclareMathOperator{\Argsh}{Argsh}
\DeclareMathOperator{\Arg}{Arg}
\DeclareMathOperator{\Aut}{Aut}
\DeclareMathOperator{\Card}{Card}
\DeclareMathOperator{\Cl}{\mathcal{C}\!\ell}
\DeclareMathOperator{\Cov}{Cov}
\DeclareMathOperator{\Ker}{Ker}
\DeclareMathOperator{\Mat}{Mat}
\DeclareMathOperator{\PGCD}{PGCD}
\DeclareMathOperator{\PPCM}{PPCM}
\DeclareMathOperator{\Supp}{Supp}
\DeclareMathOperator{\Vect}{Vect}
\DeclareMathOperator{\argmax}{argmax}
\DeclareMathOperator{\argmin}{argmin}
\DeclareMathOperator{\ch}{ch}
\DeclareMathOperator{\com}{com}
\DeclareMathOperator{\cotan}{cotan}
\DeclareMathOperator{\det}{det}
\DeclareMathOperator{\id}{id}
\DeclareMathOperator{\rg}{rg}
\DeclareMathOperator{\rk}{rk}
\DeclareMathOperator{\sh}{sh}
\DeclareMathOperator{\th}{th}
\DeclareMathOperator{\tr}{tr}

% colors and page style
\definecolor{truewhite}{HTML}{ffffff}
\definecolor{white}{HTML}{faf4ed}
\definecolor{trueblack}{HTML}{000000}
\definecolor{black}{HTML}{575279}
\definecolor{mauve}{HTML}{907aa9}
\definecolor{blue}{HTML}{286983}
\definecolor{red}{HTML}{d7827e}
\definecolor{yellow}{HTML}{ea9d34}
\definecolor{gray}{HTML}{9893a5}
\definecolor{grey}{HTML}{9893a5}
\definecolor{green}{HTML}{a0d971}

\pagecolor{white}
\color{black}

\begin{asydef}
	settings.prc = false;
	settings.render=0;

	white = rgb("faf4ed");
	black = rgb("575279");
	blue = rgb("286983");
	red = rgb("d7827e");
	yellow = rgb("f6c177");
	orange = rgb("ea9d34");
	gray = rgb("9893a5");
	grey = rgb("9893a5");
	deepcyan = rgb("56949f");
	pink = rgb("b4637a");
	magenta = rgb("eb6f92");
	green = rgb("a0d971");
	purple = rgb("907aa9");

	defaultpen(black + fontsize(8pt));

	import three;
	currentlight = nolight;
\end{asydef}

% theorems, proofs, ...

\mdfsetup{skipabove=1em,skipbelow=1em, innertopmargin=6pt, innerbottommargin=6pt,}

\declaretheoremstyle[
	headfont=\normalfont\itshape,
	numbered=no,
	postheadspace=\newline,
	headpunct={:},
	qed=\qedsymbol]{demstyle}

\declaretheorem[style=demstyle, name=Démonstration]{dem}

\newcommand\veczero{\kern-1.2pt\vec{\kern1.2pt 0}} % \vec{0} looks weird since the `0' isn't italicized

\makeatletter
\renewcommand{\title}[2]{
	\AtBeginDocument{
		\begin{titlepage}
			\begin{center}
				\vspace{10cm}
				{\Large \sc Chapitre #1}\\
				\vspace{1cm}
				{\Huge \calligra #2}\\
				\vfill
				Hugo {\sc Salou} MPI${}^{\star}$\\
				{\small Dernière mise à jour le \@date }
			\end{center}
		\end{titlepage}
	}
}

\newcommand{\titletp}[4]{
	\AtBeginDocument{
		\begin{titlepage}
			\begin{center}
				\vspace{10cm}
				{\Large \sc tp #1}\\
				\vspace{1cm}
				{\Huge \textsc{\textit{#2}}}\\
				\vfill
				{#3}\textit{MPI}${}^{\star}$\\
			\end{center}
		\end{titlepage}
	}
	\fancyfoot{}\fancyhead{}
	\fancyfoot[R]{#4 \textit{MPI}${}^{\star}$}
	\fancyhead[C]{{\sc tp #1} : #2}
	\fancyhead[R]{\thepage}
}

\newcommand{\titletd}[2]{
	\AtBeginDocument{
		\begin{titlepage}
			\begin{center}
				\vspace{10cm}
				{\Large \sc td #1}\\
				\vspace{1cm}
				{\Huge \calligra #2}\\
				\vfill
				Hugo {\sc Salou} MPI${}^{\star}$\\
				{\small Dernière mise à jour le \@date }
			\end{center}
		\end{titlepage}
	}
}
\makeatother

\newcommand{\sign}{
	\null
	\vfill
	\begin{center}
		{
			\fontfamily{ccr}\selectfont
			\textit{\textbf{\.{\"i}}}
		}
	\end{center}
	\vfill
	\null
}

\renewcommand{\thefootnote}{\emph{\alph{footnote}}}

% figure support
\usepackage{import}
\usepackage{xifthen}
\pdfminorversion=7
\usepackage{pdfpages}
\usepackage{transparent}
\newcommand{\incfig}[1]{%
	\def\svgwidth{\columnwidth}
	\import{./figures/}{#1.pdf_tex}
}

\pdfsuppresswarningpagegroup=1
\ctikzset{tripoles/european not symbol=circle}

\newcommand{\missingpart}{{\large\color{red} Il manque quelque chose ici\ldots}}


\fancyhead[R]{Hugo {\sc Salou}\/ MPI}
\fancyhead[L]{TD\textsubscript2 -- Exercice 7}

\begin{document}
	\let\thesection\relax
	\section{Exercice 3}

{\bf Indication}\/ : pour la $G$, on applique la relation de {\sc Chasles}\/ : l'intégrale $\int_0^7 \mathrm{e}^{-x}\ln x~\mathrm{d}x$\/ converge si et seulement si $\int_0^1 \mathrm{e}^{-x}\ln x\mathrm{d}x$\/ converge et $\int_1^7\mathrm{e}^{-x}\ln x~\mathrm{d}x$\/ converge (qui n'est même pas impropre).

L'intégrale $H = \int_0^1 \frac{\mathrm{e}^{\sin t}}{t}~\mathrm{d}t$\/ est impropre en 0. On sait que $\sin t \tendsto{t\to 0} 0$, et donc, par continuité de la fonction $\exp$\/ en $0$, $\mathrm{e}^{\sin t}\tendsto{t\to 0}e^0 = 1$.
Ainsi, $\frac{\mathrm{e}^{\sin t}}{t} = \mathrm{e}^{\sin t} \times \frac{1}{t} \simi_{t\to 0} \frac{1}{t}$\/ qui ne change pas de signe. Or, $\int_0^1 \frac{1}{t}~\mathrm{d}t$\/ diverge, donc l'intégrale $H$\/ diverge.

L'intégrale $I = \int_{1}^{+\infty} \frac{\mathrm{e}^{\sin t}}{t}~\mathrm{d}t$\/ est impropre en $+\infty$. Par croissance de la fonction exponentielle, on a $\frac{\mathrm{e}^{\sin t}}{t} \ge \frac{\mathrm{e}^{-1}}{t} \ge 0$. Or, l'intégrale $\int_{1}^{+\infty} \frac{1}{t}~\mathrm{d}t$\/ diverge, donc l'intégrale diverge aussi.

L'intégrale $K$, est l'intégrale d'une fonction Gau\ss ienne, et elle est impropre en $+\infty$. On la \guillemotleft~découpe~\guillemotright\ : \[
	\int_{0}^{+\infty} \mathrm{e}^{-x^2}~\mathrm{d}x \text{ converge si et seulement si } \int_{0}^{1} \mathrm{e}^{-x^2}~\mathrm{d}x \text{ converge et } \int_{1}^{+\infty} \mathrm{e}^{-x^2}~\mathrm{d}x \text{ converge.}
\] L'intégrale $\int_{0}^{1} \mathrm{e}^{-x^2}~\mathrm{d}x$\/ n'est même pas impropre, elle converge donc. Et, pour $x \in [1,+\infty[$, on sait, comme $x^2 \ge x$, $0 \le \mathrm{e}^{-x^2} \le \mathrm{e}^{-x}$. Or, $\int_{1}^{+\infty} \mathrm{e}^{-x}~\mathrm{d}x$\/ converge donc $\int_{0}^{+\infty} \mathrm{e}^{-x^2}~\mathrm{d}x$\/ aussi.
On calculera la valeur de cette intégrale dans le {\sc td}\/ \guillemotleft~Intégrales paramétrées.~\guillemotright

Autre méthode pour déterminer la nature de $K$\/ : 
$\mathrm{e}^{-x^2} = \po(\mathrm{e}^{-x})$\/ car $\mathrm{e}^{-x^2} = \underbrace{\mathrm{e}^{-x^2 + x}}_{\to 0} \times \mathrm{e}^{-x}$, car $\mathrm{e}^{-x^2 + x} = \mathrm{e}^{-x^2 \left( 1 - \frac{1}{x} \right)}$\/ et $-x^2\left( 1 - \frac{1}{x} \right) \to -\infty \times 1$.
Et $\int_0^{+\infty} \mathrm{e}^{-x}~\mathrm{d}x$\/ converge donc $\int_0^{+\infty} \mathrm{e}^{-x^2}~\mathrm{d}x$\/ converge.

\begin{figure}[H]
	\centering
	\begin{asy}
		import graph;
		size(10cm);
		draw((-10, 0) -- (10, 0), Arrow(TeXHead));
		draw((0, -3) -- (0, 5), Arrow(TeXHead));
		real f(real x) {
			return 4*exp(-(x/4)^2);
		}
		draw(graph(f, -10, 10), magenta);
	\end{asy}
	\caption{Courbe Gau\ss ienne}
\end{figure}

L'intégrale $F = \int_{7}^{+\infty} \mathrm{e}^{-x}\ln x~\mathrm{d}x$\/ est impropre en $+\infty$. Attention : la fonction n'est pas \guillemotleft~{\color{red} faussement impropre en $+\infty$}.~\guillemotright\ Mais, on peut remarquer que \[
	\mathrm{e}^{-x} \ln x = \mathrm{e}^{-\frac{x}{2}} \underbrace{\mathrm{e}^{-\frac{x}{2}} \ln x}_{\tendsto{x\to +\infty} 0} = \po(\mathrm{e}^{-\frac{x}{2}})
.\] Or, $\int_{7}^{+\infty} \mathrm{e}^{-x}~\mathrm{d}x$\/ converge donc l'intégrale $F$\/ converge aussi.

L'intégrale $G = \int_{0}^{7} \mathrm{e}^{-x}\ln x~\mathrm{d}x$\/ est impropre en 0.
Or, $\mathrm{e}^{-x}\ln x \simi_{x\to 0} \ln x$\/ qui ne change pas de signe au voisinage de 0. Or, $\int_{0}^{7}  \ln x~\mathrm{d}x$\/ converge donc l'intégrale $G$\/ converge également.

L'intégrale $E = \int_{1}^{+\infty} \frac{\ln x}{\sqrt{x}}~\mathrm{d}x$\/ est impropre en $+\infty$. Or, $\forall x \ge \mathrm{e}$, $\frac{\ln(x)}{\sqrt{x}} \ge \frac{1}{\sqrt{x}} \ge 0$\/ converge.
Or, $\int_{1}^{+\infty}  \frac{1}{x^{\sfrac{1}{2}}}~\mathrm{d}x$\/ diverge d'après le critère de {\sc Riemann}\/ en $+\infty$\/ car $\frac{1}{2} < 1$.
D'où l'intégrale $D$\/ diverge.

Autre méthode : intégration par parties. On peut même arriver à calculer une primitive de ${\ln x}\:/{\sqrt{x}}$.

L'intégrale $D = \int_{0}^{1} \frac{\ln x}{\sqrt{x}}~\mathrm{d}x$\/ est impropre en 0. On peut remarque que \[
	0 \le -\frac{\ln x}{\sqrt{x}} = -\frac{x^{0{,}1} \ln x}{x^{0{,}6}} = \po\left( \frac{1}{x^{0{,}6}} \right) \quad\text{car}\quad x^{0{,}1} \ln x \tendsto{x\to 0} 0
\] par croissances comparées.
Or, $\int_{0}^{1} \frac{1}{x^{0{,}6}}~\mathrm{d}x$\/ converge d'après le critère de {\sc Riemann}. D'où $-D$\/ converge et donc $D$\/ converge.

L'intégrale $J = \int_{1}^{+\infty} \frac{\sin t}{\sqrt{t} + \sin t}~\mathrm{d}t$\/ est impropre en $+\infty$. On calcule
\[
	f(t) = \frac{\sin t}{\sqrt{t} + \sin t} = \frac{\sin t}{\sqrt{t}} \times \frac{1}{1+\frac{\sin t}{\sqrt{t}}}
\] et $\frac{\sin t}{\sqrt{t}} \tendsto{t\to +\infty} 0$. D'où \[
	\frac{1}{1+\frac{\sin t}{\sqrt{t}}} = 1  - \frac{\sin t}{\sqrt{t}} + \frac{\sin^2 t}{t} + \po\left( \frac{\sin^2 t}{t} \right)
\] et donc \[
	f(t) = \frac{\cos t}{\sqrt{t}} + \frac{\sin^2 t}{t} + \po\left( \frac{\sin^2 t}{t} \right)
.\]
L'intégrale $\int_{1}^{+\infty} \frac{\sin t}{\sqrt{t}}~\mathrm{d}t$\/ est impropre en $+\infty$.
Soit $x \ge 1$. On calcule avec une intégration par parties,
\[
	\int_{1}^{x} \sin t \times \frac{1}{\sqrt{t}}~\mathrm{d}t = \int_{1}^{x} u'(t)\cdot v(t)~\mathrm{d}t
\] où $u(t) = - \cos t$\/ et $v(t) = \frac{1}{\sqrt{t}} = t^{-\frac{1}{2}}$. Donc
\begin{align*}
	\int_{1}^{x} \frac{\sin t}{\sqrt{t}}~\mathrm{d}t &= \Big[f(t)g(t)\Big]_1^x - \int_{1}^{x} f(t)\cdot g'(t)~\mathrm{d}t\\
	&= \left[ - \frac{\cos t}{\sqrt{t}} \right]_1^x - \int_{1}^{x} (-\cos t)\left( -\frac{1}{2}t^{-\frac{3}{2}} \right)~\mathrm{d}t \\
\end{align*}
D'où \[
	\int_{1}^{x} \frac{\sin t}{\sqrt{t}}~\mathrm{d}t = \cos 1 - \frac{\cos x}{\sqrt{x}} - \frac{1}{2} \int_{1}^{x} \frac{\cos t}{t^{\sfrac{3}{2}}}~\mathrm{d}t
.\]
Or, d'une part $\cos x \times \frac{1}{\sqrt{x}} \tendsto{x\to +\infty} 0$\/ car $\cos$\/ est bornée et $\frac{1}{\sqrt{x}}\tendsto{x\to +\infty} 0$.
Et, d'autre part $\int_{1}^{+\infty} \frac{\cos t}{t^{\sfrac{3}{2}}}~\mathrm{d}t$\/ converge car $\forall t \in [1,+\infty[$, $\left| \frac{\cos t}{t^{\sfrac{3}{2}}} \right| \le \frac{1}{t^{\sfrac{3}{2}}}$\/ et $\int_{1}^{+\infty} \frac{1}{t^{\sfrac{3}{2}}}~\mathrm{d}t$\/ converge.
Pour le 2\tsup{nd} terme du développement limité, on fait une {\sc ipp}, on trouve un terme en $\frac{1}{t^2}$\/ et donc son intégrale converge par critère de {\sc Riemann}. S'il y a des problèmes, voir en {\sc td}.
On étudie maintenant le 3\tsup{ème} terme : \[
	\int_{1}^{+\infty}  \po\left( \frac{\sin t}{t} \right) ~\mathrm{d}t \text{ converge car } \int_{1}^{+\infty} \frac{\sin^2 t}{t} ~\mathrm{d}t \text{ converge et } t\mapsto \frac{\sin^2 t}{t} \text{ est positive}.
\]
Autre méthode : on a \[
	\frac{\sin^2 t}{t} + \po\left( \frac{\sin^2 t}{t} \right) \simi_{t\to +\infty} \frac{\sin^2 t}{t} \text{ qui ne change pas de signe}
.\] Or, $\int_{1}^{+\infty} \frac{\sin^2 t}{t}~\mathrm{d}t$\/ converge et donc \[
	\int_{1}^{+\infty} \left( \frac{\sin^2 t}{t} + \po\left( \frac{\sin^2 t}{t} \right) \right) ~\mathrm{d}t
.\]

\end{document}

		\section{Vocabulaire des automates}

On représente, ci-dessous, l'automate $\mathcal{A}$\/ décrit dans l'énoncé.
\begin{figure}[H]
	\centering
	\tikzfig{automate-ex8}
	\caption{Automate décrit dans l'énoncé de l'exercice 8}
\end{figure}

\begin{enumerate}
	\item Cet automate n'est pas complet : à l'état 0, la lecture d'un $a$\/ peut conduire à l'état 0 ou bien à l'état 1.
	\item Le mot $baba$\/ est reconnu par $\mathcal{A}$\/ mais pas le mot $cabcb$.
	\item L'automate reconnaît les mots dont la 3\tsup{ème} lettre du mot, en partant de la fin, est un $a$.
\end{enumerate}



		\section{Complétion d'automate}

\begin{enumerate}
	\item Non, cet automate n'est pas complet. Par exemple, la lecture d'un $b$\/ à l'état 1 est impossible.
	\item Cet automate reconnaît le langage $L = \mathcal{L}\big(a \cdot b\cdot (a \mid b)^*\big)$.
	\item~

		\begin{figure}[H]
			\centering
			\tikzfig{automate-ex9}
			\caption{Automate complet équivalent à $\mathcal{A}$}
		\end{figure}
\end{enumerate}



		\section{Exercice supplémentaire 1}

\slshape
\begin{enumerate}
	\item Montrer que l'ensemble des langages reconnaissables est stable par complémentaire.
	\item Montrer que l'ensemble des langages reconnaissables est stable par intersection.
\end{enumerate}
\upshape

\begin{enumerate}
	\item Soient $\mathcal{A} = (\Sigma, \mathcal{Q}, I, F, \delta)$\/ et $\mathcal{A}' = (\Sigma, \mathcal{Q}', I', F', \delta')$\/ deux automates déterministes complets, tels que $\mathcal{L}(\mathcal{A}) = \mathcal{L}(\mathcal{A}')$. Alors \[
		\mathcal{L}\big(\Sigma, \mathcal{Q}', I', \mathcal{Q}'\setminus F', \delta')\big) = \Sigma^* \setminus \mathcal{L}(\mathcal{A})
	.\]
	\item On utilise les lois de {\sc De Morgan}\/ en passant au complémentaire les deux automates, puis l'union (que l'on a vu en cours), et on repasse au complémentaire.
\end{enumerate}





	}
	\def\addmacros#1{#1}
}
{
	\td[4]{Langages et expressions régulières (2)}
	\minitoc
	\renewcommand{\cwd}{../td/td04/}
	\addmacros{
		\section{Déterminisation de taille exponentielle}

\begin{enumerate}
	\item En notant $n$\/ le nombre d'états de $\mathcal{A}$, alors le nombre d'états de $\det(\mathcal{A})$\/ est, au plus, $2^n$. En effet, les états sont des éléments de $\wp(Q)$\/ et $|\wp(Q)| = 2^n$.
	\item $\mathcal{A}$ : \tikzfig{ex1-q2a} $\mathcal{B}$ : \tikzfig{ex1-q2b}\\
		mais $\det(\mathcal{A})$\/ : \tikzfig{ex1-q2c}
	\item $\mathcal{A}_n$\/ : \tikzfig{ex1-q3}
	\item $\mathcal{A}_3$\/ : \tikzfig{ex1-q4}\\
		et, $\det(\mathcal{A}_3)$\/ : \tikzfig{ex1-q4b}
	\item Soit $i_0 = \max \{k \in \left\llbracket 1,n \right\rrbracket  \mid u_k \neq v_k\}$. Soit $m \in \Sigma^{i_0}$\/ tel que $u \cdot m \in L_n$\/ mais $v\cdot m \not\in L_n$. Or, $\delta^*(i, u\cdot m) = \delta^*(\delta^*(i,u), m)$\/ et $\delta^*(i,v\cdot m) = \delta^*(\delta^*(i,v),m)$. D'où $\delta^*(i,u\cdot m) \in F$\/ et $\delta^*(i,v\cdot m) \not\in F$. Ce qui est absurde.
	\item Ainsi, l'application \begin{align*}
			f: \Sigma^* &\longrightarrow Q \\
			u &\longmapsto \delta^*(i,u)
		\end{align*} est injective. D'où, $\mathcal{D}_n = |Q| \ge |\Sigma^*| = 2^n$.
	\item D'où, d'après les questions 1 et 6, on en déduit que le nombre d'états utilisés pour la déterminisation de $\mathcal{A}_n$\/ est de $\mathcal{D}_n \ge 2^n$.
\end{enumerate}

		\section{Suppression des $\varepsilon$-transitions}

\begin{figure}[H]
	\centering
	\tikzfig{ex2}
\end{figure}

		\section{Exercice 3 : Échantillonnage}

\paragraph{Q.\ 1}~

\begin{algorithm}[H]
	\centering
	\begin{algorithmic}[1]
		\Entree $T$\/ un tableau à $n$\/ éléments, et $k \in \N$\/ avec $k \le n$\/
		\State $T \gets \text{Mélanger}(T)$\/
		\State $R \gets T[0..k]$
		\State\Return $R$\/
	\end{algorithmic}
	\caption{Échantillonnage naïf}
\end{algorithm}

\paragraph{Q.\ 2}
Un invariant de boucle est \guillemotleft~$\forall p \in \left\llbracket 0,I-1 \right\rrbracket,\:P(T[p] \in \textsf{Res}) = \frac{k}{I}$\/ et $\forall p \in \left\llbracket I,n \right\rrbracket,\:T[p] \not\in \textsf{Res}$~\guillemotright

\paragraph{Q.\ 3}
Notons $\ubar{I}$\/ et $\underline{\textsf{Res}}$\/ l'état des variables avant un tour de boucle ; et, $\bar{I}$\/ et $\overline{\textsf{Res}}$\/ l'état des variables après un tour de boucle.
\begin{itemize}
	\item Pour $k = I$, on a
		\begin{enumerate}
			\item $\forall p \in \left\llbracket 0,k-1 \right\rrbracket$, $P(T[p] \in \textsf{Res}) = 1$,
			\item $\forall p \in \left\llbracket k, n-1 \right\rrbracket$, $T[p] \not\in \textsf{Res}$,
			\item $I \le n$.
		\end{enumerate}
	\item Supposons $\ubar{I}$, et $\underline{\textsf{Res}}$\/ vérifiant l'invariant et la condition de boucle. Alors, on a
		\begin{enumerate}
			\item $\forall p \in \left\llbracket 0, \ubar{I} - 1 \right\rrbracket$, $P(T[p] \in \underline{\textsf{Res}}) = \frac{k}{\ubar{I}}$,
			\item $\forall p \in \left\llbracket \ubar{I}, n-1 \right\rrbracket$, $T[p] \not\in \underline{\textsf{Res}}$,
			\item $\ubar{I} < n$, la condition de boucle.
		\end{enumerate}
		Soit $j \in \left\llbracket 0, \ubar{I} \right\rrbracket$.
		On a $\bar{I} = \ubar{I} + 1$.
		\begin{itemize}
			\item[\sc Cas 1] $j < k$, et donc $\overline{\textsf{Res}}(j) = T[\ubar{I}]$, et $\forall \ell \neq j$, $\overline{\textsf{Res}}[\ell] = \underline{\textsf{Res}}[\ell]$.
			\item[\sc Cas 2] $j \ge  k$, et donc $\forall \ell,\:\overline{\textsf{Res}}[\ell] = \underline{\textsf{Res}}[\ell]$.
		\end{itemize}

		\begin{enumerate}
			\item Soit $p \in \left\llbracket 0, \ubar{I} \right\rrbracket $. Montrons $P(T[p] \not\in \overline{\textsf{Res}}) = \frac{k}{\bar{I}}$. Si $p < \ubar{I}$, alors
				\begin{align*}
					P(T[p] \in \overline{\textsf{Res}}) &= P\big(T[p] \in \underline{\textsf{Res}} \cap j \neq p\big) \\
					&= \frac{k}{\ubar{I}} \times \frac{\ubar{I}}{\ubar{I} + 1} \\
					&= \frac{k}{\bar{I}}.
				\end{align*}
				Si $P = \ubar{I}$, alors d'après 2.\ $T[p] \not\in \underline{\textsf{Res}}$, donc $P(T[p] \in \overline{\textsf{Res}}) = P(j < k) = \frac{k}{\bar{I} + 1}$.
		\end{enumerate}
\end{itemize}





		\section{Exercice 4}

\paragraph{Q.\ 1}

\begin{algo}
	\textsl{Entrée} : Un automate $\mathcal{A}$\/ ;\\
	\textsl{Sortie} : $\mathcal{L}(\mathcal{A}) = \O$\/ ;\\
	On fait un parcours en largeur depuis les états initiaux et on regarde si on atteint un état final.
\end{algo}

\begin{algo}[Nathan F.]
	{\sl Entrée} : Deux automates $\mathcal{A}$\/ et $\mathcal{B}$\/ \\
	{\sl Sortie} : $\mathcal{L}(\mathcal{A}) = \mathcal{L}(\mathcal{B})$\/ ;
	Soit $\mathcal{C}$\/ l'automate reconnaissant $\mathcal{L}(\mathcal{A}) \mathbin\triangle \mathcal{L}(\mathcal{B})$. On retourne $\mathcal{L}(\mathcal{C}) \mathrel{\ds\mathop=^?} \O$\/ à l'aide de l'algorithme précédent.
\end{algo}

Autre possibilité, on procède par double inclusion :

\begin{algo}[$\subseteq$]
	{\sl Entrée} : Deux automates $\mathcal{A}$\/ et $\mathcal{B}$\/ \\
	{\sl Sortie} : $\mathcal{L}(\mathcal{A}) \subseteq \mathcal{L}(\mathcal{B})$\/ ;
	On retourne $\mathcal{A} \setminus \mathcal{B} \mathrel{\ds\mathop=^?} \O$.
\end{algo}

\paragraph{Q.\ 2}
L'algorithme reconnaissant $\mathcal{L}(\mathcal{A}) \mathrel\triangle \mathcal{L}(\mathcal{B})$\/ doit être déterminisé, sa complexité est donc au moins de $2^n$.

		\section{Automates pour le calcul de l'addition en binaire}
\subsection{Nombres de même tailles}

\paragraph{Q.\ 1}~
\begin{figure}[H]
	\centering
	\tikzfig{ex5-q1}
\end{figure}

\paragraph{Q.\ 2}

Pour $r \in \{0,1\}$, il existe une exécution dans $\mathcal{A}$\/ étiquetée par \[
	(u_0,v_0, w_0)(u_1,v_1,w_1)\ldots(u_{n-1},v_{n-1},w_{n-1})
\] menant à $r$\/ si et seulement si \[
	\overline{u_0\ldots u_{n-1}}^2 + \overline{v_0\ldots v_{n-1}}^2 = \overline{w_0\ldots w_{n-1}}^2 + r\:2^{n},
\] ce qui est équivalent à si et seulement si \[
	\overline{u_0\ldots u_{n-1}0}^2 + \overline{v_0\ldots v_{n-1}0}^2 = \overline{w_0\ldots w_{n-1}r}^2
.\]

\paragraph{Q.\ 3}
Prouvons-le par récurrence.
\begin{itemize}
	\item Pour $n = 0$, il existe une exécution dans $\mathcal{A}$\/ étiquetée par $\varepsilon$\/ menant à $r=0$ si et seulement si $\overline{\varepsilon}^2 + \overline{\varepsilon}^2 = 0 = \overline{\varepsilon}^2 + 0 \times 2^0$. De même, il existe une exécution dans $\mathcal{A}$\/ étiquetée par $\varepsilon$\/ menant à $r=1$ si et seulement si $\overline{\varepsilon}^2 + \overline{\varepsilon}^2 = 0 = 1 = \overline{\varepsilon}^2 + 1 \times 2^0$.
\end{itemize}

	}
	\def\addmacros#1{#1}
}
{
	\td[5]{Langages et expressions régulières (3)}
	\minitoc
	\renewcommand{\cwd}{../td/td05/}
	\addmacros{
		\section{Déterminisation de taille exponentielle}

\begin{enumerate}
	\item En notant $n$\/ le nombre d'états de $\mathcal{A}$, alors le nombre d'états de $\det(\mathcal{A})$\/ est, au plus, $2^n$. En effet, les états sont des éléments de $\wp(Q)$\/ et $|\wp(Q)| = 2^n$.
	\item $\mathcal{A}$ : \tikzfig{ex1-q2a} $\mathcal{B}$ : \tikzfig{ex1-q2b}\\
		mais $\det(\mathcal{A})$\/ : \tikzfig{ex1-q2c}
	\item $\mathcal{A}_n$\/ : \tikzfig{ex1-q3}
	\item $\mathcal{A}_3$\/ : \tikzfig{ex1-q4}\\
		et, $\det(\mathcal{A}_3)$\/ : \tikzfig{ex1-q4b}
	\item Soit $i_0 = \max \{k \in \left\llbracket 1,n \right\rrbracket  \mid u_k \neq v_k\}$. Soit $m \in \Sigma^{i_0}$\/ tel que $u \cdot m \in L_n$\/ mais $v\cdot m \not\in L_n$. Or, $\delta^*(i, u\cdot m) = \delta^*(\delta^*(i,u), m)$\/ et $\delta^*(i,v\cdot m) = \delta^*(\delta^*(i,v),m)$. D'où $\delta^*(i,u\cdot m) \in F$\/ et $\delta^*(i,v\cdot m) \not\in F$. Ce qui est absurde.
	\item Ainsi, l'application \begin{align*}
			f: \Sigma^* &\longrightarrow Q \\
			u &\longmapsto \delta^*(i,u)
		\end{align*} est injective. D'où, $\mathcal{D}_n = |Q| \ge |\Sigma^*| = 2^n$.
	\item D'où, d'après les questions 1 et 6, on en déduit que le nombre d'états utilisés pour la déterminisation de $\mathcal{A}_n$\/ est de $\mathcal{D}_n \ge 2^n$.
\end{enumerate}

		\section{Exercice 4}

\paragraph{Q.\ 1}

\begin{algo}
	\textsl{Entrée} : Un automate $\mathcal{A}$\/ ;\\
	\textsl{Sortie} : $\mathcal{L}(\mathcal{A}) = \O$\/ ;\\
	On fait un parcours en largeur depuis les états initiaux et on regarde si on atteint un état final.
\end{algo}

\begin{algo}[Nathan F.]
	{\sl Entrée} : Deux automates $\mathcal{A}$\/ et $\mathcal{B}$\/ \\
	{\sl Sortie} : $\mathcal{L}(\mathcal{A}) = \mathcal{L}(\mathcal{B})$\/ ;
	Soit $\mathcal{C}$\/ l'automate reconnaissant $\mathcal{L}(\mathcal{A}) \mathbin\triangle \mathcal{L}(\mathcal{B})$. On retourne $\mathcal{L}(\mathcal{C}) \mathrel{\ds\mathop=^?} \O$\/ à l'aide de l'algorithme précédent.
\end{algo}

Autre possibilité, on procède par double inclusion :

\begin{algo}[$\subseteq$]
	{\sl Entrée} : Deux automates $\mathcal{A}$\/ et $\mathcal{B}$\/ \\
	{\sl Sortie} : $\mathcal{L}(\mathcal{A}) \subseteq \mathcal{L}(\mathcal{B})$\/ ;
	On retourne $\mathcal{A} \setminus \mathcal{B} \mathrel{\ds\mathop=^?} \O$.
\end{algo}

\paragraph{Q.\ 2}
L'algorithme reconnaissant $\mathcal{L}(\mathcal{A}) \mathrel\triangle \mathcal{L}(\mathcal{B})$\/ doit être déterminisé, sa complexité est donc au moins de $2^n$.

		\section{Automates pour le calcul de l'addition en binaire}
\subsection{Nombres de même tailles}

\paragraph{Q.\ 1}~
\begin{figure}[H]
	\centering
	\tikzfig{ex5-q1}
\end{figure}

\paragraph{Q.\ 2}

Pour $r \in \{0,1\}$, il existe une exécution dans $\mathcal{A}$\/ étiquetée par \[
	(u_0,v_0, w_0)(u_1,v_1,w_1)\ldots(u_{n-1},v_{n-1},w_{n-1})
\] menant à $r$\/ si et seulement si \[
	\overline{u_0\ldots u_{n-1}}^2 + \overline{v_0\ldots v_{n-1}}^2 = \overline{w_0\ldots w_{n-1}}^2 + r\:2^{n},
\] ce qui est équivalent à si et seulement si \[
	\overline{u_0\ldots u_{n-1}0}^2 + \overline{v_0\ldots v_{n-1}0}^2 = \overline{w_0\ldots w_{n-1}r}^2
.\]

\paragraph{Q.\ 3}
Prouvons-le par récurrence.
\begin{itemize}
	\item Pour $n = 0$, il existe une exécution dans $\mathcal{A}$\/ étiquetée par $\varepsilon$\/ menant à $r=0$ si et seulement si $\overline{\varepsilon}^2 + \overline{\varepsilon}^2 = 0 = \overline{\varepsilon}^2 + 0 \times 2^0$. De même, il existe une exécution dans $\mathcal{A}$\/ étiquetée par $\varepsilon$\/ menant à $r=1$ si et seulement si $\overline{\varepsilon}^2 + \overline{\varepsilon}^2 = 0 = 1 = \overline{\varepsilon}^2 + 1 \times 2^0$.
\end{itemize}

		\section{Exercice 6 : Langages reconnaissables ou non}

\paragraph{Q.\ 7}
{\slshape Le carré d'un langage est le langage $L_2 = \{u\cdot u \mid u \in L\}$. Si $L$\/ est reconnaissable, $L_2$\/ est-il nécessairement reconnaissable ?}

Avec $\Sigma = \{a,b\}$, soit $L = \mathcal{L}(a^* \cdot b^*)$. On a donc $L_2 = \{a^n \cdot b^m \cdot a^n \cdot b^m  \mid (n,m) \in \N^2\}$. Supposons $L_2$\/ reconnaissable. Soit $\mathcal{A}$\/ un automate à $n$\/ états reconnaissant $L_2$.
On pose $u = a^{2n} \cdot b^n \cdot a^{2n} \cdot b^n \in L_2$. D'après le lemme de l'étoile, il existe $(x,y,z) \in (\Sigma^*)^3$\/ tel que $u = x\cdot y\cdot z$, $|xy| \le n$, $\mathcal{L}(x\cdot y^* \cdot z) \subseteq L_2$, et $y \neq \varepsilon$. Ainsi, il existe $m \in \left\llbracket 1,n \right\rrbracket$\/ et $p \in \left\llbracket 1,n \right\rrbracket$\/ tels que $y = a^{m}$, $x = a^{p}$\/ et $z = a^{2n-m-p} \cdot  b^{n} \cdot a^{2n} \cdot b^n$. Et alors, $x\cdot y^2\cdot z = a^{p}\cdot a^{2m} \cdot a^{n-m-p} \cdot  b^n \cdot a^{2n}\cdot b^n = a^{2n+m} \cdot b^n \cdot a^{2n} \cdot b^n \not\in L_2$.

\paragraph{Q.\ 5} {\slshape Le langage $L_5 = \{a^{n^3}  \mid n \in \N\}$\/ est-il reconnaissable ?}
Soit $\mathcal{A}$\/ un automate à $N$\/ états, et soit $u = a^{N^3}$.
D'après le lemme de l'étoile, il existe $(x,y,z) \in (\Sigma^*)^3$\/ tel que $u = x\cdot y\cdot z$, $|xy| \le N$, $\mathcal{L}(x\cdot y^*\cdot z) \subseteq L_5$\/ et $y \neq \varepsilon$.
D'où $x\cdot y^{0}\cdot z \in L$, et donc $a^{N^3 - i} \in L$, avec $i\le N$. Or, $\forall k \in \N,\:N^3 - i \neq k^3$, ce qui est absurde.


	}
	\def\addmacros#1{#1}
}
{
	\td[6]{Algorithmes probabilistes}
	\minitoc
	\renewcommand{\cwd}{../td/td06/}
	\addmacros{
		\section{Déterminisation de taille exponentielle}

\begin{enumerate}
	\item En notant $n$\/ le nombre d'états de $\mathcal{A}$, alors le nombre d'états de $\det(\mathcal{A})$\/ est, au plus, $2^n$. En effet, les états sont des éléments de $\wp(Q)$\/ et $|\wp(Q)| = 2^n$.
	\item $\mathcal{A}$ : \tikzfig{ex1-q2a} $\mathcal{B}$ : \tikzfig{ex1-q2b}\\
		mais $\det(\mathcal{A})$\/ : \tikzfig{ex1-q2c}
	\item $\mathcal{A}_n$\/ : \tikzfig{ex1-q3}
	\item $\mathcal{A}_3$\/ : \tikzfig{ex1-q4}\\
		et, $\det(\mathcal{A}_3)$\/ : \tikzfig{ex1-q4b}
	\item Soit $i_0 = \max \{k \in \left\llbracket 1,n \right\rrbracket  \mid u_k \neq v_k\}$. Soit $m \in \Sigma^{i_0}$\/ tel que $u \cdot m \in L_n$\/ mais $v\cdot m \not\in L_n$. Or, $\delta^*(i, u\cdot m) = \delta^*(\delta^*(i,u), m)$\/ et $\delta^*(i,v\cdot m) = \delta^*(\delta^*(i,v),m)$. D'où $\delta^*(i,u\cdot m) \in F$\/ et $\delta^*(i,v\cdot m) \not\in F$. Ce qui est absurde.
	\item Ainsi, l'application \begin{align*}
			f: \Sigma^* &\longrightarrow Q \\
			u &\longmapsto \delta^*(i,u)
		\end{align*} est injective. D'où, $\mathcal{D}_n = |Q| \ge |\Sigma^*| = 2^n$.
	\item D'où, d'après les questions 1 et 6, on en déduit que le nombre d'états utilisés pour la déterminisation de $\mathcal{A}_n$\/ est de $\mathcal{D}_n \ge 2^n$.
\end{enumerate}

		\section{Suppression des $\varepsilon$-transitions}

\begin{figure}[H]
	\centering
	\tikzfig{ex2}
\end{figure}

		\section{Exercice 3 : Échantillonnage}

\paragraph{Q.\ 1}~

\begin{algorithm}[H]
	\centering
	\begin{algorithmic}[1]
		\Entree $T$\/ un tableau à $n$\/ éléments, et $k \in \N$\/ avec $k \le n$\/
		\State $T \gets \text{Mélanger}(T)$\/
		\State $R \gets T[0..k]$
		\State\Return $R$\/
	\end{algorithmic}
	\caption{Échantillonnage naïf}
\end{algorithm}

\paragraph{Q.\ 2}
Un invariant de boucle est \guillemotleft~$\forall p \in \left\llbracket 0,I-1 \right\rrbracket,\:P(T[p] \in \textsf{Res}) = \frac{k}{I}$\/ et $\forall p \in \left\llbracket I,n \right\rrbracket,\:T[p] \not\in \textsf{Res}$~\guillemotright

\paragraph{Q.\ 3}
Notons $\ubar{I}$\/ et $\underline{\textsf{Res}}$\/ l'état des variables avant un tour de boucle ; et, $\bar{I}$\/ et $\overline{\textsf{Res}}$\/ l'état des variables après un tour de boucle.
\begin{itemize}
	\item Pour $k = I$, on a
		\begin{enumerate}
			\item $\forall p \in \left\llbracket 0,k-1 \right\rrbracket$, $P(T[p] \in \textsf{Res}) = 1$,
			\item $\forall p \in \left\llbracket k, n-1 \right\rrbracket$, $T[p] \not\in \textsf{Res}$,
			\item $I \le n$.
		\end{enumerate}
	\item Supposons $\ubar{I}$, et $\underline{\textsf{Res}}$\/ vérifiant l'invariant et la condition de boucle. Alors, on a
		\begin{enumerate}
			\item $\forall p \in \left\llbracket 0, \ubar{I} - 1 \right\rrbracket$, $P(T[p] \in \underline{\textsf{Res}}) = \frac{k}{\ubar{I}}$,
			\item $\forall p \in \left\llbracket \ubar{I}, n-1 \right\rrbracket$, $T[p] \not\in \underline{\textsf{Res}}$,
			\item $\ubar{I} < n$, la condition de boucle.
		\end{enumerate}
		Soit $j \in \left\llbracket 0, \ubar{I} \right\rrbracket$.
		On a $\bar{I} = \ubar{I} + 1$.
		\begin{itemize}
			\item[\sc Cas 1] $j < k$, et donc $\overline{\textsf{Res}}(j) = T[\ubar{I}]$, et $\forall \ell \neq j$, $\overline{\textsf{Res}}[\ell] = \underline{\textsf{Res}}[\ell]$.
			\item[\sc Cas 2] $j \ge  k$, et donc $\forall \ell,\:\overline{\textsf{Res}}[\ell] = \underline{\textsf{Res}}[\ell]$.
		\end{itemize}

		\begin{enumerate}
			\item Soit $p \in \left\llbracket 0, \ubar{I} \right\rrbracket $. Montrons $P(T[p] \not\in \overline{\textsf{Res}}) = \frac{k}{\bar{I}}$. Si $p < \ubar{I}$, alors
				\begin{align*}
					P(T[p] \in \overline{\textsf{Res}}) &= P\big(T[p] \in \underline{\textsf{Res}} \cap j \neq p\big) \\
					&= \frac{k}{\ubar{I}} \times \frac{\ubar{I}}{\ubar{I} + 1} \\
					&= \frac{k}{\bar{I}}.
				\end{align*}
				Si $P = \ubar{I}$, alors d'après 2.\ $T[p] \not\in \underline{\textsf{Res}}$, donc $P(T[p] \in \overline{\textsf{Res}}) = P(j < k) = \frac{k}{\bar{I} + 1}$.
		\end{enumerate}
\end{itemize}





	}
	\def\addmacros#1{#1}
}
{
	\td[7]{Décidabilité, Calculabilité}
	\minitoc
	\renewcommand{\cwd}{../td/td07/}
	\addmacros{
		\section{Filtre RC double}

\begin{enumerate}
	\item En basse fréquence, un condensateur est équivalent à un interrupteur ouvert. En haute fréquence, un condensateur est équivalent à un interrupteur fermé. D'où, le circuit est un filtre passe-bas.
	\item Par une loi des nœuds, et une loi des mailles, on trouve que
		\[
			\ubar{H}(\mathrm{j}\omega) = \dfrac{1}{1 - \left( \dfrac{\omega}{\omega_0} \right)^2 + \mathrm{j} \dfrac{\omega}{Q\cdot \omega_0}}
		\] en notant $\omega_0 = 1 / {RC}$ et $Q = 1 / 3$
	\item On représente le diagramme de \textsc{Bode} du filtre dans la figure ci-dessous.
		\begin{figure}[H]
			\centering
			\includesvg[width=\linewidth]{figures/bode-1.svg}
			\caption{Diagramme de \textsc{Bode} du filtre (échelle logarithmique)}
		\end{figure}
	\item On calcule $\omega_0 \simeq 6\:\mathrm{rad/s}$, ce qui correspond à une fréquence de coupure de $1\:\mathrm{kHz}$. Le signal de sortie est donc \[
			s(t) = \frac{2E}{3\pi}\cdot \sin(\omega t)
		,\] on le représente sur la figure ci-dessous. En effet, on a un déphasage de $-\pi / 2$, et un gain valant $1 / 3$ à $\omega \simeq \omega_0$.
		\begin{figure}[H]
			\centering
			\includesvg[width=\linewidth]{figures/signal-1.svg}
			\caption{Signal résultant}
		\end{figure}
\end{enumerate}

		\section{Tri topologique}

\begin{figure}[H]
	\centering
	\tikzfig{ex2-q1}
	\caption{Exemple de graphe}
\end{figure}

\begin{enumerate}
	\item Dans le graphe ci-dessus, $a \to c \to b$\/ est un tri topologique mais pas un parcours.
	\item Dans le même graphe, $b \to a \to c$\/ est un parcours mais pas un tri topologique.
	\item Supposons que $L_1$\/ possède un prédécesseur, on le note $L_i$\/ où $i > 1$. Ainsi, $(L_i, L_1) \in A$\/ et donc $i < 1$, ce qui est absurde. De même pour le dernier.
	\item Il existe un tri topologique si, et seulement si le graphe est acyclique.
		\begin{itemize}
			\item[``$\implies$'']
				Soit $L_1,\ldots,L_n$\/ un tri topologique. Montrons que le graphe est acyclique.
				Par l'absurde, on suppose le graphe non acyclique : il existe $(i,j) \in \llbracket 1,n \rrbracket^2$\/ avec $i \neq j$\/ tels que $T_i \to \cdots \to T_j$\/ et $T_j \to \cdots \to T_i$ soient deux chemins valides. Ainsi, comme le tri est topologique et par récurrence, $i \le j$\/ et $j \le i$\/ et donc $i = j$, ce qui est absurde car $i$\/ et $j$\/ sont supposés différents. Le graphe est donc acyclique.
			\item[``$\impliedby$'']
				Soit $G$\/ un graphe tel que tous les sommets possèdent une arrête entrante. On suppose par l'absurde ce graphe acyclique.
				Soit $x_0$\/ un sommet du graphe.
				On construit par récurrence $x_0,x_1,\ldots,x_n,x_{n+1},\ldots$ les successeurs successifs. Il y a un nombre fini de sommets donc deux sommets sont identiques. Donc, il y a nécessairement un cycle, ce qui est absurde.
				\begin{algorithm}[H]
					\centering
					\begin{algorithmic}[1]
						\Entree $G = (S, A)$\/ un graphe acyclique
						\Sortie $\mathrm{Res}$\/ un tri topologique.
						\State $\mathrm{Res} \gets [\quad]$\/
						\While{$G \neq \O$}
							\State Soit $x$\/ un sommet de $G$\/ sans prédécesseur
							\State $G \gets \big(S \setminus \{x\}, A \cap (S \setminus \{x\})^2\big)$\/ 
							\State $\mathrm{Res} \gets \mathrm{Res} \cdot [x]$\/
						\EndWhile
						\State\Return $\mathrm{Res}$\/
					\end{algorithmic}
					\caption{Génération d'un tri topologique d'un graphe acyclique}
				\end{algorithm}
		\end{itemize}
	\item~
		\begin{algorithm}[H]
			\centering
			\begin{algorithmic}[1]
				\Entree $G = (S, A)$\/ un graphe
				\Sortie $\mathrm{Res}$\/ un tri topologique, ou un cycle
				\State $\mathrm{Res} \gets [\quad]$\/
				\While{$G \neq \O$}
					\If{il existe $x$ sans prédécesseurs}
						\State Soit $x$\/ un sommet de $G$\/ sans prédécesseur
						\State $G \gets \big(S \setminus \{x\}, A \cap (S \setminus \{x\})^2\big)$\/ 
						\State $\mathrm{Res} \gets \mathrm{Res} \cdot [x]$\/
					\Else
						\State Soit $x \in S$\/ 
						\State Soit $x \gets x_1 \gets x_2 \gets \cdots \gets x_i$\/ la suite des prédécesseurs
						\State\Return $x_i,x_{i+1},\ldots,x_i$, un cycle
					\EndIf
				\EndWhile
				\State\Return $\mathrm{Res}$\/
			\end{algorithmic}
			\caption{Génération d'un tri topologique d'un graphe}
		\end{algorithm}
	\item On utilise la représentation par liste d'adjacence, et on stocke le nombre de prédécesseurs que l'on décroit à chaque choix de sommet.
	\item On essaie de trouver un tri topologique, et on voit si l'on trouve un cycle.
\end{enumerate}

		\section{Formules duales}

\begin{enumerate}
	\item On définit par induction $(\cdot)^\star$\/ comme
		\begin{multicols}{3}
			\begin{itemize}
				\item $\top^\star = \bot$\/ ;
				\item $\bot^\star = \top$\/ ;
				\item $(G \lor H)^\star = G^\star \land H^\star$\/ ;
				\item $(G \land H)^\star = G^\star \lor H^\star$\/ ;
				\item $(\lnot G)^\star = \lnot G^\star$\/ ;
				\item $p^\star = p$.
			\end{itemize}
		\end{multicols}
	\item Soit $\rho \in \mathds{B}^{\mathcal{P}}$. Montrons, par induction, $P(H) : ``\left\llbracket H^\star \right\rrbracket^\rho = \left\llbracket \lnot H \right\rrbracket^{\bar \rho}"$\/ où $\bar{\rho} : p \mapsto \overline{\rho(p)}$.
		\begin{itemize}
			\item On a $\left\llbracket \bot^\star \right\rrbracket^\rho = \left\llbracket \top \right\rrbracket^\rho = \mathbf{V}$, et $\left\llbracket \lnot \bot \right\rrbracket^{\bar\rho} = \left\llbracket \top \right\rrbracket^{\bar\rho} = \mathbf{V}$, d'où $P(\bot)$.
			\item On a $\left\llbracket \top^\star \right\rrbracket^\rho = \left\llbracket \bot \right\rrbracket^\rho = \mathbf{F}$, et $\left\llbracket \lnot \top \right\rrbracket^{\bar\rho} = \left\llbracket \bot \right\rrbracket^{\bar\rho} = \mathbf{F}$, d'où $P(\top)$.
			\item Soit $p \in \mathcal{P}$. On a $\left\llbracket p^\star  \right\rrbracket^\rho = \left\llbracket p \right\rrbracket^\rho = \rho(p)$, et $\left\llbracket \lnot p \right\rrbracket^{\bar\rho} = \overline{\left\llbracket p \right\rrbracket^{\bar\rho}} = \overline{\bar\rho(p)} = \overline{\overline{\rho(p)}} = \rho(p)$, d'où~$P(p)$.
		\end{itemize}
		Soient $F$\/ et $G$\/ deux formules.
		\begin{itemize}
			\item On a
				\begin{align*}
					\left\llbracket (F \land G)^\star  \right\rrbracket^\rho &= \left\llbracket F^\star  \lor G^\star \right\rrbracket^\rho\\
					&= \left\llbracket F^\star \right\rrbracket^\rho + \left\llbracket G^\star \right\rrbracket^\rho \\
					&= \left\llbracket \lnot F \right\rrbracket^{\bar\rho} + \left\llbracket \lnot G \right\rrbracket^{\bar\rho} \\
					&= \left\llbracket \lnot F \lor \lnot G \right\rrbracket^{\bar\rho} \\
					&= \left\llbracket \lnot (F \land G) \right\rrbracket^{\bar\rho} \\
				\end{align*}
				d'où $P(F \land G)$.
			\item On a
				\begin{align*}
					\left\llbracket (F \lor G)^\star  \right\rrbracket^\rho &= \left\llbracket F^\star  \land G^\star \right\rrbracket^\rho\\
					&= \left\llbracket F^\star \right\rrbracket^\rho \cdot \left\llbracket G^\star \right\rrbracket^\rho \\
					&= \left\llbracket \lnot F \right\rrbracket^{\bar\rho} \cdot \left\llbracket \lnot G \right\rrbracket^{\bar\rho} \\
					&= \left\llbracket \lnot F \land \lnot G \right\rrbracket^{\bar\rho} \\
					&= \left\llbracket \lnot (F \lor G) \right\rrbracket^{\bar\rho} \\
				\end{align*}
				d'où $P(F \lor G)$.
			\item On a \[
					\left\llbracket (\lnot F)^\star  \right\rrbracket^\rho = \left\llbracket \lnot (F^\star) \right\rrbracket^\rho = \overline{\left\llbracket F^\star \right\rrbracket^\rho} = \overline{\left\llbracket \lnot F \right\rrbracket^{\bar\rho}} = \left\llbracket \lnot (\lnot F) \right\rrbracket^{\bar\rho}.
					\] d'où $P(\lnot F)$.
		\end{itemize}
		Par induction, on en conclut que $P(F)$\/ est vraie pour toute formule $F$.
	\item Soit $G$\/ une formule valide. Alors, par définition, $G \equiv \top$. Or, d'après la question précédente, $G^\star \equiv (\top)^\star = \bot$. Ainsi, $G^\star $\/ n'est pas satisfiable.
\end{enumerate}



		\section{Un lemme d'itération}

		\section{Ambigüité}

		\section{Regexp Crossword}
\begin{center}
	\url{https://regexcrossword.com/}
\end{center}

		\documentclass[a4paper]{article}

\usepackage[margin=1in]{geometry}
\usepackage[utf8]{inputenc}
\usepackage[T1]{fontenc}
\usepackage{mathrsfs}
\usepackage{textcomp}
\usepackage[french]{babel}
\usepackage{amsmath}
\usepackage{amssymb}
\usepackage{cancel}
\usepackage{frcursive}
\usepackage[inline]{asymptote}
\usepackage{tikz}
\usepackage[european,straightvoltages,europeanresistors]{circuitikz}
\usepackage{tikz-cd}
\usepackage{tkz-tab}
\usepackage[b]{esvect}
\usepackage[framemethod=TikZ]{mdframed}
\usepackage{centernot}
\usepackage{diagbox}
\usepackage{dsfont}
\usepackage{fancyhdr}
\usepackage{float}
\usepackage{graphicx}
\usepackage{listings}
\usepackage{multicol}
\usepackage{nicematrix}
\usepackage{pdflscape}
\usepackage{stmaryrd}
\usepackage{xfrac}
\usepackage{hep-math-font}
\usepackage{amsthm}
\usepackage{thmtools}
\usepackage{indentfirst}
\usepackage[framemethod=TikZ]{mdframed}
\usepackage{accents}
\usepackage{soulutf8}
\usepackage{mathtools}
\usepackage{bodegraph}
\usepackage{slashbox}
\usepackage{enumitem}
\usepackage{calligra}
\usepackage{cinzel}
\usepackage{BOONDOX-calo}

% Tikz
\usetikzlibrary{babel}
\usetikzlibrary{positioning}
\usetikzlibrary{calc}

% global settings
\frenchspacing
\reversemarginpar
\setuldepth{a}

%\everymath{\displaystyle}

\frenchbsetup{StandardLists=true}

\def\asydir{asy}

%\sisetup{exponent-product=\cdot,output-decimal-marker={,},separate-uncertainty,range-phrase=\;à\;,locale=FR}

\setlength{\parskip}{1em}

\theoremstyle{definition}

% Changing math
\let\emptyset\varnothing
\let\ge\geqslant
\let\le\leqslant
\let\preceq\preccurlyeq
\let\succeq\succcurlyeq
\let\ds\displaystyle
\let\ts\textstyle

\newcommand{\C}{\mathds{C}}
\newcommand{\R}{\mathds{R}}
\newcommand{\Z}{\mathds{Z}}
\newcommand{\N}{\mathds{N}}
\newcommand{\Q}{\mathds{Q}}

\renewcommand{\O}{\emptyset}

\newcommand\ubar[1]{\underaccent{\bar}{#1}}

\renewcommand\Re{\expandafter\mathfrak{Re}}
\renewcommand\Im{\expandafter\mathfrak{Im}}

\let\slantedpartial\partial
\DeclareRobustCommand{\partial}{\text{\rotatebox[origin=t]{20}{\scalebox{0.95}[1]{$\slantedpartial$}}}\hspace{-1pt}}

% merging two maths characters w/ \charfusion
\makeatletter
\def\moverlay{\mathpalette\mov@rlay}
\def\mov@rlay#1#2{\leavevmode\vtop{%
   \baselineskip\z@skip \lineskiplimit-\maxdimen
   \ialign{\hfil$\m@th#1##$\hfil\cr#2\crcr}}}
\newcommand{\charfusion}[3][\mathord]{
    #1{\ifx#1\mathop\vphantom{#2}\fi
        \mathpalette\mov@rlay{#2\cr#3}
      }
    \ifx#1\mathop\expandafter\displaylimits\fi}
\makeatother

% custom math commands
\newcommand{\T}{{\!\!\,\top}}
\newcommand{\avrt}[1]{\rotatebox{-90}{$#1$}}
\newcommand{\bigcupdot}{\charfusion[\mathop]{\bigcup}{\cdot}}
\newcommand{\cupdot}{\charfusion[\mathbin]{\cup}{\cdot}}
%\newcommand{\danger}{{\large\fontencoding{U}\fontfamily{futs}\selectfont\char 66\relax}\;}
\newcommand{\tendsto}[1]{\xrightarrow[#1]{}}
\newcommand{\vrt}[1]{\rotatebox{90}{$#1$}}
\newcommand{\tsup}[1]{\textsuperscript{\underline{#1}}}
\newcommand{\tsub}[1]{\textsubscript{#1}}

\renewcommand{\mod}[1]{~\left[ #1 \right]}
\renewcommand{\t}{{}^t\!}
\newcommand{\s}{\text{\calligra s}}

% custom units / constants
%\DeclareSIUnit{\litre}{\ell}
\let\hbar\hslash

% header / footer
\pagestyle{fancy}
\fancyhead{} \fancyfoot{}
\fancyfoot[C]{\thepage}

% fonts
\let\sc\scshape
\let\bf\bfseries
\let\it\itshape
\let\sl\slshape

% custom math operators
\let\th\relax
\let\det\relax
\DeclareMathOperator*{\codim}{codim}
\DeclareMathOperator*{\dom}{dom}
\DeclareMathOperator*{\gO}{O}
\DeclareMathOperator*{\po}{\text{\cursive o}}
\DeclareMathOperator*{\sgn}{sgn}
\DeclareMathOperator*{\simi}{\sim}
\DeclareMathOperator{\Arccos}{Arccos}
\DeclareMathOperator{\Arcsin}{Arcsin}
\DeclareMathOperator{\Arctan}{Arctan}
\DeclareMathOperator{\Argsh}{Argsh}
\DeclareMathOperator{\Arg}{Arg}
\DeclareMathOperator{\Aut}{Aut}
\DeclareMathOperator{\Card}{Card}
\DeclareMathOperator{\Cl}{\mathcal{C}\!\ell}
\DeclareMathOperator{\Cov}{Cov}
\DeclareMathOperator{\Ker}{Ker}
\DeclareMathOperator{\Mat}{Mat}
\DeclareMathOperator{\PGCD}{PGCD}
\DeclareMathOperator{\PPCM}{PPCM}
\DeclareMathOperator{\Supp}{Supp}
\DeclareMathOperator{\Vect}{Vect}
\DeclareMathOperator{\argmax}{argmax}
\DeclareMathOperator{\argmin}{argmin}
\DeclareMathOperator{\ch}{ch}
\DeclareMathOperator{\com}{com}
\DeclareMathOperator{\cotan}{cotan}
\DeclareMathOperator{\det}{det}
\DeclareMathOperator{\id}{id}
\DeclareMathOperator{\rg}{rg}
\DeclareMathOperator{\rk}{rk}
\DeclareMathOperator{\sh}{sh}
\DeclareMathOperator{\th}{th}
\DeclareMathOperator{\tr}{tr}

% colors and page style
\definecolor{truewhite}{HTML}{ffffff}
\definecolor{white}{HTML}{faf4ed}
\definecolor{trueblack}{HTML}{000000}
\definecolor{black}{HTML}{575279}
\definecolor{mauve}{HTML}{907aa9}
\definecolor{blue}{HTML}{286983}
\definecolor{red}{HTML}{d7827e}
\definecolor{yellow}{HTML}{ea9d34}
\definecolor{gray}{HTML}{9893a5}
\definecolor{grey}{HTML}{9893a5}
\definecolor{green}{HTML}{a0d971}

\pagecolor{white}
\color{black}

\begin{asydef}
	settings.prc = false;
	settings.render=0;

	white = rgb("faf4ed");
	black = rgb("575279");
	blue = rgb("286983");
	red = rgb("d7827e");
	yellow = rgb("f6c177");
	orange = rgb("ea9d34");
	gray = rgb("9893a5");
	grey = rgb("9893a5");
	deepcyan = rgb("56949f");
	pink = rgb("b4637a");
	magenta = rgb("eb6f92");
	green = rgb("a0d971");
	purple = rgb("907aa9");

	defaultpen(black + fontsize(8pt));

	import three;
	currentlight = nolight;
\end{asydef}

% theorems, proofs, ...

\mdfsetup{skipabove=1em,skipbelow=1em, innertopmargin=6pt, innerbottommargin=6pt,}

\declaretheoremstyle[
	headfont=\normalfont\itshape,
	numbered=no,
	postheadspace=\newline,
	headpunct={:},
	qed=\qedsymbol]{demstyle}

\declaretheorem[style=demstyle, name=Démonstration]{dem}

\newcommand\veczero{\kern-1.2pt\vec{\kern1.2pt 0}} % \vec{0} looks weird since the `0' isn't italicized

\makeatletter
\renewcommand{\title}[2]{
	\AtBeginDocument{
		\begin{titlepage}
			\begin{center}
				\vspace{10cm}
				{\Large \sc Chapitre #1}\\
				\vspace{1cm}
				{\Huge \calligra #2}\\
				\vfill
				Hugo {\sc Salou} MPI${}^{\star}$\\
				{\small Dernière mise à jour le \@date }
			\end{center}
		\end{titlepage}
	}
}

\newcommand{\titletp}[4]{
	\AtBeginDocument{
		\begin{titlepage}
			\begin{center}
				\vspace{10cm}
				{\Large \sc tp #1}\\
				\vspace{1cm}
				{\Huge \textsc{\textit{#2}}}\\
				\vfill
				{#3}\textit{MPI}${}^{\star}$\\
			\end{center}
		\end{titlepage}
	}
	\fancyfoot{}\fancyhead{}
	\fancyfoot[R]{#4 \textit{MPI}${}^{\star}$}
	\fancyhead[C]{{\sc tp #1} : #2}
	\fancyhead[R]{\thepage}
}

\newcommand{\titletd}[2]{
	\AtBeginDocument{
		\begin{titlepage}
			\begin{center}
				\vspace{10cm}
				{\Large \sc td #1}\\
				\vspace{1cm}
				{\Huge \calligra #2}\\
				\vfill
				Hugo {\sc Salou} MPI${}^{\star}$\\
				{\small Dernière mise à jour le \@date }
			\end{center}
		\end{titlepage}
	}
}
\makeatother

\newcommand{\sign}{
	\null
	\vfill
	\begin{center}
		{
			\fontfamily{ccr}\selectfont
			\textit{\textbf{\.{\"i}}}
		}
	\end{center}
	\vfill
	\null
}

\renewcommand{\thefootnote}{\emph{\alph{footnote}}}

% figure support
\usepackage{import}
\usepackage{xifthen}
\pdfminorversion=7
\usepackage{pdfpages}
\usepackage{transparent}
\newcommand{\incfig}[1]{%
	\def\svgwidth{\columnwidth}
	\import{./figures/}{#1.pdf_tex}
}

\pdfsuppresswarningpagegroup=1
\ctikzset{tripoles/european not symbol=circle}

\newcommand{\missingpart}{{\large\color{red} Il manque quelque chose ici\ldots}}


\fancyhead[R]{Hugo {\sc Salou}\/ MPI}
\fancyhead[L]{TD\textsubscript2 -- Exercice 7}

\begin{document}
	\let\thesection\relax
	\section{Exercice 3}

{\bf Indication}\/ : pour la $G$, on applique la relation de {\sc Chasles}\/ : l'intégrale $\int_0^7 \mathrm{e}^{-x}\ln x~\mathrm{d}x$\/ converge si et seulement si $\int_0^1 \mathrm{e}^{-x}\ln x\mathrm{d}x$\/ converge et $\int_1^7\mathrm{e}^{-x}\ln x~\mathrm{d}x$\/ converge (qui n'est même pas impropre).

L'intégrale $H = \int_0^1 \frac{\mathrm{e}^{\sin t}}{t}~\mathrm{d}t$\/ est impropre en 0. On sait que $\sin t \tendsto{t\to 0} 0$, et donc, par continuité de la fonction $\exp$\/ en $0$, $\mathrm{e}^{\sin t}\tendsto{t\to 0}e^0 = 1$.
Ainsi, $\frac{\mathrm{e}^{\sin t}}{t} = \mathrm{e}^{\sin t} \times \frac{1}{t} \simi_{t\to 0} \frac{1}{t}$\/ qui ne change pas de signe. Or, $\int_0^1 \frac{1}{t}~\mathrm{d}t$\/ diverge, donc l'intégrale $H$\/ diverge.

L'intégrale $I = \int_{1}^{+\infty} \frac{\mathrm{e}^{\sin t}}{t}~\mathrm{d}t$\/ est impropre en $+\infty$. Par croissance de la fonction exponentielle, on a $\frac{\mathrm{e}^{\sin t}}{t} \ge \frac{\mathrm{e}^{-1}}{t} \ge 0$. Or, l'intégrale $\int_{1}^{+\infty} \frac{1}{t}~\mathrm{d}t$\/ diverge, donc l'intégrale diverge aussi.

L'intégrale $K$, est l'intégrale d'une fonction Gau\ss ienne, et elle est impropre en $+\infty$. On la \guillemotleft~découpe~\guillemotright\ : \[
	\int_{0}^{+\infty} \mathrm{e}^{-x^2}~\mathrm{d}x \text{ converge si et seulement si } \int_{0}^{1} \mathrm{e}^{-x^2}~\mathrm{d}x \text{ converge et } \int_{1}^{+\infty} \mathrm{e}^{-x^2}~\mathrm{d}x \text{ converge.}
\] L'intégrale $\int_{0}^{1} \mathrm{e}^{-x^2}~\mathrm{d}x$\/ n'est même pas impropre, elle converge donc. Et, pour $x \in [1,+\infty[$, on sait, comme $x^2 \ge x$, $0 \le \mathrm{e}^{-x^2} \le \mathrm{e}^{-x}$. Or, $\int_{1}^{+\infty} \mathrm{e}^{-x}~\mathrm{d}x$\/ converge donc $\int_{0}^{+\infty} \mathrm{e}^{-x^2}~\mathrm{d}x$\/ aussi.
On calculera la valeur de cette intégrale dans le {\sc td}\/ \guillemotleft~Intégrales paramétrées.~\guillemotright

Autre méthode pour déterminer la nature de $K$\/ : 
$\mathrm{e}^{-x^2} = \po(\mathrm{e}^{-x})$\/ car $\mathrm{e}^{-x^2} = \underbrace{\mathrm{e}^{-x^2 + x}}_{\to 0} \times \mathrm{e}^{-x}$, car $\mathrm{e}^{-x^2 + x} = \mathrm{e}^{-x^2 \left( 1 - \frac{1}{x} \right)}$\/ et $-x^2\left( 1 - \frac{1}{x} \right) \to -\infty \times 1$.
Et $\int_0^{+\infty} \mathrm{e}^{-x}~\mathrm{d}x$\/ converge donc $\int_0^{+\infty} \mathrm{e}^{-x^2}~\mathrm{d}x$\/ converge.

\begin{figure}[H]
	\centering
	\begin{asy}
		import graph;
		size(10cm);
		draw((-10, 0) -- (10, 0), Arrow(TeXHead));
		draw((0, -3) -- (0, 5), Arrow(TeXHead));
		real f(real x) {
			return 4*exp(-(x/4)^2);
		}
		draw(graph(f, -10, 10), magenta);
	\end{asy}
	\caption{Courbe Gau\ss ienne}
\end{figure}

L'intégrale $F = \int_{7}^{+\infty} \mathrm{e}^{-x}\ln x~\mathrm{d}x$\/ est impropre en $+\infty$. Attention : la fonction n'est pas \guillemotleft~{\color{red} faussement impropre en $+\infty$}.~\guillemotright\ Mais, on peut remarquer que \[
	\mathrm{e}^{-x} \ln x = \mathrm{e}^{-\frac{x}{2}} \underbrace{\mathrm{e}^{-\frac{x}{2}} \ln x}_{\tendsto{x\to +\infty} 0} = \po(\mathrm{e}^{-\frac{x}{2}})
.\] Or, $\int_{7}^{+\infty} \mathrm{e}^{-x}~\mathrm{d}x$\/ converge donc l'intégrale $F$\/ converge aussi.

L'intégrale $G = \int_{0}^{7} \mathrm{e}^{-x}\ln x~\mathrm{d}x$\/ est impropre en 0.
Or, $\mathrm{e}^{-x}\ln x \simi_{x\to 0} \ln x$\/ qui ne change pas de signe au voisinage de 0. Or, $\int_{0}^{7}  \ln x~\mathrm{d}x$\/ converge donc l'intégrale $G$\/ converge également.

L'intégrale $E = \int_{1}^{+\infty} \frac{\ln x}{\sqrt{x}}~\mathrm{d}x$\/ est impropre en $+\infty$. Or, $\forall x \ge \mathrm{e}$, $\frac{\ln(x)}{\sqrt{x}} \ge \frac{1}{\sqrt{x}} \ge 0$\/ converge.
Or, $\int_{1}^{+\infty}  \frac{1}{x^{\sfrac{1}{2}}}~\mathrm{d}x$\/ diverge d'après le critère de {\sc Riemann}\/ en $+\infty$\/ car $\frac{1}{2} < 1$.
D'où l'intégrale $D$\/ diverge.

Autre méthode : intégration par parties. On peut même arriver à calculer une primitive de ${\ln x}\:/{\sqrt{x}}$.

L'intégrale $D = \int_{0}^{1} \frac{\ln x}{\sqrt{x}}~\mathrm{d}x$\/ est impropre en 0. On peut remarque que \[
	0 \le -\frac{\ln x}{\sqrt{x}} = -\frac{x^{0{,}1} \ln x}{x^{0{,}6}} = \po\left( \frac{1}{x^{0{,}6}} \right) \quad\text{car}\quad x^{0{,}1} \ln x \tendsto{x\to 0} 0
\] par croissances comparées.
Or, $\int_{0}^{1} \frac{1}{x^{0{,}6}}~\mathrm{d}x$\/ converge d'après le critère de {\sc Riemann}. D'où $-D$\/ converge et donc $D$\/ converge.

L'intégrale $J = \int_{1}^{+\infty} \frac{\sin t}{\sqrt{t} + \sin t}~\mathrm{d}t$\/ est impropre en $+\infty$. On calcule
\[
	f(t) = \frac{\sin t}{\sqrt{t} + \sin t} = \frac{\sin t}{\sqrt{t}} \times \frac{1}{1+\frac{\sin t}{\sqrt{t}}}
\] et $\frac{\sin t}{\sqrt{t}} \tendsto{t\to +\infty} 0$. D'où \[
	\frac{1}{1+\frac{\sin t}{\sqrt{t}}} = 1  - \frac{\sin t}{\sqrt{t}} + \frac{\sin^2 t}{t} + \po\left( \frac{\sin^2 t}{t} \right)
\] et donc \[
	f(t) = \frac{\cos t}{\sqrt{t}} + \frac{\sin^2 t}{t} + \po\left( \frac{\sin^2 t}{t} \right)
.\]
L'intégrale $\int_{1}^{+\infty} \frac{\sin t}{\sqrt{t}}~\mathrm{d}t$\/ est impropre en $+\infty$.
Soit $x \ge 1$. On calcule avec une intégration par parties,
\[
	\int_{1}^{x} \sin t \times \frac{1}{\sqrt{t}}~\mathrm{d}t = \int_{1}^{x} u'(t)\cdot v(t)~\mathrm{d}t
\] où $u(t) = - \cos t$\/ et $v(t) = \frac{1}{\sqrt{t}} = t^{-\frac{1}{2}}$. Donc
\begin{align*}
	\int_{1}^{x} \frac{\sin t}{\sqrt{t}}~\mathrm{d}t &= \Big[f(t)g(t)\Big]_1^x - \int_{1}^{x} f(t)\cdot g'(t)~\mathrm{d}t\\
	&= \left[ - \frac{\cos t}{\sqrt{t}} \right]_1^x - \int_{1}^{x} (-\cos t)\left( -\frac{1}{2}t^{-\frac{3}{2}} \right)~\mathrm{d}t \\
\end{align*}
D'où \[
	\int_{1}^{x} \frac{\sin t}{\sqrt{t}}~\mathrm{d}t = \cos 1 - \frac{\cos x}{\sqrt{x}} - \frac{1}{2} \int_{1}^{x} \frac{\cos t}{t^{\sfrac{3}{2}}}~\mathrm{d}t
.\]
Or, d'une part $\cos x \times \frac{1}{\sqrt{x}} \tendsto{x\to +\infty} 0$\/ car $\cos$\/ est bornée et $\frac{1}{\sqrt{x}}\tendsto{x\to +\infty} 0$.
Et, d'autre part $\int_{1}^{+\infty} \frac{\cos t}{t^{\sfrac{3}{2}}}~\mathrm{d}t$\/ converge car $\forall t \in [1,+\infty[$, $\left| \frac{\cos t}{t^{\sfrac{3}{2}}} \right| \le \frac{1}{t^{\sfrac{3}{2}}}$\/ et $\int_{1}^{+\infty} \frac{1}{t^{\sfrac{3}{2}}}~\mathrm{d}t$\/ converge.
Pour le 2\tsup{nd} terme du développement limité, on fait une {\sc ipp}, on trouve un terme en $\frac{1}{t^2}$\/ et donc son intégrale converge par critère de {\sc Riemann}. S'il y a des problèmes, voir en {\sc td}.
On étudie maintenant le 3\tsup{ème} terme : \[
	\int_{1}^{+\infty}  \po\left( \frac{\sin t}{t} \right) ~\mathrm{d}t \text{ converge car } \int_{1}^{+\infty} \frac{\sin^2 t}{t} ~\mathrm{d}t \text{ converge et } t\mapsto \frac{\sin^2 t}{t} \text{ est positive}.
\]
Autre méthode : on a \[
	\frac{\sin^2 t}{t} + \po\left( \frac{\sin^2 t}{t} \right) \simi_{t\to +\infty} \frac{\sin^2 t}{t} \text{ qui ne change pas de signe}
.\] Or, $\int_{1}^{+\infty} \frac{\sin^2 t}{t}~\mathrm{d}t$\/ converge et donc \[
	\int_{1}^{+\infty} \left( \frac{\sin^2 t}{t} + \po\left( \frac{\sin^2 t}{t} \right) \right) ~\mathrm{d}t
.\]

\end{document}

		\section{Vocabulaire des automates}

On représente, ci-dessous, l'automate $\mathcal{A}$\/ décrit dans l'énoncé.
\begin{figure}[H]
	\centering
	\tikzfig{automate-ex8}
	\caption{Automate décrit dans l'énoncé de l'exercice 8}
\end{figure}

\begin{enumerate}
	\item Cet automate n'est pas complet : à l'état 0, la lecture d'un $a$\/ peut conduire à l'état 0 ou bien à l'état 1.
	\item Le mot $baba$\/ est reconnu par $\mathcal{A}$\/ mais pas le mot $cabcb$.
	\item L'automate reconnaît les mots dont la 3\tsup{ème} lettre du mot, en partant de la fin, est un $a$.
\end{enumerate}



	}
	\def\addmacros#1{#1}
}
{
	\td[8]{Classe \textbf{P}, classe \textbf{NP}}
	\minitoc
	\renewcommand{\cwd}{../td/td08/}
	\addmacros{
		\section{Filtre RC double}

\begin{enumerate}
	\item En basse fréquence, un condensateur est équivalent à un interrupteur ouvert. En haute fréquence, un condensateur est équivalent à un interrupteur fermé. D'où, le circuit est un filtre passe-bas.
	\item Par une loi des nœuds, et une loi des mailles, on trouve que
		\[
			\ubar{H}(\mathrm{j}\omega) = \dfrac{1}{1 - \left( \dfrac{\omega}{\omega_0} \right)^2 + \mathrm{j} \dfrac{\omega}{Q\cdot \omega_0}}
		\] en notant $\omega_0 = 1 / {RC}$ et $Q = 1 / 3$
	\item On représente le diagramme de \textsc{Bode} du filtre dans la figure ci-dessous.
		\begin{figure}[H]
			\centering
			\includesvg[width=\linewidth]{figures/bode-1.svg}
			\caption{Diagramme de \textsc{Bode} du filtre (échelle logarithmique)}
		\end{figure}
	\item On calcule $\omega_0 \simeq 6\:\mathrm{rad/s}$, ce qui correspond à une fréquence de coupure de $1\:\mathrm{kHz}$. Le signal de sortie est donc \[
			s(t) = \frac{2E}{3\pi}\cdot \sin(\omega t)
		,\] on le représente sur la figure ci-dessous. En effet, on a un déphasage de $-\pi / 2$, et un gain valant $1 / 3$ à $\omega \simeq \omega_0$.
		\begin{figure}[H]
			\centering
			\includesvg[width=\linewidth]{figures/signal-1.svg}
			\caption{Signal résultant}
		\end{figure}
\end{enumerate}

		\section{Tri topologique}

\begin{figure}[H]
	\centering
	\tikzfig{ex2-q1}
	\caption{Exemple de graphe}
\end{figure}

\begin{enumerate}
	\item Dans le graphe ci-dessus, $a \to c \to b$\/ est un tri topologique mais pas un parcours.
	\item Dans le même graphe, $b \to a \to c$\/ est un parcours mais pas un tri topologique.
	\item Supposons que $L_1$\/ possède un prédécesseur, on le note $L_i$\/ où $i > 1$. Ainsi, $(L_i, L_1) \in A$\/ et donc $i < 1$, ce qui est absurde. De même pour le dernier.
	\item Il existe un tri topologique si, et seulement si le graphe est acyclique.
		\begin{itemize}
			\item[``$\implies$'']
				Soit $L_1,\ldots,L_n$\/ un tri topologique. Montrons que le graphe est acyclique.
				Par l'absurde, on suppose le graphe non acyclique : il existe $(i,j) \in \llbracket 1,n \rrbracket^2$\/ avec $i \neq j$\/ tels que $T_i \to \cdots \to T_j$\/ et $T_j \to \cdots \to T_i$ soient deux chemins valides. Ainsi, comme le tri est topologique et par récurrence, $i \le j$\/ et $j \le i$\/ et donc $i = j$, ce qui est absurde car $i$\/ et $j$\/ sont supposés différents. Le graphe est donc acyclique.
			\item[``$\impliedby$'']
				Soit $G$\/ un graphe tel que tous les sommets possèdent une arrête entrante. On suppose par l'absurde ce graphe acyclique.
				Soit $x_0$\/ un sommet du graphe.
				On construit par récurrence $x_0,x_1,\ldots,x_n,x_{n+1},\ldots$ les successeurs successifs. Il y a un nombre fini de sommets donc deux sommets sont identiques. Donc, il y a nécessairement un cycle, ce qui est absurde.
				\begin{algorithm}[H]
					\centering
					\begin{algorithmic}[1]
						\Entree $G = (S, A)$\/ un graphe acyclique
						\Sortie $\mathrm{Res}$\/ un tri topologique.
						\State $\mathrm{Res} \gets [\quad]$\/
						\While{$G \neq \O$}
							\State Soit $x$\/ un sommet de $G$\/ sans prédécesseur
							\State $G \gets \big(S \setminus \{x\}, A \cap (S \setminus \{x\})^2\big)$\/ 
							\State $\mathrm{Res} \gets \mathrm{Res} \cdot [x]$\/
						\EndWhile
						\State\Return $\mathrm{Res}$\/
					\end{algorithmic}
					\caption{Génération d'un tri topologique d'un graphe acyclique}
				\end{algorithm}
		\end{itemize}
	\item~
		\begin{algorithm}[H]
			\centering
			\begin{algorithmic}[1]
				\Entree $G = (S, A)$\/ un graphe
				\Sortie $\mathrm{Res}$\/ un tri topologique, ou un cycle
				\State $\mathrm{Res} \gets [\quad]$\/
				\While{$G \neq \O$}
					\If{il existe $x$ sans prédécesseurs}
						\State Soit $x$\/ un sommet de $G$\/ sans prédécesseur
						\State $G \gets \big(S \setminus \{x\}, A \cap (S \setminus \{x\})^2\big)$\/ 
						\State $\mathrm{Res} \gets \mathrm{Res} \cdot [x]$\/
					\Else
						\State Soit $x \in S$\/ 
						\State Soit $x \gets x_1 \gets x_2 \gets \cdots \gets x_i$\/ la suite des prédécesseurs
						\State\Return $x_i,x_{i+1},\ldots,x_i$, un cycle
					\EndIf
				\EndWhile
				\State\Return $\mathrm{Res}$\/
			\end{algorithmic}
			\caption{Génération d'un tri topologique d'un graphe}
		\end{algorithm}
	\item On utilise la représentation par liste d'adjacence, et on stocke le nombre de prédécesseurs que l'on décroit à chaque choix de sommet.
	\item On essaie de trouver un tri topologique, et on voit si l'on trouve un cycle.
\end{enumerate}

		\section{Formules duales}

\begin{enumerate}
	\item On définit par induction $(\cdot)^\star$\/ comme
		\begin{multicols}{3}
			\begin{itemize}
				\item $\top^\star = \bot$\/ ;
				\item $\bot^\star = \top$\/ ;
				\item $(G \lor H)^\star = G^\star \land H^\star$\/ ;
				\item $(G \land H)^\star = G^\star \lor H^\star$\/ ;
				\item $(\lnot G)^\star = \lnot G^\star$\/ ;
				\item $p^\star = p$.
			\end{itemize}
		\end{multicols}
	\item Soit $\rho \in \mathds{B}^{\mathcal{P}}$. Montrons, par induction, $P(H) : ``\left\llbracket H^\star \right\rrbracket^\rho = \left\llbracket \lnot H \right\rrbracket^{\bar \rho}"$\/ où $\bar{\rho} : p \mapsto \overline{\rho(p)}$.
		\begin{itemize}
			\item On a $\left\llbracket \bot^\star \right\rrbracket^\rho = \left\llbracket \top \right\rrbracket^\rho = \mathbf{V}$, et $\left\llbracket \lnot \bot \right\rrbracket^{\bar\rho} = \left\llbracket \top \right\rrbracket^{\bar\rho} = \mathbf{V}$, d'où $P(\bot)$.
			\item On a $\left\llbracket \top^\star \right\rrbracket^\rho = \left\llbracket \bot \right\rrbracket^\rho = \mathbf{F}$, et $\left\llbracket \lnot \top \right\rrbracket^{\bar\rho} = \left\llbracket \bot \right\rrbracket^{\bar\rho} = \mathbf{F}$, d'où $P(\top)$.
			\item Soit $p \in \mathcal{P}$. On a $\left\llbracket p^\star  \right\rrbracket^\rho = \left\llbracket p \right\rrbracket^\rho = \rho(p)$, et $\left\llbracket \lnot p \right\rrbracket^{\bar\rho} = \overline{\left\llbracket p \right\rrbracket^{\bar\rho}} = \overline{\bar\rho(p)} = \overline{\overline{\rho(p)}} = \rho(p)$, d'où~$P(p)$.
		\end{itemize}
		Soient $F$\/ et $G$\/ deux formules.
		\begin{itemize}
			\item On a
				\begin{align*}
					\left\llbracket (F \land G)^\star  \right\rrbracket^\rho &= \left\llbracket F^\star  \lor G^\star \right\rrbracket^\rho\\
					&= \left\llbracket F^\star \right\rrbracket^\rho + \left\llbracket G^\star \right\rrbracket^\rho \\
					&= \left\llbracket \lnot F \right\rrbracket^{\bar\rho} + \left\llbracket \lnot G \right\rrbracket^{\bar\rho} \\
					&= \left\llbracket \lnot F \lor \lnot G \right\rrbracket^{\bar\rho} \\
					&= \left\llbracket \lnot (F \land G) \right\rrbracket^{\bar\rho} \\
				\end{align*}
				d'où $P(F \land G)$.
			\item On a
				\begin{align*}
					\left\llbracket (F \lor G)^\star  \right\rrbracket^\rho &= \left\llbracket F^\star  \land G^\star \right\rrbracket^\rho\\
					&= \left\llbracket F^\star \right\rrbracket^\rho \cdot \left\llbracket G^\star \right\rrbracket^\rho \\
					&= \left\llbracket \lnot F \right\rrbracket^{\bar\rho} \cdot \left\llbracket \lnot G \right\rrbracket^{\bar\rho} \\
					&= \left\llbracket \lnot F \land \lnot G \right\rrbracket^{\bar\rho} \\
					&= \left\llbracket \lnot (F \lor G) \right\rrbracket^{\bar\rho} \\
				\end{align*}
				d'où $P(F \lor G)$.
			\item On a \[
					\left\llbracket (\lnot F)^\star  \right\rrbracket^\rho = \left\llbracket \lnot (F^\star) \right\rrbracket^\rho = \overline{\left\llbracket F^\star \right\rrbracket^\rho} = \overline{\left\llbracket \lnot F \right\rrbracket^{\bar\rho}} = \left\llbracket \lnot (\lnot F) \right\rrbracket^{\bar\rho}.
					\] d'où $P(\lnot F)$.
		\end{itemize}
		Par induction, on en conclut que $P(F)$\/ est vraie pour toute formule $F$.
	\item Soit $G$\/ une formule valide. Alors, par définition, $G \equiv \top$. Or, d'après la question précédente, $G^\star \equiv (\top)^\star = \bot$. Ainsi, $G^\star $\/ n'est pas satisfiable.
\end{enumerate}



		\section{Un lemme d'itération}

	}
	\def\addmacros#1{#1}
}
{
	\td[9]{Algorithmique des graphes}
	\minitoc
	\renewcommand{\cwd}{../td/td09/}
	\addmacros{
		\section{Filtre RC double}

\begin{enumerate}
	\item En basse fréquence, un condensateur est équivalent à un interrupteur ouvert. En haute fréquence, un condensateur est équivalent à un interrupteur fermé. D'où, le circuit est un filtre passe-bas.
	\item Par une loi des nœuds, et une loi des mailles, on trouve que
		\[
			\ubar{H}(\mathrm{j}\omega) = \dfrac{1}{1 - \left( \dfrac{\omega}{\omega_0} \right)^2 + \mathrm{j} \dfrac{\omega}{Q\cdot \omega_0}}
		\] en notant $\omega_0 = 1 / {RC}$ et $Q = 1 / 3$
	\item On représente le diagramme de \textsc{Bode} du filtre dans la figure ci-dessous.
		\begin{figure}[H]
			\centering
			\includesvg[width=\linewidth]{figures/bode-1.svg}
			\caption{Diagramme de \textsc{Bode} du filtre (échelle logarithmique)}
		\end{figure}
	\item On calcule $\omega_0 \simeq 6\:\mathrm{rad/s}$, ce qui correspond à une fréquence de coupure de $1\:\mathrm{kHz}$. Le signal de sortie est donc \[
			s(t) = \frac{2E}{3\pi}\cdot \sin(\omega t)
		,\] on le représente sur la figure ci-dessous. En effet, on a un déphasage de $-\pi / 2$, et un gain valant $1 / 3$ à $\omega \simeq \omega_0$.
		\begin{figure}[H]
			\centering
			\includesvg[width=\linewidth]{figures/signal-1.svg}
			\caption{Signal résultant}
		\end{figure}
\end{enumerate}

		\section{Tri topologique}

\begin{figure}[H]
	\centering
	\tikzfig{ex2-q1}
	\caption{Exemple de graphe}
\end{figure}

\begin{enumerate}
	\item Dans le graphe ci-dessus, $a \to c \to b$\/ est un tri topologique mais pas un parcours.
	\item Dans le même graphe, $b \to a \to c$\/ est un parcours mais pas un tri topologique.
	\item Supposons que $L_1$\/ possède un prédécesseur, on le note $L_i$\/ où $i > 1$. Ainsi, $(L_i, L_1) \in A$\/ et donc $i < 1$, ce qui est absurde. De même pour le dernier.
	\item Il existe un tri topologique si, et seulement si le graphe est acyclique.
		\begin{itemize}
			\item[``$\implies$'']
				Soit $L_1,\ldots,L_n$\/ un tri topologique. Montrons que le graphe est acyclique.
				Par l'absurde, on suppose le graphe non acyclique : il existe $(i,j) \in \llbracket 1,n \rrbracket^2$\/ avec $i \neq j$\/ tels que $T_i \to \cdots \to T_j$\/ et $T_j \to \cdots \to T_i$ soient deux chemins valides. Ainsi, comme le tri est topologique et par récurrence, $i \le j$\/ et $j \le i$\/ et donc $i = j$, ce qui est absurde car $i$\/ et $j$\/ sont supposés différents. Le graphe est donc acyclique.
			\item[``$\impliedby$'']
				Soit $G$\/ un graphe tel que tous les sommets possèdent une arrête entrante. On suppose par l'absurde ce graphe acyclique.
				Soit $x_0$\/ un sommet du graphe.
				On construit par récurrence $x_0,x_1,\ldots,x_n,x_{n+1},\ldots$ les successeurs successifs. Il y a un nombre fini de sommets donc deux sommets sont identiques. Donc, il y a nécessairement un cycle, ce qui est absurde.
				\begin{algorithm}[H]
					\centering
					\begin{algorithmic}[1]
						\Entree $G = (S, A)$\/ un graphe acyclique
						\Sortie $\mathrm{Res}$\/ un tri topologique.
						\State $\mathrm{Res} \gets [\quad]$\/
						\While{$G \neq \O$}
							\State Soit $x$\/ un sommet de $G$\/ sans prédécesseur
							\State $G \gets \big(S \setminus \{x\}, A \cap (S \setminus \{x\})^2\big)$\/ 
							\State $\mathrm{Res} \gets \mathrm{Res} \cdot [x]$\/
						\EndWhile
						\State\Return $\mathrm{Res}$\/
					\end{algorithmic}
					\caption{Génération d'un tri topologique d'un graphe acyclique}
				\end{algorithm}
		\end{itemize}
	\item~
		\begin{algorithm}[H]
			\centering
			\begin{algorithmic}[1]
				\Entree $G = (S, A)$\/ un graphe
				\Sortie $\mathrm{Res}$\/ un tri topologique, ou un cycle
				\State $\mathrm{Res} \gets [\quad]$\/
				\While{$G \neq \O$}
					\If{il existe $x$ sans prédécesseurs}
						\State Soit $x$\/ un sommet de $G$\/ sans prédécesseur
						\State $G \gets \big(S \setminus \{x\}, A \cap (S \setminus \{x\})^2\big)$\/ 
						\State $\mathrm{Res} \gets \mathrm{Res} \cdot [x]$\/
					\Else
						\State Soit $x \in S$\/ 
						\State Soit $x \gets x_1 \gets x_2 \gets \cdots \gets x_i$\/ la suite des prédécesseurs
						\State\Return $x_i,x_{i+1},\ldots,x_i$, un cycle
					\EndIf
				\EndWhile
				\State\Return $\mathrm{Res}$\/
			\end{algorithmic}
			\caption{Génération d'un tri topologique d'un graphe}
		\end{algorithm}
	\item On utilise la représentation par liste d'adjacence, et on stocke le nombre de prédécesseurs que l'on décroit à chaque choix de sommet.
	\item On essaie de trouver un tri topologique, et on voit si l'on trouve un cycle.
\end{enumerate}

		\section{Formules duales}

\begin{enumerate}
	\item On définit par induction $(\cdot)^\star$\/ comme
		\begin{multicols}{3}
			\begin{itemize}
				\item $\top^\star = \bot$\/ ;
				\item $\bot^\star = \top$\/ ;
				\item $(G \lor H)^\star = G^\star \land H^\star$\/ ;
				\item $(G \land H)^\star = G^\star \lor H^\star$\/ ;
				\item $(\lnot G)^\star = \lnot G^\star$\/ ;
				\item $p^\star = p$.
			\end{itemize}
		\end{multicols}
	\item Soit $\rho \in \mathds{B}^{\mathcal{P}}$. Montrons, par induction, $P(H) : ``\left\llbracket H^\star \right\rrbracket^\rho = \left\llbracket \lnot H \right\rrbracket^{\bar \rho}"$\/ où $\bar{\rho} : p \mapsto \overline{\rho(p)}$.
		\begin{itemize}
			\item On a $\left\llbracket \bot^\star \right\rrbracket^\rho = \left\llbracket \top \right\rrbracket^\rho = \mathbf{V}$, et $\left\llbracket \lnot \bot \right\rrbracket^{\bar\rho} = \left\llbracket \top \right\rrbracket^{\bar\rho} = \mathbf{V}$, d'où $P(\bot)$.
			\item On a $\left\llbracket \top^\star \right\rrbracket^\rho = \left\llbracket \bot \right\rrbracket^\rho = \mathbf{F}$, et $\left\llbracket \lnot \top \right\rrbracket^{\bar\rho} = \left\llbracket \bot \right\rrbracket^{\bar\rho} = \mathbf{F}$, d'où $P(\top)$.
			\item Soit $p \in \mathcal{P}$. On a $\left\llbracket p^\star  \right\rrbracket^\rho = \left\llbracket p \right\rrbracket^\rho = \rho(p)$, et $\left\llbracket \lnot p \right\rrbracket^{\bar\rho} = \overline{\left\llbracket p \right\rrbracket^{\bar\rho}} = \overline{\bar\rho(p)} = \overline{\overline{\rho(p)}} = \rho(p)$, d'où~$P(p)$.
		\end{itemize}
		Soient $F$\/ et $G$\/ deux formules.
		\begin{itemize}
			\item On a
				\begin{align*}
					\left\llbracket (F \land G)^\star  \right\rrbracket^\rho &= \left\llbracket F^\star  \lor G^\star \right\rrbracket^\rho\\
					&= \left\llbracket F^\star \right\rrbracket^\rho + \left\llbracket G^\star \right\rrbracket^\rho \\
					&= \left\llbracket \lnot F \right\rrbracket^{\bar\rho} + \left\llbracket \lnot G \right\rrbracket^{\bar\rho} \\
					&= \left\llbracket \lnot F \lor \lnot G \right\rrbracket^{\bar\rho} \\
					&= \left\llbracket \lnot (F \land G) \right\rrbracket^{\bar\rho} \\
				\end{align*}
				d'où $P(F \land G)$.
			\item On a
				\begin{align*}
					\left\llbracket (F \lor G)^\star  \right\rrbracket^\rho &= \left\llbracket F^\star  \land G^\star \right\rrbracket^\rho\\
					&= \left\llbracket F^\star \right\rrbracket^\rho \cdot \left\llbracket G^\star \right\rrbracket^\rho \\
					&= \left\llbracket \lnot F \right\rrbracket^{\bar\rho} \cdot \left\llbracket \lnot G \right\rrbracket^{\bar\rho} \\
					&= \left\llbracket \lnot F \land \lnot G \right\rrbracket^{\bar\rho} \\
					&= \left\llbracket \lnot (F \lor G) \right\rrbracket^{\bar\rho} \\
				\end{align*}
				d'où $P(F \lor G)$.
			\item On a \[
					\left\llbracket (\lnot F)^\star  \right\rrbracket^\rho = \left\llbracket \lnot (F^\star) \right\rrbracket^\rho = \overline{\left\llbracket F^\star \right\rrbracket^\rho} = \overline{\left\llbracket \lnot F \right\rrbracket^{\bar\rho}} = \left\llbracket \lnot (\lnot F) \right\rrbracket^{\bar\rho}.
					\] d'où $P(\lnot F)$.
		\end{itemize}
		Par induction, on en conclut que $P(F)$\/ est vraie pour toute formule $F$.
	\item Soit $G$\/ une formule valide. Alors, par définition, $G \equiv \top$. Or, d'après la question précédente, $G^\star \equiv (\top)^\star = \bot$. Ainsi, $G^\star $\/ n'est pas satisfiable.
\end{enumerate}



		\section{Un lemme d'itération}

		\section{Ambigüité}

	}
	\def\addmacros#1{#1}
}
{
	\td[10]{Preuves en logique propositionnelle}
	\minitoc
	\renewcommand{\cwd}{../td/td10/}
	\addmacros{
		\begin{landscape}
			\begin{multicols}{2}
				\section{Filtre RC double}

\begin{enumerate}
	\item En basse fréquence, un condensateur est équivalent à un interrupteur ouvert. En haute fréquence, un condensateur est équivalent à un interrupteur fermé. D'où, le circuit est un filtre passe-bas.
	\item Par une loi des nœuds, et une loi des mailles, on trouve que
		\[
			\ubar{H}(\mathrm{j}\omega) = \dfrac{1}{1 - \left( \dfrac{\omega}{\omega_0} \right)^2 + \mathrm{j} \dfrac{\omega}{Q\cdot \omega_0}}
		\] en notant $\omega_0 = 1 / {RC}$ et $Q = 1 / 3$
	\item On représente le diagramme de \textsc{Bode} du filtre dans la figure ci-dessous.
		\begin{figure}[H]
			\centering
			\includesvg[width=\linewidth]{figures/bode-1.svg}
			\caption{Diagramme de \textsc{Bode} du filtre (échelle logarithmique)}
		\end{figure}
	\item On calcule $\omega_0 \simeq 6\:\mathrm{rad/s}$, ce qui correspond à une fréquence de coupure de $1\:\mathrm{kHz}$. Le signal de sortie est donc \[
			s(t) = \frac{2E}{3\pi}\cdot \sin(\omega t)
		,\] on le représente sur la figure ci-dessous. En effet, on a un déphasage de $-\pi / 2$, et un gain valant $1 / 3$ à $\omega \simeq \omega_0$.
		\begin{figure}[H]
			\centering
			\includesvg[width=\linewidth]{figures/signal-1.svg}
			\caption{Signal résultant}
		\end{figure}
\end{enumerate}

				\section{Tri topologique}

\begin{figure}[H]
	\centering
	\tikzfig{ex2-q1}
	\caption{Exemple de graphe}
\end{figure}

\begin{enumerate}
	\item Dans le graphe ci-dessus, $a \to c \to b$\/ est un tri topologique mais pas un parcours.
	\item Dans le même graphe, $b \to a \to c$\/ est un parcours mais pas un tri topologique.
	\item Supposons que $L_1$\/ possède un prédécesseur, on le note $L_i$\/ où $i > 1$. Ainsi, $(L_i, L_1) \in A$\/ et donc $i < 1$, ce qui est absurde. De même pour le dernier.
	\item Il existe un tri topologique si, et seulement si le graphe est acyclique.
		\begin{itemize}
			\item[``$\implies$'']
				Soit $L_1,\ldots,L_n$\/ un tri topologique. Montrons que le graphe est acyclique.
				Par l'absurde, on suppose le graphe non acyclique : il existe $(i,j) \in \llbracket 1,n \rrbracket^2$\/ avec $i \neq j$\/ tels que $T_i \to \cdots \to T_j$\/ et $T_j \to \cdots \to T_i$ soient deux chemins valides. Ainsi, comme le tri est topologique et par récurrence, $i \le j$\/ et $j \le i$\/ et donc $i = j$, ce qui est absurde car $i$\/ et $j$\/ sont supposés différents. Le graphe est donc acyclique.
			\item[``$\impliedby$'']
				Soit $G$\/ un graphe tel que tous les sommets possèdent une arrête entrante. On suppose par l'absurde ce graphe acyclique.
				Soit $x_0$\/ un sommet du graphe.
				On construit par récurrence $x_0,x_1,\ldots,x_n,x_{n+1},\ldots$ les successeurs successifs. Il y a un nombre fini de sommets donc deux sommets sont identiques. Donc, il y a nécessairement un cycle, ce qui est absurde.
				\begin{algorithm}[H]
					\centering
					\begin{algorithmic}[1]
						\Entree $G = (S, A)$\/ un graphe acyclique
						\Sortie $\mathrm{Res}$\/ un tri topologique.
						\State $\mathrm{Res} \gets [\quad]$\/
						\While{$G \neq \O$}
							\State Soit $x$\/ un sommet de $G$\/ sans prédécesseur
							\State $G \gets \big(S \setminus \{x\}, A \cap (S \setminus \{x\})^2\big)$\/ 
							\State $\mathrm{Res} \gets \mathrm{Res} \cdot [x]$\/
						\EndWhile
						\State\Return $\mathrm{Res}$\/
					\end{algorithmic}
					\caption{Génération d'un tri topologique d'un graphe acyclique}
				\end{algorithm}
		\end{itemize}
	\item~
		\begin{algorithm}[H]
			\centering
			\begin{algorithmic}[1]
				\Entree $G = (S, A)$\/ un graphe
				\Sortie $\mathrm{Res}$\/ un tri topologique, ou un cycle
				\State $\mathrm{Res} \gets [\quad]$\/
				\While{$G \neq \O$}
					\If{il existe $x$ sans prédécesseurs}
						\State Soit $x$\/ un sommet de $G$\/ sans prédécesseur
						\State $G \gets \big(S \setminus \{x\}, A \cap (S \setminus \{x\})^2\big)$\/ 
						\State $\mathrm{Res} \gets \mathrm{Res} \cdot [x]$\/
					\Else
						\State Soit $x \in S$\/ 
						\State Soit $x \gets x_1 \gets x_2 \gets \cdots \gets x_i$\/ la suite des prédécesseurs
						\State\Return $x_i,x_{i+1},\ldots,x_i$, un cycle
					\EndIf
				\EndWhile
				\State\Return $\mathrm{Res}$\/
			\end{algorithmic}
			\caption{Génération d'un tri topologique d'un graphe}
		\end{algorithm}
	\item On utilise la représentation par liste d'adjacence, et on stocke le nombre de prédécesseurs que l'on décroit à chaque choix de sommet.
	\item On essaie de trouver un tri topologique, et on voit si l'on trouve un cycle.
\end{enumerate}

			\end{multicols}
			\section{Formules duales}

\begin{enumerate}
	\item On définit par induction $(\cdot)^\star$\/ comme
		\begin{multicols}{3}
			\begin{itemize}
				\item $\top^\star = \bot$\/ ;
				\item $\bot^\star = \top$\/ ;
				\item $(G \lor H)^\star = G^\star \land H^\star$\/ ;
				\item $(G \land H)^\star = G^\star \lor H^\star$\/ ;
				\item $(\lnot G)^\star = \lnot G^\star$\/ ;
				\item $p^\star = p$.
			\end{itemize}
		\end{multicols}
	\item Soit $\rho \in \mathds{B}^{\mathcal{P}}$. Montrons, par induction, $P(H) : ``\left\llbracket H^\star \right\rrbracket^\rho = \left\llbracket \lnot H \right\rrbracket^{\bar \rho}"$\/ où $\bar{\rho} : p \mapsto \overline{\rho(p)}$.
		\begin{itemize}
			\item On a $\left\llbracket \bot^\star \right\rrbracket^\rho = \left\llbracket \top \right\rrbracket^\rho = \mathbf{V}$, et $\left\llbracket \lnot \bot \right\rrbracket^{\bar\rho} = \left\llbracket \top \right\rrbracket^{\bar\rho} = \mathbf{V}$, d'où $P(\bot)$.
			\item On a $\left\llbracket \top^\star \right\rrbracket^\rho = \left\llbracket \bot \right\rrbracket^\rho = \mathbf{F}$, et $\left\llbracket \lnot \top \right\rrbracket^{\bar\rho} = \left\llbracket \bot \right\rrbracket^{\bar\rho} = \mathbf{F}$, d'où $P(\top)$.
			\item Soit $p \in \mathcal{P}$. On a $\left\llbracket p^\star  \right\rrbracket^\rho = \left\llbracket p \right\rrbracket^\rho = \rho(p)$, et $\left\llbracket \lnot p \right\rrbracket^{\bar\rho} = \overline{\left\llbracket p \right\rrbracket^{\bar\rho}} = \overline{\bar\rho(p)} = \overline{\overline{\rho(p)}} = \rho(p)$, d'où~$P(p)$.
		\end{itemize}
		Soient $F$\/ et $G$\/ deux formules.
		\begin{itemize}
			\item On a
				\begin{align*}
					\left\llbracket (F \land G)^\star  \right\rrbracket^\rho &= \left\llbracket F^\star  \lor G^\star \right\rrbracket^\rho\\
					&= \left\llbracket F^\star \right\rrbracket^\rho + \left\llbracket G^\star \right\rrbracket^\rho \\
					&= \left\llbracket \lnot F \right\rrbracket^{\bar\rho} + \left\llbracket \lnot G \right\rrbracket^{\bar\rho} \\
					&= \left\llbracket \lnot F \lor \lnot G \right\rrbracket^{\bar\rho} \\
					&= \left\llbracket \lnot (F \land G) \right\rrbracket^{\bar\rho} \\
				\end{align*}
				d'où $P(F \land G)$.
			\item On a
				\begin{align*}
					\left\llbracket (F \lor G)^\star  \right\rrbracket^\rho &= \left\llbracket F^\star  \land G^\star \right\rrbracket^\rho\\
					&= \left\llbracket F^\star \right\rrbracket^\rho \cdot \left\llbracket G^\star \right\rrbracket^\rho \\
					&= \left\llbracket \lnot F \right\rrbracket^{\bar\rho} \cdot \left\llbracket \lnot G \right\rrbracket^{\bar\rho} \\
					&= \left\llbracket \lnot F \land \lnot G \right\rrbracket^{\bar\rho} \\
					&= \left\llbracket \lnot (F \lor G) \right\rrbracket^{\bar\rho} \\
				\end{align*}
				d'où $P(F \lor G)$.
			\item On a \[
					\left\llbracket (\lnot F)^\star  \right\rrbracket^\rho = \left\llbracket \lnot (F^\star) \right\rrbracket^\rho = \overline{\left\llbracket F^\star \right\rrbracket^\rho} = \overline{\left\llbracket \lnot F \right\rrbracket^{\bar\rho}} = \left\llbracket \lnot (\lnot F) \right\rrbracket^{\bar\rho}.
					\] d'où $P(\lnot F)$.
		\end{itemize}
		Par induction, on en conclut que $P(F)$\/ est vraie pour toute formule $F$.
	\item Soit $G$\/ une formule valide. Alors, par définition, $G \equiv \top$. Or, d'après la question précédente, $G^\star \equiv (\top)^\star = \bot$. Ainsi, $G^\star $\/ n'est pas satisfiable.
\end{enumerate}



			\section{Un lemme d'itération}

			\section{Ambigüité}

			\section{Regexp Crossword}
\begin{center}
	\url{https://regexcrossword.com/}
\end{center}

			\documentclass[a4paper]{article}

\usepackage[margin=1in]{geometry}
\usepackage[utf8]{inputenc}
\usepackage[T1]{fontenc}
\usepackage{mathrsfs}
\usepackage{textcomp}
\usepackage[french]{babel}
\usepackage{amsmath}
\usepackage{amssymb}
\usepackage{cancel}
\usepackage{frcursive}
\usepackage[inline]{asymptote}
\usepackage{tikz}
\usepackage[european,straightvoltages,europeanresistors]{circuitikz}
\usepackage{tikz-cd}
\usepackage{tkz-tab}
\usepackage[b]{esvect}
\usepackage[framemethod=TikZ]{mdframed}
\usepackage{centernot}
\usepackage{diagbox}
\usepackage{dsfont}
\usepackage{fancyhdr}
\usepackage{float}
\usepackage{graphicx}
\usepackage{listings}
\usepackage{multicol}
\usepackage{nicematrix}
\usepackage{pdflscape}
\usepackage{stmaryrd}
\usepackage{xfrac}
\usepackage{hep-math-font}
\usepackage{amsthm}
\usepackage{thmtools}
\usepackage{indentfirst}
\usepackage[framemethod=TikZ]{mdframed}
\usepackage{accents}
\usepackage{soulutf8}
\usepackage{mathtools}
\usepackage{bodegraph}
\usepackage{slashbox}
\usepackage{enumitem}
\usepackage{calligra}
\usepackage{cinzel}
\usepackage{BOONDOX-calo}

% Tikz
\usetikzlibrary{babel}
\usetikzlibrary{positioning}
\usetikzlibrary{calc}

% global settings
\frenchspacing
\reversemarginpar
\setuldepth{a}

%\everymath{\displaystyle}

\frenchbsetup{StandardLists=true}

\def\asydir{asy}

%\sisetup{exponent-product=\cdot,output-decimal-marker={,},separate-uncertainty,range-phrase=\;à\;,locale=FR}

\setlength{\parskip}{1em}

\theoremstyle{definition}

% Changing math
\let\emptyset\varnothing
\let\ge\geqslant
\let\le\leqslant
\let\preceq\preccurlyeq
\let\succeq\succcurlyeq
\let\ds\displaystyle
\let\ts\textstyle

\newcommand{\C}{\mathds{C}}
\newcommand{\R}{\mathds{R}}
\newcommand{\Z}{\mathds{Z}}
\newcommand{\N}{\mathds{N}}
\newcommand{\Q}{\mathds{Q}}

\renewcommand{\O}{\emptyset}

\newcommand\ubar[1]{\underaccent{\bar}{#1}}

\renewcommand\Re{\expandafter\mathfrak{Re}}
\renewcommand\Im{\expandafter\mathfrak{Im}}

\let\slantedpartial\partial
\DeclareRobustCommand{\partial}{\text{\rotatebox[origin=t]{20}{\scalebox{0.95}[1]{$\slantedpartial$}}}\hspace{-1pt}}

% merging two maths characters w/ \charfusion
\makeatletter
\def\moverlay{\mathpalette\mov@rlay}
\def\mov@rlay#1#2{\leavevmode\vtop{%
   \baselineskip\z@skip \lineskiplimit-\maxdimen
   \ialign{\hfil$\m@th#1##$\hfil\cr#2\crcr}}}
\newcommand{\charfusion}[3][\mathord]{
    #1{\ifx#1\mathop\vphantom{#2}\fi
        \mathpalette\mov@rlay{#2\cr#3}
      }
    \ifx#1\mathop\expandafter\displaylimits\fi}
\makeatother

% custom math commands
\newcommand{\T}{{\!\!\,\top}}
\newcommand{\avrt}[1]{\rotatebox{-90}{$#1$}}
\newcommand{\bigcupdot}{\charfusion[\mathop]{\bigcup}{\cdot}}
\newcommand{\cupdot}{\charfusion[\mathbin]{\cup}{\cdot}}
%\newcommand{\danger}{{\large\fontencoding{U}\fontfamily{futs}\selectfont\char 66\relax}\;}
\newcommand{\tendsto}[1]{\xrightarrow[#1]{}}
\newcommand{\vrt}[1]{\rotatebox{90}{$#1$}}
\newcommand{\tsup}[1]{\textsuperscript{\underline{#1}}}
\newcommand{\tsub}[1]{\textsubscript{#1}}

\renewcommand{\mod}[1]{~\left[ #1 \right]}
\renewcommand{\t}{{}^t\!}
\newcommand{\s}{\text{\calligra s}}

% custom units / constants
%\DeclareSIUnit{\litre}{\ell}
\let\hbar\hslash

% header / footer
\pagestyle{fancy}
\fancyhead{} \fancyfoot{}
\fancyfoot[C]{\thepage}

% fonts
\let\sc\scshape
\let\bf\bfseries
\let\it\itshape
\let\sl\slshape

% custom math operators
\let\th\relax
\let\det\relax
\DeclareMathOperator*{\codim}{codim}
\DeclareMathOperator*{\dom}{dom}
\DeclareMathOperator*{\gO}{O}
\DeclareMathOperator*{\po}{\text{\cursive o}}
\DeclareMathOperator*{\sgn}{sgn}
\DeclareMathOperator*{\simi}{\sim}
\DeclareMathOperator{\Arccos}{Arccos}
\DeclareMathOperator{\Arcsin}{Arcsin}
\DeclareMathOperator{\Arctan}{Arctan}
\DeclareMathOperator{\Argsh}{Argsh}
\DeclareMathOperator{\Arg}{Arg}
\DeclareMathOperator{\Aut}{Aut}
\DeclareMathOperator{\Card}{Card}
\DeclareMathOperator{\Cl}{\mathcal{C}\!\ell}
\DeclareMathOperator{\Cov}{Cov}
\DeclareMathOperator{\Ker}{Ker}
\DeclareMathOperator{\Mat}{Mat}
\DeclareMathOperator{\PGCD}{PGCD}
\DeclareMathOperator{\PPCM}{PPCM}
\DeclareMathOperator{\Supp}{Supp}
\DeclareMathOperator{\Vect}{Vect}
\DeclareMathOperator{\argmax}{argmax}
\DeclareMathOperator{\argmin}{argmin}
\DeclareMathOperator{\ch}{ch}
\DeclareMathOperator{\com}{com}
\DeclareMathOperator{\cotan}{cotan}
\DeclareMathOperator{\det}{det}
\DeclareMathOperator{\id}{id}
\DeclareMathOperator{\rg}{rg}
\DeclareMathOperator{\rk}{rk}
\DeclareMathOperator{\sh}{sh}
\DeclareMathOperator{\th}{th}
\DeclareMathOperator{\tr}{tr}

% colors and page style
\definecolor{truewhite}{HTML}{ffffff}
\definecolor{white}{HTML}{faf4ed}
\definecolor{trueblack}{HTML}{000000}
\definecolor{black}{HTML}{575279}
\definecolor{mauve}{HTML}{907aa9}
\definecolor{blue}{HTML}{286983}
\definecolor{red}{HTML}{d7827e}
\definecolor{yellow}{HTML}{ea9d34}
\definecolor{gray}{HTML}{9893a5}
\definecolor{grey}{HTML}{9893a5}
\definecolor{green}{HTML}{a0d971}

\pagecolor{white}
\color{black}

\begin{asydef}
	settings.prc = false;
	settings.render=0;

	white = rgb("faf4ed");
	black = rgb("575279");
	blue = rgb("286983");
	red = rgb("d7827e");
	yellow = rgb("f6c177");
	orange = rgb("ea9d34");
	gray = rgb("9893a5");
	grey = rgb("9893a5");
	deepcyan = rgb("56949f");
	pink = rgb("b4637a");
	magenta = rgb("eb6f92");
	green = rgb("a0d971");
	purple = rgb("907aa9");

	defaultpen(black + fontsize(8pt));

	import three;
	currentlight = nolight;
\end{asydef}

% theorems, proofs, ...

\mdfsetup{skipabove=1em,skipbelow=1em, innertopmargin=6pt, innerbottommargin=6pt,}

\declaretheoremstyle[
	headfont=\normalfont\itshape,
	numbered=no,
	postheadspace=\newline,
	headpunct={:},
	qed=\qedsymbol]{demstyle}

\declaretheorem[style=demstyle, name=Démonstration]{dem}

\newcommand\veczero{\kern-1.2pt\vec{\kern1.2pt 0}} % \vec{0} looks weird since the `0' isn't italicized

\makeatletter
\renewcommand{\title}[2]{
	\AtBeginDocument{
		\begin{titlepage}
			\begin{center}
				\vspace{10cm}
				{\Large \sc Chapitre #1}\\
				\vspace{1cm}
				{\Huge \calligra #2}\\
				\vfill
				Hugo {\sc Salou} MPI${}^{\star}$\\
				{\small Dernière mise à jour le \@date }
			\end{center}
		\end{titlepage}
	}
}

\newcommand{\titletp}[4]{
	\AtBeginDocument{
		\begin{titlepage}
			\begin{center}
				\vspace{10cm}
				{\Large \sc tp #1}\\
				\vspace{1cm}
				{\Huge \textsc{\textit{#2}}}\\
				\vfill
				{#3}\textit{MPI}${}^{\star}$\\
			\end{center}
		\end{titlepage}
	}
	\fancyfoot{}\fancyhead{}
	\fancyfoot[R]{#4 \textit{MPI}${}^{\star}$}
	\fancyhead[C]{{\sc tp #1} : #2}
	\fancyhead[R]{\thepage}
}

\newcommand{\titletd}[2]{
	\AtBeginDocument{
		\begin{titlepage}
			\begin{center}
				\vspace{10cm}
				{\Large \sc td #1}\\
				\vspace{1cm}
				{\Huge \calligra #2}\\
				\vfill
				Hugo {\sc Salou} MPI${}^{\star}$\\
				{\small Dernière mise à jour le \@date }
			\end{center}
		\end{titlepage}
	}
}
\makeatother

\newcommand{\sign}{
	\null
	\vfill
	\begin{center}
		{
			\fontfamily{ccr}\selectfont
			\textit{\textbf{\.{\"i}}}
		}
	\end{center}
	\vfill
	\null
}

\renewcommand{\thefootnote}{\emph{\alph{footnote}}}

% figure support
\usepackage{import}
\usepackage{xifthen}
\pdfminorversion=7
\usepackage{pdfpages}
\usepackage{transparent}
\newcommand{\incfig}[1]{%
	\def\svgwidth{\columnwidth}
	\import{./figures/}{#1.pdf_tex}
}

\pdfsuppresswarningpagegroup=1
\ctikzset{tripoles/european not symbol=circle}

\newcommand{\missingpart}{{\large\color{red} Il manque quelque chose ici\ldots}}


\fancyhead[R]{Hugo {\sc Salou}\/ MPI}
\fancyhead[L]{TD\textsubscript2 -- Exercice 7}

\begin{document}
	\let\thesection\relax
	\section{Exercice 3}

{\bf Indication}\/ : pour la $G$, on applique la relation de {\sc Chasles}\/ : l'intégrale $\int_0^7 \mathrm{e}^{-x}\ln x~\mathrm{d}x$\/ converge si et seulement si $\int_0^1 \mathrm{e}^{-x}\ln x\mathrm{d}x$\/ converge et $\int_1^7\mathrm{e}^{-x}\ln x~\mathrm{d}x$\/ converge (qui n'est même pas impropre).

L'intégrale $H = \int_0^1 \frac{\mathrm{e}^{\sin t}}{t}~\mathrm{d}t$\/ est impropre en 0. On sait que $\sin t \tendsto{t\to 0} 0$, et donc, par continuité de la fonction $\exp$\/ en $0$, $\mathrm{e}^{\sin t}\tendsto{t\to 0}e^0 = 1$.
Ainsi, $\frac{\mathrm{e}^{\sin t}}{t} = \mathrm{e}^{\sin t} \times \frac{1}{t} \simi_{t\to 0} \frac{1}{t}$\/ qui ne change pas de signe. Or, $\int_0^1 \frac{1}{t}~\mathrm{d}t$\/ diverge, donc l'intégrale $H$\/ diverge.

L'intégrale $I = \int_{1}^{+\infty} \frac{\mathrm{e}^{\sin t}}{t}~\mathrm{d}t$\/ est impropre en $+\infty$. Par croissance de la fonction exponentielle, on a $\frac{\mathrm{e}^{\sin t}}{t} \ge \frac{\mathrm{e}^{-1}}{t} \ge 0$. Or, l'intégrale $\int_{1}^{+\infty} \frac{1}{t}~\mathrm{d}t$\/ diverge, donc l'intégrale diverge aussi.

L'intégrale $K$, est l'intégrale d'une fonction Gau\ss ienne, et elle est impropre en $+\infty$. On la \guillemotleft~découpe~\guillemotright\ : \[
	\int_{0}^{+\infty} \mathrm{e}^{-x^2}~\mathrm{d}x \text{ converge si et seulement si } \int_{0}^{1} \mathrm{e}^{-x^2}~\mathrm{d}x \text{ converge et } \int_{1}^{+\infty} \mathrm{e}^{-x^2}~\mathrm{d}x \text{ converge.}
\] L'intégrale $\int_{0}^{1} \mathrm{e}^{-x^2}~\mathrm{d}x$\/ n'est même pas impropre, elle converge donc. Et, pour $x \in [1,+\infty[$, on sait, comme $x^2 \ge x$, $0 \le \mathrm{e}^{-x^2} \le \mathrm{e}^{-x}$. Or, $\int_{1}^{+\infty} \mathrm{e}^{-x}~\mathrm{d}x$\/ converge donc $\int_{0}^{+\infty} \mathrm{e}^{-x^2}~\mathrm{d}x$\/ aussi.
On calculera la valeur de cette intégrale dans le {\sc td}\/ \guillemotleft~Intégrales paramétrées.~\guillemotright

Autre méthode pour déterminer la nature de $K$\/ : 
$\mathrm{e}^{-x^2} = \po(\mathrm{e}^{-x})$\/ car $\mathrm{e}^{-x^2} = \underbrace{\mathrm{e}^{-x^2 + x}}_{\to 0} \times \mathrm{e}^{-x}$, car $\mathrm{e}^{-x^2 + x} = \mathrm{e}^{-x^2 \left( 1 - \frac{1}{x} \right)}$\/ et $-x^2\left( 1 - \frac{1}{x} \right) \to -\infty \times 1$.
Et $\int_0^{+\infty} \mathrm{e}^{-x}~\mathrm{d}x$\/ converge donc $\int_0^{+\infty} \mathrm{e}^{-x^2}~\mathrm{d}x$\/ converge.

\begin{figure}[H]
	\centering
	\begin{asy}
		import graph;
		size(10cm);
		draw((-10, 0) -- (10, 0), Arrow(TeXHead));
		draw((0, -3) -- (0, 5), Arrow(TeXHead));
		real f(real x) {
			return 4*exp(-(x/4)^2);
		}
		draw(graph(f, -10, 10), magenta);
	\end{asy}
	\caption{Courbe Gau\ss ienne}
\end{figure}

L'intégrale $F = \int_{7}^{+\infty} \mathrm{e}^{-x}\ln x~\mathrm{d}x$\/ est impropre en $+\infty$. Attention : la fonction n'est pas \guillemotleft~{\color{red} faussement impropre en $+\infty$}.~\guillemotright\ Mais, on peut remarquer que \[
	\mathrm{e}^{-x} \ln x = \mathrm{e}^{-\frac{x}{2}} \underbrace{\mathrm{e}^{-\frac{x}{2}} \ln x}_{\tendsto{x\to +\infty} 0} = \po(\mathrm{e}^{-\frac{x}{2}})
.\] Or, $\int_{7}^{+\infty} \mathrm{e}^{-x}~\mathrm{d}x$\/ converge donc l'intégrale $F$\/ converge aussi.

L'intégrale $G = \int_{0}^{7} \mathrm{e}^{-x}\ln x~\mathrm{d}x$\/ est impropre en 0.
Or, $\mathrm{e}^{-x}\ln x \simi_{x\to 0} \ln x$\/ qui ne change pas de signe au voisinage de 0. Or, $\int_{0}^{7}  \ln x~\mathrm{d}x$\/ converge donc l'intégrale $G$\/ converge également.

L'intégrale $E = \int_{1}^{+\infty} \frac{\ln x}{\sqrt{x}}~\mathrm{d}x$\/ est impropre en $+\infty$. Or, $\forall x \ge \mathrm{e}$, $\frac{\ln(x)}{\sqrt{x}} \ge \frac{1}{\sqrt{x}} \ge 0$\/ converge.
Or, $\int_{1}^{+\infty}  \frac{1}{x^{\sfrac{1}{2}}}~\mathrm{d}x$\/ diverge d'après le critère de {\sc Riemann}\/ en $+\infty$\/ car $\frac{1}{2} < 1$.
D'où l'intégrale $D$\/ diverge.

Autre méthode : intégration par parties. On peut même arriver à calculer une primitive de ${\ln x}\:/{\sqrt{x}}$.

L'intégrale $D = \int_{0}^{1} \frac{\ln x}{\sqrt{x}}~\mathrm{d}x$\/ est impropre en 0. On peut remarque que \[
	0 \le -\frac{\ln x}{\sqrt{x}} = -\frac{x^{0{,}1} \ln x}{x^{0{,}6}} = \po\left( \frac{1}{x^{0{,}6}} \right) \quad\text{car}\quad x^{0{,}1} \ln x \tendsto{x\to 0} 0
\] par croissances comparées.
Or, $\int_{0}^{1} \frac{1}{x^{0{,}6}}~\mathrm{d}x$\/ converge d'après le critère de {\sc Riemann}. D'où $-D$\/ converge et donc $D$\/ converge.

L'intégrale $J = \int_{1}^{+\infty} \frac{\sin t}{\sqrt{t} + \sin t}~\mathrm{d}t$\/ est impropre en $+\infty$. On calcule
\[
	f(t) = \frac{\sin t}{\sqrt{t} + \sin t} = \frac{\sin t}{\sqrt{t}} \times \frac{1}{1+\frac{\sin t}{\sqrt{t}}}
\] et $\frac{\sin t}{\sqrt{t}} \tendsto{t\to +\infty} 0$. D'où \[
	\frac{1}{1+\frac{\sin t}{\sqrt{t}}} = 1  - \frac{\sin t}{\sqrt{t}} + \frac{\sin^2 t}{t} + \po\left( \frac{\sin^2 t}{t} \right)
\] et donc \[
	f(t) = \frac{\cos t}{\sqrt{t}} + \frac{\sin^2 t}{t} + \po\left( \frac{\sin^2 t}{t} \right)
.\]
L'intégrale $\int_{1}^{+\infty} \frac{\sin t}{\sqrt{t}}~\mathrm{d}t$\/ est impropre en $+\infty$.
Soit $x \ge 1$. On calcule avec une intégration par parties,
\[
	\int_{1}^{x} \sin t \times \frac{1}{\sqrt{t}}~\mathrm{d}t = \int_{1}^{x} u'(t)\cdot v(t)~\mathrm{d}t
\] où $u(t) = - \cos t$\/ et $v(t) = \frac{1}{\sqrt{t}} = t^{-\frac{1}{2}}$. Donc
\begin{align*}
	\int_{1}^{x} \frac{\sin t}{\sqrt{t}}~\mathrm{d}t &= \Big[f(t)g(t)\Big]_1^x - \int_{1}^{x} f(t)\cdot g'(t)~\mathrm{d}t\\
	&= \left[ - \frac{\cos t}{\sqrt{t}} \right]_1^x - \int_{1}^{x} (-\cos t)\left( -\frac{1}{2}t^{-\frac{3}{2}} \right)~\mathrm{d}t \\
\end{align*}
D'où \[
	\int_{1}^{x} \frac{\sin t}{\sqrt{t}}~\mathrm{d}t = \cos 1 - \frac{\cos x}{\sqrt{x}} - \frac{1}{2} \int_{1}^{x} \frac{\cos t}{t^{\sfrac{3}{2}}}~\mathrm{d}t
.\]
Or, d'une part $\cos x \times \frac{1}{\sqrt{x}} \tendsto{x\to +\infty} 0$\/ car $\cos$\/ est bornée et $\frac{1}{\sqrt{x}}\tendsto{x\to +\infty} 0$.
Et, d'autre part $\int_{1}^{+\infty} \frac{\cos t}{t^{\sfrac{3}{2}}}~\mathrm{d}t$\/ converge car $\forall t \in [1,+\infty[$, $\left| \frac{\cos t}{t^{\sfrac{3}{2}}} \right| \le \frac{1}{t^{\sfrac{3}{2}}}$\/ et $\int_{1}^{+\infty} \frac{1}{t^{\sfrac{3}{2}}}~\mathrm{d}t$\/ converge.
Pour le 2\tsup{nd} terme du développement limité, on fait une {\sc ipp}, on trouve un terme en $\frac{1}{t^2}$\/ et donc son intégrale converge par critère de {\sc Riemann}. S'il y a des problèmes, voir en {\sc td}.
On étudie maintenant le 3\tsup{ème} terme : \[
	\int_{1}^{+\infty}  \po\left( \frac{\sin t}{t} \right) ~\mathrm{d}t \text{ converge car } \int_{1}^{+\infty} \frac{\sin^2 t}{t} ~\mathrm{d}t \text{ converge et } t\mapsto \frac{\sin^2 t}{t} \text{ est positive}.
\]
Autre méthode : on a \[
	\frac{\sin^2 t}{t} + \po\left( \frac{\sin^2 t}{t} \right) \simi_{t\to +\infty} \frac{\sin^2 t}{t} \text{ qui ne change pas de signe}
.\] Or, $\int_{1}^{+\infty} \frac{\sin^2 t}{t}~\mathrm{d}t$\/ converge et donc \[
	\int_{1}^{+\infty} \left( \frac{\sin^2 t}{t} + \po\left( \frac{\sin^2 t}{t} \right) \right) ~\mathrm{d}t
.\]

\end{document}

			\section{Vocabulaire des automates}

On représente, ci-dessous, l'automate $\mathcal{A}$\/ décrit dans l'énoncé.
\begin{figure}[H]
	\centering
	\tikzfig{automate-ex8}
	\caption{Automate décrit dans l'énoncé de l'exercice 8}
\end{figure}

\begin{enumerate}
	\item Cet automate n'est pas complet : à l'état 0, la lecture d'un $a$\/ peut conduire à l'état 0 ou bien à l'état 1.
	\item Le mot $baba$\/ est reconnu par $\mathcal{A}$\/ mais pas le mot $cabcb$.
	\item L'automate reconnaît les mots dont la 3\tsup{ème} lettre du mot, en partant de la fin, est un $a$.
\end{enumerate}



			\section{Complétion d'automate}

\begin{enumerate}
	\item Non, cet automate n'est pas complet. Par exemple, la lecture d'un $b$\/ à l'état 1 est impossible.
	\item Cet automate reconnaît le langage $L = \mathcal{L}\big(a \cdot b\cdot (a \mid b)^*\big)$.
	\item~

		\begin{figure}[H]
			\centering
			\tikzfig{automate-ex9}
			\caption{Automate complet équivalent à $\mathcal{A}$}
		\end{figure}
\end{enumerate}



			\section{Exercice 10}

\begin{enumerate}
	\item On a $u_6 \leadsto \hat{v}_\mathrm{e}$, $u_3 \leadsto \hat{v}_\mathrm{c}$\/ (composante continue), $u_2 \leadsto \hat{v}_\mathrm{d}$\/ (discontinuités), $u_4 \leadsto \hat{v}_\mathrm{a}$\/ (composante continue), $u_5 \leadsto \hat{v}_\mathrm{f}$\/ (nombre de fréquences) et $u_1 \leadsto \hat{v}_\mathrm{b}$.
	\item
		Le filtre donnant le signal $\hat{v}_\mathrm{g}$\/ est un passe-bandes (les hautes et basses fréquences sont éliminées) et c'est un filtre non linéaire (de nouvelles fréquences apparaissent).

		Le filtre donnant le signal $\hat{v}_\mathrm{h}$\/ est un filtre passe-bas.

		Le filtre donnant le signal $\hat{v}_\mathrm{i}$\/ est un filtre passe-haut dont sa fréquence de coupure est inférieure à $1\:\mathrm{kHz}$.
\end{enumerate}


			\section{Déterminisation 1}

\begin{enumerate}
	\item \tikzfig{ex11-1}
	\item \tikzfig{ex11-2}
\end{enumerate}

			\section{Déterminisation 2}

\begin{enumerate}
	\item \tikzfig{ex12-1}
	\item \tikzfig{ex12-2}
	\item \tikzfig{ex12-3}
\end{enumerate}

			\section{Exercice 14}

\begin{enumerate}
	\item Grâce aux impédances d'entrée infinies, on a $u_r = r i_D(t)$\/ et $u_1 = K_\text{m}\,r\,i_D(t)\,u_D(t)$. C'est un Wattmètre car le produit $i_D(t) \times u_D(t)$\/ correspond à la puissance reçue par $D$ (car elle est positive). Ce Wattmètre est analogique car $i_D(t)$\/ et $u_D(t)$\/ sont des grandeurs qui varient dans un interval continu.

	\item On a $\omega_\text{c} = \frac{1}{RC}$\/ d'où $f_\text{c} = \frac{1}{2\pi\,R\,C}\circeq \frac{1}{0{,}6}\:\mathrm{Hz} \circeq 2\:\mathrm{Hz}$.

		La fonction de ce circuit est de calculer la moyenne d'un signal (c'est donc un moyenneur) : il filtre les fréquences supérieures à quelques $\:\mathrm{Hz}$. On n'obtient donc que la composante continue comme montré sur la figure ci-dessous.

		\begin{figure}[H]
			\centering
			\begin{asy}
				import graph;
				size(7cm);
				real f(real x) { return (-atan((x - 2)*3)/pi + 1/2)*7; }
				draw((-1, 0) -- (15, 0), Arrow(TeXHead));
				draw((0, -1) -- (0, 8), Arrow(TeXHead));
				draw(graph(f, 0, 15), green);
				draw((5,0)--(5,5), orange);
				draw((0,0)--(0,5), orange);
				dot((5,5), orange);
				dot((0,5), orange);
			\end{asy}
			\caption{Moyenneur}
		\end{figure}
		
		{\color{red} On ne passe pas par les complexes} mais on revient aux définitions réelles d'un signal sinusoïdal :
		\begin{align*}
			u_1(t) &= K_\text{m} r\,i_D(t)\,u_D(t) \\
			&= K_\text{m}\,r\,I_D\cos(2\pi f + \varphi_i)\,U_D \cos(2\pi f t + \varphi_u) \\
			&= K_\text{m}\,r\,I_D\,U_D \times \frac{1}{2}\Big(\cos(2\pi\times 2f t + \varphi_i + \varphi_u) - \cos(\varphi_u - \varphi_i)\Big). \\
		\end{align*}
		On en déduit donc que \[
			u_s(t) \sim K_\text{m}\,r\,\underbrace{\frac{1}{2}\,U_D\,I_D\cos(\varphi_u - \varphi_i)}_{\langle p(t)\rangle}
		.\]
	\item
		\begin{itemize}
			\item Si $D = R$, on a
				\[
					\langle p(t) \rangle = \frac{1}{2}\,U_D\,I_D \cos(\overbrace{\varphi_u - \varphi_i}^{0}) = \frac{1}{2} R I_D^2 = \frac{U_D^2}{2R}
				.\] 
			\item Si $D = C$, on a $\ubar{u}_C = \ubar{Z}_C \ubar{i}_C = \frac{1}{\mathrm{j}C\omega} \ubar{i}_C$. Or, $\Arg(\ubar{Z}_C) = \Arg\left( \frac{\ubar{u}}{\ubar{i}} \right) = \Arg(\ubar{u}) - \Arg(\ubar{i}) = \varphi_u - \varphi_i = -\frac{\pi}{2}$. On en conclut donc que $\langle p(t) \rangle = 0$.
			\item Si $D = L$, on a $\Arg(\ubar{Z}_L) = \varphi_u - \varphi_i = \frac{\pi}{2}$\/ et donc $\langle p(t) \rangle  = 0$.
		\end{itemize}
\end{enumerate}

		\end{landscape}
	}
	\def\addmacros#1{#1}
}
{
	\td[11]{Preuves}
	\minitoc
	\renewcommand{\cwd}{../td/td11/}
	\addmacros{
		\section{Filtre RC double}

\begin{enumerate}
	\item En basse fréquence, un condensateur est équivalent à un interrupteur ouvert. En haute fréquence, un condensateur est équivalent à un interrupteur fermé. D'où, le circuit est un filtre passe-bas.
	\item Par une loi des nœuds, et une loi des mailles, on trouve que
		\[
			\ubar{H}(\mathrm{j}\omega) = \dfrac{1}{1 - \left( \dfrac{\omega}{\omega_0} \right)^2 + \mathrm{j} \dfrac{\omega}{Q\cdot \omega_0}}
		\] en notant $\omega_0 = 1 / {RC}$ et $Q = 1 / 3$
	\item On représente le diagramme de \textsc{Bode} du filtre dans la figure ci-dessous.
		\begin{figure}[H]
			\centering
			\includesvg[width=\linewidth]{figures/bode-1.svg}
			\caption{Diagramme de \textsc{Bode} du filtre (échelle logarithmique)}
		\end{figure}
	\item On calcule $\omega_0 \simeq 6\:\mathrm{rad/s}$, ce qui correspond à une fréquence de coupure de $1\:\mathrm{kHz}$. Le signal de sortie est donc \[
			s(t) = \frac{2E}{3\pi}\cdot \sin(\omega t)
		,\] on le représente sur la figure ci-dessous. En effet, on a un déphasage de $-\pi / 2$, et un gain valant $1 / 3$ à $\omega \simeq \omega_0$.
		\begin{figure}[H]
			\centering
			\includesvg[width=\linewidth]{figures/signal-1.svg}
			\caption{Signal résultant}
		\end{figure}
\end{enumerate}

		\section{Tri topologique}

\begin{figure}[H]
	\centering
	\tikzfig{ex2-q1}
	\caption{Exemple de graphe}
\end{figure}

\begin{enumerate}
	\item Dans le graphe ci-dessus, $a \to c \to b$\/ est un tri topologique mais pas un parcours.
	\item Dans le même graphe, $b \to a \to c$\/ est un parcours mais pas un tri topologique.
	\item Supposons que $L_1$\/ possède un prédécesseur, on le note $L_i$\/ où $i > 1$. Ainsi, $(L_i, L_1) \in A$\/ et donc $i < 1$, ce qui est absurde. De même pour le dernier.
	\item Il existe un tri topologique si, et seulement si le graphe est acyclique.
		\begin{itemize}
			\item[``$\implies$'']
				Soit $L_1,\ldots,L_n$\/ un tri topologique. Montrons que le graphe est acyclique.
				Par l'absurde, on suppose le graphe non acyclique : il existe $(i,j) \in \llbracket 1,n \rrbracket^2$\/ avec $i \neq j$\/ tels que $T_i \to \cdots \to T_j$\/ et $T_j \to \cdots \to T_i$ soient deux chemins valides. Ainsi, comme le tri est topologique et par récurrence, $i \le j$\/ et $j \le i$\/ et donc $i = j$, ce qui est absurde car $i$\/ et $j$\/ sont supposés différents. Le graphe est donc acyclique.
			\item[``$\impliedby$'']
				Soit $G$\/ un graphe tel que tous les sommets possèdent une arrête entrante. On suppose par l'absurde ce graphe acyclique.
				Soit $x_0$\/ un sommet du graphe.
				On construit par récurrence $x_0,x_1,\ldots,x_n,x_{n+1},\ldots$ les successeurs successifs. Il y a un nombre fini de sommets donc deux sommets sont identiques. Donc, il y a nécessairement un cycle, ce qui est absurde.
				\begin{algorithm}[H]
					\centering
					\begin{algorithmic}[1]
						\Entree $G = (S, A)$\/ un graphe acyclique
						\Sortie $\mathrm{Res}$\/ un tri topologique.
						\State $\mathrm{Res} \gets [\quad]$\/
						\While{$G \neq \O$}
							\State Soit $x$\/ un sommet de $G$\/ sans prédécesseur
							\State $G \gets \big(S \setminus \{x\}, A \cap (S \setminus \{x\})^2\big)$\/ 
							\State $\mathrm{Res} \gets \mathrm{Res} \cdot [x]$\/
						\EndWhile
						\State\Return $\mathrm{Res}$\/
					\end{algorithmic}
					\caption{Génération d'un tri topologique d'un graphe acyclique}
				\end{algorithm}
		\end{itemize}
	\item~
		\begin{algorithm}[H]
			\centering
			\begin{algorithmic}[1]
				\Entree $G = (S, A)$\/ un graphe
				\Sortie $\mathrm{Res}$\/ un tri topologique, ou un cycle
				\State $\mathrm{Res} \gets [\quad]$\/
				\While{$G \neq \O$}
					\If{il existe $x$ sans prédécesseurs}
						\State Soit $x$\/ un sommet de $G$\/ sans prédécesseur
						\State $G \gets \big(S \setminus \{x\}, A \cap (S \setminus \{x\})^2\big)$\/ 
						\State $\mathrm{Res} \gets \mathrm{Res} \cdot [x]$\/
					\Else
						\State Soit $x \in S$\/ 
						\State Soit $x \gets x_1 \gets x_2 \gets \cdots \gets x_i$\/ la suite des prédécesseurs
						\State\Return $x_i,x_{i+1},\ldots,x_i$, un cycle
					\EndIf
				\EndWhile
				\State\Return $\mathrm{Res}$\/
			\end{algorithmic}
			\caption{Génération d'un tri topologique d'un graphe}
		\end{algorithm}
	\item On utilise la représentation par liste d'adjacence, et on stocke le nombre de prédécesseurs que l'on décroit à chaque choix de sommet.
	\item On essaie de trouver un tri topologique, et on voit si l'on trouve un cycle.
\end{enumerate}

		\section{Formules duales}

\begin{enumerate}
	\item On définit par induction $(\cdot)^\star$\/ comme
		\begin{multicols}{3}
			\begin{itemize}
				\item $\top^\star = \bot$\/ ;
				\item $\bot^\star = \top$\/ ;
				\item $(G \lor H)^\star = G^\star \land H^\star$\/ ;
				\item $(G \land H)^\star = G^\star \lor H^\star$\/ ;
				\item $(\lnot G)^\star = \lnot G^\star$\/ ;
				\item $p^\star = p$.
			\end{itemize}
		\end{multicols}
	\item Soit $\rho \in \mathds{B}^{\mathcal{P}}$. Montrons, par induction, $P(H) : ``\left\llbracket H^\star \right\rrbracket^\rho = \left\llbracket \lnot H \right\rrbracket^{\bar \rho}"$\/ où $\bar{\rho} : p \mapsto \overline{\rho(p)}$.
		\begin{itemize}
			\item On a $\left\llbracket \bot^\star \right\rrbracket^\rho = \left\llbracket \top \right\rrbracket^\rho = \mathbf{V}$, et $\left\llbracket \lnot \bot \right\rrbracket^{\bar\rho} = \left\llbracket \top \right\rrbracket^{\bar\rho} = \mathbf{V}$, d'où $P(\bot)$.
			\item On a $\left\llbracket \top^\star \right\rrbracket^\rho = \left\llbracket \bot \right\rrbracket^\rho = \mathbf{F}$, et $\left\llbracket \lnot \top \right\rrbracket^{\bar\rho} = \left\llbracket \bot \right\rrbracket^{\bar\rho} = \mathbf{F}$, d'où $P(\top)$.
			\item Soit $p \in \mathcal{P}$. On a $\left\llbracket p^\star  \right\rrbracket^\rho = \left\llbracket p \right\rrbracket^\rho = \rho(p)$, et $\left\llbracket \lnot p \right\rrbracket^{\bar\rho} = \overline{\left\llbracket p \right\rrbracket^{\bar\rho}} = \overline{\bar\rho(p)} = \overline{\overline{\rho(p)}} = \rho(p)$, d'où~$P(p)$.
		\end{itemize}
		Soient $F$\/ et $G$\/ deux formules.
		\begin{itemize}
			\item On a
				\begin{align*}
					\left\llbracket (F \land G)^\star  \right\rrbracket^\rho &= \left\llbracket F^\star  \lor G^\star \right\rrbracket^\rho\\
					&= \left\llbracket F^\star \right\rrbracket^\rho + \left\llbracket G^\star \right\rrbracket^\rho \\
					&= \left\llbracket \lnot F \right\rrbracket^{\bar\rho} + \left\llbracket \lnot G \right\rrbracket^{\bar\rho} \\
					&= \left\llbracket \lnot F \lor \lnot G \right\rrbracket^{\bar\rho} \\
					&= \left\llbracket \lnot (F \land G) \right\rrbracket^{\bar\rho} \\
				\end{align*}
				d'où $P(F \land G)$.
			\item On a
				\begin{align*}
					\left\llbracket (F \lor G)^\star  \right\rrbracket^\rho &= \left\llbracket F^\star  \land G^\star \right\rrbracket^\rho\\
					&= \left\llbracket F^\star \right\rrbracket^\rho \cdot \left\llbracket G^\star \right\rrbracket^\rho \\
					&= \left\llbracket \lnot F \right\rrbracket^{\bar\rho} \cdot \left\llbracket \lnot G \right\rrbracket^{\bar\rho} \\
					&= \left\llbracket \lnot F \land \lnot G \right\rrbracket^{\bar\rho} \\
					&= \left\llbracket \lnot (F \lor G) \right\rrbracket^{\bar\rho} \\
				\end{align*}
				d'où $P(F \lor G)$.
			\item On a \[
					\left\llbracket (\lnot F)^\star  \right\rrbracket^\rho = \left\llbracket \lnot (F^\star) \right\rrbracket^\rho = \overline{\left\llbracket F^\star \right\rrbracket^\rho} = \overline{\left\llbracket \lnot F \right\rrbracket^{\bar\rho}} = \left\llbracket \lnot (\lnot F) \right\rrbracket^{\bar\rho}.
					\] d'où $P(\lnot F)$.
		\end{itemize}
		Par induction, on en conclut que $P(F)$\/ est vraie pour toute formule $F$.
	\item Soit $G$\/ une formule valide. Alors, par définition, $G \equiv \top$. Or, d'après la question précédente, $G^\star \equiv (\top)^\star = \bot$. Ainsi, $G^\star $\/ n'est pas satisfiable.
\end{enumerate}



		\section{Un lemme d'itération}

		\section{Ambigüité}

		\section{Regexp Crossword}
\begin{center}
	\url{https://regexcrossword.com/}
\end{center}

		\documentclass[a4paper]{article}

\usepackage[margin=1in]{geometry}
\usepackage[utf8]{inputenc}
\usepackage[T1]{fontenc}
\usepackage{mathrsfs}
\usepackage{textcomp}
\usepackage[french]{babel}
\usepackage{amsmath}
\usepackage{amssymb}
\usepackage{cancel}
\usepackage{frcursive}
\usepackage[inline]{asymptote}
\usepackage{tikz}
\usepackage[european,straightvoltages,europeanresistors]{circuitikz}
\usepackage{tikz-cd}
\usepackage{tkz-tab}
\usepackage[b]{esvect}
\usepackage[framemethod=TikZ]{mdframed}
\usepackage{centernot}
\usepackage{diagbox}
\usepackage{dsfont}
\usepackage{fancyhdr}
\usepackage{float}
\usepackage{graphicx}
\usepackage{listings}
\usepackage{multicol}
\usepackage{nicematrix}
\usepackage{pdflscape}
\usepackage{stmaryrd}
\usepackage{xfrac}
\usepackage{hep-math-font}
\usepackage{amsthm}
\usepackage{thmtools}
\usepackage{indentfirst}
\usepackage[framemethod=TikZ]{mdframed}
\usepackage{accents}
\usepackage{soulutf8}
\usepackage{mathtools}
\usepackage{bodegraph}
\usepackage{slashbox}
\usepackage{enumitem}
\usepackage{calligra}
\usepackage{cinzel}
\usepackage{BOONDOX-calo}

% Tikz
\usetikzlibrary{babel}
\usetikzlibrary{positioning}
\usetikzlibrary{calc}

% global settings
\frenchspacing
\reversemarginpar
\setuldepth{a}

%\everymath{\displaystyle}

\frenchbsetup{StandardLists=true}

\def\asydir{asy}

%\sisetup{exponent-product=\cdot,output-decimal-marker={,},separate-uncertainty,range-phrase=\;à\;,locale=FR}

\setlength{\parskip}{1em}

\theoremstyle{definition}

% Changing math
\let\emptyset\varnothing
\let\ge\geqslant
\let\le\leqslant
\let\preceq\preccurlyeq
\let\succeq\succcurlyeq
\let\ds\displaystyle
\let\ts\textstyle

\newcommand{\C}{\mathds{C}}
\newcommand{\R}{\mathds{R}}
\newcommand{\Z}{\mathds{Z}}
\newcommand{\N}{\mathds{N}}
\newcommand{\Q}{\mathds{Q}}

\renewcommand{\O}{\emptyset}

\newcommand\ubar[1]{\underaccent{\bar}{#1}}

\renewcommand\Re{\expandafter\mathfrak{Re}}
\renewcommand\Im{\expandafter\mathfrak{Im}}

\let\slantedpartial\partial
\DeclareRobustCommand{\partial}{\text{\rotatebox[origin=t]{20}{\scalebox{0.95}[1]{$\slantedpartial$}}}\hspace{-1pt}}

% merging two maths characters w/ \charfusion
\makeatletter
\def\moverlay{\mathpalette\mov@rlay}
\def\mov@rlay#1#2{\leavevmode\vtop{%
   \baselineskip\z@skip \lineskiplimit-\maxdimen
   \ialign{\hfil$\m@th#1##$\hfil\cr#2\crcr}}}
\newcommand{\charfusion}[3][\mathord]{
    #1{\ifx#1\mathop\vphantom{#2}\fi
        \mathpalette\mov@rlay{#2\cr#3}
      }
    \ifx#1\mathop\expandafter\displaylimits\fi}
\makeatother

% custom math commands
\newcommand{\T}{{\!\!\,\top}}
\newcommand{\avrt}[1]{\rotatebox{-90}{$#1$}}
\newcommand{\bigcupdot}{\charfusion[\mathop]{\bigcup}{\cdot}}
\newcommand{\cupdot}{\charfusion[\mathbin]{\cup}{\cdot}}
%\newcommand{\danger}{{\large\fontencoding{U}\fontfamily{futs}\selectfont\char 66\relax}\;}
\newcommand{\tendsto}[1]{\xrightarrow[#1]{}}
\newcommand{\vrt}[1]{\rotatebox{90}{$#1$}}
\newcommand{\tsup}[1]{\textsuperscript{\underline{#1}}}
\newcommand{\tsub}[1]{\textsubscript{#1}}

\renewcommand{\mod}[1]{~\left[ #1 \right]}
\renewcommand{\t}{{}^t\!}
\newcommand{\s}{\text{\calligra s}}

% custom units / constants
%\DeclareSIUnit{\litre}{\ell}
\let\hbar\hslash

% header / footer
\pagestyle{fancy}
\fancyhead{} \fancyfoot{}
\fancyfoot[C]{\thepage}

% fonts
\let\sc\scshape
\let\bf\bfseries
\let\it\itshape
\let\sl\slshape

% custom math operators
\let\th\relax
\let\det\relax
\DeclareMathOperator*{\codim}{codim}
\DeclareMathOperator*{\dom}{dom}
\DeclareMathOperator*{\gO}{O}
\DeclareMathOperator*{\po}{\text{\cursive o}}
\DeclareMathOperator*{\sgn}{sgn}
\DeclareMathOperator*{\simi}{\sim}
\DeclareMathOperator{\Arccos}{Arccos}
\DeclareMathOperator{\Arcsin}{Arcsin}
\DeclareMathOperator{\Arctan}{Arctan}
\DeclareMathOperator{\Argsh}{Argsh}
\DeclareMathOperator{\Arg}{Arg}
\DeclareMathOperator{\Aut}{Aut}
\DeclareMathOperator{\Card}{Card}
\DeclareMathOperator{\Cl}{\mathcal{C}\!\ell}
\DeclareMathOperator{\Cov}{Cov}
\DeclareMathOperator{\Ker}{Ker}
\DeclareMathOperator{\Mat}{Mat}
\DeclareMathOperator{\PGCD}{PGCD}
\DeclareMathOperator{\PPCM}{PPCM}
\DeclareMathOperator{\Supp}{Supp}
\DeclareMathOperator{\Vect}{Vect}
\DeclareMathOperator{\argmax}{argmax}
\DeclareMathOperator{\argmin}{argmin}
\DeclareMathOperator{\ch}{ch}
\DeclareMathOperator{\com}{com}
\DeclareMathOperator{\cotan}{cotan}
\DeclareMathOperator{\det}{det}
\DeclareMathOperator{\id}{id}
\DeclareMathOperator{\rg}{rg}
\DeclareMathOperator{\rk}{rk}
\DeclareMathOperator{\sh}{sh}
\DeclareMathOperator{\th}{th}
\DeclareMathOperator{\tr}{tr}

% colors and page style
\definecolor{truewhite}{HTML}{ffffff}
\definecolor{white}{HTML}{faf4ed}
\definecolor{trueblack}{HTML}{000000}
\definecolor{black}{HTML}{575279}
\definecolor{mauve}{HTML}{907aa9}
\definecolor{blue}{HTML}{286983}
\definecolor{red}{HTML}{d7827e}
\definecolor{yellow}{HTML}{ea9d34}
\definecolor{gray}{HTML}{9893a5}
\definecolor{grey}{HTML}{9893a5}
\definecolor{green}{HTML}{a0d971}

\pagecolor{white}
\color{black}

\begin{asydef}
	settings.prc = false;
	settings.render=0;

	white = rgb("faf4ed");
	black = rgb("575279");
	blue = rgb("286983");
	red = rgb("d7827e");
	yellow = rgb("f6c177");
	orange = rgb("ea9d34");
	gray = rgb("9893a5");
	grey = rgb("9893a5");
	deepcyan = rgb("56949f");
	pink = rgb("b4637a");
	magenta = rgb("eb6f92");
	green = rgb("a0d971");
	purple = rgb("907aa9");

	defaultpen(black + fontsize(8pt));

	import three;
	currentlight = nolight;
\end{asydef}

% theorems, proofs, ...

\mdfsetup{skipabove=1em,skipbelow=1em, innertopmargin=6pt, innerbottommargin=6pt,}

\declaretheoremstyle[
	headfont=\normalfont\itshape,
	numbered=no,
	postheadspace=\newline,
	headpunct={:},
	qed=\qedsymbol]{demstyle}

\declaretheorem[style=demstyle, name=Démonstration]{dem}

\newcommand\veczero{\kern-1.2pt\vec{\kern1.2pt 0}} % \vec{0} looks weird since the `0' isn't italicized

\makeatletter
\renewcommand{\title}[2]{
	\AtBeginDocument{
		\begin{titlepage}
			\begin{center}
				\vspace{10cm}
				{\Large \sc Chapitre #1}\\
				\vspace{1cm}
				{\Huge \calligra #2}\\
				\vfill
				Hugo {\sc Salou} MPI${}^{\star}$\\
				{\small Dernière mise à jour le \@date }
			\end{center}
		\end{titlepage}
	}
}

\newcommand{\titletp}[4]{
	\AtBeginDocument{
		\begin{titlepage}
			\begin{center}
				\vspace{10cm}
				{\Large \sc tp #1}\\
				\vspace{1cm}
				{\Huge \textsc{\textit{#2}}}\\
				\vfill
				{#3}\textit{MPI}${}^{\star}$\\
			\end{center}
		\end{titlepage}
	}
	\fancyfoot{}\fancyhead{}
	\fancyfoot[R]{#4 \textit{MPI}${}^{\star}$}
	\fancyhead[C]{{\sc tp #1} : #2}
	\fancyhead[R]{\thepage}
}

\newcommand{\titletd}[2]{
	\AtBeginDocument{
		\begin{titlepage}
			\begin{center}
				\vspace{10cm}
				{\Large \sc td #1}\\
				\vspace{1cm}
				{\Huge \calligra #2}\\
				\vfill
				Hugo {\sc Salou} MPI${}^{\star}$\\
				{\small Dernière mise à jour le \@date }
			\end{center}
		\end{titlepage}
	}
}
\makeatother

\newcommand{\sign}{
	\null
	\vfill
	\begin{center}
		{
			\fontfamily{ccr}\selectfont
			\textit{\textbf{\.{\"i}}}
		}
	\end{center}
	\vfill
	\null
}

\renewcommand{\thefootnote}{\emph{\alph{footnote}}}

% figure support
\usepackage{import}
\usepackage{xifthen}
\pdfminorversion=7
\usepackage{pdfpages}
\usepackage{transparent}
\newcommand{\incfig}[1]{%
	\def\svgwidth{\columnwidth}
	\import{./figures/}{#1.pdf_tex}
}

\pdfsuppresswarningpagegroup=1
\ctikzset{tripoles/european not symbol=circle}

\newcommand{\missingpart}{{\large\color{red} Il manque quelque chose ici\ldots}}


\fancyhead[R]{Hugo {\sc Salou}\/ MPI}
\fancyhead[L]{TD\textsubscript2 -- Exercice 7}

\begin{document}
	\let\thesection\relax
	\section{Exercice 3}

{\bf Indication}\/ : pour la $G$, on applique la relation de {\sc Chasles}\/ : l'intégrale $\int_0^7 \mathrm{e}^{-x}\ln x~\mathrm{d}x$\/ converge si et seulement si $\int_0^1 \mathrm{e}^{-x}\ln x\mathrm{d}x$\/ converge et $\int_1^7\mathrm{e}^{-x}\ln x~\mathrm{d}x$\/ converge (qui n'est même pas impropre).

L'intégrale $H = \int_0^1 \frac{\mathrm{e}^{\sin t}}{t}~\mathrm{d}t$\/ est impropre en 0. On sait que $\sin t \tendsto{t\to 0} 0$, et donc, par continuité de la fonction $\exp$\/ en $0$, $\mathrm{e}^{\sin t}\tendsto{t\to 0}e^0 = 1$.
Ainsi, $\frac{\mathrm{e}^{\sin t}}{t} = \mathrm{e}^{\sin t} \times \frac{1}{t} \simi_{t\to 0} \frac{1}{t}$\/ qui ne change pas de signe. Or, $\int_0^1 \frac{1}{t}~\mathrm{d}t$\/ diverge, donc l'intégrale $H$\/ diverge.

L'intégrale $I = \int_{1}^{+\infty} \frac{\mathrm{e}^{\sin t}}{t}~\mathrm{d}t$\/ est impropre en $+\infty$. Par croissance de la fonction exponentielle, on a $\frac{\mathrm{e}^{\sin t}}{t} \ge \frac{\mathrm{e}^{-1}}{t} \ge 0$. Or, l'intégrale $\int_{1}^{+\infty} \frac{1}{t}~\mathrm{d}t$\/ diverge, donc l'intégrale diverge aussi.

L'intégrale $K$, est l'intégrale d'une fonction Gau\ss ienne, et elle est impropre en $+\infty$. On la \guillemotleft~découpe~\guillemotright\ : \[
	\int_{0}^{+\infty} \mathrm{e}^{-x^2}~\mathrm{d}x \text{ converge si et seulement si } \int_{0}^{1} \mathrm{e}^{-x^2}~\mathrm{d}x \text{ converge et } \int_{1}^{+\infty} \mathrm{e}^{-x^2}~\mathrm{d}x \text{ converge.}
\] L'intégrale $\int_{0}^{1} \mathrm{e}^{-x^2}~\mathrm{d}x$\/ n'est même pas impropre, elle converge donc. Et, pour $x \in [1,+\infty[$, on sait, comme $x^2 \ge x$, $0 \le \mathrm{e}^{-x^2} \le \mathrm{e}^{-x}$. Or, $\int_{1}^{+\infty} \mathrm{e}^{-x}~\mathrm{d}x$\/ converge donc $\int_{0}^{+\infty} \mathrm{e}^{-x^2}~\mathrm{d}x$\/ aussi.
On calculera la valeur de cette intégrale dans le {\sc td}\/ \guillemotleft~Intégrales paramétrées.~\guillemotright

Autre méthode pour déterminer la nature de $K$\/ : 
$\mathrm{e}^{-x^2} = \po(\mathrm{e}^{-x})$\/ car $\mathrm{e}^{-x^2} = \underbrace{\mathrm{e}^{-x^2 + x}}_{\to 0} \times \mathrm{e}^{-x}$, car $\mathrm{e}^{-x^2 + x} = \mathrm{e}^{-x^2 \left( 1 - \frac{1}{x} \right)}$\/ et $-x^2\left( 1 - \frac{1}{x} \right) \to -\infty \times 1$.
Et $\int_0^{+\infty} \mathrm{e}^{-x}~\mathrm{d}x$\/ converge donc $\int_0^{+\infty} \mathrm{e}^{-x^2}~\mathrm{d}x$\/ converge.

\begin{figure}[H]
	\centering
	\begin{asy}
		import graph;
		size(10cm);
		draw((-10, 0) -- (10, 0), Arrow(TeXHead));
		draw((0, -3) -- (0, 5), Arrow(TeXHead));
		real f(real x) {
			return 4*exp(-(x/4)^2);
		}
		draw(graph(f, -10, 10), magenta);
	\end{asy}
	\caption{Courbe Gau\ss ienne}
\end{figure}

L'intégrale $F = \int_{7}^{+\infty} \mathrm{e}^{-x}\ln x~\mathrm{d}x$\/ est impropre en $+\infty$. Attention : la fonction n'est pas \guillemotleft~{\color{red} faussement impropre en $+\infty$}.~\guillemotright\ Mais, on peut remarquer que \[
	\mathrm{e}^{-x} \ln x = \mathrm{e}^{-\frac{x}{2}} \underbrace{\mathrm{e}^{-\frac{x}{2}} \ln x}_{\tendsto{x\to +\infty} 0} = \po(\mathrm{e}^{-\frac{x}{2}})
.\] Or, $\int_{7}^{+\infty} \mathrm{e}^{-x}~\mathrm{d}x$\/ converge donc l'intégrale $F$\/ converge aussi.

L'intégrale $G = \int_{0}^{7} \mathrm{e}^{-x}\ln x~\mathrm{d}x$\/ est impropre en 0.
Or, $\mathrm{e}^{-x}\ln x \simi_{x\to 0} \ln x$\/ qui ne change pas de signe au voisinage de 0. Or, $\int_{0}^{7}  \ln x~\mathrm{d}x$\/ converge donc l'intégrale $G$\/ converge également.

L'intégrale $E = \int_{1}^{+\infty} \frac{\ln x}{\sqrt{x}}~\mathrm{d}x$\/ est impropre en $+\infty$. Or, $\forall x \ge \mathrm{e}$, $\frac{\ln(x)}{\sqrt{x}} \ge \frac{1}{\sqrt{x}} \ge 0$\/ converge.
Or, $\int_{1}^{+\infty}  \frac{1}{x^{\sfrac{1}{2}}}~\mathrm{d}x$\/ diverge d'après le critère de {\sc Riemann}\/ en $+\infty$\/ car $\frac{1}{2} < 1$.
D'où l'intégrale $D$\/ diverge.

Autre méthode : intégration par parties. On peut même arriver à calculer une primitive de ${\ln x}\:/{\sqrt{x}}$.

L'intégrale $D = \int_{0}^{1} \frac{\ln x}{\sqrt{x}}~\mathrm{d}x$\/ est impropre en 0. On peut remarque que \[
	0 \le -\frac{\ln x}{\sqrt{x}} = -\frac{x^{0{,}1} \ln x}{x^{0{,}6}} = \po\left( \frac{1}{x^{0{,}6}} \right) \quad\text{car}\quad x^{0{,}1} \ln x \tendsto{x\to 0} 0
\] par croissances comparées.
Or, $\int_{0}^{1} \frac{1}{x^{0{,}6}}~\mathrm{d}x$\/ converge d'après le critère de {\sc Riemann}. D'où $-D$\/ converge et donc $D$\/ converge.

L'intégrale $J = \int_{1}^{+\infty} \frac{\sin t}{\sqrt{t} + \sin t}~\mathrm{d}t$\/ est impropre en $+\infty$. On calcule
\[
	f(t) = \frac{\sin t}{\sqrt{t} + \sin t} = \frac{\sin t}{\sqrt{t}} \times \frac{1}{1+\frac{\sin t}{\sqrt{t}}}
\] et $\frac{\sin t}{\sqrt{t}} \tendsto{t\to +\infty} 0$. D'où \[
	\frac{1}{1+\frac{\sin t}{\sqrt{t}}} = 1  - \frac{\sin t}{\sqrt{t}} + \frac{\sin^2 t}{t} + \po\left( \frac{\sin^2 t}{t} \right)
\] et donc \[
	f(t) = \frac{\cos t}{\sqrt{t}} + \frac{\sin^2 t}{t} + \po\left( \frac{\sin^2 t}{t} \right)
.\]
L'intégrale $\int_{1}^{+\infty} \frac{\sin t}{\sqrt{t}}~\mathrm{d}t$\/ est impropre en $+\infty$.
Soit $x \ge 1$. On calcule avec une intégration par parties,
\[
	\int_{1}^{x} \sin t \times \frac{1}{\sqrt{t}}~\mathrm{d}t = \int_{1}^{x} u'(t)\cdot v(t)~\mathrm{d}t
\] où $u(t) = - \cos t$\/ et $v(t) = \frac{1}{\sqrt{t}} = t^{-\frac{1}{2}}$. Donc
\begin{align*}
	\int_{1}^{x} \frac{\sin t}{\sqrt{t}}~\mathrm{d}t &= \Big[f(t)g(t)\Big]_1^x - \int_{1}^{x} f(t)\cdot g'(t)~\mathrm{d}t\\
	&= \left[ - \frac{\cos t}{\sqrt{t}} \right]_1^x - \int_{1}^{x} (-\cos t)\left( -\frac{1}{2}t^{-\frac{3}{2}} \right)~\mathrm{d}t \\
\end{align*}
D'où \[
	\int_{1}^{x} \frac{\sin t}{\sqrt{t}}~\mathrm{d}t = \cos 1 - \frac{\cos x}{\sqrt{x}} - \frac{1}{2} \int_{1}^{x} \frac{\cos t}{t^{\sfrac{3}{2}}}~\mathrm{d}t
.\]
Or, d'une part $\cos x \times \frac{1}{\sqrt{x}} \tendsto{x\to +\infty} 0$\/ car $\cos$\/ est bornée et $\frac{1}{\sqrt{x}}\tendsto{x\to +\infty} 0$.
Et, d'autre part $\int_{1}^{+\infty} \frac{\cos t}{t^{\sfrac{3}{2}}}~\mathrm{d}t$\/ converge car $\forall t \in [1,+\infty[$, $\left| \frac{\cos t}{t^{\sfrac{3}{2}}} \right| \le \frac{1}{t^{\sfrac{3}{2}}}$\/ et $\int_{1}^{+\infty} \frac{1}{t^{\sfrac{3}{2}}}~\mathrm{d}t$\/ converge.
Pour le 2\tsup{nd} terme du développement limité, on fait une {\sc ipp}, on trouve un terme en $\frac{1}{t^2}$\/ et donc son intégrale converge par critère de {\sc Riemann}. S'il y a des problèmes, voir en {\sc td}.
On étudie maintenant le 3\tsup{ème} terme : \[
	\int_{1}^{+\infty}  \po\left( \frac{\sin t}{t} \right) ~\mathrm{d}t \text{ converge car } \int_{1}^{+\infty} \frac{\sin^2 t}{t} ~\mathrm{d}t \text{ converge et } t\mapsto \frac{\sin^2 t}{t} \text{ est positive}.
\]
Autre méthode : on a \[
	\frac{\sin^2 t}{t} + \po\left( \frac{\sin^2 t}{t} \right) \simi_{t\to +\infty} \frac{\sin^2 t}{t} \text{ qui ne change pas de signe}
.\] Or, $\int_{1}^{+\infty} \frac{\sin^2 t}{t}~\mathrm{d}t$\/ converge et donc \[
	\int_{1}^{+\infty} \left( \frac{\sin^2 t}{t} + \po\left( \frac{\sin^2 t}{t} \right) \right) ~\mathrm{d}t
.\]

\end{document}

	}
	\def\addmacros#1{#1}
}
{
	\td[12]{Algorithmes d'approximation}
	\minitoc
	\renewcommand{\cwd}{../td/td12/}
	\addmacros{
		\section{Filtre RC double}

\begin{enumerate}
	\item En basse fréquence, un condensateur est équivalent à un interrupteur ouvert. En haute fréquence, un condensateur est équivalent à un interrupteur fermé. D'où, le circuit est un filtre passe-bas.
	\item Par une loi des nœuds, et une loi des mailles, on trouve que
		\[
			\ubar{H}(\mathrm{j}\omega) = \dfrac{1}{1 - \left( \dfrac{\omega}{\omega_0} \right)^2 + \mathrm{j} \dfrac{\omega}{Q\cdot \omega_0}}
		\] en notant $\omega_0 = 1 / {RC}$ et $Q = 1 / 3$
	\item On représente le diagramme de \textsc{Bode} du filtre dans la figure ci-dessous.
		\begin{figure}[H]
			\centering
			\includesvg[width=\linewidth]{figures/bode-1.svg}
			\caption{Diagramme de \textsc{Bode} du filtre (échelle logarithmique)}
		\end{figure}
	\item On calcule $\omega_0 \simeq 6\:\mathrm{rad/s}$, ce qui correspond à une fréquence de coupure de $1\:\mathrm{kHz}$. Le signal de sortie est donc \[
			s(t) = \frac{2E}{3\pi}\cdot \sin(\omega t)
		,\] on le représente sur la figure ci-dessous. En effet, on a un déphasage de $-\pi / 2$, et un gain valant $1 / 3$ à $\omega \simeq \omega_0$.
		\begin{figure}[H]
			\centering
			\includesvg[width=\linewidth]{figures/signal-1.svg}
			\caption{Signal résultant}
		\end{figure}
\end{enumerate}

		\section{Tri topologique}

\begin{figure}[H]
	\centering
	\tikzfig{ex2-q1}
	\caption{Exemple de graphe}
\end{figure}

\begin{enumerate}
	\item Dans le graphe ci-dessus, $a \to c \to b$\/ est un tri topologique mais pas un parcours.
	\item Dans le même graphe, $b \to a \to c$\/ est un parcours mais pas un tri topologique.
	\item Supposons que $L_1$\/ possède un prédécesseur, on le note $L_i$\/ où $i > 1$. Ainsi, $(L_i, L_1) \in A$\/ et donc $i < 1$, ce qui est absurde. De même pour le dernier.
	\item Il existe un tri topologique si, et seulement si le graphe est acyclique.
		\begin{itemize}
			\item[``$\implies$'']
				Soit $L_1,\ldots,L_n$\/ un tri topologique. Montrons que le graphe est acyclique.
				Par l'absurde, on suppose le graphe non acyclique : il existe $(i,j) \in \llbracket 1,n \rrbracket^2$\/ avec $i \neq j$\/ tels que $T_i \to \cdots \to T_j$\/ et $T_j \to \cdots \to T_i$ soient deux chemins valides. Ainsi, comme le tri est topologique et par récurrence, $i \le j$\/ et $j \le i$\/ et donc $i = j$, ce qui est absurde car $i$\/ et $j$\/ sont supposés différents. Le graphe est donc acyclique.
			\item[``$\impliedby$'']
				Soit $G$\/ un graphe tel que tous les sommets possèdent une arrête entrante. On suppose par l'absurde ce graphe acyclique.
				Soit $x_0$\/ un sommet du graphe.
				On construit par récurrence $x_0,x_1,\ldots,x_n,x_{n+1},\ldots$ les successeurs successifs. Il y a un nombre fini de sommets donc deux sommets sont identiques. Donc, il y a nécessairement un cycle, ce qui est absurde.
				\begin{algorithm}[H]
					\centering
					\begin{algorithmic}[1]
						\Entree $G = (S, A)$\/ un graphe acyclique
						\Sortie $\mathrm{Res}$\/ un tri topologique.
						\State $\mathrm{Res} \gets [\quad]$\/
						\While{$G \neq \O$}
							\State Soit $x$\/ un sommet de $G$\/ sans prédécesseur
							\State $G \gets \big(S \setminus \{x\}, A \cap (S \setminus \{x\})^2\big)$\/ 
							\State $\mathrm{Res} \gets \mathrm{Res} \cdot [x]$\/
						\EndWhile
						\State\Return $\mathrm{Res}$\/
					\end{algorithmic}
					\caption{Génération d'un tri topologique d'un graphe acyclique}
				\end{algorithm}
		\end{itemize}
	\item~
		\begin{algorithm}[H]
			\centering
			\begin{algorithmic}[1]
				\Entree $G = (S, A)$\/ un graphe
				\Sortie $\mathrm{Res}$\/ un tri topologique, ou un cycle
				\State $\mathrm{Res} \gets [\quad]$\/
				\While{$G \neq \O$}
					\If{il existe $x$ sans prédécesseurs}
						\State Soit $x$\/ un sommet de $G$\/ sans prédécesseur
						\State $G \gets \big(S \setminus \{x\}, A \cap (S \setminus \{x\})^2\big)$\/ 
						\State $\mathrm{Res} \gets \mathrm{Res} \cdot [x]$\/
					\Else
						\State Soit $x \in S$\/ 
						\State Soit $x \gets x_1 \gets x_2 \gets \cdots \gets x_i$\/ la suite des prédécesseurs
						\State\Return $x_i,x_{i+1},\ldots,x_i$, un cycle
					\EndIf
				\EndWhile
				\State\Return $\mathrm{Res}$\/
			\end{algorithmic}
			\caption{Génération d'un tri topologique d'un graphe}
		\end{algorithm}
	\item On utilise la représentation par liste d'adjacence, et on stocke le nombre de prédécesseurs que l'on décroit à chaque choix de sommet.
	\item On essaie de trouver un tri topologique, et on voit si l'on trouve un cycle.
\end{enumerate}

		\section{Formules duales}

\begin{enumerate}
	\item On définit par induction $(\cdot)^\star$\/ comme
		\begin{multicols}{3}
			\begin{itemize}
				\item $\top^\star = \bot$\/ ;
				\item $\bot^\star = \top$\/ ;
				\item $(G \lor H)^\star = G^\star \land H^\star$\/ ;
				\item $(G \land H)^\star = G^\star \lor H^\star$\/ ;
				\item $(\lnot G)^\star = \lnot G^\star$\/ ;
				\item $p^\star = p$.
			\end{itemize}
		\end{multicols}
	\item Soit $\rho \in \mathds{B}^{\mathcal{P}}$. Montrons, par induction, $P(H) : ``\left\llbracket H^\star \right\rrbracket^\rho = \left\llbracket \lnot H \right\rrbracket^{\bar \rho}"$\/ où $\bar{\rho} : p \mapsto \overline{\rho(p)}$.
		\begin{itemize}
			\item On a $\left\llbracket \bot^\star \right\rrbracket^\rho = \left\llbracket \top \right\rrbracket^\rho = \mathbf{V}$, et $\left\llbracket \lnot \bot \right\rrbracket^{\bar\rho} = \left\llbracket \top \right\rrbracket^{\bar\rho} = \mathbf{V}$, d'où $P(\bot)$.
			\item On a $\left\llbracket \top^\star \right\rrbracket^\rho = \left\llbracket \bot \right\rrbracket^\rho = \mathbf{F}$, et $\left\llbracket \lnot \top \right\rrbracket^{\bar\rho} = \left\llbracket \bot \right\rrbracket^{\bar\rho} = \mathbf{F}$, d'où $P(\top)$.
			\item Soit $p \in \mathcal{P}$. On a $\left\llbracket p^\star  \right\rrbracket^\rho = \left\llbracket p \right\rrbracket^\rho = \rho(p)$, et $\left\llbracket \lnot p \right\rrbracket^{\bar\rho} = \overline{\left\llbracket p \right\rrbracket^{\bar\rho}} = \overline{\bar\rho(p)} = \overline{\overline{\rho(p)}} = \rho(p)$, d'où~$P(p)$.
		\end{itemize}
		Soient $F$\/ et $G$\/ deux formules.
		\begin{itemize}
			\item On a
				\begin{align*}
					\left\llbracket (F \land G)^\star  \right\rrbracket^\rho &= \left\llbracket F^\star  \lor G^\star \right\rrbracket^\rho\\
					&= \left\llbracket F^\star \right\rrbracket^\rho + \left\llbracket G^\star \right\rrbracket^\rho \\
					&= \left\llbracket \lnot F \right\rrbracket^{\bar\rho} + \left\llbracket \lnot G \right\rrbracket^{\bar\rho} \\
					&= \left\llbracket \lnot F \lor \lnot G \right\rrbracket^{\bar\rho} \\
					&= \left\llbracket \lnot (F \land G) \right\rrbracket^{\bar\rho} \\
				\end{align*}
				d'où $P(F \land G)$.
			\item On a
				\begin{align*}
					\left\llbracket (F \lor G)^\star  \right\rrbracket^\rho &= \left\llbracket F^\star  \land G^\star \right\rrbracket^\rho\\
					&= \left\llbracket F^\star \right\rrbracket^\rho \cdot \left\llbracket G^\star \right\rrbracket^\rho \\
					&= \left\llbracket \lnot F \right\rrbracket^{\bar\rho} \cdot \left\llbracket \lnot G \right\rrbracket^{\bar\rho} \\
					&= \left\llbracket \lnot F \land \lnot G \right\rrbracket^{\bar\rho} \\
					&= \left\llbracket \lnot (F \lor G) \right\rrbracket^{\bar\rho} \\
				\end{align*}
				d'où $P(F \lor G)$.
			\item On a \[
					\left\llbracket (\lnot F)^\star  \right\rrbracket^\rho = \left\llbracket \lnot (F^\star) \right\rrbracket^\rho = \overline{\left\llbracket F^\star \right\rrbracket^\rho} = \overline{\left\llbracket \lnot F \right\rrbracket^{\bar\rho}} = \left\llbracket \lnot (\lnot F) \right\rrbracket^{\bar\rho}.
					\] d'où $P(\lnot F)$.
		\end{itemize}
		Par induction, on en conclut que $P(F)$\/ est vraie pour toute formule $F$.
	\item Soit $G$\/ une formule valide. Alors, par définition, $G \equiv \top$. Or, d'après la question précédente, $G^\star \equiv (\top)^\star = \bot$. Ainsi, $G^\star $\/ n'est pas satisfiable.
\end{enumerate}



	}
	\def\addmacros#1{#1}
}
{
	\td[13]{Jeux}
	\minitoc
	\renewcommand{\cwd}{../td/td13/}
	\addmacros{
		\section{Filtre RC double}

\begin{enumerate}
	\item En basse fréquence, un condensateur est équivalent à un interrupteur ouvert. En haute fréquence, un condensateur est équivalent à un interrupteur fermé. D'où, le circuit est un filtre passe-bas.
	\item Par une loi des nœuds, et une loi des mailles, on trouve que
		\[
			\ubar{H}(\mathrm{j}\omega) = \dfrac{1}{1 - \left( \dfrac{\omega}{\omega_0} \right)^2 + \mathrm{j} \dfrac{\omega}{Q\cdot \omega_0}}
		\] en notant $\omega_0 = 1 / {RC}$ et $Q = 1 / 3$
	\item On représente le diagramme de \textsc{Bode} du filtre dans la figure ci-dessous.
		\begin{figure}[H]
			\centering
			\includesvg[width=\linewidth]{figures/bode-1.svg}
			\caption{Diagramme de \textsc{Bode} du filtre (échelle logarithmique)}
		\end{figure}
	\item On calcule $\omega_0 \simeq 6\:\mathrm{rad/s}$, ce qui correspond à une fréquence de coupure de $1\:\mathrm{kHz}$. Le signal de sortie est donc \[
			s(t) = \frac{2E}{3\pi}\cdot \sin(\omega t)
		,\] on le représente sur la figure ci-dessous. En effet, on a un déphasage de $-\pi / 2$, et un gain valant $1 / 3$ à $\omega \simeq \omega_0$.
		\begin{figure}[H]
			\centering
			\includesvg[width=\linewidth]{figures/signal-1.svg}
			\caption{Signal résultant}
		\end{figure}
\end{enumerate}

		\section{Tri topologique}

\begin{figure}[H]
	\centering
	\tikzfig{ex2-q1}
	\caption{Exemple de graphe}
\end{figure}

\begin{enumerate}
	\item Dans le graphe ci-dessus, $a \to c \to b$\/ est un tri topologique mais pas un parcours.
	\item Dans le même graphe, $b \to a \to c$\/ est un parcours mais pas un tri topologique.
	\item Supposons que $L_1$\/ possède un prédécesseur, on le note $L_i$\/ où $i > 1$. Ainsi, $(L_i, L_1) \in A$\/ et donc $i < 1$, ce qui est absurde. De même pour le dernier.
	\item Il existe un tri topologique si, et seulement si le graphe est acyclique.
		\begin{itemize}
			\item[``$\implies$'']
				Soit $L_1,\ldots,L_n$\/ un tri topologique. Montrons que le graphe est acyclique.
				Par l'absurde, on suppose le graphe non acyclique : il existe $(i,j) \in \llbracket 1,n \rrbracket^2$\/ avec $i \neq j$\/ tels que $T_i \to \cdots \to T_j$\/ et $T_j \to \cdots \to T_i$ soient deux chemins valides. Ainsi, comme le tri est topologique et par récurrence, $i \le j$\/ et $j \le i$\/ et donc $i = j$, ce qui est absurde car $i$\/ et $j$\/ sont supposés différents. Le graphe est donc acyclique.
			\item[``$\impliedby$'']
				Soit $G$\/ un graphe tel que tous les sommets possèdent une arrête entrante. On suppose par l'absurde ce graphe acyclique.
				Soit $x_0$\/ un sommet du graphe.
				On construit par récurrence $x_0,x_1,\ldots,x_n,x_{n+1},\ldots$ les successeurs successifs. Il y a un nombre fini de sommets donc deux sommets sont identiques. Donc, il y a nécessairement un cycle, ce qui est absurde.
				\begin{algorithm}[H]
					\centering
					\begin{algorithmic}[1]
						\Entree $G = (S, A)$\/ un graphe acyclique
						\Sortie $\mathrm{Res}$\/ un tri topologique.
						\State $\mathrm{Res} \gets [\quad]$\/
						\While{$G \neq \O$}
							\State Soit $x$\/ un sommet de $G$\/ sans prédécesseur
							\State $G \gets \big(S \setminus \{x\}, A \cap (S \setminus \{x\})^2\big)$\/ 
							\State $\mathrm{Res} \gets \mathrm{Res} \cdot [x]$\/
						\EndWhile
						\State\Return $\mathrm{Res}$\/
					\end{algorithmic}
					\caption{Génération d'un tri topologique d'un graphe acyclique}
				\end{algorithm}
		\end{itemize}
	\item~
		\begin{algorithm}[H]
			\centering
			\begin{algorithmic}[1]
				\Entree $G = (S, A)$\/ un graphe
				\Sortie $\mathrm{Res}$\/ un tri topologique, ou un cycle
				\State $\mathrm{Res} \gets [\quad]$\/
				\While{$G \neq \O$}
					\If{il existe $x$ sans prédécesseurs}
						\State Soit $x$\/ un sommet de $G$\/ sans prédécesseur
						\State $G \gets \big(S \setminus \{x\}, A \cap (S \setminus \{x\})^2\big)$\/ 
						\State $\mathrm{Res} \gets \mathrm{Res} \cdot [x]$\/
					\Else
						\State Soit $x \in S$\/ 
						\State Soit $x \gets x_1 \gets x_2 \gets \cdots \gets x_i$\/ la suite des prédécesseurs
						\State\Return $x_i,x_{i+1},\ldots,x_i$, un cycle
					\EndIf
				\EndWhile
				\State\Return $\mathrm{Res}$\/
			\end{algorithmic}
			\caption{Génération d'un tri topologique d'un graphe}
		\end{algorithm}
	\item On utilise la représentation par liste d'adjacence, et on stocke le nombre de prédécesseurs que l'on décroit à chaque choix de sommet.
	\item On essaie de trouver un tri topologique, et on voit si l'on trouve un cycle.
\end{enumerate}

		\section{Formules duales}

\begin{enumerate}
	\item On définit par induction $(\cdot)^\star$\/ comme
		\begin{multicols}{3}
			\begin{itemize}
				\item $\top^\star = \bot$\/ ;
				\item $\bot^\star = \top$\/ ;
				\item $(G \lor H)^\star = G^\star \land H^\star$\/ ;
				\item $(G \land H)^\star = G^\star \lor H^\star$\/ ;
				\item $(\lnot G)^\star = \lnot G^\star$\/ ;
				\item $p^\star = p$.
			\end{itemize}
		\end{multicols}
	\item Soit $\rho \in \mathds{B}^{\mathcal{P}}$. Montrons, par induction, $P(H) : ``\left\llbracket H^\star \right\rrbracket^\rho = \left\llbracket \lnot H \right\rrbracket^{\bar \rho}"$\/ où $\bar{\rho} : p \mapsto \overline{\rho(p)}$.
		\begin{itemize}
			\item On a $\left\llbracket \bot^\star \right\rrbracket^\rho = \left\llbracket \top \right\rrbracket^\rho = \mathbf{V}$, et $\left\llbracket \lnot \bot \right\rrbracket^{\bar\rho} = \left\llbracket \top \right\rrbracket^{\bar\rho} = \mathbf{V}$, d'où $P(\bot)$.
			\item On a $\left\llbracket \top^\star \right\rrbracket^\rho = \left\llbracket \bot \right\rrbracket^\rho = \mathbf{F}$, et $\left\llbracket \lnot \top \right\rrbracket^{\bar\rho} = \left\llbracket \bot \right\rrbracket^{\bar\rho} = \mathbf{F}$, d'où $P(\top)$.
			\item Soit $p \in \mathcal{P}$. On a $\left\llbracket p^\star  \right\rrbracket^\rho = \left\llbracket p \right\rrbracket^\rho = \rho(p)$, et $\left\llbracket \lnot p \right\rrbracket^{\bar\rho} = \overline{\left\llbracket p \right\rrbracket^{\bar\rho}} = \overline{\bar\rho(p)} = \overline{\overline{\rho(p)}} = \rho(p)$, d'où~$P(p)$.
		\end{itemize}
		Soient $F$\/ et $G$\/ deux formules.
		\begin{itemize}
			\item On a
				\begin{align*}
					\left\llbracket (F \land G)^\star  \right\rrbracket^\rho &= \left\llbracket F^\star  \lor G^\star \right\rrbracket^\rho\\
					&= \left\llbracket F^\star \right\rrbracket^\rho + \left\llbracket G^\star \right\rrbracket^\rho \\
					&= \left\llbracket \lnot F \right\rrbracket^{\bar\rho} + \left\llbracket \lnot G \right\rrbracket^{\bar\rho} \\
					&= \left\llbracket \lnot F \lor \lnot G \right\rrbracket^{\bar\rho} \\
					&= \left\llbracket \lnot (F \land G) \right\rrbracket^{\bar\rho} \\
				\end{align*}
				d'où $P(F \land G)$.
			\item On a
				\begin{align*}
					\left\llbracket (F \lor G)^\star  \right\rrbracket^\rho &= \left\llbracket F^\star  \land G^\star \right\rrbracket^\rho\\
					&= \left\llbracket F^\star \right\rrbracket^\rho \cdot \left\llbracket G^\star \right\rrbracket^\rho \\
					&= \left\llbracket \lnot F \right\rrbracket^{\bar\rho} \cdot \left\llbracket \lnot G \right\rrbracket^{\bar\rho} \\
					&= \left\llbracket \lnot F \land \lnot G \right\rrbracket^{\bar\rho} \\
					&= \left\llbracket \lnot (F \lor G) \right\rrbracket^{\bar\rho} \\
				\end{align*}
				d'où $P(F \lor G)$.
			\item On a \[
					\left\llbracket (\lnot F)^\star  \right\rrbracket^\rho = \left\llbracket \lnot (F^\star) \right\rrbracket^\rho = \overline{\left\llbracket F^\star \right\rrbracket^\rho} = \overline{\left\llbracket \lnot F \right\rrbracket^{\bar\rho}} = \left\llbracket \lnot (\lnot F) \right\rrbracket^{\bar\rho}.
					\] d'où $P(\lnot F)$.
		\end{itemize}
		Par induction, on en conclut que $P(F)$\/ est vraie pour toute formule $F$.
	\item Soit $G$\/ une formule valide. Alors, par définition, $G \equiv \top$. Or, d'après la question précédente, $G^\star \equiv (\top)^\star = \bot$. Ainsi, $G^\star $\/ n'est pas satisfiable.
\end{enumerate}



		\section{Un lemme d'itération}

		\section{Ambigüité}

	}
	\def\addmacros#1{#1}
}
{
	\td[14]{Grammaires non contextuelles (1)}
	\minitoc
	\renewcommand{\cwd}{../td/td14/}
	\addmacros{
		\section{Filtre RC double}

\begin{enumerate}
	\item En basse fréquence, un condensateur est équivalent à un interrupteur ouvert. En haute fréquence, un condensateur est équivalent à un interrupteur fermé. D'où, le circuit est un filtre passe-bas.
	\item Par une loi des nœuds, et une loi des mailles, on trouve que
		\[
			\ubar{H}(\mathrm{j}\omega) = \dfrac{1}{1 - \left( \dfrac{\omega}{\omega_0} \right)^2 + \mathrm{j} \dfrac{\omega}{Q\cdot \omega_0}}
		\] en notant $\omega_0 = 1 / {RC}$ et $Q = 1 / 3$
	\item On représente le diagramme de \textsc{Bode} du filtre dans la figure ci-dessous.
		\begin{figure}[H]
			\centering
			\includesvg[width=\linewidth]{figures/bode-1.svg}
			\caption{Diagramme de \textsc{Bode} du filtre (échelle logarithmique)}
		\end{figure}
	\item On calcule $\omega_0 \simeq 6\:\mathrm{rad/s}$, ce qui correspond à une fréquence de coupure de $1\:\mathrm{kHz}$. Le signal de sortie est donc \[
			s(t) = \frac{2E}{3\pi}\cdot \sin(\omega t)
		,\] on le représente sur la figure ci-dessous. En effet, on a un déphasage de $-\pi / 2$, et un gain valant $1 / 3$ à $\omega \simeq \omega_0$.
		\begin{figure}[H]
			\centering
			\includesvg[width=\linewidth]{figures/signal-1.svg}
			\caption{Signal résultant}
		\end{figure}
\end{enumerate}

		\section{Tri topologique}

\begin{figure}[H]
	\centering
	\tikzfig{ex2-q1}
	\caption{Exemple de graphe}
\end{figure}

\begin{enumerate}
	\item Dans le graphe ci-dessus, $a \to c \to b$\/ est un tri topologique mais pas un parcours.
	\item Dans le même graphe, $b \to a \to c$\/ est un parcours mais pas un tri topologique.
	\item Supposons que $L_1$\/ possède un prédécesseur, on le note $L_i$\/ où $i > 1$. Ainsi, $(L_i, L_1) \in A$\/ et donc $i < 1$, ce qui est absurde. De même pour le dernier.
	\item Il existe un tri topologique si, et seulement si le graphe est acyclique.
		\begin{itemize}
			\item[``$\implies$'']
				Soit $L_1,\ldots,L_n$\/ un tri topologique. Montrons que le graphe est acyclique.
				Par l'absurde, on suppose le graphe non acyclique : il existe $(i,j) \in \llbracket 1,n \rrbracket^2$\/ avec $i \neq j$\/ tels que $T_i \to \cdots \to T_j$\/ et $T_j \to \cdots \to T_i$ soient deux chemins valides. Ainsi, comme le tri est topologique et par récurrence, $i \le j$\/ et $j \le i$\/ et donc $i = j$, ce qui est absurde car $i$\/ et $j$\/ sont supposés différents. Le graphe est donc acyclique.
			\item[``$\impliedby$'']
				Soit $G$\/ un graphe tel que tous les sommets possèdent une arrête entrante. On suppose par l'absurde ce graphe acyclique.
				Soit $x_0$\/ un sommet du graphe.
				On construit par récurrence $x_0,x_1,\ldots,x_n,x_{n+1},\ldots$ les successeurs successifs. Il y a un nombre fini de sommets donc deux sommets sont identiques. Donc, il y a nécessairement un cycle, ce qui est absurde.
				\begin{algorithm}[H]
					\centering
					\begin{algorithmic}[1]
						\Entree $G = (S, A)$\/ un graphe acyclique
						\Sortie $\mathrm{Res}$\/ un tri topologique.
						\State $\mathrm{Res} \gets [\quad]$\/
						\While{$G \neq \O$}
							\State Soit $x$\/ un sommet de $G$\/ sans prédécesseur
							\State $G \gets \big(S \setminus \{x\}, A \cap (S \setminus \{x\})^2\big)$\/ 
							\State $\mathrm{Res} \gets \mathrm{Res} \cdot [x]$\/
						\EndWhile
						\State\Return $\mathrm{Res}$\/
					\end{algorithmic}
					\caption{Génération d'un tri topologique d'un graphe acyclique}
				\end{algorithm}
		\end{itemize}
	\item~
		\begin{algorithm}[H]
			\centering
			\begin{algorithmic}[1]
				\Entree $G = (S, A)$\/ un graphe
				\Sortie $\mathrm{Res}$\/ un tri topologique, ou un cycle
				\State $\mathrm{Res} \gets [\quad]$\/
				\While{$G \neq \O$}
					\If{il existe $x$ sans prédécesseurs}
						\State Soit $x$\/ un sommet de $G$\/ sans prédécesseur
						\State $G \gets \big(S \setminus \{x\}, A \cap (S \setminus \{x\})^2\big)$\/ 
						\State $\mathrm{Res} \gets \mathrm{Res} \cdot [x]$\/
					\Else
						\State Soit $x \in S$\/ 
						\State Soit $x \gets x_1 \gets x_2 \gets \cdots \gets x_i$\/ la suite des prédécesseurs
						\State\Return $x_i,x_{i+1},\ldots,x_i$, un cycle
					\EndIf
				\EndWhile
				\State\Return $\mathrm{Res}$\/
			\end{algorithmic}
			\caption{Génération d'un tri topologique d'un graphe}
		\end{algorithm}
	\item On utilise la représentation par liste d'adjacence, et on stocke le nombre de prédécesseurs que l'on décroit à chaque choix de sommet.
	\item On essaie de trouver un tri topologique, et on voit si l'on trouve un cycle.
\end{enumerate}

		\section{Formules duales}

\begin{enumerate}
	\item On définit par induction $(\cdot)^\star$\/ comme
		\begin{multicols}{3}
			\begin{itemize}
				\item $\top^\star = \bot$\/ ;
				\item $\bot^\star = \top$\/ ;
				\item $(G \lor H)^\star = G^\star \land H^\star$\/ ;
				\item $(G \land H)^\star = G^\star \lor H^\star$\/ ;
				\item $(\lnot G)^\star = \lnot G^\star$\/ ;
				\item $p^\star = p$.
			\end{itemize}
		\end{multicols}
	\item Soit $\rho \in \mathds{B}^{\mathcal{P}}$. Montrons, par induction, $P(H) : ``\left\llbracket H^\star \right\rrbracket^\rho = \left\llbracket \lnot H \right\rrbracket^{\bar \rho}"$\/ où $\bar{\rho} : p \mapsto \overline{\rho(p)}$.
		\begin{itemize}
			\item On a $\left\llbracket \bot^\star \right\rrbracket^\rho = \left\llbracket \top \right\rrbracket^\rho = \mathbf{V}$, et $\left\llbracket \lnot \bot \right\rrbracket^{\bar\rho} = \left\llbracket \top \right\rrbracket^{\bar\rho} = \mathbf{V}$, d'où $P(\bot)$.
			\item On a $\left\llbracket \top^\star \right\rrbracket^\rho = \left\llbracket \bot \right\rrbracket^\rho = \mathbf{F}$, et $\left\llbracket \lnot \top \right\rrbracket^{\bar\rho} = \left\llbracket \bot \right\rrbracket^{\bar\rho} = \mathbf{F}$, d'où $P(\top)$.
			\item Soit $p \in \mathcal{P}$. On a $\left\llbracket p^\star  \right\rrbracket^\rho = \left\llbracket p \right\rrbracket^\rho = \rho(p)$, et $\left\llbracket \lnot p \right\rrbracket^{\bar\rho} = \overline{\left\llbracket p \right\rrbracket^{\bar\rho}} = \overline{\bar\rho(p)} = \overline{\overline{\rho(p)}} = \rho(p)$, d'où~$P(p)$.
		\end{itemize}
		Soient $F$\/ et $G$\/ deux formules.
		\begin{itemize}
			\item On a
				\begin{align*}
					\left\llbracket (F \land G)^\star  \right\rrbracket^\rho &= \left\llbracket F^\star  \lor G^\star \right\rrbracket^\rho\\
					&= \left\llbracket F^\star \right\rrbracket^\rho + \left\llbracket G^\star \right\rrbracket^\rho \\
					&= \left\llbracket \lnot F \right\rrbracket^{\bar\rho} + \left\llbracket \lnot G \right\rrbracket^{\bar\rho} \\
					&= \left\llbracket \lnot F \lor \lnot G \right\rrbracket^{\bar\rho} \\
					&= \left\llbracket \lnot (F \land G) \right\rrbracket^{\bar\rho} \\
				\end{align*}
				d'où $P(F \land G)$.
			\item On a
				\begin{align*}
					\left\llbracket (F \lor G)^\star  \right\rrbracket^\rho &= \left\llbracket F^\star  \land G^\star \right\rrbracket^\rho\\
					&= \left\llbracket F^\star \right\rrbracket^\rho \cdot \left\llbracket G^\star \right\rrbracket^\rho \\
					&= \left\llbracket \lnot F \right\rrbracket^{\bar\rho} \cdot \left\llbracket \lnot G \right\rrbracket^{\bar\rho} \\
					&= \left\llbracket \lnot F \land \lnot G \right\rrbracket^{\bar\rho} \\
					&= \left\llbracket \lnot (F \lor G) \right\rrbracket^{\bar\rho} \\
				\end{align*}
				d'où $P(F \lor G)$.
			\item On a \[
					\left\llbracket (\lnot F)^\star  \right\rrbracket^\rho = \left\llbracket \lnot (F^\star) \right\rrbracket^\rho = \overline{\left\llbracket F^\star \right\rrbracket^\rho} = \overline{\left\llbracket \lnot F \right\rrbracket^{\bar\rho}} = \left\llbracket \lnot (\lnot F) \right\rrbracket^{\bar\rho}.
					\] d'où $P(\lnot F)$.
		\end{itemize}
		Par induction, on en conclut que $P(F)$\/ est vraie pour toute formule $F$.
	\item Soit $G$\/ une formule valide. Alors, par définition, $G \equiv \top$. Or, d'après la question précédente, $G^\star \equiv (\top)^\star = \bot$. Ainsi, $G^\star $\/ n'est pas satisfiable.
\end{enumerate}



		\section{Un lemme d'itération}

		\section{Ambigüité}

		\section{Regexp Crossword}
\begin{center}
	\url{https://regexcrossword.com/}
\end{center}

		\documentclass[a4paper]{article}

\usepackage[margin=1in]{geometry}
\usepackage[utf8]{inputenc}
\usepackage[T1]{fontenc}
\usepackage{mathrsfs}
\usepackage{textcomp}
\usepackage[french]{babel}
\usepackage{amsmath}
\usepackage{amssymb}
\usepackage{cancel}
\usepackage{frcursive}
\usepackage[inline]{asymptote}
\usepackage{tikz}
\usepackage[european,straightvoltages,europeanresistors]{circuitikz}
\usepackage{tikz-cd}
\usepackage{tkz-tab}
\usepackage[b]{esvect}
\usepackage[framemethod=TikZ]{mdframed}
\usepackage{centernot}
\usepackage{diagbox}
\usepackage{dsfont}
\usepackage{fancyhdr}
\usepackage{float}
\usepackage{graphicx}
\usepackage{listings}
\usepackage{multicol}
\usepackage{nicematrix}
\usepackage{pdflscape}
\usepackage{stmaryrd}
\usepackage{xfrac}
\usepackage{hep-math-font}
\usepackage{amsthm}
\usepackage{thmtools}
\usepackage{indentfirst}
\usepackage[framemethod=TikZ]{mdframed}
\usepackage{accents}
\usepackage{soulutf8}
\usepackage{mathtools}
\usepackage{bodegraph}
\usepackage{slashbox}
\usepackage{enumitem}
\usepackage{calligra}
\usepackage{cinzel}
\usepackage{BOONDOX-calo}

% Tikz
\usetikzlibrary{babel}
\usetikzlibrary{positioning}
\usetikzlibrary{calc}

% global settings
\frenchspacing
\reversemarginpar
\setuldepth{a}

%\everymath{\displaystyle}

\frenchbsetup{StandardLists=true}

\def\asydir{asy}

%\sisetup{exponent-product=\cdot,output-decimal-marker={,},separate-uncertainty,range-phrase=\;à\;,locale=FR}

\setlength{\parskip}{1em}

\theoremstyle{definition}

% Changing math
\let\emptyset\varnothing
\let\ge\geqslant
\let\le\leqslant
\let\preceq\preccurlyeq
\let\succeq\succcurlyeq
\let\ds\displaystyle
\let\ts\textstyle

\newcommand{\C}{\mathds{C}}
\newcommand{\R}{\mathds{R}}
\newcommand{\Z}{\mathds{Z}}
\newcommand{\N}{\mathds{N}}
\newcommand{\Q}{\mathds{Q}}

\renewcommand{\O}{\emptyset}

\newcommand\ubar[1]{\underaccent{\bar}{#1}}

\renewcommand\Re{\expandafter\mathfrak{Re}}
\renewcommand\Im{\expandafter\mathfrak{Im}}

\let\slantedpartial\partial
\DeclareRobustCommand{\partial}{\text{\rotatebox[origin=t]{20}{\scalebox{0.95}[1]{$\slantedpartial$}}}\hspace{-1pt}}

% merging two maths characters w/ \charfusion
\makeatletter
\def\moverlay{\mathpalette\mov@rlay}
\def\mov@rlay#1#2{\leavevmode\vtop{%
   \baselineskip\z@skip \lineskiplimit-\maxdimen
   \ialign{\hfil$\m@th#1##$\hfil\cr#2\crcr}}}
\newcommand{\charfusion}[3][\mathord]{
    #1{\ifx#1\mathop\vphantom{#2}\fi
        \mathpalette\mov@rlay{#2\cr#3}
      }
    \ifx#1\mathop\expandafter\displaylimits\fi}
\makeatother

% custom math commands
\newcommand{\T}{{\!\!\,\top}}
\newcommand{\avrt}[1]{\rotatebox{-90}{$#1$}}
\newcommand{\bigcupdot}{\charfusion[\mathop]{\bigcup}{\cdot}}
\newcommand{\cupdot}{\charfusion[\mathbin]{\cup}{\cdot}}
%\newcommand{\danger}{{\large\fontencoding{U}\fontfamily{futs}\selectfont\char 66\relax}\;}
\newcommand{\tendsto}[1]{\xrightarrow[#1]{}}
\newcommand{\vrt}[1]{\rotatebox{90}{$#1$}}
\newcommand{\tsup}[1]{\textsuperscript{\underline{#1}}}
\newcommand{\tsub}[1]{\textsubscript{#1}}

\renewcommand{\mod}[1]{~\left[ #1 \right]}
\renewcommand{\t}{{}^t\!}
\newcommand{\s}{\text{\calligra s}}

% custom units / constants
%\DeclareSIUnit{\litre}{\ell}
\let\hbar\hslash

% header / footer
\pagestyle{fancy}
\fancyhead{} \fancyfoot{}
\fancyfoot[C]{\thepage}

% fonts
\let\sc\scshape
\let\bf\bfseries
\let\it\itshape
\let\sl\slshape

% custom math operators
\let\th\relax
\let\det\relax
\DeclareMathOperator*{\codim}{codim}
\DeclareMathOperator*{\dom}{dom}
\DeclareMathOperator*{\gO}{O}
\DeclareMathOperator*{\po}{\text{\cursive o}}
\DeclareMathOperator*{\sgn}{sgn}
\DeclareMathOperator*{\simi}{\sim}
\DeclareMathOperator{\Arccos}{Arccos}
\DeclareMathOperator{\Arcsin}{Arcsin}
\DeclareMathOperator{\Arctan}{Arctan}
\DeclareMathOperator{\Argsh}{Argsh}
\DeclareMathOperator{\Arg}{Arg}
\DeclareMathOperator{\Aut}{Aut}
\DeclareMathOperator{\Card}{Card}
\DeclareMathOperator{\Cl}{\mathcal{C}\!\ell}
\DeclareMathOperator{\Cov}{Cov}
\DeclareMathOperator{\Ker}{Ker}
\DeclareMathOperator{\Mat}{Mat}
\DeclareMathOperator{\PGCD}{PGCD}
\DeclareMathOperator{\PPCM}{PPCM}
\DeclareMathOperator{\Supp}{Supp}
\DeclareMathOperator{\Vect}{Vect}
\DeclareMathOperator{\argmax}{argmax}
\DeclareMathOperator{\argmin}{argmin}
\DeclareMathOperator{\ch}{ch}
\DeclareMathOperator{\com}{com}
\DeclareMathOperator{\cotan}{cotan}
\DeclareMathOperator{\det}{det}
\DeclareMathOperator{\id}{id}
\DeclareMathOperator{\rg}{rg}
\DeclareMathOperator{\rk}{rk}
\DeclareMathOperator{\sh}{sh}
\DeclareMathOperator{\th}{th}
\DeclareMathOperator{\tr}{tr}

% colors and page style
\definecolor{truewhite}{HTML}{ffffff}
\definecolor{white}{HTML}{faf4ed}
\definecolor{trueblack}{HTML}{000000}
\definecolor{black}{HTML}{575279}
\definecolor{mauve}{HTML}{907aa9}
\definecolor{blue}{HTML}{286983}
\definecolor{red}{HTML}{d7827e}
\definecolor{yellow}{HTML}{ea9d34}
\definecolor{gray}{HTML}{9893a5}
\definecolor{grey}{HTML}{9893a5}
\definecolor{green}{HTML}{a0d971}

\pagecolor{white}
\color{black}

\begin{asydef}
	settings.prc = false;
	settings.render=0;

	white = rgb("faf4ed");
	black = rgb("575279");
	blue = rgb("286983");
	red = rgb("d7827e");
	yellow = rgb("f6c177");
	orange = rgb("ea9d34");
	gray = rgb("9893a5");
	grey = rgb("9893a5");
	deepcyan = rgb("56949f");
	pink = rgb("b4637a");
	magenta = rgb("eb6f92");
	green = rgb("a0d971");
	purple = rgb("907aa9");

	defaultpen(black + fontsize(8pt));

	import three;
	currentlight = nolight;
\end{asydef}

% theorems, proofs, ...

\mdfsetup{skipabove=1em,skipbelow=1em, innertopmargin=6pt, innerbottommargin=6pt,}

\declaretheoremstyle[
	headfont=\normalfont\itshape,
	numbered=no,
	postheadspace=\newline,
	headpunct={:},
	qed=\qedsymbol]{demstyle}

\declaretheorem[style=demstyle, name=Démonstration]{dem}

\newcommand\veczero{\kern-1.2pt\vec{\kern1.2pt 0}} % \vec{0} looks weird since the `0' isn't italicized

\makeatletter
\renewcommand{\title}[2]{
	\AtBeginDocument{
		\begin{titlepage}
			\begin{center}
				\vspace{10cm}
				{\Large \sc Chapitre #1}\\
				\vspace{1cm}
				{\Huge \calligra #2}\\
				\vfill
				Hugo {\sc Salou} MPI${}^{\star}$\\
				{\small Dernière mise à jour le \@date }
			\end{center}
		\end{titlepage}
	}
}

\newcommand{\titletp}[4]{
	\AtBeginDocument{
		\begin{titlepage}
			\begin{center}
				\vspace{10cm}
				{\Large \sc tp #1}\\
				\vspace{1cm}
				{\Huge \textsc{\textit{#2}}}\\
				\vfill
				{#3}\textit{MPI}${}^{\star}$\\
			\end{center}
		\end{titlepage}
	}
	\fancyfoot{}\fancyhead{}
	\fancyfoot[R]{#4 \textit{MPI}${}^{\star}$}
	\fancyhead[C]{{\sc tp #1} : #2}
	\fancyhead[R]{\thepage}
}

\newcommand{\titletd}[2]{
	\AtBeginDocument{
		\begin{titlepage}
			\begin{center}
				\vspace{10cm}
				{\Large \sc td #1}\\
				\vspace{1cm}
				{\Huge \calligra #2}\\
				\vfill
				Hugo {\sc Salou} MPI${}^{\star}$\\
				{\small Dernière mise à jour le \@date }
			\end{center}
		\end{titlepage}
	}
}
\makeatother

\newcommand{\sign}{
	\null
	\vfill
	\begin{center}
		{
			\fontfamily{ccr}\selectfont
			\textit{\textbf{\.{\"i}}}
		}
	\end{center}
	\vfill
	\null
}

\renewcommand{\thefootnote}{\emph{\alph{footnote}}}

% figure support
\usepackage{import}
\usepackage{xifthen}
\pdfminorversion=7
\usepackage{pdfpages}
\usepackage{transparent}
\newcommand{\incfig}[1]{%
	\def\svgwidth{\columnwidth}
	\import{./figures/}{#1.pdf_tex}
}

\pdfsuppresswarningpagegroup=1
\ctikzset{tripoles/european not symbol=circle}

\newcommand{\missingpart}{{\large\color{red} Il manque quelque chose ici\ldots}}


\fancyhead[R]{Hugo {\sc Salou}\/ MPI}
\fancyhead[L]{TD\textsubscript2 -- Exercice 7}

\begin{document}
	\let\thesection\relax
	\section{Exercice 3}

{\bf Indication}\/ : pour la $G$, on applique la relation de {\sc Chasles}\/ : l'intégrale $\int_0^7 \mathrm{e}^{-x}\ln x~\mathrm{d}x$\/ converge si et seulement si $\int_0^1 \mathrm{e}^{-x}\ln x\mathrm{d}x$\/ converge et $\int_1^7\mathrm{e}^{-x}\ln x~\mathrm{d}x$\/ converge (qui n'est même pas impropre).

L'intégrale $H = \int_0^1 \frac{\mathrm{e}^{\sin t}}{t}~\mathrm{d}t$\/ est impropre en 0. On sait que $\sin t \tendsto{t\to 0} 0$, et donc, par continuité de la fonction $\exp$\/ en $0$, $\mathrm{e}^{\sin t}\tendsto{t\to 0}e^0 = 1$.
Ainsi, $\frac{\mathrm{e}^{\sin t}}{t} = \mathrm{e}^{\sin t} \times \frac{1}{t} \simi_{t\to 0} \frac{1}{t}$\/ qui ne change pas de signe. Or, $\int_0^1 \frac{1}{t}~\mathrm{d}t$\/ diverge, donc l'intégrale $H$\/ diverge.

L'intégrale $I = \int_{1}^{+\infty} \frac{\mathrm{e}^{\sin t}}{t}~\mathrm{d}t$\/ est impropre en $+\infty$. Par croissance de la fonction exponentielle, on a $\frac{\mathrm{e}^{\sin t}}{t} \ge \frac{\mathrm{e}^{-1}}{t} \ge 0$. Or, l'intégrale $\int_{1}^{+\infty} \frac{1}{t}~\mathrm{d}t$\/ diverge, donc l'intégrale diverge aussi.

L'intégrale $K$, est l'intégrale d'une fonction Gau\ss ienne, et elle est impropre en $+\infty$. On la \guillemotleft~découpe~\guillemotright\ : \[
	\int_{0}^{+\infty} \mathrm{e}^{-x^2}~\mathrm{d}x \text{ converge si et seulement si } \int_{0}^{1} \mathrm{e}^{-x^2}~\mathrm{d}x \text{ converge et } \int_{1}^{+\infty} \mathrm{e}^{-x^2}~\mathrm{d}x \text{ converge.}
\] L'intégrale $\int_{0}^{1} \mathrm{e}^{-x^2}~\mathrm{d}x$\/ n'est même pas impropre, elle converge donc. Et, pour $x \in [1,+\infty[$, on sait, comme $x^2 \ge x$, $0 \le \mathrm{e}^{-x^2} \le \mathrm{e}^{-x}$. Or, $\int_{1}^{+\infty} \mathrm{e}^{-x}~\mathrm{d}x$\/ converge donc $\int_{0}^{+\infty} \mathrm{e}^{-x^2}~\mathrm{d}x$\/ aussi.
On calculera la valeur de cette intégrale dans le {\sc td}\/ \guillemotleft~Intégrales paramétrées.~\guillemotright

Autre méthode pour déterminer la nature de $K$\/ : 
$\mathrm{e}^{-x^2} = \po(\mathrm{e}^{-x})$\/ car $\mathrm{e}^{-x^2} = \underbrace{\mathrm{e}^{-x^2 + x}}_{\to 0} \times \mathrm{e}^{-x}$, car $\mathrm{e}^{-x^2 + x} = \mathrm{e}^{-x^2 \left( 1 - \frac{1}{x} \right)}$\/ et $-x^2\left( 1 - \frac{1}{x} \right) \to -\infty \times 1$.
Et $\int_0^{+\infty} \mathrm{e}^{-x}~\mathrm{d}x$\/ converge donc $\int_0^{+\infty} \mathrm{e}^{-x^2}~\mathrm{d}x$\/ converge.

\begin{figure}[H]
	\centering
	\begin{asy}
		import graph;
		size(10cm);
		draw((-10, 0) -- (10, 0), Arrow(TeXHead));
		draw((0, -3) -- (0, 5), Arrow(TeXHead));
		real f(real x) {
			return 4*exp(-(x/4)^2);
		}
		draw(graph(f, -10, 10), magenta);
	\end{asy}
	\caption{Courbe Gau\ss ienne}
\end{figure}

L'intégrale $F = \int_{7}^{+\infty} \mathrm{e}^{-x}\ln x~\mathrm{d}x$\/ est impropre en $+\infty$. Attention : la fonction n'est pas \guillemotleft~{\color{red} faussement impropre en $+\infty$}.~\guillemotright\ Mais, on peut remarquer que \[
	\mathrm{e}^{-x} \ln x = \mathrm{e}^{-\frac{x}{2}} \underbrace{\mathrm{e}^{-\frac{x}{2}} \ln x}_{\tendsto{x\to +\infty} 0} = \po(\mathrm{e}^{-\frac{x}{2}})
.\] Or, $\int_{7}^{+\infty} \mathrm{e}^{-x}~\mathrm{d}x$\/ converge donc l'intégrale $F$\/ converge aussi.

L'intégrale $G = \int_{0}^{7} \mathrm{e}^{-x}\ln x~\mathrm{d}x$\/ est impropre en 0.
Or, $\mathrm{e}^{-x}\ln x \simi_{x\to 0} \ln x$\/ qui ne change pas de signe au voisinage de 0. Or, $\int_{0}^{7}  \ln x~\mathrm{d}x$\/ converge donc l'intégrale $G$\/ converge également.

L'intégrale $E = \int_{1}^{+\infty} \frac{\ln x}{\sqrt{x}}~\mathrm{d}x$\/ est impropre en $+\infty$. Or, $\forall x \ge \mathrm{e}$, $\frac{\ln(x)}{\sqrt{x}} \ge \frac{1}{\sqrt{x}} \ge 0$\/ converge.
Or, $\int_{1}^{+\infty}  \frac{1}{x^{\sfrac{1}{2}}}~\mathrm{d}x$\/ diverge d'après le critère de {\sc Riemann}\/ en $+\infty$\/ car $\frac{1}{2} < 1$.
D'où l'intégrale $D$\/ diverge.

Autre méthode : intégration par parties. On peut même arriver à calculer une primitive de ${\ln x}\:/{\sqrt{x}}$.

L'intégrale $D = \int_{0}^{1} \frac{\ln x}{\sqrt{x}}~\mathrm{d}x$\/ est impropre en 0. On peut remarque que \[
	0 \le -\frac{\ln x}{\sqrt{x}} = -\frac{x^{0{,}1} \ln x}{x^{0{,}6}} = \po\left( \frac{1}{x^{0{,}6}} \right) \quad\text{car}\quad x^{0{,}1} \ln x \tendsto{x\to 0} 0
\] par croissances comparées.
Or, $\int_{0}^{1} \frac{1}{x^{0{,}6}}~\mathrm{d}x$\/ converge d'après le critère de {\sc Riemann}. D'où $-D$\/ converge et donc $D$\/ converge.

L'intégrale $J = \int_{1}^{+\infty} \frac{\sin t}{\sqrt{t} + \sin t}~\mathrm{d}t$\/ est impropre en $+\infty$. On calcule
\[
	f(t) = \frac{\sin t}{\sqrt{t} + \sin t} = \frac{\sin t}{\sqrt{t}} \times \frac{1}{1+\frac{\sin t}{\sqrt{t}}}
\] et $\frac{\sin t}{\sqrt{t}} \tendsto{t\to +\infty} 0$. D'où \[
	\frac{1}{1+\frac{\sin t}{\sqrt{t}}} = 1  - \frac{\sin t}{\sqrt{t}} + \frac{\sin^2 t}{t} + \po\left( \frac{\sin^2 t}{t} \right)
\] et donc \[
	f(t) = \frac{\cos t}{\sqrt{t}} + \frac{\sin^2 t}{t} + \po\left( \frac{\sin^2 t}{t} \right)
.\]
L'intégrale $\int_{1}^{+\infty} \frac{\sin t}{\sqrt{t}}~\mathrm{d}t$\/ est impropre en $+\infty$.
Soit $x \ge 1$. On calcule avec une intégration par parties,
\[
	\int_{1}^{x} \sin t \times \frac{1}{\sqrt{t}}~\mathrm{d}t = \int_{1}^{x} u'(t)\cdot v(t)~\mathrm{d}t
\] où $u(t) = - \cos t$\/ et $v(t) = \frac{1}{\sqrt{t}} = t^{-\frac{1}{2}}$. Donc
\begin{align*}
	\int_{1}^{x} \frac{\sin t}{\sqrt{t}}~\mathrm{d}t &= \Big[f(t)g(t)\Big]_1^x - \int_{1}^{x} f(t)\cdot g'(t)~\mathrm{d}t\\
	&= \left[ - \frac{\cos t}{\sqrt{t}} \right]_1^x - \int_{1}^{x} (-\cos t)\left( -\frac{1}{2}t^{-\frac{3}{2}} \right)~\mathrm{d}t \\
\end{align*}
D'où \[
	\int_{1}^{x} \frac{\sin t}{\sqrt{t}}~\mathrm{d}t = \cos 1 - \frac{\cos x}{\sqrt{x}} - \frac{1}{2} \int_{1}^{x} \frac{\cos t}{t^{\sfrac{3}{2}}}~\mathrm{d}t
.\]
Or, d'une part $\cos x \times \frac{1}{\sqrt{x}} \tendsto{x\to +\infty} 0$\/ car $\cos$\/ est bornée et $\frac{1}{\sqrt{x}}\tendsto{x\to +\infty} 0$.
Et, d'autre part $\int_{1}^{+\infty} \frac{\cos t}{t^{\sfrac{3}{2}}}~\mathrm{d}t$\/ converge car $\forall t \in [1,+\infty[$, $\left| \frac{\cos t}{t^{\sfrac{3}{2}}} \right| \le \frac{1}{t^{\sfrac{3}{2}}}$\/ et $\int_{1}^{+\infty} \frac{1}{t^{\sfrac{3}{2}}}~\mathrm{d}t$\/ converge.
Pour le 2\tsup{nd} terme du développement limité, on fait une {\sc ipp}, on trouve un terme en $\frac{1}{t^2}$\/ et donc son intégrale converge par critère de {\sc Riemann}. S'il y a des problèmes, voir en {\sc td}.
On étudie maintenant le 3\tsup{ème} terme : \[
	\int_{1}^{+\infty}  \po\left( \frac{\sin t}{t} \right) ~\mathrm{d}t \text{ converge car } \int_{1}^{+\infty} \frac{\sin^2 t}{t} ~\mathrm{d}t \text{ converge et } t\mapsto \frac{\sin^2 t}{t} \text{ est positive}.
\]
Autre méthode : on a \[
	\frac{\sin^2 t}{t} + \po\left( \frac{\sin^2 t}{t} \right) \simi_{t\to +\infty} \frac{\sin^2 t}{t} \text{ qui ne change pas de signe}
.\] Or, $\int_{1}^{+\infty} \frac{\sin^2 t}{t}~\mathrm{d}t$\/ converge et donc \[
	\int_{1}^{+\infty} \left( \frac{\sin^2 t}{t} + \po\left( \frac{\sin^2 t}{t} \right) \right) ~\mathrm{d}t
.\]

\end{document}

		\section{Vocabulaire des automates}

On représente, ci-dessous, l'automate $\mathcal{A}$\/ décrit dans l'énoncé.
\begin{figure}[H]
	\centering
	\tikzfig{automate-ex8}
	\caption{Automate décrit dans l'énoncé de l'exercice 8}
\end{figure}

\begin{enumerate}
	\item Cet automate n'est pas complet : à l'état 0, la lecture d'un $a$\/ peut conduire à l'état 0 ou bien à l'état 1.
	\item Le mot $baba$\/ est reconnu par $\mathcal{A}$\/ mais pas le mot $cabcb$.
	\item L'automate reconnaît les mots dont la 3\tsup{ème} lettre du mot, en partant de la fin, est un $a$.
\end{enumerate}



	}
	\def\addmacros#1{#1}
}
{
	\td[15]{Grammaires non contextuelles (2)}
	\minitoc
	\renewcommand{\cwd}{../td/td15/}
	\addmacros{
		\section{Filtre RC double}

\begin{enumerate}
	\item En basse fréquence, un condensateur est équivalent à un interrupteur ouvert. En haute fréquence, un condensateur est équivalent à un interrupteur fermé. D'où, le circuit est un filtre passe-bas.
	\item Par une loi des nœuds, et une loi des mailles, on trouve que
		\[
			\ubar{H}(\mathrm{j}\omega) = \dfrac{1}{1 - \left( \dfrac{\omega}{\omega_0} \right)^2 + \mathrm{j} \dfrac{\omega}{Q\cdot \omega_0}}
		\] en notant $\omega_0 = 1 / {RC}$ et $Q = 1 / 3$
	\item On représente le diagramme de \textsc{Bode} du filtre dans la figure ci-dessous.
		\begin{figure}[H]
			\centering
			\includesvg[width=\linewidth]{figures/bode-1.svg}
			\caption{Diagramme de \textsc{Bode} du filtre (échelle logarithmique)}
		\end{figure}
	\item On calcule $\omega_0 \simeq 6\:\mathrm{rad/s}$, ce qui correspond à une fréquence de coupure de $1\:\mathrm{kHz}$. Le signal de sortie est donc \[
			s(t) = \frac{2E}{3\pi}\cdot \sin(\omega t)
		,\] on le représente sur la figure ci-dessous. En effet, on a un déphasage de $-\pi / 2$, et un gain valant $1 / 3$ à $\omega \simeq \omega_0$.
		\begin{figure}[H]
			\centering
			\includesvg[width=\linewidth]{figures/signal-1.svg}
			\caption{Signal résultant}
		\end{figure}
\end{enumerate}

		\section{Tri topologique}

\begin{figure}[H]
	\centering
	\tikzfig{ex2-q1}
	\caption{Exemple de graphe}
\end{figure}

\begin{enumerate}
	\item Dans le graphe ci-dessus, $a \to c \to b$\/ est un tri topologique mais pas un parcours.
	\item Dans le même graphe, $b \to a \to c$\/ est un parcours mais pas un tri topologique.
	\item Supposons que $L_1$\/ possède un prédécesseur, on le note $L_i$\/ où $i > 1$. Ainsi, $(L_i, L_1) \in A$\/ et donc $i < 1$, ce qui est absurde. De même pour le dernier.
	\item Il existe un tri topologique si, et seulement si le graphe est acyclique.
		\begin{itemize}
			\item[``$\implies$'']
				Soit $L_1,\ldots,L_n$\/ un tri topologique. Montrons que le graphe est acyclique.
				Par l'absurde, on suppose le graphe non acyclique : il existe $(i,j) \in \llbracket 1,n \rrbracket^2$\/ avec $i \neq j$\/ tels que $T_i \to \cdots \to T_j$\/ et $T_j \to \cdots \to T_i$ soient deux chemins valides. Ainsi, comme le tri est topologique et par récurrence, $i \le j$\/ et $j \le i$\/ et donc $i = j$, ce qui est absurde car $i$\/ et $j$\/ sont supposés différents. Le graphe est donc acyclique.
			\item[``$\impliedby$'']
				Soit $G$\/ un graphe tel que tous les sommets possèdent une arrête entrante. On suppose par l'absurde ce graphe acyclique.
				Soit $x_0$\/ un sommet du graphe.
				On construit par récurrence $x_0,x_1,\ldots,x_n,x_{n+1},\ldots$ les successeurs successifs. Il y a un nombre fini de sommets donc deux sommets sont identiques. Donc, il y a nécessairement un cycle, ce qui est absurde.
				\begin{algorithm}[H]
					\centering
					\begin{algorithmic}[1]
						\Entree $G = (S, A)$\/ un graphe acyclique
						\Sortie $\mathrm{Res}$\/ un tri topologique.
						\State $\mathrm{Res} \gets [\quad]$\/
						\While{$G \neq \O$}
							\State Soit $x$\/ un sommet de $G$\/ sans prédécesseur
							\State $G \gets \big(S \setminus \{x\}, A \cap (S \setminus \{x\})^2\big)$\/ 
							\State $\mathrm{Res} \gets \mathrm{Res} \cdot [x]$\/
						\EndWhile
						\State\Return $\mathrm{Res}$\/
					\end{algorithmic}
					\caption{Génération d'un tri topologique d'un graphe acyclique}
				\end{algorithm}
		\end{itemize}
	\item~
		\begin{algorithm}[H]
			\centering
			\begin{algorithmic}[1]
				\Entree $G = (S, A)$\/ un graphe
				\Sortie $\mathrm{Res}$\/ un tri topologique, ou un cycle
				\State $\mathrm{Res} \gets [\quad]$\/
				\While{$G \neq \O$}
					\If{il existe $x$ sans prédécesseurs}
						\State Soit $x$\/ un sommet de $G$\/ sans prédécesseur
						\State $G \gets \big(S \setminus \{x\}, A \cap (S \setminus \{x\})^2\big)$\/ 
						\State $\mathrm{Res} \gets \mathrm{Res} \cdot [x]$\/
					\Else
						\State Soit $x \in S$\/ 
						\State Soit $x \gets x_1 \gets x_2 \gets \cdots \gets x_i$\/ la suite des prédécesseurs
						\State\Return $x_i,x_{i+1},\ldots,x_i$, un cycle
					\EndIf
				\EndWhile
				\State\Return $\mathrm{Res}$\/
			\end{algorithmic}
			\caption{Génération d'un tri topologique d'un graphe}
		\end{algorithm}
	\item On utilise la représentation par liste d'adjacence, et on stocke le nombre de prédécesseurs que l'on décroit à chaque choix de sommet.
	\item On essaie de trouver un tri topologique, et on voit si l'on trouve un cycle.
\end{enumerate}

		\section{Formules duales}

\begin{enumerate}
	\item On définit par induction $(\cdot)^\star$\/ comme
		\begin{multicols}{3}
			\begin{itemize}
				\item $\top^\star = \bot$\/ ;
				\item $\bot^\star = \top$\/ ;
				\item $(G \lor H)^\star = G^\star \land H^\star$\/ ;
				\item $(G \land H)^\star = G^\star \lor H^\star$\/ ;
				\item $(\lnot G)^\star = \lnot G^\star$\/ ;
				\item $p^\star = p$.
			\end{itemize}
		\end{multicols}
	\item Soit $\rho \in \mathds{B}^{\mathcal{P}}$. Montrons, par induction, $P(H) : ``\left\llbracket H^\star \right\rrbracket^\rho = \left\llbracket \lnot H \right\rrbracket^{\bar \rho}"$\/ où $\bar{\rho} : p \mapsto \overline{\rho(p)}$.
		\begin{itemize}
			\item On a $\left\llbracket \bot^\star \right\rrbracket^\rho = \left\llbracket \top \right\rrbracket^\rho = \mathbf{V}$, et $\left\llbracket \lnot \bot \right\rrbracket^{\bar\rho} = \left\llbracket \top \right\rrbracket^{\bar\rho} = \mathbf{V}$, d'où $P(\bot)$.
			\item On a $\left\llbracket \top^\star \right\rrbracket^\rho = \left\llbracket \bot \right\rrbracket^\rho = \mathbf{F}$, et $\left\llbracket \lnot \top \right\rrbracket^{\bar\rho} = \left\llbracket \bot \right\rrbracket^{\bar\rho} = \mathbf{F}$, d'où $P(\top)$.
			\item Soit $p \in \mathcal{P}$. On a $\left\llbracket p^\star  \right\rrbracket^\rho = \left\llbracket p \right\rrbracket^\rho = \rho(p)$, et $\left\llbracket \lnot p \right\rrbracket^{\bar\rho} = \overline{\left\llbracket p \right\rrbracket^{\bar\rho}} = \overline{\bar\rho(p)} = \overline{\overline{\rho(p)}} = \rho(p)$, d'où~$P(p)$.
		\end{itemize}
		Soient $F$\/ et $G$\/ deux formules.
		\begin{itemize}
			\item On a
				\begin{align*}
					\left\llbracket (F \land G)^\star  \right\rrbracket^\rho &= \left\llbracket F^\star  \lor G^\star \right\rrbracket^\rho\\
					&= \left\llbracket F^\star \right\rrbracket^\rho + \left\llbracket G^\star \right\rrbracket^\rho \\
					&= \left\llbracket \lnot F \right\rrbracket^{\bar\rho} + \left\llbracket \lnot G \right\rrbracket^{\bar\rho} \\
					&= \left\llbracket \lnot F \lor \lnot G \right\rrbracket^{\bar\rho} \\
					&= \left\llbracket \lnot (F \land G) \right\rrbracket^{\bar\rho} \\
				\end{align*}
				d'où $P(F \land G)$.
			\item On a
				\begin{align*}
					\left\llbracket (F \lor G)^\star  \right\rrbracket^\rho &= \left\llbracket F^\star  \land G^\star \right\rrbracket^\rho\\
					&= \left\llbracket F^\star \right\rrbracket^\rho \cdot \left\llbracket G^\star \right\rrbracket^\rho \\
					&= \left\llbracket \lnot F \right\rrbracket^{\bar\rho} \cdot \left\llbracket \lnot G \right\rrbracket^{\bar\rho} \\
					&= \left\llbracket \lnot F \land \lnot G \right\rrbracket^{\bar\rho} \\
					&= \left\llbracket \lnot (F \lor G) \right\rrbracket^{\bar\rho} \\
				\end{align*}
				d'où $P(F \lor G)$.
			\item On a \[
					\left\llbracket (\lnot F)^\star  \right\rrbracket^\rho = \left\llbracket \lnot (F^\star) \right\rrbracket^\rho = \overline{\left\llbracket F^\star \right\rrbracket^\rho} = \overline{\left\llbracket \lnot F \right\rrbracket^{\bar\rho}} = \left\llbracket \lnot (\lnot F) \right\rrbracket^{\bar\rho}.
					\] d'où $P(\lnot F)$.
		\end{itemize}
		Par induction, on en conclut que $P(F)$\/ est vraie pour toute formule $F$.
	\item Soit $G$\/ une formule valide. Alors, par définition, $G \equiv \top$. Or, d'après la question précédente, $G^\star \equiv (\top)^\star = \bot$. Ainsi, $G^\star $\/ n'est pas satisfiable.
\end{enumerate}



		\section{Un lemme d'itération}

		\section{Ambigüité}

	}
	\def\addmacros#1{#1}
}
{
	\td[16]{Concurrence}
	\minitoc
	\renewcommand{\cwd}{../td/td16/}
	\addmacros{
		\section{Filtre RC double}

\begin{enumerate}
	\item En basse fréquence, un condensateur est équivalent à un interrupteur ouvert. En haute fréquence, un condensateur est équivalent à un interrupteur fermé. D'où, le circuit est un filtre passe-bas.
	\item Par une loi des nœuds, et une loi des mailles, on trouve que
		\[
			\ubar{H}(\mathrm{j}\omega) = \dfrac{1}{1 - \left( \dfrac{\omega}{\omega_0} \right)^2 + \mathrm{j} \dfrac{\omega}{Q\cdot \omega_0}}
		\] en notant $\omega_0 = 1 / {RC}$ et $Q = 1 / 3$
	\item On représente le diagramme de \textsc{Bode} du filtre dans la figure ci-dessous.
		\begin{figure}[H]
			\centering
			\includesvg[width=\linewidth]{figures/bode-1.svg}
			\caption{Diagramme de \textsc{Bode} du filtre (échelle logarithmique)}
		\end{figure}
	\item On calcule $\omega_0 \simeq 6\:\mathrm{rad/s}$, ce qui correspond à une fréquence de coupure de $1\:\mathrm{kHz}$. Le signal de sortie est donc \[
			s(t) = \frac{2E}{3\pi}\cdot \sin(\omega t)
		,\] on le représente sur la figure ci-dessous. En effet, on a un déphasage de $-\pi / 2$, et un gain valant $1 / 3$ à $\omega \simeq \omega_0$.
		\begin{figure}[H]
			\centering
			\includesvg[width=\linewidth]{figures/signal-1.svg}
			\caption{Signal résultant}
		\end{figure}
\end{enumerate}

		\section{Tri topologique}

\begin{figure}[H]
	\centering
	\tikzfig{ex2-q1}
	\caption{Exemple de graphe}
\end{figure}

\begin{enumerate}
	\item Dans le graphe ci-dessus, $a \to c \to b$\/ est un tri topologique mais pas un parcours.
	\item Dans le même graphe, $b \to a \to c$\/ est un parcours mais pas un tri topologique.
	\item Supposons que $L_1$\/ possède un prédécesseur, on le note $L_i$\/ où $i > 1$. Ainsi, $(L_i, L_1) \in A$\/ et donc $i < 1$, ce qui est absurde. De même pour le dernier.
	\item Il existe un tri topologique si, et seulement si le graphe est acyclique.
		\begin{itemize}
			\item[``$\implies$'']
				Soit $L_1,\ldots,L_n$\/ un tri topologique. Montrons que le graphe est acyclique.
				Par l'absurde, on suppose le graphe non acyclique : il existe $(i,j) \in \llbracket 1,n \rrbracket^2$\/ avec $i \neq j$\/ tels que $T_i \to \cdots \to T_j$\/ et $T_j \to \cdots \to T_i$ soient deux chemins valides. Ainsi, comme le tri est topologique et par récurrence, $i \le j$\/ et $j \le i$\/ et donc $i = j$, ce qui est absurde car $i$\/ et $j$\/ sont supposés différents. Le graphe est donc acyclique.
			\item[``$\impliedby$'']
				Soit $G$\/ un graphe tel que tous les sommets possèdent une arrête entrante. On suppose par l'absurde ce graphe acyclique.
				Soit $x_0$\/ un sommet du graphe.
				On construit par récurrence $x_0,x_1,\ldots,x_n,x_{n+1},\ldots$ les successeurs successifs. Il y a un nombre fini de sommets donc deux sommets sont identiques. Donc, il y a nécessairement un cycle, ce qui est absurde.
				\begin{algorithm}[H]
					\centering
					\begin{algorithmic}[1]
						\Entree $G = (S, A)$\/ un graphe acyclique
						\Sortie $\mathrm{Res}$\/ un tri topologique.
						\State $\mathrm{Res} \gets [\quad]$\/
						\While{$G \neq \O$}
							\State Soit $x$\/ un sommet de $G$\/ sans prédécesseur
							\State $G \gets \big(S \setminus \{x\}, A \cap (S \setminus \{x\})^2\big)$\/ 
							\State $\mathrm{Res} \gets \mathrm{Res} \cdot [x]$\/
						\EndWhile
						\State\Return $\mathrm{Res}$\/
					\end{algorithmic}
					\caption{Génération d'un tri topologique d'un graphe acyclique}
				\end{algorithm}
		\end{itemize}
	\item~
		\begin{algorithm}[H]
			\centering
			\begin{algorithmic}[1]
				\Entree $G = (S, A)$\/ un graphe
				\Sortie $\mathrm{Res}$\/ un tri topologique, ou un cycle
				\State $\mathrm{Res} \gets [\quad]$\/
				\While{$G \neq \O$}
					\If{il existe $x$ sans prédécesseurs}
						\State Soit $x$\/ un sommet de $G$\/ sans prédécesseur
						\State $G \gets \big(S \setminus \{x\}, A \cap (S \setminus \{x\})^2\big)$\/ 
						\State $\mathrm{Res} \gets \mathrm{Res} \cdot [x]$\/
					\Else
						\State Soit $x \in S$\/ 
						\State Soit $x \gets x_1 \gets x_2 \gets \cdots \gets x_i$\/ la suite des prédécesseurs
						\State\Return $x_i,x_{i+1},\ldots,x_i$, un cycle
					\EndIf
				\EndWhile
				\State\Return $\mathrm{Res}$\/
			\end{algorithmic}
			\caption{Génération d'un tri topologique d'un graphe}
		\end{algorithm}
	\item On utilise la représentation par liste d'adjacence, et on stocke le nombre de prédécesseurs que l'on décroit à chaque choix de sommet.
	\item On essaie de trouver un tri topologique, et on voit si l'on trouve un cycle.
\end{enumerate}

		\section{Formules duales}

\begin{enumerate}
	\item On définit par induction $(\cdot)^\star$\/ comme
		\begin{multicols}{3}
			\begin{itemize}
				\item $\top^\star = \bot$\/ ;
				\item $\bot^\star = \top$\/ ;
				\item $(G \lor H)^\star = G^\star \land H^\star$\/ ;
				\item $(G \land H)^\star = G^\star \lor H^\star$\/ ;
				\item $(\lnot G)^\star = \lnot G^\star$\/ ;
				\item $p^\star = p$.
			\end{itemize}
		\end{multicols}
	\item Soit $\rho \in \mathds{B}^{\mathcal{P}}$. Montrons, par induction, $P(H) : ``\left\llbracket H^\star \right\rrbracket^\rho = \left\llbracket \lnot H \right\rrbracket^{\bar \rho}"$\/ où $\bar{\rho} : p \mapsto \overline{\rho(p)}$.
		\begin{itemize}
			\item On a $\left\llbracket \bot^\star \right\rrbracket^\rho = \left\llbracket \top \right\rrbracket^\rho = \mathbf{V}$, et $\left\llbracket \lnot \bot \right\rrbracket^{\bar\rho} = \left\llbracket \top \right\rrbracket^{\bar\rho} = \mathbf{V}$, d'où $P(\bot)$.
			\item On a $\left\llbracket \top^\star \right\rrbracket^\rho = \left\llbracket \bot \right\rrbracket^\rho = \mathbf{F}$, et $\left\llbracket \lnot \top \right\rrbracket^{\bar\rho} = \left\llbracket \bot \right\rrbracket^{\bar\rho} = \mathbf{F}$, d'où $P(\top)$.
			\item Soit $p \in \mathcal{P}$. On a $\left\llbracket p^\star  \right\rrbracket^\rho = \left\llbracket p \right\rrbracket^\rho = \rho(p)$, et $\left\llbracket \lnot p \right\rrbracket^{\bar\rho} = \overline{\left\llbracket p \right\rrbracket^{\bar\rho}} = \overline{\bar\rho(p)} = \overline{\overline{\rho(p)}} = \rho(p)$, d'où~$P(p)$.
		\end{itemize}
		Soient $F$\/ et $G$\/ deux formules.
		\begin{itemize}
			\item On a
				\begin{align*}
					\left\llbracket (F \land G)^\star  \right\rrbracket^\rho &= \left\llbracket F^\star  \lor G^\star \right\rrbracket^\rho\\
					&= \left\llbracket F^\star \right\rrbracket^\rho + \left\llbracket G^\star \right\rrbracket^\rho \\
					&= \left\llbracket \lnot F \right\rrbracket^{\bar\rho} + \left\llbracket \lnot G \right\rrbracket^{\bar\rho} \\
					&= \left\llbracket \lnot F \lor \lnot G \right\rrbracket^{\bar\rho} \\
					&= \left\llbracket \lnot (F \land G) \right\rrbracket^{\bar\rho} \\
				\end{align*}
				d'où $P(F \land G)$.
			\item On a
				\begin{align*}
					\left\llbracket (F \lor G)^\star  \right\rrbracket^\rho &= \left\llbracket F^\star  \land G^\star \right\rrbracket^\rho\\
					&= \left\llbracket F^\star \right\rrbracket^\rho \cdot \left\llbracket G^\star \right\rrbracket^\rho \\
					&= \left\llbracket \lnot F \right\rrbracket^{\bar\rho} \cdot \left\llbracket \lnot G \right\rrbracket^{\bar\rho} \\
					&= \left\llbracket \lnot F \land \lnot G \right\rrbracket^{\bar\rho} \\
					&= \left\llbracket \lnot (F \lor G) \right\rrbracket^{\bar\rho} \\
				\end{align*}
				d'où $P(F \lor G)$.
			\item On a \[
					\left\llbracket (\lnot F)^\star  \right\rrbracket^\rho = \left\llbracket \lnot (F^\star) \right\rrbracket^\rho = \overline{\left\llbracket F^\star \right\rrbracket^\rho} = \overline{\left\llbracket \lnot F \right\rrbracket^{\bar\rho}} = \left\llbracket \lnot (\lnot F) \right\rrbracket^{\bar\rho}.
					\] d'où $P(\lnot F)$.
		\end{itemize}
		Par induction, on en conclut que $P(F)$\/ est vraie pour toute formule $F$.
	\item Soit $G$\/ une formule valide. Alors, par définition, $G \equiv \top$. Or, d'après la question précédente, $G^\star \equiv (\top)^\star = \bot$. Ainsi, $G^\star $\/ n'est pas satisfiable.
\end{enumerate}



		\section{Un lemme d'itération}

		\section{Ambigüité}

	}
	\def\addmacros#1{#1}
}
{
	\td[17]{Concurrence}
	\minitoc
	\renewcommand{\cwd}{../td/td17/}
	\addmacros{
		\section{Filtre RC double}

\begin{enumerate}
	\item En basse fréquence, un condensateur est équivalent à un interrupteur ouvert. En haute fréquence, un condensateur est équivalent à un interrupteur fermé. D'où, le circuit est un filtre passe-bas.
	\item Par une loi des nœuds, et une loi des mailles, on trouve que
		\[
			\ubar{H}(\mathrm{j}\omega) = \dfrac{1}{1 - \left( \dfrac{\omega}{\omega_0} \right)^2 + \mathrm{j} \dfrac{\omega}{Q\cdot \omega_0}}
		\] en notant $\omega_0 = 1 / {RC}$ et $Q = 1 / 3$
	\item On représente le diagramme de \textsc{Bode} du filtre dans la figure ci-dessous.
		\begin{figure}[H]
			\centering
			\includesvg[width=\linewidth]{figures/bode-1.svg}
			\caption{Diagramme de \textsc{Bode} du filtre (échelle logarithmique)}
		\end{figure}
	\item On calcule $\omega_0 \simeq 6\:\mathrm{rad/s}$, ce qui correspond à une fréquence de coupure de $1\:\mathrm{kHz}$. Le signal de sortie est donc \[
			s(t) = \frac{2E}{3\pi}\cdot \sin(\omega t)
		,\] on le représente sur la figure ci-dessous. En effet, on a un déphasage de $-\pi / 2$, et un gain valant $1 / 3$ à $\omega \simeq \omega_0$.
		\begin{figure}[H]
			\centering
			\includesvg[width=\linewidth]{figures/signal-1.svg}
			\caption{Signal résultant}
		\end{figure}
\end{enumerate}

		\section{Tri topologique}

\begin{figure}[H]
	\centering
	\tikzfig{ex2-q1}
	\caption{Exemple de graphe}
\end{figure}

\begin{enumerate}
	\item Dans le graphe ci-dessus, $a \to c \to b$\/ est un tri topologique mais pas un parcours.
	\item Dans le même graphe, $b \to a \to c$\/ est un parcours mais pas un tri topologique.
	\item Supposons que $L_1$\/ possède un prédécesseur, on le note $L_i$\/ où $i > 1$. Ainsi, $(L_i, L_1) \in A$\/ et donc $i < 1$, ce qui est absurde. De même pour le dernier.
	\item Il existe un tri topologique si, et seulement si le graphe est acyclique.
		\begin{itemize}
			\item[``$\implies$'']
				Soit $L_1,\ldots,L_n$\/ un tri topologique. Montrons que le graphe est acyclique.
				Par l'absurde, on suppose le graphe non acyclique : il existe $(i,j) \in \llbracket 1,n \rrbracket^2$\/ avec $i \neq j$\/ tels que $T_i \to \cdots \to T_j$\/ et $T_j \to \cdots \to T_i$ soient deux chemins valides. Ainsi, comme le tri est topologique et par récurrence, $i \le j$\/ et $j \le i$\/ et donc $i = j$, ce qui est absurde car $i$\/ et $j$\/ sont supposés différents. Le graphe est donc acyclique.
			\item[``$\impliedby$'']
				Soit $G$\/ un graphe tel que tous les sommets possèdent une arrête entrante. On suppose par l'absurde ce graphe acyclique.
				Soit $x_0$\/ un sommet du graphe.
				On construit par récurrence $x_0,x_1,\ldots,x_n,x_{n+1},\ldots$ les successeurs successifs. Il y a un nombre fini de sommets donc deux sommets sont identiques. Donc, il y a nécessairement un cycle, ce qui est absurde.
				\begin{algorithm}[H]
					\centering
					\begin{algorithmic}[1]
						\Entree $G = (S, A)$\/ un graphe acyclique
						\Sortie $\mathrm{Res}$\/ un tri topologique.
						\State $\mathrm{Res} \gets [\quad]$\/
						\While{$G \neq \O$}
							\State Soit $x$\/ un sommet de $G$\/ sans prédécesseur
							\State $G \gets \big(S \setminus \{x\}, A \cap (S \setminus \{x\})^2\big)$\/ 
							\State $\mathrm{Res} \gets \mathrm{Res} \cdot [x]$\/
						\EndWhile
						\State\Return $\mathrm{Res}$\/
					\end{algorithmic}
					\caption{Génération d'un tri topologique d'un graphe acyclique}
				\end{algorithm}
		\end{itemize}
	\item~
		\begin{algorithm}[H]
			\centering
			\begin{algorithmic}[1]
				\Entree $G = (S, A)$\/ un graphe
				\Sortie $\mathrm{Res}$\/ un tri topologique, ou un cycle
				\State $\mathrm{Res} \gets [\quad]$\/
				\While{$G \neq \O$}
					\If{il existe $x$ sans prédécesseurs}
						\State Soit $x$\/ un sommet de $G$\/ sans prédécesseur
						\State $G \gets \big(S \setminus \{x\}, A \cap (S \setminus \{x\})^2\big)$\/ 
						\State $\mathrm{Res} \gets \mathrm{Res} \cdot [x]$\/
					\Else
						\State Soit $x \in S$\/ 
						\State Soit $x \gets x_1 \gets x_2 \gets \cdots \gets x_i$\/ la suite des prédécesseurs
						\State\Return $x_i,x_{i+1},\ldots,x_i$, un cycle
					\EndIf
				\EndWhile
				\State\Return $\mathrm{Res}$\/
			\end{algorithmic}
			\caption{Génération d'un tri topologique d'un graphe}
		\end{algorithm}
	\item On utilise la représentation par liste d'adjacence, et on stocke le nombre de prédécesseurs que l'on décroit à chaque choix de sommet.
	\item On essaie de trouver un tri topologique, et on voit si l'on trouve un cycle.
\end{enumerate}

		\section{Formules duales}

\begin{enumerate}
	\item On définit par induction $(\cdot)^\star$\/ comme
		\begin{multicols}{3}
			\begin{itemize}
				\item $\top^\star = \bot$\/ ;
				\item $\bot^\star = \top$\/ ;
				\item $(G \lor H)^\star = G^\star \land H^\star$\/ ;
				\item $(G \land H)^\star = G^\star \lor H^\star$\/ ;
				\item $(\lnot G)^\star = \lnot G^\star$\/ ;
				\item $p^\star = p$.
			\end{itemize}
		\end{multicols}
	\item Soit $\rho \in \mathds{B}^{\mathcal{P}}$. Montrons, par induction, $P(H) : ``\left\llbracket H^\star \right\rrbracket^\rho = \left\llbracket \lnot H \right\rrbracket^{\bar \rho}"$\/ où $\bar{\rho} : p \mapsto \overline{\rho(p)}$.
		\begin{itemize}
			\item On a $\left\llbracket \bot^\star \right\rrbracket^\rho = \left\llbracket \top \right\rrbracket^\rho = \mathbf{V}$, et $\left\llbracket \lnot \bot \right\rrbracket^{\bar\rho} = \left\llbracket \top \right\rrbracket^{\bar\rho} = \mathbf{V}$, d'où $P(\bot)$.
			\item On a $\left\llbracket \top^\star \right\rrbracket^\rho = \left\llbracket \bot \right\rrbracket^\rho = \mathbf{F}$, et $\left\llbracket \lnot \top \right\rrbracket^{\bar\rho} = \left\llbracket \bot \right\rrbracket^{\bar\rho} = \mathbf{F}$, d'où $P(\top)$.
			\item Soit $p \in \mathcal{P}$. On a $\left\llbracket p^\star  \right\rrbracket^\rho = \left\llbracket p \right\rrbracket^\rho = \rho(p)$, et $\left\llbracket \lnot p \right\rrbracket^{\bar\rho} = \overline{\left\llbracket p \right\rrbracket^{\bar\rho}} = \overline{\bar\rho(p)} = \overline{\overline{\rho(p)}} = \rho(p)$, d'où~$P(p)$.
		\end{itemize}
		Soient $F$\/ et $G$\/ deux formules.
		\begin{itemize}
			\item On a
				\begin{align*}
					\left\llbracket (F \land G)^\star  \right\rrbracket^\rho &= \left\llbracket F^\star  \lor G^\star \right\rrbracket^\rho\\
					&= \left\llbracket F^\star \right\rrbracket^\rho + \left\llbracket G^\star \right\rrbracket^\rho \\
					&= \left\llbracket \lnot F \right\rrbracket^{\bar\rho} + \left\llbracket \lnot G \right\rrbracket^{\bar\rho} \\
					&= \left\llbracket \lnot F \lor \lnot G \right\rrbracket^{\bar\rho} \\
					&= \left\llbracket \lnot (F \land G) \right\rrbracket^{\bar\rho} \\
				\end{align*}
				d'où $P(F \land G)$.
			\item On a
				\begin{align*}
					\left\llbracket (F \lor G)^\star  \right\rrbracket^\rho &= \left\llbracket F^\star  \land G^\star \right\rrbracket^\rho\\
					&= \left\llbracket F^\star \right\rrbracket^\rho \cdot \left\llbracket G^\star \right\rrbracket^\rho \\
					&= \left\llbracket \lnot F \right\rrbracket^{\bar\rho} \cdot \left\llbracket \lnot G \right\rrbracket^{\bar\rho} \\
					&= \left\llbracket \lnot F \land \lnot G \right\rrbracket^{\bar\rho} \\
					&= \left\llbracket \lnot (F \lor G) \right\rrbracket^{\bar\rho} \\
				\end{align*}
				d'où $P(F \lor G)$.
			\item On a \[
					\left\llbracket (\lnot F)^\star  \right\rrbracket^\rho = \left\llbracket \lnot (F^\star) \right\rrbracket^\rho = \overline{\left\llbracket F^\star \right\rrbracket^\rho} = \overline{\left\llbracket \lnot F \right\rrbracket^{\bar\rho}} = \left\llbracket \lnot (\lnot F) \right\rrbracket^{\bar\rho}.
					\] d'où $P(\lnot F)$.
		\end{itemize}
		Par induction, on en conclut que $P(F)$\/ est vraie pour toute formule $F$.
	\item Soit $G$\/ une formule valide. Alors, par définition, $G \equiv \top$. Or, d'après la question précédente, $G^\star \equiv (\top)^\star = \bot$. Ainsi, $G^\star $\/ n'est pas satisfiable.
\end{enumerate}



	}
	\def\addmacros#1{#1}
}
\def\prefix{\textsc{td bonus}}
{
	\td[1]{Complexité amortie}
	\minitoc
	\renewcommand{\cwd}{../td/td_b1/}
	\addmacros{
		\section{Filtre RC double}

\begin{enumerate}
	\item En basse fréquence, un condensateur est équivalent à un interrupteur ouvert. En haute fréquence, un condensateur est équivalent à un interrupteur fermé. D'où, le circuit est un filtre passe-bas.
	\item Par une loi des nœuds, et une loi des mailles, on trouve que
		\[
			\ubar{H}(\mathrm{j}\omega) = \dfrac{1}{1 - \left( \dfrac{\omega}{\omega_0} \right)^2 + \mathrm{j} \dfrac{\omega}{Q\cdot \omega_0}}
		\] en notant $\omega_0 = 1 / {RC}$ et $Q = 1 / 3$
	\item On représente le diagramme de \textsc{Bode} du filtre dans la figure ci-dessous.
		\begin{figure}[H]
			\centering
			\includesvg[width=\linewidth]{figures/bode-1.svg}
			\caption{Diagramme de \textsc{Bode} du filtre (échelle logarithmique)}
		\end{figure}
	\item On calcule $\omega_0 \simeq 6\:\mathrm{rad/s}$, ce qui correspond à une fréquence de coupure de $1\:\mathrm{kHz}$. Le signal de sortie est donc \[
			s(t) = \frac{2E}{3\pi}\cdot \sin(\omega t)
		,\] on le représente sur la figure ci-dessous. En effet, on a un déphasage de $-\pi / 2$, et un gain valant $1 / 3$ à $\omega \simeq \omega_0$.
		\begin{figure}[H]
			\centering
			\includesvg[width=\linewidth]{figures/signal-1.svg}
			\caption{Signal résultant}
		\end{figure}
\end{enumerate}

		\section{Tri topologique}

\begin{figure}[H]
	\centering
	\tikzfig{ex2-q1}
	\caption{Exemple de graphe}
\end{figure}

\begin{enumerate}
	\item Dans le graphe ci-dessus, $a \to c \to b$\/ est un tri topologique mais pas un parcours.
	\item Dans le même graphe, $b \to a \to c$\/ est un parcours mais pas un tri topologique.
	\item Supposons que $L_1$\/ possède un prédécesseur, on le note $L_i$\/ où $i > 1$. Ainsi, $(L_i, L_1) \in A$\/ et donc $i < 1$, ce qui est absurde. De même pour le dernier.
	\item Il existe un tri topologique si, et seulement si le graphe est acyclique.
		\begin{itemize}
			\item[``$\implies$'']
				Soit $L_1,\ldots,L_n$\/ un tri topologique. Montrons que le graphe est acyclique.
				Par l'absurde, on suppose le graphe non acyclique : il existe $(i,j) \in \llbracket 1,n \rrbracket^2$\/ avec $i \neq j$\/ tels que $T_i \to \cdots \to T_j$\/ et $T_j \to \cdots \to T_i$ soient deux chemins valides. Ainsi, comme le tri est topologique et par récurrence, $i \le j$\/ et $j \le i$\/ et donc $i = j$, ce qui est absurde car $i$\/ et $j$\/ sont supposés différents. Le graphe est donc acyclique.
			\item[``$\impliedby$'']
				Soit $G$\/ un graphe tel que tous les sommets possèdent une arrête entrante. On suppose par l'absurde ce graphe acyclique.
				Soit $x_0$\/ un sommet du graphe.
				On construit par récurrence $x_0,x_1,\ldots,x_n,x_{n+1},\ldots$ les successeurs successifs. Il y a un nombre fini de sommets donc deux sommets sont identiques. Donc, il y a nécessairement un cycle, ce qui est absurde.
				\begin{algorithm}[H]
					\centering
					\begin{algorithmic}[1]
						\Entree $G = (S, A)$\/ un graphe acyclique
						\Sortie $\mathrm{Res}$\/ un tri topologique.
						\State $\mathrm{Res} \gets [\quad]$\/
						\While{$G \neq \O$}
							\State Soit $x$\/ un sommet de $G$\/ sans prédécesseur
							\State $G \gets \big(S \setminus \{x\}, A \cap (S \setminus \{x\})^2\big)$\/ 
							\State $\mathrm{Res} \gets \mathrm{Res} \cdot [x]$\/
						\EndWhile
						\State\Return $\mathrm{Res}$\/
					\end{algorithmic}
					\caption{Génération d'un tri topologique d'un graphe acyclique}
				\end{algorithm}
		\end{itemize}
	\item~
		\begin{algorithm}[H]
			\centering
			\begin{algorithmic}[1]
				\Entree $G = (S, A)$\/ un graphe
				\Sortie $\mathrm{Res}$\/ un tri topologique, ou un cycle
				\State $\mathrm{Res} \gets [\quad]$\/
				\While{$G \neq \O$}
					\If{il existe $x$ sans prédécesseurs}
						\State Soit $x$\/ un sommet de $G$\/ sans prédécesseur
						\State $G \gets \big(S \setminus \{x\}, A \cap (S \setminus \{x\})^2\big)$\/ 
						\State $\mathrm{Res} \gets \mathrm{Res} \cdot [x]$\/
					\Else
						\State Soit $x \in S$\/ 
						\State Soit $x \gets x_1 \gets x_2 \gets \cdots \gets x_i$\/ la suite des prédécesseurs
						\State\Return $x_i,x_{i+1},\ldots,x_i$, un cycle
					\EndIf
				\EndWhile
				\State\Return $\mathrm{Res}$\/
			\end{algorithmic}
			\caption{Génération d'un tri topologique d'un graphe}
		\end{algorithm}
	\item On utilise la représentation par liste d'adjacence, et on stocke le nombre de prédécesseurs que l'on décroit à chaque choix de sommet.
	\item On essaie de trouver un tri topologique, et on voit si l'on trouve un cycle.
\end{enumerate}

		\section{Formules duales}

\begin{enumerate}
	\item On définit par induction $(\cdot)^\star$\/ comme
		\begin{multicols}{3}
			\begin{itemize}
				\item $\top^\star = \bot$\/ ;
				\item $\bot^\star = \top$\/ ;
				\item $(G \lor H)^\star = G^\star \land H^\star$\/ ;
				\item $(G \land H)^\star = G^\star \lor H^\star$\/ ;
				\item $(\lnot G)^\star = \lnot G^\star$\/ ;
				\item $p^\star = p$.
			\end{itemize}
		\end{multicols}
	\item Soit $\rho \in \mathds{B}^{\mathcal{P}}$. Montrons, par induction, $P(H) : ``\left\llbracket H^\star \right\rrbracket^\rho = \left\llbracket \lnot H \right\rrbracket^{\bar \rho}"$\/ où $\bar{\rho} : p \mapsto \overline{\rho(p)}$.
		\begin{itemize}
			\item On a $\left\llbracket \bot^\star \right\rrbracket^\rho = \left\llbracket \top \right\rrbracket^\rho = \mathbf{V}$, et $\left\llbracket \lnot \bot \right\rrbracket^{\bar\rho} = \left\llbracket \top \right\rrbracket^{\bar\rho} = \mathbf{V}$, d'où $P(\bot)$.
			\item On a $\left\llbracket \top^\star \right\rrbracket^\rho = \left\llbracket \bot \right\rrbracket^\rho = \mathbf{F}$, et $\left\llbracket \lnot \top \right\rrbracket^{\bar\rho} = \left\llbracket \bot \right\rrbracket^{\bar\rho} = \mathbf{F}$, d'où $P(\top)$.
			\item Soit $p \in \mathcal{P}$. On a $\left\llbracket p^\star  \right\rrbracket^\rho = \left\llbracket p \right\rrbracket^\rho = \rho(p)$, et $\left\llbracket \lnot p \right\rrbracket^{\bar\rho} = \overline{\left\llbracket p \right\rrbracket^{\bar\rho}} = \overline{\bar\rho(p)} = \overline{\overline{\rho(p)}} = \rho(p)$, d'où~$P(p)$.
		\end{itemize}
		Soient $F$\/ et $G$\/ deux formules.
		\begin{itemize}
			\item On a
				\begin{align*}
					\left\llbracket (F \land G)^\star  \right\rrbracket^\rho &= \left\llbracket F^\star  \lor G^\star \right\rrbracket^\rho\\
					&= \left\llbracket F^\star \right\rrbracket^\rho + \left\llbracket G^\star \right\rrbracket^\rho \\
					&= \left\llbracket \lnot F \right\rrbracket^{\bar\rho} + \left\llbracket \lnot G \right\rrbracket^{\bar\rho} \\
					&= \left\llbracket \lnot F \lor \lnot G \right\rrbracket^{\bar\rho} \\
					&= \left\llbracket \lnot (F \land G) \right\rrbracket^{\bar\rho} \\
				\end{align*}
				d'où $P(F \land G)$.
			\item On a
				\begin{align*}
					\left\llbracket (F \lor G)^\star  \right\rrbracket^\rho &= \left\llbracket F^\star  \land G^\star \right\rrbracket^\rho\\
					&= \left\llbracket F^\star \right\rrbracket^\rho \cdot \left\llbracket G^\star \right\rrbracket^\rho \\
					&= \left\llbracket \lnot F \right\rrbracket^{\bar\rho} \cdot \left\llbracket \lnot G \right\rrbracket^{\bar\rho} \\
					&= \left\llbracket \lnot F \land \lnot G \right\rrbracket^{\bar\rho} \\
					&= \left\llbracket \lnot (F \lor G) \right\rrbracket^{\bar\rho} \\
				\end{align*}
				d'où $P(F \lor G)$.
			\item On a \[
					\left\llbracket (\lnot F)^\star  \right\rrbracket^\rho = \left\llbracket \lnot (F^\star) \right\rrbracket^\rho = \overline{\left\llbracket F^\star \right\rrbracket^\rho} = \overline{\left\llbracket \lnot F \right\rrbracket^{\bar\rho}} = \left\llbracket \lnot (\lnot F) \right\rrbracket^{\bar\rho}.
					\] d'où $P(\lnot F)$.
		\end{itemize}
		Par induction, on en conclut que $P(F)$\/ est vraie pour toute formule $F$.
	\item Soit $G$\/ une formule valide. Alors, par définition, $G \equiv \top$. Or, d'après la question précédente, $G^\star \equiv (\top)^\star = \bot$. Ainsi, $G^\star $\/ n'est pas satisfiable.
\end{enumerate}



		\section{Un lemme d'itération}

	}
	\def\addmacros#1{#1}
}
\def\prefix{\textsc{td bonus}}
{
	\td[2]{Diviser pour régner}
	\minitoc
	\renewcommand{\cwd}{../td/td_b2/}
	\addmacros{
		\section{Filtre RC double}

\begin{enumerate}
	\item En basse fréquence, un condensateur est équivalent à un interrupteur ouvert. En haute fréquence, un condensateur est équivalent à un interrupteur fermé. D'où, le circuit est un filtre passe-bas.
	\item Par une loi des nœuds, et une loi des mailles, on trouve que
		\[
			\ubar{H}(\mathrm{j}\omega) = \dfrac{1}{1 - \left( \dfrac{\omega}{\omega_0} \right)^2 + \mathrm{j} \dfrac{\omega}{Q\cdot \omega_0}}
		\] en notant $\omega_0 = 1 / {RC}$ et $Q = 1 / 3$
	\item On représente le diagramme de \textsc{Bode} du filtre dans la figure ci-dessous.
		\begin{figure}[H]
			\centering
			\includesvg[width=\linewidth]{figures/bode-1.svg}
			\caption{Diagramme de \textsc{Bode} du filtre (échelle logarithmique)}
		\end{figure}
	\item On calcule $\omega_0 \simeq 6\:\mathrm{rad/s}$, ce qui correspond à une fréquence de coupure de $1\:\mathrm{kHz}$. Le signal de sortie est donc \[
			s(t) = \frac{2E}{3\pi}\cdot \sin(\omega t)
		,\] on le représente sur la figure ci-dessous. En effet, on a un déphasage de $-\pi / 2$, et un gain valant $1 / 3$ à $\omega \simeq \omega_0$.
		\begin{figure}[H]
			\centering
			\includesvg[width=\linewidth]{figures/signal-1.svg}
			\caption{Signal résultant}
		\end{figure}
\end{enumerate}

		\section{Tri topologique}

\begin{figure}[H]
	\centering
	\tikzfig{ex2-q1}
	\caption{Exemple de graphe}
\end{figure}

\begin{enumerate}
	\item Dans le graphe ci-dessus, $a \to c \to b$\/ est un tri topologique mais pas un parcours.
	\item Dans le même graphe, $b \to a \to c$\/ est un parcours mais pas un tri topologique.
	\item Supposons que $L_1$\/ possède un prédécesseur, on le note $L_i$\/ où $i > 1$. Ainsi, $(L_i, L_1) \in A$\/ et donc $i < 1$, ce qui est absurde. De même pour le dernier.
	\item Il existe un tri topologique si, et seulement si le graphe est acyclique.
		\begin{itemize}
			\item[``$\implies$'']
				Soit $L_1,\ldots,L_n$\/ un tri topologique. Montrons que le graphe est acyclique.
				Par l'absurde, on suppose le graphe non acyclique : il existe $(i,j) \in \llbracket 1,n \rrbracket^2$\/ avec $i \neq j$\/ tels que $T_i \to \cdots \to T_j$\/ et $T_j \to \cdots \to T_i$ soient deux chemins valides. Ainsi, comme le tri est topologique et par récurrence, $i \le j$\/ et $j \le i$\/ et donc $i = j$, ce qui est absurde car $i$\/ et $j$\/ sont supposés différents. Le graphe est donc acyclique.
			\item[``$\impliedby$'']
				Soit $G$\/ un graphe tel que tous les sommets possèdent une arrête entrante. On suppose par l'absurde ce graphe acyclique.
				Soit $x_0$\/ un sommet du graphe.
				On construit par récurrence $x_0,x_1,\ldots,x_n,x_{n+1},\ldots$ les successeurs successifs. Il y a un nombre fini de sommets donc deux sommets sont identiques. Donc, il y a nécessairement un cycle, ce qui est absurde.
				\begin{algorithm}[H]
					\centering
					\begin{algorithmic}[1]
						\Entree $G = (S, A)$\/ un graphe acyclique
						\Sortie $\mathrm{Res}$\/ un tri topologique.
						\State $\mathrm{Res} \gets [\quad]$\/
						\While{$G \neq \O$}
							\State Soit $x$\/ un sommet de $G$\/ sans prédécesseur
							\State $G \gets \big(S \setminus \{x\}, A \cap (S \setminus \{x\})^2\big)$\/ 
							\State $\mathrm{Res} \gets \mathrm{Res} \cdot [x]$\/
						\EndWhile
						\State\Return $\mathrm{Res}$\/
					\end{algorithmic}
					\caption{Génération d'un tri topologique d'un graphe acyclique}
				\end{algorithm}
		\end{itemize}
	\item~
		\begin{algorithm}[H]
			\centering
			\begin{algorithmic}[1]
				\Entree $G = (S, A)$\/ un graphe
				\Sortie $\mathrm{Res}$\/ un tri topologique, ou un cycle
				\State $\mathrm{Res} \gets [\quad]$\/
				\While{$G \neq \O$}
					\If{il existe $x$ sans prédécesseurs}
						\State Soit $x$\/ un sommet de $G$\/ sans prédécesseur
						\State $G \gets \big(S \setminus \{x\}, A \cap (S \setminus \{x\})^2\big)$\/ 
						\State $\mathrm{Res} \gets \mathrm{Res} \cdot [x]$\/
					\Else
						\State Soit $x \in S$\/ 
						\State Soit $x \gets x_1 \gets x_2 \gets \cdots \gets x_i$\/ la suite des prédécesseurs
						\State\Return $x_i,x_{i+1},\ldots,x_i$, un cycle
					\EndIf
				\EndWhile
				\State\Return $\mathrm{Res}$\/
			\end{algorithmic}
			\caption{Génération d'un tri topologique d'un graphe}
		\end{algorithm}
	\item On utilise la représentation par liste d'adjacence, et on stocke le nombre de prédécesseurs que l'on décroit à chaque choix de sommet.
	\item On essaie de trouver un tri topologique, et on voit si l'on trouve un cycle.
\end{enumerate}

	}
	\def\addmacros#1{#1}
}
\def\prefix{\textsc{td bonus}}
{
	\td[3]{Invariants plus complexes}
	\minitoc
	\renewcommand{\cwd}{../td/td_b3/}
	\addmacros{
	}
	\def\addmacros#1{#1}
}


\part{Travaux Pratiques}
\def\prefix{\textsc{tp}}
\renewcommand{\chaptername}{Travaux pratiques}


{
	\tp[1]{Logique propositionnelle}
	\minitoc
	\renewcommand{\cwd}{../tps/tp01/}
	\addmacros{
		\section{Filtre RC double}

\begin{enumerate}
	\item En basse fréquence, un condensateur est équivalent à un interrupteur ouvert. En haute fréquence, un condensateur est équivalent à un interrupteur fermé. D'où, le circuit est un filtre passe-bas.
	\item Par une loi des nœuds, et une loi des mailles, on trouve que
		\[
			\ubar{H}(\mathrm{j}\omega) = \dfrac{1}{1 - \left( \dfrac{\omega}{\omega_0} \right)^2 + \mathrm{j} \dfrac{\omega}{Q\cdot \omega_0}}
		\] en notant $\omega_0 = 1 / {RC}$ et $Q = 1 / 3$
	\item On représente le diagramme de \textsc{Bode} du filtre dans la figure ci-dessous.
		\begin{figure}[H]
			\centering
			\includesvg[width=\linewidth]{figures/bode-1.svg}
			\caption{Diagramme de \textsc{Bode} du filtre (échelle logarithmique)}
		\end{figure}
	\item On calcule $\omega_0 \simeq 6\:\mathrm{rad/s}$, ce qui correspond à une fréquence de coupure de $1\:\mathrm{kHz}$. Le signal de sortie est donc \[
			s(t) = \frac{2E}{3\pi}\cdot \sin(\omega t)
		,\] on le représente sur la figure ci-dessous. En effet, on a un déphasage de $-\pi / 2$, et un gain valant $1 / 3$ à $\omega \simeq \omega_0$.
		\begin{figure}[H]
			\centering
			\includesvg[width=\linewidth]{figures/signal-1.svg}
			\caption{Signal résultant}
		\end{figure}
\end{enumerate}

		\section{Tri topologique}

\begin{figure}[H]
	\centering
	\tikzfig{ex2-q1}
	\caption{Exemple de graphe}
\end{figure}

\begin{enumerate}
	\item Dans le graphe ci-dessus, $a \to c \to b$\/ est un tri topologique mais pas un parcours.
	\item Dans le même graphe, $b \to a \to c$\/ est un parcours mais pas un tri topologique.
	\item Supposons que $L_1$\/ possède un prédécesseur, on le note $L_i$\/ où $i > 1$. Ainsi, $(L_i, L_1) \in A$\/ et donc $i < 1$, ce qui est absurde. De même pour le dernier.
	\item Il existe un tri topologique si, et seulement si le graphe est acyclique.
		\begin{itemize}
			\item[``$\implies$'']
				Soit $L_1,\ldots,L_n$\/ un tri topologique. Montrons que le graphe est acyclique.
				Par l'absurde, on suppose le graphe non acyclique : il existe $(i,j) \in \llbracket 1,n \rrbracket^2$\/ avec $i \neq j$\/ tels que $T_i \to \cdots \to T_j$\/ et $T_j \to \cdots \to T_i$ soient deux chemins valides. Ainsi, comme le tri est topologique et par récurrence, $i \le j$\/ et $j \le i$\/ et donc $i = j$, ce qui est absurde car $i$\/ et $j$\/ sont supposés différents. Le graphe est donc acyclique.
			\item[``$\impliedby$'']
				Soit $G$\/ un graphe tel que tous les sommets possèdent une arrête entrante. On suppose par l'absurde ce graphe acyclique.
				Soit $x_0$\/ un sommet du graphe.
				On construit par récurrence $x_0,x_1,\ldots,x_n,x_{n+1},\ldots$ les successeurs successifs. Il y a un nombre fini de sommets donc deux sommets sont identiques. Donc, il y a nécessairement un cycle, ce qui est absurde.
				\begin{algorithm}[H]
					\centering
					\begin{algorithmic}[1]
						\Entree $G = (S, A)$\/ un graphe acyclique
						\Sortie $\mathrm{Res}$\/ un tri topologique.
						\State $\mathrm{Res} \gets [\quad]$\/
						\While{$G \neq \O$}
							\State Soit $x$\/ un sommet de $G$\/ sans prédécesseur
							\State $G \gets \big(S \setminus \{x\}, A \cap (S \setminus \{x\})^2\big)$\/ 
							\State $\mathrm{Res} \gets \mathrm{Res} \cdot [x]$\/
						\EndWhile
						\State\Return $\mathrm{Res}$\/
					\end{algorithmic}
					\caption{Génération d'un tri topologique d'un graphe acyclique}
				\end{algorithm}
		\end{itemize}
	\item~
		\begin{algorithm}[H]
			\centering
			\begin{algorithmic}[1]
				\Entree $G = (S, A)$\/ un graphe
				\Sortie $\mathrm{Res}$\/ un tri topologique, ou un cycle
				\State $\mathrm{Res} \gets [\quad]$\/
				\While{$G \neq \O$}
					\If{il existe $x$ sans prédécesseurs}
						\State Soit $x$\/ un sommet de $G$\/ sans prédécesseur
						\State $G \gets \big(S \setminus \{x\}, A \cap (S \setminus \{x\})^2\big)$\/ 
						\State $\mathrm{Res} \gets \mathrm{Res} \cdot [x]$\/
					\Else
						\State Soit $x \in S$\/ 
						\State Soit $x \gets x_1 \gets x_2 \gets \cdots \gets x_i$\/ la suite des prédécesseurs
						\State\Return $x_i,x_{i+1},\ldots,x_i$, un cycle
					\EndIf
				\EndWhile
				\State\Return $\mathrm{Res}$\/
			\end{algorithmic}
			\caption{Génération d'un tri topologique d'un graphe}
		\end{algorithm}
	\item On utilise la représentation par liste d'adjacence, et on stocke le nombre de prédécesseurs que l'on décroit à chaque choix de sommet.
	\item On essaie de trouver un tri topologique, et on voit si l'on trouve un cycle.
\end{enumerate}

		\section{Formules duales}

\begin{enumerate}
	\item On définit par induction $(\cdot)^\star$\/ comme
		\begin{multicols}{3}
			\begin{itemize}
				\item $\top^\star = \bot$\/ ;
				\item $\bot^\star = \top$\/ ;
				\item $(G \lor H)^\star = G^\star \land H^\star$\/ ;
				\item $(G \land H)^\star = G^\star \lor H^\star$\/ ;
				\item $(\lnot G)^\star = \lnot G^\star$\/ ;
				\item $p^\star = p$.
			\end{itemize}
		\end{multicols}
	\item Soit $\rho \in \mathds{B}^{\mathcal{P}}$. Montrons, par induction, $P(H) : ``\left\llbracket H^\star \right\rrbracket^\rho = \left\llbracket \lnot H \right\rrbracket^{\bar \rho}"$\/ où $\bar{\rho} : p \mapsto \overline{\rho(p)}$.
		\begin{itemize}
			\item On a $\left\llbracket \bot^\star \right\rrbracket^\rho = \left\llbracket \top \right\rrbracket^\rho = \mathbf{V}$, et $\left\llbracket \lnot \bot \right\rrbracket^{\bar\rho} = \left\llbracket \top \right\rrbracket^{\bar\rho} = \mathbf{V}$, d'où $P(\bot)$.
			\item On a $\left\llbracket \top^\star \right\rrbracket^\rho = \left\llbracket \bot \right\rrbracket^\rho = \mathbf{F}$, et $\left\llbracket \lnot \top \right\rrbracket^{\bar\rho} = \left\llbracket \bot \right\rrbracket^{\bar\rho} = \mathbf{F}$, d'où $P(\top)$.
			\item Soit $p \in \mathcal{P}$. On a $\left\llbracket p^\star  \right\rrbracket^\rho = \left\llbracket p \right\rrbracket^\rho = \rho(p)$, et $\left\llbracket \lnot p \right\rrbracket^{\bar\rho} = \overline{\left\llbracket p \right\rrbracket^{\bar\rho}} = \overline{\bar\rho(p)} = \overline{\overline{\rho(p)}} = \rho(p)$, d'où~$P(p)$.
		\end{itemize}
		Soient $F$\/ et $G$\/ deux formules.
		\begin{itemize}
			\item On a
				\begin{align*}
					\left\llbracket (F \land G)^\star  \right\rrbracket^\rho &= \left\llbracket F^\star  \lor G^\star \right\rrbracket^\rho\\
					&= \left\llbracket F^\star \right\rrbracket^\rho + \left\llbracket G^\star \right\rrbracket^\rho \\
					&= \left\llbracket \lnot F \right\rrbracket^{\bar\rho} + \left\llbracket \lnot G \right\rrbracket^{\bar\rho} \\
					&= \left\llbracket \lnot F \lor \lnot G \right\rrbracket^{\bar\rho} \\
					&= \left\llbracket \lnot (F \land G) \right\rrbracket^{\bar\rho} \\
				\end{align*}
				d'où $P(F \land G)$.
			\item On a
				\begin{align*}
					\left\llbracket (F \lor G)^\star  \right\rrbracket^\rho &= \left\llbracket F^\star  \land G^\star \right\rrbracket^\rho\\
					&= \left\llbracket F^\star \right\rrbracket^\rho \cdot \left\llbracket G^\star \right\rrbracket^\rho \\
					&= \left\llbracket \lnot F \right\rrbracket^{\bar\rho} \cdot \left\llbracket \lnot G \right\rrbracket^{\bar\rho} \\
					&= \left\llbracket \lnot F \land \lnot G \right\rrbracket^{\bar\rho} \\
					&= \left\llbracket \lnot (F \lor G) \right\rrbracket^{\bar\rho} \\
				\end{align*}
				d'où $P(F \lor G)$.
			\item On a \[
					\left\llbracket (\lnot F)^\star  \right\rrbracket^\rho = \left\llbracket \lnot (F^\star) \right\rrbracket^\rho = \overline{\left\llbracket F^\star \right\rrbracket^\rho} = \overline{\left\llbracket \lnot F \right\rrbracket^{\bar\rho}} = \left\llbracket \lnot (\lnot F) \right\rrbracket^{\bar\rho}.
					\] d'où $P(\lnot F)$.
		\end{itemize}
		Par induction, on en conclut que $P(F)$\/ est vraie pour toute formule $F$.
	\item Soit $G$\/ une formule valide. Alors, par définition, $G \equiv \top$. Or, d'après la question précédente, $G^\star \equiv (\top)^\star = \bot$. Ainsi, $G^\star $\/ n'est pas satisfiable.
\end{enumerate}



	}
	\def\addmacros#1{#1}
}
{
	% No `main.tex' file for tp02
	\def\addmacros#1{#1}
}
{
	\tp[3]{Langages et expressions régulières (2)}
	\minitoc
	\renewcommand{\cwd}{../tps/tp03/}
	\addmacros{
	}
	\def\addmacros#1{#1}
}
{
	% No `main.tex' file for tp04
	\def\addmacros#1{#1}
}
{
	% No `main.tex' file for tp05
	\def\addmacros#1{#1}
}
{
	% No `main.tex' file for tp06
	\def\addmacros#1{#1}
}
{
	\tp[7]{Algorithme de Kosaraju en \textsc{OCaml}}
	\minitoc
	\renewcommand{\cwd}{../tps/tp07/}
	\addmacros{
		\section{Filtre RC double}

\begin{enumerate}
	\item En basse fréquence, un condensateur est équivalent à un interrupteur ouvert. En haute fréquence, un condensateur est équivalent à un interrupteur fermé. D'où, le circuit est un filtre passe-bas.
	\item Par une loi des nœuds, et une loi des mailles, on trouve que
		\[
			\ubar{H}(\mathrm{j}\omega) = \dfrac{1}{1 - \left( \dfrac{\omega}{\omega_0} \right)^2 + \mathrm{j} \dfrac{\omega}{Q\cdot \omega_0}}
		\] en notant $\omega_0 = 1 / {RC}$ et $Q = 1 / 3$
	\item On représente le diagramme de \textsc{Bode} du filtre dans la figure ci-dessous.
		\begin{figure}[H]
			\centering
			\includesvg[width=\linewidth]{figures/bode-1.svg}
			\caption{Diagramme de \textsc{Bode} du filtre (échelle logarithmique)}
		\end{figure}
	\item On calcule $\omega_0 \simeq 6\:\mathrm{rad/s}$, ce qui correspond à une fréquence de coupure de $1\:\mathrm{kHz}$. Le signal de sortie est donc \[
			s(t) = \frac{2E}{3\pi}\cdot \sin(\omega t)
		,\] on le représente sur la figure ci-dessous. En effet, on a un déphasage de $-\pi / 2$, et un gain valant $1 / 3$ à $\omega \simeq \omega_0$.
		\begin{figure}[H]
			\centering
			\includesvg[width=\linewidth]{figures/signal-1.svg}
			\caption{Signal résultant}
		\end{figure}
\end{enumerate}

	}
	\def\addmacros#1{#1}
}
{
	% No `main.tex' file for tp08
	\def\addmacros#1{#1}
}
{
	% No `main.tex' file for tp09
	\def\addmacros#1{#1}
}
{
	% No `main.tex' file for tp10
	\def\addmacros#1{#1}
}
\def\prefix{\textsc{tp bonus}}
{
	% No `main.tex' file for tp_b1
	\def\addmacros#1{#1}
}
\def\prefix{\textsc{tp bonus}}
{
	% No `main.tex' file for tp_b2
	\def\addmacros#1{#1}
}


\part{Annexes}
\def\prefix{Annexe}
\renewcommand{\chaptername}{Annexe}
\useroman


{
	\chap[1]{Complexité amortie}
	\minitoc
	\renewcommand{\cwd}{../cours/annexeA/}
	\addmacros{
		Avec une fonction $\mathrm{PA}(n)$\/ ayant une complexité en $\mathcal{O}(f(n))$, on considère le problème ci-dessous.
		\begin{lstlisting}[language=caml]
			for i = 0 to n - 1 do
				%*$\mathrm{PA}$*)(i)
			done
		\end{lstlisting}
		Cet algorithme a une complexité en $\mathcal{O}(n f(n))$. Mais, parfois, cette complexité est trop approximative : parfois, des sommes mathématiques se compensent : \[
			\sum_{i=0}^{n-1} 2^i = 2^n \neq \cancel{\Theta}(n 2^n)
		.\]
	
		On considère le problème ci-dessous.
		\begin{lstlisting}[language=c]
			for (int i = 0; i < n; i = i + 1) (
				// calcul qui ne coute pas cher
			}
		\end{lstlisting}
		Le calcul \texttt{i = i + 1} est parfois plus coûteux que le calcul dans la boucle.
		Par exemple, un algorithme permettant de faire ce calcul est celui ci-dessous.
		\begin{algorithm}[H]
			\centering
			\begin{algorithmic}[1]
				\Entree un entier $n$, représenté sous la forme d'un tableau de \textit{bit} $T$\/ (où les \textit{bit}s de poids forts sont à droite
				\State $I \gets \mathrm{len}(T) - 1$\/ 
				\While{$T[I] = 1$}
				\State $T[I] \gets 0$
				\State $I \gets I - 1$
				\EndWhile
				\State $T[I] \gets 1$
			\end{algorithmic}
			\caption{Calcul de $n + 1$\/ avec un tableau de \textit{bit}s}
		\end{algorithm}
		Avec un tel algorithme, on a une complexité, dans le pire des cas, en $\Theta(\log_2 n)$.
		Ainsi, en modifiant le code, on peut avoir une complexité importante.
		\begin{lstlisting}[language=c]
			// n est une valeur donnee par l'utilisateur
			for (int i = 0; i < 2%*$^\texttt{n}$*); i = i + 1) (
				// calcul qui ne coute pas cher
			}
		\end{lstlisting}
		La complexité de cet algorithme, que l'on nommera $\mathcal{A}$\/ dans la suite,  est en $\mathcal{O}(\texttt{n}\:2^\texttt{n})$, car le \texttt{i = i + 1} coûte, au pire des cas, $\texttt{n}$.
		En réalisant des mesures, et en graphant le temps de cet algorithme divisé par $2^\texttt{n}$, on remarque que ce le ratio n'est pas une droite de coefficient directeur $\texttt{n}$, mais une constante (à partir d'un certain rang).
		On doit donc faire une étude plus précise de la complexité, et faire le calcul de la somme plus proprement.
		Une étude plus fine nous montre que l'algorithme est beaucoup plus long pour les entiers en plaisances de deux, mais les autres nombres, on n'a pas besoin d'autant de calcul.
		Faisons cette étude plus fine.
	
		On nomme $\mathcal{T}$\/ l'ensemble des tableaux de taille $n$\/ contenant des valeurs dans $\{0,1\}$. On a, \[
			\mathrm{Co\hat ut}_\mathcal{A}(n) = \sum_{t \in \mathcal{T}} \mathrm{Co\hat ut}_\text{Incr}(t)
		.\]Partitionnons $\mathcal{T}$\/ : \[
			\mathcal{T}_i = \Bigg\{
				\begin{array}{|c|c|c|c|c|c|}
					\hline
					\ldots & i + 1 & i & i - 1 & \ldots & 0\\ \hline
					\ldots & \mathbf{0} & \mathbf{1} & \mathbf{1} & \ldots & \mathbf{1}\\ \hline
				\end{array} \in \mathcal{T}
			\Bigg\}
		.\]  Si $t \in \mathcal{T}_i$, on sait que $\mathrm{Co\hat ut}_\text{Incr}(t) = i + 1$.
		Ainsi,
		\begin{align*}
			\sum_{t \in \mathcal{T}} \mathrm{Co\hat ut}_\text{Incr}(t)
			&= \sum_{i=0}^{n-1} \sum_{t \in \mathcal{T}_i} \mathrm{Co\hat ut}_\text{Incr}(t) \\
			&= \sum_{i=0}^{n-1} |\mathcal{T}_i| \cdot (i + 1) \\
			&= \sum_{i=0}^{n-1} 2^{n - 1 - i}\:(i+1) \\
			&= 2^n \sum_{i=1}^n \frac{i}{2^i}\\
			&= 2^n \times \mathcal{O}(1) \\
			&= \mathcal{O}(2^n) \\
		\end{align*}
		ce qui explique les résultats trouvés précédemment.
		L'incrementation \texttt{i = i + 1} est donc en $\mathcal{O}(1)$, et non en $\mathcal{O}(\log_2 \texttt{i})$.
	
		\begin{defn}
			Étant donnée une structure (des éléments d'un type de données abstrait,~\textsc{tda}), $\mathds{F}$, munie d'opérations $\mathds{O}$ opérant sur $\mathds{F}$\/ munis de fonctions de coût \[
				\forall o \in \mathds{O}, \quad C_o : \mathds{F} \to \R^+
			.\]
			Étant donné un élément initial $f_0 \in \mathds{F}$\/ et une suite d'opérations $(o_1, \ldots, o_n) \in \mathds{O}^n$, cela conduit donc à une suite d'éléments \[
				f_0 \overset{o_1}\leadsto f_1 \overset{o_2}\leadsto f_2 \leadsto \cdots \leadsto f_n.
			\] On appelle \textit{complexité} de cette séquence, notée $\tilde c$, \[
				\tilde C\big((o_1, \ldots, o_n), f_0\big)  = \sum_{i=0}^n C_{o_i}(f_{i-1})
			.\]
			On appelle alors \textit{complexité amortie} depuis $f_0 \in \mathds{F}$\/ la suite \[
				C_\mathrm{A}(f_0, n) = \frac{1}{n} \sup_{(o_1, \ldots, o_n) \in \mathds{O}^n} \tilde C\big((o_1, \ldots, o_n), f_0\big)
			.\]
		\end{defn}
	
		\begin{exm}[Tableaux dynamiques]
			On s'intéresse aux tableaux à longueur variable : on alloue un tableau de petite taille, et on alloue plus de mémoire au besoin.
			On a une structure de tableau dynamique :
			\begin{algorithm}[H]
				\centering
				\begin{algorithmic}[1]
					\State Soit $\mathrm{taille}' = f\big(\mathrm{len}(T)\big)$ \Comment{$f$ reste à déterminer}
					\State On alloue $T'$\/ de taille $\mathrm{taille}'$
					\State On recopie $T$\/ dans $T'$\/
					\State $T \gets T'$\/
				\end{algorithmic}
				\caption{$\textsc{Agrandit}(T)$, fonction agrandissant le tableau $t$\/}
			\end{algorithm}
			Cet algorithme a une complexité $\mathrm{Co\hat ut}_\textsc{Agrandit}(n) = n + f(n)$.
			On suppose que le tableau $T$\/ est rempli jusqu'à la $r$-ième case.
			\begin{algorithm}[H]
				\centering
				\begin{algorithmic}[1]
					\If{$\mathrm{len}(T) = r$}
					\State $\textsc{Agrandit}(T)$
					\EndIf
					\State $T[r] \gets x$
					\State $r \gets r + 1$
				\end{algorithmic}
				\caption{$\textsc{Ajout}(T,x)$, ajout d'un élément dans le tableau}
			\end{algorithm}
			On choisit la fonction $f$.
			\begin{description}
				\item[Cas 1] On choisit $f(n) = n + 1$. Soit une suite de $n$\/ opérations \textsc{Ajout} depuis un tableau de taille 1, où $r = 0$. Ainsi, \[
						f_0\overset{\textsc{Ajout}}\leadsto f_1 \overset{\textsc{Ajout}}\leadsto \cdots \leadsto f_i \leadsto \cdots \leadsto f_{n-1} \overset{\textsc{Ajout}}\leadsto f_n
					.\] La complexité de cette suite d'opérations est \[
						\tilde C\big((o_1, \ldots, o_n), f_0) = n + 2 \cdot \frac{n(n+1)}{2},
					\] d'où la complexité amortie est de $C_\mathrm{A}(f_0, n) = \Theta(n)$.
				\item[Cas 2] On choisit $f(n) = 2n$. On somme les complexités : $2n$\/ (clairement par dessin). Ainsi, $C_\mathrm{A}(f_0, n) = \Theta(1)$.
			\end{description}
		\end{exm}
	
		\begin{met}[du potentiel]
			Considérons une fonction $h : \mathds{F} \to \R^+$\/ dite \textit{de potentiel} telle que $h(f_0) = 0$.
			Intéressons nous alors à $\ubar{C}_o(f) = C_o(f) + h(\bar{f}) - h(f)$, où $f \overset o\leadsto \bar{f}$.
			Soit alors \[
				f_0 \overset{o_1}\leadsto f_1 \overset{o_1}\leadsto f_2 \leadsto \cdots \leadsto f_n
			\]une suite d'opérations. Alors,
			\begin{align*}
				\sum_{i=1}^n \ubar{C}_{o_i}(f_{i-1}) &= \sum_{i=1}^n \Big(C_{o_i}(f_{i-1}) + h(f_i) - h(f_{i-1})\Big)\\
				&= \bigg(\sum_{i=1}^n C_{o_i}(f_{i-1})\bigg) + \underbrace{h(f_n) - h(f_0)}_{\ge  0} \\
			\end{align*}
			par télescopage. Ainsi, \[
				\sum_{i=1}^n C_{o_i}(f_{i-1}) \le \sum_{i=1}^n \ubar{C}_{o_i}(f_{i-1})
			.\]
		\end{met}
	
		\begin{exm}
			On applique la méthode du potentiel au cas 2 de l'exemple ci-avant.
			On rappelle que $\mathds{F}$\/ est l'ensemble des tableaux. On pose la fonction \begin{align*}
				h: \mathds{F} &\longrightarrow \R^+ \\
				(T, r) &\longmapsto 6\left(r - \frac{\mathrm{len}(T)}{2}\right)
			\end{align*}
			Inspectons alors \[
				\ubar{C}_\textsc{Ajout}(T, r) = C_\textsc{Ajout}(\ubar{T}, \ubar{r}) + 6\bar{r} - 3\:\mathrm{len}(\bar{T}) - 3\ubar{r} + 3\:\mathrm{len}(\ubar{T})
			.\]
			Si $\mathrm{len}(\ubar{T}) = \ubar{r}$, alors $C_\textsc{Ajout}(\ubar{T}, \ubar{r}) = 3\:\mathrm{len}(\ubar{T})$\/ et $\mathrm{len}(\bar{T}) = 2\: \mathrm{len}(\ubar{T})$\/ et $\bar{r} = \ubar{r} + 1$.
			D'où, \[
				\ubar{C}_\textsc{Ajout}(\ubar{T}, \ubar{r}) = 3\:\mathrm{len}(\ubar{T}) + 6\ubar{r} + 6 - 6\:\mathrm{len}(\ubar{T}) - 6\ubar{r} + 3\:\mathrm{len}(\ubar{T}) = 6
			.\] 
			Sinon, $\mathrm{len}(\ubar{T}) > \ubar{r}$, alors $\mathrm{len}(\bar{T}) = \mathrm{len}(\ubar{T})$\/ et $\bar{r} = \ubar{r} + 1$.
			Ainsi,
			\begin{align*}
				\ubar{C}_\textsc{Ajout}(\ubar{T}, \ubar{r}) = 1 + 6 (\ubar{r} + 1) - 6\:\mathrm{len}(\bar{T}) - 6 \ubar{r} + 6\:\mathrm{len}(\ubar{T}) = 7
			\end{align*}
			D'où \[
				\sum_{i=1}^n C_{o_i}(f_{i-1}) \le \sum_{i=1}^n \ubar{C}_{o_i}(f_{i-1}) \le 7n
			.\] Le coût amorti est en $\mathcal{O}(1)$.
		\end{exm}
		\begin{exm}[Méthode du Banquier]
			On encode une file avec deux piles.
			Au moment de défiler, on doit potentiellement transvaser une pile dans une autre.
			Avec la méthode du Banquier, on a l'\textit{intuition} que le coût amorti est constant.
	
			On pose $\mathds{F}$\/ l'ensemble des couples de piles $(p_1, p_2)$.
			On a \[
				C_\text{défiler}\big((p_1, p_2)\big) = \begin{cases}
					\mathrm{taille}\ p_1 + 1 \quad& \text{si $p_2$ est vide}\\
					1 \quad& \text{ sinon}
				\end{cases}, \text{ et } C_\text{enfiler}\big((p_1, p_2)\big) = 1
			.\] Soit $h$\/ la fonction de potentiel définie comme \begin{align*}
				h: \mathds{F} &\longrightarrow \R^+ \\
				(p_1, p_2) &\longmapsto \mathrm{taille}\ p_1
			\end{align*}
			Étudions alors $\ubar{C}_\text{défiler}\big((p_1, p_2)\big)$.
			\begin{itemize}
				\item Si $p_2$\/ est vide, alors
					\begin{align*}
						\ubar{C}_\text{défiler}\big((p_1, p_2)\big) &= C_\text{défiler}\big((p_1,p_2)\big) + h\big((\bar{p}_1, \bar{p}_2)\big) - h\big((p_1, p_2)\big)\\
						&= \mathrm{taille} \ p_1 + 1 + \overbrace{\mathrm{taille}\ \bar{p}_1}^{=0} - \mathrm{taille}\ p_1 \\
						&= 1.
					\end{align*}
				\item Si $p_2$\/ n'est pas vide, alors \[
						\ubar{C}_\text{défiler}\big((p_1, p_2)\big) = 1 + \underbrace{\mathrm{taille}\  \bar{p}_1}_{\substack{\ds=\\ \ds \mathrm{taille}\ p_1}} - \mathrm{taille}\ p_1
					.\]
			\end{itemize}
			D'où, pour $(p_1, p_2) \in \mathds{F}$, $\ubar{C}_ \text{défiler}\big((p_1, p_2)\big) \le 1$.
			De plus,
			\begin{align*}
				\ubar{C}_ \text{ enfiler}\big((p_1, p_2)\big)
				&= C_ \text{enfiler}\big((p_1, p_2)\big) + h\big((\bar{p}_1, \bar{p}_2)\big) - h\big((p_1, p_2)\big) \\
				&= 1 + \mathrm{taille}\ \bar{p}_1 - \mathrm{taille}\ p_1 \\
				&= 2 \\
			\end{align*}
			Finalement, pour toute séquence d'opérations $o_1, \ldots, o_n$\/ initialisée à la file vide $f_0$, on a
			\begin{align*}
				\frac{1}{n} \tilde C\big((o_1,\ldots,o_n), f_0\big)
				&= \frac{1}{n} \sum_{i=0}^n C_{o_i}(f_{i-1}) \\
				&\le \frac{1}{n} \sum_{i=1}^n \ubar{C}_{o_i}(f_{i-1}) \\
				&\le 2
			\end{align*}
			D'où, un coût amorti constant.
		\end{exm}
	}
	\def\addmacros#1{#1}
}
{
	\chap[2]{Algorithmes \textsc{Dijkstra} et $A^*$}
	\minitoc
	\renewcommand{\cwd}{../cours/annexeB/}
	\addmacros{
		On s'intéresse, dans cette annexe, à l'algorithme $A^*$.
		Cette annexe se situe à l'intersection des chapitres sur les graphes, et sur les jeux.
		L'algorithme $A^*$ est une modification de l'algorithme de \textsc{Dijkstra}.
		Dans cette annexe, on prouvera la correction de l'algorithme $A^*$.
	
		On se place dans le contexte d'exécution d'un algorithme de calcul de plus cours chemin utilisant un tableau de distances $\mu$, et le manipulant en n'effectuant que des opérations \textsc{Relâcher}.
		Notons le graphe $G = (V,E)$, le sommet source $s$.
		Notons également $d(\cdot,\cdot)$ la distance induite par les arêtes du graphe $G$.
		De plus, on notera $c(\cdot,\cdot)$ les coûts (positifs, non nuls) d'une arête de $G$.
		Notons $\ell(\cdot)$ les rongeurs des chemins.
	
		\begin{numlem}
			\[
				\forall (u,v) \in E,\quad d(s,v) \le d(s, u) + c(u,v)
			.\]
		\end{numlem}
	
		\begin{prv}
			Soit $(u,v) \in E$.
			Soit $\gamma_u$ un plus court chemin de $s$ à $u$.
			Alors, $\gamma_u \cdot v$ est un chemin de $s$ à $v$ :
			\[
				\ell(\gamma_u \cdot v) = \ell(\gamma_u) + c(u,v) = d(s,u) + c(u,v) \ge d(s,v).
			\]
		\end{prv}
	
		\begin{numlem}
			Pour tout sommet $u$, la valeur de $\mu[u]$ est décroissant à mesure que l'algorithme s'exécute.
		\end{numlem}
	
		\begin{prv}
			Soit $\ubar{\mu}$ et $\bar\mu$ les valeurs de $\mu$ avant et après une opération $\textsc{Relâcher}(x,y)$.
			Pour tout sommet $v \neq y$, $\bar\mu[v] = \ubar\mu[v]$.
			De plus, par disjonction de cas,
			\begin{itemize}
				\item ou bien $\bar\mu[y] = \ubar\mu[y]$, \textsc{ok}.
				\item ou bien $\bar\mu[y] = \ubar\mu[x] + c(x,y)$ lorsque $\ubar\mu[x] + c(x,y) \le \ubar\mu[y]$, donc $\bar\mu[y] \le \ubar\mu[y]$, \textsc{ok}.
			\end{itemize}
		\end{prv}
	
		\begin{numlem}
			Supposons que l'algorithme ait initialisé $\mu$ de la manière suivante : \[
				\forall u \in V,\quad\quad \mu[u] = \begin{cases}
					+\infty & \text{ si } u \neq s\\
					0 & \text{ sinon}.
				\end{cases}
			\] Alors, tout au long de l'exécution de l'algorithme, pour tout sommet $u$, $\mu[u] \ge d(s,u)$.
		\end{numlem}
	
		\begin{prv}
			\begin{description}
				\item[Initialement] La propriété est vraie par hypothèse.
				\item[Hérédité] Supposons vrai jusqu'à un certain état $\ubar\mu$, pour une opération $\textsc{Relâcher}(x,y)$.
					Pour tout sommet $v \neq y$, $\ubar\mu[v] = \bar\mu[v] \ge d(s,v)$.
					De plus, par disjonction de cas,
					\begin{itemize}
						\item si $\bar\mu[y] = \ubar\mu[y] \ge d(s,y)$ ;
						\item sinon si $\ubar\mu[y] = \ubar\mu[x] + c(x,y) \ge d(s,x) + c(x,y) \ge d(s,y)$ par hypothèse de récurrence, puis par lemme 1.
					\end{itemize}
			\end{description}
		\end{prv}
	
		\begin{crlr}
			Si \guillemotleft~à un moment~\guillemotright\ $\mu[u] = d(s,u)$, alors \guillemotleft~pour toujours après~\guillemotright\ $\mu[u] = d(s,u)$.
			\qed
		\end{crlr}
	
		\begin{numlem}
			Si $(s, \ldots, u, v)$ est un plus court chemin de $s$ à $v$ tel que $\ubar\mu[u] = d(s,u)$ \guillemotleft~à un certain moment de l'exécution de l'algorithme.~\guillemotright\@ Notons $\bar\mu$ obtenu par $\textsc{Relâcher}(u,v)$.
		\end{numlem}
	
		\begin{prv}
			On a \[
				\bar\mu = \begin{cases}
					\ubar\mu[v] & \text{ si } \ubar\mu[v] < \ubar\mu[u] + c(u,v)\\
					\ubar\mu[u] + c(u,v) &\text{ sinon}.
				\end{cases}
			\]Par disjonction de cas,
			\begin{itemize}
				\item si $\ubar\mu[v] < \ubar\mu[u] + c(u,v) = d(s, u) + c(u,v) = d(s,v)$, et donc, en utilisant le lemme 3, $\bar\mu[v] = \ubar\mu[v] = d(s,v)$.
				\item sinon, $\bar\mu[v] = \ubar\mu[u] + c(u,v) = d(u,v) + c(u, v) = d(s,v)$.
			\end{itemize}
		\end{prv}
	
		\begin{numlem}
			Soit $(s=x_0, x_1, x_2, \ldots, x_n)$ un plus court chemin. Si on effectue des opérations \hbox{$\textsc{Relâcher}(x_i, x_{i+1})$} dans l'ordre $0 \to n - 1$, possiblement entremêlés avec d'autres opérations $\textsc{Relâcher}$, alors pour tout $i \in \llbracket 0,n \rrbracket$, $\mu_{\text{final}}[x_i] = d(s, x_i)$.
		\end{numlem}
	
		\begin{prv}[par récurrence]
			\begin{itemize}
				\item Initialement, $\mu[x_0] = d(s, x_0) = d(s,s)$.
				\item Et, pour tout les $i$ inférieurs stricts, $\mu[x_i] = d(s, x_i)$, on conclut par le lemme 4.
			\end{itemize}
		\end{prv}
	
		(De ce lemme découle l'algorithme de \textsc{Bellman-Ford}.)
	
		\begin{crlr}
			L'algorithme \textsc{Dijkstra} est correct.
		\end{crlr}
	
		\begin{prv}
			Soit $t \in V$, un sommet du graphe. Soit $(s = x_0, x_1, \ldots, x_{p-1}, x_p = t)$ un plus court chemin de $s$ à $t$. Montrons que $\mu_{\text{final}}[t] = d(s, t)$.
			En utilisant le lemme 5, il suffit de montrer que \textsc{Dijkstra} relâche les arêtes dans cet ordre.
			Supposons les sommets extraits $\mathrm{todo}$ dans l'ordre $x_0, \ldots, x_i$, pour $i \in \llbracket 0,p-1 \rrbracket$.
			Par l'absurde, supposons que \textsc{Dijkstra} sorte $x_k$ de $\mathrm{todo}$ pour $k \in \llbracket i+2, p \rrbracket$.
			\guillemotleft~À ce moment là,~\guillemotright\ on a \[
				d(s, x_k) \le \mu[x_k] \le \mu[x_{i+1}] \le d(s, x_{i+1}),
			\]d'après le lemme 5, ce qui est absurde ($k > i + 1$).
		\end{prv}
	
		\begin{crlr}
			L'algorithme $A^*$ est correct.
		\end{crlr}
	
		\begin{algorithm}[H]
			\centering
			\begin{algorithmic}[1]
				\Procedure{Relâcher}{$u,v$}
				\If{$\mu[v] > \mu[u] + c(u,v)$}
				\State $\mu[v] \gets \mu[u] + c(u,v)$
				\State $\pi[v] \gets u$
				\State $\eta[v] \gets \mu[v] + h(v)$
				\EndIf
				\EndProcedure
			\end{algorithmic}
			\caption{Algorithme $A^*$ (partiel)}
		\end{algorithm}
	
		\begin{prv}
			Par l'absurde, supposons que non.
			Soit $t \in V$, un sommet du graphe, tel que $\mu_{\text{final}}[t] \neq d(s,t)$.
			Donc $d = \mu_{\text{final}}[t] > d(s,t) = d^*$.
			Soit $(s = x_0, x_1, \ldots, x_{p-1}, x_p = t)$ un plus court chemin de $s$ à $t$ de longueur $d^*$.
			L'algorithme commence par visiter $x_0 = s$ et on relâche les arêtes sortantes.
			Alors, $\mu[x_i] = d(s, x_1)$ et $\eta[x_1] = \mu[x_1] + \mu[x_1] + h(x_1) \le d(s, x_1) + d(x_1, t) = d(s,t) = d^* < d$ par hypothèse.
			\guillemotleft~À ce state,~\guillemotright\ $\eta[t] = \mu[t] + h(t) \ge d + 0$.
			Ainsi, $x_1$ devrait être choisi avant $t$. À un tel moment, $\mu[x_1] = d(s, x_1)$, on relâche alors ses arêtes sortantes ; en particulier $x_1$ et $x_2$. Ceci assure alors que $\mu[x_2] = d(s, x_2)$, et $\eta[x_2] = \mu[x_2] + h(x_2) \le d(sn x_2) + d(x_2, t) \le d(s,t) = d^* < d$.
			\guillemotleft~De proche en proche,~\guillemotright\ alors que l'on choisit $x_{p-1}$ dans $\mathrm{todo}$, on a $\mu[x_{p-1}] = d(s, x_{p-1})$.
			On relâche alors $\mu[x_p] = d(s, x_p) = d^*$.
			Or, $d = \mu_{\text{final}}[t] \le \mu_{\text{à ce moment}}[t]$. Absurde.
		\end{prv}
	
		\begin{exm}[ré-entrée dans $\mathrm{todo}$]
			\begin{comment}
				     b (h = 6)
						/ \
				 1 /   \ 1
					/  3  \     5
				 s - - - a - - - - t
			        (h = 0)   (h = 0)
			\end{comment}
			Exécution de l'algorithme $A^*$ sur l'entrée ci-dessus.
			La pile $\mathrm{todo}$ est vaut donc $\cancel s, \cancel a, \cancel b, \cancel t, \cancel a$.
		\end{exm}
	}
	\def\addmacros#1{#1}
}
{
	\chap[3]{Diviser pour régner}
	\minitoc
	\renewcommand{\cwd}{../cours/annexeC/}
	\addmacros{
		\section{Motivation}

\lettrine On place au centre de la classe 40 bonbons. On en distribue un chacun. Si, par exemple, chacun choisit un bonbon et, au \textit{top} départ, prennent celui choisi.
Il est probable que plusieurs choisissent le même. Comme gérer lorsque plusieurs essaient d'accéder à la mémoire ?

Deuxièmement, sur l'ordinateur, plusieurs applications tournent en même temps. Pour le moment, on considérait qu'un seul programme était exécuté, mais, le \textsc{pc} ne s'arrête pas pendant l'exécution du programme.

On s'intéresse à la notion de \guillemotleft~processus~\guillemotright\ qui représente une tâche à réaliser.
On ne peut pas assigner un processus à une unité de calcul, mais on peut \guillemotleft~allumer~\guillemotright\ et \guillemotleft~éteindre~\guillemotright\ un processus.
Le programme allumant et éteignant les processus est \guillemotleft~l'ordonnanceur.~\guillemotright\@ Il doit aussi s'occuper de la mémoire du processus (chaque processus à sa mémoire séparée).

On s'intéresse, dans ce chapitre, à des programmes qui \guillemotleft~partent du même~\guillemotright\ : un programme peut créer un \guillemotleft~fil d'exécution~\guillemotright\ (en anglais, \textit{thread}). Le programme peut gérer les fils d'exécution qu'il a créé, et éventuellement les arrêter.
Les fils d'exécutions partagent la mémoire du programme qui les a créé.

En C, une tâche est représenté par une fonction de type \lstinline[language=c]!void* tache(void* arg)!. Le type \lstinline[language=c]!void*!\ est l'équivalent du type \lstinline[language=caml]!'a! : on peut le \textit{cast} à un autre type (comme \lstinline[language=c]-char*-).

\begin{lstlisting}[language=c,caption=Création de \textit{thread}s en C]
void* tache(void* arg) {
	printf("%s\n", (char*) arg);
	return NULL;
}

int main() {
	pthread_t p1, p2;

	printf("main: begin\n");

	pthread_create(&p1, NULL, tache, "A");
	pthread_create(&p2, NULL, tache, "B");

	pthread_join(p1, NULL);
	pthread_join(p2, NULL);

	printf("main: end\n");

	return 0;
}
\end{lstlisting}

\begin{lstlisting}[language=c,caption=Mémoire dans les \textit{thread}s en C]
int max = 10;
volatile int counter = 0;

void* tache(void* arg) {
	char* letter = arg;
	int i;

	printf("%s begin [addr of i: %p] \n", letter, &i);

	for(i = 0; i < max; i++) {
		counter = counter + 1;
	}

	printf("%s : done\n", letter);
	return NULL;
}

int main() {
	pthread_t p1, p2;

	printf("main: begin\n");

	pthread_create(&p1, NULL, tache, "A");
	pthread_create(&p2, NULL, tache, "B");

	pthread_join(p1, NULL);
	pthread_join(p2, NULL);

	printf("main: end\n");

	return 0;
}
\end{lstlisting}

Dans les \textit{thread}s, les variables locales (comme \texttt{i}) sont séparées en mémoire. Mais, la variable \texttt{counter} est modifiée, mais elle ne correspond pas forcément à $2 \times \texttt{max}$. En effet, si \texttt{p1} et \texttt{p2} essaient d'exécuter au même moment de réaliser l'opération \lstinline[language=c]-counter = counter + 1-, ils peuvent récupérer deux valeurs identiques de \texttt{counter}, ajouter 1, puis réassigner \texttt{counter}.
Ils \guillemotleft~se marchent sur les pieds.~\guillemotright\ 

Parmi les opérations, on distingue certaines dénommées \guillemotleft~atomiques~\guillemotright\ qui ne peuvent pas être séparées. L'opération \lstinline[language=c]-i++- n'est pas atomique, mais la lecture et l'écriture mémoire le sont.

\begin{defn}
	On dit d'une variable qu'elle est \textit{atomique} lorsque l'ordonnanceur ne l'interrompt pas.
\end{defn}

\begin{exm}
	L'opération \lstinline[language=c]-counter = counter + 1- exécutée en série peut être représentée comme ci-dessous. Avec \texttt{counter} valant 40, cette exécution donne 42.
	\begin{table}[H]
		\centering
		\begin{tabular}{l|l}
			Exécution du fil A & Exécution du fil B\\ \hline
			(1)~$\mathrm{reg}_1 \gets \texttt{counter}$ & (4)~$\mathrm{reg}_2 \gets \texttt{counter}$ \\
			(2)~$\mathrm{reg}_1{++}$ & (5)~$\mathrm{reg}_2{++}$ \\
			(3)~$\texttt{counter} \gets \mathrm{reg}_1$ & (6)~$\texttt{counter} \gets \mathrm{reg}_2$
		\end{tabular}
	\end{table}
	\noindent Mais, avec l'exécution en simultanée, la valeur de \texttt{counter} sera 41.
	\begin{table}[H]
		\centering
		\begin{tabular}{l|l}
			Exécution du fil A & Exécution du fil B\\ \hline
			(1)~$\mathrm{reg}_1 \gets \texttt{counter}$ & (2)~$\mathrm{reg}_2 \gets \texttt{counter}$ \\
			(3)~$\mathrm{reg}_1{++}$ & (5)~$\mathrm{reg}_2{++}$ \\
			(4)~$\texttt{counter} \gets \mathrm{reg}_1$ & (6)~$\texttt{counter} \gets \mathrm{reg}_2$
		\end{tabular}
	\end{table}
	\noindent Il y a \textit{entrelacement} des deux fils d'exécution.
\end{exm}

\begin{rmk}[Problèmes de la programmation concurrentielle]
	\begin{itemize}
		\item Problème d'accès en mémoire,
		\item Problème du rendez-vous,\footnote{Lorsque deux programmes terminent, ils doivent s'attendre pour donner leurs valeurs.}
		\item Problème du producteur-consommateur,\footnote{Certains programmes doivent ralentir ou accélérer.}
		\item Problème de l'entreblocage,\footnote{\textit{c.f.} exemple ci-après.}
		\item Problème famine, du dîner des philosophes.\footnote{Les philosophes mangent autour d'une table, et mangent du riz avec des baguettes. Ils décident de n'acheter qu'une seule baguette par personne. Un philosophe peut, ou penser, ou manger. Mais, pour manger, ils ont besoin de deux baguettes. S'ils ne mangent pas, ils meurent.}
	\end{itemize}
\end{rmk}

\begin{exm}[Problème de l'entreblocage]~

	\begin{table}[H]
		\centering
		\begin{tabular}{l|l|l}
			Fil A & Fil B & Fil C\\ \hline
			RDV(C) & RDV(A) & RDV(B)\\
			RDV(B) & RDV(C) & RDV(A)\\
		\end{tabular}
		\caption{Problème de l'entreblocage}
	\end{table}
\end{exm}

Comment résoudre le problème des deux incrementations ? Il suffit de \guillemotleft~mettre un verrou.~\guillemotright\ Le premier fil d'exécution \guillemotleft~s'enferme~\guillemotright\ avec l'expression \lstinline[language=c]!count++!, le second fil d'exécution attend que l'autre sorte pour pouvoir entrer et s'enfermer à son tour.


	}
	\def\addmacros#1{#1}
}
{
	\chap[4]{Lemme d'\textsc{Arden} et retour sur le théorème de \textsc{Kleene}}
	\minitoc
	\renewcommand{\cwd}{../cours/annexeD/}
	\addmacros{
		\begin{exm}[Lemme d'\textsc{Arden}]
			Soient $K$ et $L$ deux langages. Résoudre $X = K\cdot X \cup L$ pour $X$ un langage.
			(On trouve~$X = K^* \cdot L$.) On suppose que $\varepsilon \not\in K$.
			On procède par double-inclusion.
			\begin{itemize}
				\item[``$\supseteq$''] Soit $X$ un langage tel que $X = K\cdot X \cup L$.
					Montrons par récurrence \guillemotleft~si $w$ est un mot de $X$ de taille $n$, alors $w \in K^* \cdot L$.
					\begin{itemize}
						\item Si $n= 0$, alors $w \in L$ car $\varepsilon \not\in  K$.
							Ainsi, $w = \varepsilon \cdot w$ et $\varepsilon \in K^*$. On en déduit que $w \in K^* \cdot L$.
						\item Si $|w| = n$, alors
							\begin{itemize}
								\item si $w \in L$, alors $w = \varepsilon \cdot w$ et donc $w \in K^* L$.
								\item si $w = v \cdot w'$ où $v \in K$ et $w' \in X$, alors $|w'| < |w|$. Ainsi, par hypothèse de récurrence, $w' \in K^* \cdot L$. Ainsi, $v \cdot w' \in K^* \cdot L$.
							\end{itemize}
					\end{itemize}
					Ainsi, $X \subseteq K^* \cdot L$.
				\item[``$\subseteq$'']
					Soit $w \in K^* \cdot L$. Il existe donc $n \in \N$, $(v_1, \ldots, v_n) \in K^n$ et $w' \in L$ tels que $w = v_1 \cdot \ldots \cdot v_n \cdot  w'$.
					Alors, $w' \in X$ donc $v_n \cdot w' \in X$ donc \ldots donc $v_1 v_2 \ldots v_n w' \in X$.
					Ainsi, $w \in X$.
			\end{itemize}
		\end{exm}
	
		\begin{exm}
			On considère l'automate ci-dessous.
			\begin{figure}[H]
				\centering
				\tikzfig{auto-ex}
				\caption{Automate exemple ($\mathcal{A}$)}
			\end{figure}
			On pose $X_i = \mathcal{L}\big((\Sigma, \mathcal{Q}, \{i\}, F, \delta)\big)$, où $x_i$ est l'unique point de départ.
			Ainsi, $\mathcal{L}(\mathcal{A}) = \bigcup_{i \in  I} X_i$.
			Déterminons les valeurs de $X_1$, $X_2$ et $X_3$.
			On applique un algorithme similaire au \guillemotleft~pivot de Gau\ss.~\guillemotright\ 
	
			\begin{align*}
				\left.\begin{array}{rl}
					X_1 &= \{a\} \cdot X_2 \cup \{a\} X_1\\
					X_2 &= \{b\} \cdot X_1 \cup \{a\} \cdot X_3 \cup \{\varepsilon\}\\
					X_3 &= \{b\} X_3 \cup \{\varepsilon\}
				\end{array}\right\}
				\iff& 
				\begin{cases}
					X_1&= \{a\}^* \cdot \{a\} \cdot X_2\\
					X_2&= \mathcal{L}(ba^* \cdot a) X_2 \cup \{a\} X_3 \cup \{\varepsilon\}\\
					X_3&= \{b\} \cdot X_3 \cup \{\varepsilon\}
				\end{cases}\\
				\iff& 
				\begin{cases}
					X_1&= \mathcal{L}(a^* \cdot a) X_2\\
					X_2&= \mathcal{L}\big((b\cdot a^* \cdot a)^*\big) \cdot \big(\{a\} X_3 \cup \{\varepsilon\}\big)\\
					X_3&= \{b\} X_3 \cup \{\varepsilon\}
				\end{cases}\\
				\iff& \begin{cases}
					X_1&= \mathcal{L}(a^* \cdot a) X_2\\
					X_2&= \mathcal{L}\big((b\cdot a^* \cdot a)^*\big) \cdot \big(\{a\} X_3 \cup \{\varepsilon\}\big)\\
					X_3&= \mathcal{L}(b^*)
				\end{cases} \\
				\iff& \begin{cases}
					X_1&= \mathcal{L}(a^* \cdot a) X_2\\
					X_2&= \mathcal{L}\big((ba^*a)^* \cdot (ab^*  \mid \varepsilon)\big)\\
					X_3&= \mathcal{L}(b^*)
				\end{cases} \\
				\iff& \begin{cases}
					X_1&= \mathcal{L}\big(a^* \cdot a \cdot (ba^*a)^* \cdot (ab^*  \mid \varepsilon)\big)\\
					X_2&= \mathcal{L}\big((ba^*a)^* \cdot (ab^*  \mid \varepsilon)\big)\\
					X_3&= \mathcal{L}(b^*)
				\end{cases}
			\end{align*}
		\end{exm}
	
		On peut généraliser la méthode employée dans l'exemple précédent pour montrer que tout langage reconnaissable est régulier.
	}
	\def\addmacros#1{#1}
}
{
	\chap[5]{Tas et files de priorités}
	\minitoc
	\renewcommand{\cwd}{../cours/annexeE/}
	\addmacros{
		L'objectif d'une file de priorité est de récupérer l'élément de priorité minimale.
		On organise cette structure de données sous forme d'un arbre tournois.\footnote{Un arbre tournois n'est pas un arbre binaire de recherche.}
		Un arbre tournois est un arbre dont la priorité d'un nœud est supérieur à celle de ses fils.
		On impose une structure supplémentaire, l'arbre doit être parfait : l'arbre est complet jusqu'à l'avant dernier niveau, où il est replis à gauche.
		 On définit plusieurs opérations sur cette file de priorité (de type \texttt{fp}, où les éléments sont de type \texttt{elem}) :
		\begin{itemize}
			\item $\texttt{insérer} : \texttt{fp} \to \texttt{elem} \to \texttt{fp}$ qui insère un élément,
			\item $\texttt{lire\_min} : \texttt{fp} \to \texttt{elem}$ qui récupère l'élément de priorité minimale,
			\item $\texttt{supprimer\_min} : \texttt{fp} \to \texttt{fp}$ qui supprime l'élément de priorité minimale,
			\item ($\texttt{diminuer\_priorité} : \texttt{fp} \to \texttt{elem} \to \texttt{fp}$),\footnote{Cette opération est parfois omise car trop compliquée à implémenter.}
			\item $\texttt{créer} : (\:) \to \texttt{fp}$.
		\end{itemize}
		On définit un type \texttt{btree}, représentant un arbre binaire, et on implémente les opérations ci-dessous en \textsc{OCaml}.
		\begin{lstlisting}[language=caml,caption=Définition du type \texttt{btree}]
	type 'a btree =
	| Node of 'a * 'a btree * 'a btree
	| Empty
		\end{lstlisting}
		Pour l'opération \texttt{créer}, on retourne \texttt{Empty} (cela donne une complexité en $\Theta(1)$). 
		Pour l'opération \texttt{insérer}, on insère l'élément comme feuille (de manière à conserver la propriété de l'arbre parfait), et on inverse le nœud avec son parent jusqu'à ce que la propriété soit vérifiée ($\Theta(\log_2 n)$).
		Pour l'opération \texttt{lire\_min}, on lit la racine ($\Theta(1)$).
		Pour l'opération \texttt{supprimer\_min}, on permute la racine et le dernier nœud (\textit{i.e.} le nœud le plus à droite de hauteur maximale), et on restore la structure d'arbre tournois en permutant un nœud et son fils de valeur minimale, et en répétant ($\Theta(\log_2 n)$).
		Pour trouver le dernier nœud, on garde en mémoire cet emplacement.
		On peut aussi implémenter cet algorithme avec un tableau ($\triangleright$ \textsc{tp}), ou avec une liste triée (mais la complexité est moins bien).
	}
	\def\addmacros#1{#1}
}
{
	\chap[6]{Arithmétique}
	\minitoc
	\renewcommand{\cwd}{../cours/annexeF/}
	\addmacros{
		Un des premiers algorithmes codé est l'algorithme d'Euclide pour calculer le \textsc{pgcd}. Pour $a \neq 0$, on a $a \wedge 0 = a$ et $a \wedge b = b \wedge (a\ \mathrm{mod}\ b)$.
		On peut le coder en \textsc{OCaml} avec la fonction \texttt{euclid} suivante.
	
		\begin{lstlisting}[language=caml,caption=Algorithme d'Euclide calculant le \textsc{pgcd}]
	let rec euclid (a: int) (b: int): int =
		(* Hyp: a >= b et a != 0 *)
		if b = 0 then a
		else euclide b (a mod b)
		\end{lstlisting}
		
		Quelle est la complexité de cet algorithme ?
		On représente le nombre d'appels récursifs à \texttt{euclid}, et on devine une courbe logarithmique.
		En notant $(u_n)$ les divisions euclidiennes réalisées et $(q_n)$ les quotients, ainsi, on $u_n = q_{n-1} \cdot u_{n-1} + u_{n-2}$.
		Alors, $\texttt{euclid}(u_n, u_{n-1}) = \cdots = \texttt{euclid}(u_3, u_2) = \texttt{euclid}(u_2, u_1) = \texttt{euclid}(u_1, u_0)$.
	
		En fixant la complexité, on cherche les valeurs de $(u_n)$ maximisant les appels récursifs.
		On peut montrer par récurrence que si $\texttt{euclid}(a,b)$ conduit à $n$ appels récursifs de \texttt{euclid}, alors $a \ge F_n$ et $b \ge F_{n-1}$, où $(F_n)_{n\in\N}$\/ est la suite de Fibonacci.
	
		En effet, soit un tel couple $(a,b)$. Alors, $(b, a\ \mathrm{mod}\ b)$ conduit à $n - 1$ appels récursifs donc~$b \ge F_{n-1}$ et~$a\ \mathrm{mod}\ b \ge F_{n-2}$ par hypothèse de recurrence.
		Et, $a = bq + (a\ \mathrm{mod}\ b)$ et donc $a \ge F_{n-1} + F_{n-2} = F_n$.
	
		De plus, pour tout $n \in \N \setminus \{0,1\}$, $F_n \ge \varphi^{n-2}$ où $\varphi$ est le nombre d'or.\footnote{C'est la solution positive de $X^2 - X - 1 = 0$.}
		En effet, $F_2 = 1 \ge \varphi^0 = 1$ et $F_3 = 2 \ge \varphi^1 = \varphi = (1 + \sqrt{5}) / 2$. Et, $F_n = F_{n-1} + F_{n-2} \ge \varphi^{n-3} + \varphi^{n-4} \ge \varphi^{n-4}(1 + \varphi) \ge \varphi^{n-2}$.
	
		Soient $(p,q)$, où $p \ge q$, une entrée de l'algorithme d'Euclide. Si l'appel $\texttt{euclid}(p,q)$ conduit à plus de $\left\lceil \log_\varphi p \right\rceil + 4$ appels, alors $p \ge F_{\left\lceil \log_\varphi p \right\rceil + 4} \ge \varphi^{\left\lceil \log_\varphi p \right\rceil + 4 - 2} > \varphi^{\log_\varphi p} = p$, ce qui est absurde.
	
		Ceci conduit à une complexité en $\mathcal{O}(\log p)$.
	
		\bigskip
	
		Soit $n$ un entier premier.
		Pour l'algorithme RSA, on cherche un inverse de $a \in \sfrac{\Z}{n\Z}$ : on cherche $b \in \sfrac\Z{n\Z}$ tel que $ab \equiv 1 \mod 1$. D'après le théorème de Bézout, on a $au + nv = 1$ car $a \wedge n = 1$. L'inverse est $v$. D'où l'importance des coefficients de Bézout.
	
		Comment calculer les coefficients de Bézout ?
		On peut utiliser l'algorithme d'Euclide.
		On pose $r_n$\/ la valeur de \texttt{a} après $n$ appels récursifs.
	
		\begin{table}[H]
			\centering
			\begin{tabular}{c|c|c|c|c}
				$r_i$ & $u_i$ & & $v_i$ &\\ \hline \hline
				$r_0 = a$ & $1$ & $a$ & $0$ & $b$\\
				$r_1 = b$ & $0$ & $b$ & $1$ & $a$\\
				$\vdots$ & $\vdots$ & $\vdots$ & $\vdots$ & $\vdots$ \\
				$r_{i-2}$ & $u_{i-2}$ & $a$ & $v_{i-2}$ & $b$\\
				$r_{i-1}$ & $u_{i-1}$ & $a$ & $v_{i-1}$ & $b$ \\
			\end{tabular}
			\caption{Valeurs de $r_i$ avec invariant $r_i = a u_i + b v_i$}
		\end{table}
	
		Alors,
		\begin{align*}
			r_i &= u_{i-2} a + v_{i-2} b - (r_{i-2} / r_{i-1}) (u_{i-1}a + v_{i-1} b)\\
			&= \big(u_{i-2} - (r_{i-2}/r_{i-1}) u_{i-1}\big) a + \big(v_{i-2} - (r_{i-2}/r_{i-1}) v_{i-1}\big) b \\
		\end{align*}
		Ainsi, on a bien $\mathrm{pgcd}(a,b) = u_{n-1} a + v_{n-1} b$.
	}
	\def\addmacros#1{#1}
}
{
	\chap[7]{Arbres rouges-noirs}
	\minitoc
	\renewcommand{\cwd}{../cours/annexeG/}
	\addmacros{
		Un arbre rouge-noir est un cas particulier des arbres binaires de recherches.
		On l'utilise notamment pour représenter des ensembles, on veut donc réaliser deux opérations simples : l'insertion et le test d'appartenance.
		Initialement, on pense représenter un ensemble par une liste triée.
		Mais, on utilise plutôt un arbre binaire pour représenter des données avec une hauteur logarithmique, contrairement à une hauteur linéaire.
		Les arbres binaires de recherches sont des arbres dans lesquels ont peut réaliser une dichotomie.
		\begin{lstlisting}[language=caml,caption=Arbre binaire de recherche]
	type 'a btree = E | N of 'a * 'a btree * 'a btree
	
	let rec mem (x: 'a) (t: 'a btree): bool =
		match t with
		| E -> false
		| N(y, g, d) ->
				if x < y      then mem x g
				else if x = y then true
				else               mem x d
	
	let rec insere (x: 'a) (t: 'a btree): 'a btree =
		match t with
		| E -> false
		| N(y, g, d) ->
				if x < y      then N(y, insere x g, d)
				else if x = y then t
				else               N(y, g, insere x d)
		\end{lstlisting}
		La fonction \texttt{mem} permet de réaliser ce test d'appartenance et la fonction \texttt{insere} insère l'insertion dans l'arbre.
		Ainsi, on peut représenter un ensemble avec le type \texttt{'a btree}.
	
		Mais, cet arbre peut être déséquilibré, et l'utilisation de la dichotomie ne donne pas de résultats très avantageux.
		On utilise donc un arbre \textit{auto-équilibrant}, comme les \textsc{avl} du 1er \textsc{dm}. Pour les \textsc{avl}, la différence de hauteur est $-1$, $0$ ou $1$.
	
		On introduit donc le concept d'arbre rouge-noir.
		Un arbre rouge-noir est un arbre parfait, qui a une certaine \guillemotleft~élasticité.~\guillemotright\@
		On colorie chaque nœuds pour imposer des contraintes sur cette élasticité.
		Les branches de l'arbre a une longueur de rupture.
		Un arbre contenant uniquement des nœuds noirs est un arbre parfait.
		Et, entre deux nœuds noirs, on peut insérer un nœud rouge.
		Un arbre rouge-noir vérifie donc les trois propriétés suivantes :
		\begin{enumerate}[label=(\textit{\alph*})]
			\item la racine est noire,
			\item le père d'un nœud rouge est noir,
			\item la hauteur noir de chaque feuille externe est constante. \hfill [important]
		\end{enumerate}
		Une \textit{feuille externe} est, dans le code \textsc{OCaml}, l'expression \texttt{E} ; et, la \textit{hauteur noir} d'une feuille externe est le nombre de nœuds noirs depuis la racine.
		On définit la \textit{hauteur noir} d'un arbre comme la hauteur noir de chaque feuille externe (qui est constante).
	
		Le problème est l'insertion d'un nœud.
		Insérer un nœud noir est, en général, plus dangereux car il modifie la hauteur noir de tout l'arbre.
		On préfère donc insérer un nœud rouge, sauf dans le cas de la racine.
	
		On considère donc la propriété (\textit{c}) comme invariante. En effet, corriger un arbre pour valider la propriété (\textit{a}) ou la propriété (\textit{b}) est bien plus simple.
	
		On traite tous les cas dans le diaporama sur \textit{cahier-de-prépa}, et on réalise l'exemple sur les lettres A, L, G, O, R, I, T, H, M, E.
	
		\bigskip
		
		Pour supprimer un nœud dans un arbre binaire classique, on peut le remplacer par le maximal de son sous-arbre droit, ou le minimum de son sous-arbre gauche. Si on supprime un nœud ayant un seul fils, on n'a qu'à re-brancher le sous-arbre.
		Pour les arbres rouges-noirs, c'est supprimer un nœud noir qui pose problème.
		Pour cela, on introduit les nœuds doublement noirs, qui comptent pour deux dans la hauteur noir.
		Ainsi, on supprime le nœud noir et on remplace les autres nœuds par des nœuds doublement noirs.
		L'algorithme n'a donc qu'à faire remonter le nœud doublement noir, jusqu'à la racine, où il sera transformé en nœud simplement noir.
	
		Un arbre de hauteur $h$ a une hauteur noir $\mathrm{bh} \ge h / 2$. Et, la taille, \textit{i.e.} le nombre de nœuds, est supérieure à $2^{\mathrm{bh}} - 1$.
		On conclut que $h \le 2 \log_2(\mathrm{taille} + 1)$.
		De même par la propriété (\textit{c}) permet de conclure que $h = \Theta(\log_2 \mathrm{taille})$.
	}
	\def\addmacros#1{#1}
}
{
	\chap[8]{Complexité moyenne}
	\minitoc
	\renewcommand{\cwd}{../cours/annexeH/}
	\addmacros{
		Dans les annexes et cours précédents, on a vu la complexité \guillemotleft~pire cas~\guillemotright\ et la complexité amortie.
		On considère les nombres d'opérations possibles pour toute entrée de taille $n$.
		La complexité \guillemotleft~pire cas~\guillemotright\ est la complexité obtenue en prenant le $\max$ d'opération possibilité.
		Pour la complexité moyenne, on suppose que chaque ensemble d'entrée de taille $n$ est munit d'une probabilité $P_n$.
		Par exemple, on considère qu'une entrée est une permutation de $n$ éléments, \textit{i.e.}\ un élément de $\mathfrak{S}_n$.
		On suppose que chaque entrée arrive avec équiprobabilité. Ainsi, $\forall \sigma \in \mathfrak{S}_n$, $P_n(\sigma) = 1/n!$.
		Ainsi, on a \[
			C_{\max}(n) = \max_{\sigma \in \mathfrak{S}_n} C(s) \text{ et } C_{\mathrm{moy}} = \sum_{\sigma \in \mathfrak{S}_n} P(\sigma) \cdot C(\sigma)
		.\]
		La complexité moyenne est la moyenne des complexité pondérées par les probabilités.
	
		\begin{exm}
			On considère l'algorithme ci-dessous.
			\begin{algorithm}[H]
				\centering
				\begin{algorithmic}[1]
					\Entree $\sigma \in \mathfrak{S}_n$ et $i \in \llbracket 1,n \rrbracket$
					\Sortie $j \in \llbracket 1,n \rrbracket$ tel que $\sigma(j) = i$.
					\For{$j \in \llbracket 1,n \rrbracket$}
					\If{$\sigma(j) = i$} \Return $j$
					\EndIf
					\EndFor
				\end{algorithmic}
				\caption{Calcul d'inverse d'une permutation}
			\end{algorithm}
			\noindent
			On munit $\mathfrak{S}_n$ de la probabilité uniforme.
			Soit $i \in \llbracket 1,n \rrbracket$.
			Notons, pour tout $j \in \llbracket 1,n \rrbracket$, $\mathfrak{S}_n^j = \{\sigma \in \mathfrak{S}_n  \mid \sigma(j) = i\}$.
			Remarquons que $\mathfrak{S}_n = \bigcupdot_{j=1}^n \mathfrak{S}_n^j$.
			Ainsi,
			\begin{align*}
				C_{\mathrm{moy}} &= \sum_{j=1}^n \sum_{\sigma \in \mathfrak{S}_n^j}P_n(\sigma) C(\sigma)\\
				&= \sum_{j=1}^n j \times \frac{|\mathfrak{S}_n^j|}{n!}\\
				&= \sum_{j=1}^n j \times \frac{(n+1)!}{n!}\\
				&= \frac{1}{n} \cdot \frac{n(n+1)}{2}\\
				&= \frac{n+1}{2}
			\end{align*}
		\end{exm}
	}
	\def\addmacros#1{#1}
}
{
	\chap[9]{Preuves de correction pour les fonctions récursives}
	\minitoc
	\renewcommand{\cwd}{../cours/annexeI/}
	\addmacros{
		On considère l'insertion dans un arbre binaire de recherche.
		Démontrons qu'elle est correcte.
		On adopte les notations de l'annexe G sur les arbres rouges-noirs.
		Montrons que, pour tout ABR $t$, et pour tout étiquette $e \in \mathds{E}$, 
		\[
			\mathrm{\acute{e}tiquettes}(\texttt{insertion}(t, x)) = \mathrm{\acute{e}tiquettes}(t) \cup \{x\}
		.\]
		Montrons le par induction.
		\begin{enumerate}
			\item Si $t = \texttt{E}$. Soit $x \in \mathds{E}$ : \[
					\mathrm{\acute{e}tiquettes}(\texttt{insertion}(\texttt{E},x)) = \mathrm{\acute{e}tiquettes}(N(x, \texttt{E},\texttt{E})) = \{x\} = \mathord{\underbrace{\mathrm{\acute{e}tiquettes}(\texttt{E})}_{\O}} \cup \{x\}
				.\]
			\item Si $t = \texttt{N}(y, g, d)$. Soit $x \in \mathds{E}$.
				\begin{itemize}
					\item si $x < y$, alors
						\begin{align*}
							\mathrm{\acute{e}tiquettes}(\texttt{insertion}(t,x))
							&= \mathrm{\acute{e}tiquettes}(\texttt{N}(y, \texttt{insertion}(g,x), d)) \\
							&= \{x\} \cup \mathrm{\acute{e}tiquettes}(\texttt{insertion}(g,x)) \cup \mathrm{\acute{e}tiquettes}(d) \\
							&= \{y\}  \cup \mathrm{\acute{e}tiquettes}(g) \cup \{x\} \cup \mathrm{\acute{e}tiquettes}(d) \\
							&= \mathrm{\acute{e}tiquettes}(\texttt{N}(y,g,d)) \cup \{x\} \\
							&= \mathrm{\acute{e}tiquettes}(t) \cup \{x\} \\
						\end{align*}
					\item on procède de même pour les autres cas.
				\end{itemize}
		\end{enumerate}
		On procède de même pour les autres propriétés.
	}
	\def\addmacros#1{#1}
}
