\section{Gestion du graphe et du graphe transposé}
\begin{enumerate}
	\item~
		\begin{lstlisting}[language=caml,caption=Type \texttt{graphe}]
type graphe = int list array
		\end{lstlisting}
	\item~
		\begin{lstlisting}[language=caml,caption=Transposée d'un graphe]
let transpose (g: graphe): graphe =
let n = Array.length g in
let g' = Array.make n [] in
for i = 0 to n - 1 do
	List.iter (fun j -> g'.(j) <- i :: g'.(j)) g.(i)
done;
g'
		\end{lstlisting}
\end{enumerate}
\section{Gestion des points de régénération d’un parcours}
\begin{enumerate}[start=3]
	\item Avec comme ordre de priorité $(a, b, c, d, e, f, g)$, on a le parcours
		\[
			a \to b \to c \to d \to e \to f \to g
		.\]
		Avec l'ordre de priorité $(f, e, g, c, d, b, a)$, on a le parcours \[
			f \to e \leadsto g \to c \to d \to b \leadsto a
		.\]
	\item~
		\begin{lstlisting}[language=caml,caption={Générateur du tableau $\llbracket 0,\texttt{n} \rrbracket$}]
exception Plus_d_entiers

let cree_entiers (n: int): (int ref) * (unit -> unit) =
let elem = ref 0 in
let next () =
	if !elem < n then elem := !elem + 1
	else raise Plus_d_entiers
in (elem, next)
		\end{lstlisting}
	\item~
		\begin{lstlisting}[language=caml,caption={Générateur d'un tableau \texttt{tab}}]
let cree_enumerateur_tab (tab: int array): (int ref) * (unit -> unit) =
let i = ref 0 in
let elem = ref tab.(0) in
let next () =
	i := !i + 1;
	elem := tab.(!i)
in
(elem, next)
		\end{lstlisting}
\end{enumerate}
