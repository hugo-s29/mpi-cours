\section{Raisonner par induction sur une grammaire}

\begin{enumerate}
	\item Montrons le par induction.
		\begin{itemize}
			\item \textbf{Cas $\mathrm{S} \to a \mathrm{S}$.}
				Soit $w = a w'$ un mot, où $ba$ n'est pas un sous-mot de $w'$.
				Alors, $ba$ n'est pas un sous-mot de $w=aw'$.
			\item \textbf{Cas $\mathrm{S} \to \mathrm{S}b$.}
				Soit $w = w' b$ un mot, où $ba$ n'est pas un sous-mot de $w'$.
				Alors, $ba$ n'est pas un sous-mot de $w=w'b$.
			\item \textbf{Cas $S \to \varepsilon  \mid a  \mid b$.}
				Le mot $ba$ n'est pas un sous-mot de $a$, ni de $b$, ni de $\varepsilon$.
		\end{itemize}
	\item Montrons, par double-inclusion, que $\mathcal{L}(\mathcal{G}) = \mathcal{L}(a^* b^*)$.
		\begin{itemize}
			\item[``$\supseteq$'']
				Soit $w \in \mathcal{L}(a^*b^*)$. Il existe $m$ et $n$ deux entiers tels que $w = a^n b^m$.
				On applique la dérivation \[
					\underbrace{\mathrm{S} \Rightarrow a \mathrm{S} \Rightarrow aa \mathrm{S} \Rightarrow \cdots \Rightarrow a^n \mathrm{S}}_{n \text{ fois}} \mathrel{\Rightarrow} \underbrace{a^n \mathrm{S} b \Rightarrow a^n \mathrm{S} bb \Rightarrow \cdots \Rightarrow a^n \mathrm{S} b^m}_{m \text{ fois}} \Rightarrow a^n b^m
				.\]
			\item[``$\subseteq$''] D'après la question 1, on a ce sens de l'inclusion.
		\end{itemize}
\end{enumerate}

