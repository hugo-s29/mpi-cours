\section{Exemples de dérivations}

\begin{enumerate}
	\item Les dérivations (b), et (d) sont vraies.
	\item Les dérivations (a), (c), et (d).
	\item On a $\{ab,ba,aab\} \subseteq \mathcal{L}(\mathcal{G})$. En effet,
		\begin{itemize}
			\item $\mathrm{S} \Rightarrow \mathrm{D} \Rightarrow a \mathrm{T} b \Rightarrow ab$,
			\item $\mathrm{S} \Rightarrow \mathrm{D} \Rightarrow b \mathrm{T} a \Rightarrow ba$,
			\item $\mathrm{S} \Rightarrow \mathrm{D} \Rightarrow a \mathrm{T} b \Rightarrow a \mathrm{X} b \Rightarrow aab$.
		\end{itemize}
	\item On a $\varepsilon \not\in \mathcal{L}(\mathcal{G})$, $aa \not\in \mathcal{L}(\mathcal{G})$ et $bb \not\in \mathcal{L}(\mathcal{G})$.
	\item Le langage $\mathcal{L}(\mathcal{G})$ est l'ensemble des mots non palindromes.
\end{enumerate}
