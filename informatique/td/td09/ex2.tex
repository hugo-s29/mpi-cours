\section{Tri topologique}

\begin{figure}[H]
	\centering
	\tikzfig{ex2-q1}
	\caption{Exemple de graphe}
\end{figure}

\begin{enumerate}
	\item Dans le graphe ci-dessus, $a \to c \to b$\/ est un tri topologique mais pas un parcours.
	\item Dans le même graphe, $b \to a \to c$\/ est un parcours mais pas un tri topologique.
	\item Supposons que $L_1$\/ possède un prédécesseur, on le note $L_i$\/ où $i > 1$. Ainsi, $(L_i, L_1) \in A$\/ et donc $i < 1$, ce qui est absurde. De même pour le dernier.
	\item Il existe un tri topologique si, et seulement si le graphe est acyclique.
		\begin{itemize}
			\item[``$\implies$'']
				Soit $L_1,\ldots,L_n$\/ un tri topologique. Montrons que le graphe est acyclique.
				Par l'absurde, on suppose le graphe non acyclique : il existe $(i,j) \in \llbracket 1,n \rrbracket^2$\/ avec $i \neq j$\/ tels que $T_i \to \cdots \to T_j$\/ et $T_j \to \cdots \to T_i$ soient deux chemins valides. Ainsi, comme le tri est topologique et par récurrence, $i \le j$\/ et $j \le i$\/ et donc $i = j$, ce qui est absurde car $i$\/ et $j$\/ sont supposés différents. Le graphe est donc acyclique.
			\item[``$\impliedby$'']
				Soit $G$\/ un graphe tel que tous les sommets possèdent une arrête entrante. On suppose par l'absurde ce graphe acyclique.
				Soit $x_0$\/ un sommet du graphe.
				On construit par récurrence $x_0,x_1,\ldots,x_n,x_{n+1},\ldots$ les successeurs successifs. Il y a un nombre fini de sommets donc deux sommets sont identiques. Donc, il y a nécessairement un cycle, ce qui est absurde.
				\begin{algorithm}[H]
					\centering
					\begin{algorithmic}[1]
						\Entree $G = (S, A)$\/ un graphe acyclique
						\Sortie $\mathrm{Res}$\/ un tri topologique.
						\State $\mathrm{Res} \gets [\quad]$\/
						\While{$G \neq \O$}
							\State Soit $x$\/ un sommet de $G$\/ sans prédécesseur
							\State $G \gets \big(S \setminus \{x\}, A \cap (S \setminus \{x\})^2\big)$\/ 
							\State $\mathrm{Res} \gets \mathrm{Res} \cdot [x]$\/
						\EndWhile
						\State\Return $\mathrm{Res}$\/
					\end{algorithmic}
					\caption{Génération d'un tri topologique d'un graphe acyclique}
				\end{algorithm}
		\end{itemize}
	\item~
		\begin{algorithm}[H]
			\centering
			\begin{algorithmic}[1]
				\Entree $G = (S, A)$\/ un graphe
				\Sortie $\mathrm{Res}$\/ un tri topologique, ou un cycle
				\State $\mathrm{Res} \gets [\quad]$\/
				\While{$G \neq \O$}
					\If{il existe $x$ sans prédécesseurs}
						\State Soit $x$\/ un sommet de $G$\/ sans prédécesseur
						\State $G \gets \big(S \setminus \{x\}, A \cap (S \setminus \{x\})^2\big)$\/ 
						\State $\mathrm{Res} \gets \mathrm{Res} \cdot [x]$\/
					\Else
						\State Soit $x \in S$\/ 
						\State Soit $x \gets x_1 \gets x_2 \gets \cdots \gets x_i$\/ la suite des prédécesseurs
						\State\Return $x_i,x_{i+1},\ldots,x_i$, un cycle
					\EndIf
				\EndWhile
				\State\Return $\mathrm{Res}$\/
			\end{algorithmic}
			\caption{Génération d'un tri topologique d'un graphe}
		\end{algorithm}
	\item On utilise la représentation par liste d'adjacence, et on stocke le nombre de prédécesseurs que l'on décroit à chaque choix de sommet.
	\item On essaie de trouver un tri topologique, et on voit si l'on trouve un cycle.
\end{enumerate}
