\section{Logique avec \texttt{If}}

\setbox\tmpbox=\hbox{\Large $\mathbf{\mathcal{F}}$}
\subsection{Représentatibilité des fonctions booléennes par formules de \box\tmpbox\!\textsubscript{\texttt{if}}}

\begin{enumerate}
	\item On pose $G = {\tt if}\/\ p\ {\tt then}\/\ \top\ {\tt else}\/\ (\underbrace{{\tt if}\/\ q\ {\tt then}\/\ \top\ {\tt else}\/\ r}_{A})$, et on a
		\begin{align*}
			\left\llbracket G \right\rrbracket^\rho
			&= \left\llbracket p \right\rrbracket^\rho \cdot \left\llbracket \top \right\rrbracket^\rho + \overline{\left\llbracket p \right\rrbracket^\rho}  \cdot \left\llbracket A \right\rrbracket^\rho \\
			&= \rho(p) + \overline{\rho(p)} \cdot  \left( \left\llbracket q \right\rrbracket^\rho \cdot \left\llbracket \top \right\rrbracket^\rho + \overline{\left\llbracket q \right\rrbracket^\rho} \cdot \left\llbracket r \right\rrbracket^\rho \right) \\
			&= \rho(p) + \overline{\rho(p)} \cdot  \left( \rho(q) + \overline{\rho(q)} \cdot \rho(r) \right) \\
			&= \rho(p) + \overline{\rho(p)}\cdot \rho(q) + \overline{\rho(p)} \cdot  \overline{\rho(q)} \cdot \rho(r) \\
			&= \rho(p) + \rho(q) + \rho(r). \\
		\end{align*}
	\item \[
			\left\llbracket {\tt if}\/\ C\ {\tt then}\/\ G\ {\tt else}\/\ H \right\rrbracket^\rho = \begin{cases}
				\left\llbracket G \right\rrbracket^\rho &\text{ si }\left\llbracket C \right\rrbracket^\rho = {\bfm V}\\
				\left\llbracket H \right\rrbracket^\rho &\text{ if } \left\llbracket C \right\rrbracket^\rho = {\bfm F}
			\end{cases}
		.\]
	\item Soit $G \in \mathds{F}$.
		\begin{itemize}
			\item[Cas 1] Soit $\mathcal{P} = \{p\}$\/
				\begin{itemize}
					\item Sous-cas 1 : $f: \rho \mapsto {\bfm V}$\/ est associée à $\top$.
					\item Sous-cas 2 :  la fonction dont la table de vérité est ci-dessous est associée à $p$.
						\begin{center}
							\begin{tabular}{c|c}
								$p$&$f$\\ \hline
								${\bfm F}$ & ${\bfm F}$\/ \\
								${\bfm V}$ & ${\bfm V}$\/ \\
							\end{tabular}
						\end{center}
					\item Sous-cas 3 :  la fonction dont la table de vérité est ci-dessous est associée à $\bar{p}$.
						\begin{center}
							\begin{tabular}{c|c}
								$p$&$f$\\ \hline
								${\bfm F}$ & ${\bfm V}$\/ \\
								${\bfm V}$ & ${\bfm F}$\/ \\
							\end{tabular}
						\end{center}
					\item Sous-cas 4 : $f: \rho \mapsto {\bfm F}$\/ est associée à $\bot$.
				\end{itemize}
			\item[Cas 2] Soit $\mathcal{P} = \{p_1, \ldots,p_n\}$\/ et on pose \[
					P_r : ``\,\forall f : \mathds{B}^{\{p_1, \ldots, p_r\}}\to \mathds{B},\,\exists G \in \mathcal{F}_{{\tt if}},\,\left\llbracket G \right\rrbracket = f."
				\]
				Soit $r \in  \left\llbracket 2,n \right\rrbracket$\/ et $f$\/ une fonction booléenne définie sur $\mathds{B}^{\{p_1, \ldots, p_r\}}$\/ à valeurs dans $\mathds{B}$. Soit \begin{align*}
					g: \mathds{B}^{\{p_1,\ldots,p_{r-1}\}} &\longrightarrow \mathds{B} \\
					\rho' &\longmapsto f\big(\rho' \uplus (p_r \mapsto {\bfm V}\/)\big).
				\end{align*}
				où $\uplus$\/ est défini comme dans l'exemple $(p\mapsto {\bfm V}, q\mapsto {\bfm F}) \uplus (r \mapsto {\bfm V}) = (p\mapsto {\bfm V}, q\mapsto {\bfm F}, r \mapsto {\bfm V})$.
				Soit alors $G$\/ par hypothèse de récurrence tel que $\left\llbracket G \right\rrbracket = g$.
				Soit \begin{align*}
					h: \mathds{B}^{\{p_1, \ldots,p_{r-1}\}} &\longrightarrow \mathds{B} \\
					\rho' &\longmapsto f\big(\rho'\uplus (p_r \mapsto {\bfm F})\big).
				\end{align*}
				Soit alors $H$\/ par hypothèse de récurrence tel que $\left\llbracket H \right\rrbracket = h$.

				On pose alors $A = {\tt if}\/\ p_r\ {\tt then}\/\ G\ {\tt else}\/\ H$. Montrons que $\left\llbracket A \right\rrbracket = f$. Soit $\rho \in \mathds{B}^{\{p_1,\ldots,p_r\}}$.
				\begin{itemize}
					\item Si $\rho(p_r) = {\bfm V}$\/ alors
						\begin{align*}
							\left\llbracket A \right\rrbracket^\rho &=  \left\llbracket G \right\rrbracket^\rho\\
							&= \left\llbracket G \right\rrbracket^{\rho_{|\{p_1,\ldots,p_{r-1}\}}}\\
							&= g\left(\rho_{|\{p_1, \ldots, p_{r-1}\}}\right) \\
							&= f\left( \rho_{|\{p_1,\ldots,p_{r-1}\}} \uplus (p_r \mapsto {\bfm V}) \right) \\
							&= f(\rho) \\
						\end{align*}
					\item Si $\rho(p_r) = \mathbf{F}$, alors
						\begin{align*}
							\left\llbracket A \right\rrbracket^\rho &= \left\llbracket H \right\rrbracket^\rho \\
							&= \left\llbracket H \right\rrbracket^{\rho_{|\{p_1,\ldots,p_{r-1}\}}} \\
							&= h\left( \rho_{|\{p_1,\ldots,p_{r-1}\}} \right) \\
							&= f\left(\rho_{|\{p_1,\ldots,p_{r-1}\}}\uplus (p_r \mapsto \mathbf{F})\right) \\
							&= f(\rho) \\
						\end{align*}
				\end{itemize}
		\end{itemize}
\end{enumerate}

