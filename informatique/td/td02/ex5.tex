\section{Axiomatisation algèbre de \textsc{Boole}}

\begin{enumerate}
	\item On pose, pour $t \in \mathds{T}$, $P(t) : ``t \simeq 0 \text{ ou } t \simeq 1{,}"$ et on démontre cette propriété par induction.
		\begin{itemize}
			\item On a bien $P(\mathbf{0})$\/ et $P(\mathbf{1})$.
			\item Soient $t_1,t_2 \in \mathds{T}$.
				\begin{enumerate}
					\item Si $t_1 \simeq \mathbf{1}$, alors $\bar{t}_1 \simeq \bar{\mathbf{1}} \simeq \bar{\bar{\mathbf{0}}} \simeq \mathbf{0}$ ; si $t_1 \simeq \mathbf{0}$, alors $\bar{t}_1 \simeq \bar{\mathbf{0}} \simeq \mathbf{1}$.
					\item Si $t_1 \simeq \mathbf{0}$, alors $t_1 \cdot t_2 \simeq \mathbf{0} \cdot t_2 \simeq \mathbf{0}$\/ ; si $t_1 \simeq \mathbf{1}$, alors $t_1 \cdot t_2 \simeq \mathbf{1} \cdot t_2 \simeq t_2$\/ (qui est équivalent à $\mathbf{0}$\/ ou $\mathbf{1}$).
					\item Si $t_1 \simeq \mathbf{1}$, alors $t_1 + t_2 \simeq \mathbf{1} + t_2 \simeq \mathbf{1}$ ; si $t_1 \simeq \mathbf{0}$, alors $t_1 + t_2 \simeq \mathbf{0} + t_2 \simeq t_2$\/ (qui est équivalent à $\mathbf{0}$\/ ou $\mathbf{1}$)
				\end{enumerate}
		\end{itemize}
	\item En reprenant les relations trouvées dans la question précédente, on construit les tables ci-dessous.
		\[
			\begin{array}{cc|c}
				t_1&t_2&t_1\cdot t_2\\ \hline
				\mathbf{0}&\mathbf{0}&\mathbf{0} \\
				\mathbf{0}&\mathbf{1}&\mathbf{0} \\
				\mathbf{1}&\mathbf{0}&\mathbf{0} \\
				\mathbf{1}&\mathbf{1}&\mathbf{1} \\
			\end{array}
			\qquad\qquad
			\begin{array}{cc|c}
				t_1&t_2&t_1 +t_2\\ \hline
				\mathbf{0}&\mathbf{0}&\mathbf{0} \\
				\mathbf{0}&\mathbf{1}&\mathbf{1} \\
				\mathbf{1}&\mathbf{0}&\mathbf{1} \\
				\mathbf{1}&\mathbf{1}&\mathbf{1} \\
			\end{array}
			\qquad\qquad
			\begin{array}{c|c}
				t_1 & \bar{t}_1\\ \hline
				\mathbf{0}&\mathbf{1}\\
				\mathbf{1}&\mathbf{0}
			\end{array}
		\]
\end{enumerate}


