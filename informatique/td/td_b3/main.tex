\documentclass[a4paper]{article}

\usepackage[margin=1in]{geometry}
\usepackage[utf8]{inputenc}
\usepackage[T1]{fontenc}
\usepackage{mathrsfs}
\usepackage{textcomp}
\usepackage[french]{babel}
\usepackage{amsmath}
\usepackage{amssymb}
\usepackage{cancel}
\usepackage{frcursive}
\usepackage[inline]{asymptote}
\usepackage{tikz}
\usepackage[european,straightvoltages,europeanresistors]{circuitikz}
\usepackage{tikz-cd}
\usepackage{tkz-tab}
\usepackage[b]{esvect}
\usepackage[framemethod=TikZ]{mdframed}
\usepackage{centernot}
\usepackage{diagbox}
\usepackage{dsfont}
\usepackage{fancyhdr}
\usepackage{float}
\usepackage{graphicx}
\usepackage{listings}
\usepackage{multicol}
\usepackage{nicematrix}
\usepackage{pdflscape}
\usepackage{stmaryrd}
\usepackage{xfrac}
\usepackage{hep-math-font}
\usepackage{amsthm}
\usepackage{thmtools}
\usepackage{indentfirst}
\usepackage[framemethod=TikZ]{mdframed}
\usepackage{accents}
\usepackage{soulutf8}
\usepackage{mathtools}
\usepackage{bodegraph}
\usepackage{slashbox}
\usepackage{enumitem}
\usepackage{calligra}
\usepackage{cinzel}
\usepackage{BOONDOX-calo}

% Tikz
\usetikzlibrary{babel}
\usetikzlibrary{positioning}
\usetikzlibrary{calc}

% global settings
\frenchspacing
\reversemarginpar
\setuldepth{a}

%\everymath{\displaystyle}

\frenchbsetup{StandardLists=true}

\def\asydir{asy}

%\sisetup{exponent-product=\cdot,output-decimal-marker={,},separate-uncertainty,range-phrase=\;à\;,locale=FR}

\setlength{\parskip}{1em}

\theoremstyle{definition}

% Changing math
\let\emptyset\varnothing
\let\ge\geqslant
\let\le\leqslant
\let\preceq\preccurlyeq
\let\succeq\succcurlyeq
\let\ds\displaystyle
\let\ts\textstyle

\newcommand{\C}{\mathds{C}}
\newcommand{\R}{\mathds{R}}
\newcommand{\Z}{\mathds{Z}}
\newcommand{\N}{\mathds{N}}
\newcommand{\Q}{\mathds{Q}}

\renewcommand{\O}{\emptyset}

\newcommand\ubar[1]{\underaccent{\bar}{#1}}

\renewcommand\Re{\expandafter\mathfrak{Re}}
\renewcommand\Im{\expandafter\mathfrak{Im}}

\let\slantedpartial\partial
\DeclareRobustCommand{\partial}{\text{\rotatebox[origin=t]{20}{\scalebox{0.95}[1]{$\slantedpartial$}}}\hspace{-1pt}}

% merging two maths characters w/ \charfusion
\makeatletter
\def\moverlay{\mathpalette\mov@rlay}
\def\mov@rlay#1#2{\leavevmode\vtop{%
   \baselineskip\z@skip \lineskiplimit-\maxdimen
   \ialign{\hfil$\m@th#1##$\hfil\cr#2\crcr}}}
\newcommand{\charfusion}[3][\mathord]{
    #1{\ifx#1\mathop\vphantom{#2}\fi
        \mathpalette\mov@rlay{#2\cr#3}
      }
    \ifx#1\mathop\expandafter\displaylimits\fi}
\makeatother

% custom math commands
\newcommand{\T}{{\!\!\,\top}}
\newcommand{\avrt}[1]{\rotatebox{-90}{$#1$}}
\newcommand{\bigcupdot}{\charfusion[\mathop]{\bigcup}{\cdot}}
\newcommand{\cupdot}{\charfusion[\mathbin]{\cup}{\cdot}}
%\newcommand{\danger}{{\large\fontencoding{U}\fontfamily{futs}\selectfont\char 66\relax}\;}
\newcommand{\tendsto}[1]{\xrightarrow[#1]{}}
\newcommand{\vrt}[1]{\rotatebox{90}{$#1$}}
\newcommand{\tsup}[1]{\textsuperscript{\underline{#1}}}
\newcommand{\tsub}[1]{\textsubscript{#1}}

\renewcommand{\mod}[1]{~\left[ #1 \right]}
\renewcommand{\t}{{}^t\!}
\newcommand{\s}{\text{\calligra s}}

% custom units / constants
%\DeclareSIUnit{\litre}{\ell}
\let\hbar\hslash

% header / footer
\pagestyle{fancy}
\fancyhead{} \fancyfoot{}
\fancyfoot[C]{\thepage}

% fonts
\let\sc\scshape
\let\bf\bfseries
\let\it\itshape
\let\sl\slshape

% custom math operators
\let\th\relax
\let\det\relax
\DeclareMathOperator*{\codim}{codim}
\DeclareMathOperator*{\dom}{dom}
\DeclareMathOperator*{\gO}{O}
\DeclareMathOperator*{\po}{\text{\cursive o}}
\DeclareMathOperator*{\sgn}{sgn}
\DeclareMathOperator*{\simi}{\sim}
\DeclareMathOperator{\Arccos}{Arccos}
\DeclareMathOperator{\Arcsin}{Arcsin}
\DeclareMathOperator{\Arctan}{Arctan}
\DeclareMathOperator{\Argsh}{Argsh}
\DeclareMathOperator{\Arg}{Arg}
\DeclareMathOperator{\Aut}{Aut}
\DeclareMathOperator{\Card}{Card}
\DeclareMathOperator{\Cl}{\mathcal{C}\!\ell}
\DeclareMathOperator{\Cov}{Cov}
\DeclareMathOperator{\Ker}{Ker}
\DeclareMathOperator{\Mat}{Mat}
\DeclareMathOperator{\PGCD}{PGCD}
\DeclareMathOperator{\PPCM}{PPCM}
\DeclareMathOperator{\Supp}{Supp}
\DeclareMathOperator{\Vect}{Vect}
\DeclareMathOperator{\argmax}{argmax}
\DeclareMathOperator{\argmin}{argmin}
\DeclareMathOperator{\ch}{ch}
\DeclareMathOperator{\com}{com}
\DeclareMathOperator{\cotan}{cotan}
\DeclareMathOperator{\det}{det}
\DeclareMathOperator{\id}{id}
\DeclareMathOperator{\rg}{rg}
\DeclareMathOperator{\rk}{rk}
\DeclareMathOperator{\sh}{sh}
\DeclareMathOperator{\th}{th}
\DeclareMathOperator{\tr}{tr}

% colors and page style
\definecolor{truewhite}{HTML}{ffffff}
\definecolor{white}{HTML}{faf4ed}
\definecolor{trueblack}{HTML}{000000}
\definecolor{black}{HTML}{575279}
\definecolor{mauve}{HTML}{907aa9}
\definecolor{blue}{HTML}{286983}
\definecolor{red}{HTML}{d7827e}
\definecolor{yellow}{HTML}{ea9d34}
\definecolor{gray}{HTML}{9893a5}
\definecolor{grey}{HTML}{9893a5}
\definecolor{green}{HTML}{a0d971}

\pagecolor{white}
\color{black}

\begin{asydef}
	settings.prc = false;
	settings.render=0;

	white = rgb("faf4ed");
	black = rgb("575279");
	blue = rgb("286983");
	red = rgb("d7827e");
	yellow = rgb("f6c177");
	orange = rgb("ea9d34");
	gray = rgb("9893a5");
	grey = rgb("9893a5");
	deepcyan = rgb("56949f");
	pink = rgb("b4637a");
	magenta = rgb("eb6f92");
	green = rgb("a0d971");
	purple = rgb("907aa9");

	defaultpen(black + fontsize(8pt));

	import three;
	currentlight = nolight;
\end{asydef}

% theorems, proofs, ...

\mdfsetup{skipabove=1em,skipbelow=1em, innertopmargin=6pt, innerbottommargin=6pt,}

\declaretheoremstyle[
	headfont=\normalfont\itshape,
	numbered=no,
	postheadspace=\newline,
	headpunct={:},
	qed=\qedsymbol]{demstyle}

\declaretheorem[style=demstyle, name=Démonstration]{dem}

\newcommand\veczero{\kern-1.2pt\vec{\kern1.2pt 0}} % \vec{0} looks weird since the `0' isn't italicized

\makeatletter
\renewcommand{\title}[2]{
	\AtBeginDocument{
		\begin{titlepage}
			\begin{center}
				\vspace{10cm}
				{\Large \sc Chapitre #1}\\
				\vspace{1cm}
				{\Huge \calligra #2}\\
				\vfill
				Hugo {\sc Salou} MPI${}^{\star}$\\
				{\small Dernière mise à jour le \@date }
			\end{center}
		\end{titlepage}
	}
}

\newcommand{\titletp}[4]{
	\AtBeginDocument{
		\begin{titlepage}
			\begin{center}
				\vspace{10cm}
				{\Large \sc tp #1}\\
				\vspace{1cm}
				{\Huge \textsc{\textit{#2}}}\\
				\vfill
				{#3}\textit{MPI}${}^{\star}$\\
			\end{center}
		\end{titlepage}
	}
	\fancyfoot{}\fancyhead{}
	\fancyfoot[R]{#4 \textit{MPI}${}^{\star}$}
	\fancyhead[C]{{\sc tp #1} : #2}
	\fancyhead[R]{\thepage}
}

\newcommand{\titletd}[2]{
	\AtBeginDocument{
		\begin{titlepage}
			\begin{center}
				\vspace{10cm}
				{\Large \sc td #1}\\
				\vspace{1cm}
				{\Huge \calligra #2}\\
				\vfill
				Hugo {\sc Salou} MPI${}^{\star}$\\
				{\small Dernière mise à jour le \@date }
			\end{center}
		\end{titlepage}
	}
}
\makeatother

\newcommand{\sign}{
	\null
	\vfill
	\begin{center}
		{
			\fontfamily{ccr}\selectfont
			\textit{\textbf{\.{\"i}}}
		}
	\end{center}
	\vfill
	\null
}

\renewcommand{\thefootnote}{\emph{\alph{footnote}}}

% figure support
\usepackage{import}
\usepackage{xifthen}
\pdfminorversion=7
\usepackage{pdfpages}
\usepackage{transparent}
\newcommand{\incfig}[1]{%
	\def\svgwidth{\columnwidth}
	\import{./figures/}{#1.pdf_tex}
}

\pdfsuppresswarningpagegroup=1
\ctikzset{tripoles/european not symbol=circle}

\newcommand{\missingpart}{{\large\color{red} Il manque quelque chose ici\ldots}}


\titletd{B 3}{Invariants plus complexes}

\begin{document} % EXACT
	\begin{lstlisting}[language=caml,caption=Tri par insertion]
let swap (t: int array) (i: int) (j: int): unit =
	let tmp = t.(i) in
	t.(i) <- t.(j);
	t.(j) <- tmp

let range (t: int array) (idx: int): unit =
	(* idx : endroit mal range, t[0, idx - 1] : trie *)
	let curseur = ref idx in
	while (!curseur > 0 && t.(!curseur - 1) > t.(!curseur)) do
		swap t (!curseur) (!curseur - 1);
		curseur := !curseur - 1
	done

let tri_insertion (t: int array): unit =
	let n = Array.length t in
	for i = 0 to n - 1 do
		range t i
	done
	\end{lstlisting}

	\begin{prop}
		Spot $t$ un tableau d'entiers de taille $n$.
		La fonction \texttt{tri\_insertion} appelée sur $t$ termine et produit un tableau $t'$ tel que
		\begin{itemize}
			\item $t'$ est trié,
			\item $t'$ est une permutation de $t$.
		\end{itemize}
	\end{prop}

	\begin{prop}
		Soit $t$ un tableau d'entiers de taille $n$.
		Soit $\mathrm{idx} \in \llbracket 0,n-1 \rrbracket$.
		Si le sous-tableau $t[0,\mathrm{idx} - 1]$ est trié par ordre croissant, alors $\texttt{range}\ t\ \mathrm{idx}$\/ termine et produit un tableau $t'$ tel que
		\begin{itemize}
			\item $t'$ est trié par ordre croissant,
			\item $t'$ est une permutation de $t$.
		\end{itemize}
	\end{prop}

	\begin{prv}
		Soit $t_0$ un tableau de taille $n$.
		Soit $\mathrm{idx} \in \llbracket 0,n-1 \rrbracket$ tel que $t_0[0,\mathrm{idx}-1]$ est trié.
		On considère $(\mathcal{I})$ le système suivant : \[
			(\mathcal{I}): \begin{cases}
				\texttt{curseur} \ge 0 \quad\quad & (\mathcal{P}_1)\\
				t[\texttt{curseur}, \mathrm{idx}] \text{ est trié}\quad\quad & (\mathcal{P}_2)\\
				t[0, \texttt{curseur} - 1] \text{ est trié}\quad\quad & (\mathcal{P}_3)\\
				t \text{ est une permutation de } t_0\quad\quad & (\mathcal{P}_4)
			\end{cases}
		.\]
		Montrons que $(\mathcal{I})$ est un invariant.
		\begin{itemize}
			\item \textbf{Initialisation.} Au début, $t = t_0$ et $\texttt{curseur} = \mathrm{idx}$. Les propriétés de $\mathcal{I}$ sont vérifiées.
			\item \textbf{Hérédité.} Soit $\ubar{t}$, $\underline{\texttt{curseur}}$, et $\bar{t}$, $\overline{\texttt{curseur}}$\/ les valeurs de $t$ et \texttt{curseur} avant (resp. après) une itération de boucle.
				Par hypothèse de récurrence, $\underline{\texttt{curseur}} \ge 0$, le sous-tableau $\ubar{t}[\underline{\texttt{curseur}}, \mathrm{idx}]$ est trié, $\ubar{t}[0,\underline{\texttt{curseur}} - 1]$, $\ubar{t}$ est une permutation de $t_0$.
				De plus, par condition de boucle, $\underline{\texttt{curseur}} > 0$ et $\ubar{t}[\underline{\texttt{curseur}} - 1] > \ubar{t}[\underline{\texttt{curseur}}]$.
				D'après le code, on a également $\overline{\texttt{curseur}} = \underline{\texttt{curseur}} - 1$, et pour tout $i \in \llbracket 0,n-1 \rrbracket$, où $i \not\in \{\underline{\texttt{curseur}}, \underline{\texttt{curseur}} - 1\}$, $\bar{t}[i] = \ubar{t}[i]$.
				Et, $\bar{t} [\underline{\texttt{curseur}}] = \ubar{t}[\underline{\texttt{curseur}} - 1]$, et $\bar{t}[\underline{\texttt{curseur}} - 1] = \bar{t}[\underline{\texttt{curseur}}]$.
				On en déduit trivialement que $\bar{t}$ est une permutation de $\ubar{t}$, qui est une permutation de $t_0$, d'où $(\mathcal{P}_4)$.
				De plus, $\overline{\texttt{curseur}} \ge 0$, d'où $(\mathcal{P}_1)$.
				Montrons $(\mathcal{P}_2)$, \textit{i.e.}\ $\forall i \in \llbracket \overline{\texttt{curseur}}, \mathrm{idx} - 1 \rrbracket$, $\bar{t}[i] \le \bar{t}[i+1]$.
				Soit donc $i \in \llbracket \overline{\texttt{curseur}}, \mathrm{idx} - 1 \rrbracket$.
				\begin{itemize}
					\item Si $i = \overline{\texttt{curseur}}$, alors $\bar{t}[i] = \bar{t}[\overline{\texttt{curseur}}] = \bar{t}[\underline{\texttt{curseur}} - 1] = \ubar{t}[\underline{\texttt{curseur}}] < \ubar{t}[\underline{\texttt{curseur}} - 1] < \bar{t}[\underline{\texttt{curseur}}] < \bar{t}[\overline{\texttt{curseur}} + 1] < \bar{t}[i+1]$.
					\item Si $i \in \llbracket \overline{\texttt{curseur}} + 1, \mathrm{idx} - 1 \rrbracket$, alors $\bar{t}[i] = \ubar{t}[i] \le \ubar{t}[i + 1] \le \bar{t}[i+1]$.
				\end{itemize}
				On en déduit $(\mathcal{P}_2)$.
				Montrons $(\mathcal{P}_3)$.
				Soit $i \in \llbracket 0, \overline{\texttt{curseur}} - 2 \rrbracket$. Alors, $\bar{t}[i] = \ubar{t}[i] \le \ubar{t}[i + 1] \le \bar{t}[i+1]$, d'où $(\mathcal{P}_3)$.
		\end{itemize}
		On enrichit l'invariant en ajoutant la propriété $(\mathcal{P}_5)$ : $\forall i \in \llbracket 0, \texttt{curseur} - 1 \rrbracket$, $\forall j \in \llbracket \texttt{curseur} + 1, \mathrm{idx} \rrbracket$, $\ubar{t}[i] \le \ubar{t}[j]$.
	\end{prv}
\end{document}
