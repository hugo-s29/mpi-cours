\section{Exercice 4}

\paragraph{Q.\ 1}

\begin{algo}
	\textsl{Entrée} : Un automate $\mathcal{A}$\/ ;\\
	\textsl{Sortie} : $\mathcal{L}(\mathcal{A}) = \O$\/ ;\\
	On fait un parcours en largeur depuis les états initiaux et on regarde si on atteint un état final.
\end{algo}

\begin{algo}[Nathan F.]
	{\sl Entrée} : Deux automates $\mathcal{A}$\/ et $\mathcal{B}$\/ \\
	{\sl Sortie} : $\mathcal{L}(\mathcal{A}) = \mathcal{L}(\mathcal{B})$\/ ;
	Soit $\mathcal{C}$\/ l'automate reconnaissant $\mathcal{L}(\mathcal{A}) \mathbin\triangle \mathcal{L}(\mathcal{B})$. On retourne $\mathcal{L}(\mathcal{C}) \mathrel{\ds\mathop=^?} \O$\/ à l'aide de l'algorithme précédent.
\end{algo}

Autre possibilité, on procède par double inclusion :

\begin{algo}[$\subseteq$]
	{\sl Entrée} : Deux automates $\mathcal{A}$\/ et $\mathcal{B}$\/ \\
	{\sl Sortie} : $\mathcal{L}(\mathcal{A}) \subseteq \mathcal{L}(\mathcal{B})$\/ ;
	On retourne $\mathcal{A} \setminus \mathcal{B} \mathrel{\ds\mathop=^?} \O$.
\end{algo}

\paragraph{Q.\ 2}
L'algorithme reconnaissant $\mathcal{L}(\mathcal{A}) \mathrel\triangle \mathcal{L}(\mathcal{B})$\/ doit être déterminisé, sa complexité est donc au moins de $2^n$.
