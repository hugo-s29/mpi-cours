\section{Exercice 1 : Vérification d'égalité polynomiale}

\begin{enumerate}
	\item Étant donnés deux tableaux représentant deux polynômes, on peut calculer leurs produit en concaténant ce tableau. La complexité du produit de polynômes avec cet algorithme est en $\mathcal{O}(nm)$\/ où $n$\/ est le degré du 1er polynôme, et $m$\/ est le degré du second. En effet, \textit{dans le pire des cas}, tous les polynômes représentant les deux polynômes sont des monômes, or, la concaténation étant en $\mathcal{O}(nm)$ (pour un tableau de taille $n$\/ et un de taille $m$). D'où la complexité en $\mathcal{O}(nm)$.
	\item Afin d'évaluer ces polynômes, on utilise l'algorithme de \textsc{Horner}, qui est en $\mathcal{O}(n)$, donc en temps linéaire.
	\item En développant ces polynômes, la complexité serait en $\mathcal{O}(n^3)$. En effet, la multiplication de deux polynômes de degrés $n$\/ a une complexité en $\mathcal{O}(n^2)$. D'où la complexité en $\mathcal{O}(n^3)$\/ pour la multiplication de deux polynômes ayant chacun un degré $n$.
	\item Un polynôme de degré $n$\/ a, au plus, $n$\/ racines. D'où, le polynôme $P - Q$, a au plus $n$\/ racines (où $n = \max(\deg P, \deg Q)$). Ainsi, s'il a $n+1$\/ racines, c'est alors le polynôme nul, et donc $P = Q$.
		\begin{algorithm}[H]
			\centering
			\begin{algorithmic}
				\State {\bf Entrée}\/ : $P = (P_{i})_{i \in \left\llbracket 1,m \right\rrbracket }$\/ et $Q = (Q_j)_{j \in \left\llbracket 1,p \right\rrbracket }$\/ deux polynômes
				\State $n \gets \deg P$\/
				\For{$i \in \left\llbracket 0,n \right\rrbracket$}
					\If{$P(i) \neq Q(i)$} \Comment{Avec l'algorithme de \textsc{Horner}, évaluation en $\mathcal{O}(n)$}
						\State \Return {\sc Non}\/
					\EndIf
				\EndFor
				\State \Return {\sc Oui}\/
			\end{algorithmic}
			\caption{Algorithme déterministe pout tester l'égalité polynomiale en $\mathcal{O}(n^2)$}
		\end{algorithm}

	\item~
		\begin{algorithm}[H]
			\centering
			\begin{algorithmic}[1]
				\Entree $P = (P_{i})_{i \in \left\llbracket 1,n \right\rrbracket }$\/ et $Q = (Q_j)_{j \in \left\llbracket 1,n \right\rrbracket }$\/ deux polynômes, et $k \in \N$\/ un entier
				\State $x \gets \mathcal{U}(\left\llbracket 1,k\times n \right\rrbracket)$\/ \phantom{$\frac00$}
				\If{$P(x) \neq Q(x)$}
					\State\Return {\sc Non}\/
				\EndIf
				\State\Return {\sc Oui}\/
			\end{algorithmic}
			\caption{Algorithme probabiliste pout tester l'égalité polynomiale en $\mathcal{O}(n)$}
		\end{algorithm}

		Soit $X$\/ la variable aléatoire de $\mathcal{U}(\left\llbracket 1,k\times n \right\rrbracket)$.
		L'événement ``$P \neq Q$\/ mais l'algorithme retourne {\sc Oui}'' arrive si $X \in \{j \in \left\llbracket 1,kn \right\rrbracket  \mid P(j) = Q(j)\} = A$. Or $|A| \le n$, et $A \subseteq \left\llbracket 1,kn \right\rrbracket$. Ainsi, l'événement a une probabilité de $\frac{1}{k}$.
\end{enumerate}
