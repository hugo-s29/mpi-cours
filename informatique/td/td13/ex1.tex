\section{Attracteurs pour le jeu de la soustraction généralisé}

\begin{enumerate}
	\item On considère une partie nulle.
		\begin{itemize}
			\item Le nombre d'allumettes, à chaque tours, décroît strictement dans $(\N, {\le})$. Or, $(\N, {\le})$ est bien fondé, donc la partie se termine.
			\item Il existe donc un joueur $j$ qui prend la dernière allumette. Ce joueur perd et l'autre gagne, donc la partie est non nulle.
		\end{itemize}
	\item Montrons que
		\begin{gather*}
			\mathcal{A} = \{(\mathrm{A},i)  \mid i \in \llbracket 1,n \rrbracket,\: i\not\equiv 1\mod{k + 1} \} \cup \{(\mathrm{B}, i)  \mid i \in \llbracket 1,n \rrbracket,\: i \equiv 1 \mod {k + 1}\}\\
			\mathcal{B} = \{(\mathrm{B},i)  \mid i \in \llbracket 1,n \rrbracket,\: i\not\equiv 1\mod{k + 1} \} \cup \{(\mathrm{A}, i)  \mid i \in \llbracket 1,n \rrbracket,\: i \equiv 1 \mod {k + 1}\}\\
		\end{gather*}
		On pose, pour tout $p \in \N$, 
\end{enumerate}
