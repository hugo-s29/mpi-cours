\section{Rendez-vous à l'aide de sémaphores}

\subsection{Deux fils d'exécutions se rencontrent une fois}

\begin{enumerate}
	\item
		On considère les fils $P_1$ et $P_2$. Dans le programme principal, on crée deux sémaphores~$\mathcal{S}_1$ et $\mathcal{S}_2$, initialisés à $0$.
		\begin{multicols}{2}
			\begin{algorithm}[H]
				\centering
				\begin{algorithmic}[1]
					\State $A_1$
					\State $\texttt{release}\ \mathcal{S}_2$
					\State $\texttt{acquire}\ \mathcal{S}_1$
					\State $B_1$
				\end{algorithmic}
				\caption{Fil d'exécution $P_1$}
			\end{algorithm}
			\begin{algorithm}[H]
				\centering
				\begin{algorithmic}[1]
					\State $A_2$
					\State $\texttt{release}\ \mathcal{S}_1$
					\State $\texttt{acquire}\ \mathcal{S}_2$
					\State $B_2$
				\end{algorithmic}
				\caption{Fil d'exécution $P_2$}
			\end{algorithm}
		\end{multicols}
		Pour généraliser, on considère 3 sémaphores $\mathcal{S}_1$, $\mathcal{S}_2$ et $\mathcal{S}_3$ initialement à $0$. On définit donc les fils $P_1$, $P_2$ et $P_3$ définis ci-dessous.
		\begin{multicols}{3}
			\begin{algorithm}[H]
				\centering
				\begin{algorithmic}[1]
					\State $A_1$
					\State $\texttt{release}\ \mathcal{S}_2$
					\State $\texttt{release}\ \mathcal{S}_3$
					\State $\texttt{acquire}\ \mathcal{S}_1$
					\State $\texttt{acquire}\ \mathcal{S}_1$
					\State $B_1$
				\end{algorithmic}
				\caption{Fil d'exécution $P_1$}
			\end{algorithm}
			\begin{algorithm}[H]
				\centering
				\begin{algorithmic}[1]
					\State $A_2$
					\State $\texttt{release}\ \mathcal{S}_1$
					\State $\texttt{release}\ \mathcal{S}_3$
					\State $\texttt{acquire}\ \mathcal{S}_2$
					\State $\texttt{acquire}\ \mathcal{S}_2$
					\State $B_2$
				\end{algorithmic}
				\caption{Fil d'exécution $P_2$}
			\end{algorithm}
			\begin{algorithm}[H]
				\centering
				\begin{algorithmic}[1]
					\State $A_3$
					\State $\texttt{release}\ \mathcal{S}_1$
					\State $\texttt{release}\ \mathcal{S}_2$
					\State $\texttt{acquire}\ \mathcal{S}_3$
					\State $\texttt{acquire}\ \mathcal{S}_3$
					\State $B_3$
				\end{algorithmic}
				\caption{Fil d'exécution $P_3$}
			\end{algorithm}
		\end{multicols}
	\item
		On perds l'indépendance des fils $P_1$ et $P_2$.
		Pour généraliser la solution, on considère les algorithmes ci-dessous.

		\begin{multicols}{3}
			\begin{algorithm}[H]
				\centering
				\begin{algorithmic}[1]
					\State $A_1$
				\end{algorithmic}
				\caption{Fil $P_1$}
			\end{algorithm}
			\begin{algorithm}[H]
				\centering
				\begin{algorithmic}[1]
					\State $B_1$
				\end{algorithmic}
				\caption{Fil $P_1'$}
			\end{algorithm}
			\begin{algorithm}[H]
				\centering
				\begin{algorithmic}[1]
					\State $A_2$
				\end{algorithmic}
				\caption{Fil $P_2$}
			\end{algorithm}
			\begin{algorithm}[H]
				\centering
				\begin{algorithmic}[1]
					\State $B_2$
				\end{algorithmic}
				\caption{Fil $P_2'$}
			\end{algorithm}
			\begin{algorithm}[H]
				\centering
				\begin{algorithmic}[1]
					\State $A_3$
				\end{algorithmic}
				\caption{Fil $P_3$}
			\end{algorithm}
			\begin{algorithm}[H]
				\centering
				\begin{algorithmic}[1]
					\State $B_3$
				\end{algorithmic}
				\caption{Fil $P_3'$}
			\end{algorithm}
		\end{multicols}

		\begin{algorithm}[H]
			\centering
			\begin{algorithmic}[1]
				\State lancer $P_1$, $P_2$ et $P_3$ en parallèle
				\State attendre la fin de $P_1$, $P_2$ et $P_3$
				\State lancer $P_1'$, $P_2'$ et $P_3'$ en parallèle
			\end{algorithmic}
			\caption{Programme principal}
		\end{algorithm}

	\item Non, les instructions $B_2$ et $A_1$ pourraient être exécutées en parallèle, mais ce n'est pas possible avec la subdivision des fils.
\end{enumerate}

\subsection{Plusieurs fils se rencontrent une fois}

\begin{enumerate}[start=4]
	\item On considère l'algorithme ci-dessous.
		\begin{algorithm}[H]
			\centering
			\begin{algorithmic}[1]
				\Procedure{CréeBarrière}{}
				\State Soit un verrou $\mathcal{V}$.
				\State Soit $i = 0$.
				\State Soit un sémaphore $\mathcal{S}$ initialisé à $0$.
				\State\Return $\mathcal{b} = (\mathcal{V}, \mathcal{S}, i)$.
				\EndProcedure
				\Procedure{AppelBarrière}{$b$}
				\State $(\mathcal{V}, \mathcal{S}, i) \gets \mathcal{b}$
				\State $\texttt{lock}(\mathcal{V})$
				\State $i \gets i + 1$
				\If{$i = n$}
				\For{$k \in \llbracket 1,n \rrbracket$}
				\State $\texttt{release}\ \mathcal{S}$
				\EndFor
				\EndIf
				\State $\texttt{unlock}(\mathcal{V})$
				\State $\texttt{acquire}\ \mathcal{S}$
				\EndProcedure
			\end{algorithmic}
			\caption{Implémentation de la structure barrière $\mathcal{b}$}
		\end{algorithm}
\end{enumerate}

\subsection{Plusieurs fils d'exécution se rencontrent plusieurs fois}

\begin{enumerate}[start=5]
	\item 
	\item On considère l'algorithme suivant.
		\begin{algorithm}[H]
			\centering
			\begin{algorithmic}[1]
				\Procedure{CréeBarrière}{$n$}
				\State Soit un verrou $\mathcal{V}$.
				\State Soit $i = 0$, et soit $\mathrm{nb} = i$.
				\State Soit un sémaphore $\mathcal{S}$ initialisé à $0$.
				\State\Return $\mathcal{b} = (\mathcal{V}, \mathcal{S}, i, \mathrm{nb})$.
				\EndProcedure
				\Procedure{AppelBarrière}{$b$}
				\State $(\mathcal{V}, \mathcal{S}, i) \gets \mathcal{b}$
				\State $\texttt{lock}(\mathcal{V})$
				\State $i \gets i + 1$
				\If{$i = n$}
				\For{$k \in \llbracket 1,n \rrbracket$}
				\State $\texttt{release}\ \mathcal{S}$
				\EndFor
				\EndIf
				\State $\texttt{unlock}(\mathcal{V})$
				\State $\texttt{acquire}\ \mathcal{S}$
				\EndProcedure
				\Procedure{JeNeViendraisPlus}{$\mathcal{b}$}
				\State $(\mathcal{V}, \mathcal{S}, i, \mathrm{nb}) \gets \mathcal{b}$
				\State $\texttt{lock}(\mathcal{V})$\/
				\State $\mathrm{nb} \gets \mathrm{nb} - 1$
				\If {$i = \mathrm{nb}$}
				\For{$k \in \llbracket 1,\mathrm{nb} \rrbracket$}
				\State $\texttt{release}\ \mathcal{S}$
				\EndFor
				\State $i \gets 0$
				\EndIf
				\State $\texttt{unlock}(\mathcal{V})$
				\EndProcedure
			\end{algorithmic}
			\caption{Implémentation de la structure barrière robuste $\mathcal{b}$}
		\end{algorithm}
\end{enumerate}

