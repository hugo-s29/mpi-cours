\begin{prv}[par réduction de \textsc{Sat} à 3\textsc{sat}]
	Soit $G$\/ une formule sur $\mathcal{Q}$, un ensemble de variables propositionnelles. Pour toute sous-formule $H$, on note 
	\begin{itemize}
		\item $x_H$\/ une variable propositionnelle,
		\item $K_H$\/ une formule définie par
			\begin{itemize}
				\item si $H = \top$, $H = \bot$ ou $H = p \in \mathcal{Q}$, alors $K_H = H$,
				\item si $H = \lnot H_1$, alors $K_H = \lnot x_{H_1}$,
				\item si $H = H_1 \odot H_2$, avec $\odot \in \{{\to},{\lor},{\land},{\leftrightarrow}\}$, alors $K_H = x_{H_1} \odot x_{H_2}$.
			\end{itemize}
	\end{itemize}
	Définissons alors la formule \[
		K = \bigwedge_{H \text{ sous-formule de } G} (x_H \leftrightarrow K_H)
	.\]
	On note aussi, si $\mathcal{Q} \subseteq \mathcal{Q}'$\/ deux ensembles de variables propositionnelles, et $\rho \in \mathds{B}^\mathcal{Q}$, on note $\rho' \sqsupseteq \rho$\/ dès lors que $\mathrm{def}(\rho') = \mathcal{Q}'$\/ et $\forall x \in \mathcal{Q}$, $\rho(x) = \rho'(x)$.
	On pose $\mathcal{Q}' = \mathcal{Q} \cup \{x_H  \mid H \text{ sous-formule de } G \}$.
	On considère à présent le lemme suivant.
	\begin{lem}
		Soit $\rho \in \mathds{B}^\mathcal{Q}$. Il existe $\rho' \in \mathds{B}^{\mathcal{Q}'}$\/ tel que $\rho' \sqsupseteq \rho$\/ et $\left\llbracket K \right\rrbracket^\rho = \mathbf{V}$.
	\end{lem}
	\noindent Prouvons ce lemme.
	\begin{prvk}
		On définit \[
			\rho'(x) = \begin{cases}
				\rho(x) &\quad \text{ si } x \in \mathcal{Q}\\
				\left\llbracket H \right\rrbracket^\rho &\quad \text{ si } x = x_H.
			\end{cases}
		\]
		Soit alors $H$\/ une sous-formule de $G$.
		\begin{itemize}
			\item Si $H = \top $, alors $\rho'(x_H) = \left\llbracket H \right\rrbracket^\rho = \left\llbracket x_H \right\rrbracket^{\rho'}$, et $\left\llbracket K_H \right\rrbracket^{\rho'} = \left\llbracket H \right\rrbracket^{\rho'} = \left\llbracket \top \right\rrbracket^{\rho'} = \left\llbracket H \right\rrbracket^{\rho}$. Ainsi, $\left\llbracket x_H \leftrightarrow K_H \right\rrbracket^{\rho'} = \mathbf{V}$.
			\item Si $H = \lnot H_1$, alors $\left\llbracket x_H \right\rrbracket^{\rho'} = \left\llbracket H \right\rrbracket^\rho$, et \[
					\left\llbracket K_H \right\rrbracket^{\rho'} = \left\llbracket \lnot x_{H_1} \right\rrbracket^{\rho'} = \overline{\left\llbracket x_{H_1} \right\rrbracket^{\smash{\substack{~\\[1mm]\rho'}}}} = \overline{\left\llbracket H_1 \right\rrbracket^\rho} = \left\llbracket H \right\rrbracket^\rho
				.\] On en déduit que $\left\llbracket x_H \leftrightarrow K_H \right\rrbracket^{\rho'} = \mathbf{V}$.
			\item Si $H = H_1 \land H_2$, alors $\left\llbracket x_H \right\rrbracket^{\rho'} = \left\llbracket H \right\rrbracket^\rho$, et 
				\begin{align*}
					\left\llbracket K_H \right\rrbracket^{\rho'} &= \left\llbracket x_{H_1} \land x_{H_2} \right\rrbracket^{\rho'}\\
					&= \left\llbracket x_{H_1} \right\rrbracket^{\rho'} \cdot \left\llbracket x_{H_2} \right\rrbracket^{\rho'} \\
					&= \left\llbracket H_1 \right\rrbracket^\rho \cdot \left\llbracket H_2 \right\rrbracket^\rho \\
					&= \left\llbracket H_1 \land H_2 \right\rrbracket^\rho \\
					&= \left\llbracket H \right\rrbracket^\rho \\
				\end{align*}
			\item De même pour les autres cas\ldots
		\end{itemize}
		On a donc
		\begin{align*}
			\Big\llbracket \bigwedge_{H \text{ sous-formule de } G} x_H \leftrightarrow K_H\Big\rrbracket^{\rho'}
			&= \bigdot_{H \text{ sous-formule de } G} \left\llbracket x_H \leftrightarrow K_H \right\rrbracket^{\rho'} \\
			&= \mathbf{V} \\
		\end{align*}
		et donc $\left\llbracket K \right\rrbracket^{\rho'} = \mathbf{V}$.
	\end{prvk}

	\begin{lem}
		Pour tout environnement propositionnel $\rho \in \mathds{B}^\mathcal{Q}$, et pour tout $\rho' \in \mathds{B}^{\mathcal{Q}'}$. Si $\left\llbracket K \right\rrbracket^{\rho'} = \mathbf{V}$, alors $\rho(x_{G}) = \left\llbracket G \right\rrbracket^\rho$.
	\end{lem}

	\begin{prvk}
		Soient $\rho \in \mathds{B}^{\mathcal{Q}}$\/ et $\rho' \in \mathds{B}^{\mathcal{Q}'}$\/ deux environnements propositionnels.
		\textit{Montrons, par induction sur les sous-formules de $G$, pour toute sous-formule $H$\/ de $G$, que $\rho'(x_H) = \left\llbracket H \right\rrbracket^\rho$.}
		\begin{itemize}
			\item Si $H = \top$, alors, comme $\left\llbracket K \right\rrbracket^{\rho'} = \mathbf{V}$, on a $\left\llbracket x_H \leftrightarrow K_H \right\rrbracket^{\rho'} = \mathbf{V}$, donc $\rho'(x_H) = \left\llbracket H \right\rrbracket^{\rho'} = \left\llbracket H \right\rrbracket^\rho$.
			\item Si $H = \lnot H_1$, avec $\rho'(x_{H_1}) = \left\llbracket H_1 \right\rrbracket^\rho$\/ par hypothèse d'induction, alors $\left\llbracket K \right\rrbracket^{\rho'} = \mathbf{V}$\/ donc $\left\llbracket x_H \leftrightarrow K_H \right\rrbracket^{\rho'} = \mathbf{V}$, d'où
				\begin{align*}
					\rho'(x_H) &= \left\llbracket K_H \right\rrbracket^{\rho'}\\
					&= \left\llbracket \lnot x_{H_1} \right\rrbracket^{\rho'} \\
					&= \overline{\left\llbracket x_{H_1} \right\rrbracket^{\smash{\substack{~\\[1mm]\rho'}}}} \\
					&= \overline{\left\llbracket H_1 \right\rrbracket^{\smash{\substack{~\\\rho}}}} \\
					&= \left\llbracket \lnot H_1 \right\rrbracket^{\rho} \\
					&= \left\llbracket H \right\rrbracket^\rho \\
				\end{align*}
			\item Si $H = H_1 \land H_2$, avec $\rho'(x_{H_1}) = \left\llbracket H_1 \right\rrbracket^\rho$, et $\rho'(x_{H_2}) = \left\llbracket H_2 \right\rrbracket^\rho$\/ par hypothèse d'induction, alors $\left\llbracket K \right\rrbracket^\rho = \mathbf{V}$\/ donc $\left\llbracket x_H \leftrightarrow K_H \right\rrbracket^{\rho'} = \mathbf{V}$\/ d'où 
				\begin{align*}
					\rho'(x_H) &= \left\llbracket K_H \right\rrbracket^{\rho'} \\
					&= \left\llbracket x_{H_1} \land x_{H_2} \right\rrbracket^{\rho'} \\
					&= \left\llbracket x_{H_1} \right\rrbracket^{\rho'}\cdot \left\llbracket x_{H_2} \right\rrbracket^{\rho'} \\
					&= \left\llbracket H_1 \right\rrbracket^\rho \cdot  \left\llbracket H_2 \right\rrbracket^\rho \\
					&= \left\llbracket H_1 \land H_2 \right\rrbracket^\rho \\
					&= \left\llbracket H \right\rrbracket^\rho \\
				\end{align*}
			\item De même pour les autres cas\ldots
		\end{itemize}
	\end{prvk}

	À l'instance $G$\/ du problème \textsc{Sat}, on associe donc la formule $K \land x_G$.
	\begin{itemize}
		\item[``$\implies$''] Soit $G \in \textsc{Sat}^+$, soit alors $\rho$\/ tel que $\left\llbracket G \right\rrbracket^\rho = \mathbf{V}$, alors il existe, d'après le premier lemme, un environnement $\rho'$\/ tel que $\left\llbracket K \right\rrbracket^{\rho'} = \mathbf{V}$. Le second lemme nous donne alors $\left\llbracket x_G \right\rrbracket^{\rho'} = \left\llbracket G \right\rrbracket^\rho = \mathbf{V}$ donc $\left\llbracket K \land x_G \right\rrbracket^{\rho'}$\/ d'où $\left\llbracket K \land x_G \right\rrbracket \in \textsc{Sat}^+$.
		\item[``$\impliedby$''] Si $K \land x_G \in \textsc{Sat}^+$, il existe donc $\rho'$\/ tel que $\left\llbracket K \land x_G \right\rrbracket^{\rho'} = \mathbf{V}$\/ donc $\left\llbracket K \right\rrbracket^{\rho'} = \mathbf{V}$\/ et $\rho'(x_G) = \mathbf{V}$. Soit alors $\rho$\/ la restriction de $\rho'$\/ à $\mathcal{Q}$. Ainsi, d'après le second lemme, $\left\llbracket G \right\rrbracket^\rho = \rho'(x_G) = \mathbf{V}$.
	\end{itemize}
\end{prv}

\begin{rmk}
	Pour tout $n \ge 3$, le problème $n$-\textsc{cnf}-\textsc{sat} est \textbf{NP}-difficile. Ceci peut être prouvé par réduction avec la fonction identité à 3\textsc{sat}.
\end{rmk}

\begin{defn}
	On dit qu'un problème est \textbf{NP}-complet s'il est dans la classe \textbf{NP} et dans la classe \textbf{NP}-difficile : \[
		\text{\textbf{NP}-complet} = \text{\textbf{NP}-difficile} \cap \text{\textbf{NP}}
	.\]
	\index{problème!\textbf{NP}-complet}
\end{defn}

\begin{rmk}
	Le problème \textsc{Sat} est \textbf{NP}-complet.
\end{rmk}
