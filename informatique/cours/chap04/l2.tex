%On peut maintenant considérer différents exemples de chaînes de caractères, par exemple, un programme \textsc{OCaml}.

\begin{prop}
	Tout langage régulier est décidable.
\end{prop}

\begin{prv}
	Soit $L$\/ un langage régulier. Montrons que $L$\/ est décidable. On utilise alors la fonction~\textsc{OCaml} suivante de reconnaissance d'un mot dans un automate (que l'on a déjà codé en~\textsc{tp}).
\end{prv}

\begin{prop}[stabilité des langages décidables]
	Un langage décidable est stable par
	\begin{enumerate}
		\item union ;
		\item intersection ;
		\item complémentaire.
	\end{enumerate}
\end{prop}

\begin{prv}
	\begin{enumerate}
		\item Soient $L_1$\/ et $L_2$\/ deux langages décidables. Montrons que $L_1 \cup L_2$\/ est décidable. Soit $\texttt{decide}_1 : \texttt{string} \to \texttt{bool}$\/ la fonction décidant le langage $L_1$.\footnotemark\@ Soit $\texttt{decide}_2 : \texttt{string} \to \texttt{bool}$\/ la fonction décidant du langage $L_2$. On construit alors la fonction
			\begin{lstlisting}[language=caml,caption=Fonction \textsc{OCaml}reconnaissant l'union de deux langages décidables]
let decide (w : string) : bool =
	(decide%*$_1$*) w) || (decide%*$_2$*) w)
			\end{lstlisting}
	\end{enumerate}
\end{prv}
\footnotetext{\textit{i.e.} $\forall w \in \Sigma^*$, $w \in L_1 \iff \texttt{decide}_1(w) = \texttt{true}$}

\begin{rmk}
	$\O$\/ est décidable (par \verb|fun s -> false|) ; $\Sigma^*$\/ est décidable (par \verb|fun s -> true|).
\end{rmk}

\subsection{Sérialisation}

\begin{defn}
	Étant donné un type \textsc{OCaml} \texttt{t}, on appelle \textit{sérialisation calculable} de ce type \texttt{t}, la donnée d'une fonction \texttt{f} \textsc{OCaml} de type $\texttt{t}\to \texttt{string}$\/ qui soit telle que
	\begin{itemize}
		\item pour tout $e : \texttt{t}$, $(\texttt{f}\ e)$\/ est bien parenthésée ;
		\item \texttt{f} est injective ;
		\item la réciproque de \texttt{f} (définie sur $\mathrm{Im}(\mathtt{f})$) est définissable en \textsc{OCaml}.
	\end{itemize}
	\index{sérialisation}
\end{defn}

\begin{exm}
	Le type \texttt{int} est sérialisable par la fonction \verb|string_of_int|.
\end{exm}

\begin{prop}
	Soit $\texttt{t}_\texttt{a}$\/ et $\texttt{t}_\texttt{b}$\/ deux types \textsc{OCaml} sérialisables, alors le type $\texttt{t}_\texttt{a} * \texttt{t}_\texttt{b}$ est sérialisable.
\end{prop}

\begin{prv}
	Soit $\varphi_\texttt{a}$\/ et $\varphi_\texttt{b}$\/ les fonctions de sérialisation des types $\texttt{t}_\texttt{a}$\/ et $\texttt{t}_\texttt{b}$. On définit alors la fonction
		\begin{lstlisting}[language=caml,caption=Fonction \textsc{OCaml} sérialisant le produit cartésien de deux types sérialisables]
let %*$\varphi$*) ((%*$a$*),%*$b$*)): %*$\texttt{t}_\texttt{a}$*) * %*$\texttt{t}_\texttt{b}$*)) =
	"(" ^ (%*$\varphi_\texttt{a}$*) %*$a$*)) ^ "),(" ^ (%*$\varphi_\texttt{b}$*) %*$b$*)) ^ ")"
	\end{lstlisting}
	On remarque que
	\begin{itemize}
		\item $\varphi$\/ est à valeur dans les chaînes de caractères bien parenthésées ;
		\item $\varphi$\/ est injective (par identification de parenthèses et injectivité de $\varphi_\texttt{a}$\/ et $\varphi_\texttt{b}$) ;
		\item la réciproque de $\varphi$\/ est décidable en \textsc{OCaml} (preuve à faire en \textsc{OCaml}).
	\end{itemize}
\end{prv}

\begin{rmk}
	Une programme \textsc{OCaml} est trivialement sérialisable : c'est déjà une chaîne de caractères.
\end{rmk}

\begin{exm}
	La sérialisation de la fonction
	\begin{lstlisting}[language=caml,caption=Fonction factorielle en \textsc{OCaml}]
let rec fact (n : int) : int = 
	if n = 0 then 1
	else n * (fact (n-1))
	\end{lstlisting}
	est la chaîne de caractère 
	\begin{lstlisting}[language=caml]
"let rec fact (n : int) : int = 
	if n = 0 then 1
	else n * (fact (n-1))"%*.*)
	\end{lstlisting}
\end{exm}

\subsection{Machine universelle}

Soit l'ensemble $\mathscr{O}$ des chaînes de caractères qui sont des sérialisations de programme \textsc{OCaml} valide.

\begin{exm}
	La sérialisation de la fonction \texttt{fact} trouvée précédemment est un élément de $\mathscr{O}$. Mais, $\text{``}\texttt{let}\text{''}\not\in \mathscr{O}$.
\end{exm}

\begin{defn}
	Soit la fonction $\texttt{interprète} : \mathscr{O} \times \Sigma^* \to \Sigma^*$ définie par \[
		\texttt{interprète}(\mathcal{M}, w) = \begin{cases}
			w' \text{ tel que } w \xrightarrow[\mathcal{M}]{} w'\\
			\text{non défini sinon}.
		\end{cases}
	\]
	\index{fonction \texttt{interprète}}
\end{defn}

\begin{thm}
	La fonction \texttt{interprète} est calculable. On appelle \textit{machine universelle} un programme \textsc{OCaml} la calculant.
	\index{machine universelle}
\end{thm}

\begin{prv}
	On considère \texttt{utop} ou \texttt{WinCaml}.
\end{prv}

De même, considérons le problème suivant \[
	\begin{cases}
		\text{\textbf{Entrée}} &: M \in \mathscr{O}, w \in \Sigma^*, n \in \N\\
		\text{\textbf{Sortie}} &: M \text{ se termine-t-elle sur $w$\/ en moins de $n$\/ étapes élémentaires ?}
	\end{cases}
\] Ce problème est décidable.

\subsection{Théorème de l'\textsc{Arrêt}}

Dans cette sous-section, on considère le problème \[
	\text{\textsc{Arrêt}} : \begin{cases}
		\text{\textbf{Entrée}}&: M \in \mathscr{O}, w \in \Sigma^*\\
		\text{\textbf{Sortie}} &: M \text{ s'arrête-t-elle sur } w.\text{\footnotemark}
	\end{cases}
\] \footnotetext{\textit{i.e.} $\exists ? w' \in \Sigma^*,\:w \xrightarrow[M]{} w'$}

\begin{thm}
	Le problème \textsc{Arrêt} est indécidable.
\end{thm}

\begin{prv}
	Par l'absurde : supposons \textsc{Arrêt} décidable.
	Soit $\texttt{arret} : \texttt{string} \to \texttt{bool}$\/ prenant en argument une chaîne de caractères qui est la sérialisation d'un couple $M$\/ code d'une machine ($M \subseteq \Sigma^*$), et $w \in \Sigma^*$, et retournant \texttt{true} si et seulement si $M$\/ s'arrête sur l'entrée $w$. Si l'entrée n'est pas une sérialisation convenable, on retourne \texttt{false}. On crée le programme suivant
	\begin{lstlisting}[language=caml,caption=Programme paradoxe prouvant que le problème \textsc{Arrêt} est indécidable]
let paradoxe (w : string) : bool =
	if arret (serialise_couple w w) then
		(while true do () done; true)
	else false
	\end{lstlisting}
	où \texttt{serialise\_couple} est une fonction sérialisant un couple avec l'algorithme trouvé précédemment. Soit $S_\texttt{paradoxe}$\/ la sérialisation de la fonction \texttt{paradoxe}.

	Quid de $(\texttt{paradoxe}\ S_{\texttt{paradoxe}})$ : soit $c = (\texttt{serialise\_couple}\ S_{\texttt{paradoxe}}\ S_{\texttt{paradoxe}})$. La chaîne $c$\/ est donc la sérialisation d'un couple dont la première composante est le code de la fonction \texttt{paradoxe}. De plus, \texttt{arret} se termine sur toute entrée.
	\begin{itemize}
		\item Si $(\texttt{arret}\ c) = \texttt{false}$, alors on va dans la branche \texttt{else} et l'exécution de \texttt{paradoxe} sur $S_{\texttt{paradoxe}}$ termine. Mais, comme $(\texttt{arret}\ c) = \texttt{false}$, \texttt{paradoxe} ne se termine pas sur $S_{\texttt{paradoxe}}$.
		\item Si $(\texttt{arret}\ c) = \texttt{true}$, alors on va dans la branche \texttt{then}, et donc $(\texttt{paradoxe}\ S_\texttt{paradoxe})$\/ ne se termine pas. Mais, comme $(\texttt{arret}\ c) = \texttt{true}$, alors $(\texttt{paradoxe}\ S_\texttt{paradoxe})$\/ se termine.
	\end{itemize}
	On en conclut que le problème \textsc{Arrêt} est indécidable.
\end{prv}

\begin{crlr}
	Il existe des problèmes indécidables.
\end{crlr}

\subsection{Réduction}

\begin{figure}[H]
	\centering
	\begin{asy}
		size(10cm);
		import roundedpath;
		label("$\Sigma^* \owns w$", (-3, 0), align=W);
		draw(roundedpath(box((-2, -1), (3.5, 1)), 0.25));
		fill(roundedpath(box((-1.5,-0.25),(-0.5,0.25)), 0.1), magenta);
		draw(roundedpath(box((0.5,-0.25),(1.5,0.25)), 0.1), deepcyan);
		label("$f$", (-1, 0), white);
		label("\LARGE$Q$", (-2, 1), align=2*SE);
		label("$R$", (0.5, 0.25), deepcyan, align=2*SE);
		draw((-3, 0)--(-1.5, 0), Arrow(TeXHead));
		draw((-0.5, 0)--(0.5, 0), Arrow(TeXHead));
		draw((1.5, 0)--(4.5, 0), Arrow(TeXHead));
		label("$f(w)$", (0, 0), magenta, align=N);
		label("$\quad f(w)\in R^+\iff w\in Q^+$", (1.55, 0), align=NE);
	\end{asy}
	\caption{Structure d'un sous-problème}
\end{figure}

\begin{defn}
	Soit $Q$\/ et $R$\/ deux problèmes de décision. On dit que $Q$\/ \textit{se réduit au problème} $R$ s'il existe $f : \mathcal{E}_Q \to \mathcal{E}_R$\/ totale et calculable, telle que \[
		w \in Q^+ \iff f(w) \in R^+
	.\] On note alors $Q \preceq R$.
	\index{réduction}
\end{defn}

\begin{prop}
	Si $Q \preceq R$, et que $R$\/ est décidable, alors $Q$\/ est décidable.
\end{prop}

\begin{prv}
	Soit $\texttt{decide}_R : \texttt{string} \to \texttt{bool}$\/ décidant $R$ \textit{i.e.} $(\texttt{decide}_R\ w) = \texttt{true} \iff w \in R^+$. Soit~$f : \mathcal{E}_Q \to \mathcal{E}_R$\/ totale calculable, telle que $w \in Q^+ \iff f(w) \in R^+$. On doit coder la fonction suivante.
	\begin{lstlisting}[language=caml,caption=Fonction décidant un sous-problème]
let %*$\texttt{decide}_Q$*) (w: string) : bool =
	(%*$\texttt{decide}_R$*) (%*$f$*) w))
	\end{lstlisting}
	La fonction $\texttt{decide}_Q$\/ décide bien le problème $Q$. En effet,
	\begin{align*}
		(\texttt{decide}_Q\ \texttt{w}) = \texttt{true} \iff& (\texttt{decide}_R\ (f\ \texttt{w}))\\
		\iff& (f\ \texttt{w}) \in R^+\\
		\iff& w \in Q^+.
	\end{align*}
\end{prv}

\begin{crlr}
	Si $Q \preceq R$, et $Q$\/ non décidable, alors $R$\/ non décidable.
\end{crlr}

\begin{exm}
	On considère le problème \[
		\text{\textsc{NonVide}} : \begin{cases}
			\text{\textbf{Entrée}} &: M \in \mathscr{O}\\
			\text{\textbf{Sortie}} &: \mathcal{L}(M) \neq \O\ ?\:.
		\end{cases}
	\] Le problème \textsc{NonVide} est indécidable.
\end{exm}

\begin{exm}
	Les problèmes \[
		\begin{cases}
			\text{\textbf{Entrée}} &: M_1 \text{ et } M_2 \text{ deux machines}\\
			\text{\textbf{Sortie}} &: \mathcal{L}(M_1) \cup \mathcal{L}(M_2) = \Sigma^*
		\end{cases}
	\] et \[
		\begin{cases}
			\text{\textbf{Entrée}} &: M_1 \text{ et } M_2 \text{ deux machines}\\
			\text{\textbf{Sortie}} &: \mathcal{L}(M_1) \cap \mathcal{L}(M_2) = \O
		\end{cases}
	\] sont indécidables.
\end{exm}



