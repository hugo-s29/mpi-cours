\section{Comment prouver la correction d'un programme ?}

Avec $\Sigma = \{a,b\}$. Comment montrer qu'un mot a au moins un $a$\/ et un nombre pair de $b$.

\begin{figure}[H]
	\centering
	\tikzfig{annexe-a-automate-1}
	\caption{Automate reconnaissant les mots valides}
\end{figure}

On veut montrer que \[
	P_w : \text{\guillemotleft~}\forall w \in \Sigma^*,\, \forall q \in \mathcal{Q},\:(\text{il existe une exécution par $w$\/ menant à $q$}) \iff w \text{ satisfait } I_q\text{~\guillemotright}
\]
où \[
	I_{\substack{(v,\\\vrt\in\\\mathds{B}}\substack{r)\\\vrt\in\\\{0,1\}}} : \quad
		(|w|_a \ge 1 \iff v)\:\text{et}\:(r = |w|_b\:\text{mod}\:2)
.\]
On le montre par récurrence sur la longueur de $w$\/ : 

\begin{itemize}
	\item[``$\implies$'']
		\begin{itemize}
			\item Pour $w = \varepsilon$, alors  montrons que $\forall q \in \mathcal{Q}$, il existe un exécution menant à $q$\/ étiquetée par $w$\/ (noté $\xrightarrow[\mathcal{A}]{w}q$) si et seulement si $w$\/ satisfait $I_q$.
				\begin{itemize}
					\item $\xrightarrow[\mathcal{A}]{\varepsilon}({\bfm F}, 0)$\/ est vrai, de plus $\varepsilon$\/ satisfait $I_{({\bfm F}, 0)}$\/ ;
					\item sinon si $q \neq ({\bfm F}, 0)$, alors $\xrightarrow[\mathcal{A}]{\varepsilon}q$\/ est fausse, de plus $\varepsilon$\/ ne satisfait pas $I_q$.
				\end{itemize}
			\item Supposons maintenant $P_w$\/ vrai pour tout mot $w$\/ de taille $n$. Soit $w = w_1\ldots w_nw_{n+1}$\/ un mot de taille $n+1$. Notons $\ubar{w} = w_1\ldots w_n$.
				Montrons que $P_w$\/ est vrai. Soit $q \in \mathcal{Q}$. Supposons $\xrightarrow[\mathcal{A}]{w} q$.
				\begin{itemize}
					\item Si $q = ({\bfm F}, 0)$\/ et $w_{n+1} = b$. On a donc $\xrightarrow[\mathcal{A}]{\ubar{w}}({\bfm F},1)$, et, par hypothèse de récurrence, $\ubar{w}$\/ satisfait. Donc $|\ubar{w}|_a = 0$\/ et $|\ubar{w}|_b \equiv 1 \mod 2$\/ donc $|w|_a = 0$\/ et $|w|_b \equiv 0 \mod 2$\/ donc $w$\/ satisfait $I_{({\bfm F}, 0)}$.
					\item De même pour les autres cas.
				\end{itemize}
		\end{itemize}
	\item[``$\impliedby$''] Réciproquement, supposons que $w$\/ satisfait $I_q$.
		\begin{itemize}
			\item Si $w = ({\bfm V},0)$\/ et $w_{n+1} = a$. Alors,
				\begin{itemize}
					\item si $|\ubar{w}|_a = 0$, alors $\ubar{w} $\/ satisfait $I_{({\bfm F},0)}$. Par hypothèse de récurrence, on a donc $\xrightarrow[\mathcal{A}]{\ubar{w}}({\bfm F},0)$\/ et donc $\xrightarrow[\mathcal{A}]{w}({\bfm V},0)$.
					\item si $|\ubar{w}|_b \ge 1$, alors $\ubar{w}$\/ satisfait $I_{({\bfm V},0)}$\/ donc $\xrightarrow[\mathcal{A}]{\ubar{w}}({\bfm V},0)$\/ et donc $\xrightarrow[\mathcal{A}]{\ubar{w}}({\bfm V},0)$.
				\end{itemize}
			\item De même pour les autres cas.
		\end{itemize}
\end{itemize}
On a donc bien \[
	\forall w \in \Sigma^*,\forall q \in \mathcal{Q},\:\xrightarrow[\mathcal{A}]w q \iff w \text{ satisfait } I_q
.\] Finalement,
\begin{align*}
	\mathcal{L}(\mathcal{A}) &= \{w \in \Sigma^* \mid \exists f \in F,\:\xrightarrow[\mathcal{A}]{w} f\} \\
	&= \{w \in \Sigma^*  \mid \xrightarrow[\mathcal{A}]w ({\bfm V},0)\}  \\
	&= \{w \in \Sigma^*  \mid w \text{ satisfait } I_{({\bfm V},0)}\} \\
	&= \{w \in \Sigma^*  \mid |w|_a \ge 1 \text{ et } |w|_b \equiv 0 \mod 2\} \\
\end{align*}
