\section{\scshape Hors-programme}

\begin{defn}
	On appelle monoïde un ensemble $M$\/ muni d'une loi ``$\cdot$'' interne associative admettant un élément neutre $1_M$.
\end{defn}

\begin{defn}
	Étant donné deux monoïdes $M$\/ et $N$, on appelle morphisme de monoïdes une fonction $\mu : M \to N$\/ telle que
	\begin{enumerate}
		\item $\mu(1_M) = 1_N$\/ ;
		\item $\mu(x \cdot_M y) = \mu(x) \cdot_N \mu(y)$.
	\end{enumerate}
\end{defn}

\begin{exm}
	$|\:\cdot\:|\:: (\Sigma^*, \cdot) \to (\N, +)$\/ est un morphisme de monoïdes.
\end{exm}

\begin{defn}
	Un langage $L$\/ est dit reconnu par un monoïde $M$, un morphisme $\mu : \Sigma^* \to M$\/ et un ensemble $P \subseteq M$\/ si $L = \mu^{-1}(P)$.
\end{defn}

\begin{exm}
	L'ensemble $\{a^{n^3} \mid n \in \N\}$\/ est reconnu par le morphisme $|\:\cdot\:|$\/ et l'ensemble $P = \{n^3  \mid n \in \N\}$.
\end{exm}

\begin{thm}
	Un langage est régulier si et seulement s'il est reconnu par un monoïde fini.
\end{thm}

\begin{exm}
	L'ensemble $\{a^{2n}  \mid n \in \N\}$\/ est un langage régulier. En effet, on a $M = \ZdZ$, $P = \{0\}$\/ et \begin{align*}
		\mu: \Sigma^* &\longrightarrow \ZdZ \\
		w &\longmapsto |w|\ \mathrm{mod}\ 2.
	\end{align*}

	\begin{figure}[H]
		\centering
		\tikzfig{automate-monoide-1}
		\caption{Automate reconnaissant $\mu^{-1}(P) = L$}
	\end{figure}
\end{exm}

\begin{prv}
	\begin{itemize}
		\item[``$\implies$''] Soit $L \in \wp(\Sigma^*)$\/ reconnu par un monoïde $M$ fini, un morphisme $\mu$\/ et un ensemble $P$ : $L = \mu^{-1}(P)$. Posons $\mathcal{A} = (\Sigma', \mathcal{Q}, I, F, \delta)$\/ avec

			\vspace{-5mm}
			\begin{multicols}{4}
				\[\Sigma' = \Sigma\]
				\[\mathcal{Q} = M\]
				\[I = \{1_M\}\]
				\[F = P\]
			\end{multicols}
			\vspace{-7mm}\[
				\delta = \{(q,\ell, q') \in \mathcal{Q} \times \Sigma \times \mathcal{Q}  \mid q \cdot \mu(\ell) = q'\}.
			\]
			Montrons que $\mathcal{L}(\mathcal{A}) = L$.
			Soit $w \in \mathcal{L}(\mathcal{A})$. Il existe une exécution acceptante \[
				1_M = q_0 \xrightarrow{w_1} q_1 \to \cdots \xrightarrow{w_n} q_n \in P
			.\] Or, $\mu(w_1\ldots w_n) = \prod_{i=1}^n \mu(w_i) = q_0 \prod_{i=1}^n \mu(w_i) = q_0 \mu(w_1) \cdot \prod_{i=1}^n \mu(w_i) = q_1 \prod_{i=1}^n \mu(w_i) = q_n \in P$.
	\end{itemize}
\end{prv}

