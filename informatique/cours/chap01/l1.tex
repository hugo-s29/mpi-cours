\section{Motivation}

\subsection{1\tsup{ère} motivation}

Il y a une grande différence entre les mathématiques et l'informatique : la gestion de l'infini.
On n'a pas de mémoire infinie sur un ordinateur.
Par exemple, pour représenter $\pi$\/ ou $\sqrt{2}$, on ne peut pas stocker un nombre infini de décimales. Ce n'est pas une question de base, ces nombres ont aussi des décimales infinies dans une base 2.

Par exemple, si on ne veut utiliser $\sqrt{2}$\/ seulement pour plus tard le mettre au carré. On peut définir une structure en {\tt C}\/ comme celle qui suit
\begin{lstlisting}[language=c, caption=$\sqrt{2}$\/ sous forme de structure]
typedef struct {
	int carre;
	bool sign;
};
\end{lstlisting}
On a pu décrire $\sqrt{2}$\/ comme cela car il y a une certaine régularité dans ce nombre.

On définit des relations entre ces objets. Dans ce chapitre, on va commencer par étudier autre chose : les mots.
Les mots sont utilisés, tout d'abord, pour entrer une liste de lettres mais aussi le programme lui-même. En effet, il y a une liste infinie de code possibles en {\tt C}.

\subsection{2\tsup{nde} motivation}

Les ordinateurs sont complexes ; il peut être dans une multitude d'états.
On représente une succession de tâches (le symbole {\raisebox{-2pt}{\tikz\node [style=new style 0] at (0, 0) {};}} représente une tâche) : 
\begin{figure}[H]
	\centering
	\scalebox{1.5}{\tikzfig{computer-states}}
	\caption{États d'un ordinateur}
\end{figure}
\noindent Par exemple, pour une boucle infinie, on a un cycle dans le graphe ci-dessus. Ou, s'il atteint un certain nœud, on a un {\it bug}.

Mais, pour résoudre ce problème, on peut forcer le nombre d'état d'un ordinateur ; par exemple, dire qu'un ordinateur a 17 états.

On décide donc de représenter mathématiquement un ordinateur afin de pouvoir faire des preuves avec. Et, c'est l'objet de ce chapitre.

\section{Mots et langages, rappels}

\begin{defn}
	On appelle {\it alphabet}\/ un ensemble fini, d'éléments qu'on appelle {\it lettres}.
\end{defn}

\begin{defn}
	On appelle une {\it mot}\/ sur $\Sigma$ (où $\Sigma$ est un alphabet) une suite finie de lettres de $\Sigma$.

	La {\it longueur}\/ d'un mot est le nombre de lettres, comptées avec leurs multiplicité. On la note $|w|$\/ pour un mot $w$.
	Si $|w| = n \in \N^*$, on indexe les lettres de $w$\/ pour $(w_i)_{i\in\left\llbracket 1,n \right\rrbracket}$ et on écrit alors \[
		w = w_1 w_2 w_3 \ldots w_n
	.\]

	Il existe un unique mot de longueur $0$\/ appelé {\it mot vide}, on le note $\varepsilon$.
\end{defn}

\begin{defn}
	Si $\Sigma$ est un alphabet, on note
	\begin{itemize}
		\item $\Sigma^n$\/ les mots de longueur $n$\/ ;
		\item $\Sigma^*$\/ les mots de longueurs positives ou nulle ;
		\item $\Sigma^+$\/ les mots de longueurs strictement positives.
	\end{itemize}
\end{defn}

\begin{rmk}
	\[
		\Sigma^* = \bigcup_{n \in \N} \Sigma^n \quad\quad\quad \Sigma^+ = \bigcup_{n \in \N^*} \Sigma^n
	.\]
\end{rmk}

\begin{defn}
	Soit $\Sigma$ un alphabet. Soit $x$\/ et $y$\/ deux mots de $\Sigma^*$. Notons
	\begin{gather*}
		x = x_1x_2x_3\ldots x_n\\
		y = y_1y_2y_3\ldots y_n.\\
	\end{gather*}
	On définit alors {\it concaténation}\/ notée $x \cdot y$\/ l'opération définie comme \[
		x \cdot y = x_1x_2x_3\ldots x_n y_1y_2\ldots y_n
	.\]
\end{defn}

\begin{rmk}
	\begin{itemize}
		\item $\cdot $\/ est une opération interne ;
		\item $\varepsilon$\/ est neutre pour $\cdot$\/ : \hfill $\forall n \in \Sigma^*,\,\varepsilon \cdot x = x \cdot \varepsilon = x$. \hfill \hbox{}
		\item $\cdot$\/ est associatif.
	\end{itemize}

	(On dit que c'est un monoïde.)
\end{rmk}

\begin{rmk}
	L'opération $\cdot $\/ n'est pas commutative.
\end{rmk}

\begin{defn}
	Soit $x$\/ et $y$\/ deux mots sur l'alphabet $\Sigma$. On dit que 
	\begin{itemize}
		\item $x$\/ est un {\it préfixe}\/ de $y$\/ si \hfill $\exists v \in \Sigma^*,\,y = x \cdot v$\/ ;\hfill\hbox{}
		\item $x$\/ est un {\it suffixe}\/ de $y$\/ si \hfill $\exists v \in \Sigma^*,\,y = v\cdot x$\/ ; \hfill\hbox{}
		\item $x$\/ est un {\it facteur}\/ de $y$\/ si \hfill $\exists (v, w) \in \big(\Sigma^*\big)^2,\,y = v\cdot x\cdot w$. \hfill\hbox{}
	\end{itemize}
\end{defn}

\begin{exm}
	\begin{itemize}
		\item $a$\/ est facteur de $aa$\/ ;
		\item $\varepsilon$\/ est un facteur de $a$.
	\end{itemize}
\end{exm}

\begin{defn}
	On dit que $x$\/ est un {\it sous-mot}\/ de $y$\/ si $x$\/ est une suite extraite de $y$. Par exemple \[
		a\:\cancel{b}\:a\:a\:\cancel{b}\:a \qquad \longrightarrow \qquad a\:a\:a\:a
	.\]
\end{defn}

\begin{defn}
	Un {\it langage}\/ est un ensemble de mots. C'est donc un élément de $\wp(\Sigma^*)$.
\end{defn}

\begin{rmk}[\danger\!\!]
	$\O \neq \{\varepsilon\}$.
\end{rmk}

\section{Langage régulier}

\subsection{Opérations sur les langages}

\begin{rmk}
	Les langages sont des ensembles. On peut donc leurs appliquer des opérations ensemblistes.
\end{rmk}

\begin{defn}
	Soient $L_1, L_2 \in \wp(\Sigma^*)$ deux langages. On définit la {\it concaténation}\/ de deux langages, notée $L_1 \cdot L_2$\/ : \[
		L_1 \cdot L_2 = \{u \cdot v  \mid u \in L_1,\,v \in L_2\}
	.\]
\end{defn}

\begin{rmk}
	\begin{itemize}
		\item L'opération $\cdot$\/ (langages) a $\{\varepsilon\}$\/ pour neutre.
		\item L'opération $\cdot$\/ (langages) a $\O$\/ pour élément absorbant : \[
					\forall L \in \wp(\Sigma^*),\:L\cdot \O = \O\cdot L = \O
			.\]
		\item L'opération $\cdot $\/ est distributive : soient $K, L, M$\/ trois langages ; on a 
			\begin{gather*}
				K \cdot (L \cup M) = (K \cdot  L) \cup (K \cdot M) ;\\
				(L \cup M) \cdot K = (L\cdot K) \cup (M\cdot K).
			\end{gather*}
	\end{itemize}
\end{rmk}

\begin{defn}
	Étant donné un langage $L$, on définit par récurrence : 
	\begin{itemize}
		\item $L^0 = \{\varepsilon\}$\/ ;
		\item $L^{n+1} = L^n \cdot L = L \cdot L^n$.\protect\footnotemark
	\end{itemize}
	On note alors \[
		L^{\red *} = \bigcup_{n \in \N} L^n \qquad \text{ et } \qquad L^+ = \bigcup_{n \in \N} L^n
	.\]
\end{defn}
\footnotetext{La deuxième égalité est assurée par l'associativité de l'opération $\cdot $.}

\begin{rmk}
	On a $\Sigma^* = \Sigma^{\red *}$. On notera donc $\Sigma^*$\/ dans tous les cas.
\end{rmk}

\begin{rmk}
	Si $\varepsilon \in L$, alors $L^* = L^+$. En effet, $L^* = L^+ \cup \{\varepsilon\}$.
\end{rmk}

\begin{rmk}
	On nomme l'application $L \mapsto L^*$\/ l'{\it étoile de \sc Kleene}.
\end{rmk}

\begin{rmk}
	Avec $L$\/ et $K$\/ deux alphabets, on a \[|L\cdot K| \le |L|\:|K|.\]
	En effet, avec $K = \{a,aa\}$, on a $K^2 = \{aa, aaa, aaaa\}$.
\end{rmk}

\begin{exm}
	Avec $L = \{a, ab\}$, on a
	\begin{itemize}
		\item $L^1 = L$\/ ;
		\item $L^2 = \{aa,\,aab,\,aba\,abab\}$\/ ;
		\item $L^* \supseteq \{\varepsilon,\,a,\,ab,\,aa,\ldots,aaaaaaa\ldots\}$.
	\end{itemize}
\end{exm}

\begin{defn}
	Soit $\Sigma$\/ un alphabet.
	On appelle {\it ensemble des langages réguliers}, noté $\mathrm{LR}$.
	Le plus petit ensemble tel que 
	\begin{itemize}
		\item $\O \subset \mathrm{LR}$\/ ;
		\item Si $a \in \Sigma$, alors $a \in \mathrm{LR}$\/ ;
		\item Si $L_1 \in \mathrm{LR}$\/ et $L_2 \in \mathrm{LR}$, on a $L_1 \cup L_2 \in \mathrm{LR}$\/ ;
		\item Si $L_1 \in \mathrm{LR}$\/ et $L_2 \in \mathrm{LR}$, on a $L_1 \cdot L_2 \in \mathrm{LR}$\/ ;
		\item Si $L \in \mathrm{LR}$, alors $L^* \in \mathrm{LR}$.
	\end{itemize}
\end{defn}

\begin{exm}
	\begin{itemize}
		\item Soit $\Sigma = \{t,o\}$.
			Est-ce que $\{toto\} \in \mathrm{LR}$\/ ?
			On sait déjà que $\{t\}$\/ et $\{o\}$\/ sont déjà des langages réguliers.
			Or, \[
				\{toto\} = \{t\} \cdot \{o\} \cdot \{t\} \cdot \{o\} 
			.\] Et donc $\{toto\} \in \mathrm{LR}$.
		\item On a $\{\varepsilon\} = \O^* \in \mathrm{LR}$.
		\item On a \[
				\begin{array}{rccccl}
					\{&x_{11}&x_{12}&\ldots&x_{1,n_1}\\
					&x_{21}&x_{22}&\ldots&x_{2,n_2}\\
					&\vdots&\vdots&\ddots&\vdots\\
					&x_{m,1}&x_{m,2}&\ldots&x_{m,n_m}&\} \in \mathrm{LR}
				\end{array}
			\] car $\{x_{i,1}, x_{i,2}, \ldots, x_{i,n_i}\} \in \mathrm{LR}$\/ et c'est stable par union finie.
		\item Avec $\Sigma = \{a,b\}$, on a
			\[
				L = \{\varepsilon,\,a,\,aa,\,aaa,\ldots\} = \underbrace{\Big(\overbrace{\{a\}}^{\in \mathrm{LR}}\Big)^*}_{\in \mathrm{LR}}
			.\]
		\item $\Sigma^* \in \mathrm{LR}$.
		\item Contre-exemple : $\{a^n \mid a \text{ premier}\:\} \not\in \mathrm{LR}$.
	\end{itemize}
\end{exm}

\subsection{Expressions régulières}

On représente $\O$\/ l'ensemble vide, $\_|\_$\/ l'union de deux expressions régulières, $\_\cdot\_$\/ la concaténation de deux expression régulières, et, $\_^*$\/ l'étoile de {\sc Klein}\/ pour les expressions régulières.

On cherche à représenter informatiquement ces expressions à l'aide d'une règle de construction nommée.

\begin{defn}
	Étant donné un alphabet $\Sigma$, on définit $\Reg(\Sigma)$\/ défini par induction nommée à partir des règles :
	\begin{multicols}{2}
		\begin{itemize}
			\item $\red \O\Big|^0$\/ ;
			\item $\red |~\Big|^2$\/ ;
			\item $\red \cdot~\Big|^2$\/ ;
			\item $\red *\:\Big|^1$\/ ;
			\item $\red L\Big|^1_\Sigma$\/ ;
			\item $\red \varepsilon\Big|^0$.
		\end{itemize}
	\end{multicols}
\end{defn}

Ces règles peuvent être définies en OCamL (douteux) de la façon suivante :
\begin{lstlisting}[language=caml, caption=Règles des expressions régulières en OCamL]
type regex =
	|%*$~\red\O$*)
	|%*$~\red\mid~$*) of regex * regex
	|%*$~\red\cdot~$*) of regex * regex
	|%*$~\red\ast~$*) of regex
	|%*$~\red L$*) of char
	|%*$~\red\varepsilon$*)
\end{lstlisting}
\noindent On peut donc écrire \[
	\big((\O^*) \mid (a\cdot b) \big) \quad \longrightarrow\quad \red |(\red *(\red \O()), \red \cdot(\red L(a),\red L(b)))
.\]

A-t-on $\red |(\red L(a), \red L(b)) \mathrel{\mathop=^?} \red |(\red L(b), \red L(a))$\/ ? Non, sinon on risque d'avoir une boucle infinie au moment de l'évaluation de cette expression.

On peut simplifier la notation : au lieu d'écrire $\red |(\red *(\red \O()), \red \cdot(\red L(a),\red L(b)))$, on note $\O^* \mid (a\cdot b)$.

\begin{defn}
	On définit \begin{align*}
		\mathcal{L}: \Reg(\Sigma) &\longrightarrow \wp(\Sigma^*) \\
		\O &\longmapsto \O\\
		a &\longmapsto \{a\}\\
		\varepsilon &\longmapsto \{\varepsilon\}\\
		e_1\cdot e_2 &\longmapsto \mathcal{L}(e_1) \cdot  \mathcal{L}(e_2)\\
		e_1 \mid e_2 &\longmapsto \mathcal{L}(e_1) \cup \mathcal{L}(e_2)\\
		e^* &\longmapsto \mathcal{L}(e)^*\\
	\end{align*}
\end{defn}

