\documentclass[a4paper]{article}

\usepackage[margin=1in]{geometry}
\usepackage[utf8]{inputenc}
\usepackage[T1]{fontenc}
\usepackage{mathrsfs}
\usepackage{textcomp}
\usepackage[french]{babel}
\usepackage{amsmath}
\usepackage{amssymb}
\usepackage{cancel}
\usepackage{frcursive}
\usepackage[inline]{asymptote}
\usepackage{tikz}
\usepackage[european,straightvoltages,europeanresistors]{circuitikz}
\usepackage{tikz-cd}
\usepackage{tkz-tab}
\usepackage[b]{esvect}
\usepackage[framemethod=TikZ]{mdframed}
\usepackage{centernot}
\usepackage{diagbox}
\usepackage{dsfont}
\usepackage{fancyhdr}
\usepackage{float}
\usepackage{graphicx}
\usepackage{listings}
\usepackage{multicol}
\usepackage{nicematrix}
\usepackage{pdflscape}
\usepackage{stmaryrd}
\usepackage{xfrac}
\usepackage{hep-math-font}
\usepackage{amsthm}
\usepackage{thmtools}
\usepackage{indentfirst}
\usepackage[framemethod=TikZ]{mdframed}
\usepackage{accents}
\usepackage{soulutf8}
\usepackage{mathtools}
\usepackage{bodegraph}
\usepackage{slashbox}
\usepackage{enumitem}
\usepackage{calligra}
\usepackage{cinzel}
\usepackage{BOONDOX-calo}

% Tikz
\usetikzlibrary{babel}
\usetikzlibrary{positioning}
\usetikzlibrary{calc}

% global settings
\frenchspacing
\reversemarginpar
\setuldepth{a}

%\everymath{\displaystyle}

\frenchbsetup{StandardLists=true}

\def\asydir{asy}

%\sisetup{exponent-product=\cdot,output-decimal-marker={,},separate-uncertainty,range-phrase=\;à\;,locale=FR}

\setlength{\parskip}{1em}

\theoremstyle{definition}

% Changing math
\let\emptyset\varnothing
\let\ge\geqslant
\let\le\leqslant
\let\preceq\preccurlyeq
\let\succeq\succcurlyeq
\let\ds\displaystyle
\let\ts\textstyle

\newcommand{\C}{\mathds{C}}
\newcommand{\R}{\mathds{R}}
\newcommand{\Z}{\mathds{Z}}
\newcommand{\N}{\mathds{N}}
\newcommand{\Q}{\mathds{Q}}

\renewcommand{\O}{\emptyset}

\newcommand\ubar[1]{\underaccent{\bar}{#1}}

\renewcommand\Re{\expandafter\mathfrak{Re}}
\renewcommand\Im{\expandafter\mathfrak{Im}}

\let\slantedpartial\partial
\DeclareRobustCommand{\partial}{\text{\rotatebox[origin=t]{20}{\scalebox{0.95}[1]{$\slantedpartial$}}}\hspace{-1pt}}

% merging two maths characters w/ \charfusion
\makeatletter
\def\moverlay{\mathpalette\mov@rlay}
\def\mov@rlay#1#2{\leavevmode\vtop{%
   \baselineskip\z@skip \lineskiplimit-\maxdimen
   \ialign{\hfil$\m@th#1##$\hfil\cr#2\crcr}}}
\newcommand{\charfusion}[3][\mathord]{
    #1{\ifx#1\mathop\vphantom{#2}\fi
        \mathpalette\mov@rlay{#2\cr#3}
      }
    \ifx#1\mathop\expandafter\displaylimits\fi}
\makeatother

% custom math commands
\newcommand{\T}{{\!\!\,\top}}
\newcommand{\avrt}[1]{\rotatebox{-90}{$#1$}}
\newcommand{\bigcupdot}{\charfusion[\mathop]{\bigcup}{\cdot}}
\newcommand{\cupdot}{\charfusion[\mathbin]{\cup}{\cdot}}
%\newcommand{\danger}{{\large\fontencoding{U}\fontfamily{futs}\selectfont\char 66\relax}\;}
\newcommand{\tendsto}[1]{\xrightarrow[#1]{}}
\newcommand{\vrt}[1]{\rotatebox{90}{$#1$}}
\newcommand{\tsup}[1]{\textsuperscript{\underline{#1}}}
\newcommand{\tsub}[1]{\textsubscript{#1}}

\renewcommand{\mod}[1]{~\left[ #1 \right]}
\renewcommand{\t}{{}^t\!}
\newcommand{\s}{\text{\calligra s}}

% custom units / constants
%\DeclareSIUnit{\litre}{\ell}
\let\hbar\hslash

% header / footer
\pagestyle{fancy}
\fancyhead{} \fancyfoot{}
\fancyfoot[C]{\thepage}

% fonts
\let\sc\scshape
\let\bf\bfseries
\let\it\itshape
\let\sl\slshape

% custom math operators
\let\th\relax
\let\det\relax
\DeclareMathOperator*{\codim}{codim}
\DeclareMathOperator*{\dom}{dom}
\DeclareMathOperator*{\gO}{O}
\DeclareMathOperator*{\po}{\text{\cursive o}}
\DeclareMathOperator*{\sgn}{sgn}
\DeclareMathOperator*{\simi}{\sim}
\DeclareMathOperator{\Arccos}{Arccos}
\DeclareMathOperator{\Arcsin}{Arcsin}
\DeclareMathOperator{\Arctan}{Arctan}
\DeclareMathOperator{\Argsh}{Argsh}
\DeclareMathOperator{\Arg}{Arg}
\DeclareMathOperator{\Aut}{Aut}
\DeclareMathOperator{\Card}{Card}
\DeclareMathOperator{\Cl}{\mathcal{C}\!\ell}
\DeclareMathOperator{\Cov}{Cov}
\DeclareMathOperator{\Ker}{Ker}
\DeclareMathOperator{\Mat}{Mat}
\DeclareMathOperator{\PGCD}{PGCD}
\DeclareMathOperator{\PPCM}{PPCM}
\DeclareMathOperator{\Supp}{Supp}
\DeclareMathOperator{\Vect}{Vect}
\DeclareMathOperator{\argmax}{argmax}
\DeclareMathOperator{\argmin}{argmin}
\DeclareMathOperator{\ch}{ch}
\DeclareMathOperator{\com}{com}
\DeclareMathOperator{\cotan}{cotan}
\DeclareMathOperator{\det}{det}
\DeclareMathOperator{\id}{id}
\DeclareMathOperator{\rg}{rg}
\DeclareMathOperator{\rk}{rk}
\DeclareMathOperator{\sh}{sh}
\DeclareMathOperator{\th}{th}
\DeclareMathOperator{\tr}{tr}

% colors and page style
\definecolor{truewhite}{HTML}{ffffff}
\definecolor{white}{HTML}{faf4ed}
\definecolor{trueblack}{HTML}{000000}
\definecolor{black}{HTML}{575279}
\definecolor{mauve}{HTML}{907aa9}
\definecolor{blue}{HTML}{286983}
\definecolor{red}{HTML}{d7827e}
\definecolor{yellow}{HTML}{ea9d34}
\definecolor{gray}{HTML}{9893a5}
\definecolor{grey}{HTML}{9893a5}
\definecolor{green}{HTML}{a0d971}

\pagecolor{white}
\color{black}

\begin{asydef}
	settings.prc = false;
	settings.render=0;

	white = rgb("faf4ed");
	black = rgb("575279");
	blue = rgb("286983");
	red = rgb("d7827e");
	yellow = rgb("f6c177");
	orange = rgb("ea9d34");
	gray = rgb("9893a5");
	grey = rgb("9893a5");
	deepcyan = rgb("56949f");
	pink = rgb("b4637a");
	magenta = rgb("eb6f92");
	green = rgb("a0d971");
	purple = rgb("907aa9");

	defaultpen(black + fontsize(8pt));

	import three;
	currentlight = nolight;
\end{asydef}

% theorems, proofs, ...

\mdfsetup{skipabove=1em,skipbelow=1em, innertopmargin=6pt, innerbottommargin=6pt,}

\declaretheoremstyle[
	headfont=\normalfont\itshape,
	numbered=no,
	postheadspace=\newline,
	headpunct={:},
	qed=\qedsymbol]{demstyle}

\declaretheorem[style=demstyle, name=Démonstration]{dem}

\newcommand\veczero{\kern-1.2pt\vec{\kern1.2pt 0}} % \vec{0} looks weird since the `0' isn't italicized

\makeatletter
\renewcommand{\title}[2]{
	\AtBeginDocument{
		\begin{titlepage}
			\begin{center}
				\vspace{10cm}
				{\Large \sc Chapitre #1}\\
				\vspace{1cm}
				{\Huge \calligra #2}\\
				\vfill
				Hugo {\sc Salou} MPI${}^{\star}$\\
				{\small Dernière mise à jour le \@date }
			\end{center}
		\end{titlepage}
	}
}

\newcommand{\titletp}[4]{
	\AtBeginDocument{
		\begin{titlepage}
			\begin{center}
				\vspace{10cm}
				{\Large \sc tp #1}\\
				\vspace{1cm}
				{\Huge \textsc{\textit{#2}}}\\
				\vfill
				{#3}\textit{MPI}${}^{\star}$\\
			\end{center}
		\end{titlepage}
	}
	\fancyfoot{}\fancyhead{}
	\fancyfoot[R]{#4 \textit{MPI}${}^{\star}$}
	\fancyhead[C]{{\sc tp #1} : #2}
	\fancyhead[R]{\thepage}
}

\newcommand{\titletd}[2]{
	\AtBeginDocument{
		\begin{titlepage}
			\begin{center}
				\vspace{10cm}
				{\Large \sc td #1}\\
				\vspace{1cm}
				{\Huge \calligra #2}\\
				\vfill
				Hugo {\sc Salou} MPI${}^{\star}$\\
				{\small Dernière mise à jour le \@date }
			\end{center}
		\end{titlepage}
	}
}
\makeatother

\newcommand{\sign}{
	\null
	\vfill
	\begin{center}
		{
			\fontfamily{ccr}\selectfont
			\textit{\textbf{\.{\"i}}}
		}
	\end{center}
	\vfill
	\null
}

\renewcommand{\thefootnote}{\emph{\alph{footnote}}}

% figure support
\usepackage{import}
\usepackage{xifthen}
\pdfminorversion=7
\usepackage{pdfpages}
\usepackage{transparent}
\newcommand{\incfig}[1]{%
	\def\svgwidth{\columnwidth}
	\import{./figures/}{#1.pdf_tex}
}

\pdfsuppresswarningpagegroup=1
\ctikzset{tripoles/european not symbol=circle}

\newcommand{\missingpart}{{\large\color{red} Il manque quelque chose ici\ldots}}


\titleanx{F}{Arithmétique}

\begin{document} %% EXACT
	Un des premiers algorithmes codé est l'algorithme d'Euclide pour calculer le \textsc{pgcd}. Pour $a \neq 0$, on a $a \wedge 0 = a$ et $a \wedge b = b \wedge (a\ \mathrm{mod}\ b)$.
	On peut le coder en \textsc{OCaml} avec la fonction \texttt{euclid} suivante.

	\begin{lstlisting}[language=caml,caption=Algorithme d'Euclide calculant le \textsc{pgcd}]
let rec euclid (a: int) (b: int): int =
	(* Hyp: a >= b et a != 0 *)
	if b = 0 then a
	else euclide b (a mod b)
	\end{lstlisting}
	
	Quelle est la complexité de cet algorithme ?
	On représente le nombre d'appels récursifs à \texttt{euclid}, et on devine une courbe logarithmique.
	En notant $(u_n)$ les divisions euclidiennes réalisées et $(q_n)$ les quotients, ainsi, on $u_n = q_{n-1} \cdot u_{n-1} + u_{n-2}$.
	Alors, $\texttt{euclid}(u_n, u_{n-1}) = \cdots = \texttt{euclid}(u_3, u_2) = \texttt{euclid}(u_2, u_1) = \texttt{euclid}(u_1, u_0)$.

	En fixant la complexité, on cherche les valeurs de $(u_n)$ maximisant les appels récursifs.
	On peut montrer par récurrence que si $\texttt{euclid}(a,b)$ conduit à $n$ appels récursifs de \texttt{euclid}, alors $a \ge F_n$ et $b \ge F_{n-1}$, où $(F_n)_{n\in\N}$\/ est la suite de Fibonacci.

	En effet, soit un tel couple $(a,b)$. Alors, $(b, a\ \mathrm{mod}\ b)$ conduit à $n - 1$ appels récursifs donc~$b \ge F_{n-1}$ et~$a\ \mathrm{mod}\ b \ge F_{n-2}$ par hypothèse de recurrence.
	Et, $a = bq + (a\ \mathrm{mod}\ b)$ et donc $a \ge F_{n-1} + F_{n-2} = F_n$.

	De plus, pour tout $n \in \N \setminus \{0,1\}$, $F_n \ge \varphi^{n-2}$ où $\varphi$ est le nombre d'or.\footnote{C'est la solution positive de $X^2 - X - 1 = 0$.}
	En effet, $F_2 = 1 \ge \varphi^0 = 1$ et $F_3 = 2 \ge \varphi^1 = \varphi = (1 + \sqrt{5}) / 2$. Et, $F_n = F_{n-1} + F_{n-2} \ge \varphi^{n-3} + \varphi^{n-4} \ge \varphi^{n-4}(1 + \varphi) \ge \varphi^{n-2}$.

	Soient $(p,q)$, où $p \ge q$, une entrée de l'algorithme d'Euclide. Si l'appel $\texttt{euclid}(p,q)$ conduit à plus de $\left\lceil \log_\varphi p \right\rceil + 4$ appels, alors $p \ge F_{\left\lceil \log_\varphi p \right\rceil + 4} \ge \varphi^{\left\lceil \log_\varphi p \right\rceil + 4 - 2} > \varphi^{\log_\varphi p} = p$, ce qui est absurde.

	Ceci conduit à une complexité en $\mathcal{O}(\log p)$.

	\bigskip

	Soit $n$ un entier premier.
	Pour l'algorithme RSA, on cherche un inverse de $a \in \sfrac{\Z}{n\Z}$ : on cherche $b \in \sfrac\Z{n\Z}$ tel que $ab \equiv 1 \mod 1$. D'après le théorème de Bézout, on a $au + nv = 1$ car $a \wedge n = 1$. L'inverse est $v$. D'où l'importance des coefficients de Bézout.

	Comment calculer les coefficients de Bézout ?
	On peut utiliser l'algorithme d'Euclide.
	On pose $r_n$\/ la valeur de \texttt{a} après $n$ appels récursifs.

	\begin{table}[H]
		\centering
		\begin{tabular}{c|c|c|c|c}
			$r_i$ & $u_i$ & & $v_i$ &\\ \hline \hline
			$r_0 = a$ & $1$ & $a$ & $0$ & $b$\\
			$r_1 = b$ & $0$ & $b$ & $1$ & $a$\\
			$\vdots$ & $\vdots$ & $\vdots$ & $\vdots$ & $\vdots$ \\
			$r_{i-2}$ & $u_{i-2}$ & $a$ & $v_{i-2}$ & $b$\\
			$r_{i-1}$ & $u_{i-1}$ & $a$ & $v_{i-1}$ & $b$ \\
		\end{tabular}
		\caption{Valeurs de $r_i$ avec invariant $r_i = a u_i + b v_i$}
	\end{table}

	Alors,
	\begin{align*}
		r_i &= u_{i-2} a + v_{i-2} b - (r_{i-2} / r_{i-1}) (u_{i-1}a + v_{i-1} b)\\
		&= \big(u_{i-2} - (r_{i-2}/r_{i-1}) u_{i-1}\big) a + \big(v_{i-2} - (r_{i-2}/r_{i-1}) v_{i-1}\big) b \\
	\end{align*}
	Ainsi, on a bien $\mathrm{pgcd}(a,b) = u_{n-1} a + v_{n-1} b$.
\end{document}
