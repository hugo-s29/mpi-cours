Ce chapitre se rattache à la branche de l'informatique des langages formels. On l'a étudié au chapitre 1 avec les automates, et au chapitre 4 avec les machines.
Pour le moment, nous avons 5 classes de langages : (1) les langages finis, (2) les langages locaux, (3) les langages réguliers, (4) les langages décidables en temps polynomial, (5) les langages décidables.
L'objectif de ce chapitre se situe entre les points (3) et (4).

Intéressons-nous à un langage particulier, le langage des programmes \textsc{OCaml} avec une syntaxe valide.
Ce langage est il régulier ? Non. On peut considérer l'expression\[
	e_n = \:\: \text{\guillemotleft~}\underbrace{((\cdots (}_n 0 \underbrace{)\cdots ))}_n\text{~\guillemotright}
\]qui est une expression \textsc{OCaml} valide. Par application du lemme de l'étoile, il n'est pas reconnaissable par un automate à $N$ états en considérant $e_{N+1}$.

L'ensemble $\mathcal{B}$ des mots bien parenthésés est le plus petit ensemble tel que
\begin{itemize}
	\item $\varepsilon \in \mathcal{B}$
	\item si $u \in \mathcal{B}$ et $v \in \mathcal{B}$, alors $u \cdot v \in \mathcal{B}$
	\item si $u \in \mathcal{B}$, alors $\red( \cdot u \cdot \red) \in \mathcal{B}$.
\end{itemize}
Une telle définition de langage est appelée une grammaire. Un autre exemple de langage est la grammaire de la langue française :
\begin{itemize}
	\item $\text{phrase} : \text{sujet} + \text{verbe} + \text{complément}$,
	\item $\text{complément} : \text{COD} + \text{complément}$,
	\item $\text{complément} : \text{CCL} + \text{complément}$,
	\item $\text{complément} : \varepsilon$.
\end{itemize}
Avec cette définition\footnote{On néglige les règles manquantes à cette définition de la grammaire française.}, on peut reconnaître des phrases simples comme \[
	\text{\guillemotleft}~\underbrace{\text{Matthieu}}_{\text{sujet}}\ \underbrace{\text{aime}}_{\text{verbe}}\ \underbrace{\text{les trains}}_{\text{complément}}.\text{~\guillemotright}
\]

\section{Définition, vocabulaire, propriétés}
\subsection{Grammaires non contextuelles}

\begin{defn}
	On se munit d'un alphabet $\Sigma$ qu'on appelle \textit{terminaux}. On se munit d'un ensemble de symboles $\mathcal{V}$ qu'on appelle \textit{non-terminaux}.
	On suppose $\mathcal{V} \cap \Sigma$.
\end{defn}

\begin{exm}
	Dans la suite, on pose $\Sigma = \{\mathord{\red(},\mathord{\red)}\}$ et $\mathcal{V} = \{\mathrm{B}\}$.
\end{exm}

\begin{defn}
	On appelle \textit{règle de production} la donnée
	\begin{itemize}
		\item d'un symbole $V \in \mathcal{V}$,
		\item d'un mot $w_1 w_2 \ldots w_n$ sur l'alphabet $\mathcal{V} \cup \Sigma$,
	\end{itemize}
	que l'on note $V \to w_1w_2\ldots w_n$.
\end{defn}

\begin{exm}
	L'ensemble $\mathcal{B}$ est décrit par les règles
	\begin{itemize}
		\item $\mathrm{B} \to \varepsilon$,
		\item $\mathrm{B} \to \mathrm{BB}$,
		\item $\mathrm{B} \to \mathord{\red(} \mathrm{B} \mathord{\red)}$.
	\end{itemize}
\end{exm}

\begin{defn}[Grammaire non contextuelle]
	Une \textit{grammaire non contextuelle} est la donnée de
	\begin{itemize}
		\item un alphabet de non-terminaux $\mathcal{V}$,
		\item un alphabet de terminaux $\Sigma$,\footnotemark
		\item un ensemble fini de règles de production $P$,
		\item un symbole initial $S \in \mathcal{V}$,
	\end{itemize}
	que l'on note $(\mathcal{V}, \Sigma, P, S)$.
\end{defn}
\footnotetext{avec ``terminaux/non-terminaux'' vient l'implication que $\mathcal{V} \cap \Sigma = \O$}

\begin{exm}
	On note dans la suite \[
		\mathcal{G} = \big(\{\mathrm{B}\}, \{\mathord{\red(},\mathord{\red)}\}, \{\mathrm{B}\to \varepsilon,\mathrm{B}\to \mathrm{BB},\mathrm{B}\to \mathord{\red(} \mathrm{B}\mathord{\red)}\}, \mathrm{B}\big)
	.\]
\end{exm}

\paragraph{Interlude.}
Avec cette définition, on espère pouvoir appliquer ces règles pour définir des mots valides. Par exemple,
\[
	\mathrm{B} \overset?\leadsto \mathrm{BB} \overset?\leadsto \red(\mathrm{B}\red) \mathrm{B} \overset?\leadsto \red(\mathrm{B}\red)\red(\mathrm{B}\red) \overset?\leadsto \red(\mathrm{BB}\red) \red(\mathrm{B}\red) \overset?\leadsto \red(\mathrm{BB}\red)\red(\red) \overset?\leadsto
	\red(\mathrm{B}\red)\red(\red) \overset?\leadsto \red(\red(\mathrm{B}\red)\red)\red(\red) \overset?\leadsto\red{(())()}
.\]
C'est l'objectif de la sous-section suivante.

\subsection{Dérivation}

\begin{defn}[Dérivation immédiate]
	Soit $\mathcal{G} = (\mathcal{V}, \Sigma, P, S)$ une grammaire. Soient $u$ et $v$ deux mots de $(\Sigma \cup \mathcal{V})^*$.
	On dit que $v$ \textit{dérive immédiatement} de $u$ dans la grammaire~$\mathcal{G}$, que l'on note $u \Rightarrow v$, si
	\begin{itemize}
		\item il existe $x$ et $y$ deux mots de $(\Sigma \cup \mathcal{V})^*$
		\item il existe $(V \to w_1 w_2 \ldots w_n) \in P$
	\end{itemize}
	tels que $u = x \cdot V \cdot y$ et $v = x \cdot w_1 w_2 \ldots w_n \cdot y$.
\end{defn}

\begin{exm}
	%Dans l'exemple de l'interlude, on pose $u = \overbrace{\red(}^x\ \mathrm{B}\ \overbrace{\red)\red(\mathrm{B}\red)}^y$ et $v = \underbrace{\red(}_x\ \overline{\mathrm{BB}}}\ \underbrace{\red)\red(\mathrm{B}\red)}_y$.
	Ainsi, $u \Rightarrow v$.
\end{exm}

\begin{lem}[de composition]
	Soit $\mathcal{G} = (\mathcal{V}, \Sigma, P, S)$ une grammaire.
	Soient $u, v \in (\Sigma \cup \mathcal{V})^*$ tels que $u \Rightarrow v$.
	Soit $w \in (\Sigma \cup \mathcal{V})^*$.
	On a $u\cdot w \Rightarrow v \cdot w$ et $w\cdot u\Rightarrow w \cdot v$.
\end{lem}

\begin{prv}
	À faire à la maison.
\end{prv}

On propose trois définitions différentes mais équivalentes pour la dérivation $\Rst$.

\begin{defn}
	Étant donné une grammaire $\mathcal{G} = (\mathcal{V}, \Sigma, P, S)$, on étend $\Rightarrow$ en sa cloture réflexive et transitive, que l'on note $\Rst$, appelé \textit{dérivation}.
	
\end{defn}

\begin{defn}
	On définit $\Rst$ comme $u \Rst v {\ds\iffdef} \exists n \in \N,\:\exists (u_0,u_1,\ldots,u_n) \in \big((\mathcal{V} \cup \Sigma)^*)^{n+1}$ tels que $u_0 = u$, $u_n = v$ et $\forall i \in \llbracket 0,n-1 \rrbracket,\: u_i \Rightarrow u_{i+1}$.
\end{defn}


\begin{defn}
	Soit la suite $(\Rightarrow^n)_{n\in\N}$ définie inductivement par
	\begin{itemize}
		\item $\ds u \Rightarrow^0 v \iffdef u = v$,
		\item $\ds u \Rightarrow^{n+1} v \iffdef u\Rightarrow^n v \text{ ou } \exists w \in (\Sigma \cup \mathcal{V})^*,\: u \Rightarrow w \text{ et } w \Rightarrow^n v$.
	\end{itemize}
	On pose alors $\mathord{\Rst} = \bigcup_{n \in \N} \mathord{\Rightarrow^n}$.
\end{defn}

\begin{lem}[de composition$^*$]
	Soit $\mathcal{G} = (\mathcal{V}, \Sigma, P, S)$ une grammaire, et soient $u, v \in (\Sigma \cup \mathcal{V})^*$ tels que $u \Rightarrow^p v$, pour $p \in \N$.
	Et, soient $(w,t) \in (\Sigma \cup \mathcal{V})^2$ tels que $w \Rightarrow^q t$, pour $q \in \N$.
	Alors, $uw \Rightarrow^{p+q} vt$.
\end{lem}

\begin{prv}
	Soit $u_0u_1\ldots u_p$ tels que $u_0 = u$, $u_p = v$ et $\forall i \in \llbracket 0,p-1 \rrbracket$, $u_i \Rightarrow u_{i+1}$.
	Soit $w_0w_1\ldots w_n$ tels que $w_0 = w$, $w_q = t$ et $\forall i \in \llbracket 0,q-1 \rrbracket$, $w_i \Rightarrow w_{i+1}$.
	Soit alors la dérivation \[
		uw = u_0 w \Rightarrow u_1 w \Rightarrow u_2 w \Rightarrow \cdots \Rightarrow u_p w \Rightarrow u_p w_1 \Rightarrow u_p w_2 \Rightarrow \cdots \Rightarrow u_p w_q = vt
	.\]
	On a donc $uw \Rightarrow^{p+q} vt$.
\end{prv}

\begin{defn}
	Soit $\mathcal{G} = (\mathcal{V}, \Sigma, P, S)$ une grammaire. Son langage est défini par \[
		\mathcal{L}(\mathcal{G}) = \{u \in \Sigma^*  \mid S \Rst u\}
	.\]
\end{defn}

\begin{exm}
	Dans l'exemple précédent, on a \[
		\mathcal{L}(\mathcal{G}) = \mathcal{B}
	.\] 
\end{exm}
