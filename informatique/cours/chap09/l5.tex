\subsection{Lien avec les langages décidables}

\begin{prop}
	Les langages des grammaires non contextuelles sont des langages décidables.
	\qed
\end{prop}

La preuve de cette propriété est dans le \textsc{td} 15. (On peut montrer, par programmation dynamique, que l'appartenance d'un mot à une grammaire non contextuelle est calculable en temps polynômial, en $\mathcal{O}(n^3)$.)

\begin{exm}
	$\triangleright$ \textsc{td} 15, exercice 1.
\end{exm}

\begin{exm}
	On considère l'alphabet $\Sigma = \{C, L\!P, R\!P\}$ et les règles de production $\mathrm{S} \to \mathrm{TS}  \mid C$ et $\mathrm{T} \to L\!P \cdot \mathrm{S} \cdot R\!P$.
	Codons un programme reconnaissant la grammaire définie par ces règles.
	On représente $\Sigma$ par le type \texttt{token} où $C$ correspond à \texttt{C}, $L\!P$ à \texttt{LP} et $R\!P$ à \texttt{RP}.

	\begin{lstlisting}[language=caml,caption=Programme reconnaissant une grammaire $\mathcal{G}$]
type token = C | LP | RP
type word = token list

type non_term = S | T

(* closed derivation tree *)
type cdt =
  | Node of non_term * cdt list
  | Leaf of token option


exception Non

let rec parse_s (w: word): cdt * word =
  match w with
  | C ::  w' -> (Node(S, [Leaf(Some C)]), w')
  | _        ->
      let (g, w')  = parse_t w  in
      let (d, w'') = parse_s w' in
      (Node (S, [g; d]), w'')
and parse_t (w: word): cdt * word =
  match w with
  | LP :: w' -> begin
      let (u, w'') = parse_s w' in
      match w'' with
      | RP :: w''' -> (Node (T,
					[Leaf (Some LP); u; Leaf (Some RP)]), w''')
      | _          -> raise Non
    end
  | _ -> raise Non
	\end{lstlisting}
\end{exm}


