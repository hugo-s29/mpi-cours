Avec une représentation implicite du graphe, on ne stocke pas l'entièreté du graphe directement mais on ne récupère que les successeurs d'un sommet en particulier.

Par exemple, pour écrire un moteur de recherche, on utilise aussi une représentation implicite pour le graphe \textit{internet} ; on ne peut pas \textit{juste} télécharger l'entièreté du graphe.

\paragraph{Attracteur.}

\begin{rmk}
	On suppose que les deux joueurs (Alice et Bob) jouent intelligemment.
	On se met à la place d'Alice. Bob joue le ``mieux'' pour lui, et le ``pire'' pour nous.
	Nous jouerons le ``mieux'' pour nous.
	Ainsi, Bob jouera vers un état de valeur minimale, nous jouerons vers un état de valeur maximale.
\end{rmk}

\begin{figure}[H]
	\centering
	\begin{tikzpicture}[scale=2,minimum size=7.5mm,inner sep=0pt]
		\node[draw,thick,red,circle] (A0) at (0, 0) {A};
		\node[draw,thick,rectangle,mauve] (B1) at (-2, -1.5) {B};
		\node[draw,thick,rectangle,mauve] (B2) at (0, -1.5) {B};
		\node[draw,thick,rectangle,mauve] (B3) at (2, -1.5) {B};
		\node[draw,thick,red,circle] (A11) at (-2.5, -3) {A};
		\node[draw,thick,red,circle] (A12) at (-2, -3) {A};
		\node[draw,thick,red,circle] (A13) at (-1.5, -3) {A};
		\node[draw,thick,red,circle] (A21) at (-0.5, -3) {A};
		\node[draw,thick,red,circle] (A22) at (0, -3) {A};
		\node[draw,thick,red,circle] (A23) at (0.5, -3) {A};
		\node[draw,thick,red,circle] (A31) at (1.5, -3) {A};
		\node[draw,thick,red,circle] (A32) at (2, -3) {A};
		\node[draw,thick,red,circle] (A33) at (2.5, -3) {A};
		\draw[->, bend right] (A0) to node[fill=white]{max} (B1);
		\draw[->] (A0) to node[fill=white]{max} (B2);
		\draw[->, bend left] (A0) to node[fill=white]{max} (B3);
		\draw[->, bend right] (B1) to node[fill=white]{min} (A11);
		\draw[->] (B1) to node[fill=white]{min} (A12);
		\draw[->, bend left] (B1) to node[fill=white]{min} (A13);
		\draw[->, bend right] (B2) to node[fill=white]{min} (A21);
		\draw[->] (B2) to node[fill=white]{min} (A22);
		\draw[->, bend left] (B2) to node[fill=white]{min} (A23);
		\draw[->, bend right] (B3) to node[fill=white]{min} (A31);
		\draw[->] (B3) to node[fill=white]{min} (A32);
		\draw[->, bend left] (B3) to node[fill=white]{min} (A33);
	\end{tikzpicture}
	\caption{Stratégie de minimisation pour Bob, et de maximisation pour Alice}
\end{figure}

\begin{defn}
	On définit la suite d'ensembles $(\mathcal{A}_i)_{i\in\N}$\/ par $\mathcal{A}_0 = \mathcal{O}_\mathrm{A}$, et, pour tout $n \in \N$,
	\begin{align*}
		\mathcal{A}_{n+1} = \mathcal{A}_n &\cup \{s_\mathrm{B} \in V_\mathrm{B}^\star  \mid \forall s_\mathrm{A} \in \mathrm{succ}(s_\mathrm{B}),\: s_\mathrm{A} \in \mathcal{A}_n \}\\
		&\cup \{s_\mathrm{A} \in V_\mathrm{A}^\star  \mid \exists s_\mathrm{B} \in \mathrm{succ}(s_\mathrm{A}),\: s_\mathrm{B} \in \mathcal{A}_n \}.\\
	\end{align*}
	On a appelle alors \textit{attracteur d'Alice} les états $\mathcal{A} = \bigcup_{n \in \N} \mathcal{A}_n$.
\end{defn}

\begin{rmk}
	La suite des $(\mathcal{A}_n)_{n \in \N}$\/ est croissante pour l'inclusion $\subseteq$, et ultimement stationnaire car le graphe est fini : \[
		\exists N \in \N,\: \forall n \ge N,\quad\quad \mathcal{A}_n = \mathcal{A}_N
	.\]
\end{rmk}

De même, on définit la suite d'ensembles $(\mathcal{B}_n)_{n \in \N}$\/ et l'ensemble $\mathcal{B}$\/ des attracteurs de Bob, en permutant les deux joueurs dans la définition précédente.

\begin{exm}
	\todo{Ajouter exemple, \textit{c.f.\ cahier de prépa}}
	Si Bob joue intelligemment, il est garanti de gagner.
\end{exm}

\begin{rmk}
	On remarque que les ensembles $\bigcup_{n \in \N} \mathcal{A}_n$\/ et $\bigcup_{n\in \N} \mathcal{B}_n$\/ ne forment pas une partition de l'ensemble des états : l'intersection est bien vide, mais l'union de ces deux ensembles ne couvre pas l'ensemble des états.
\end{rmk}

\begin{defn}[Stratégie induite par les attracteurs]
	Soit $\mathcal{A}$\/ l'ensemble des attracteurs d'Alice. On définit la stratégie suivante
	\begin{align*}
		f: V_\mathrm{A}^\star &\longrightarrow V_\mathrm{B} \\
		s_\mathrm{A} &\longmapsto \begin{cases}
			s_\mathrm{B} \in \mathrm{succ}(s_\mathrm{A}) \cap \mathcal{A} & \quad\text{ si } s_\mathrm{A} \in \mathcal{A}\\
			s_\mathrm{B} \in \mathrm{succ}(s_\mathrm{A}) \text{ quelconque} & \quad \text{ sinon}.
		\end{cases}
	\end{align*}
	Cette stratégie est nommée \textit{stratégie induite par les attracteurs}.
\end{defn}

\begin{defn}
	Soit $s \in \mathcal{A}$ un attracteur. On appelle \textit{rang} de $s$, notée $\rg(s)$\/ l'entier  \[
		\rg(s) = \min \{n \in \N  \mid s \in \mathcal{A}_n\}
	.\]
\end{defn}

\begin{lem}
	Pour tout entier $n$, pour tout état $s \in \mathcal{A}_n$,  un des trois cas est vrai :
	\begin{itemize}
		\item $s \in \mathcal{O}_\mathrm{A}$\/ ;
		\item $n > 0$, $s \in V_\mathrm{A}^\star$ et il existe $s_\mathrm{B} \in \mathrm{succ}(s)$\/ tel que $s_\mathrm{B} \in \mathcal{A}_{n-1}$\/ ;
		\item $n > 0$, $s \in V_\mathrm{B}^\star$ et pour tout $s_\mathrm{A} \in \mathrm{succ}(s)$, $s_\mathrm{A} \in \mathcal{A}_{n-1}$.
	\end{itemize}
\end{lem}

\begin{prv}
	Par récurrence.
	\begin{itemize}
		\item Si $n = 0$, alors $\mathcal{A}_0 = \mathcal{O}_\mathrm{A}$.
		\item Supposons la propriété vraie au rang $n$. Soit $s \in \mathcal{A}_{n+1}$. Alors,
			\begin{itemize}
				\item ou bien $s \in \mathcal{A}_n$, et on conclut par hypothèse de récurrence.
				\item ou bien $s \in V_\mathrm{A}^\star$, et il existe $s_\mathrm{B} \in \mathrm{succ}(s)$\/ tel que $s_\mathrm{B} \in \mathcal{A}_n$, alors \textsc{ok} ;
				\item ou bien $s \in B_\mathrm{B}^\star$, et pour tout $s_\mathrm{A} \in \mathrm{succ}(s)$, $s_\mathrm{A} \in \mathcal{A}_n$, alors \textsc{ok}.
			\end{itemize}
	\end{itemize}
\end{prv}

\begin{prop}
	On a $\mathcal{A} \cap \mathcal{B} = \O$.
\end{prop}

\begin{prv}
	Montrons par récurrence que, pour tout entier $n$, $\mathcal{A}_n \cap \mathcal{B}_n = \O$. \todo{finir cette partie de la preuve}.
	Soit $N \in \N$ tel que $\mathcal{A}_n = \mathcal{A}_N$, pour tout $n \ge N$.
	Soit $M \in \N$ tel que $\mathcal{B}_n = \mathcal{B}_M$, pour tout $n \ge M$.
	Ainsi, $\mathcal{A} = \mathcal{A}_N$\/ et $\mathcal{B} = \mathcal{B}_M$.
	On pose $K = \max(M, N)$, et on a donc \[
		\mathcal{A} \cap \mathcal{B} = \mathcal{A}_K \cap \mathcal{B}_K = \O
	.\]
\end{prv}

\begin{prop}
	Si $f$\/ est une stratégie induite par les attracteurs d'Alice $\mathcal{A}$, et que $s \in \mathcal{A}$, alors toute partie jouée selon $f$\/ depuis $s$\/ est gagnante pour Alice.
\end{prop}

\begin{prv}
	Soit $(s_n)_{n\in\llbracket 1,N + 1\llbracket}$, avec $N \in \N\cup \{+\infty\}$, une partie jouée selon $f$ depuis $s$.
	\begin{itemize}
		\item Montrons $\forall n \in \llbracket 1,N+1 \llbracket$, $s_n \in \mathcal{A}$.
			Par récurrence. On a $s_1 = s \in \mathcal{A}$, d'où l'initialisation. Si la propriété est vraie au rang $n \in \llbracket 1,N \llbracket$ (ainsi $s_{n+1}$ existe),
			\begin{itemize}
				\item si $s_n \in V_\mathrm{A}^\star$, alors $s_{n+1} = f(s_n) \in \mathrm{succ}(s_n) \cap \mathcal{A}$ donc $s_{n+1} \in \mathcal{A}$,
				\item si $s_n \in V_\mathrm{B}^\star$, on a $s_n \in \mathcal{A}$\/ et $s_{n+1} \in \mathrm{succ}(s_n)$, d'où $s_{n+1} \in \mathcal{A}$.
			\end{itemize}
			Concluant ainsi la récurrence.
		\item Montrons que $N \neq +\infty$.
			En effet, montrons que, $\forall n \in \llbracket 1,N \llbracket$, $\rg(s_{n+1}) < \rg(s_n)$.
			L'état $s_n \in \mathcal{A}$, son rang est bien défini. De plus, $s_n \in \mathcal{A}_{r}$, où $r = \rg(s_n)$. Or, $s_n \not\in \mathcal{O}_\mathrm{A}$.
			D'après le lemme, $s_{n+1} \in \mathcal{A}_{r - 1}$, donc $\rg(s_{n+1}) \le r - 1 < r$.
			Ainsi, $N \neq +\infty$\/ par existence du variant, comme l'ensemble ordonné $(\N, \le)$\/ est bien fondé.
	\end{itemize}
	Ainsi, $(s_n)_{n\in\llbracket 1,N + 1\llbracket}$\/ est une partie finie, et $\forall n \in \llbracket 1,N \rrbracket$, $s_n \in \mathcal{A}$. L'état $s_N$\/ est terminal, et donc $s_N \in \mathcal{O}_\mathrm{A}$, d'où le résultat.
\end{prv}

\begin{prop}
	Soit $s$\/ un sommet admettant une stratégie gagnante pour Alice. Alors, $s$\/ est un attracteur : $s \in \mathcal{A}$.
\end{prop}

\begin{prv}
	Montrons par récurrence forte : \guillemotleft~si $s$ est un sommet admettant une stratégie gagnante $f$ et que toute partie jouée depuis $s$ selon $f$ est de longueur au plus $n$, alors $s \in \mathcal{A}_n$.~\guillemotright\ 
	\begin{itemize}
		\item Si le sommet $s$ admet une stratégie gagnante, et toute partie depuis $s$ est de longueur $0$, alors $s \in \mathcal{O}_\mathrm{A} = \mathcal{A}_0$.
		\item Supposons, pour tout $k \in \llbracket 1,n \rrbracket$, l'hypothèse de récurrence. Montrons la propriété pour $n + 1$. Soit $s$\/ admettant une stratégie gagnante $f$ et tel que toute partie jouée depuis $s$ selon $f$ est de longueur au plus $n+1$.
			\begin{itemize}
				\item Si $s \in V_\mathrm{B}$, alors
					\begin{itemize}
						\item si $s$ est terminal, $s \in \mathcal{O}_\mathrm{A} \subseteq \mathcal{A}$.
						\item sinon, soit $s_1 \in \mathrm{succ}(s)$. Soit $\gamma$\/ une partie jouée selon $f$ depuis $s_1$. $\gamma$ est une partie jouée selon $f$ depuis $s$, elle est donc gagnante et de longueur au plus $n + 1$. Alors $\gamma$\/ est gagnante et de longueur au plus $n$ donc $s_1$ vérifie les prémisses de l'hypothèse de récurrence, et donc $s_1 \in \mathcal{A}_n$. Ceci étant vrai pour tout $s_1 \in \mathrm{succ}(s)$, on a $s \in \mathcal{A}_{n+1}$.
					\end{itemize}
				\item Si $s \in V_\mathrm{A}$, alors
					\begin{itemize}
						\item si $s$ est terminal, alors $s \in \mathcal{O}_\mathrm{A} \subseteq \mathcal{A}$.
						\item sinon, soit $s_1 = f(s)$, alors $s_1$ admet une stratégie gagnante ($f$), et toute partie jouée depuis $s_1$ est de longueur, au plus, $n$. L'état $s_1$ vérifie donc les prémisses l'hypothèse de récurrence. Ainsi $s_1 \in \mathcal{A}_n$, et donc $s$\/ a un successeur dans $\mathcal{A}_n$. On en déduit $s \in \mathcal{A}_{n+1}$.
					\end{itemize}
			\end{itemize}
	\end{itemize}
	Soit alors $s$ admettant une stratégie gagnante $f$. Ainsi, toute partie jouée depuis $s$ selon $f$ est finie. On peut donc conclure grâce à la démonstration par récurrence ci-avant.
\end{prv}


