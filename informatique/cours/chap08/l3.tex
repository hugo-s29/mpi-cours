En pratique, la détermination des attracteurs nécessite une quantité bien trop importante de calculs. On souhaite revenir à une idée développée précédemment : réaliser un $\min$-$\max$. Mais, pour cela, il est nécessaire de donner une notion de \guillemotleft~valeur~\guillemotright\ a chaque état du jeu. On définit donc une \textit{fonction d'heuristique} qui, à un état, associé sa valeur. Le choix de cette fonction reste, cependant, très subjectif.

\section{Résolution par heuristique}

On suppose définie une \textit{fonction d'heuristique} $h$ de la forme : \[
	h : \overset{\mathclap{\substack{\text{joueur}\\\downarrow}}}{\{\mathrm{A},\mathrm{B}\}}  \times \underset{\mathclap{\substack{\uparrow\\\text{états}}}}{V} \to \overbrace{\Z \cup \{+\infty,-\infty\}}^{\bar{\Z}}
.\]

On représente informatiquement l'ensemble $\bar{\Z}$\/ par le type \textsc{OCaml} \texttt{int\_bar} défini ci-dessous
\begin{lstlisting}[language=caml,caption={Type \texttt{int\_bar} représentant un élément l'ensemble $\bar{\Z}$~}]
	type int_bar =
		| PInt
		| NInt
		| F of int
\end{lstlisting}

On peut donc définir l'algorithme \textsc{MinMax}.
\begin{algorithm}[H]
	\centering
	\begin{algorithmic}[1]
		\Entree Un état $e$, un seuil de profondeur $d$, le joueur $j$
		\Sortie Un score
		\If{$\mathrm{succ}(e) = \O$}
			\If{$e \in \mathcal{O}_j$} \State\Return $+\infty$
			\ElsIf{$e \in \mathcal{O}_{\mathrm{autre}(j)}$} \State\Return $-\infty$
			\Else \State\Return $0$
			\EndIf
		\ElsIf{$d = 0$}
			\State\Return $h(j, e)$
		\Else
			\If{$j = \mathrm{joueur}(e)$}
				\State\Return $\max\{\textsc{MinMax}(e', d-1, j) \mid e' \in \mathrm{succ}(e)\}$
			\Else
				\State\Return $\min\,\{\textsc{MinMax}(e', d-1, j) \mid e' \in \mathrm{succ}(e)\}$
			\EndIf
		\EndIf
	\end{algorithmic}
	\caption{Algorithme \textsc{MinMax}}
\end{algorithm}

\begin{exm}
	\textit{c.f.\ cahier-de-prépa} On applique l'algorithme \textsc{MinMax} pour Alice.
\end{exm}

\begin{exm}[\textsc{td} 13.\ Exercice 4]
\end{exm}

On peut améliorer l'algorithme \textsc{MinMax} avec \guillemotleft~l'élagage $\alpha$, $\beta$.~\guillemotright\ 

\begin{algorithm}[H]
	\centering
	\begin{algorithmic}[1]
		\Entree Un état $e$, un seuil de profondeur $d$, le joueur $j$, et $\alpha, \beta \in \bar{\Z}$
		\Sortie Un score dans $[\alpha, \beta]$
		\If{$\mathrm{succ}(e) = \O$}
			\If{$e \in \mathcal{O}_j$} \State\Return $\beta$
			\ElsIf{$e \in \mathcal{O}_{\mathrm{autre}(j)}$} \State\Return $\alpha$
			\Else
				\If{$\beta \le 0$}\ \Return $\beta$
				\ElsIf{$\alpha \ge 0$}\ \Return $\alpha$
				\Else\ \Return $0$
				\EndIf
			\EndIf
		\ElsIf{$d = 0$}
			\If{$h(j, e) \ge \beta$}\ \Return $\beta$
			\ElsIf{$h(j, e) \le \alpha$}\ \Return $\alpha$
			\Else\ \Return $h(j, e)$
			\EndIf
		\Else
			\If{$\mathrm{joueur}(e) = j$}
				\State $v \gets \alpha$
				\For{$e' \in \mathrm{succ}(e)$}
					\State $v \gets \max(v, \textsc{MinMaxÉlagué}(e', d-1, j, v, \beta))$
					\If{$v \ge \beta$}
						\State\Return $\beta$
					\EndIf
				\EndFor
				\State\Return $v$
			\Else
				\State $u \gets \beta$
				\For{$e' \in \mathrm{succ}(e)$}
					\State $u \gets \min(u, \textsc{MinMaxÉlagué}(e', d-1, j, \alpha, u))$
					\If{$u \le \alpha$}
						\State\Return $\alpha$
					\EndIf
				\EndFor
				\State\Return $u$
			\EndIf
		\EndIf
	\end{algorithmic}
	\caption{Algorithme \textsc{MinMax} avec élagage $\alpha$, $\beta$}
\end{algorithm}

