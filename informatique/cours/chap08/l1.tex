\lettrine{D}{ans ce chapitre}, on s'intéresse à la théorie des jeux.
L'objectifs est de modéliser des interactions (économie, coopération, \ldots).
On s'intéressera, par contre, à une petite partie de la théorie des jeux : les jeux d'accessibilité.
Par exemple, les échecs et le go sont deux jeux d'accessibilité, mais on s'intéressera à des jeux beaucoup plus simples.
On cherchera les stratégies gagnantes pour ces jeux, des heuristiques.
On raccrochera ce chapitre avec la théorème des graphes.

Par exemple, on peut coder un programme répondant, \textit{plus ou moins correctement}, au jeu puissance 4. [Démonstration d'un jeu de puissance 4.]


Dans un premier temps, on s'intéresse à un jeu simple. On a 13 allumettes : \[
	|\quad|\quad|\quad|\quad|\quad|\quad|\quad|\quad|\quad|\quad|\quad|\quad|\quad|
.\]
On peut retirer une, deux ou trois allumettes. Le joueur retirant la dernière allumette perd.
On peut représenter le jeu par un graphe, et \textit{jouer} au jeu est se déplacer dans ce graphe. Ce graphe est \textit{biparti}.

\section{Jeux sur un graphe}

\begin{defn}
	On appelle \textit{arène} la donnée
	\begin{itemize}
		\item d'un graphe biparti orienté $G = (V, E)$, avec $V = V_\mathrm{A} \cupdot V_\mathrm{B}$\/ la séparation en deux sommets de ce graphes bipartis ;
		\item d'un sommet initial $s$.
	\end{itemize}
\end{defn}

\begin{defn}
	Un état du jeu (un sommet de l'arène) est dit \textit{terminal} lorsqu'il n'a pas de successeurs. On note, dans la suite, $V_\mathrm{A}^\star$\/ et $V_\mathrm{B}^\star$\/ les états non terminaux (notation non officielle).
\end{defn}

\begin{rmk}
	Les états $V_\mathrm{A}$\/ sont les états où c'est à Alice de jouer.
	Les états $V_\mathrm{B}$\/ sont les états où c'est à Bob de jouer.
\end{rmk}

\begin{exm}
	On reprend l'exemple du jeu des allumettes. La figure ci-dessus représente les états du jeu. Les états
	{\raisebox{-2pt}{\tikz\node[thick,minimum size=3.5mm,inner sep=0pt,auto,draw,circle,red] at (0,0){$i$};}}
	représentent les états de Alice (\textit{i.e.}\ des éléments de $V_\mathrm{A}$), les états
	{\raisebox{-2pt}{\tikz\node[thick,minimum size=3.5mm,,inner sep=0pt,auto,draw,rectangle,mauve] at (0,0){$j$};}}
	représentent les états de Bob (\textit{i.e.}\ des éléments de $V_\mathrm{B}$).
	Les états doublement encadrés sont terminaux, les autres sont non terminaux.
\end{exm}

\begin{figure}[H]
	\centering
	\begin{tikzpicture}[minimum size=7.5mm,scale=2,font=\small,inner sep=0pt,auto,every node/.style=thick]
		\node[draw,circle,red] (A13) at (0, 0) {13};
		\node[draw,rectangle,mauve] (B12) at (-1, -1) {12};
		\node[draw,rectangle,mauve] (B11) at ( 0, -1) {11};
		\node[draw,rectangle,mauve] (B10) at ( 1, -1) {10};
		\node[draw,circle,red] (A11) at (-2, -2) {11};
		\node[draw,circle,red] (A10) at (-1, -2) {10};
		\node[draw,circle,red] (A9)  at ( 0, -2) {9 };
		\node[draw,circle,red] (A8)  at ( 1, -2) {8 };
		\node[draw,circle,red] (A7)  at ( 2, -2) {7 };
		\node[draw,rectangle,mauve] (B9) at (-2, -3) {9};
		\node[draw,rectangle,mauve] (B8) at (-1, -3) {8};
		\node[draw,rectangle,mauve] (B7) at ( 0, -3) {7};
		\node[draw,rectangle,mauve] (B6) at ( 1, -3) {6};
		\node[draw,rectangle,mauve] (B5) at ( 2, -3) {5};
		\node[draw,rectangle,mauve] (B4) at ( 3, -3) {4};
		\node[draw,circle,red] (A6) at (-2, -4) {6};
		\node[draw,circle,red] (A5) at (-1, -4) {5};
		\node[draw,circle,red] (A4) at ( 0, -4) {4};
		\node[draw,circle,red] (A3) at ( 1, -4) {3};
		\node[draw,circle,red] (A2) at ( 2, -4) {2};
		\node[draw,circle,red,double] (A1) at ( 3, -4) {1};
		\node[draw,rectangle,mauve] (B3) at (-1, -5) {3};
		\node[draw,rectangle,mauve] (B2) at ( 0, -5) {2};
		\node[draw,rectangle,mauve,double] (B1) at ( 1, -5) {1};
		\draw[->] (A4) -- (B1);
		\draw[->] (A5) -- (B2);
		\draw[->] (A6) -- (B3);
		\draw[->] (A7) -- (B4);
		\draw[->] (A8) -- (B5);
		\draw[->] (A9) -- (B6);
		\draw[->] (A10) -- (B7);
		\draw[->] (A11) -- (B8);
		\draw[->] (A13) -- (B10);
		\draw[->] (A3) -- (B1);
		\draw[->] (A4) -- (B2);
		\draw[->] (A5) -- (B3);
		\draw[->,dashed] (A6) -- (B4);
		\draw[->] (A7) -- (B5);
		\draw[->] (A8) -- (B6);
		\draw[->] (A9) -- (B7);
		\draw[->] (A10) -- (B8);
		\draw[->] (A11) -- (B9);
		\draw[->] (A13) -- (B11);
		\draw[->] (A2) -- (B1);
		\draw[->] (A3) -- (B2);
		\draw[->] (A4) -- (B3);
		\draw[->,dashed] (A5) -- (B4);
		\draw[->,dashed] (A6) -- (B5);
		\draw[->] (A7) -- (B6);
		\draw[->] (A8) -- (B7);
		\draw[->] (A9) -- (B8);
		\draw[->] (A10) -- (B9);
		\draw[->,dashed] (A11) -- (B10);
		\draw[->] (A13) -- (B12);
		\draw[->] (B4) -- (A1);
		\draw[->] (B5) -- (A2);
		\draw[->] (B6) -- (A3);
		\draw[->] (B7) -- (A4);
		\draw[->] (B8) -- (A5);
		\draw[->] (B9) -- (A6);
		\draw[->] (B10) -- (A7);
		\draw[->] (B11) -- (A8);
		\draw[->] (B12) -- (A9);
		\draw[->,dashed] (B3) -- (A1);
		\draw[->] (B4) -- (A2);
		\draw[->] (B5) -- (A3);
		\draw[->] (B6) -- (A4);
		\draw[->] (B7) -- (A5);
		\draw[->] (B8) -- (A6);
		\draw[->,dashed] (B9) -- (A7);
		\draw[->] (B10) -- (A8);
		\draw[->] (B11) -- (A9);
		\draw[->] (B12) -- (A10);
		\draw[->,dashed] (B2) -- (A1);
		\draw[->,dashed] (B3) -- (A2);
		\draw[->] (B4) -- (A3);
		\draw[->] (B5) -- (A4);
		\draw[->] (B6) -- (A5);
		\draw[->] (B7) -- (A6);
		\draw[->,dashed] (B8) -- (A7);
		\draw[->,dashed] (B9) -- (A8);
		\draw[->] (B10) -- (A9);
		\draw[->] (B11) -- (A10);
		\draw[->] (B12) -- (A11);
	\end{tikzpicture}
	\caption{États du jeu des allumettes}
	\label{fig:jeu-allu}
\end{figure}

\begin{defn}
	Un \textit{jeu d'accessibilité} est déterminé par une arène $G = (V_\mathrm{A} \cupdot V_\mathrm{B}, E)$ de sommet initial $s$, et deux ensembles $\mathcal{O}_\mathrm{A}$\/ et $\mathcal{O}_\mathrm{B}$\/ de sommets terminaux tels que
	\begin{itemize}
		\item $\mathcal{O}_\mathrm{A} \cap \mathcal{O}_\mathrm{B} = \O$,
		\item $\mathcal{O}_\mathrm{A} \subseteq V_\mathrm{A} \cupdot V_\mathrm{B}$,
		\item et $\mathcal{O}_\mathrm{B} \subseteq  V_\mathrm{A} \cupdot V_\mathrm{B}$.
	\end{itemize}
	L'ensemble $\mathcal{O}_\mathrm{A}$\/ représente les sommets \guillemotleft~\textit{gagnants}~\guillemotright\ pour Alice.
	L'ensemble $\mathcal{O}_\mathrm{B}$\/ représente les sommets \guillemotleft~\textit{gagnants}~\guillemotright\ pour Bob.
\end{defn}

\begin{rmk}
	Ils ne forment pas nécessairement une partition des sommets terminaux.
\end{rmk}

\begin{exm}
	Dans la figure~\ref{fig:jeu-allu}, l'état gagnant pour Alice est 
	{\raisebox{-2pt}{\tikz\node[double,thick,minimum size=3.5mm,inner sep=0pt,auto,draw,circle,red] at (0,0){1};}}
	; celui de Bob est 
	{\raisebox{-2pt}{\tikz\node[double,thick,minimum size=3.5mm,,inner sep=0pt,auto,draw,rectangle,mauve] at (0,0){1};}}.
\end{exm}

\begin{defn}
	Une \textit{partie depuis un sommet $s$} dans un jeu d'accessibilité est une suite (finie ou infinie) de sommets formant un chemin dans l'arène, partant de $s$, telle que, si la partie est finie, le dernier sommet est terminal.

	On dit d'une partie
	\begin{itemize}
		\item qu'elle est \textit{gagnée par Alice} si elle est finie et le dernier sommet est dans $\mathcal{O}_\mathrm{A}$\/ ;
		\item qu'elle est \textit{gagnée par Bob} si elle est finie et le dernier sommet est dans $\mathcal{O}_\mathrm{B}$\/ ;
		\item qu'elle est nulle sinon.
	\end{itemize}
\end{defn}

\begin{exm}
	\todo{Ajouter exemple} ?
\end{exm}

\begin{defn}
	On appelle \textit{stratégie pour Alice} une fonction $f : V_\mathrm{A}^\star \to V_\mathrm{B}$\/ telle que, pour tout état $s_\mathrm{A} \in V_\mathrm{A}^\star$,  on a $\big(s_\mathrm{A}, f(s_\mathrm{A})\big) \in E$.
	
	Soit $N \in \N \cup \{+\infty\}$\/ la longueur d'une partie $(s_1, s_2, \ldots, s_n, \ldots, s_N\:?)$, cette partie est dite \textit{jouée selon une stratégie $f$} si \[
		\forall i \in \llbracket 1, N + 1 \llbracket, \quad s_i \in V_\mathrm{A}^\star  \implies s_{i+1} = f(s_i)
	.\]

	Une stratégie pour Alice est dit \textit{gagnante depuis un état $s$\/} dès lors que toute partie depuis $s$\/ jouée selon $f$ est gagnée par Alice.
	Plus généralement, une stratégie est dit \textit{gagnante} si elle est gagnante depuis l'état initial.

	\bigskip
	\centerline{------------------------------------------------}
	\medskip

	On appelle \textit{stratégie pour Bob} une fonction $f : V_\mathrm{B}^\star \to V_\mathrm{B}$\/ telle que, pour tout état $s_\mathrm{B} \in V_\mathrm{B}^\star$,  on a $\big(s_\mathrm{B}, f(s_\mathrm{B})\big) \in E$.
	
	Soit $N \in \N \cup \{+\infty\}$\/ la longueur d'une partie $(s_1, s_2, \ldots, s_n, \ldots, s_N\:?)$, cette partie est dite \textit{jouée selon une stratégie $f$} si \[
		\forall i \in \llbracket 1, N + 1 \llbracket, \quad s_i \in V_\mathrm{B}^\star  \implies s_{i+1} = f(s_i)
	.\]

	Une stratégie pour Bob est dit \textit{gagnante depuis un état $s$\/} dès lors que toute partie depuis $s$ jouée selon $f$ est gagnée par Bob.
\end{defn}

\begin{exm}
	La stratégie {\tikz\draw[-{Implies[]},mauve,thick,double] (0,0) to (0.35, 0);} est gagnante pour Bob.
	\begin{figure}[H]
		\centering
		\begin{tikzpicture}[minimum size=7.5mm,xscale=2,yscale=1.5,font=\small,inner sep=0pt,auto,every node/.style=thick,bend angle=10]
			\node[draw,circle,red] (A0) at ( 0, 0){};
			\node[draw,rectangle,mauve] (B1) at (-1,-1){};
			\node[draw,rectangle,mauve] (B2) at ( 1,-1){};
			\node[draw,circle,red] (A1) at (-1.5,-2){};
			\node[draw,circle,red,double] (A2) at (-0.5,-2){};
			\node[draw,circle,red] (A3) at (0.5,-2){};
			\node[draw,circle,red] (A4) at (1.5,-2){};
			\node[draw,rectangle,mauve,double] (B3) at (-1.75,-3){};
			\node[draw,rectangle,mauve,double] (B4) at (-1.25,-3){};
			\node[draw,rectangle,mauve,double] (B5) at (0.25,-3){};
			\node[draw,rectangle,mauve,double] (B6) at (0.75,-3){};
			\node[draw,rectangle,red,minimum size=8mm] (B7) at (1.25,-3){};
			\node[draw,rectangle,mauve,minimum size=7mm] at (1.25,-3){};
			\node[draw,rectangle,mauve] (B8) at (1.75,-3){};
			\draw[->] (A0) -- (B1); \draw[->] (A0) -- (B2);
			\draw[->] (B1) -- (A1); \draw[->] (B1) -- (A2);
			\draw[->] (B2) -- (A3); \draw[->] (B2) -- (A4);
			\draw[->] (A1) -- (B3); \draw[->] (A1) -- (B4);
			\draw[->] (A3) -- (B5); \draw[->] (A3) -- (B6);
			\draw[->] (A4) -- (B7);
			\draw[->,bend right] (A4) to (B8); \draw[->,bend right] (B8) to (A4);
			\draw[-{Implies[]},mauve,thick,double] (B8) to (A4);
			\draw[-{Implies[]},mauve,thick,double,bend right] (B2) to (A3);
			\draw[-{Implies[]},mauve,thick,double,bend left] (B1) to (A1);
		\end{tikzpicture}
		\caption{Stratégie gagnante pour Bob}
	\end{figure}
\end{exm}

\paragraph{Interlude : représentation machine des jeux.}
On représente le graphe de manière \ul{implicite}.

\begin{lstlisting}[language={[Objective]caml},caption={Représentation machine des jeux, types}]
type joueur
type etat

val joueur : etat -> move
val possibilite_moves : etat -> etat list
val init : etat
val est_gagant : etat -> joueur option
\end{lstlisting}


\begin{lstlisting}[language=caml,caption={Représentation machine du jeux des allumettes}]
type joueur = Alice | Bob
type etat = { allu : int; joueur : joueur }

let autre = function
| Alice -> Bob
| Bob -> Alice

let joueur (e: etat) = e.joueur
let init = { allu = 13; joueur = Alice }
let est_gagnant (e: etat) = 
	if e.allu = 1 then Some(autre e.joueur)
	else None
let possible_moves (e: etat) = 
	(if e.allu > 3 then
		[{ allu = e.allu - 3; joueur = autre e.joueur }] else [])
	@ (if e.allu > 2 then
		[{ allu = e.allu - 2; joueur = autre e.joueur }] else [])
	@ (if e.allu > 1 then
		[{ allu = e.allu - 1; joueur = autre e.joueur }] else [])
\end{lstlisting}


