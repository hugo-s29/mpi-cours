\section{Sémaphore}

Un sémaphore est utilisé pour assurer une propriété moins forte que le \textit{mutex} : on assure qu'il n'y a pas \guillemotleft~trop~\guillemotright\ de personnes dans la zone critique.
On fixe un nombre maximal de fils qui sont dans la zone critique et on évite un \guillemotleft~flot~\guillemotright\ de personnes ininterrompu dans la zone critique.
Par exemple, en \textsc{tp}, on n'a qu'un nombre limité d'ordinateurs. À l'entrée de la salle, on met à disposition les ordinateurs et chacun dépose le sien lorsqu'il a fini de l'utiliser.

\begin{defn}
	Le type de données abstrait \textsf{sémaphore} fournit
	\begin{itemize}
		\item le type \texttt{t} des sémaphores,
		\item une fonction d'acquisition du sémaphore $\texttt{acquire} : \texttt{t} \to (\:)$,
		\item une fonction de libération du sémaphore $\texttt{release} : \texttt{t} \to (\:)$,
		\item une fonction de création/d'initialisation du sémaphore $\texttt{make} : \N\to \texttt{t}$,
	\end{itemize}
	tels que
	\begin{itemize}
		\item lors d'une tentative d'acquisition du sémaphore : si le compteur du sémaphore est nul, alors le fil d'exécution courant est mis en attente ; sinon, le compteur est décrémenté et le fil d'exécution peut continuer ;
		\item lors de la libération du sémaphore : si un fil d'exécution est en attente, on le laisse continuer son exécution ; sinon, on incrémente le compteur.
	\end{itemize}
\end{defn}

\begin{exm}
	On considère deux \guillemotleft~types~\guillemotright\ de personnes. Les \textit{producteurs} qui peuvent écrire une donnée. Les \textit{consommateurs} qui récupère une donnée (qui consomme une donnée).
	En tant que métaphore, on considère que les données sont des pommes.
	Le producteur ne doit pas produire trop de pommes, le consommateur doit avoir des pommes à consommer.
	On considère, à présent, l'algorithme ci-dessous.

	\begin{algorithm}[H]
		\centering
		\begin{algorithmic}[1]
			\Procedure{Producteur}{}
			\State $\texttt{acquire}\ \mathrm{plein}$
			\State $\texttt{lock}()$
			\State Produit une pomme.
			\State $\texttt{unlock}()$
			\State $\texttt{release}\ \mathrm{vide}$
			\EndProcedure
			\bigskip

			\Procedure{Consommateur}{}
			\State $\texttt{acquire}\ \mathrm{vide}$
			\State $\texttt{lock}()$
			\State Mange une pomme.
			\State $\texttt{unlock}()$
			\State $\texttt{release}\ \mathrm{plein}$
			\EndProcedure
			\bigskip
			\State $\mathrm{vide} \gets \texttt{make}\ 0$
			\State $\mathrm{plein} \gets \texttt{make}\ \mathrm{nb\_pommes}$

			\bigskip

			\For{$i\in\llbracket 1,n \rrbracket$}
			\State Soit un nouveau consommateur.\Comment{\textit{i.e.}\ un fil d'exécution qui exécute la procédure $\Call{Consommateur}{}$}
			\EndFor
			\For{$j\in \llbracket 1,m \rrbracket$}
			\State Soit un nouveau producteur.\Comment{\textit{i.e.}\ un fil d'exécution qui exécute la procédure $\Call{Producteur}{}$}
			\EndFor
		\end{algorithmic}
		\caption{Utilisation de la structure \textsc{sémaphore} dans une problématique producteur/consommateur}
	\end{algorithm}
\end{exm}

