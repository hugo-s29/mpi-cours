\subsection{À $N$ fils d'exécutions}

On propose de l'algorithme de la boulangerie.
On se donne un \guillemotleft~ticket~\guillemotright\ qui donne l'ordre de passage.
En voulant accéder à la boulangerie, on prend un ticket (1 plus la valeur maximale des tickets), et on attend son tour
Pour attendre son tour, on attend que chacune des personnes n'ai un ticket inférieur au sien.
Ceci donne l'algorithme suivant.

\begin{algorithm}[H]
	\centering
	\begin{algorithmic}[1]
		\State $\mathrm{Ticket}$ est un tableau de $n$ entiers initialisés à $0$.
		\Procedure{Lock}{$\mathrm{Ticket},i$}
		\State $\mathrm{Ticket}[i] \gets 1 + \max \{\mathrm{Ticket}[j]  \mid j \in \llbracket 0,n-1 \rrbracket\} $
		\For{$j \in \llbracket 0,n-1 \rrbracket$}
		\While {$\mathrm{Ticket}[i] \neq 0$ et $\mathrm{Ticket}[j] < \mathrm{Ticket}[i]$}
		\State rien
		\EndWhile
		\EndFor
		\EndProcedure
		\Procedure{Unlock}{$\mathrm{Ticket},i$}
		\State $\mathrm{Ticket}[i] \gets 0$\/
		\EndProcedure
	\end{algorithmic}
	\caption{Tentative 1 d'implémentation du type \textsf{verrou} à $N$ fils}
\end{algorithm}

Mais, cet algorithme peuvent prendre un ticket pendant le calcul du $\max$. On utilise une variable $\mathrm{EnCalcul}$ qui dit lorsque le calcul du $\max$ est en cours.

\begin{algorithm}[H]
	\centering
	\begin{algorithmic}[1]
		\State $\mathrm{Ticket}$ est un tableau de $n$ entiers initialisés à $0$.
		\State $\mathrm{EnCalcul}$ est un tableau de $n$ booléens initialisés à $\mathbf{F}$.
		\Procedure{Lock}{$\mathrm{Ticket},i$}
		\State $\mathrm{EnCalcul}[i] \gets \mathbf{V}$
		\State $\mathrm{Ticket}[i] \gets 1 + \max \{\mathrm{Ticket}[j]  \mid j \in \llbracket 0,n-1 \rrbracket\} $
		\State $\mathrm{EnCalcul}[i] \gets \mathbf{F}$
		\For{$j \in \llbracket 0,n-1 \rrbracket$}
		\While{$\mathrm{EnCalcul}[j]$}
		\State rien
		\EndWhile
		\While {$\mathrm{Ticket}[i] \neq 0$ et $\mathrm{Ticket}[j] < \mathrm{Ticket}[i]$}
		\State rien
		\EndWhile
		\EndFor
		\EndProcedure
		\Procedure{Unlock}{$\mathrm{Ticket},i$}
		\State $\mathrm{Ticket}[i] \gets 0$\/
		\EndProcedure
	\end{algorithmic}
	\caption{Tentative 2 d'implémentation du type \textsf{verrou} à $N$ fils}
\end{algorithm}

Mais, cet algorithme laisse avoir deux personnes ayant un ticket de même valeur.
On peut départager les deux personnes avec les identifiants.

\begin{algorithm}[H]
	\centering
	\begin{algorithmic}[1]
		\State $\mathrm{Ticket}$ est un tableau de $n$ entiers initialisés à $0$.
		\State $\mathrm{EnCalcul}$ est un tableau de $n$ booléens initialisés à $\mathbf{F}$.
		\Procedure{Lock}{$\mathrm{Ticket},i$}
		\State $\mathrm{EnCalcul}[i] \gets \mathbf{V}$
		\State $\mathrm{Ticket}[i] \gets 1 + \max \{\mathrm{Ticket}[j]  \mid j \in \llbracket 0,n-1 \rrbracket\} $
		\State $\mathrm{EnCalcul}[i] \gets \mathbf{F}$
		\For{$j \in \llbracket 0,n-1 \rrbracket$}
		\While{$\mathrm{EnCalcul}[j]$}
		\State rien
		\EndWhile
		\While {{\footnotesize $\mathrm{Ticket}[j] \neq 0$ et \smash{$\underbrace{\text{[$\mathrm{Ticket}[j] < \mathrm{Ticket}[i]$ ou ($\mathrm{Ticket}[j] = \mathrm{Ticket}[i]$ et $j < i$)]}}_{\text{ordre lexicographique}}$}}}
		\State rien
		\EndWhile
		\EndFor
		\EndProcedure
		\Procedure{Unlock}{$\mathrm{Ticket},i$}
		\State $\mathrm{Ticket}[i] \gets 0$\/
		\EndProcedure
	\end{algorithmic}
	\caption{Tentative 3 d'implémentation du type \textsf{verrou} à $N$ fils -- algorithme de la boulangerie}
\end{algorithm}

Cet algorithme assure l'exclusion mutuelle.
On peut penser que ce problème peut créer le problème de famine.
Mais, non, la comparaison entre identifiants n'a lieu qu'en cas d'égalité de ticket, donc lorsque deux personnes arrivent en même temps à la boulangerie.

