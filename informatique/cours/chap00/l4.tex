\subsection{Par des formules sous formes normales ?}

\begin{defn}
	On dit d'une formule de la forme
	\begin{itemize}
		\item $p$\/ ou $\lnot p$\/ avec $p \in \mathcal{P}$, que c'est un {\it littéral}\/ ;
		\item $\bigwedge_{i=1}^n \ell_i$\/ où les $\ell_i$\/ sont des littéraux que c'est une {\it clause conjonctive}\/ ;
		\item $\bigvee_{i=1}^n \ell_i$\/ où les $\ell_i$\/ sont des littéraux que c'est une {\it clause disjonctive}\/ ;
		\item $\bigwedge_{i=1}^n D_i$\/ où les $D_i$\/ qui sont des clauses disjonctives est appelée une {\it forme normale conjonctive}\/ ;
		\item $\bigvee_{i=1}^n C_i$\/ où les $C_i$\/ qui sont des clauses conjonctives est appelée une {\it forme normale disjonctive}.
	\end{itemize}
\end{defn}

\begin{rmk}
	On prend, comme convention, que $\bigwedge_{i=1}^0 G_i = \top$\/ et $\bigvee_{i=1}^0 G_i = \bot$.
\end{rmk}

\begin{exm}
	La formule $\overbrace{(p \land \lnot q)}^{\text{clause conjonctive}} \lor \overbrace{(r \land p)}^{\text{clause conjonctive}}$\/ est donc une clause normale disjonctive.
\end{exm}

\begin{rmk}
	On écrit {\sc fmd}\/ pour une forme normale disjonctive et {\sc fnc}\/ pour une forme normale conjonctive.
\end{rmk}

\begin{exm}
	La formule $p \land q \land\lnot r$\/ est une clause conjonctive donc une {\sc fnc}\/ mais c'est aussi une {\sc fnd}.
\end{exm}

\begin{exm}
	La formule $\top $\/ est une clause conjonctive de taille 0, donc c'est une {\sc fnd}. Mais, c'est aussi une clause conjonctive de taille 0, donc c'est une {\sc fnc}. De même, la formule $\bot$\/ est une {\sc fnc}\/ et une {\sc fnd}.
\end{exm}

\begin{thm}
	Toute formule est équivalente à une formule sous {\sc fnd} et à une formule sous {\sc fnc}.
\end{thm}

\begin{prv}
	Soit $G \in \mathcal{F}$\/ une formule. Soit $\left\llbracket G \right\rrbracket$\/ la fonction booléenne associée à $G$. Alors, par le théorème précédent, il existe une formule $H$\/ telle que $\left\llbracket H \right\rrbracket = \left\llbracket G \right\rrbracket$\/ (i.e.\ $H \equiv G$) avec $H$\/ construit dans la preuve précédente sous forme normale disjonctive.
\end{prv}

\begin{exm}
	La formule $G = p \land (\lnot q \lor p)$\/ a pour table de vérité la table suivante.
	\begin{table}[H]
		\centering
		\begin{tabular}{c|c|c}
			$p$&$q$&$\left\llbracket G \right\rrbracket$\\ \hline
			$\bfm F$&$\bfm F$&$\bfm F$\\
			$\bfm F$&$\bfm V$&$\bfm F$\\
			$\bfm V$&$\bfm F$&$\bfm V$\\
			$\bfm V$&$\bfm V$&$\bfm V$\\
		\end{tabular}
		\caption{Table de vérité de $p \land (\lnot q \lor p)$}
	\end{table}
	La forme normale disjonctive équivalente à $G$\/ est $(p \land \lnot q) \lor (p \land q)$.
\end{exm}

Nous n'avons pas encore prouvé la deuxième partie du théorème mais, on essaie de trouver une formule sous {\sc fnc}\/ :
\begin{exm}
	On reprend l'exemple de la table de vérité d'une fonction inconnue.

	\begin{table}[H]
		\centering
		\begin{tabular}{c|c|c|c|c}
			$p$&$q$&$r$&$f$&$\bar f$\\[2mm]\hline
			$\bfm F$&$\bfm F$&$\bfm F$&$\bfm V$&$\bfm F$\\
			$\bfm F$&$\bfm F$&$\bfm V$&$\bfm F$&$\bfm V$\\
			$\bfm F$&$\bfm V$&$\bfm F$&$\bfm F$&$\bfm V$\\
			$\bfm F$&$\bfm V$&$\bfm V$&$\bfm V$&$\bfm F$\\
			$\bfm V$&$\bfm F$&$\bfm F$&$\bfm V$&$\bfm F$\\
			$\bfm V$&$\bfm F$&$\bfm V$&$\bfm F$&$\bfm V$\\
			$\bfm V$&$\bfm V$&$\bfm F$&$\bfm V$&$\bfm F$\\
			$\bfm V$&$\bfm V$&$\bfm V$&$\bfm F$&$\bfm V$
		\end{tabular}
		\caption{Table de vérité d'une formule inconnue (2)}
	\end{table}

	On analyse la formule $\bar{f}$\/ au lieu de $f$. Grâce à la première partie du théorème (et de la méthode pour générer cette {\sc fnd}), on a \[
		\bar{f}= (\lnot p \land \lnot q \land r)
		\lor
		(\lnot p \land q \land \lnot r)
		\lor
		(p \land \lnot q \land r)
		\lor
		(p \land q \land r)
	.\]
	Et, à l'aide des lois de {\sc De Morgan}, on a \[
		\bar{f}= (p \lor q \lor\lnot r)
		\land
		( p \lor\lnot q \lor r)
		\land
		(\lnot p \lor q \lor \lnot r)
		\land
		(\lnot p \lor \lnot q \lor \lnot r),
	\] ce qui est une {\sc fnc}.

	À l'aide de cet algorithme, on prouve facilement la 2\tsup{nde} partie du théorème.
\end{exm}

\begin{rmk}
	Il est en fait possible de transformer une formule en {\sc fnd}\/ en appliquant les règles suivantes à toutes les sous-formules jusqu'à obtention d'un point fixe.
	\begin{multicols}{2}
		\begin{itemize}
			\item $\lnot \lnot H \leadsto H$\/ ;
			\item $\lnot(G \land H) \leadsto G \lor H$\/ ;
			\item $\lnot (G \lor H) \leadsto G \land H$\/ ;
			\item $(G \lor H) \land I \leadsto (G \land I) \lor (H \land I)$\/ ;
			\item $I \land (G \lor H) \leadsto (I \land G) \lor (I \land H)$\/ ;
		\end{itemize}
	\end{multicols}
	\begin{multicols}{3}
		\begin{itemize}
			\item $H \land \top \leadsto H$\/ ;
			\item $\top \land H \leadsto H$\/ ;
			\item $H \lor \bot \leadsto H$\/ ;
			\item $\bot \lor H \leadsto H$\/ ;
			\item $\lnot \top \leadsto \bot$\/ ;
			\item $\bot \land H \leadsto \bot$\/ ;
			\item $H \land \bot \leadsto \bot$\/ ;
			\item $\lnot \bot \leadsto \top$\/ ;
			\item $\top \lor H \leadsto \top$\/ ;
			\item $H \lor \top \leadsto \top$.
		\end{itemize}
	\end{multicols}
\end{rmk}

\begin{prop}
	Soit $n \ge 2$\/ et $H_n$\/ la formule $H_n = (a_1 \lor b_1) \land (a_2 \lor b_2) \land \cdots \land (a_n \lor b_n)$\/ avec $\mathcal{P}_n = \{a_1,b_1,a_2,b_2,\ldots,a_n,b_n\}$. Alors, par application de l'algorithme précédent on obtient \[
		\bigvee_{P \in \wp(\left\llbracket 1,n \right\rrbracket)}\bigg(\bigwedge_{j = 1}^n \Big\{{\scriptstyle a_j \text{ si } j \in P \atop\scriptstyle b_j \text{ sinon }}\bigg)
	.\]
\end{prop}

\todo{}
\begin{prv}[par récurrence]
\end{prv}

\begin{rmk}
	Qu'en est-il du problème {\sc Sat} ? Le problème est-il simplifié pour les {\sc fnd}\/ ou les {\sc fnc}\/ ?
	
	Oui, pour les {\sc fnd}, le problème se simplifie. On considère, par exemple, la formule \[
		\begin{array}{cc}
			&(\ell_{11} \land \ell_{12}\land \cdots \land \ell_{1,n_1})\\
			\lor&(\ell_{21} \land \ell_{22}\land \cdots \land \ell_{2,n_2})\\
			\vdots&\vdots\\
			\lor&(\ell_{m,1} \land \ell_{m,2} \cdots \ell_{m,n_m}).
		\end{array}
	\]
	On procède en suivant l'algorithme suivant :
	(\todo{Mettre l'algorithme à part})
	Pour $i$\/ fixé, je lis la ligne $i$, puis je fabrique un environnement $\rho$.

	Par exemple, pour $(p \land \lnot q \land r \land \lnot p) \lor (q \land r \land \lnot q) \lor (p \land r)$, on a $\rho = (p \mapsto {\bfm V}, r \mapsto {\bfm V})$.

	On en conclut que {\sc Sat}\/ peut être résolu en temps linéaire dans le cas d'une forme normale disjonctive. Le problème est de construire cette {\sc fnd}.
\end{rmk}

\begin{rmk}
	Après s'être intéressé au problème {\sc Sat}, on s'intéresse au problème Valide.

	Par exemple, on considère la formule $(p \lor q \lor\lnot r \lor \lnot p) \land (p \lor \lnot r \lor p \lor r) \land (q \lor r)$.
	On peut construire $\rho = (q\mapsto {\bfm F}, r \mapsto {\bfm F})$\/ est tel que $\left\llbracket H \right\rrbracket^\rho = {\bfm F}$.

	Si on ne peut pas construire un tel environnement propositionnel, la formule vérifie le problème Valide.

	On en conclut que Valide peut être résolu en temps linéaire dans le cas d'une forme normale conjonctive. Le problème est de construire cette {\sc fnc}.
\end{rmk}

\section{Algorithme de {\scshape Quine}}

\begin{rmk}
	Une forme normale peut être vue comme un ensemble d'ensembles de littéraux (c'est la représentation que nous allons utiliser en OCamL).
\end{rmk}

\begin{exm}\hfill
	\begin{multicols}{1}
		L'ensemble $\big\{\{p,q\},\{p,r\}, \O\big\}$, a pour formule sous {\sc fnc}\/ associée $(p\lor \lnot q) \land (q \lor r) \land \bot$.

		L'ensemble $\big\{\{p,\lnot q\}, \{q,r\}, \O\big\}$\/ a pour formule sous {\sc fnd}\/ associée $(p \land \lnot q) \lor (q \land r) \lor \top$.

		L'ensemble $\O$\/ a pour formule sous {\sc fnc}\/ associée $\top$.

		L'ensemble $\O$\/ a pour formule sous {\sc fnd}\/ associée $\bot$.
	\end{multicols}
\end{exm}

\begin{lem}
	Pour toute formule $G$, pour tout variable propositionnelle et pour tout environnement propositionnel $\rho$, tel que $\rho(p) = {\bfm V}$, alors \[
		\Big\llbracket G[p \mapsto \top] \Big\rrbracket^\rho = \left\llbracket G \right\rrbracket^\rho
	.\]
\end{lem}

