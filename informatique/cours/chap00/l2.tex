\section{Syntaxe}

\begin{defn}
	On suppose donné un ensemble $\mathcal{P}$\/ de variables propositionnelles.
	\index{variable propositionnelle}
\end{defn}

\begin{defn}
	On définit alors l'ensemble des formules de la logique propositionnelle par induction nommée avec les règles :
	\begin{multicols}{3}
		\begin{itemize}
			\item $\red\lnot\big|^1$\/ ;
			\item $\red\land\big|^2$\/ ;
			\item $\red\lor\big|^2$\/ ;
			\item $\red\to\big|^2$\/ ;
			\item $\red\leftrightarrow\big|^2$\/ ;
			\item $\red\top\big|^0$\/ ;
			\item $\red\bot\big|^0$\/ ;
			\item $\red V\big|^0_{\mathcal{P}}$.
		\end{itemize}
	\end{multicols}

	On nomme l'ensemble des formules $\mathcal{F}$.
	\index{ensemble des formules}
	\index{formule logique}
\end{defn}

\begin{exm}
	\[
		\red{\lor}(\red{\land}(\red{\to}(\red{V}(P), \red{\top}(~), \red{\lnot}(\red{\bot}(~))), \red{\lor}(\red{\leftrightarrow}(\red{\top}(~),\red{\top}(~)),\red{V}(r))
	.\]

	\begin{figure}[H]
		\centering
		\Tree[.{$\red{\lor}$}
			[.{$\red{\land}$}
				[.{$\red{\to}$}
					[.{$\red{V}$} {P} ]
					[.{$\red{\top}$} {} ]
				]
				[.{$\red{\lnot}$} {$\red{\bot}$} ]
			]
			[.{$\red{\lor}$}
				[.{$\red{\leftrightarrow}$}
					{$\red{\top}$}
					{$\red{\top}$}
				]
			]
			[.{$\red{V}$} {$r$} ]
		]
		\caption{Arbre syntaxique d'une expression logique}
	\end{figure}

	Pour simplifier la syntaxe, on écrit plutôt \[
		\big((p \to \top) \land \lnot \bot\big) \lor \big( (\top \leftrightarrow \top) \lor r)
	.\]
\end{exm}

\begin{defn}[taille d'une formule]
	On définit, par induction, la taille notée ``$\mathrm{taille}$\/'' comme
	\begin{align*}
		\mathrm{taille}: \mathcal{F} &\longrightarrow \N \\
		p \in \mathcal{P} &\longmapsto 1\\
		\top &\longmapsto 1\\
		\bot &\longmapsto 1\\
		\lnot G &\longmapsto 1 + \mathrm{taille}(G)\\
		 G \to H &\longmapsto 1 + \mathrm{taille}(G) + \mathrm{taille}(H)\\
		 G \leftrightarrow H &\longmapsto 1 + \mathrm{taille}(G) + \mathrm{taille}(H)\\
		 G \land H &\longmapsto 1 + \mathrm{taille}(G) + \mathrm{taille}(H)\\
		 G \lor H &\longmapsto 1 + \mathrm{taille}(G) + \mathrm{taille}(H).
	\end{align*}
	\index{taille (d'une formule)}
\end{defn}

\begin{defn}[Ensemble des variables propositionnelles]
	On définit inductivement
	\begin{align*}
		\mathrm{vars}: \mathcal{F} &\longrightarrow \wp(\mathcal{P})\quad\hbox{}\footnotemark\\
		p \in \mathcal{P} &\longmapsto \{p\}\\
		\top,\bot &\longmapsto \O\\
		\lnot G &\longmapsto \mathrm{vars}(G)\\
		G \odot H&\longmapsto \mathrm{vars}(G) \cup \mathrm{vars}(H)\\
	\end{align*}
	où $\odot$\/ correspond à $\cup$, $\cap$, $\to$ ou $\leftrightarrow$.
	\index{variables (d'une formule)}
\end{defn}
\footnotetext{Le $\wp(E)$\/ représente ici l'ensemble des parties de $E$.}

\begin{defn}
	On appelle {\it substitution}\/ une fonction de $\mathcal{P}$\/ dans $\mathcal{F}$\/ qui est l'identité partout sauf sur un ensemble fini de variables.
	On la note alors \[
		(p_1 \mapsto H_1,\,p_2\mapsto H_2,\ldots,p_n\mapsto H_n)
	\] qui est la substitution
	\begin{align*}
		\mathcal{P} &\longrightarrow \mathcal{F} \\
		p &\longmapsto \begin{cases}
			H_i &\text{ si } p = p_i\\
			p &\text{ sinon}.
		\end{cases}
	\end{align*}
	\index{substitution}
\end{defn}

\begin{exm}
	La fonction \[
		\sigma = (p \mapsto p \lor q,\,r \mapsto p \land \top)
	\] est une substitution. On a $\sigma(p) = p \lor q$, $\sigma(r) = p \land \top$, $\sigma(q) = q$\/ et, pour toute autre variable logique $a$, $\sigma(a) = a$.
\end{exm}

\begin{defn}[Application d'une substitution à une formule]
	Étant donné une formule $G \in \mathcal{F}$\/ et une substitution $\sigma$, on définit inductivement $G[\sigma]$\/ par \[
		\begin{cases}
			\top[\sigma] = \top\\
			\bot[\sigma] = \bot\\
			p[\sigma] = \sigma(p)\\
			(\lnot G)[\sigma] = \lnot (G[\sigma])\\
			(G \odot H)[\sigma] = (G[\sigma])\odot H[\sigma]
		\end{cases}
	\]
	où $\odot$\/ correspond à $\cup$, $\cap$, $\to$ ou $\leftrightarrow$.
	\index{substitution!application à une formule}
\end{defn}

\begin{exm}
	Avec $G = p \land (q \lor \top)$\/ et $\sigma = (p \mapsto p,\,q\mapsto r \land \top)$, on a \[
		G[\sigma] = q \land \big((r \land \top) \lor \bot\big)
	.\]
\end{exm}

\begin{defn}
	On appelle parfois {\it clés}\/ d'une substitution de $\sigma$, l'ensemble des variables propositionnelles sur lequel elle n'est pas l'identité.
	\index{substitution!clés}
\end{defn}

\begin{defn}
	On définit la {\it composée}\/ de deux substitutions $\sigma$\/ et $\sigma'$\/ par 
	\begin{align*}
		\sigma \cdot  \sigma': \mathcal{P} &\longrightarrow \mathcal{F} \\
		p &\longmapsto \big(p[\sigma]\big)[\sigma].
	\end{align*}
	\index{substitution!composée}
\end{defn}

\begin{exm}
	Avec $\sigma = (p \mapsto q)$\/ et $\sigma = (q \mapsto r)$, on a \[
		\sigma' \cdot \sigma = (p \mapsto r,\,q\mapsto r)\\
	.\] En effet,
	\begin{align*}
		\sigma' \cdot \sigma(x) &= \begin{cases}
			r &\text{ si } x = p\\
			r &\text{ si } x= q\\
			x &\text{ sinon}.
		\end{cases} \\
	\end{align*}
\end{exm}

\begin{exm}
	Avec $\sigma = ( p \mapsto q \land \top )$, $\sigma' = (q \mapsto \bot,\,r \mapsto p)$, on a
	\begin{align*}
		\sigma' \cdot \sigma (x) &= \begin{cases}
			\bot \land \top &\text{ si } x = p\\
			\bot &\text{ si} x = q\\
			p &\text{ si } x = r\\
			x &\text{ sinon}
		\end{cases}\\
		&= (p\mapsto \bot \land \top,\,q\mapsto \bot,\, r \mapsto p). \\
	\end{align*}
\end{exm}

\begin{rmk}
	L'opération $\cdot$\/ est associative.
\end{rmk}

\begin{prop}
	Soient $\sigma$\/ et $\sigma'$\/ deux substitutions, on a, pour toute formule $H \in \mathcal{F}$, \[
		\big(H[\sigma]\big)[\sigma'] = H[\sigma' \cdot \sigma]
	.\]
\end{prop}

\begin{prv}
	Notons $P_G$\/ la propriété \hfill ``$(G[\sigma])[\sigma'] = G[\sigma' \cdot \sigma]$\/'' \hfill\hbox{}

	Montrons que, pour toute formule $G \in \mathcal{F}$, $P_G$\/ est vraie par induction : 
	\begin{itemize}
		\item $\big(\top[\sigma]\big)[\sigma'] \mathrel{\mathop=^{\text{(def)}}} \top = \top[\sigma' \cdot]$\/
		\item $\big(p[\sigma]\big)[\sigma'] = p[\sigma' \cdot \sigma]$\/ 
		\item à faire à la maison : le cas $\lnot $\/ et un cas $\land$.
	\end{itemize}
\end{prv}

\begin{defn}
	On appelle {\it relation sous-formule}, la relation $\diamond$ définie dans la section 3 du chapitre~$-\text{1}$.
	\index{relation d'ordre!sous-formule}
\end{defn}

\section{Sémantique}

\subsection{Algèbre de {\scshape Boole}}

\begin{defn}
	On note $\mathds{B} = \{{\bfm V},{\bfm F}\}$\/ l'{\it ensemble des booléens}.
	\index{booléens!(ensemble)}
\end{defn}

\begin{defn}
	Sur $\mathds{B}$, on définit les opérateurs
	\begin{table}[H]
		\centering
		\begin{tabular}{c|c|c}
			$a$\/ & $b$\/ &$a\cdot b$\/ \\ \hline
			${\bfm F}\/$\/&${\bfm F}\/$\/&${\bfm F}\/$\/\\
			${\bfm F}\/$\/&${\bfm V}\/$\/&${\bfm F}\/$\/\\
			${\bfm V}\/$\/&${\bfm F}\/$\/&${\bfm F}\/$\/\\
			${\bfm V}\/$\/&${\bfm V}\/$\/&${\bfm V}\/$\/\\
		\end{tabular}
		\caption{Opération $\cdot $\/ sur les booléens}
	\end{table}
	\index{booléens!opération $\cdot$}

	\begin{table}[H]
		\centering
		\begin{tabular}{c|c|c}
			$a$\/ & $b$\/ &$a+b$\/ \\ \hline
			${\bfm F}\/$\/&${\bfm F}\/$\/&${\bfm F}\/$\/\\
			${\bfm F}\/$\/&${\bfm V}\/$\/&${\bfm V}\/$\/\\
			${\bfm V}\/$\/&${\bfm F}\/$\/&${\bfm V}\/$\/\\
			${\bfm V}\/$\/&${\bfm V}\/$\/&${\bfm V}\/$\/\\
		\end{tabular}
		\caption{Opération $+$\/ sur les booléens}
	\end{table}
	\index{booléens!opération $+$}

	\begin{table}[H]
		\centering
		\begin{tabular}{c|c}
			$a$\/ &$\bar{a}$\/ \\ \hline
			${\bfm F}\/$\/&${\bfm V}\/$\/\\
			${\bfm V}\/$\/&${\bfm F}\/$\/\\
		\end{tabular}
		\caption{Opération $\bar{\square}$\/ sur les booléens}
	\end{table}
	\index{booléens!opération $\bar{\square}$}
\end{defn}

\begin{rmk}~

	\begin{table}[H]
		\centering
		\begin{tabular}{c|c|c}
			{\sc Nom}\/&$\cdot$\/&$+$\/\\ \hline
			Commutativité&$a\cdot b = b\cdot a$\/ &$a+b=b+a$\/ \\
			Neutre&${\bfm V}\/ \cdot a = a$\/ & ${\bfm F}\/ + a = a$\/ \\
			Absorbant&${\bfm F}\/ \cdot a = {\bfm F}\/$\/&${\bfm V}\/ \cdot a = {\bfm V}\/$ \\
			Associativité&$(a\cdot b)\cdot c = a\cdot (b\cdot c)$\/ &$a+(b+c)=(a+b)+c$\/ \\
			Idempotence&$a\cdot a = a$\/ & $a + a = a$\/ \\
			Distributivité&$a\cdot (b+c) = a\cdot b + a\cdot c$\/ & $a + (b\cdot c) = (a+b)\cdot (a+c)$\/ \\
			Complémentaire&$a\cdot \bar{a} = {\bfm F}\/$\/&$a+\bar{a} = {\bfm V}\/ $\/ \\
			{\sc Morgan}\/&$\overline{a\cdot b} = \bar{a}+ \bar{b}$\/&$\overline{a+b} = \bar{a}\cdot \bar{b}$\/
		\end{tabular}
		\caption{Règles dans $\mathds{B}$}
	\end{table}
\end{rmk}

\subsection{Fonctions booléennes}

\begin{defn}[Environnement propositionnel]
	On appelle {\it environnement propositionnel}\/ une fonction de $\mathcal{P}$\/ dans $\mathds{B}$.
	\index{environment propositionnel}
\end{defn}

\begin{defn}
	On appelle {\it fonction booléenne}\/ une fonction de $\mathds{B}^{\mathcal{P}}$\/ dans $\mathds{B}$. On note l'ensemble des fonctions booléennes $\mathds{F}$.
	\index{fonction booléenne}
\end{defn}

\begin{rmk}
	Si $|\mathcal{P}| = n$, alors $\left| \mathds{B}^{\mathcal{P}} \right| = 2^n$\/ et donc $|\mathds{F}| = 2^{2^n}$.
\end{rmk}

\begin{exm}
	La fonction \[
		f : \begin{pmatrix}
			(p\mapsto {\bfm F},\,q \mapsto {\bfm F}\/)\mapsto {\bfm F}\/\\
			(p\mapsto {\bfm F},\,q \mapsto {\bfm V}\/)\mapsto {\bfm V}\/\\
			(p\mapsto {\bfm V},\,q \mapsto {\bfm F}\/)\mapsto {\bfm V}\/\\
			(p\mapsto {\bfm V},\,q \mapsto {\bfm V}\/)\mapsto {\bfm V}\/\\
		\end{pmatrix} \in \mathds{F}
	\] est une fonction booléenne.
\end{exm}

\subsection{Interprétation d'une formule comme une fonction booléenne}

\begin{defn}[Interprétation]
	Étant donné une formule $G \in \mathcal{F}$ et un environnement propositionnel $\rho \in \mathds{B}^{\mathcal{P}}$, on définit l'{\it interprétation}\/ de $G$\/ dans l'environnement $\rho$\/ par
	\begin{itemize}
		\item $\left\llbracket \top \right\rrbracket^{\rho} = {\bfm V}\/$\/ ;
		\item $\left\llbracket \bot \right\rrbracket^{\rho} = {\bfm F}\/$\/ ;
		\item $\left\llbracket p \right\rrbracket^{\rho} = \rho(p)$\/ où $p \in \mathcal{P}$\/ ;
		\item $\left\llbracket \lnot G \right\rrbracket ^{\rho} = \overline{\left\llbracket G \right\rrbracket^{\rho}}$\/ ;
		\item $\left\llbracket G \land H \right\rrbracket ^{\rho} = \left\llbracket G \right\rrbracket ^{\rho} \cdot \left\llbracket H \right\rrbracket ^{\rho}$\/ ;
		\item $\left\llbracket G \lor H \right\rrbracket^{\rho} = \left\llbracket G \right\rrbracket ^{\rho} + \left\llbracket H \right\rrbracket ^{\rho}$\/ ;
		\item $\left\llbracket G \to H \right\rrbracket ^{\rho} = \overline{\left\llbracket G \right\rrbracket^{\rho}} + \left\llbracket H \right\rrbracket ^{\rho}$\/ ;
		\item $\left\llbracket G \leftrightarrow H \right\rrbracket ^{\rho} = \left( \overline{\left\llbracket G \right\rrbracket^\rho} + \left\llbracket H \right\rrbracket ^\rho \right)  \cdot \left( \overline{\left\llbracket H \right\rrbracket^\rho} + \left\llbracket G \right\rrbracket^\rho \right)$.
	\end{itemize}
	\index{formule!interprétation}
\end{defn}

\begin{exm}
	Avec $\rho = (p \mapsto {\bfm V}, q \mapsto {\bfm F}\/)$, et $G = (p \land \top) \lor (q \land \bot)$, on a
	\begin{align*}
		\left\llbracket G \right\rrbracket^\rho &= \left\llbracket (p \land \top) \lor (q \land\bot) \right\rrbracket^\rho \\
		&= \left\llbracket p \land \top \right\rrbracket^\rho + \left\llbracket q \land \bot \right\rrbracket^\rho \\
		&= \left\llbracket p \right\rrbracket^{\rho} \cdot \left\llbracket \top \right\rrbracket^{\rho} + \left\llbracket q \right\rrbracket^\rho \cdot\left\llbracket \bot \right\rrbracket^\rho \\
		&= \rho(p) \cdot {\bfm V}\/ + \rho(q) \cdot {\bfm F}\/ \\
		&= {\bfm V}\/ + {\bfm F}\/ \\
		&= {\bfm V}. \\
	\end{align*}
\end{exm}

\begin{defn}[Fonction booléenne associée à une formule]
	Étant donné une formule $G$, on note
	\begin{align*}
		\mathds{F} \ni \left\llbracket G \right\rrbracket : \mathds{B}^{\mathcal{P}} &\longrightarrow \mathds{B}  \\
		\rho &\longmapsto \left\llbracket G \right\rrbracket^{\rho}.
	\end{align*}
	\index{formule!fonction booléenne associée}
\end{defn}

\begin{exm}
	La fonction booléenne associée à $p \lor q$\/ est \[
		f : \begin{pmatrix}
			(p\mapsto {\bfm F},\,q \mapsto {\bfm F}\/)\mapsto {\bfm F}\/\\
			(p\mapsto {\bfm F},\,q \mapsto {\bfm V}\/)\mapsto {\bfm V}\/\\
			(p\mapsto {\bfm V},\,q \mapsto {\bfm F}\/)\mapsto {\bfm V}\/\\
			(p\mapsto {\bfm V},\,q \mapsto {\bfm V}\/)\mapsto {\bfm V}\/\\
		\end{pmatrix} \in \mathds{F}.
	\]
	La fonction booléenne associée à $p \lor (q \land \top)$\/ est aussi $f$ ; tout comme $(p \lor \bot) \lor (q \land \top)$.
\end{exm}

\subsection{Liens sémantiques}

\begin{defn}
	On dit que $G$\/ et $H$\/ sont {\it équivalents}\/ si et seulement si $\left\llbracket G \right\rrbracket = \left\llbracket H \right\rrbracket$.
	On note alors $G \equiv H$.
	\index{formule!équivalence}
\end{defn}

\begin{defn}[Conséquence sémantique]
	On dit que $H$\/ est {\it conséquence sémantique}\/ de $G$\/ dès lors que \[
		\forall \rho \in \mathds{B}^{\mathcal{P}},\:\big(\left\llbracket G \right\rrbracket^{\rho} = {\bfm V}\/ \big) \implies \big(\left\llbracket H \right\rrbracket^\rho = {\bfm V}\/ \big)
	.\]

	On le note $G \models H$. \index{formule!conséquence sémantique}
\end{defn}

\begin{prop}
	On a \[
		G \equiv H \iff \big(G \models H \mathrel{et} H \models G\big)
	.\]
\end{prop}

\begin{prv}
	\begin{itemize}
		\item[``$\implies$\/''] On suppose $G \equiv H$. Soit $\rho \in \mathds{B}^{\mathcal{P}}$. On suppose $\left\llbracket G \right\rrbracket^\rho = {\bfm V}\/ $\/ alors $\left\llbracket H \right\rrbracket^{\rho} = {\bfm V}\/$\/ car $\left\llbracket G \right\rrbracket = \left\llbracket H \right\rrbracket$.
			On suppose maintenant $\left\llbracket H \right\rrbracket^{\rho} = {\bfm V}\/$, et alors $\left\llbracket G \right\rrbracket^\rho = {\bfm V}\/$\/ car $\left\llbracket G \right\rrbracket = \left\llbracket H \right\rrbracket$.
		\item[``$\impliedby$\/'']  On suppose $G \models H$\/ et $H \models G$. Soit $\rho \in \mathds{B}^{\mathcal{P}}$. On suppose $\left\llbracket G \right\rrbracket^{\rho} = {\bfm V}\/$\/ alors $\left\llbracket H \right\rrbracket^\rho = {\bfm V}\/$\/ car $H \models H$\/ et donc $\left\llbracket G \right\rrbracket = \left\llbracket H \right\rrbracket$. On suppose maintenant $\left\llbracket H \right\rrbracket^\rho = {\bfm V}\/$\/ alors $\left\llbracket G \right\rrbracket^\rho = {\bfm V}\/$\/ car $G \models H$. Par contraposée, si $\left\llbracket G \right\rrbracket^\rho = {\bfm F}\/$, alors $\left\llbracket H \right\rrbracket^\rho = {\bfm F}\/$. On en déduit que $\left\llbracket G \right\rrbracket = \left\llbracket H \right\rrbracket$.
	\end{itemize}
\end{prv}

\begin{rmk}
	$\models$\/ n'est pas une relation d'ordre.
\end{rmk}


