\documentclass[a4paper]{article}

\usepackage[margin=1in]{geometry}
\usepackage[utf8]{inputenc}
\usepackage[T1]{fontenc}
\usepackage{mathrsfs}
\usepackage{textcomp}
\usepackage[french]{babel}
\usepackage{amsmath}
\usepackage{amssymb}
\usepackage{cancel}
\usepackage{frcursive}
\usepackage[inline]{asymptote}
\usepackage{tikz}
\usepackage[european,straightvoltages,europeanresistors]{circuitikz}
\usepackage{tikz-cd}
\usepackage{tkz-tab}
\usepackage[b]{esvect}
\usepackage[framemethod=TikZ]{mdframed}
\usepackage{centernot}
\usepackage{diagbox}
\usepackage{dsfont}
\usepackage{fancyhdr}
\usepackage{float}
\usepackage{graphicx}
\usepackage{listings}
\usepackage{multicol}
\usepackage{nicematrix}
\usepackage{pdflscape}
\usepackage{stmaryrd}
\usepackage{xfrac}
\usepackage{hep-math-font}
\usepackage{amsthm}
\usepackage{thmtools}
\usepackage{indentfirst}
\usepackage[framemethod=TikZ]{mdframed}
\usepackage{accents}
\usepackage{soulutf8}
\usepackage{mathtools}
\usepackage{bodegraph}
\usepackage{slashbox}
\usepackage{enumitem}
\usepackage{calligra}
\usepackage{cinzel}
\usepackage{BOONDOX-calo}

% Tikz
\usetikzlibrary{babel}
\usetikzlibrary{positioning}
\usetikzlibrary{calc}

% global settings
\frenchspacing
\reversemarginpar
\setuldepth{a}

%\everymath{\displaystyle}

\frenchbsetup{StandardLists=true}

\def\asydir{asy}

%\sisetup{exponent-product=\cdot,output-decimal-marker={,},separate-uncertainty,range-phrase=\;à\;,locale=FR}

\setlength{\parskip}{1em}

\theoremstyle{definition}

% Changing math
\let\emptyset\varnothing
\let\ge\geqslant
\let\le\leqslant
\let\preceq\preccurlyeq
\let\succeq\succcurlyeq
\let\ds\displaystyle
\let\ts\textstyle

\newcommand{\C}{\mathds{C}}
\newcommand{\R}{\mathds{R}}
\newcommand{\Z}{\mathds{Z}}
\newcommand{\N}{\mathds{N}}
\newcommand{\Q}{\mathds{Q}}

\renewcommand{\O}{\emptyset}

\newcommand\ubar[1]{\underaccent{\bar}{#1}}

\renewcommand\Re{\expandafter\mathfrak{Re}}
\renewcommand\Im{\expandafter\mathfrak{Im}}

\let\slantedpartial\partial
\DeclareRobustCommand{\partial}{\text{\rotatebox[origin=t]{20}{\scalebox{0.95}[1]{$\slantedpartial$}}}\hspace{-1pt}}

% merging two maths characters w/ \charfusion
\makeatletter
\def\moverlay{\mathpalette\mov@rlay}
\def\mov@rlay#1#2{\leavevmode\vtop{%
   \baselineskip\z@skip \lineskiplimit-\maxdimen
   \ialign{\hfil$\m@th#1##$\hfil\cr#2\crcr}}}
\newcommand{\charfusion}[3][\mathord]{
    #1{\ifx#1\mathop\vphantom{#2}\fi
        \mathpalette\mov@rlay{#2\cr#3}
      }
    \ifx#1\mathop\expandafter\displaylimits\fi}
\makeatother

% custom math commands
\newcommand{\T}{{\!\!\,\top}}
\newcommand{\avrt}[1]{\rotatebox{-90}{$#1$}}
\newcommand{\bigcupdot}{\charfusion[\mathop]{\bigcup}{\cdot}}
\newcommand{\cupdot}{\charfusion[\mathbin]{\cup}{\cdot}}
%\newcommand{\danger}{{\large\fontencoding{U}\fontfamily{futs}\selectfont\char 66\relax}\;}
\newcommand{\tendsto}[1]{\xrightarrow[#1]{}}
\newcommand{\vrt}[1]{\rotatebox{90}{$#1$}}
\newcommand{\tsup}[1]{\textsuperscript{\underline{#1}}}
\newcommand{\tsub}[1]{\textsubscript{#1}}

\renewcommand{\mod}[1]{~\left[ #1 \right]}
\renewcommand{\t}{{}^t\!}
\newcommand{\s}{\text{\calligra s}}

% custom units / constants
%\DeclareSIUnit{\litre}{\ell}
\let\hbar\hslash

% header / footer
\pagestyle{fancy}
\fancyhead{} \fancyfoot{}
\fancyfoot[C]{\thepage}

% fonts
\let\sc\scshape
\let\bf\bfseries
\let\it\itshape
\let\sl\slshape

% custom math operators
\let\th\relax
\let\det\relax
\DeclareMathOperator*{\codim}{codim}
\DeclareMathOperator*{\dom}{dom}
\DeclareMathOperator*{\gO}{O}
\DeclareMathOperator*{\po}{\text{\cursive o}}
\DeclareMathOperator*{\sgn}{sgn}
\DeclareMathOperator*{\simi}{\sim}
\DeclareMathOperator{\Arccos}{Arccos}
\DeclareMathOperator{\Arcsin}{Arcsin}
\DeclareMathOperator{\Arctan}{Arctan}
\DeclareMathOperator{\Argsh}{Argsh}
\DeclareMathOperator{\Arg}{Arg}
\DeclareMathOperator{\Aut}{Aut}
\DeclareMathOperator{\Card}{Card}
\DeclareMathOperator{\Cl}{\mathcal{C}\!\ell}
\DeclareMathOperator{\Cov}{Cov}
\DeclareMathOperator{\Ker}{Ker}
\DeclareMathOperator{\Mat}{Mat}
\DeclareMathOperator{\PGCD}{PGCD}
\DeclareMathOperator{\PPCM}{PPCM}
\DeclareMathOperator{\Supp}{Supp}
\DeclareMathOperator{\Vect}{Vect}
\DeclareMathOperator{\argmax}{argmax}
\DeclareMathOperator{\argmin}{argmin}
\DeclareMathOperator{\ch}{ch}
\DeclareMathOperator{\com}{com}
\DeclareMathOperator{\cotan}{cotan}
\DeclareMathOperator{\det}{det}
\DeclareMathOperator{\id}{id}
\DeclareMathOperator{\rg}{rg}
\DeclareMathOperator{\rk}{rk}
\DeclareMathOperator{\sh}{sh}
\DeclareMathOperator{\th}{th}
\DeclareMathOperator{\tr}{tr}

% colors and page style
\definecolor{truewhite}{HTML}{ffffff}
\definecolor{white}{HTML}{faf4ed}
\definecolor{trueblack}{HTML}{000000}
\definecolor{black}{HTML}{575279}
\definecolor{mauve}{HTML}{907aa9}
\definecolor{blue}{HTML}{286983}
\definecolor{red}{HTML}{d7827e}
\definecolor{yellow}{HTML}{ea9d34}
\definecolor{gray}{HTML}{9893a5}
\definecolor{grey}{HTML}{9893a5}
\definecolor{green}{HTML}{a0d971}

\pagecolor{white}
\color{black}

\begin{asydef}
	settings.prc = false;
	settings.render=0;

	white = rgb("faf4ed");
	black = rgb("575279");
	blue = rgb("286983");
	red = rgb("d7827e");
	yellow = rgb("f6c177");
	orange = rgb("ea9d34");
	gray = rgb("9893a5");
	grey = rgb("9893a5");
	deepcyan = rgb("56949f");
	pink = rgb("b4637a");
	magenta = rgb("eb6f92");
	green = rgb("a0d971");
	purple = rgb("907aa9");

	defaultpen(black + fontsize(8pt));

	import three;
	currentlight = nolight;
\end{asydef}

% theorems, proofs, ...

\mdfsetup{skipabove=1em,skipbelow=1em, innertopmargin=6pt, innerbottommargin=6pt,}

\declaretheoremstyle[
	headfont=\normalfont\itshape,
	numbered=no,
	postheadspace=\newline,
	headpunct={:},
	qed=\qedsymbol]{demstyle}

\declaretheorem[style=demstyle, name=Démonstration]{dem}

\newcommand\veczero{\kern-1.2pt\vec{\kern1.2pt 0}} % \vec{0} looks weird since the `0' isn't italicized

\makeatletter
\renewcommand{\title}[2]{
	\AtBeginDocument{
		\begin{titlepage}
			\begin{center}
				\vspace{10cm}
				{\Large \sc Chapitre #1}\\
				\vspace{1cm}
				{\Huge \calligra #2}\\
				\vfill
				Hugo {\sc Salou} MPI${}^{\star}$\\
				{\small Dernière mise à jour le \@date }
			\end{center}
		\end{titlepage}
	}
}

\newcommand{\titletp}[4]{
	\AtBeginDocument{
		\begin{titlepage}
			\begin{center}
				\vspace{10cm}
				{\Large \sc tp #1}\\
				\vspace{1cm}
				{\Huge \textsc{\textit{#2}}}\\
				\vfill
				{#3}\textit{MPI}${}^{\star}$\\
			\end{center}
		\end{titlepage}
	}
	\fancyfoot{}\fancyhead{}
	\fancyfoot[R]{#4 \textit{MPI}${}^{\star}$}
	\fancyhead[C]{{\sc tp #1} : #2}
	\fancyhead[R]{\thepage}
}

\newcommand{\titletd}[2]{
	\AtBeginDocument{
		\begin{titlepage}
			\begin{center}
				\vspace{10cm}
				{\Large \sc td #1}\\
				\vspace{1cm}
				{\Huge \calligra #2}\\
				\vfill
				Hugo {\sc Salou} MPI${}^{\star}$\\
				{\small Dernière mise à jour le \@date }
			\end{center}
		\end{titlepage}
	}
}
\makeatother

\newcommand{\sign}{
	\null
	\vfill
	\begin{center}
		{
			\fontfamily{ccr}\selectfont
			\textit{\textbf{\.{\"i}}}
		}
	\end{center}
	\vfill
	\null
}

\renewcommand{\thefootnote}{\emph{\alph{footnote}}}

% figure support
\usepackage{import}
\usepackage{xifthen}
\pdfminorversion=7
\usepackage{pdfpages}
\usepackage{transparent}
\newcommand{\incfig}[1]{%
	\def\svgwidth{\columnwidth}
	\import{./figures/}{#1.pdf_tex}
}

\pdfsuppresswarningpagegroup=1
\ctikzset{tripoles/european not symbol=circle}

\newcommand{\missingpart}{{\large\color{red} Il manque quelque chose ici\ldots}}


\titleanx{B}{Algorithmes {\sc Dijkstra} et $A^*$}

\begin{document} %% EXACT
	On s'intéresse, dans cette annexe, à l'algorithme $A^*$.
	Cette annexe se situe à l'intersection des chapitres sur les graphes, et sur les jeux.
	L'algorithme $A^*$ est une modification de l'algorithme de \textsc{Dijkstra}.
	Dans cette annexe, on prouvera la correction de l'algorithme $A^*$.

	On se place dans le contexte d'exécution d'un algorithme de calcul de plus cours chemin utilisant un tableau de distances $\mu$, et le manipulant en n'effectuant que des opérations \textsc{Relâcher}.
	Notons le graphe $G = (V,E)$, le sommet source $s$.
	Notons également $d(\cdot,\cdot)$ la distance induite par les arêtes du graphe $G$.
	De plus, on notera $c(\cdot,\cdot)$ les coûts (positifs, non nuls) d'une arête de $G$.
	Notons $\ell(\cdot)$ les rongeurs des chemins.

	\begin{numlem}
		\[
			\forall (u,v) \in E,\quad d(s,v) \le d(s, u) + c(u,v)
		.\]
	\end{numlem}

	\begin{prv}
		Soit $(u,v) \in E$.
		Soit $\gamma_u$ un plus court chemin de $s$ à $u$.
		Alors, $\gamma_u \cdot v$ est un chemin de $s$ à $v$ :
		\[
			\ell(\gamma_u \cdot v) = \ell(\gamma_u) + c(u,v) = d(s,u) + c(u,v) \ge d(s,v).
		\]
	\end{prv}

	\begin{numlem}
		Pour tout sommet $u$, la valeur de $\mu[u]$ est décroissant à mesure que l'algorithme s'exécute.
	\end{numlem}

	\begin{prv}
		Soit $\ubar{\mu}$ et $\bar\mu$ les valeurs de $\mu$ avant et après une opération $\textsc{Relâcher}(x,y)$.
		Pour tout sommet $v \neq y$, $\bar\mu[v] = \ubar\mu[v]$.
		De plus, par disjonction de cas,
		\begin{itemize}
			\item ou bien $\bar\mu[y] = \ubar\mu[y]$, \textsc{ok}.
			\item ou bien $\bar\mu[y] = \ubar\mu[x] + c(x,y)$ lorsque $\ubar\mu[x] + c(x,y) \le \ubar\mu[y]$, donc $\bar\mu[y] \le \ubar\mu[y]$, \textsc{ok}.
		\end{itemize}
	\end{prv}

	\begin{numlem}
		Supposons que l'algorithme ait initialisé $\mu$ de la manière suivante : \[
			\forall u \in V,\quad\quad \mu[u] = \begin{cases}
				+\infty & \text{ si } u \neq s\\
				0 & \text{ sinon}.
			\end{cases}
		\] Alors, tout au long de l'exécution de l'algorithme, pour tout sommet $u$, $\mu[u] \ge d(s,u)$.
	\end{numlem}

	\begin{prv}
		\begin{description}
			\item[Initialement] La propriété est vraie par hypothèse.
			\item[Hérédité] Supposons vrai jusqu'à un certain état $\ubar\mu$, pour une opération $\textsc{Relâcher}(x,y)$.
				Pour tout sommet $v \neq y$, $\ubar\mu[v] = \bar\mu[v] \ge d(s,v)$.
				De plus, par disjonction de cas,
				\begin{itemize}
					\item si $\bar\mu[y] = \ubar\mu[y] \ge d(s,y)$ ;
					\item sinon si $\ubar\mu[y] = \ubar\mu[x] + c(x,y) \ge d(s,x) + c(x,y) \ge d(s,y)$ par hypothèse de récurrence, puis par lemme 1.
				\end{itemize}
		\end{description}
	\end{prv}

	\begin{crlr}
		Si \guillemotleft~à un moment~\guillemotright\ $\mu[u] = d(s,u)$, alors \guillemotleft~pour toujours après~\guillemotright\ $\mu[u] = d(s,u)$.
		\qed
	\end{crlr}

	\begin{numlem}
		Si $(s, \ldots, u, v)$ est un plus court chemin de $s$ à $v$ tel que $\ubar\mu[u] = d(s,u)$ \guillemotleft~à un certain moment de l'exécution de l'algorithme.~\guillemotright\@ Notons $\bar\mu$ obtenu par $\textsc{Relâcher}(u,v)$.
	\end{numlem}

	\begin{prv}
		On a \[
			\bar\mu = \begin{cases}
				\ubar\mu[v] & \text{ si } \ubar\mu[v] < \ubar\mu[u] + c(u,v)\\
				\ubar\mu[u] + c(u,v) &\text{ sinon}.
			\end{cases}
		\]Par disjonction de cas,
		\begin{itemize}
			\item si $\ubar\mu[v] < \ubar\mu[u] + c(u,v) = d(s, u) + c(u,v) = d(s,v)$, et donc, en utilisant le lemme 3, $\bar\mu[v] = \ubar\mu[v] = d(s,v)$.
			\item sinon, $\bar\mu[v] = \ubar\mu[u] + c(u,v) = d(u,v) + c(u, v) = d(s,v)$.
		\end{itemize}
	\end{prv}

	\begin{numlem}
		Soit $(s=x_0, x_1, x_2, \ldots, x_n)$ un plus court chemin. Si on effectue des opérations \hbox{$\textsc{Relâcher}(x_i, x_{i+1})$} dans l'ordre $0 \to n - 1$, possiblement entremêlés avec d'autres opérations $\textsc{Relâcher}$, alors pour tout $i \in \llbracket 0,n \rrbracket$, $\mu_{\text{final}}[x_i] = d(s, x_i)$.
	\end{numlem}

	\begin{prv}[par récurrence]
		\begin{itemize}
			\item Initialement, $\mu[x_0] = d(s, x_0) = d(s,s)$.
			\item Et, pour tout les $i$ inférieurs stricts, $\mu[x_i] = d(s, x_i)$, on conclut par le lemme 4.
		\end{itemize}
	\end{prv}

	(De ce lemme découle l'algorithme de \textsc{Bellman-Ford}.)

	\begin{crlr}
		L'algorithme \textsc{Dijkstra} est correct.
	\end{crlr}

	\begin{prv}
		Soit $t \in V$, un sommet du graphe. Soit $(s = x_0, x_1, \ldots, x_{p-1}, x_p = t)$ un plus court chemin de $s$ à $t$. Montrons que $\mu_{\text{final}}[t] = d(s, t)$.
		En utilisant le lemme 5, il suffit de montrer que \textsc{Dijkstra} relâche les arêtes dans cet ordre.
		Supposons les sommets extraits $\mathrm{todo}$ dans l'ordre $x_0, \ldots, x_i$, pour $i \in \llbracket 0,p-1 \rrbracket$.
		Par l'absurde, supposons que \textsc{Dijkstra} sorte $x_k$ de $\mathrm{todo}$ pour $k \in \llbracket i+2, p \rrbracket$.
		\guillemotleft~À ce moment là,~\guillemotright\ on a \[
			d(s, x_k) \le \mu[x_k] \le \mu[x_{i+1}] \le d(s, x_{i+1}),
		\]d'après le lemme 5, ce qui est absurde ($k > i + 1$).
	\end{prv}

	\begin{crlr}
		L'algorithme $A^*$ est correct.
	\end{crlr}

	\begin{algorithm}[H]
		\centering
		\begin{algorithmic}[1]
			\Procedure{Relâcher}{$u,v$}
			\If{$\mu[v] > \mu[u] + c(u,v)$}
			\State $\mu[v] \gets \mu[u] + c(u,v)$
			\State $\pi[v] \gets u$
			\State $\eta[v] \gets \mu[v] + h(v)$
			\EndIf
			\EndProcedure
		\end{algorithmic}
		\caption{Algorithme $A^*$ (partiel)}
	\end{algorithm}

	\begin{prv}
		Par l'absurde, supposons que non.
		Soit $t \in V$, un sommet du graphe, tel que $\mu_{\text{final}}[t] \neq d(s,t)$.
		Donc $d = \mu_{\text{final}}[t] > d(s,t) = d^*$.
		Soit $(s = x_0, x_1, \ldots, x_{p-1}, x_p = t)$ un plus court chemin de $s$ à $t$ de longueur $d^*$.
		L'algorithme commence par visiter $x_0 = s$ et on relâche les arêtes sortantes.
		Alors, $\mu[x_i] = d(s, x_1)$ et $\eta[x_1] = \mu[x_1] + \mu[x_1] + h(x_1) \le d(s, x_1) + d(x_1, t) = d(s,t) = d^* < d$ par hypothèse.
		\guillemotleft~À ce state,~\guillemotright\ $\eta[t] = \mu[t] + h(t) \ge d + 0$.
		Ainsi, $x_1$ devrait être choisi avant $t$. À un tel moment, $\mu[x_1] = d(s, x_1)$, on relâche alors ses arêtes sortantes ; en particulier $x_1$ et $x_2$. Ceci assure alors que $\mu[x_2] = d(s, x_2)$, et $\eta[x_2] = \mu[x_2] + h(x_2) \le d(sn x_2) + d(x_2, t) \le d(s,t) = d^* < d$.
		\guillemotleft~De proche en proche,~\guillemotright\ alors que l'on choisit $x_{p-1}$ dans $\mathrm{todo}$, on a $\mu[x_{p-1}] = d(s, x_{p-1})$.
		On relâche alors $\mu[x_p] = d(s, x_p) = d^*$.
		Or, $d = \mu_{\text{final}}[t] \le \mu_{\text{à ce moment}}[t]$. Absurde.
	\end{prv}

	\begin{exm}[ré-entrée dans $\mathrm{todo}$]
		\begin{comment}
			     b (h = 6)
					/ \
			 1 /   \ 1
				/  3  \     5
			 s - - - a - - - - t
		        (h = 0)   (h = 0)
		\end{comment}
		Exécution de l'algorithme $A^*$ sur l'entrée ci-dessus.
		La pile $\mathrm{todo}$ est vaut donc $\cancel s, \cancel a, \cancel b, \cancel t, \cancel a$.
	\end{exm}
\end{document}
