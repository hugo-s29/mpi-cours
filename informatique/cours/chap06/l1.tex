L'objectif de ce chapitre, sera de ``critiquer'' le travail en logique fait précédement, puis d'apporter une solution à ce problème ; on finira par un peu de \textsc{Hors-Programme}.

\setcounter{section}{-1}

\section{Motivation}

\subsection{Tables de vérité}

Pour l'instant, pour montrer $\Gamma \models G$\/ ou $G \equiv H$, nous devons encore utiliser une table de vérité.
Par exemple, montrons \[
	\underbrace{(p \to q) \land (q \to r)}_G \models \overbrace{p\to r}^H
.\]
On réalise la table de vérité ci-dessous.

\begin{table}[H]
	\centering
	\[
		\begin{array}{c|c|c|c|c|c|cl}
			p&q&r&p\to q&q\to r&G&H\\ \hline
			\mathbf{F}&\mathbf{F}&\mathbf{F}&\mathbf{V}&\mathbf{V}&\mathbf{V}&\mathbf{V}&\gtk\\
			\mathbf{F}&\mathbf{F}&\mathbf{V}&\mathbf{V}&\mathbf{V}&\mathbf{V}&\mathbf{V}&\gtk\\
			\mathbf{F}&\mathbf{V}&\mathbf{F}&\mathbf{V}&\mathbf{F}&\mathbf{F}&\mathbf{V}&\gtk\\
			%\mathbf{F}&\mathbf{V}&\mathbf{V}&\mathbf{V}&\mathbf{V}&
			%\mathbf{V}&\mathbf{F}&\mathbf{F}&\mathbf{F}&\mathbf{F}&\mathbf{V}&\mathbf{F}&\mathbf{F}
		\end{array}
	\] \todo{Finir table de vérité}
	\caption{Table de vérité pour montrer $(p \to q) \land (q \to r) \models p\to r$\/}
\end{table}

\subsection{Équations}
Supposons $\llbracket G \rrbracket^\rho = \mathbf{V}$. Montrons que $\llbracket H \rrbracket^\rho = \mathbf{V}$.
On a
\begin{align*}
	\mathbf{V} &= \llbracket G \rrbracket^\rho\\
	&= \llbracket (p\to q) \land(q \to r) \rrbracket^\rho \\
	&\vdots \\
	&= \overline{\llbracket p \rrbracket^\rho} \cdot \overline{\llbracket q \rrbracket^\rho} + \overline{\llbracket p \rrbracket^\rho} \cdot \overline{\llbracket q \rrbracket^\rho} \cdot \llbracket r \rrbracket^\rho + \llbracket q \rrbracket^\rho \cdot \llbracket r \rrbracket^\rho \\
\end{align*}
et
\begin{align*}
	\llbracket H \rrbracket^\rho &= \llbracket p\to r \rrbracket^\rho \\
	&\vdots \\
	&= \overline{\llbracket p \rrbracket^\rho} \cdot \overline{\llbracket q \rrbracket^\rho}
	+ \overline{\llbracket p \rrbracket^\rho} \cdot \overline{\llbracket q \rrbracket^\rho}\\
	&\quad+ \llbracket r \rrbracket^\rho \cdot \overline{\llbracket q \rrbracket^\rho} \cdot \left( \llbracket p \rrbracket^\rho + \overline{\llbracket p \rrbracket^\rho} \right) \\
	&\quad + \llbracket r \rrbracket^\rho + ?
\end{align*}
\todo{finir le calcul}

\subsection{Raisonnement mathématiques}

Supposons $(p \to q) \land (q \to r)$. Montrons que $p \to r$.\\
\begin{itemize}
	\item[$\hookrightarrow$] Supposons donc $p$. Montrons $r$\/
		\begin{itemize}
			\item[$\hookrightarrow$] Montrons $q$.
				\begin{itemize}
					\item[$\hookrightarrow$] Montrons $p$, qui est une hypothèse.
					\item[$\hookrightarrow$] Montrons $p\to q$, qui est aussi une hypothèse.
				\end{itemize}
			\item Montrons $q \to r$, ce qui est vrai par hypothèse.
		\end{itemize}
\end{itemize}

On reconnaît un arbre.

\section{La déduction naturelle en logique propositionnelle}
\subsection{Séquents}

\paragraph{Objectifs de preuves.}

\begin{exm}
	Montrons $P \land Q$\/ est vrai.\\
	$\hookrightarrow$ Montrons $P$. \\
	$\hookrightarrow$ Montrons $Q$. \\
\end{exm}

\paragraph{Hypothèses courantes.}

\begin{exm}
	Montrons que $(n \in 4\N \to n \in 2\N) \land (n \in 4\N + 3 \to n \in 2\N + 1)$.\\
	\begin{itemize}
		\item[$\hookrightarrow$] Montrons $n \in 4\N \to 2\N$.
			\begin{itemize}
				\item[$\hookrightarrow$] $\underset{\text{\color{red} local}}{\text{\color{red}\underline{\color{black}Supposons $n \in 4 \N$}}}$. Montrons $n \in 2\N$.
			\end{itemize}
		\item[$\hookrightarrow$] Montrons $n \in 4\N + 3 \to  n \in 2\N + 1$.
		\begin{itemize}
			\item[$\hookrightarrow$] Supposons $n \in 4\N + 3$. Montrons $n \in 2\N+1$.
		\end{itemize}
	\end{itemize}
\end{exm}

\begin{defn}[Séquent]
	Un \textit{séquent} est la donnée 
	\begin{itemize}
		\item d'un ensemble d'hypothèses $\Gamma$\/ ;
		\item d'un objectif $G$.
	\end{itemize}
	On le typographie $\Gamma \vdash G$.

	\index{séquent}
\end{defn}

\begin{exm}
	Montrons $\O \vdash (n \in 4\N \to n \in 2\N) \land (n \in 4\N + 3 \to n \in 2\N +1)$\/
	\begin{itemize}
		\item[$\hookrightarrow$] $\O \vdash (n \in 4\N \to n \in 2\N)$\/
			\begin{itemize}
				\item[$\hookrightarrow$] $\{n \in 4\N\} \vdash n \in 2\N$\/
			\end{itemize}
		\item[$\hookrightarrow$] $\O \vdash (n \in 4\N + 3 \to n \in 2\N +1)$\/
			\begin{itemize}
				\item[$\hookrightarrow$] $\{n \in 4\N + 3\} \to n \in 2\N +1$\/
			\end{itemize}
	\end{itemize}

	On typographie cette preuve sous forme d'un arbre. \todo{Arbre à faire}
\end{exm}


