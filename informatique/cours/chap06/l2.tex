\subsection{Preuves}

\begin{defn}
	On appelle \textit{règle de construction de preuves} une règle de la forme : \[
		\begin{prooftree}
			\hypo{\Gamma_1 \vdash \varphi_1}
			\hypo{\Gamma_2 \vdash \varphi_2}
			\hypo{\Gamma_3 \vdash \varphi_3} % NOCHECK
			\hypo{\cdots}
			\hypo{\Gamma_n \vdash \varphi_n}
			\infer5[nom]{\Gamma \vdash \varphi}
		\end{prooftree}
	.\]
	On appelle $\Gamma_1 \vdash \varphi_1$, $\Gamma_2 \vdash \varphi_2$, $\Gamma_3 \vdash \varphi_3$, \ldots, $\Gamma_n \vdash \varphi_n$\/ les \textit{prémisses}, et $\Gamma \vdash \varphi$\/ la \textit{conclusion}.

	Si $n = 0$, on dit que c'est une \textit{règle de base}.
	
	\index{prémisses}
	\index{conclusion}
	\index{règle de construction de preuves}
	\index{règle de base}
\end{defn}

\begin{rmk}[Notation]
	$\Gamma$\/ est un ensemble. Alors, l'ensemble $\Gamma \cup \{\psi\}$\/ est noté $\Gamma,\psi$.
\end{rmk}

\begin{exm}
	Un \textit{axiome} est de la forme \[
		\begin{prooftree}
			\infer 0[Ax]{\Gamma, \varphi \vdash \varphi} % NOCHECK
		\end{prooftree}
	.\]
	\index{axiome}

	Une preuve de la forme, appelée \textit{introduction du \textsc{et}}, \[
		\begin{prooftree}
			\hypo{\Gamma \vdash\varphi} % NOCHECK
			\hypo{\Gamma \vdash \psi}
			\infer 2[$\land$i]{\Gamma \vdash \varphi \land \psi}
		\end{prooftree}
	\] permet de prouver un \textsc{et}. Il correspond au raisonnement mathématique suivant : supposons $\Gamma$ ; montrons $\varphi \land \psi$\/ ;
	\begin{itemize}
		\item[$\hookrightarrow$] montrons $\varphi$\/ ;
		\item[$\hookrightarrow$] montrons $\psi$.
	\end{itemize}
\end{exm}

\begin{defn}[Arbre de preuve]
	On appelle \textit{arbre de preuve} un arbre étiqueté par des séquents, et dont les liens père--fils sont des liens autorisés par les règles du système de preuves. Un \textit{système de preuves} étant un ensemble de règles.
	\index{arbre de preuve}
	\index{système de preuves}
\end{defn}

\begin{exm}[Système Jouet]
	\[
		\begin{prooftree}
			\infer 0[Ax]{\Gamma, \varphi \vdash \varphi} % NOCHECK
		\end{prooftree}
		\begin{prooftree}
			\hypo{\Gamma \vdash \varphi}
			\hypo{\Gamma \vdash \psi} % NOCHECK
			\infer 2[$\land$i]{\Gamma \vdash \varphi \land \psi}
		\end{prooftree}
		\begin{prooftree}
			\hypo{\Gamma \vdash \varphi} % NOCHECK
			\infer 1[$\lor$i,g]{\Gamma \vdash \varphi \lor\psi}
		\end{prooftree}
		\begin{prooftree}
			\hypo{\Gamma \vdash \psi} % NOCHECK
			\infer 1[$\lor$i,d]{\Gamma \vdash \varphi \lor\psi}
		\end{prooftree}
	.\] 
\end{exm}

\begin{exm}
	Avec le système précédent, \[
		\begin{prooftree}
			\infer0[Ax]{\{P,Q,R\} \vdash P}
			\infer1[$\lor$i,g]{\{P,Q,R\} \vdash P \lor Q}
			\infer0[Ax]{\{P,Q,R\} \vdash Q}
			\infer1[$\lor$i,g]{\{P,Q,R\} \vdash Q \lor\lnot R}
			\infer2[$\land$i]{\{P,Q,R\} \vdash (P \lor Q) \land(Q \lor\lnot R)}
			\infer1[$\lor$i,d]{\{P,Q,R\} \vdash(P \land X) \lor\big((P\lor Q) \land (Q \lor\lnot R)\big)}
		\end{prooftree}
	.\]
\end{exm}

\begin{defn}[Être prouvable]
	On dit d'un séquent $\Gamma \vdash G$\/ qu'il est \textit{prouvable} dans un système de preuve dès lors qu'il existe une preuve dont la racine est étiquetée par $\Gamma \vdash G$.
	\index{être prouvable}
\end{defn}

\begin{rap}[objectifs]
	On veut trouver d'autres moyens de montrer $F \models G$.
	On veut que, si $F \vdash G$, alors $F \models G$ (correction).
	Mais, on veut aussi que, si $F \models G$, alors $G \vdash F$ (complétude).
	On veut aussi qu'il existe un algorithme qui vérifie $F \models G$ (décidabilité).
\end{rap}

\begin{defn}[Correction]
	On dit d'un système de preuve qu'il est \textit{correct} dès lors que : 
	pour tout $\Gamma$, pour tout $G$, si $\Gamma \vdash G$\/ admet une preuve, alors $\Gamma \models G$.
	\index{système de preuves!correction}
\end{defn}

\begin{exm}
	\begin{enumerate}
		\item On pose la règle ``Menteur'' définie comme \[
				\begin{prooftree}
					\infer0[Menteur]{\Gamma\vdash \bot} % NOCHECK
				\end{prooftree}
			.\] Ce système de preuve n'est pas correct car $\{\top\} \not\models \bot$. Or, \[
				\begin{prooftree}
					\infer0[Menteur]{\{\top\} \vdash\bot} % NOCHECK
				\end{prooftree}
			.\]
		\item Le système jouet est correct. Montrons cela par induction sur la preuve de $\Gamma \vdash G$.
			\begin{itemize}
				\item Si la preuve de $\Gamma \vdash G$\/ est de la forme \[
						\begin{prooftree}
							\infer0[Ax]{\Gamma',G \vdash G} % NOCHECK
						\end{prooftree}
					.\] Montrons que $\Gamma' \cup \{G\} \models \{G\}$.
					Soit donc $\rho$\/ un modèle de $\Gamma \cup \{G\}$. Alors $\forall \varphi \in \Gamma \cup \{G\}$, $\llbracket \varphi \rrbracket^\rho = \mathbf{V}$. Montrons que $\rho$\/ est un modèle de $G$.
					On pose $\varphi = G$\/ ; on a donc $\llbracket G \rrbracket^\rho = \mathbf{V}$.
				\item Si la preuve de $\Gamma \vdash G$\/ est de la forme \[
						\begin{prooftree}
							\hypo{\Gamma \vdash \varphi} % NOCHECK
							\hypo{\Gamma \vdash \psi}
							\infer2[$\land$i]{\Gamma \vdash \varphi \land \psi}
						\end{prooftree}
					.\] On appelle $\pi_1$\/ la branche gauche de l'arbre, et $\pi_2$\/ la branche de droite.
					Par hypothèse d'induction sur $\pi_1$\/, $\Gamma \vdash \varphi$\/ admet une preuve (qui est une sous-preuve), donc $\Gamma\models \varphi$. De même, $\Gamma \models \psi$ avec la branche $\pi_2$.
					On en déduit que $\Gamma \models \varphi \land \psi$\/ d'après les résultats du chapitre 0.
				\item De même pour les autres cas.
			\end{itemize}
	\end{enumerate}
\end{exm}


\begin{defn}[Complétude]
	Un système de preuves est \textit{complet} dès lors que, si $\Gamma \models G$, alors il existe un preuve de $\Gamma \vdash G$.
	\index{système de preuves!complétude}
\end{defn}

\begin{exm}
	Le système de preuve ayant pour règle, pour tout $\Gamma$, et tout $G$, \[
		\begin{prooftree}
			\infer0[OP]{\Gamma \vdash G} % NOCHECK
		\end{prooftree}
	\] est complet mais pas correct.
\end{exm}

\begin{exm}
	Le système de preuve ayant pour règle, pour tout $\Gamma$, et tout $G$ tel que $\Gamma \models G$, \[
		\begin{prooftree}
			\infer0{\Gamma \vdash G} % NOCHECK
		\end{prooftree}
	\] est complet et correct.
\end{exm}

\subsection{Déduction naturelle}

On définit les différentes règles d'introduction et d'élimination suivantes.

\begin{table}[H]
	\centering
	\begin{tabular}{Sc|Sc|Sc}
		\hline
		\textsc{Symbole} & \textsc{Règle d'introduction} & \textsc{Règle d'élimination}\\ \hline \hline 
		$\top$\/&$\ptree{\infer0[$\top$i]{\Gamma \vdash \top}}$&\\ \hline
		$\bot$\/ &&$\ptree{\hypo{\Gamma\vdash\bot}\infer1[$\bot$e]{\Gamma \vdash G}}$\\ \hline
		$\lnot$\/ &$\ptree{\hypo{\Gamma,G\vdash \bot}\infer1[$\lnot$i]{\Gamma\vdash\lnot G}}$&$\ptree{\hypo{\Gamma\vdash G}\hypo{\Gamma\vdash \lnot G}\infer2[$\lnot$e]{\Gamma\vdash \bot}}$ \\ \hline
		$\to$ & $\ptree{\hypo{\Gamma,G \vdash H}\infer1[$\to$i]{\Gamma \vdash G \to H}}$ & $\ptree{\hypo{\Gamma \vdash H \to G}\hypo{\Gamma \vdash H}\infer2[$\to$e]{\Gamma \vdash G}}$\\ \hline
		$\land$\/ & $\ptree{\hypo{\Gamma \vdash G}\hypo{\Gamma \vdash H}\infer2[$\land$i]{\Gamma \vdash G \land H}}$\/ & $\ptree{\hypo{\Gamma \vdash G \land H}\infer1[$\land$e,g]{\Gamma \vdash G}}$ \quad\quad $\ptree{\hypo{\Gamma \vdash G \land H}\infer1[$\land$e,d]{\Gamma \vdash H}}$ \\ \hline
		$\lor$&$\ptree{\hypo{\Gamma\vdash G}\infer1[$\lor$i,g]{\Gamma\vdash G \lor H}}$\quad\quad$\ptree{\hypo{\Gamma\vdash H}\infer1[$\lor$i,d]{\Gamma\vdash G \lor H}}$&$\ptree{\hypo{\Gamma \vdash A \lor B}\hypo{\Gamma,A\vdash G}\hypo{\Gamma,B\vdash G}\infer3[$\lor$e]{\Gamma \vdash G}}$\\ \hline
		\multicolumn{3}{c}{}\\
		\multicolumn{3}{c}{$\ptree{\infer 0[Ax]{\Gamma, \varphi\vdash \varphi}}$}\\
		\multicolumn{3}{c}{}\\ \hline
	\end{tabular}
	\caption{Règles d'introduction et d'élimination}
\end{table}
