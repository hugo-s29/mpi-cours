\section{Arbres couvrants de poids minimum}

\begin{exm}
	On considère le graphe ci-dessous.
	\begin{figure}[H]
		\centering
		\tikzfig{ex-graphe-pondere}
		\caption{Arbre pondéré}
	\end{figure}
	On cherche à \guillemotleft~supprimer~\guillemotright\ des arrêtes de ce graphe afin d'avoir un poids total minimum, tout en conservant la connexité du graphe.
	Une structure assurant cette condition est un arbre.

	Pour résoudre ce problème, on part du graphe vide, et on ajoute les arrêtes les moins coûteuses en premier.
\end{exm}

\begin{defn}[Arbre]
	Soit $G = (S,A)$\/ un graphe non-orienté. On dit que $G$\/ est un \textit{arbre} si $G$\/ est connexe et acyclique.
	\index{arbre}
\end{defn}

\begin{defn}[Arbre couvrant]
	Étant donné un graphe non orienté pondéré par poids positifs $G = (S, A, c)$,\footnotemark\ on dit de $G' = (S', A')$\/ que c'est un \textit{arbre couvrant} de $G$\/ si $S' = S$\/ et $A' \subseteq A$, et $G'$\/ est un arbre.
	\index{arbre!couvrant}
\end{defn}
\footnotetext{on dit que $c$\/ est la fonction de pondération de ce graphe}

\begin{defn}[Arbre couvrant de poids minimum]
	Étant donné un graphe non orienté pondéré $G = (S, A, c)$\/ et un arbre couvrant $T = (S', A')$, on appelle \textit{poids} de l'arbre $T$\/ la valeur $\sum_{a \in A'} c(a)$.
	\index{arbre!couvrant!poids}

	Si $G$\/ est connexe, il admet au moins un arbre couvrant, on peut définir l'\textit{arbre couvrant de poids minimum} (\textit{\textsc{acpm}}).
	\index{arbre!couvrant!de poids minimum}
\end{defn}

On définir alors le problème \[
	\textsc{acpm}\text{\footnotemark}
	\begin{cases}
		\text{\textbf{Entrée}}&: G = (S, A, c) \text{ connexe}\\
		\text{\textbf{Sortie}}&: \text{ le poids de l'arbre couvrant de poids minimum}.
	\end{cases}
\]
\footnotetext{Arbre Couvrant de Poids Minimum}

\begin{algorithm}[H]
	\centering
	\begin{algorithmic}[1]
		\Entree $G = (S, A, c)$\/ un graphe connexe
		\Sortie Un arbre couvrant de poids minimum
		\State $B \gets \O$\/ 
		\State $U \gets \O$\/
		\While{il existe $u$\/ et $v$\/ tels que $u \nsim_B v$}
			\State Soit $\{x,y\} \in A \setminus U$\/ de poids minimal
			\If{$x \sim_B y$}
				\State $U \gets \big\{\!\{x,y\}\!\big\} \cup U$
			\Else
				\State $U \gets \big\{\!\{x,y\}\!\big\} \cup U$
				\State $B \gets \big\{\!\{x,y\}\!\big\} \cup B$
			\EndIf
		\EndWhile
		\State\Return $T = (S,B)$\/
	\end{algorithmic}
	\caption{Algorithme de \textsc{Kruskal}}
\end{algorithm}

\begin{prop}
	L'algorithme de \textsc{Kruskal} est correct.
\end{prop}

\begin{prv}
	\begin{enumerate}
		\item Il existe un arbre couvrant de poids minimum utilisant les arrêtes de $B$ ;
		\item $B \subseteq U \subseteq A$\/ ;
		\item $\forall \{u,v\} \in U$, $u \sim_B v$.
	\end{enumerate}
	Ces trois propriétés sont invariantes.
	\begin{description}
		\item[Initialement] $B = \O = U$, donc \textsc{ok}.
		\item[Propagation] Soient $\ubar{B}$\/ et $\ubar{U}$\/ (resp.\ $\bar{B}, \bar{U}$) les valeurs de $B$\/ et $U$\/ avant (resp.\ après) une itération de boucle. Supposons que $\ubar{B}$\/ et $\ubar{U}$\/ satisfont les propriétés 1, 2 et 3. Montrons que $\bar{B}$\/ et $\bar{U}$\/ les satisfont aussi.
			\begin{enumerate}
				\item[2.] On a $\{x,y\} \in A$\/ et $\ubar{B} \subseteq \ubar{U} \subseteq A$, donc \[
						\bar{B} \subseteq \ubar{B} \cup \{\!\{x,y\}\!\} \subseteq \ubar{U} \cup \{\!\{x,y\}\!\} \subseteq A
					.\]
				\item[3.] Soit $\{u,v\} \in \bar{U}$.
					\begin{itemize}
						\item Si $\{u,v\} \in \ubar{U}$, alors de 3, $u \sim_{\ubar{B}} v$. Or, $\ubar{B} \subseteq \bar{B}$\/ et donc $u \sim_{\bar{B}} v$.
						\item Sinon, $\{ u,v\} = \{x,y\}$, alors $x = u$\/ et $v = y$.
							\begin{itemize}
								\item Sous-cas 1 : $\bar{B} = \ubar{B} \cup \{\!\{x,y\}\!\}$, alors $x \sim_{\bar{B}} y$.
								\item Sous-cas 2 : $\bar{B} = \ubar{B}$, alors par condition du \textbf{si}, $x \sim_{\ubar{B}} y$\/ et donc $x \sim_{\bar{B}} y$.
							\end{itemize}
					\end{itemize}
				\item[1.]
					Soit $\mathcal{T}$\/ un \textsc{acpm} contenant $\ubar{B}$.
					\begin{itemize}
						\item Cas 1 : $\bar{B} = \ubar{B}$, \textsc{ok}
						\item Cas 2 : $\bar{B} = \ubar{B} \cup \{\!\{x,y\}\!\}$.
							\begin{itemize}
								\item Sous-cas 1 : $\{x,y\} \in \mathcal{T}$, alors $\mathcal{T}$\/ est un \textsc{acpm} qui contient $\bar{B}$.
								\item Sous-cas 2 : $\{x,y\}  \not\in \mathcal{T}$, $\mathcal{T}$\/ est un arbre couvrant, donc il contient une chaîne de $x$\/ à $y$\/ : \[
											\{\overset{\substack{x\\[-1mm]\vrt=}}{x_0},x_1\},\{x_1,x_2\},\ldots,\{x_{n-1},\underset{\substack{\vrt=\\y}}{x_n}\}
									.\]
									Or, $\forall i \in \llbracket 1,n-1 \rrbracket$, $x_i \sim_{\ubar{B}} x_{i+1}$. Par transitivité, on a donc $x = x_0 \sim_{\ubar{B}} x_n = y$, ce qui n'est pas le cas.
									Il existe donc $i_0 \in \llbracket 0,n-1 \rrbracket$, tel que $x_{i_0} \nsim_{\ubar{B}} x_{i_0 + 1}$\/ et donc $\{x_{i_0}, x_{i_0 + 1}\} \not\in \ubar{U}$. D'où, d'après 3, on a $\{x_{i_0}, x_{i_0 + 1}\}  \not\in \ubar{B}$
									Considérons alors $\mathcal{T}' = \big(\mathcal{T} \setminus \{\!\{x_{i_0},x_{i_0+1}\}\!\}\big)  \cup \{\!\{x,y\}\!\}$. Montrons que $\mathcal{T}'$\/ est un \textsc{acpm} contenant $B$, en commençant par montrer que c'est un arbre couvrant. L'arbre $\mathcal{T}'$\/ a $n-1$\/ arrêtes (autant que $\mathcal{T}$). Montrons que $\mathcal{T}'$\/ est connexe.
									Soit $(a,b) \in S^2$. $\mathcal{T}$\/ est connexe, soit donc une chaîne \[
										C : \quad a = u_0, u_1, \ldots, u_n = b
									\] de $\mathcal{T}$. Si la chaîne $C$\/ n'utilise pas l'arrête $\{x_{i_0},x_{i_0+1}\}$, alors $C$\/ est une chaîne de $\mathcal{T}'$. Sinon, on pose $\mu$\/ et $\tau$\/ tels que \[
									\underbrace{a,\ldots,x_{i_0}}_{\mu},\underbrace{x_{i_0+1},\ldots,b}_{\tau}
									.\]
									Soit alors la chaîne
									\begin{align*}
										\overbrace{a,\ldots,x_{i_0}}^{\mu},x_{i_0-1},x_{i_0-2},\ldots,x_0 = x,&\\
										\underbrace{b,\ldots,x_{i_0 + 1}}_{\tau},x_{i_0+2},\ldots,x_{n-1},x_n=y&
									\end{align*}
									qui est dans $\mathcal{T}'$.
									Montrons que le poids est minimum. Notons $P(\mathcal{T})$\/ le poids de l'arbre. On a donc \[
										P(\mathcal{T}') = P(\mathcal{T}) + c(\{x,y\}) - c(\{x_{i_0},x_{i_0+1}\})
									.\] Par choix glouton, ($\{x_{i_0}, x_{i_0+1} \not\in \ubar{U}\}$), $c(\{x,y\}) \le c(\{x_{i_0},x_{i_0+1}\})$\/ donc $P(\mathcal{T}') \le P(\mathcal{T})$, et $\mathcal{T}$\/ étant de poids min, $P(\mathcal{T}') = P(\mathcal{T})$\/ et $\mathcal{T}'$\/ est un \textsc{acpm} contenant $\bar{B}$.
							\end{itemize}
					\end{itemize}
			\end{enumerate}
	\end{description}

	Les invariants le sont.
\end{prv}

À la fin, $B$\/ induit un graphe connexe et $B$\/ est contenu dans un \textsc{acpm}, c'en est donc un.

