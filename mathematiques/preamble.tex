\usepackage[utf8]{inputenc}
\usepackage[OT2,OT1,T1]{fontenc}
\usepackage{textcomp}
\usepackage[bookmarks]{hyperref}
\usepackage[french]{babel}
\usepackage[fontsize=8pt]{fontsize}
\usepackage{amsmath, amssymb}
\usepackage{amsthm}
\usepackage{tikz}
\usepackage{mathtools}
\usepackage{tkz-tab}
\usepackage[inline]{asymptote}
\usepackage{frcursive}
\usepackage{verbatim}
\usepackage{moresize}
\usepackage{pifont}
\usepackage{xfrac}
\usepackage{thmtools}
\usepackage{diagbox}
\usepackage{centernot}
\usepackage{multicol}
\usepackage{nicematrix}
\usepackage{stmaryrd}
\usepackage{setspace}
\usepackage{cancel}
\usepackage[f]{esvect}
\usepackage{wrapfig}
\usepackage{floatflt}
\usepackage{dsfont}
\usepackage{subcaption}
\usepackage{pdflscape}
\usepackage{soulutf8}
\usepackage[Rejne]{fncychap}
\usepackage{listings}
\usepackage[framemethod=TikZ]{mdframed}
\usepackage[thinlines]{easytable}
\usepackage{yhmath}

\frenchspacing
\reversemarginpar

% better underline
\setuldepth{a}

\usetikzlibrary{babel}

\usepackage{graphicx}
\newcommand\longvdots[1]{\raisebox{1em}{\rotatebox{-90}{\hbox to #1 {\dotfill}}}}

\theoremstyle{definition}

\definecolor{green}{HTML}{60A917}
\def\asydir{asy}
\newcommand{\cwd}{.}

% figure support
\usepackage{import}
\usepackage{xifthen}
\pdfminorversion=7
\usepackage{pdfpages}
\usepackage{transparent}
\newcommand{\incfig}[1]{%
	\def\svgwidth{\columnwidth}
	\import{\cwd/figures/}{#1.pdf_tex}
}

\usepackage{mathrsfs}
\usepackage{float}
\usepackage{tikz-cd}
\usepackage{thmtools}
\usepackage{thm-restate}
\usepackage{etoolbox}
\usepackage{footnote}

\let\oldemptyset\emptyset
\let\emptyset\varnothing

\let\ge\geqslant
\let\le\leqslant
\let\preceq\preccurlyeq
\let\succeq\succcurlyeq

\renewcommand{\C}{\mathds{C}}
\newcommand{\R}{\mathds{R}}
\newcommand{\Z}{\mathds{Z}}
\newcommand{\N}{\mathds{N}}
\newcommand{\Q}{\mathds{Q}}
\renewcommand{\O}{\emptyset}


\renewcommand{\thepart}{\Roman{part}}
\newcommand{\centered}[1]{\begin{center}#1\end{center}}

\let\th\relax
\let\det\relax
\let\Im\relax
\DeclareMathOperator{\Arccos}{Arccos}
\DeclareMathOperator{\Arcsin}{Arcsin}
\DeclareMathOperator{\Arctan}{Arctan}
\DeclareMathOperator{\Argsh}{Argsh}
\DeclareMathOperator{\ch}{ch}
\DeclareMathOperator{\sh}{sh}
\DeclareMathOperator{\th}{th}
\DeclareMathOperator{\Card}{Card}
\DeclareMathOperator{\com}{com}
\DeclareMathOperator{\Ker}{Ker}
\DeclareMathOperator{\Aut}{Aut}
\DeclareMathOperator{\id}{id}
\DeclareMathOperator{\rg}{rg}
\DeclareMathOperator{\PPCM}{PPCM}
\DeclareMathOperator{\PGCD}{PGCD}
\DeclareMathOperator{\argmax}{argmax}
\DeclareMathOperator{\argmin}{argmin}
\DeclareMathOperator{\Vect}{Vect}
\DeclareMathOperator{\cotan}{cotan}
\DeclareMathOperator{\Mat}{Mat}
\DeclareMathOperator{\det}{det}
\DeclareMathOperator{\tr}{tr}
\DeclareMathOperator{\Cov}{Cov}
\DeclareMathOperator{\Supp}{Supp}
\DeclareMathOperator{\Cl}{\mathcal{C}\!\ell}
\DeclareMathOperator*{\po}{\text{\cursive o}}
\DeclareMathOperator*{\gO}{O}
\DeclareMathOperator*{\dom}{dom}
\DeclareMathOperator*{\sgn}{sgn}
\DeclareMathOperator*{\codim}{codim}
\DeclareMathOperator*{\simi}{\sim}
\DeclareMathOperator{\Im}{Im}
\DeclareMathOperator{\Sp}{Sp}
\renewcommand\Re{\mathop{\expandafter\mathfrak{Re}}}
\newcommand\Ima{\mathop{\expandafter\mathfrak{Im}}}

\pdfsuppresswarningpagegroup=1

\newcommand{\emptyenv}[2][{}] {
	\newenvironment{#2}[1][{}] {
		\vspace{-16pt}
		#1
		\vspace{16pt}
		\expandafter\noindent\comment
	}{
		\expandafter\noindent\endcomment
	}
}

\mdfsetup{skipabove=1em,skipbelow=1em, innertopmargin=6pt, innerbottommargin=6pt,}

\newmdenv[frametitlerule=true,roundcorner=5pt,subtitlebelowline=true,subtitleaboveline=true]{recap-box}

\declaretheoremstyle[
	mdframed={ },
	headpunct={:},
	headfont=\normalfont\bfseries,
	bodyfont=\normalfont,
	postheadspace=1em]{defnstyle}

\declaretheoremstyle[
	mdframed={
		rightline=false, topline=false, bottomline=false,
		innerlinewidth=0.4pt,outerlinewidth=0.4pt,
		middlelinewidth=2pt,
		linecolor=black,middlelinecolor=white,
	},
	headpunct={:},
	headfont=\normalfont\bfseries,
	bodyfont=\normalfont,
	postheadspace=1em]{thmstyle}

\declaretheoremstyle[
	headpunct={:},
	postheadspace=\newline,
	headfont=\normalfont\scshape]{rmkstyle}

\declaretheoremstyle[
	headpunct={:},
	postheadspace=\newline,
	numbered={no},
	headfont=\normalfont\scshape]{rmknstyle}

\declaretheorem[style=rmkstyle, name=Théorème]{thm}
\declaretheorem[style=rmkstyle, name=Définition, sibling=thm]{defn}
\declaretheorem[style=rmknstyle, name=Définition, sibling=thm]{defnn}
\declaretheorem[style=rmkstyle, name=Algorithme, sibling=thm]{algo}
\declaretheorem[style=rmkstyle, name=Proposition -- Définition, sibling=thm]{prop-defn}

% \declaretheorem[style=plain, thmbox={style=M, bodystyle=\normalfont}, name=Théorème]{thm}
% \declaretheorem[style=plain, thmbox={style=M, bodystyle=\normalfont}, name=Proposition]{prop}
% \declaretheorem[style=plain, thmbox={style=M, bodystyle=\normalfont}, name=Corollaire]{crlr}
% \declaretheorem[style=plain, thmbox={style=M, bodystyle=\normalfont}, name=Lemme]{lem}

\declaretheorem[style=rmkstyle, name=Axiome, sibling=thm]{axm}
\declaretheorem[style=rmkstyle, name=Proposition, sibling=thm]{prop}
\declaretheorem[style=rmknstyle, name=Proposition, sibling=thm]{propn}
\declaretheorem[style=rmkstyle, name=Corollaire, sibling=thm]{crlr}
\declaretheorem[style=rmkstyle, name=Lemme, sibling=thm]{lem}
\declaretheorem[style=rmkstyle, name=Méthode, sibling=thm]{met}

\declaretheorem[style=rmkstyle, name=Remarque, sibling=thm]{rmk}
\declaretheorem[style=rmknstyle, name=Remarque, sibling=thm]{rmkn}
\declaretheorem[style=rmknstyle, name=Rappel, sibling=thm]{rap}

\declaretheorem[style=rmkstyle, name=Exemple, sibling=thm]{exm}
\declaretheorem[style=rmknstyle, name=Exemple, sibling=thm]{exmn}
\declaretheorem[style=rmknstyle, name=Contre-exemple, sibling=thm]{cexm}
\declaretheorem[style=rmkstyle, name=Exercice, sibling=thm]{exo}
\declaretheorem[style=rmknstyle, name=Exercice, sibling=thm]{exon}
\declaretheorem[style=rmknstyle, name=Preuve]{prv}

\makeatother
\usepackage{fancyhdr}
\pagestyle{fancy}

\fancyhead[R]{}
\fancyhead[L]{\thepart}
\fancyhead[C]{\parttitle}

\fancyfoot[C]{\thepage}
\fancyfoot[L]{}
\fancyfoot[R]{}

\newcommand*\parttitle{}
\let\origpart\part
\renewcommand*{\part}[2][]{%
   \ifx\\#1\\% optional argument not present?
      \origpart{#2}%
      \renewcommand*\parttitle{#2}%
   \else
      \origpart[#1]{#2}%
      \renewcommand*\parttitle{#1}%
   \fi
}

\makeatletter

\newcommand{\tendsto}[1]{\xrightarrow[#1]{}}
\newcommand{\danger}{{\large\fontencoding{U}\fontfamily{futs}\selectfont\char 66\relax}\;}
\newcommand{\ex}{\fbox{ex}\;}
\renewcommand{\mod}[1]{~\left[ #1 \right]}
\newcommand{\todo}[1]{{\color{blue} À faire : #1}}
\newcommand*{\raisesign}[2][.7\normalbaselineskip]{\smash{\llap{\raisebox{#1}{$#2$\hspace{2\arraycolsep}}}}}
\newcommand{\vrt}[1]{\rotatebox{-90}{$#1$}}

\DeclareMathOperator{\ou}{\text{ ou }}
\DeclareMathOperator{\et}{\text{ et }}
\DeclareMathOperator{\si}{\text{ si }}
\DeclareMathOperator{\non}{\text{ non }}

\renewcommand{\title}[2]{
	\AtBeginDocument{
		\begin{titlepage}
			\begin{center}
				\vspace{10cm}
				{\Large \sc Chapitre #1}\\
				\vspace{1cm}
				{\HUGE \cursive #2}\\
				\vfill
				Hugo {\sc Salou} MPI${}^{\star}$\\
				{\ssmall Dernière mise à jour le \@date }
			\end{center}
		\end{titlepage}
	}
}

\let\bx\boxed
\newcommand{\s}{\text{\cursive s}}
\renewcommand{\t}{{}^t\!}
\newcommand{\T}{{\!\!\,\top}}
\newcommand{\eme}{\ensuremath{{}^{\text{ème}}}~}
\newcommand{\red}[1]{{\color{red}#1}}
%\let\oldfract\fract
%\renewcommand{\fract}[2]{\oldfract{\displaystyle #1}{\displaystyle #2}}
% \let\textstyle\displaystyle
% \let\scriptstyle\displaystyle
% \let\scriptscriptstyle\displaystyle
%\everymath{\displaystyle}


\makeatletter
\def\moverlay{\mathpalette\mov@rlay}
\def\mov@rlay#1#2{\leavevmode\vtop{%
   \baselineskip\z@skip \lineskiplimit-\maxdimen
   \ialign{\hfil$\m@th#1##$\hfil\cr#2\crcr}}}
\newcommand{\charfusion}[3][\mathord]{
    #1{\ifx#1\mathop\vphantom{#2}\fi
        \mathpalette\mov@rlay{#2\cr#3}
      }
    \ifx#1\mathop\expandafter\displaylimits\fi}
\makeatother

\newcommand{\cupdot}{\charfusion[\mathbin]{\cup}{\cdot}}
\newcommand{\bigcupdot}{\charfusion[\mathop]{\bigcup}{\cdot}}
\newcommand{\plusbar}{\charfusion[\mathbin]{+}{\color{blue}/}}
\let\olddiamond\diamond
\renewcommand{\diamond}{\mathbin{\olddiamond}}
\newcommand{\missingpart}{\centered{\color{red} \Large Il manque une partie du cours ici}}
%\renewcommand{\Sigma}{\ensuremath{\textstyle \sum}}

\newcommand{\cbox}[2]{{\color{#1}{\fbox{\color{black} #2}}}}
\newcommand{\clrhl}[2]{{\large\sc\color{#1} #2}}
\newcommand{\mat}[1]{\begin{pmatrix} #1 \end{pmatrix}}
\newcommand{\mathunderline}[1]{\text{\underline{$#1$}}}
\let\oldboxtimes\boxtimes
\renewcommand{\boxtimes}{\raisebox{-0.2mm}{$\;\oldboxtimes\;$}}
\newcommand{\indep}{\perp\!\!\!\!\!\:\perp}
\newcommand{\startrecap}{
	\part{Bilan}
	\setlength{\abovedisplayskip}{0pt}
	\setlength{\belowdisplayskip}{0pt}
	\setlength{\abovedisplayshortskip}{0pt}
	\setlength{\belowdisplayshortskip}{0pt}
}


\definecolor{white}{HTML}{faf4ed}
\definecolor{black}{HTML}{575279}
\definecolor{mauve}{HTML}{907aa9}
\definecolor{blue}{HTML}{286983}
\definecolor{red}{HTML}{d7827e}
\definecolor{yellow}{HTML}{ea9d34}
\definecolor{gray}{HTML}{9893a5}
\definecolor{grey}{HTML}{9893a5}

\pagecolor{white}
\color{black}

% code syntax highlighting
\lstset{ %
  backgroundcolor=\color{white},    % choose the background color
  commentstyle=\color{blue},        % comment style
  keywordstyle=\color{mauve},       % keyword style
  stringstyle=\color{pink},         % string literal style
}


\begin{asydef}
	settings.prc = false;
	settings.render=0;

	white = rgb("faf4ed");
	black = rgb("575279");
	blue = rgb("286983");
	red = rgb("d7827e");
	yellow = rgb("f6c177");
	orange = rgb("ea9d34");
	gray = rgb("9893a5");
	grey = rgb("9893a5");
	deepcyan = rgb("56949f");
	pink = rgb("b4637a");
	magenta = rgb("eb6f92");
	green = rgb("a0d971");
	purple = rgb("907aa9");

	defaultpen(black + fontsize(8pt));

	import three;
	currentlight = nolight;
\end{asydef}

\newcommand{\cyr}{
	\renewcommand\rmdefault{wncyr}
	\renewcommand\sfdefault{wncyss}
	\renewcommand\encodingdefault{OT2}
	\normalfont
	\selectfont
}

\DeclareTextFontCommand{\textcyr}{\cyr}
\newcommand{\cprime}{\char"7E }
\newcommand{\cdprime}{\char"7F }
\newcommand{\eoborotnoye}{\char’013}
\newcommand{\Eoborotnoye}{\char’003}

\newcommand{\Tchebychev}{\hspace{-4.3mm}\mbox{\textcyr{chebysh\"ev}}}
\newcommand{\Markov}{\hspace{-4mm}\textcyr{Markov}}
\newcommand{\Gauss}{Gau\ss}

\let\ds\displaystyle
\let\ts\textstyle
\newcommand{\tsup}[1]{\textsuperscript{\underline{#1}}}
\setlength{\parskip}{2mm}
\newcommand\blue[1]{{\color{blue}#1}}
\newcommand{\iffdef}{\mathop{\iff}^{\text{def.}}}

\newcommand{\snput}[1]{
	\begin{center}
		\Large \texttt{#1}
	\end{center}
	\input{#1}
}
