\documentclass[a4paper]{article}

\usepackage[margin=1in]{geometry}
\usepackage[utf8]{inputenc}
\usepackage[T1]{fontenc}
\usepackage{mathrsfs}
\usepackage{textcomp}
\usepackage[french]{babel}
\usepackage{amsmath}
\usepackage{amssymb}
\usepackage{cancel}
\usepackage{frcursive}
\usepackage[inline]{asymptote}
\usepackage{tikz}
\usepackage[european,straightvoltages,europeanresistors]{circuitikz}
\usepackage{tikz-cd}
\usepackage{tkz-tab}
\usepackage[b]{esvect}
\usepackage[framemethod=TikZ]{mdframed}
\usepackage{centernot}
\usepackage{diagbox}
\usepackage{dsfont}
\usepackage{fancyhdr}
\usepackage{float}
\usepackage{graphicx}
\usepackage{listings}
\usepackage{multicol}
\usepackage{nicematrix}
\usepackage{pdflscape}
\usepackage{stmaryrd}
\usepackage{xfrac}
\usepackage{hep-math-font}
\usepackage{amsthm}
\usepackage{thmtools}
\usepackage{indentfirst}
\usepackage[framemethod=TikZ]{mdframed}
\usepackage{accents}
\usepackage{soulutf8}
\usepackage{mathtools}
\usepackage{bodegraph}
\usepackage{slashbox}
\usepackage{enumitem}
\usepackage{calligra}
\usepackage{cinzel}
\usepackage{BOONDOX-calo}

% Tikz
\usetikzlibrary{babel}
\usetikzlibrary{positioning}
\usetikzlibrary{calc}

% global settings
\frenchspacing
\reversemarginpar
\setuldepth{a}

%\everymath{\displaystyle}

\frenchbsetup{StandardLists=true}

\def\asydir{asy}

%\sisetup{exponent-product=\cdot,output-decimal-marker={,},separate-uncertainty,range-phrase=\;à\;,locale=FR}

\setlength{\parskip}{1em}

\theoremstyle{definition}

% Changing math
\let\emptyset\varnothing
\let\ge\geqslant
\let\le\leqslant
\let\preceq\preccurlyeq
\let\succeq\succcurlyeq
\let\ds\displaystyle
\let\ts\textstyle

\newcommand{\C}{\mathds{C}}
\newcommand{\R}{\mathds{R}}
\newcommand{\Z}{\mathds{Z}}
\newcommand{\N}{\mathds{N}}
\newcommand{\Q}{\mathds{Q}}

\renewcommand{\O}{\emptyset}

\newcommand\ubar[1]{\underaccent{\bar}{#1}}

\renewcommand\Re{\expandafter\mathfrak{Re}}
\renewcommand\Im{\expandafter\mathfrak{Im}}

\let\slantedpartial\partial
\DeclareRobustCommand{\partial}{\text{\rotatebox[origin=t]{20}{\scalebox{0.95}[1]{$\slantedpartial$}}}\hspace{-1pt}}

% merging two maths characters w/ \charfusion
\makeatletter
\def\moverlay{\mathpalette\mov@rlay}
\def\mov@rlay#1#2{\leavevmode\vtop{%
   \baselineskip\z@skip \lineskiplimit-\maxdimen
   \ialign{\hfil$\m@th#1##$\hfil\cr#2\crcr}}}
\newcommand{\charfusion}[3][\mathord]{
    #1{\ifx#1\mathop\vphantom{#2}\fi
        \mathpalette\mov@rlay{#2\cr#3}
      }
    \ifx#1\mathop\expandafter\displaylimits\fi}
\makeatother

% custom math commands
\newcommand{\T}{{\!\!\,\top}}
\newcommand{\avrt}[1]{\rotatebox{-90}{$#1$}}
\newcommand{\bigcupdot}{\charfusion[\mathop]{\bigcup}{\cdot}}
\newcommand{\cupdot}{\charfusion[\mathbin]{\cup}{\cdot}}
%\newcommand{\danger}{{\large\fontencoding{U}\fontfamily{futs}\selectfont\char 66\relax}\;}
\newcommand{\tendsto}[1]{\xrightarrow[#1]{}}
\newcommand{\vrt}[1]{\rotatebox{90}{$#1$}}
\newcommand{\tsup}[1]{\textsuperscript{\underline{#1}}}
\newcommand{\tsub}[1]{\textsubscript{#1}}

\renewcommand{\mod}[1]{~\left[ #1 \right]}
\renewcommand{\t}{{}^t\!}
\newcommand{\s}{\text{\calligra s}}

% custom units / constants
%\DeclareSIUnit{\litre}{\ell}
\let\hbar\hslash

% header / footer
\pagestyle{fancy}
\fancyhead{} \fancyfoot{}
\fancyfoot[C]{\thepage}

% fonts
\let\sc\scshape
\let\bf\bfseries
\let\it\itshape
\let\sl\slshape

% custom math operators
\let\th\relax
\let\det\relax
\DeclareMathOperator*{\codim}{codim}
\DeclareMathOperator*{\dom}{dom}
\DeclareMathOperator*{\gO}{O}
\DeclareMathOperator*{\po}{\text{\cursive o}}
\DeclareMathOperator*{\sgn}{sgn}
\DeclareMathOperator*{\simi}{\sim}
\DeclareMathOperator{\Arccos}{Arccos}
\DeclareMathOperator{\Arcsin}{Arcsin}
\DeclareMathOperator{\Arctan}{Arctan}
\DeclareMathOperator{\Argsh}{Argsh}
\DeclareMathOperator{\Arg}{Arg}
\DeclareMathOperator{\Aut}{Aut}
\DeclareMathOperator{\Card}{Card}
\DeclareMathOperator{\Cl}{\mathcal{C}\!\ell}
\DeclareMathOperator{\Cov}{Cov}
\DeclareMathOperator{\Ker}{Ker}
\DeclareMathOperator{\Mat}{Mat}
\DeclareMathOperator{\PGCD}{PGCD}
\DeclareMathOperator{\PPCM}{PPCM}
\DeclareMathOperator{\Supp}{Supp}
\DeclareMathOperator{\Vect}{Vect}
\DeclareMathOperator{\argmax}{argmax}
\DeclareMathOperator{\argmin}{argmin}
\DeclareMathOperator{\ch}{ch}
\DeclareMathOperator{\com}{com}
\DeclareMathOperator{\cotan}{cotan}
\DeclareMathOperator{\det}{det}
\DeclareMathOperator{\id}{id}
\DeclareMathOperator{\rg}{rg}
\DeclareMathOperator{\rk}{rk}
\DeclareMathOperator{\sh}{sh}
\DeclareMathOperator{\th}{th}
\DeclareMathOperator{\tr}{tr}

% colors and page style
\definecolor{truewhite}{HTML}{ffffff}
\definecolor{white}{HTML}{faf4ed}
\definecolor{trueblack}{HTML}{000000}
\definecolor{black}{HTML}{575279}
\definecolor{mauve}{HTML}{907aa9}
\definecolor{blue}{HTML}{286983}
\definecolor{red}{HTML}{d7827e}
\definecolor{yellow}{HTML}{ea9d34}
\definecolor{gray}{HTML}{9893a5}
\definecolor{grey}{HTML}{9893a5}
\definecolor{green}{HTML}{a0d971}

\pagecolor{white}
\color{black}

\begin{asydef}
	settings.prc = false;
	settings.render=0;

	white = rgb("faf4ed");
	black = rgb("575279");
	blue = rgb("286983");
	red = rgb("d7827e");
	yellow = rgb("f6c177");
	orange = rgb("ea9d34");
	gray = rgb("9893a5");
	grey = rgb("9893a5");
	deepcyan = rgb("56949f");
	pink = rgb("b4637a");
	magenta = rgb("eb6f92");
	green = rgb("a0d971");
	purple = rgb("907aa9");

	defaultpen(black + fontsize(8pt));

	import three;
	currentlight = nolight;
\end{asydef}

% theorems, proofs, ...

\mdfsetup{skipabove=1em,skipbelow=1em, innertopmargin=6pt, innerbottommargin=6pt,}

\declaretheoremstyle[
	headfont=\normalfont\itshape,
	numbered=no,
	postheadspace=\newline,
	headpunct={:},
	qed=\qedsymbol]{demstyle}

\declaretheorem[style=demstyle, name=Démonstration]{dem}

\newcommand\veczero{\kern-1.2pt\vec{\kern1.2pt 0}} % \vec{0} looks weird since the `0' isn't italicized

\makeatletter
\renewcommand{\title}[2]{
	\AtBeginDocument{
		\begin{titlepage}
			\begin{center}
				\vspace{10cm}
				{\Large \sc Chapitre #1}\\
				\vspace{1cm}
				{\Huge \calligra #2}\\
				\vfill
				Hugo {\sc Salou} MPI${}^{\star}$\\
				{\small Dernière mise à jour le \@date }
			\end{center}
		\end{titlepage}
	}
}

\newcommand{\titletp}[4]{
	\AtBeginDocument{
		\begin{titlepage}
			\begin{center}
				\vspace{10cm}
				{\Large \sc tp #1}\\
				\vspace{1cm}
				{\Huge \textsc{\textit{#2}}}\\
				\vfill
				{#3}\textit{MPI}${}^{\star}$\\
			\end{center}
		\end{titlepage}
	}
	\fancyfoot{}\fancyhead{}
	\fancyfoot[R]{#4 \textit{MPI}${}^{\star}$}
	\fancyhead[C]{{\sc tp #1} : #2}
	\fancyhead[R]{\thepage}
}

\newcommand{\titletd}[2]{
	\AtBeginDocument{
		\begin{titlepage}
			\begin{center}
				\vspace{10cm}
				{\Large \sc td #1}\\
				\vspace{1cm}
				{\Huge \calligra #2}\\
				\vfill
				Hugo {\sc Salou} MPI${}^{\star}$\\
				{\small Dernière mise à jour le \@date }
			\end{center}
		\end{titlepage}
	}
}
\makeatother

\newcommand{\sign}{
	\null
	\vfill
	\begin{center}
		{
			\fontfamily{ccr}\selectfont
			\textit{\textbf{\.{\"i}}}
		}
	\end{center}
	\vfill
	\null
}

\renewcommand{\thefootnote}{\emph{\alph{footnote}}}

% figure support
\usepackage{import}
\usepackage{xifthen}
\pdfminorversion=7
\usepackage{pdfpages}
\usepackage{transparent}
\newcommand{\incfig}[1]{%
	\def\svgwidth{\columnwidth}
	\import{./figures/}{#1.pdf_tex}
}

\pdfsuppresswarningpagegroup=1
\ctikzset{tripoles/european not symbol=circle}

\newcommand{\missingpart}{{\large\color{red} Il manque quelque chose ici\ldots}}


\begin{document}
	\begin{center}
		\Huge $\mathbf{DM_2}$\/ Mathématiques
	\end{center}

	\begin{center}
		\LARGE \scshape Problème 1
	\end{center}

	\section*{Le groupe symplectique}

	\begin{enumerate}
		\item On calcule $J^2$\/ et $J^\top$ en fonction de $I_{2n}$\/ et de $J$\/ : \[
				J^2 = \begin{pmatrix}
					0_n - I_{n} & 0_n + 0_n\\
					0_n + 0_n & -I_n + 0_n
				\end{pmatrix} = -\begin{pmatrix}
					I_n&0_n\\
					0_n&I_n
				\end{pmatrix} = -I_{2n},
			\] et \[
				J^\top = \begin{pmatrix}
					0_n & I_n\\
					-I_n&0_n
				\end{pmatrix} = -J
			.\] On remarque que $J \times J^\top = J \times (-J) = -(-I_{2n}) = I_{2n}$. On en déduit que $J$\/ est inversible et $J^{-1} = J^\top$.
		\item On remarque que $J^\top \times J \times J = I_{2n} \times J = J$\/ et donc $J \in \mathcal{S}p_{2n}$.
			\begin{align*}
				\begin{pmatrix}
					I_n&-\alpha I_n\\
					0_n&I_n
				\end{pmatrix} \begin{pmatrix}
					0_n&-I_n\\
					I_n&0_n
				\end{pmatrix} \begin{pmatrix}
					I_n&0_n\\
					-\alpha I_n&I_n
				\end{pmatrix}
				&=
				\begin{pmatrix}
					-\alpha I_n & -I_n\\
					I_n&0_n
				\end{pmatrix} \begin{pmatrix}
					I_n&0_n\\
					-\alpha I_n & I_n
				\end{pmatrix}\\ &= \begin{pmatrix}
					0_n & -I_n\\
					I_n & 0_n
				\end{pmatrix} = J
			.\end{align*}
			On en déduit que $K(\alpha) \in \mathcal{S}p_{2n}$\/ pour tout réel $\alpha$.
		\item Soit $U \in \mathcal{G}_n$. Soit $L_U$\/ comme défini dans l'énoncé. On calcule ${L_U}^\top\:J\:L_U$ :
			\begin{align*}
				\begin{pmatrix}
					U^\top &0_n\\
					0_n&U^{-1}
				\end{pmatrix}\cdot \begin{pmatrix}
					0_n&-I_n\\
					I_n&0_n
				\end{pmatrix} \cdot \begin{pmatrix}
					U&0_n\\
					0_n&(U^{-1})^\top
				\end{pmatrix} &= \begin{pmatrix}
					0_n&-U^\top\\
					U^{-1}&0_n
				\end{pmatrix} \cdot \begin{pmatrix}
					U&0_n\\
					0_n&(U^{-1})^\top 
				\end{pmatrix}\\
				&= \begin{pmatrix}
					0_n&-I_n\\
					I_n&0_n
				\end{pmatrix} = J
			.\end{align*}
			On en déduit que $L_U \in \mathcal{S}p_{2n}$\/ pour toute matrice $U \in \mathcal{G}_n$.
		\item Soit $M \in \mathcal{S}p_{2n}$. On a donc $M^\top\:J\:M = J$, d'où \[
				\det(M^\top) \times \det(J) \times \det(M) = \det J \quad\text{i.e.} \det(M)^2 = 1
			\] car $\det J \neq 0$\/ (car elle est inversible) et donc, on en déduit que $\det M \in \{-1,1\}$.
		\item Soient $A$\/ et $B$\/ deux matrices de $\mathcal{S}p_{2n}$. On a donc $A^\top \:J\:A = J$, et $B^\top\:J\:B = J$. On pose $M = A \cdot B$\/ et \[
				M^\top J M = B^\top A^\top J A B = B^\top J B = J
			.\] On en déduit que $A \cdot B \in \mathcal{S}p_{2n}$.
		\item On sait que $I_{2n} \in \mathcal{S}p_{2n}$. Soit $A \in \mathcal{S}p_{2n}$. On sait que $\det A \neq 0$\/ ; la matrice $A$\/ est donc inversible.
			On a $A^\top J A = J$, d'où ${(A^\top)}^{-1}\cdot A^\top\cdot J\cdot A\cdot A^{-1} = {(A^\top)}^{-1}\cdot J\cdot A^{-1}$. On en déduit donc que $J = {(A^{-1})}^\top \cdot J\cdot A^{-1}$\/ et donc $A^{-1} \in \mathcal{S}p_{2n}$.
		\item Soit $M \in \mathcal{S}p_{2n}$. On a donc $M^\top\:J\:M = J$, d'où $(M^\top\:J\:M)^{-1} = J^{-1}$ i.e.\ $M^{-1}\:J^\top\:{(M^\top)}^{-1} = J^\top$. D'où $M\cdot M^{-1} J^\top (M^\top)^{-1} \cdot M^\top = M\cdot J M^\top$\/ et donc $J = M\cdot J\cdot M^\top$, autrement dit, \[\boxed{M^\top \in \mathcal{S}p_{2n}.}\]
		\item \marginpar{À faire\ldots}
	\end{enumerate}
	\section*{Centre de $\mathcal{S}p_{2n}$}
	\begin{enumerate}
		\item[9.] Soit $N \in \mathcal{S}p_{2n}$. On sait que $I_{2n}\cdot N = N = N\cdot I_{2n}$, d'où $I_{2n} \in \mathcal{Z}$.
			De même, $-I_{2n} \cdot N = -N = N \cdot (-I_{2n})$, d'où $-I_{2n} \in \mathcal{Z}$. On a donc \[
				\boxed{\{-I_{2n}, I_{2n}\} \subset \mathcal{Z}.}
			\]
		\item[10.] On remarque que $L = K(-1)^\top \in \mathcal{S}p_{2n}$. Or, comme $M \in \mathcal{Z}$, $L\cdot M = M \cdot L$. D'où \[
				M\cdot L = \begin{pmatrix}
					A&A+B\\
					C&C+D
				\end{pmatrix} = \begin{pmatrix}
					A+C&B+D\\
					C&D
				\end{pmatrix} = L \cdot M
			.\] On en déduit que $C = 0_n$, $A = D$.
			On sait que $L^\top \in \mathcal{S}p_{2n}$, d'où $L^\top \cdot M = M\cdot L^\top$\/ et donc \[
				\begin{pmatrix}
					A + B & B\\
					C + D & D
				\end{pmatrix} = \begin{pmatrix}
					A & B\\
					A + C & B + D
				\end{pmatrix}
			.\] On en déduit donc que $B = 0_n$. La matrice $M$\/ est donc de la forme \[
				M = \begin{pmatrix}
					A&0_n\\
					0_n&A
				\end{pmatrix}
			.\] Or, $\det M = (\det A)^2 \neq 0$\/ et donc $\det A \neq 0$. On en déduit que la matrice $A$\/ est inversible.
		\item[11.] On calcule $M \cdot L_U$\/ puis $L_U\cdot M$\/ :
			\begin{align*}
				M \cdot L_U &= \begin{pmatrix}
					A&0_n\\
					0_n&A
				\end{pmatrix} \cdot \begin{pmatrix}
					U&0_n\\
					0_n&(U^{-1})^\top
				\end{pmatrix} \\
				&= \begin{pmatrix}
					A\cdot U&0_n\\
					0_n&A\cdot (U^{-1})^\top
				\end{pmatrix} ; \\
			\end{align*}
			\begin{align*}
				L_U\cdot M &= \begin{pmatrix}
					U&0_n\\
					0_n&(U^{-1})^\top
				\end{pmatrix} \cdot \begin{pmatrix}
					A&0_n\\
					0_n&A
				\end{pmatrix} \\
				&= \begin{pmatrix}
					U\cdot A& 0_n\\
					0_n&(U^{-1})^\top \cdot A
				\end{pmatrix}. \\
			\end{align*}
			Or, $M \in \mathcal{Z}$\/ et $L_U \in \mathcal{S}p_{2n}$, d'où $M\cdot L_U = L_U\cdot M$. On en déduit donc que $U\cdot A = A \cdot U$.
		\item[12.] Les seules matrices commutant dans $\mathcal{G}_n$\/ sont les matrices $I_n$\/ et $-I_n$, on a donc $A \in \{I_n,-I_n\}$\/ et donc $M = I_{2n}$\/ ou $M = -I_{2n}$, d'où $\mathcal{Z} \subset \{-I_{2n}, I_{2n}\}$. On en déduit que \[
			\boxed{\mathcal{Z} = \{-I_{2n}, I_{2n}\}.}
		\]
	\end{enumerate}
	\section*{Déterminant d'une matrice symplectique}
	\begin{enumerate}
		\item[13.]
			On pose $M = {a\:b\choose c\:d} \in \mathscr{M}_2$\/ et on calcule $M^\top \cdot J\cdot M$\/ :
			\begin{align*}
				\begin{pmatrix}
					a&c\\
					b&d
				\end{pmatrix} \cdot \begin{pmatrix}
					0&-1\\
					1&0
				\end{pmatrix}  \cdot \begin{pmatrix}
					a&b\\
					c&d
				\end{pmatrix}
				&= \begin{pmatrix}
					c&-a\\
					d&-b
				\end{pmatrix} \cdot \begin{pmatrix}
					a&b\\
					c&d
				\end{pmatrix} \\
				&= \begin{pmatrix}
					ac-ca&cb-ad\\
					da-bc&db-bd
				\end{pmatrix} \\
				&= \begin{pmatrix}
					0&-\det M\\
					\det M&0
				\end{pmatrix} \\
			\end{align*}
			Or, \[
				M \in \mathcal{S}p_2 \iff M^\top \cdot J\cdot M = J
			.\]
			D'où \[
				\begin{pmatrix}
					a&b\\
					c&d
				\end{pmatrix} \in \mathcal{S}p_2 \iff \begin{cases}
					\det M = 1\\
					-\det M = -1
				\end{cases} \iff \det M = 1
			.\]
			On a montré que \[
				\boxed{M \in \mathcal{S}p_{2} \iff \det M = 1.}
			\]
		\item[14.]On procède à une analyse-synthèse.
			\begin{itemize}
				\item[\sc Analyse] Soient $Q,U,V,W \in \mathscr{M}_n$.
					On calcule \[
						\begin{pmatrix}
							I_n & Q\\
							0_n&I_n
						\end{pmatrix} \cdot \begin{pmatrix}
							U&0_n\\
							V&W
						\end{pmatrix} = \begin{pmatrix}
							U+QV&QW\\
							V&W
						\end{pmatrix}
					.\]
					Ainsi, en comparant avec les coefficients de la matrice ${A\:B\choose C\:D}$, on en déduit que \[
						D = W \qquad V = C\qquad Q = B \cdot D^{-1}\qquad U = A - B \cdot D^{-1}\cdot C
					.\]
				\item[{\sc Synthèse}] On pose $W = D$, $V = C$, $Q = B\cdot D^{-1}$\/ et $U = A - BD^{-1}C$. On calcule \[
					\begin{align*}
						\begin{pmatrix}
							I_n & Q\\
							0_n&I_n
						\end{pmatrix} \cdot \begin{pmatrix}
							U&0_n\\
							V&W
						\end{pmatrix} &= \begin{pmatrix}
							U+QV&QW\\
							V&W
						\end{pmatrix}\\ &= \begin{pmatrix}
							A-BD^{-1}C+BD^{-1}C&BD^{-1}\cdot D\\
							C&D
						\end{pmatrix}\\ &= \begin{pmatrix}
							A&B\\
							C&D
						\end{pmatrix}.
					\end{align*}
			\end{itemize}
	\end{enumerate}
\end{document}
