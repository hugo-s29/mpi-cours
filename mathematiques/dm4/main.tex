\documentclass[a4paper]{article}

\usepackage[margin=1in]{geometry}
\usepackage[utf8]{inputenc}
\usepackage[T1]{fontenc}
\usepackage{mathrsfs}
\usepackage{textcomp}
\usepackage[french]{babel}
\usepackage{amsmath}
\usepackage{amssymb}
\usepackage{cancel}
\usepackage{frcursive}
\usepackage[inline]{asymptote}
\usepackage{tikz}
\usepackage[european,straightvoltages,europeanresistors]{circuitikz}
\usepackage{tikz-cd}
\usepackage{tkz-tab}
\usepackage[b]{esvect}
\usepackage[framemethod=TikZ]{mdframed}
\usepackage{centernot}
\usepackage{diagbox}
\usepackage{dsfont}
\usepackage{fancyhdr}
\usepackage{float}
\usepackage{graphicx}
\usepackage{listings}
\usepackage{multicol}
\usepackage{nicematrix}
\usepackage{pdflscape}
\usepackage{stmaryrd}
\usepackage{xfrac}
\usepackage{hep-math-font}
\usepackage{amsthm}
\usepackage{thmtools}
\usepackage{indentfirst}
\usepackage[framemethod=TikZ]{mdframed}
\usepackage{accents}
\usepackage{soulutf8}
\usepackage{mathtools}
\usepackage{bodegraph}
\usepackage{slashbox}
\usepackage{enumitem}
\usepackage{calligra}
\usepackage{cinzel}
\usepackage{BOONDOX-calo}

% Tikz
\usetikzlibrary{babel}
\usetikzlibrary{positioning}
\usetikzlibrary{calc}

% global settings
\frenchspacing
\reversemarginpar
\setuldepth{a}

%\everymath{\displaystyle}

\frenchbsetup{StandardLists=true}

\def\asydir{asy}

%\sisetup{exponent-product=\cdot,output-decimal-marker={,},separate-uncertainty,range-phrase=\;à\;,locale=FR}

\setlength{\parskip}{1em}

\theoremstyle{definition}

% Changing math
\let\emptyset\varnothing
\let\ge\geqslant
\let\le\leqslant
\let\preceq\preccurlyeq
\let\succeq\succcurlyeq
\let\ds\displaystyle
\let\ts\textstyle

\newcommand{\C}{\mathds{C}}
\newcommand{\R}{\mathds{R}}
\newcommand{\Z}{\mathds{Z}}
\newcommand{\N}{\mathds{N}}
\newcommand{\Q}{\mathds{Q}}

\renewcommand{\O}{\emptyset}

\newcommand\ubar[1]{\underaccent{\bar}{#1}}

\renewcommand\Re{\expandafter\mathfrak{Re}}
\renewcommand\Im{\expandafter\mathfrak{Im}}

\let\slantedpartial\partial
\DeclareRobustCommand{\partial}{\text{\rotatebox[origin=t]{20}{\scalebox{0.95}[1]{$\slantedpartial$}}}\hspace{-1pt}}

% merging two maths characters w/ \charfusion
\makeatletter
\def\moverlay{\mathpalette\mov@rlay}
\def\mov@rlay#1#2{\leavevmode\vtop{%
   \baselineskip\z@skip \lineskiplimit-\maxdimen
   \ialign{\hfil$\m@th#1##$\hfil\cr#2\crcr}}}
\newcommand{\charfusion}[3][\mathord]{
    #1{\ifx#1\mathop\vphantom{#2}\fi
        \mathpalette\mov@rlay{#2\cr#3}
      }
    \ifx#1\mathop\expandafter\displaylimits\fi}
\makeatother

% custom math commands
\newcommand{\T}{{\!\!\,\top}}
\newcommand{\avrt}[1]{\rotatebox{-90}{$#1$}}
\newcommand{\bigcupdot}{\charfusion[\mathop]{\bigcup}{\cdot}}
\newcommand{\cupdot}{\charfusion[\mathbin]{\cup}{\cdot}}
%\newcommand{\danger}{{\large\fontencoding{U}\fontfamily{futs}\selectfont\char 66\relax}\;}
\newcommand{\tendsto}[1]{\xrightarrow[#1]{}}
\newcommand{\vrt}[1]{\rotatebox{90}{$#1$}}
\newcommand{\tsup}[1]{\textsuperscript{\underline{#1}}}
\newcommand{\tsub}[1]{\textsubscript{#1}}

\renewcommand{\mod}[1]{~\left[ #1 \right]}
\renewcommand{\t}{{}^t\!}
\newcommand{\s}{\text{\calligra s}}

% custom units / constants
%\DeclareSIUnit{\litre}{\ell}
\let\hbar\hslash

% header / footer
\pagestyle{fancy}
\fancyhead{} \fancyfoot{}
\fancyfoot[C]{\thepage}

% fonts
\let\sc\scshape
\let\bf\bfseries
\let\it\itshape
\let\sl\slshape

% custom math operators
\let\th\relax
\let\det\relax
\DeclareMathOperator*{\codim}{codim}
\DeclareMathOperator*{\dom}{dom}
\DeclareMathOperator*{\gO}{O}
\DeclareMathOperator*{\po}{\text{\cursive o}}
\DeclareMathOperator*{\sgn}{sgn}
\DeclareMathOperator*{\simi}{\sim}
\DeclareMathOperator{\Arccos}{Arccos}
\DeclareMathOperator{\Arcsin}{Arcsin}
\DeclareMathOperator{\Arctan}{Arctan}
\DeclareMathOperator{\Argsh}{Argsh}
\DeclareMathOperator{\Arg}{Arg}
\DeclareMathOperator{\Aut}{Aut}
\DeclareMathOperator{\Card}{Card}
\DeclareMathOperator{\Cl}{\mathcal{C}\!\ell}
\DeclareMathOperator{\Cov}{Cov}
\DeclareMathOperator{\Ker}{Ker}
\DeclareMathOperator{\Mat}{Mat}
\DeclareMathOperator{\PGCD}{PGCD}
\DeclareMathOperator{\PPCM}{PPCM}
\DeclareMathOperator{\Supp}{Supp}
\DeclareMathOperator{\Vect}{Vect}
\DeclareMathOperator{\argmax}{argmax}
\DeclareMathOperator{\argmin}{argmin}
\DeclareMathOperator{\ch}{ch}
\DeclareMathOperator{\com}{com}
\DeclareMathOperator{\cotan}{cotan}
\DeclareMathOperator{\det}{det}
\DeclareMathOperator{\id}{id}
\DeclareMathOperator{\rg}{rg}
\DeclareMathOperator{\rk}{rk}
\DeclareMathOperator{\sh}{sh}
\DeclareMathOperator{\th}{th}
\DeclareMathOperator{\tr}{tr}

% colors and page style
\definecolor{truewhite}{HTML}{ffffff}
\definecolor{white}{HTML}{faf4ed}
\definecolor{trueblack}{HTML}{000000}
\definecolor{black}{HTML}{575279}
\definecolor{mauve}{HTML}{907aa9}
\definecolor{blue}{HTML}{286983}
\definecolor{red}{HTML}{d7827e}
\definecolor{yellow}{HTML}{ea9d34}
\definecolor{gray}{HTML}{9893a5}
\definecolor{grey}{HTML}{9893a5}
\definecolor{green}{HTML}{a0d971}

\pagecolor{white}
\color{black}

\begin{asydef}
	settings.prc = false;
	settings.render=0;

	white = rgb("faf4ed");
	black = rgb("575279");
	blue = rgb("286983");
	red = rgb("d7827e");
	yellow = rgb("f6c177");
	orange = rgb("ea9d34");
	gray = rgb("9893a5");
	grey = rgb("9893a5");
	deepcyan = rgb("56949f");
	pink = rgb("b4637a");
	magenta = rgb("eb6f92");
	green = rgb("a0d971");
	purple = rgb("907aa9");

	defaultpen(black + fontsize(8pt));

	import three;
	currentlight = nolight;
\end{asydef}

% theorems, proofs, ...

\mdfsetup{skipabove=1em,skipbelow=1em, innertopmargin=6pt, innerbottommargin=6pt,}

\declaretheoremstyle[
	headfont=\normalfont\itshape,
	numbered=no,
	postheadspace=\newline,
	headpunct={:},
	qed=\qedsymbol]{demstyle}

\declaretheorem[style=demstyle, name=Démonstration]{dem}

\newcommand\veczero{\kern-1.2pt\vec{\kern1.2pt 0}} % \vec{0} looks weird since the `0' isn't italicized

\makeatletter
\renewcommand{\title}[2]{
	\AtBeginDocument{
		\begin{titlepage}
			\begin{center}
				\vspace{10cm}
				{\Large \sc Chapitre #1}\\
				\vspace{1cm}
				{\Huge \calligra #2}\\
				\vfill
				Hugo {\sc Salou} MPI${}^{\star}$\\
				{\small Dernière mise à jour le \@date }
			\end{center}
		\end{titlepage}
	}
}

\newcommand{\titletp}[4]{
	\AtBeginDocument{
		\begin{titlepage}
			\begin{center}
				\vspace{10cm}
				{\Large \sc tp #1}\\
				\vspace{1cm}
				{\Huge \textsc{\textit{#2}}}\\
				\vfill
				{#3}\textit{MPI}${}^{\star}$\\
			\end{center}
		\end{titlepage}
	}
	\fancyfoot{}\fancyhead{}
	\fancyfoot[R]{#4 \textit{MPI}${}^{\star}$}
	\fancyhead[C]{{\sc tp #1} : #2}
	\fancyhead[R]{\thepage}
}

\newcommand{\titletd}[2]{
	\AtBeginDocument{
		\begin{titlepage}
			\begin{center}
				\vspace{10cm}
				{\Large \sc td #1}\\
				\vspace{1cm}
				{\Huge \calligra #2}\\
				\vfill
				Hugo {\sc Salou} MPI${}^{\star}$\\
				{\small Dernière mise à jour le \@date }
			\end{center}
		\end{titlepage}
	}
}
\makeatother

\newcommand{\sign}{
	\null
	\vfill
	\begin{center}
		{
			\fontfamily{ccr}\selectfont
			\textit{\textbf{\.{\"i}}}
		}
	\end{center}
	\vfill
	\null
}

\renewcommand{\thefootnote}{\emph{\alph{footnote}}}

% figure support
\usepackage{import}
\usepackage{xifthen}
\pdfminorversion=7
\usepackage{pdfpages}
\usepackage{transparent}
\newcommand{\incfig}[1]{%
	\def\svgwidth{\columnwidth}
	\import{./figures/}{#1.pdf_tex}
}

\pdfsuppresswarningpagegroup=1
\ctikzset{tripoles/european not symbol=circle}

\newcommand{\missingpart}{{\large\color{red} Il manque quelque chose ici\ldots}}

%\usepackage{concmath}
%\usepackage{tgschola}

\let\bfseries\scshape

\begin{document}
	\begin{center}
		\Huge \textbf{DM\textsubscript4 Mathématiques}
	\end{center}

	\begin{center}
		\LARGE \textsc{Problème 1}
	\end{center}
	
	\begin{enumerate}
		\item Les événements $E_1$\/ et $E_2$\/ sont certains. Au 1\tsup{er} et au 2\tsup{nd} duel, le gagnant n'est pas encore désigné, peu importe les gagnants de ces duels. Pour calculer la probabilité de l'événement $E_3$, on passe au complémentaire : l'événement $\bar{E}_3$\/ correspond à \guillemotleft~le joueur 0 ou le joueur 1 ne gagne pas le duel.~\guillemotright\@ Ainsi, en notant $G^i_k$\/ l'événement \guillemotleft~le joueur~$A_k$\/ gagne le $i$-ème duel,~\guillemotright\ on a $\bar{E}_3 = (G_0^1 \cap G_0^2 \cap G_0^3) \cup (G_1^1 \cap G_1^2 \cap G_1^3)$, et cette union est disjointe. D'où 
			\begin{align*}
				P(\bar{E}_3) &= P(G_0^1 \cap G_0^2 \cap G_0^3) + P(G_1^1 \cap G_1^2 \cap G_1^3)\\
				&= {}\mathbin{\phantom+} P(G_0^1) \times P(G_0^2  \mid G_0^1) \times P(G_0^3  \mid G_0^1 \cap G_0^2) \\
				&\mathrel{\phantom=}{} + P(G_1^1) \times P(G_1^2  \mid G_1^1) \times P(G_1^3  \mid G_1^1 \cap G_1^2)\\
				&= 2 \times \left( \frac{1}{2} \right)^3 = \frac{1}{4} \\
			\end{align*}
			On en déduit que $P(E_3) = 1 - P(\bar{E}_3) = \frac{3}{4}$.
			On a bien $\frac{1}{2}P(E_2) + \frac{1}{4}P(E_1) = \frac{1}{2} + \frac{1}{4} = \frac{3}{4} = P(E_3)$.
		\item Soit $n\ge 3$. On pose $U_k$\/ l'événement \guillemotleft~il n'y a pas encore de gagnant désigné et le joueur $A_k$\/ remporte le duel $k$,~\guillemotright\ et $V_k$\/ l'événement \guillemotleft~il n'y a pas encore de gagnant désigné et le joueur $A_{k-1}$\/ remporte le duel $k$.~\guillemotright\@ Ainsi, $E_n = U_n \cup V_n$ et cette union est disjointe. Ainsi, $P(E_n) = P(U_n) + P(V_n)$.
			D'une part, on a que $U_n = E_{n-1} \cap G^k_k$, donc $P(U_n) = P(E_{n-1}) \times P(G_k^k  \mid E_{n-1}) = \frac{1}{2}P(E_{n-1})$.
			D'autre part, on a $V_n = E_{n-2} \cap G_{k-1}^{k-1} \cap G_{k-1}^k$, d'où $P(V_n) = P(E_{n-2}) \times P(G_{k-1}^{k-1}  \mid E_{n-2}) \times P(G_{k-1}^k  \mid E_{n-2} \cap G_{k-1}^{k-1}) = \frac{1}{2} \times \frac{1}{2} \times  P(E_{n-2})$.
			On en déduit donc que\\
			\null\hfill$\forall n \ge 3,\quad P(E_n) = \frac{1}{2}P(E_{n-1}) + \frac{1}{4} P(E_{n-2}).\hfill(\mathcal{R}_1)$
		\item On pose, pour $n \ge 3$, $u_n = P(E_n)$. Ainsi, d'après $(\mathcal{R}_1)$, \[
				\forall n \ge 3,\quad u_n = \frac{1}{2} u_{n-1} + \frac{1}{4} u_{n-2}
			.\]
			L'équation caractéristique de $(\mathcal{R}_1)$\/ est $x^2 = \frac{1}{2} x + \frac{1}{4}$. On résout donc $4x^2 - 2x - 1 = 0$. Le discriminant de ce trinôme est~$\Delta = 20 > 0$. On en déduit que les racines de cette équation caractéristique sont \[
				x_1 = \frac{2 + \sqrt{20}}{8} = \frac{1 + \sqrt{5}}{4} \qquad \text{ et } \qquad x_2 = \frac{2 + \sqrt{20}}{8} = \frac{1 - \sqrt{5}}{4}
			.\] Ainsi, il existe deux constantes réelles $\lambda$\/ et $\mu$\/ que l'on peut déterminer à l'aide de $u_1$\/ et $u_2$, telles que \[
				P(E_n) = u_n = \lambda\times {x_1}^n + \mu\times{x_2}^n
			.\]
		\item L'événement $E_{n+1}$\/ est inclus dans $E_n$, ainsi la suite $(E_n)_{n\in\N}$\/ est décroissante (au sens de l'inclusion). Ainsi, par continuité décroissante, on a \[
				P\Big(\bigcap_{n=2}^\infty E_n\Big) = \lim_{n\to \infty} P(E_n) = 0
			\] comme $|r_1| < 1$\/ et $|r_2| < 1$. L'événement, que l'on notera $W$, \guillemotleft~le tournoi désignera un vainqueur~\guillemotright\ est le complémentaire de l'événement $\bigcap_{n=2}^{\infty} E_n$. Ainsi, $P(W) = 1 - P\big(\bigcap_{n=2}^\infty E_n\big) = 0$.
	\end{enumerate}

	\begin{center}
		\LARGE \textsc{Problème 2}
	\end{center}

	\begin{enumerate}
		\item Soient $u,v \in \exists $. On a
			\begin{align*}
				\det\big(G(u,v)\big) &= \left<u \mid u \right>\:\left<v \mid v \right> - \left<v \mid u \right>\:\left<u \mid v \right>\\
				&= \|u\|^2\:\|v\|^2 - \left<u \mid v \right>^2 \text{ par symétrie }\\
				&= \big(\|u\|\:\|v\| - \left<u \mid v \right>\big)
				\big(\|u\|\:\|v\| + \left<u \mid v \right>\big)\\
				&= \big(\|u\|\:\|v\| - \left<u \mid v \right>\big)
				\big(\|-u\|\:\|v\| - \left<(-u) \mid v \right>\big)\\
				&\ge 0 \text{ par inégalité de \textsc{Cauchy-Schwarz}}.
			\end{align*}
			Ce déterminant est nul si, et seulement si $u$\/ et $v$\/ sont colinéaires (d'après l'égalité de \textsc{Cauchy-Scharz}). Ainsi, $u$\/ et $v$\/ non colinéaires est un condition nécessaire et suffisante pour que $\det G(u,v)$\/ soit strictement positif.
		\item
			\begin{enumerate}
				\item On calcule, pour $(i,j) \in \llbracket 1,n \rrbracket^2$,
					\begin{align*}
						\big(G(v_1,\ldots,v_n)\big)_{i,j} &= \left<v_i  \mid v_j \right>\\
																							&= \Big< \sum_{k=1}^n a_{k,i} e_k \:\Big|\: \sum_{k=1}^n a_{k,j} e_k\Big>\\
						&= \sum_{k=1}^n a_{k,i}\Big<e_k \:\Big|\sum_{p=1}^n a_{p,j} e_p \Big> \\
						&= \sum_{k=1}^n \sum_{p=1}^n a_{k,i} a_{p,j} \left<e_k  \mid e_p \right>  \\
						&= \sum_{k=1}^n a_{k,i} a_{k,j} \text{ car la base } (e_1,\ldots,e_n)  \text{ est orthonormée }\\
						&= \sum_{k=1}^n (A^\top)_{i,k}\:(A)_{k,j} \\
						&= (A^\top \cdot A)_{i,j} \\
					\end{align*}
					D'où $G(v_1,\ldots,v_n) = A^\top \cdot A$.
				\item On a $\det G(v_1,\ldots,v_n) = \det (A^\top \cdot A) = \det A^\top \times \det A = \det^2 A \ge 0$.
				\item On cherche à montrer que $\Ker A = \Ker G(v_1, \ldots, v_n)$. Soit $X \in \mathcal{M}_{n,1}(\R)$. Montrons que $A\cdot X = 0$\/ si, et seulement si $A^\top \cdot A\cdot X = 0$, d'après (a). On remarque que, si $A\cdot X = 0$, alors $A^\top \cdot (A\cdot X) = 0$.
					Réciproquement, si $A^\top \cdot A \cdot X = 0$, alors $(A\cdot X)^\top \cdot A\cdot X = X^\top \cdot A^\top \cdot A^\top \cdot X = 0$.
					Mais, avec le produit scalaire canonique sur $\mathcal{M}_{n,n}(\R)$, on a $\left<AX \mid AX \right> = 0$, d'où $AX = 0$.
					D'après (a), $\Ker G(v_1, \ldots, v_n) = A^\top \cdot A$. Ainsi, d'après le théorème du rang,
					\begin{align*}
						\rg A = \dim(\Im A) &= \dim \mathcal{M}_{n,1}(\R) - \dim(\Ker A)\\
						&= \dim \mathcal{M}_{n,1}(\R) - \dim G(v_1, \ldots, v_n)\\
						&= \dim\!\big(\!\Im G(v_1, \ldots, v_n)\big)\\
						&= \rg G(v_1, \ldots, v_n)
					\end{align*}
				\item On sait que, pour $j \in \llbracket 1,n \rrbracket$, $v_j = \sum_{i=0}^n a_{i,j} e_j$. On a donc bien $\dim (\Im A) = \dim \Vect(v_1, \ldots, v_n)$. D'où, d'après la question précédente, \[
					\dim \Vect(v_1, \ldots, v_n) = \dim\!\big(\!\Im G(v_1, \ldots, v_n)\big)
				.\]
			\end{enumerate}
		\item
			\begin{enumerate}
				\item On a
					\begin{align*}
						G(v_1, \ldots, v_n, z) &= \begin{pmatrix}
							\left<v_1 \mid v_1 \right> & \ldots & \left<v_1 \mid v_n \right> & \left<v_1 \mid z \right>\\
							\vdots & \ddots & \vdots & \vdots\\
							\left<v_n  \mid v_1 \right> & \ldots & \left<v_n  \mid v_n \right> & \left<v_n  \mid z \right>\\
							\left<z  \mid v_1 \right> & \ldots & \left<z  \mid v_n \right> & \left<z  \mid z \right>
						\end{pmatrix}\\
						&= \begin{pmatrix}
							\left<v_1 \mid v_1 \right> & \ldots & \left<v_1 \mid v_n \right> & 0\\
							\vdots & \ddots & \vdots & \vdots\\
							\left<v_n  \mid v_1 \right> & \ldots & \left<v_n  \mid v_n \right> & 0 \\
							0 & \ldots & 0 & \|z\|^2
						\end{pmatrix} \\
					\end{align*}
					Le déterminant de cette matrice diagonale par blocs est le produit des déterminants de chaque bloc, d'où \[
						\det G(v_1, \ldots, v_n, z) = \det G(v_1, \ldots, v_n) \cdot \|z\|^2
					.\]
				\item On exprime $y \in F$\/ dans la base $(v_1, \ldots, v_n)$\/ : soient $y_1, \ldots, y_n$\/ tels que $y = \sum_{i=0}^n y_i v_i$.
					\begin{align*}
						G(v_1, \ldots, v_n, y+z) &= \begin{pmatrix}
							\left<v_1 \mid v_1 \right> & \ldots & \left<v_1 \mid v_n \right> & \left<v_1 \mid y + z \right>\\
							\vdots & \ddots & \vdots & \vdots\\
							\left<v_n  \mid v_1 \right> & \ldots & \left<v_n  \mid v_n \right> & \left<v_n  \mid y + z \right>\\
							\left<y+z  \mid v_1 \right> & \ldots & \left<y+z  \mid v_n \right> & \left<y + z  \mid y + z \right>
						\end{pmatrix}\\
						&= \begin{pmatrix}
							\left<v_1 \mid v_1 \right> & \ldots & \left<v_1 \mid v_n \right> & \left<v_1 \mid z \right>\\
							\vdots & \ddots & \vdots & \vdots\\
							\left<v_n  \mid v_1 \right> & \ldots & \left<v_n  \mid v_n \right> & \left<v_n  \mid z \right>\\
							\left<y+z  \mid v_1 \right> & \ldots & \left<y+z  \mid v_n \right> & \left<z  \mid y + z \right>
						\end{pmatrix}\\
					\end{align*}
					en appliquant soustrayant les $p$\/ premières colonnes, multipliées par $y_i$\/ : $C_{n+1} \gets C_{n+1} - \sum_{i=1}^n y_i\:C_i$, où les $C_i$\/ sont les colonnes de la matrice.
					Ainsi, on a \[
						G(v_1, \ldots, v_n, y + z)
						= \begin{pmatrix}
							\left<v_1 \mid v_1 \right> & \ldots & \left<v_1 \mid v_n \right> & 0\\
							\vdots & \ddots & \vdots & \vdots\\
							\left<v_n  \mid v_1 \right> & \ldots & \left<v_n  \mid v_n \right> & 0\\
							\left<y+z  \mid v_1 \right> & \ldots & \left<y+z  \mid v_n \right> & \left<z \mid z \right>
						\end{pmatrix}
					.\] Cette matrice est triangulaire par blocs, d'où, \[
						\det G(v_1, \ldots, v_n, y + z) = \det G(v_1, \ldots, v_n) \cdot \|z\|^2
					.\]
				\item Soit $x \in E$. On pose $y = p(x) \in F$\/ et $z = x - p(x) \in F^\perp$. D'où, d'après la question précédente, \[
					\mathrm{d}(x,F) = \|z\| = \sqrt{\frac{\det G(v_1, \ldots, v_n, x)}{\det G(v_1,\ldots,v_n)}}
				.\] La racine carrée est bien définie d'après la question (2b).
			\end{enumerate}
		\item
			\begin{enumerate}
				\item On remarque que, pour tout couple $(i,j) \in \llbracket 0,n - 1 \rrbracket^2$, on a \[
						\left<X^i  \mid X^j \right> = \int_{0}^{1} t^{i}\cdot t^{j}~\mathrm{d}t = \left[ \frac{t^{i+j+1}}{i+j+1} \right]_0^1 = \frac{1}{i + j + 1} = (H_n)_{i,j}
					.\]Ainsi, la matrice $H_n$\/ est donc la matrice de \textsc{Gram} pour le produit scalaire dans $\R_{n-1}[X]$\/ : $H_n = G(1, X, \ldots, X^{n-1})$.
					La famille $(1, X, \ldots, X^{n-1})$\/ étant une base de $\R_{n-1}[X]$, elle est libre, d'où $\det G(1, X, \ldots, X^n) \neq 0$, d'après la question (2a) car $\det A = \det_\mathcal{B}(1, X, \ldots, X^n)$, pour une base orthonormalisée $\mathcal{B}$. La matrice $H_n$\/ est donc inversible.
				\item D'après le théorème des moindres carrés, la fonction $P \in \R_{n-1}[X] \mapsto \|X^n-P\|$\/ atteint un minimum pour $P = p(X^n)$, où $p$\/ est la projection orthogonale de $\R_n[X]$\/ sur $\R_{n-1}[X]$.
					D'où, en posant $p(X^n) = a_0 + a_1 X + \cdots + a_{n-1}X^{n-1}$, la fonction $f$\/ admet un minimum avec $(a_0, a_1, \ldots, a_{n-1})$, les coefficients de $p(X^n)$. Avec ces coefficients, la valeur de $f$\/ est alors $\|X^n - p(X^n)\|^2$.
					Or, $\|X^n - p(X^n)\| = \mathrm{d}(X^n, \R_{n-1}[X])$, et, d'après la question (3c), on a, \[
						\|X^n - p(X^n)\|^2 = \frac{\det G(1, X, \ldots, X^{n-1}, X^n)}{\det G(1, X, \ldots, X^{n-1})} = \frac{\det H_{n+1}}{\det H_n}
					.\]
			\end{enumerate}
	\end{enumerate}

	\begin{center}
		\LARGE \textsc{Problème 3}
	\end{center}
	\begin{enumerate}
		\item
			\begin{enumerate}
				\item On considère la série entière $\sum \frac{x^n}{n!}$, dont la somme vaut la fonction $\exp$.
					La série $\sum \frac{1}{n!}$\/ converge, d'où $(\mathcal{P}_1)$.
					La limite $\lim_{x\to 1^-} \exp x$\/ existe et est finie ; elle vaut~$\mathrm{e}$, d'où $(\mathcal{P}_2)$.
				\item On considère la série entière $\sum (-x)^n$, qui converge vers la fonction $f: x \mapsto \frac{1}{1+x}$. La série $\sum (-1)^n$\/ diverge, elle ne vérifie donc pas $(\mathcal{P}_1)$. Mais, $f$\/ admet une limite finie en 1 : $f(1) = \frac{1}{2}$, d'où $(\mathcal{P}_2)$.
				\item On considère la série entière $\sum \frac{x^n}{n}$, qui converge vers la fonction $f:x\mapsto \ln(1-x)$. La série $\sum \frac{1}{n}$\/ diverge, elle ne vérifie donc pas $(\mathcal{P}_1)$.
					Et, $\lim_{x\to 1^-} \ln(1-x) = -\infty$, elle ne vérifie donc pas $(\mathcal{P}_2)$.
				\item 
			\end{enumerate}
	\end{enumerate}
\end{document}
