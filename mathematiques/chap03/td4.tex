\section{Exercice 6}\relax
{\slshape Quelle est la nature de l'intégrale \[
	A = \int_{0}^{+\infty} \left( x+2-\sqrt{x^2 + 4x + 1} \right)~\mathrm{d}x\ ?
\]}

L'intégrale $A$\/ est impropre en $+\infty$. On calcule donc $a(M) = \int_{0}^{M} \left( x + 2 - \sqrt{x^2 + 4x + 1} \right) ~\mathrm{d}x$\/ et on étudie sa limite quand $M$\/ tend vers $+\infty$.

Soit $x \in \R$. On pose $f : x \mapsto x + 2 + \sqrt{x^2 + 4x + 1}$, et on effectue un développement limité :
\begin{align*}
	f(x) &= x + 2 - \Big( 1 + (x^2 + 4x)\Big)^{\frac{1}{2}} \\
	&= x + 2 - 1 - \frac{1}{2}(x^2 + 4x) + \po(x) \\
	&= \frac{x^2}{2} + 1 + - x + \po(x)  \\
	&= \frac{x^2}{2} + \po(x^2) \\
\end{align*}
Or, l'intégrale $\int_{0}^{+\infty} \frac{x^2}{2}~\mathrm{d}x$\/ diverge par critère de {\sc Riemann}, et $\int_{0}^{+\infty} \po(x^2)~\mathrm{d}x$ aussi, et donc \[
	\boxed{\text{l'intégrale } A \text{ diverge}.}
\]

\begin{comment}
\section{Exercice 4}
\begin{enumerate}
	\item
		\begin{figure}[H]
			\centering
			\begin{asy}
				import graph;
				real f(real t) { return 1/t - floor(1/t);}
				bool3 fcheck(real t) { return t > 0 && f(t) < 10; }
				draw((0,0)--(10,0), Arrow(TeXHead));
				draw((0,0)--(0,10), Arrow(TeXHead));
				draw(graph(f, 0, 10, 500, fcheck), magenta);
				size(5cm);
			\end{asy}
			\caption{La fonction $\ds f(t) = \frac{1}{t} - \left\lfloor \frac{1}{t} \right\rfloor$}
		\end{figure}
		La fonction $t \mapsto \frac{1}{t}$\/ est continue par morceaux sur $]0,+\infty[$, et la fonction $u \mapsto \left\lfloor u \right\rfloor$\/ est continue par morceaux. Or, comme la composée et la différence de deux fonctions continues par morceaux est continue par morceaux, on en déduit que la fonction $f$\/ est continue par morceaux.
	\item
		L'intégrale $\int_{0}^{1} f(t)~\mathrm{d}t$.
\end{enumerate}
\end{comment}
