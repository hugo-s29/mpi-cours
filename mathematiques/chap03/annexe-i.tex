\section*{Annexe I : Bilan de la khôlle n\tsup{o}3}

\subsection*{Exercice 3}

L'intégrale $A = \int_{0}^{1} \frac{\ln t}{t - 1}~\mathrm{d}t$\/ est généralisée en 0 {\bf et}\/ en 1. Il est important d'écrire que \guillemotleft~l'intégrale $A$\/ converge si et seulement si les intégrales $\int_{0}^{1/2} \frac{\ln t}{t-1}~\mathrm{d}t$\/ et $\int_{1/2}^{1} \frac{\ln t}{t-1}~\mathrm{d}t$\/ convergent.~\guillemotright\ On tente un développement limité de la fonction dans l'intégrale $\int_{1/2}^{1} \frac{\ln t}{t - 1}~\mathrm{d}t$, mais il faut faire un changement de variables pour pouvoir faire un $\mathrm{DL}$\/ en 0.

On peut faire le changement de variables $t = 1 - u$, qui est strictement monotone. On a $\mathrm{d}t = - \mathrm{d}u$, d'où \[
	\int_{1 / 2}^{1} \frac{\ln t}{t - 1}~\mathrm{d}t = \int_{1 / 2}^{0} \frac{\ln (1-u)}{-u}~-\mathrm{d}u = - \int_{0}^{1 / 2} \frac{\ln(1-u)}{u}~\mathrm{d}u
.\]
Or, $\ln(1-u) = u + \po(u)$, et donc \[
	\frac{\ln(1-u)}{u} = \frac{-u+ \po(u)}{u} = -1 + \po(1) \tendsto{u \to 0} - 1
.\]
D'où, l'intégrale $\int_{0}^{1 / 2} \frac{\ln(1-u)}{u}~\mathrm{d}u$\/ est faussement impropre, donc elle converge.

Mais, au lieu de faire un changement de variables dans l'intégrale, il est préférable de faire le changement de variables dans la fonction et de montrer qu'elle est faussement impropre directement.
