\section{Exercice 8}

\paragraph{Q.\ 6}
On considère la fonction $f$\/ définie sur $\left[ 0,\frac{\pi}{2} \right]$\/ par \[
	f(0) = 0 \qquad\text{et}\qquad\forall t \in \left] 0,\frac{\pi}{2} \right],\:f(t) = \frac{1}{\sin t} - \frac{1}{t}
.\]
Montrons que $f$\/ est de classe $\mathscr{C}^1$.

\begin{enumerate}
	\item Étudions la limite de $f$\/ en $0^+_{\neq}$. En effet $f$\/ est continue en 0 si et seulement si $f(t) \tendsto{t\to 0} f(0)$. On fait un développement limité :
		\begin{align*}
			f(t) &= \frac{1}{t - \frac{t^3}{6} + \po(t^3)} - \frac{1}{t}\\
			&= \frac{1}{t} \times \Bigg( \frac{1}{1-\underbrace{\tfrac{t^2}{6} + \po(t^2)}_{\to 0}} - 1 \Bigg) \\
			&= \frac{1}{t} \times \left( 1 + \frac{t^2}{6} + \po(t^2) - 1 \right)  \\
			&= \frac{t}{6} + \po(t) \tendsto{t\to 0} 0 \\
		\end{align*}
		Or $f(0) = 0$. D'où $f(t) \tendsto{t\to 0} f(0)$.
	\item Étudions la dérivabilité de $f$\/ en 0 : soit $h > 0$, on calcule \[
			\frac{f(0+h) -f(0)}{h} = \frac{f(h)}{h} = \frac{1}{h\sin h} - \frac{1}{h^2} = \frac{\frac{h}{6} + \po(h)}{h} = \frac{1}{6} + \po(1) \tendsto{h\to 0} \frac{1}{6}
		.\]
		On en déduit que $f$\/ est dérivable en 0 et $f'(0) = \frac{1}{6}$.
	\item $f'$\/ est continue si et seulement si $f'(t) \tendsto{t\to 0} f'(0) = \frac{1}{6}$. On sait, comme $f$\/ est dérivable comme somme et composée de fonctions dérivables, d'où \[
		\forall t \in \left] 0, \frac{\pi}{2} \right],\:f'(t) = \frac{\mathrm{d}}{\mathrm{d}t} \left( \frac{1}{\sin t} - \frac{1}{t} \right) = - \frac{\cos t}{\sin^2 t} + \frac{1}{t^2}
	.\]
	On fait un développement limité de $f'$\/ :
	\begin{align*}
		f'(t) &= -\frac{1 - \frac{t^2}{2} + \po(t^2)}{\left( t - \frac{t^3}{6} + \po(t^3) \right)} + \frac{1}{t^2}\\
		&= \frac{1}{t^3} \left( - \frac{1-\frac{t^2}{2} + \po(t^2)}{\left(1 - \frac{t^2}{6} + \po(t^2)\right)^2} + 1 \right) \\
		&= \frac{1}{t^2} \left( -\left(1-\frac{t^2}{2} + \po(t^2)\right) \times \underset{\smash{\sim (1+\cdots)^{-2}}}{\boxed{\frac{1}{\left( 1-\frac{t^2}{6} + \po(t^2) \right)^2}}} + 1 \right) \\
		&= \frac{1}{t^2} \left( -\left( 1-\frac{t^2}{2} + \po(t^2) \right) \times \left( 1 + (-2)\left( -\frac{t^2}{6} \right) + \po(t^2) \right) \right) \\
		&= \frac{1}{t^2} \left( -\left( 1 + \left( \frac{1}{3} - \frac{1}{2} \right)t^2 + \po(t^2) \right) + 1 \right) \\
		&= \frac{1}{6} + \po(1) \\
		&\tendsto{t\to 0} \frac{1}{6}
	\end{align*}
	Or, $\frac{1}{6} = f'(0)$\/ d'après 2.\ d'où $f'(t) \tendsto{t\to 0} f'(0)$.
\end{enumerate}

On aurait pu ne pas faire la partie 1. En effet, la dérivabilité de $f$\/ implique sa continuité. On peut donc utiliser une autre méthode : montrer la continuité de $f$\/ puis, on montre d'un coup que $f$\/ est dérivable et que la dérivée est continue à l'aide du théorème de la limite de la dérivée.

\begin{rap}
	Soit $f$\/ continue sur $[a,b]$\/ et dérivable sur $]a,b[$. Alors, \[
		f'(t) \tendsto{t\to a} \ell \in \R \implies \begin{cases}
			f \text{ est dérivable en } a\\
			f'(a) = \ell\\
			f' \text{ est continue en } a.
		\end{cases}
	\]
\end{rap}


