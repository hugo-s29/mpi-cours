\begin{defn}
	Le nombre d'éléments inversibles de $\sfrac\Z{n\Z}$ (\textit{i.e.} le nombre d'entiers de $\llbracket 1,n \rrbracket$ premiers avec $n$) est noté $\varphi(n)$.
	L'application $\varphi : n \mapsto \varphi(n)$ est appelée l'\textit{indicatrice d'Euler}.
\end{defn}

\begin{exmn}
	On a $\varphi(8)$ car les entiers de $\llbracket 1,7 \rrbracket$ premiers avec $8$ sont 1, 3, 5, 7.
\end{exmn}

\begin{met}[Comment calculer l'indicatrice d'Euler]~\\[-2\baselineskip]
	\begin{enumerate}[label=(\textit{\roman*})]
		\item Si $p$ est premier, alors $\varphi(p) = p - 1$ car tous les éléments de $\llbracket 1,p-1 \rrbracket$ sont premiers avec $p$.
			Et, $\forall k \in \N^*$, on a $\varphi(p^k) = p^k \cdot (1 - \sfrac{1}{p})$.
		\item Si $a$ et $b$ sont premiers entre eux, alors $\varphi(ab) = \varphi(a) \cdot \varphi(b)$. En effet, d'après le théorème chinois, il y a autant d'éléments inversibles dans $\sfrac\Z{(ab)\Z}$ et dans $\sfrac\Z{a\Z} \times \sfrac\Z{b\Z}$ par isomorphisme.
		\item Si $p_1, \ldots, p_k$ sont les diviseurs premiers de $n$, alors \[
				\varphi(n) = n \times \left( 1 - \frac{1}{p_1} \right) \times \cdots \times \left( 1 - \frac{1}{p_k} \right)
			.\]En effet, on a $n = p_1^{\alpha_1} \times \cdots \times p_k^{\alpha_k}$, d'où $\varphi(n) = \varphi(p_1^{\alpha_1}) \cdot \ldots \cdot \varphi(p_k^{\alpha_k})$.
			Calculons $\varphi(p_1^{\alpha_1})$ : on cherche tous les entiers de $\llbracket 1,p_1^{\alpha_1} \rrbracket$ premiers avec $p_1^{\alpha_1}$. On cherche donc tous les entiers $\llbracket 1,p_1^{\alpha_1} \rrbracket$. Les multiples de $p_1$ dans $\llbracket 1,p_1^{\alpha_1} \rrbracket$ sont $p_1$, $2p_1$, \ldots, $p_1^{\alpha_1 - 1} \times p_1$ : il y en a $p_1^{\alpha_1 - 1}$.
			Il y a donc $p_1^{\alpha_1} - p_1^{\alpha_1 - 1}$ non multiples de $p_1$.
			D'où, $\varphi(p_1^{\alpha_1}) = p_1^{\alpha_1} \cdot (1 - \sfrac 1{p_1})$.
			Ainsi, on en déduit la formule de $\varphi(n)$ précédente.
	\end{enumerate}
\end{met}

\section{L'ordre d'un élément}

Si $a$ est un élément d'un groupe $(G, \cdot)$ d'élément neutre $1$, alors l'ensemble  $\{\ldots, a^{-2}, a^{-1}, 1, a^1, a^2, \ldots\} = \{a^k  \mid k \in \Z\}$ est un sous-groupe de $G$, appelé le sous-groupe engendré par $a$, et il est noté $\langle a\rangle$. C'est le plus petit sous-groupe de $G$ contenant $a$.

Si ce sous-groupe est un ensemble fini, alors son cardinal est appelé l'\textit{ordre} de $a$. L'ordre de $a$ est le plus petit entier $k$ strictement positif tel que $a^k = 1$. Et, les entiers $k$ tels que $a^k = 1$ sont les multiples de l'ordre de $a$. On dit que le groupe $G$ est \textit{monogène} s'il est, lui-même, engendré par un élément et qu'il est \textit{cyclique} s'il est monogène et fini.

\begin{exo}
	\begin{slshape}
		Décomposer en cycle disjoints la permutation $\sigma$ du groupe symétrique $\mathfrak{S}_7$ et en déduire l'ordre de $\sigma$, où \[
			\sigma = \begin{pmatrix}
				1 & 2 & 3 & 4 & 5 & 6 & 7\\
				5 & 4 & 7 & 2 & 6 & 1 & 3
			\end{pmatrix}
		.\]
	\end{slshape}

	Les images successives de $1$ sont $1$, $5$ et $6$ ; celles de $2$ sont $2$ et $4$ ; celles de $3$ sont $3$ et $7$.
	D'où, $\sigma = \begin{pmatrix}
		1 & 5 & 6
	\end{pmatrix} \cdot \begin{pmatrix}
		2 & 4
	\end{pmatrix} \cdot \begin{pmatrix}
		3 & 7
	\end{pmatrix}$.
	L'ordre de $\sigma$ est 6, ce qui correspond au plus petit commun multiple des ordres des cycles (donc $\mathrm{ppcm}(2, 2, 3)$).

	\begin{figure}[H]
		\centering
		\begin{asy}
size(5cm);

real rho = 0.15; // circles

void draw_cycle(pair O, real r ...int[] nums) {
	int n = nums.length;
	real eps = (15 / r) * 0.8;

	for(int i = 0; i < n; ++i) {
		real theta_1 = (360/n) * (i+1);
		real theta_2 = (360/n) * i;

		pair C = O + dir(theta_2) * r;

		draw(circle(C, rho));
		label("$" + string(nums[i]) + "$", C);
		draw(arc(O, r, theta_2+eps, theta_1-eps), Arrow(TeXHead));
	}
}

draw_cycle((0,0), 0.4, 1, 5, 6);
draw_cycle((1, 0), 0.2, 2, 4);
draw_cycle((2, 0), 0.2, 3, 7);
		\end{asy}
		\caption{Décomposition en cycles de la permutation $\sigma$}
	\end{figure}
\end{exo}

\begin{exmn}
	Le groupe $(\Z, +)$ est monogène car $\Z = \langle 1 \rangle$.
	Mais, $(\mathfrak{S}_n, \cdot)$ n'est pas monogène.
\end{exmn}

\begin{prop}
	Si $G$ est un groupe fini, alors $\forall a \in G$, $o(a)  \mid \Card G$, où $o(a)$ est l'ordre de $a$.\qed
\end{prop}

\begin{crlr}
	\begin{description}
		\item[Théorème d'Euler.] Si $a \in \Z$ est premier avec $n \in \N^*$, alors $a^{\varphi(n)} \equiv 1 \mod n$.
		\item[Petit théorème de Fermat.] Si $p$ est un nombre premier, alors $\forall a \in \Z$, on a $a^p \equiv a \mod p$.
	\end{description}
\end{crlr}

\begin{exo}
	\begin{slshape}
		Calculer $\varphi(10)$ et en déduire que le dernier chiffre de l'écriture décimale de $3^{345}$ est $3$.
		Calculer $\varphi(100)$ et en déduire que les deux derniers chiffres de l'écriture décimale de $3^{345}$ sont $4$ et $3$.
	\end{slshape}

	On trouve $\varphi(10) = 4$ car les entiers de $\llbracket 1,10 \rrbracket$ premiers avec $10$ sont $1$, $3$, $7$ et $9$.
	On trouve aussi $\varphi(100) = 40$ car $100 = 4 \times 25 = 2^2 \times 5^2$, d'où les diviseurs premiers de $100$ sont $2$ et $5$, et donc $\varphi(100) = 100 \times \left(1 - \frac{1}{2}\right) \times \left( 1 - \frac{1}{5} \right) = 50 \times 4 / 5 = 40$.

	On cherche $r \in \llbracket 0,9 \rrbracket$ tel que $3^{385} \equiv 1 \mod {10}$.
	On a $3 \wedge 10 = 1$, d'où, d'après le théorème d'Euler, $3^{\varphi(10)} \equiv 1 \mod 10$ et donc $3^4 \equiv 1 \mod 10$.
	Or, \[
		3^{345} = 3^{344} \times 3 = (3^4)^{86} \times 3 \equiv 1^{86} \times 3 \equiv 3 \mod {10}
	.\]Donc $r = 3$.

	On cherche $r \in \llbracket 0,99 \rrbracket$ tel que $3^{345} \equiv r \mod {100}$.
	De même, d'après le théorème d'Euler, $3^{\varphi(100)} \equiv 1 \mod{100}$, d'où $3^{40} \equiv 1 \mod {100}$.
	Or, $3^{345} = (3^{40})^{8} \times 3^{25}\equiv 3^{25} \mod{100}$.
	Et, $3^{25} = 3^5 \times (3^5)^4 = 3 \cdot 81 \cdot (3^5)^4 \equiv 43 \times (3^4)^5 \equiv 43 \mod {100}$.
	D'où, $3^{345} \equiv 43 \mod {100}$.
\end{exo}

