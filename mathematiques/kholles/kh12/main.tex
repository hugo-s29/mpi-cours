\documentclass[a4paper]{article}

\usepackage[margin=1in]{geometry}
\usepackage[utf8]{inputenc}
\usepackage[T1]{fontenc}
\usepackage{mathrsfs}
\usepackage{textcomp}
\usepackage[french]{babel}
\usepackage{amsmath}
\usepackage{amssymb}
\usepackage{cancel}
\usepackage{frcursive}
\usepackage[inline]{asymptote}
\usepackage{tikz}
\usepackage[european,straightvoltages,europeanresistors]{circuitikz}
\usepackage{tikz-cd}
\usepackage{tkz-tab}
\usepackage[b]{esvect}
\usepackage[framemethod=TikZ]{mdframed}
\usepackage{centernot}
\usepackage{diagbox}
\usepackage{dsfont}
\usepackage{fancyhdr}
\usepackage{float}
\usepackage{graphicx}
\usepackage{listings}
\usepackage{multicol}
\usepackage{nicematrix}
\usepackage{pdflscape}
\usepackage{stmaryrd}
\usepackage{xfrac}
\usepackage{hep-math-font}
\usepackage{amsthm}
\usepackage{thmtools}
\usepackage{indentfirst}
\usepackage[framemethod=TikZ]{mdframed}
\usepackage{accents}
\usepackage{soulutf8}
\usepackage{mathtools}
\usepackage{bodegraph}
\usepackage{slashbox}
\usepackage{enumitem}
\usepackage{calligra}
\usepackage{cinzel}
\usepackage{BOONDOX-calo}

% Tikz
\usetikzlibrary{babel}
\usetikzlibrary{positioning}
\usetikzlibrary{calc}

% global settings
\frenchspacing
\reversemarginpar
\setuldepth{a}

%\everymath{\displaystyle}

\frenchbsetup{StandardLists=true}

\def\asydir{asy}

%\sisetup{exponent-product=\cdot,output-decimal-marker={,},separate-uncertainty,range-phrase=\;à\;,locale=FR}

\setlength{\parskip}{1em}

\theoremstyle{definition}

% Changing math
\let\emptyset\varnothing
\let\ge\geqslant
\let\le\leqslant
\let\preceq\preccurlyeq
\let\succeq\succcurlyeq
\let\ds\displaystyle
\let\ts\textstyle

\newcommand{\C}{\mathds{C}}
\newcommand{\R}{\mathds{R}}
\newcommand{\Z}{\mathds{Z}}
\newcommand{\N}{\mathds{N}}
\newcommand{\Q}{\mathds{Q}}

\renewcommand{\O}{\emptyset}

\newcommand\ubar[1]{\underaccent{\bar}{#1}}

\renewcommand\Re{\expandafter\mathfrak{Re}}
\renewcommand\Im{\expandafter\mathfrak{Im}}

\let\slantedpartial\partial
\DeclareRobustCommand{\partial}{\text{\rotatebox[origin=t]{20}{\scalebox{0.95}[1]{$\slantedpartial$}}}\hspace{-1pt}}

% merging two maths characters w/ \charfusion
\makeatletter
\def\moverlay{\mathpalette\mov@rlay}
\def\mov@rlay#1#2{\leavevmode\vtop{%
   \baselineskip\z@skip \lineskiplimit-\maxdimen
   \ialign{\hfil$\m@th#1##$\hfil\cr#2\crcr}}}
\newcommand{\charfusion}[3][\mathord]{
    #1{\ifx#1\mathop\vphantom{#2}\fi
        \mathpalette\mov@rlay{#2\cr#3}
      }
    \ifx#1\mathop\expandafter\displaylimits\fi}
\makeatother

% custom math commands
\newcommand{\T}{{\!\!\,\top}}
\newcommand{\avrt}[1]{\rotatebox{-90}{$#1$}}
\newcommand{\bigcupdot}{\charfusion[\mathop]{\bigcup}{\cdot}}
\newcommand{\cupdot}{\charfusion[\mathbin]{\cup}{\cdot}}
%\newcommand{\danger}{{\large\fontencoding{U}\fontfamily{futs}\selectfont\char 66\relax}\;}
\newcommand{\tendsto}[1]{\xrightarrow[#1]{}}
\newcommand{\vrt}[1]{\rotatebox{90}{$#1$}}
\newcommand{\tsup}[1]{\textsuperscript{\underline{#1}}}
\newcommand{\tsub}[1]{\textsubscript{#1}}

\renewcommand{\mod}[1]{~\left[ #1 \right]}
\renewcommand{\t}{{}^t\!}
\newcommand{\s}{\text{\calligra s}}

% custom units / constants
%\DeclareSIUnit{\litre}{\ell}
\let\hbar\hslash

% header / footer
\pagestyle{fancy}
\fancyhead{} \fancyfoot{}
\fancyfoot[C]{\thepage}

% fonts
\let\sc\scshape
\let\bf\bfseries
\let\it\itshape
\let\sl\slshape

% custom math operators
\let\th\relax
\let\det\relax
\DeclareMathOperator*{\codim}{codim}
\DeclareMathOperator*{\dom}{dom}
\DeclareMathOperator*{\gO}{O}
\DeclareMathOperator*{\po}{\text{\cursive o}}
\DeclareMathOperator*{\sgn}{sgn}
\DeclareMathOperator*{\simi}{\sim}
\DeclareMathOperator{\Arccos}{Arccos}
\DeclareMathOperator{\Arcsin}{Arcsin}
\DeclareMathOperator{\Arctan}{Arctan}
\DeclareMathOperator{\Argsh}{Argsh}
\DeclareMathOperator{\Arg}{Arg}
\DeclareMathOperator{\Aut}{Aut}
\DeclareMathOperator{\Card}{Card}
\DeclareMathOperator{\Cl}{\mathcal{C}\!\ell}
\DeclareMathOperator{\Cov}{Cov}
\DeclareMathOperator{\Ker}{Ker}
\DeclareMathOperator{\Mat}{Mat}
\DeclareMathOperator{\PGCD}{PGCD}
\DeclareMathOperator{\PPCM}{PPCM}
\DeclareMathOperator{\Supp}{Supp}
\DeclareMathOperator{\Vect}{Vect}
\DeclareMathOperator{\argmax}{argmax}
\DeclareMathOperator{\argmin}{argmin}
\DeclareMathOperator{\ch}{ch}
\DeclareMathOperator{\com}{com}
\DeclareMathOperator{\cotan}{cotan}
\DeclareMathOperator{\det}{det}
\DeclareMathOperator{\id}{id}
\DeclareMathOperator{\rg}{rg}
\DeclareMathOperator{\rk}{rk}
\DeclareMathOperator{\sh}{sh}
\DeclareMathOperator{\th}{th}
\DeclareMathOperator{\tr}{tr}

% colors and page style
\definecolor{truewhite}{HTML}{ffffff}
\definecolor{white}{HTML}{faf4ed}
\definecolor{trueblack}{HTML}{000000}
\definecolor{black}{HTML}{575279}
\definecolor{mauve}{HTML}{907aa9}
\definecolor{blue}{HTML}{286983}
\definecolor{red}{HTML}{d7827e}
\definecolor{yellow}{HTML}{ea9d34}
\definecolor{gray}{HTML}{9893a5}
\definecolor{grey}{HTML}{9893a5}
\definecolor{green}{HTML}{a0d971}

\pagecolor{white}
\color{black}

\begin{asydef}
	settings.prc = false;
	settings.render=0;

	white = rgb("faf4ed");
	black = rgb("575279");
	blue = rgb("286983");
	red = rgb("d7827e");
	yellow = rgb("f6c177");
	orange = rgb("ea9d34");
	gray = rgb("9893a5");
	grey = rgb("9893a5");
	deepcyan = rgb("56949f");
	pink = rgb("b4637a");
	magenta = rgb("eb6f92");
	green = rgb("a0d971");
	purple = rgb("907aa9");

	defaultpen(black + fontsize(8pt));

	import three;
	currentlight = nolight;
\end{asydef}

% theorems, proofs, ...

\mdfsetup{skipabove=1em,skipbelow=1em, innertopmargin=6pt, innerbottommargin=6pt,}

\declaretheoremstyle[
	headfont=\normalfont\itshape,
	numbered=no,
	postheadspace=\newline,
	headpunct={:},
	qed=\qedsymbol]{demstyle}

\declaretheorem[style=demstyle, name=Démonstration]{dem}

\newcommand\veczero{\kern-1.2pt\vec{\kern1.2pt 0}} % \vec{0} looks weird since the `0' isn't italicized

\makeatletter
\renewcommand{\title}[2]{
	\AtBeginDocument{
		\begin{titlepage}
			\begin{center}
				\vspace{10cm}
				{\Large \sc Chapitre #1}\\
				\vspace{1cm}
				{\Huge \calligra #2}\\
				\vfill
				Hugo {\sc Salou} MPI${}^{\star}$\\
				{\small Dernière mise à jour le \@date }
			\end{center}
		\end{titlepage}
	}
}

\newcommand{\titletp}[4]{
	\AtBeginDocument{
		\begin{titlepage}
			\begin{center}
				\vspace{10cm}
				{\Large \sc tp #1}\\
				\vspace{1cm}
				{\Huge \textsc{\textit{#2}}}\\
				\vfill
				{#3}\textit{MPI}${}^{\star}$\\
			\end{center}
		\end{titlepage}
	}
	\fancyfoot{}\fancyhead{}
	\fancyfoot[R]{#4 \textit{MPI}${}^{\star}$}
	\fancyhead[C]{{\sc tp #1} : #2}
	\fancyhead[R]{\thepage}
}

\newcommand{\titletd}[2]{
	\AtBeginDocument{
		\begin{titlepage}
			\begin{center}
				\vspace{10cm}
				{\Large \sc td #1}\\
				\vspace{1cm}
				{\Huge \calligra #2}\\
				\vfill
				Hugo {\sc Salou} MPI${}^{\star}$\\
				{\small Dernière mise à jour le \@date }
			\end{center}
		\end{titlepage}
	}
}
\makeatother

\newcommand{\sign}{
	\null
	\vfill
	\begin{center}
		{
			\fontfamily{ccr}\selectfont
			\textit{\textbf{\.{\"i}}}
		}
	\end{center}
	\vfill
	\null
}

\renewcommand{\thefootnote}{\emph{\alph{footnote}}}

% figure support
\usepackage{import}
\usepackage{xifthen}
\pdfminorversion=7
\usepackage{pdfpages}
\usepackage{transparent}
\newcommand{\incfig}[1]{%
	\def\svgwidth{\columnwidth}
	\import{./figures/}{#1.pdf_tex}
}

\pdfsuppresswarningpagegroup=1
\ctikzset{tripoles/european not symbol=circle}

\newcommand{\missingpart}{{\large\color{red} Il manque quelque chose ici\ldots}}


\def\khollenum{12}

\fancyhead[L]{Hugo {\sc Salou}\/ MPI$^\star$}
\fancyhead[R]{\scshape Khôlle n\tsup{o} \khollenum}

\begin{document}
	\begin{center}
		\bfseries\scshape\Huge Khôlle n\tsup{o} \khollenum
	\end{center}

	\paragraph{Exercice 2.}
	Posons les événements $P_n$\/ \guillemotleft~le 1\tsup{er} \textsc{Pile} apparaît au $n$-ième lancer,~\guillemotright\ et $B$\/ \guillemotleft~tirer une boule blanche.~\guillemotright\@ Les événements $(P_n)_{n \in \N}$\/ forment un système quasi-complet d'événements, d'où, d'après la formule des probabilités totales,
	\begin{align*}
		P(B) &= \sum_{k=1}^\infty P(B \cap P_n) \\
		&= \sum_{k=1}^\infty P(B  \mid P_n) \times P(P_n)\\
	\end{align*}
	Or, par équiprobabilité, pour tout $n \in \N$, $P(P_n) = \left(\frac{1}{2}\right)^n$.
	De plus et aussi par équiprobabilité, pour tout $n \in \N$, $P(B  \mid P_n) = \frac{1}{n}$ car il y a une seule boule blanche, et $n - 1$\/ boules noires.
	On en déduit donc que \[
		P(B) = \sum_{k=1}^\infty \frac{1}{n 2^n}
	.\] On reconnaît le développement en série entière de $-\ln(1 - x)$, dans le cas $x = \frac{1}{2}$. Or, le rayon de convergence de cette série est de 1, et $\frac{1}{2} \in {]{-1},1[}$, la série numérique $\sum \frac{1}{n 2^n}$\/ converge donc. On en déduit que \[
		\boxed{P(B) = -\ln\left(1 - \frac{1}{2}\right) = -\ln \frac{1}{2} = \ln 2.}
	\]

	\paragraph{Exercice 3.}
	\begin{enumerate}
		\item Soit $x \in {]-R,R[}$. On calcule :
			\begin{align*}
				(1 - x - x^2) \cdot f(x) &= (1 - x - x^2) \sum_{n=0}^\infty u_n x^n \\
				&= \sum_{n = 0}^\infty u_n x^n - \sum_{n=0}^\infty u_n x^{n+1} -\sum_{n=0}^\infty u_n x^{n+2} \\
				&= \sum_{n=0}^\infty u_n x^n - \sum_{n=1}^\infty u_{n-1} x^n - \sum_{n=2}^\infty u_{n-2} x^n \\
				&= \sum_{n=2}^\infty (u_n - u_{n-1} - u_{n-2})x^n + u_0 - u_1 + u_1 x \\
				&= \sum_{n=0}^\infty 0\cdot x^n + u_0 - u_1 + u_1x \\
				&= u_0 + u_1(x - 1) \\
			\end{align*}
			On en déduit que \[
				\boxed{(1 - x - x^2) \cdot f(x) = x - 1.}
			\]
		\item Prouvons le par récurrence forte : posons $P_n$\/ le prédicat \guillemotleft~$u_n \le 2^n$.~\guillemotright\ 
			\begin{itemize}
				\item On a bien $u_0 = 0 \le 2^0 = 1$, d'où $P_0$.
				\item On a bien $u_1 = 1 \le 2^1 = 1$, d'où $P_0$.
				\item Soit $n \in \N$, avec $n \ge 1$. Supposons, pour $k \le n$, $P_k$ vrai. Montrons $P_{n+1}$.
					Par définition de la suite, on a \[
						u_{n+1} = u_n + u_{n-1} \le 2^n + 2^{n-1} \le 2^n + 2^n \le 2^{n+1}
					,\] d'où $P_{n+1}$.
			\end{itemize}
			Par récurrence forte, on en déduit que, \ul{pour tout $n \in \N$, $u_n \le 2^n$}.
			Or, la série entière $\sum 2^n x^n = \sum (2x)^n$ est une série géométrique, dont le rayon de convergence vaut $\frac{1}{2}$. On en déduit que \[
				\boxed{R \ge \frac{1}{2}.}
			\]
		\item Résolvons l'équation caractéristique de la suite $(u_n)$\/ : \[
					(C) : \quad\quad z^2 - z - 1 = 0
			.\] Le discriminant du polynôme $X^2 - X - 1$\/ vaut $\Delta = 5 > 0$ ; les solutions $\varphi$\/ et $\psi$\/ sont donc \[
				\varphi = \frac{1 + \sqrt{5}}{2} \quad\quad \quad \psi = \frac{1 - \sqrt{5}}{2}
			.\] Ainsi, il existe $A$\/ et $B$\/ deux réels tels que, pour $n \in \N$, \[
				u_n = A \varphi^n + B \psi^n
			.\]  Or, $u_0 = A + B = 0$, et \[
				u_1 = A\varphi + B \psi = \frac{A + B + \sqrt{5} (A - B)}{2} = (A-B) \cdot \frac{\sqrt{5}}{2} = 1
			.\] D'où $B = -A$, et donc $A - B = 2A = 2 / \sqrt{5}$. Ainsi, on en déduit que  \[
				A = \frac{1}{\sqrt{5}} \quad\quad \text{ et } \quad\quad B = -\frac{1}{\sqrt{5}}
			.\] On en déduit le terme général de la suite $(u_n)_{n\in\N}$\/ : \[
				\forall n \in \N,\quad u_n = \frac{1}{\sqrt{5}} \cdot (\varphi^n - \psi^n)
			.\]
			D'où, $\sum u_n x^n = \frac{1}{\sqrt{5}} \sum \varphi^n\,x^n - \frac{1}{\sqrt{5}} \sum \psi^n\,x^n = \frac{1}{\sqrt{5}}\big(\sum (\varphi x)^n - \sum (\psi x)^n\big)$. La série entière $\sum (\varphi x)^n$\/ est géométrique, et a pour rayon de convergence $\frac{1}{\varphi}$\/ ; de même, la série entière $\sum (\psi x)^n$\/ a pour rayon de convergence $\frac{1}{\psi}$.
			Comme $\frac{1}{\varphi} \neq \frac{1}{\psi}$, on en déduit que le rayon de convergence de la série entière $\sum u_n x^n$\/ vaut  \[
				\boxed{R = \min\left(\frac{1}{\varphi}, \frac{1}{\psi}\right) = -\frac{1 - \sqrt{5}}{2}.}
			\]
	\end{enumerate}

	\paragraph{Exercice 1.}
	\begin{enumerate}
		\item Soit $S \in \mathcal{S}_n(\R)$, et soit $M \in \mathcal{M}_n(\R)$.
			On sait que $\mathcal{S}_n(\R) \perp \mathcal{A}_n(\R)$, et $\mathcal{S}_n(\R) \oplus \mathcal{A}_n(\R)$.
			Et, nous avons l'égalité \[
			(\star)\quad\quad M = \underbrace{\frac{M - M^\top}{2}}_{\in \mathcal{A}_n(\R)} + \underbrace{\frac{M+M^\top}{2}}_{\in \mathcal{S}_n(\R)}
			.\] D'après le théorème de \textsc{Pythagore}, \[
				\|M - S\|^2 = \left\| \frac{M-M^\top}{2}\right\|^2 + \left\|\frac{M+M^\top}{2} - S\right\|^2 \le \left\| \frac{M-M^\top}{2}\right\|^2
			.\] Ainsi, comme la norme est positive ou nulle, et par croissance de la fonction racine carrée, on en déduit que \[
				\boxed{\|M - S\| \le \left\| \frac{M-M^\top}{2}\right\|.}
			\]
		\item L'inéquation ci-dessus est vraie pour toute matrice symétrique $S$. En particulier, si $S = \frac{M - M^\top}{2} \in \mathcal{S}_n(\R)$, alors \[
			M - S = \frac{M - M^\top}{2} \quad\quad \text{ d'où } \quad\quad \|M - S\| = \left\|\frac{M - M^\top}{2}\right\|,
		\] d'après $(\star)$. Or, par définition $d\big(M, \mathcal{S}_n(\R)\big)$ est le minimum des normes $\|M - S\|$, d'où \[
			\boxed{d\big(M, \mathcal{S}_n(\R)\big) = \left\|\frac{M - M^\top}{2}\right\|.}
		\] 
	\end{enumerate}

	\begin{comment}
	\paragraph{Exercice 4.}
	\def\N{\ensuremath{\mathds{N} \setminus \{0,1\}}} % douteux, mais ça marche
	\begin{enumerate}
		\item Soit $(u_n)_{n\in\N}$\/ le terme général de la série $\sum \frac{1}{n^a + (-1)^n}$. On a, si $a > 1$, \[
				\frac{1}{n^a}\simi_{n\to +\infty} \frac{1}{n^a + 1} \ge u_n \ge \frac{1}{n^a - 1} \simi_{n\to +\infty} \frac{1}{n^a}
			.\] La série de \textsc{Riemann} $\sum \frac{1}{n^a}$\/ converge si, et seulement si $a > 1$. Ainsi, si $a > 1$, la série $\sum u_n$\/ converge, car, à partir d'un certain rang $N$, la suite $(u_n)_{n \ge N}$\/ est positive.

			Si $a = 1$, alors $u_n = \frac{1}{n + (-1)^n} \sim \frac{1}{n}$, et la série $\sum \frac{1}{n}$\/ ne converge pas. On en déduit que la série $\sum u_n$\/ diverge.

			Si $a < 1$, alors la série $\sum u_n$\/ diverge grossièrement. En effet, on considère les deux sous-suites $(u_{2n})_{n \in \N}$\/ et $(u_{2n + 1})_{n \in \N}$ : 
			\[
				u_{2n} = \frac{1}{(2n)^a + 1} \tendsto{n\to +\infty} 1
				\quad\quad \text{ et }\quad\quad
				u_{2n+1} = \frac{1}{(2n+1)^a - 1} \tendsto{n\to +\infty} -1
			.\] Par unicité de la limite, la suite $(u_n)_{n\in\N}$\/ ne converge pas vers 0, la série $\sum u_n$\/ diverge grossièrement.
		\item Soit $(v_n)_{n \in \N}$\/ le terme général de la série $\sum \frac{(-1)^n}{n^a + (-1)^n}$.

			Si $a = 1$, alors $v_n = (-1)^n\frac{1}{n + (-1)^n} \sim \frac{(-1)^n}{n}$, et la série $\sum \frac{(-1)^n}{n}$\/ converge (vers $\ln 2$). On en déduit que la série $\sum v_n$\/ converge.

			Si $a > 1$, on considère le rang $N$\/ (défini à la question précédente, tel que la série $(u_n)_{n \ge N}$\/ soit positive.
			Ainsi, $(|v_n|)_{n \ge N} = (u_n)_{n \ge N}$, d'où la série $\sum\:|v_n|$\/ converge d'après la question précédente, la série $\sum v_n$\/ converge donc.

			Si $a < 1$, alors on considère les sous-suites $(v_{2n})_{n\in\N}$\/ et $(v_{2n + 1})_{n\in\N}$ : \[
				v_{2n} = \frac{1}{(2n)^a + 1} \tendsto{n \to +\infty} 1
				\quad\quad \text{ et } \quad\quad
				v_{2n+1} = \frac{-1}{(2n+1)^a - 1} \tendsto{n \to +\infty} 1
			.\] Par unicité de la limite, la suite $(v_n)_{n \in \N}$\/ converge vers $1 \neq 0$. La série $\sum v_n$\/ diverge donc grossièrement.
		\item Soit $R$\/ le rayon de convergence de la série $\sum u_n x^n$.
			Soit $x \neq 0$. On calcule :
			\[
				\frac{|u_{n+1}\:x^{n+1}|}{|u_n\:x^n|} \sim \left(\frac{n}{n+1}\right)^\alpha\:|x|
				= \left(\frac{1}{1 + \frac{1}{n}}\right)^\alpha\:|x| \tendsto{n\to +\infty} |x|
			.\]
	\end{enumerate}
	\end{comment}
\end{document}


