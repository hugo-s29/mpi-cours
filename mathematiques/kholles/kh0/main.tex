\documentclass[a4paper]{article}

\usepackage[margin=1in]{geometry}
\usepackage[utf8]{inputenc}
\usepackage[T1]{fontenc}
\usepackage{mathrsfs}
\usepackage{textcomp}
\usepackage[french]{babel}
\usepackage{amsmath}
\usepackage{amssymb}
\usepackage{cancel}
\usepackage{frcursive}
\usepackage[inline]{asymptote}
\usepackage{tikz}
\usepackage[european,straightvoltages,europeanresistors]{circuitikz}
\usepackage{tikz-cd}
\usepackage{tkz-tab}
\usepackage[b]{esvect}
\usepackage[framemethod=TikZ]{mdframed}
\usepackage{centernot}
\usepackage{diagbox}
\usepackage{dsfont}
\usepackage{fancyhdr}
\usepackage{float}
\usepackage{graphicx}
\usepackage{listings}
\usepackage{multicol}
\usepackage{nicematrix}
\usepackage{pdflscape}
\usepackage{stmaryrd}
\usepackage{xfrac}
\usepackage{hep-math-font}
\usepackage{amsthm}
\usepackage{thmtools}
\usepackage{indentfirst}
\usepackage[framemethod=TikZ]{mdframed}
\usepackage{accents}
\usepackage{soulutf8}
\usepackage{mathtools}
\usepackage{bodegraph}
\usepackage{slashbox}
\usepackage{enumitem}
\usepackage{calligra}
\usepackage{cinzel}
\usepackage{BOONDOX-calo}

% Tikz
\usetikzlibrary{babel}
\usetikzlibrary{positioning}
\usetikzlibrary{calc}

% global settings
\frenchspacing
\reversemarginpar
\setuldepth{a}

%\everymath{\displaystyle}

\frenchbsetup{StandardLists=true}

\def\asydir{asy}

%\sisetup{exponent-product=\cdot,output-decimal-marker={,},separate-uncertainty,range-phrase=\;à\;,locale=FR}

\setlength{\parskip}{1em}

\theoremstyle{definition}

% Changing math
\let\emptyset\varnothing
\let\ge\geqslant
\let\le\leqslant
\let\preceq\preccurlyeq
\let\succeq\succcurlyeq
\let\ds\displaystyle
\let\ts\textstyle

\newcommand{\C}{\mathds{C}}
\newcommand{\R}{\mathds{R}}
\newcommand{\Z}{\mathds{Z}}
\newcommand{\N}{\mathds{N}}
\newcommand{\Q}{\mathds{Q}}

\renewcommand{\O}{\emptyset}

\newcommand\ubar[1]{\underaccent{\bar}{#1}}

\renewcommand\Re{\expandafter\mathfrak{Re}}
\renewcommand\Im{\expandafter\mathfrak{Im}}

\let\slantedpartial\partial
\DeclareRobustCommand{\partial}{\text{\rotatebox[origin=t]{20}{\scalebox{0.95}[1]{$\slantedpartial$}}}\hspace{-1pt}}

% merging two maths characters w/ \charfusion
\makeatletter
\def\moverlay{\mathpalette\mov@rlay}
\def\mov@rlay#1#2{\leavevmode\vtop{%
   \baselineskip\z@skip \lineskiplimit-\maxdimen
   \ialign{\hfil$\m@th#1##$\hfil\cr#2\crcr}}}
\newcommand{\charfusion}[3][\mathord]{
    #1{\ifx#1\mathop\vphantom{#2}\fi
        \mathpalette\mov@rlay{#2\cr#3}
      }
    \ifx#1\mathop\expandafter\displaylimits\fi}
\makeatother

% custom math commands
\newcommand{\T}{{\!\!\,\top}}
\newcommand{\avrt}[1]{\rotatebox{-90}{$#1$}}
\newcommand{\bigcupdot}{\charfusion[\mathop]{\bigcup}{\cdot}}
\newcommand{\cupdot}{\charfusion[\mathbin]{\cup}{\cdot}}
%\newcommand{\danger}{{\large\fontencoding{U}\fontfamily{futs}\selectfont\char 66\relax}\;}
\newcommand{\tendsto}[1]{\xrightarrow[#1]{}}
\newcommand{\vrt}[1]{\rotatebox{90}{$#1$}}
\newcommand{\tsup}[1]{\textsuperscript{\underline{#1}}}
\newcommand{\tsub}[1]{\textsubscript{#1}}

\renewcommand{\mod}[1]{~\left[ #1 \right]}
\renewcommand{\t}{{}^t\!}
\newcommand{\s}{\text{\calligra s}}

% custom units / constants
%\DeclareSIUnit{\litre}{\ell}
\let\hbar\hslash

% header / footer
\pagestyle{fancy}
\fancyhead{} \fancyfoot{}
\fancyfoot[C]{\thepage}

% fonts
\let\sc\scshape
\let\bf\bfseries
\let\it\itshape
\let\sl\slshape

% custom math operators
\let\th\relax
\let\det\relax
\DeclareMathOperator*{\codim}{codim}
\DeclareMathOperator*{\dom}{dom}
\DeclareMathOperator*{\gO}{O}
\DeclareMathOperator*{\po}{\text{\cursive o}}
\DeclareMathOperator*{\sgn}{sgn}
\DeclareMathOperator*{\simi}{\sim}
\DeclareMathOperator{\Arccos}{Arccos}
\DeclareMathOperator{\Arcsin}{Arcsin}
\DeclareMathOperator{\Arctan}{Arctan}
\DeclareMathOperator{\Argsh}{Argsh}
\DeclareMathOperator{\Arg}{Arg}
\DeclareMathOperator{\Aut}{Aut}
\DeclareMathOperator{\Card}{Card}
\DeclareMathOperator{\Cl}{\mathcal{C}\!\ell}
\DeclareMathOperator{\Cov}{Cov}
\DeclareMathOperator{\Ker}{Ker}
\DeclareMathOperator{\Mat}{Mat}
\DeclareMathOperator{\PGCD}{PGCD}
\DeclareMathOperator{\PPCM}{PPCM}
\DeclareMathOperator{\Supp}{Supp}
\DeclareMathOperator{\Vect}{Vect}
\DeclareMathOperator{\argmax}{argmax}
\DeclareMathOperator{\argmin}{argmin}
\DeclareMathOperator{\ch}{ch}
\DeclareMathOperator{\com}{com}
\DeclareMathOperator{\cotan}{cotan}
\DeclareMathOperator{\det}{det}
\DeclareMathOperator{\id}{id}
\DeclareMathOperator{\rg}{rg}
\DeclareMathOperator{\rk}{rk}
\DeclareMathOperator{\sh}{sh}
\DeclareMathOperator{\th}{th}
\DeclareMathOperator{\tr}{tr}

% colors and page style
\definecolor{truewhite}{HTML}{ffffff}
\definecolor{white}{HTML}{faf4ed}
\definecolor{trueblack}{HTML}{000000}
\definecolor{black}{HTML}{575279}
\definecolor{mauve}{HTML}{907aa9}
\definecolor{blue}{HTML}{286983}
\definecolor{red}{HTML}{d7827e}
\definecolor{yellow}{HTML}{ea9d34}
\definecolor{gray}{HTML}{9893a5}
\definecolor{grey}{HTML}{9893a5}
\definecolor{green}{HTML}{a0d971}

\pagecolor{white}
\color{black}

\begin{asydef}
	settings.prc = false;
	settings.render=0;

	white = rgb("faf4ed");
	black = rgb("575279");
	blue = rgb("286983");
	red = rgb("d7827e");
	yellow = rgb("f6c177");
	orange = rgb("ea9d34");
	gray = rgb("9893a5");
	grey = rgb("9893a5");
	deepcyan = rgb("56949f");
	pink = rgb("b4637a");
	magenta = rgb("eb6f92");
	green = rgb("a0d971");
	purple = rgb("907aa9");

	defaultpen(black + fontsize(8pt));

	import three;
	currentlight = nolight;
\end{asydef}

% theorems, proofs, ...

\mdfsetup{skipabove=1em,skipbelow=1em, innertopmargin=6pt, innerbottommargin=6pt,}

\declaretheoremstyle[
	headfont=\normalfont\itshape,
	numbered=no,
	postheadspace=\newline,
	headpunct={:},
	qed=\qedsymbol]{demstyle}

\declaretheorem[style=demstyle, name=Démonstration]{dem}

\newcommand\veczero{\kern-1.2pt\vec{\kern1.2pt 0}} % \vec{0} looks weird since the `0' isn't italicized

\makeatletter
\renewcommand{\title}[2]{
	\AtBeginDocument{
		\begin{titlepage}
			\begin{center}
				\vspace{10cm}
				{\Large \sc Chapitre #1}\\
				\vspace{1cm}
				{\Huge \calligra #2}\\
				\vfill
				Hugo {\sc Salou} MPI${}^{\star}$\\
				{\small Dernière mise à jour le \@date }
			\end{center}
		\end{titlepage}
	}
}

\newcommand{\titletp}[4]{
	\AtBeginDocument{
		\begin{titlepage}
			\begin{center}
				\vspace{10cm}
				{\Large \sc tp #1}\\
				\vspace{1cm}
				{\Huge \textsc{\textit{#2}}}\\
				\vfill
				{#3}\textit{MPI}${}^{\star}$\\
			\end{center}
		\end{titlepage}
	}
	\fancyfoot{}\fancyhead{}
	\fancyfoot[R]{#4 \textit{MPI}${}^{\star}$}
	\fancyhead[C]{{\sc tp #1} : #2}
	\fancyhead[R]{\thepage}
}

\newcommand{\titletd}[2]{
	\AtBeginDocument{
		\begin{titlepage}
			\begin{center}
				\vspace{10cm}
				{\Large \sc td #1}\\
				\vspace{1cm}
				{\Huge \calligra #2}\\
				\vfill
				Hugo {\sc Salou} MPI${}^{\star}$\\
				{\small Dernière mise à jour le \@date }
			\end{center}
		\end{titlepage}
	}
}
\makeatother

\newcommand{\sign}{
	\null
	\vfill
	\begin{center}
		{
			\fontfamily{ccr}\selectfont
			\textit{\textbf{\.{\"i}}}
		}
	\end{center}
	\vfill
	\null
}

\renewcommand{\thefootnote}{\emph{\alph{footnote}}}

% figure support
\usepackage{import}
\usepackage{xifthen}
\pdfminorversion=7
\usepackage{pdfpages}
\usepackage{transparent}
\newcommand{\incfig}[1]{%
	\def\svgwidth{\columnwidth}
	\import{./figures/}{#1.pdf_tex}
}

\pdfsuppresswarningpagegroup=1
\ctikzset{tripoles/european not symbol=circle}

\newcommand{\missingpart}{{\large\color{red} Il manque quelque chose ici\ldots}}


\begin{document}
	\begin{center}
		\bfseries\scshape\Huge Khôlle n\tsup{o} 0
	\end{center}

	\paragraph{Exercice 1}
	Cette limite est, à première vue, une forme indéterminée. On procède à un développement limité après avoir réarrangés les différents termes de l'expression. Soit $x$ un réel non nul.
	On a
	\begin{align*}
		\frac{(1+x)^{\frac{\ln x}{x}}-x}{x(x^x - 1)} &= \frac{\mathrm{e}^{\frac{1}{x}\ln(x)\ln(1+x)} - x}{x^{x+1} - x} \\
		&= \frac{\mathrm{e}^{\frac{1}{x}\ln(x)\big(x - \frac{x^2}{2} + \po(x^2)\big)} - x}{x^{x+1} - x} \\
		&= \frac{\mathrm{e}^{\ln(x)\times \big(1 + \frac{x}{2}+\po(x)\big)} - x}{x^{x+1} - x} \\
		&= \frac{\cancel x\big(x^{\frac{x}{2} + \po(x)} - 1\big)}{\cancel x(x^x - 1)} \\
		&= \frac{\mathrm{e}^{\left( \frac{x}{2} + \po(x) \right) \ln x} - 1}{\mathrm{e}^{x \ln x} - 1} \\
		&= \frac{\frac{x}{2} \ln x + \po(x \ln x) + 1 - 1}{x \ln x + 1 - 1 + \po(x \ln x)} \\
		&= \frac{\frac{1}{2} + \po(1)}{1+ \po(1)} \\
		&\tendsto{x\to 0^+} \frac{1}{2} \\
	\end{align*}
	\bigskip
	\bigskip
	\paragraph{Exercice 2}
	\begin{enumerate}
		\item Comme $\frac{\ln n}{n^2 - 1} \sim \frac{\ln n }{n^2} = \frac{\ln n}{n^{0{,}5}} \times \frac{1}{n^{1{,}5}} = \po\left( \frac{1}{n^{1{,}5}} \right)$\/ et $\sum \frac{1}{n^{1{,}5}}$\/ converge par critère de {\sc Riemann}, la série $\sum b_n$\/ converge donc.
		\item Soit $x \in {[0,1]}$. On procède par récurrence. Soit $P$, le prédicat défini par \[
				P(n) : \quad``\:x + \cdots + x^{n-1} \ge (n-1)\:x^{n+1}.\:"
			\]
			\begin{itemize}
				\item On suppose $n = 2$. On a bien, comme $x \in [0,1]$, $x \ge x^2$. D'où $P(2)$.
				\item On suppose que le prédicat $P$\/ est vrai jusqu'à un certain rang $n \ge 2$. Soit $n \ge 2$\/ tel que $P(n)$\/ soit vrai i.e.\ \[
						x + \cdots + x^{n-1} \ge (n-1)\:x^{n+1}
					.\]
					On a donc, en multipliant par $x$\/ de chaque côtés de l'inégalité \[
						x^2 + \cdots + x^{n} \ge (n-1)x^{n+2}
					.\]
					On ajoute $x$\/ à chacun des membres et on utilise l'inégalité $x \ge x^{n+2}$\/ : \[
						x + x^2 + \cdots + x^n \ge (n-1)\:x^{n+2} + x^{n+2} \ge n(x^{n+2}.
					\] D'où, $P(n+1)$.
			\end{itemize}
			On conclut, par récurrence, que, pour tout $n \ge 2$, $x + \cdots + x^{n-1} \ge (n-1)\:x^{n+1}$. Soit $n \ge 2$. On en déduit que \[
				\frac{x^n}{1 + x + \cdots + x^{n-1}} \le \frac{x^n}{1 + (n-1)x^{n+1}}
			.\] Et, par croissance de l'intégrale, on a donc \[
				\boxed{a_n = \int_{0}^{1} \frac{x^n}{1 + x + \cdots + x^{n-1}}~\mathrm{d}x \le \int_{0}^{1} \frac{x^n}{1 + (n-1)x^{n+1}}~\mathrm{d}x.}
			\]
		\item Soit $k \in \left\llbracket 2,n \right\rrbracket$.
			\begin{align*}
				\int_{0}^{1} \frac{x^k}{1 + (k-1)x^{k+1}}~\mathrm{d}x &= \bigg[\ln\big(1 + (k-1) \times x^{k+1}\big) \times  \frac{1}{k+1} \times \frac{1}{k-1} \bigg]_0^1 \\
				&= \frac{1}{k^2 - 1} \Big(\ln(1 + (k-1)) - \ln 1\Big) \\
				&= \frac{\ln k}{k^2 - 1} \\
			\end{align*}
			Or, d'après la question 1., on sait que la série $\sum \frac{\ln n}{n^2 - 1}$\/ converge. Et, on sait que $\forall n \in \N\setminus \{0,1\},\: 0 \le a_n \le b_n$. D'où, par le théorème des gendarmes, \fbox{$\sum a_n$\/ converge.}
	\end{enumerate}
	\bigskip
	\bigskip
	\paragraph{Exercice 3}
	\begin{enumerate}
		\item Prouvons ce résultat par récurrence. On pose, pour $n \in \N^*$, \[
				P(n):\quad``\:C_n + i S_n = \mathrm{e}^{in\theta/2} \times \frac{\sin\big((n+1)\theta / 2\big)}{\sin\big(\theta / 2\big)}.\:"
			\]
			\begin{itemize}
				\item On pose $n = 1$. On a 
					\[
						C_1 + i S_1 = 1 + \cos \theta + i \sin \theta
					\] et, \[
						\mathrm{e}^{i\theta/2} \times \frac{\sin \theta}{\sin \frac{\theta}{2}} = \frac{\sin \theta}{\sin \frac{\theta}{2}} \left( \cos \frac{\theta}{2} + i \sin \frac{\theta}{2} \right) = i \sin \theta + \sin \theta \times \cotan \frac{\theta}{2} = i \sin \theta + 1 + \cos \theta
					\] car \[
						\frac{1 + \cos \theta}{\sin \theta} = \frac{2 \cos^2 \frac{\theta}{2}}{2 \sin \frac{\theta}{2} \cos \frac{\theta}{2}} = \cotan \frac{\theta}{2}
					.\]
				\item 
			\end{itemize}
	\end{enumerate}
\end{document}
