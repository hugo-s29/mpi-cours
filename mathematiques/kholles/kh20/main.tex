\documentclass[a4paper]{article}

\usepackage[margin=1in]{geometry}
\usepackage[utf8]{inputenc}
\usepackage[T1]{fontenc}
\usepackage{mathrsfs}
\usepackage{textcomp}
\usepackage[french]{babel}
\usepackage{amsmath}
\usepackage{amssymb}
\usepackage{cancel}
\usepackage{frcursive}
\usepackage[inline]{asymptote}
\usepackage{tikz}
\usepackage[european,straightvoltages,europeanresistors]{circuitikz}
\usepackage{tikz-cd}
\usepackage{tkz-tab}
\usepackage[b]{esvect}
\usepackage[framemethod=TikZ]{mdframed}
\usepackage{centernot}
\usepackage{diagbox}
\usepackage{dsfont}
\usepackage{fancyhdr}
\usepackage{float}
\usepackage{graphicx}
\usepackage{listings}
\usepackage{multicol}
\usepackage{nicematrix}
\usepackage{pdflscape}
\usepackage{stmaryrd}
\usepackage{xfrac}
\usepackage{hep-math-font}
\usepackage{amsthm}
\usepackage{thmtools}
\usepackage{indentfirst}
\usepackage[framemethod=TikZ]{mdframed}
\usepackage{accents}
\usepackage{soulutf8}
\usepackage{mathtools}
\usepackage{bodegraph}
\usepackage{slashbox}
\usepackage{enumitem}
\usepackage{calligra}
\usepackage{cinzel}
\usepackage{BOONDOX-calo}

% Tikz
\usetikzlibrary{babel}
\usetikzlibrary{positioning}
\usetikzlibrary{calc}

% global settings
\frenchspacing
\reversemarginpar
\setuldepth{a}

%\everymath{\displaystyle}

\frenchbsetup{StandardLists=true}

\def\asydir{asy}

%\sisetup{exponent-product=\cdot,output-decimal-marker={,},separate-uncertainty,range-phrase=\;à\;,locale=FR}

\setlength{\parskip}{1em}

\theoremstyle{definition}

% Changing math
\let\emptyset\varnothing
\let\ge\geqslant
\let\le\leqslant
\let\preceq\preccurlyeq
\let\succeq\succcurlyeq
\let\ds\displaystyle
\let\ts\textstyle

\newcommand{\C}{\mathds{C}}
\newcommand{\R}{\mathds{R}}
\newcommand{\Z}{\mathds{Z}}
\newcommand{\N}{\mathds{N}}
\newcommand{\Q}{\mathds{Q}}

\renewcommand{\O}{\emptyset}

\newcommand\ubar[1]{\underaccent{\bar}{#1}}

\renewcommand\Re{\expandafter\mathfrak{Re}}
\renewcommand\Im{\expandafter\mathfrak{Im}}

\let\slantedpartial\partial
\DeclareRobustCommand{\partial}{\text{\rotatebox[origin=t]{20}{\scalebox{0.95}[1]{$\slantedpartial$}}}\hspace{-1pt}}

% merging two maths characters w/ \charfusion
\makeatletter
\def\moverlay{\mathpalette\mov@rlay}
\def\mov@rlay#1#2{\leavevmode\vtop{%
   \baselineskip\z@skip \lineskiplimit-\maxdimen
   \ialign{\hfil$\m@th#1##$\hfil\cr#2\crcr}}}
\newcommand{\charfusion}[3][\mathord]{
    #1{\ifx#1\mathop\vphantom{#2}\fi
        \mathpalette\mov@rlay{#2\cr#3}
      }
    \ifx#1\mathop\expandafter\displaylimits\fi}
\makeatother

% custom math commands
\newcommand{\T}{{\!\!\,\top}}
\newcommand{\avrt}[1]{\rotatebox{-90}{$#1$}}
\newcommand{\bigcupdot}{\charfusion[\mathop]{\bigcup}{\cdot}}
\newcommand{\cupdot}{\charfusion[\mathbin]{\cup}{\cdot}}
%\newcommand{\danger}{{\large\fontencoding{U}\fontfamily{futs}\selectfont\char 66\relax}\;}
\newcommand{\tendsto}[1]{\xrightarrow[#1]{}}
\newcommand{\vrt}[1]{\rotatebox{90}{$#1$}}
\newcommand{\tsup}[1]{\textsuperscript{\underline{#1}}}
\newcommand{\tsub}[1]{\textsubscript{#1}}

\renewcommand{\mod}[1]{~\left[ #1 \right]}
\renewcommand{\t}{{}^t\!}
\newcommand{\s}{\text{\calligra s}}

% custom units / constants
%\DeclareSIUnit{\litre}{\ell}
\let\hbar\hslash

% header / footer
\pagestyle{fancy}
\fancyhead{} \fancyfoot{}
\fancyfoot[C]{\thepage}

% fonts
\let\sc\scshape
\let\bf\bfseries
\let\it\itshape
\let\sl\slshape

% custom math operators
\let\th\relax
\let\det\relax
\DeclareMathOperator*{\codim}{codim}
\DeclareMathOperator*{\dom}{dom}
\DeclareMathOperator*{\gO}{O}
\DeclareMathOperator*{\po}{\text{\cursive o}}
\DeclareMathOperator*{\sgn}{sgn}
\DeclareMathOperator*{\simi}{\sim}
\DeclareMathOperator{\Arccos}{Arccos}
\DeclareMathOperator{\Arcsin}{Arcsin}
\DeclareMathOperator{\Arctan}{Arctan}
\DeclareMathOperator{\Argsh}{Argsh}
\DeclareMathOperator{\Arg}{Arg}
\DeclareMathOperator{\Aut}{Aut}
\DeclareMathOperator{\Card}{Card}
\DeclareMathOperator{\Cl}{\mathcal{C}\!\ell}
\DeclareMathOperator{\Cov}{Cov}
\DeclareMathOperator{\Ker}{Ker}
\DeclareMathOperator{\Mat}{Mat}
\DeclareMathOperator{\PGCD}{PGCD}
\DeclareMathOperator{\PPCM}{PPCM}
\DeclareMathOperator{\Supp}{Supp}
\DeclareMathOperator{\Vect}{Vect}
\DeclareMathOperator{\argmax}{argmax}
\DeclareMathOperator{\argmin}{argmin}
\DeclareMathOperator{\ch}{ch}
\DeclareMathOperator{\com}{com}
\DeclareMathOperator{\cotan}{cotan}
\DeclareMathOperator{\det}{det}
\DeclareMathOperator{\id}{id}
\DeclareMathOperator{\rg}{rg}
\DeclareMathOperator{\rk}{rk}
\DeclareMathOperator{\sh}{sh}
\DeclareMathOperator{\th}{th}
\DeclareMathOperator{\tr}{tr}

% colors and page style
\definecolor{truewhite}{HTML}{ffffff}
\definecolor{white}{HTML}{faf4ed}
\definecolor{trueblack}{HTML}{000000}
\definecolor{black}{HTML}{575279}
\definecolor{mauve}{HTML}{907aa9}
\definecolor{blue}{HTML}{286983}
\definecolor{red}{HTML}{d7827e}
\definecolor{yellow}{HTML}{ea9d34}
\definecolor{gray}{HTML}{9893a5}
\definecolor{grey}{HTML}{9893a5}
\definecolor{green}{HTML}{a0d971}

\pagecolor{white}
\color{black}

\begin{asydef}
	settings.prc = false;
	settings.render=0;

	white = rgb("faf4ed");
	black = rgb("575279");
	blue = rgb("286983");
	red = rgb("d7827e");
	yellow = rgb("f6c177");
	orange = rgb("ea9d34");
	gray = rgb("9893a5");
	grey = rgb("9893a5");
	deepcyan = rgb("56949f");
	pink = rgb("b4637a");
	magenta = rgb("eb6f92");
	green = rgb("a0d971");
	purple = rgb("907aa9");

	defaultpen(black + fontsize(8pt));

	import three;
	currentlight = nolight;
\end{asydef}

% theorems, proofs, ...

\mdfsetup{skipabove=1em,skipbelow=1em, innertopmargin=6pt, innerbottommargin=6pt,}

\declaretheoremstyle[
	headfont=\normalfont\itshape,
	numbered=no,
	postheadspace=\newline,
	headpunct={:},
	qed=\qedsymbol]{demstyle}

\declaretheorem[style=demstyle, name=Démonstration]{dem}

\newcommand\veczero{\kern-1.2pt\vec{\kern1.2pt 0}} % \vec{0} looks weird since the `0' isn't italicized

\makeatletter
\renewcommand{\title}[2]{
	\AtBeginDocument{
		\begin{titlepage}
			\begin{center}
				\vspace{10cm}
				{\Large \sc Chapitre #1}\\
				\vspace{1cm}
				{\Huge \calligra #2}\\
				\vfill
				Hugo {\sc Salou} MPI${}^{\star}$\\
				{\small Dernière mise à jour le \@date }
			\end{center}
		\end{titlepage}
	}
}

\newcommand{\titletp}[4]{
	\AtBeginDocument{
		\begin{titlepage}
			\begin{center}
				\vspace{10cm}
				{\Large \sc tp #1}\\
				\vspace{1cm}
				{\Huge \textsc{\textit{#2}}}\\
				\vfill
				{#3}\textit{MPI}${}^{\star}$\\
			\end{center}
		\end{titlepage}
	}
	\fancyfoot{}\fancyhead{}
	\fancyfoot[R]{#4 \textit{MPI}${}^{\star}$}
	\fancyhead[C]{{\sc tp #1} : #2}
	\fancyhead[R]{\thepage}
}

\newcommand{\titletd}[2]{
	\AtBeginDocument{
		\begin{titlepage}
			\begin{center}
				\vspace{10cm}
				{\Large \sc td #1}\\
				\vspace{1cm}
				{\Huge \calligra #2}\\
				\vfill
				Hugo {\sc Salou} MPI${}^{\star}$\\
				{\small Dernière mise à jour le \@date }
			\end{center}
		\end{titlepage}
	}
}
\makeatother

\newcommand{\sign}{
	\null
	\vfill
	\begin{center}
		{
			\fontfamily{ccr}\selectfont
			\textit{\textbf{\.{\"i}}}
		}
	\end{center}
	\vfill
	\null
}

\renewcommand{\thefootnote}{\emph{\alph{footnote}}}

% figure support
\usepackage{import}
\usepackage{xifthen}
\pdfminorversion=7
\usepackage{pdfpages}
\usepackage{transparent}
\newcommand{\incfig}[1]{%
	\def\svgwidth{\columnwidth}
	\import{./figures/}{#1.pdf_tex}
}

\pdfsuppresswarningpagegroup=1
\ctikzset{tripoles/european not symbol=circle}

\newcommand{\missingpart}{{\large\color{red} Il manque quelque chose ici\ldots}}


\def\khollenum{20}

\fancyhead[L]{Hugo {\sc Salou}\/ MPI$^\star$}
\fancyhead[R]{\scshape Khôlle n\tsup{o} \khollenum}

\begin{document}
	\begin{center}
		\bfseries\scshape\Huge Khôlle n\tsup{o} \khollenum
	\end{center}

	\paragraph{Exercice 1.}
	\begin{enumerate}
		\item Pour tout entier $k \in \N$, et pour tout $t \in {]0,1]}$, on a $|t^{2k} \ln t| \le |{\ln t}| = -\ln t$. Et, l'intégrale $\int_{0}^{1} |{\ln t}|~\mathrm{d}t = -\int_{0}^{1} \ln t~\mathrm{d}t$ converge.
			D'où, l'intégrale $\int_{0}^{1} |t^{2k} \ln t|~\mathrm{d}t$ converge.
			On en déduit que l'intégrale $I_k$ converge, pour tout entier $k \in \N$.

			Soit $x \in {]0,1]}$.
			On calcule
			\begin{align*}
				\int_{x}^{1} t^{2k} \ln t~\mathrm{d}t &= \left[ \frac{t^{2k+1}}{2k+1} \ln t \right]_x^1 - \int_{x}^{1} \frac{t^{2k+1}}{2k+1} \cdot \frac{1}{t}~\mathrm{d}t && \substack{\text{ car $\ln$ et $t \mapsto t^{2k}$ sont $\mathcal{C}^1$}\hfill\null\\\text{par intégration par parties sur le segment $[x,1]$}}\\
				&= \frac{-x^{2k+1}}{2k+1} - \int_{x}^{1} \frac{t^{2k}}{2k+1}~\mathrm{d}t \\
				&= -\frac{x^{2k+1}}{2k+1} - \left[ \frac{t^{2k+1}}{(2k+1)^2} \right]_x^1  \\
				&= -\frac{x^{2k+1}}{2k+1} - \frac{1}{(2k+1)^2} + \frac{x^{2k+1}}{(2k+1)^2} \\
				&\tendsto{x\to 0} -\frac{1}{(2k+1)^2}.
			\end{align*}
			On en déduit que, pour tout $k \in \N$, $I_k = -1 / (2k+1)^2$.
		\item La série de fonctions $\sum f_k$, où $f_k : t \mapsto t^{2k} \ln t$, converge simplement vers la fonction $t \mapsto \ln t / (1 - t^2)$ sur $]0,1[$.
			Les fonctions $f_k$ sont continues et intégrables : \[
				\int_{0}^{1} |f_k(t)|~\mathrm{d}t = -\int_{0}^{1} t^{2k} \ln t~\mathrm{d}t = -I_k
			.\]
			Et, la série $\sum (-I_k) = -\sum 1 / (2k+1)^2$ converge car, pour tout $k \in \N$, $0 \le 1/(2k+1)^2 \le 1 / k^2$ et la série $\sum 1/k^2$ converge.
			Ainsi, la fonction $g : t \mapsto \ln t / (1 - t^2)$ est intégrable sur $]0,+\infty[$, donc $f = -g$ l'est aussi ; et,
			\begin{align*}
				\int_{0}^{1} \frac{\ln t}{t^2 - 1}~\mathrm{d}t &= -\sum_{k=0}^\infty \int_{0}^{1} f_k(t)~\mathrm{d}t\\
				&= \sum_{k=0}^\infty (-I_k)\\
				&= \sum_{k=0}^\infty \frac{1}{(2k+1)^2}\\
				&= \sum_{k=1}^\infty \frac{1}{k^2} - \sum_{k=1}^\infty \frac{1}{(2k)^2} && \substack{\text{ par somme termes pairs/impairs}\\ \text{pour la série $\sum 1 / k^2$ }} \\
				&= \frac{\pi^2}{6} - \frac{1}{4} \sum_{k=1}^\infty \frac{1}{k^2} \\
				&= \frac{\pi^2}{6}\left( 1 - \frac{1}{4} \right)  \\
				&= \frac{\pi^2}{8}. \\
			\end{align*}
	\end{enumerate}

	\paragraph{Exercice 2.}
	\begin{enumerate}
		\item La fonction $\ln$\/ est de classe $\mathcal{C}^2$ sur $]0,+\infty[$ donc deux fois dérivable sur cet intervalle et, pour tout $x > 0$, $\ln'' x = -1/x^2 < 0$. La fonction $\ln$\/ est donc concave sur $]0,+\infty[$.
			Soient $a$, $b$ et $c$ trois réels strictement positifs.
			Par concavité de la fonction $\ln$, on a $\ln\big((a + b + c) / 3\big) \le (\ln a + \ln b + \ln c) / 3$.
			D'où, $\ln\big((a+b+c) / 3\big) \le \ln(abc) / 3 = \ln \sqrt[3]{abc} $.
			Par croissance de $\ln$, on en déduit que \[
				\frac{a+b+c}{3} \ge \sqrt[3]{abc} 
			.\]
		\item La fonction $f$ est de classe $\mathcal{C}^1$ sur $(\R^+_*)^2$, donc différentiable sur $(\R^+_*)^2$. On calcule $\nabla f(x,y) = \big(1 - \sfrac1{yx^2},1 - \sfrac1{xy^2}\big)$.
			On procède par analyse-synthèse.
			\begin{description}
				\item[Analyse.] On suppose que $f$ atteint un extremum local en $(x,y) \in (\R_+^*)^2$. Ainsi, $\nabla f(x,y) = 0$.
					D'où, $1 = 1 / yx^2$ et $1 = 1/xy^2$. On en déduit que $yx^2 = xy^2$, d'où $x = y$.
					Or, $1 = 1 / yx^2 = 1 / x^3$, d'où $x = y = 1$.
					Ainsi, si un extremum est atteint, il sera en $(1,1)$.
				\item[Synthèse.] Montrons que $f$ atteint un minimum en $(1,1)$.
					Montrons ainsi que $f(x,y) \ge f(1,1) = 3$ pour tout vecteur $(x,y) \in (\R^+_*)^2$.
					\[
						f(x,y) = x + y + \frac{1}{xy} \ge 3 \sqrt[3]{\frac{xy}{xy}} = 3 = f(1,1)
					.\]
			\end{description}
			On en déduit que $f$ atteint un minimum global en $(1,1)$, et n'a pas de maximum (local ou global) et n'a pas d'autres minima (locaux ou globaux).
	\end{enumerate}

	\paragraph{Exercice 3.}
	\begin{enumerate}
		\item L'univers $\Omega$ est l'ensemble des $(n+1)$-uplets d'éléments de $\llbracket 1,n \rrbracket$ : $\Omega = \llbracket 1,n \rrbracket^{n+1}$.
			Soit $k \in \llbracket 2,n+1 \rrbracket$. L'événement $(X = k)$\/ n'est pas vide car le $(n+1)$-uplet \[
				u_k = (1, 2, \ldots, {k-2}, 1, 1, \ldots, 1)
			\] est un élément de $(X = k)$.
			D'où, par croissance de la probabilité, $P(\{u_k\}) \le P(X = k)$, car $\{u_k\} \subset (X=k)$.
			Or, par équiprobabilité, $P(\{u_k\}) = 1 / \Card \Omega = 1 / n^{n+1} > 0$.
			On en déduit que $P(X = k) > 0$.
		\item Montrons $P(X > k) \neq 0$. On a $(X > k) = (X \ge k + 1)$, d'où $P(X > k) = P({X \ge k+1}) > 0$ car $k + 1 \in \llbracket 2,n \rrbracket$.
			On sait que $(X > k+ 1) \cap (X > k) = (X > k + 1)$ car $(X > k) \subset (X > k + 1)$.
			On a montré précédemment que $P(X > k) \neq 0$.
			On en déduit, par définition des probabilités conditionnelles, \[
				P(X > k + 1) = P\big((X > k+ 1) \cap (X > k)\big) = P(X > k + 1  \mid X > k) \cdot P(X > k)
			.\]
		\item Au $(k + 1)$-ième lancer, on choisit une boule parmi $n$ avec équiprobabilité.
			Pour que $X > k + 1$ sachant que $X > k$, il faut tirer une boule non tirée, il y en a $n - k$.
			D'où, $P(X > k + 1  \mid X > k) = (n-k) / n$.
		\item Montrons, par récurrence sur $k$, la propriété $\mathcal{P}(k)$ : \guillemotleft~$P(X > k) = n! / (n^k \cdot (n-k)!)$.~\guillemotright
			\begin{itemize}
				\item Pour $k = 0$, on a $P(X > 0) = 1 = n! / (n^0 \cdot n!)$.
					Ainsi, $\mathcal{P}(0)$ est vraie.
				\item On suppose $\mathcal{P}(k)$ vraie, montrons que $\mathcal{P}(k+1)$ est aussi vraie. En appliquant l'égalité des probabilités trouvées à la question précédente, qui est valide comme $k \in \llbracket 1,n - 1 \rrbracket$, on calcule
					\begin{align*}
						P(X > k + 1) &= P(X > k + 1  \mid X > k) \cdot P(X > k)\\
						&= \frac{n - k}{n}\cdot \frac{n!}{n^k \cdot (n-k)!} \\
						&= \frac{n!}{n^{k+1} \cdot (n-k+1)!}. \\
					\end{align*}
					Ainsi, $\mathcal{P}(n+1)$ est vraie.
			\end{itemize}
			On en déduit, par récurrence, que $P(X > k) = n! / \big(n^k \cdot (n -k)!\big)$, pour tout $k \in \llbracket 1,n-1 \rrbracket$.
			De plus, pour $k = n$, on a $P(X > n) = 0$ car il n'y a que $n$ boules dans l'urne.
	\end{enumerate}

\end{document}


