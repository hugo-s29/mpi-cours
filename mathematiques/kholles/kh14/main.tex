\documentclass[a4paper]{article}

\usepackage[margin=1in]{geometry}
\usepackage[utf8]{inputenc}
\usepackage[T1]{fontenc}
\usepackage{mathrsfs}
\usepackage{textcomp}
\usepackage[french]{babel}
\usepackage{amsmath}
\usepackage{amssymb}
\usepackage{cancel}
\usepackage{frcursive}
\usepackage[inline]{asymptote}
\usepackage{tikz}
\usepackage[european,straightvoltages,europeanresistors]{circuitikz}
\usepackage{tikz-cd}
\usepackage{tkz-tab}
\usepackage[b]{esvect}
\usepackage[framemethod=TikZ]{mdframed}
\usepackage{centernot}
\usepackage{diagbox}
\usepackage{dsfont}
\usepackage{fancyhdr}
\usepackage{float}
\usepackage{graphicx}
\usepackage{listings}
\usepackage{multicol}
\usepackage{nicematrix}
\usepackage{pdflscape}
\usepackage{stmaryrd}
\usepackage{xfrac}
\usepackage{hep-math-font}
\usepackage{amsthm}
\usepackage{thmtools}
\usepackage{indentfirst}
\usepackage[framemethod=TikZ]{mdframed}
\usepackage{accents}
\usepackage{soulutf8}
\usepackage{mathtools}
\usepackage{bodegraph}
\usepackage{slashbox}
\usepackage{enumitem}
\usepackage{calligra}
\usepackage{cinzel}
\usepackage{BOONDOX-calo}

% Tikz
\usetikzlibrary{babel}
\usetikzlibrary{positioning}
\usetikzlibrary{calc}

% global settings
\frenchspacing
\reversemarginpar
\setuldepth{a}

%\everymath{\displaystyle}

\frenchbsetup{StandardLists=true}

\def\asydir{asy}

%\sisetup{exponent-product=\cdot,output-decimal-marker={,},separate-uncertainty,range-phrase=\;à\;,locale=FR}

\setlength{\parskip}{1em}

\theoremstyle{definition}

% Changing math
\let\emptyset\varnothing
\let\ge\geqslant
\let\le\leqslant
\let\preceq\preccurlyeq
\let\succeq\succcurlyeq
\let\ds\displaystyle
\let\ts\textstyle

\newcommand{\C}{\mathds{C}}
\newcommand{\R}{\mathds{R}}
\newcommand{\Z}{\mathds{Z}}
\newcommand{\N}{\mathds{N}}
\newcommand{\Q}{\mathds{Q}}

\renewcommand{\O}{\emptyset}

\newcommand\ubar[1]{\underaccent{\bar}{#1}}

\renewcommand\Re{\expandafter\mathfrak{Re}}
\renewcommand\Im{\expandafter\mathfrak{Im}}

\let\slantedpartial\partial
\DeclareRobustCommand{\partial}{\text{\rotatebox[origin=t]{20}{\scalebox{0.95}[1]{$\slantedpartial$}}}\hspace{-1pt}}

% merging two maths characters w/ \charfusion
\makeatletter
\def\moverlay{\mathpalette\mov@rlay}
\def\mov@rlay#1#2{\leavevmode\vtop{%
   \baselineskip\z@skip \lineskiplimit-\maxdimen
   \ialign{\hfil$\m@th#1##$\hfil\cr#2\crcr}}}
\newcommand{\charfusion}[3][\mathord]{
    #1{\ifx#1\mathop\vphantom{#2}\fi
        \mathpalette\mov@rlay{#2\cr#3}
      }
    \ifx#1\mathop\expandafter\displaylimits\fi}
\makeatother

% custom math commands
\newcommand{\T}{{\!\!\,\top}}
\newcommand{\avrt}[1]{\rotatebox{-90}{$#1$}}
\newcommand{\bigcupdot}{\charfusion[\mathop]{\bigcup}{\cdot}}
\newcommand{\cupdot}{\charfusion[\mathbin]{\cup}{\cdot}}
%\newcommand{\danger}{{\large\fontencoding{U}\fontfamily{futs}\selectfont\char 66\relax}\;}
\newcommand{\tendsto}[1]{\xrightarrow[#1]{}}
\newcommand{\vrt}[1]{\rotatebox{90}{$#1$}}
\newcommand{\tsup}[1]{\textsuperscript{\underline{#1}}}
\newcommand{\tsub}[1]{\textsubscript{#1}}

\renewcommand{\mod}[1]{~\left[ #1 \right]}
\renewcommand{\t}{{}^t\!}
\newcommand{\s}{\text{\calligra s}}

% custom units / constants
%\DeclareSIUnit{\litre}{\ell}
\let\hbar\hslash

% header / footer
\pagestyle{fancy}
\fancyhead{} \fancyfoot{}
\fancyfoot[C]{\thepage}

% fonts
\let\sc\scshape
\let\bf\bfseries
\let\it\itshape
\let\sl\slshape

% custom math operators
\let\th\relax
\let\det\relax
\DeclareMathOperator*{\codim}{codim}
\DeclareMathOperator*{\dom}{dom}
\DeclareMathOperator*{\gO}{O}
\DeclareMathOperator*{\po}{\text{\cursive o}}
\DeclareMathOperator*{\sgn}{sgn}
\DeclareMathOperator*{\simi}{\sim}
\DeclareMathOperator{\Arccos}{Arccos}
\DeclareMathOperator{\Arcsin}{Arcsin}
\DeclareMathOperator{\Arctan}{Arctan}
\DeclareMathOperator{\Argsh}{Argsh}
\DeclareMathOperator{\Arg}{Arg}
\DeclareMathOperator{\Aut}{Aut}
\DeclareMathOperator{\Card}{Card}
\DeclareMathOperator{\Cl}{\mathcal{C}\!\ell}
\DeclareMathOperator{\Cov}{Cov}
\DeclareMathOperator{\Ker}{Ker}
\DeclareMathOperator{\Mat}{Mat}
\DeclareMathOperator{\PGCD}{PGCD}
\DeclareMathOperator{\PPCM}{PPCM}
\DeclareMathOperator{\Supp}{Supp}
\DeclareMathOperator{\Vect}{Vect}
\DeclareMathOperator{\argmax}{argmax}
\DeclareMathOperator{\argmin}{argmin}
\DeclareMathOperator{\ch}{ch}
\DeclareMathOperator{\com}{com}
\DeclareMathOperator{\cotan}{cotan}
\DeclareMathOperator{\det}{det}
\DeclareMathOperator{\id}{id}
\DeclareMathOperator{\rg}{rg}
\DeclareMathOperator{\rk}{rk}
\DeclareMathOperator{\sh}{sh}
\DeclareMathOperator{\th}{th}
\DeclareMathOperator{\tr}{tr}

% colors and page style
\definecolor{truewhite}{HTML}{ffffff}
\definecolor{white}{HTML}{faf4ed}
\definecolor{trueblack}{HTML}{000000}
\definecolor{black}{HTML}{575279}
\definecolor{mauve}{HTML}{907aa9}
\definecolor{blue}{HTML}{286983}
\definecolor{red}{HTML}{d7827e}
\definecolor{yellow}{HTML}{ea9d34}
\definecolor{gray}{HTML}{9893a5}
\definecolor{grey}{HTML}{9893a5}
\definecolor{green}{HTML}{a0d971}

\pagecolor{white}
\color{black}

\begin{asydef}
	settings.prc = false;
	settings.render=0;

	white = rgb("faf4ed");
	black = rgb("575279");
	blue = rgb("286983");
	red = rgb("d7827e");
	yellow = rgb("f6c177");
	orange = rgb("ea9d34");
	gray = rgb("9893a5");
	grey = rgb("9893a5");
	deepcyan = rgb("56949f");
	pink = rgb("b4637a");
	magenta = rgb("eb6f92");
	green = rgb("a0d971");
	purple = rgb("907aa9");

	defaultpen(black + fontsize(8pt));

	import three;
	currentlight = nolight;
\end{asydef}

% theorems, proofs, ...

\mdfsetup{skipabove=1em,skipbelow=1em, innertopmargin=6pt, innerbottommargin=6pt,}

\declaretheoremstyle[
	headfont=\normalfont\itshape,
	numbered=no,
	postheadspace=\newline,
	headpunct={:},
	qed=\qedsymbol]{demstyle}

\declaretheorem[style=demstyle, name=Démonstration]{dem}

\newcommand\veczero{\kern-1.2pt\vec{\kern1.2pt 0}} % \vec{0} looks weird since the `0' isn't italicized

\makeatletter
\renewcommand{\title}[2]{
	\AtBeginDocument{
		\begin{titlepage}
			\begin{center}
				\vspace{10cm}
				{\Large \sc Chapitre #1}\\
				\vspace{1cm}
				{\Huge \calligra #2}\\
				\vfill
				Hugo {\sc Salou} MPI${}^{\star}$\\
				{\small Dernière mise à jour le \@date }
			\end{center}
		\end{titlepage}
	}
}

\newcommand{\titletp}[4]{
	\AtBeginDocument{
		\begin{titlepage}
			\begin{center}
				\vspace{10cm}
				{\Large \sc tp #1}\\
				\vspace{1cm}
				{\Huge \textsc{\textit{#2}}}\\
				\vfill
				{#3}\textit{MPI}${}^{\star}$\\
			\end{center}
		\end{titlepage}
	}
	\fancyfoot{}\fancyhead{}
	\fancyfoot[R]{#4 \textit{MPI}${}^{\star}$}
	\fancyhead[C]{{\sc tp #1} : #2}
	\fancyhead[R]{\thepage}
}

\newcommand{\titletd}[2]{
	\AtBeginDocument{
		\begin{titlepage}
			\begin{center}
				\vspace{10cm}
				{\Large \sc td #1}\\
				\vspace{1cm}
				{\Huge \calligra #2}\\
				\vfill
				Hugo {\sc Salou} MPI${}^{\star}$\\
				{\small Dernière mise à jour le \@date }
			\end{center}
		\end{titlepage}
	}
}
\makeatother

\newcommand{\sign}{
	\null
	\vfill
	\begin{center}
		{
			\fontfamily{ccr}\selectfont
			\textit{\textbf{\.{\"i}}}
		}
	\end{center}
	\vfill
	\null
}

\renewcommand{\thefootnote}{\emph{\alph{footnote}}}

% figure support
\usepackage{import}
\usepackage{xifthen}
\pdfminorversion=7
\usepackage{pdfpages}
\usepackage{transparent}
\newcommand{\incfig}[1]{%
	\def\svgwidth{\columnwidth}
	\import{./figures/}{#1.pdf_tex}
}

\pdfsuppresswarningpagegroup=1
\ctikzset{tripoles/european not symbol=circle}

\newcommand{\missingpart}{{\large\color{red} Il manque quelque chose ici\ldots}}


\def\khollenum{14}

\fancyhead[L]{Hugo {\sc Salou}\/ MPI$^\star$}
\fancyhead[R]{\scshape Khôlle n\tsup{o} \khollenum}

\begin{document}
	\begin{center}
		\bfseries\scshape\Huge Khôlle n\tsup{o} \khollenum
	\end{center}

	\paragraph{Exercice 1.}
	\begin{enumerate}
		\item On note $A$\/ l'événement \guillemotleft~$X$\/ est pair.~\guillemotright\@ On a $A = \bigcup_{n \in \N} (X = 2n)$, et cette union est disjointe. D'où,
			\begin{align*}
				P(A) = \sum_{n = 0}^\infty P(X = 2n) &= \sum_{n=0}^\infty \mathrm{e}^{-\lambda} \cdot \frac{\lambda^{2n}}{(2n)!} \\
				&= \mathrm{e}^{-\lambda} \sum_{n=0}^\infty \frac{\lambda^{2n}}{(2n)!} \\
			\end{align*}
			Or, la série entière $\sum \frac{x^{2n}}{(2n)!}$\/ a pour rayon de convergence $+\infty$, et \[
				\forall x \in \R,\quad \sum_{n=0}^\infty \frac{x^{2n}}{(2n)!} = \ch x
			.\] D'où, $P(A) = \mathrm{e}^{-\lambda} \cdot \ch \lambda$.

			On pose $B$\/ l'événement \guillemotleft~$X$\/ est impair.~\guillemotright\@ On a $B = \bigcup_{n \in \N} (X = 2n+1)$, et cette union est disjointe. D'où,
			\begin{align*}
				P(B) = \sum_{n=0}^\infty P(X = 2n+1) &= \sum_{n=0}^\infty \mathrm{e}^{-\lambda} \frac{x^{2n+1}}{(2n+1)!} \\
				&= \mathrm{e}^{-\lambda}\sum_{n=0}^\infty \frac{\lambda^{2n+1}}{(2n+1)!} \\
			\end{align*}
			Or, la série entière $\sum \frac{x^{2n+1}}{(2n+1)!}$\/ a pour rayon de convergence $+\infty$, et, \[
				\forall x \in \R, \quad \sum_{n=0}^\infty \frac{x^{2n+1}}{(2n+1)!} = \sh x
			.\] D'où, $P(B) = \mathrm{e}^{-\lambda} \cdot \sh \lambda$.

			Or, pour tout réel $x$, $\ch x \ge \sh x$. Ainsi, $P(A) \ge P(B)$.
		\item Soit $n > \lambda - 1$.
			On sait que $(X  \ge n) = \bigcup_{i \in \N}(X = i+ n)$, et cette union est disjointe. D'où,
			\begin{align*}
				P(X \ge n) = \sum_{i=0}^\infty P(X = i + n) &= \sum_{i=0}^\infty \mathrm{e}^{-\lambda} \cdot \frac{\lambda^{i+n}}{(i+n)!}\\ &= \mathrm{e}^{-\lambda}\cdot \lambda^n \sum_{i=0}^\infty \frac{\lambda^i}{(i+n)!}\\
				&\le \mathrm{e}^{-\lambda} \lambda^n \sum_{i=0}^\infty \frac{\lambda^i}{n! \times (n+1)^i} \\
				&\le \mathrm{e}^{-\lambda}\cdot \frac{\lambda^n}{n!} \sum_{i=0}^n \left( \frac{\lambda}{n+1} \right)^i
			\end{align*}
			Et, la série entière $\sum x^n$\/ est géométrique, son rayon de convergence est 1, et \[
				\forall x \in {]-1,1[},\quad \sum_{n=0}^\infty x^n = \frac{1}{1-x}
			.\]
			Or, par hypothèse, $0 < \lambda < n + 1$, donc $0 <\frac{\lambda}{n+1} < 1$.
			D'où, \[
				P(X \ge n) \le \mathrm{e}^{-\lambda}\: \frac{\lambda^n}{n!} \cdot \frac{1}{1 - \frac{\lambda}{n+1}}
			.\]
		\item D'une part, on a $P(X \ge n) \ge P(X = n)$\/ car $(X \ge n) \supset (X = n)$. D'autre part, $P(X \ge n) \le P(X = n)/ (1 - \frac{\lambda}{n + 1})$, et \[
				\frac{1}{1-\frac{\lambda}{n+1}} \tendsto{n\to \infty} 1
			.\] D'où, $P(X \ge n) \simi_{n\to +\infty} P(X = n)$.

			On a donc $P(X \ge n) = P(X = n) + \po\big(P(X = n)\big)$. Or, $(X \ge n) = (X = n) \cup (X > n)$, et cette union est disjointe, d'où,
			\begin{align*}
				P(X > n) = P(X \ge n) - P(X = n) &= P(X = n) + \po\big(P(X = n)\big) - P(X = n)\\
				&= \po\big(P(X = n)\big)
			\end{align*}
	\end{enumerate}

	\paragraph{Exercice 2.}
	\begin{enumerate}[start=0]
		\item Soit $x \in \R^*$, et soit $n \in \N^*$.
			\begin{align*}
				\left| \frac{\sqrt{n+1}\:x^{n+1}}{\sqrt{n}\:x^n} \right| = |x| \sqrt{1 + \frac{1}{n}}  \tendsto{n\to +\infty} |x|
			\end{align*}
			D'après le critère de \textsc{d'Alembert}, la série entière $\sum \sqrt{n}\:x^n$\/ a pour rayon de convergence~1.
		\item Soit $x \in {]0,1[}$. On a $\sqrt{n}\:x^n \ge x^n$, pour $n \in \N^*$. D'où, \[
				g(x) = \sum_{n=0}^\infty \sqrt{n}\:x^n \ge \sum_{n=0}^\infty x^n = \frac{1}{1-x}
			.\] Comme $|x| < 1$, on en déduit que \[
				g(x) \ge \frac{1}{1-x} \ge \frac{x}{1-x}
			.\]
			Et, $\frac{x}{1-x} \tendsto{x\to 1^-} +\infty$, d'où $g(x) \tendsto{x\to 1^-} +\infty$.
		\item Soit $x \in {]-1,1[}$. On calcule
			\begin{align*}
				f(x) = (1-x) g(x) &= g(x) - x\:g(x) \\
				&= \sum_{n=0}^\infty \sqrt{n}\:x^n - x\sum_{n=0}^\infty \sqrt{n}\:x^n \\
				&= \sum_{n=0}^\infty \sqrt{n}\:x^n - \sum_{n=0}^\infty \sqrt{n}\:x^{n+1} \\
				&= \sum_{n=1}^\infty \sqrt{n} x^n - \sum_{n=1}^\infty \sqrt{n - 1} x^n \\
				&= \sum_{n=1}^\infty (\sqrt{n} - \sqrt{n-1})x^n \\
			\end{align*}
		\item On calcule, pour $N \in \N$,
			\[
				\sum_{n=1}^N (\sqrt{n} - \sqrt{n-1}) = \sum_{n=1}^N \sqrt{n} - \sum_{n=0}^{N-1} \sqrt{n} = \sqrt{N} - \sqrt{0} = \sqrt{N} \tendsto{N\to \infty} +\infty
			\] 
			par télescopage. D'où, la série $\sum(\sqrt{n} -\sqrt{n-1})$\/ diverge.
			On pose, $h : x \mapsto \sqrt{x} - \sqrt{x-1}$, dérivable sur $]1, +\infty[$, et \[
				\forall x > 1,\quad h'(x) = \frac{1}{2\sqrt{x}} - \frac{1}{2\sqrt{x-1}} < 0
			\] car $2\sqrt{x} > 2 \sqrt{x-1}$, et par décroissance de la fonction inverse.
			Et, \[
				\forall n > 1,\quad\frac{\sqrt{n}}{\sqrt{n-1}} = \sqrt{\frac{1}{1-\frac{1}{n}}} \tendsto{n\to \infty} 1,
			\] d'où $\sqrt{n-1} = \sqrt{n} + \po(n)$. Ainsi, $\sqrt{n} - \sqrt{n-1} = \po(n) \tendsto{n\to \infty} 0$.
			D'où, la suite $(\sqrt{n} -\sqrt{n-1})$\/ tend vers 0 en décroissant. D'après le théorème des séries alternées, on en déduit que la série $\sum (\sqrt{n} - \sqrt{n-1})\cdot (-1)^n$\/ converge.
		\item On en déduit que le rayon de convergence de la série $\sum \big(\sqrt{n} - \sqrt{n-1}\big)$\/ est 1 (d'après la question précédente).
		\item D'après la question 3, la série $\sum (\sqrt{n} - \sqrt{n-1})\:(-1)^n$\/ converge. D'après le théorème radial d'\textsc{Abel}, on en déduit que \[
				f(x) = \sum_{n=1}^\infty (\sqrt{n} - \sqrt{n-1}) x^n \tendsto{x\to -1^+} \sum_{n=1}^\infty (\sqrt{n} - \sqrt{n-1}) (-1)^n = f(-1)
			.\] Or, $\smash{g(x) = f(x) / (1-x) \tendsto{x\to-1^+} f(-1) / 2}$. On en déduit que $g(x)$\/ admet une limite finie en $-1^+$.
	\end{enumerate}

	\paragraph{Exercice 3.}
	Soit $x \in \R^*$.
	\[
		\left| \frac{S_{n+1} x^{n+1}}{S_n x^n} \right| = |x| \cdot \frac{a_n + S_n}{S_n} = |x| \cdot \left(1 + \frac{a_n}{S_n}\right) \tendsto{n\to \infty} |x|
	.\] D'où, par le critère de \textsc{d'Alembert}, le rayon de convergence de la série $\sum S_n x^n$\/ est donc 1.

	Et, $a_n > 0$, donc, pour tout $x \in \R$, $0 \le \sum a_n x^n \le \sum S_n x^n$, et la série $\sum S_n x^n$\/ converge pour tout $x \in {]-1,1[}$, la série $\sum a_n x^n$\/ converge aussi pour tout $x \in {]-1,1[}$. On en déduit que le rayon de convergence de cette série entière est supérieur ou égal à 1. De plus, la série $\sum a_n$\/ diverge car $(S_n)$\/ diverge, d'où le rayon de convergence de la série entière $\sum a_n x^n$\/ est inférieur ou égal à 1.
	Le rayon de convergence de la série $\sum a_n x^n$\/ est donc égal à 1.
\end{document}


