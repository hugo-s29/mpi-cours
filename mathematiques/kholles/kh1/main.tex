\documentclass[a4paper]{article}

\usepackage[margin=1in]{geometry}
\usepackage[utf8]{inputenc}
\usepackage[T1]{fontenc}
\usepackage{mathrsfs}
\usepackage{textcomp}
\usepackage[french]{babel}
\usepackage{amsmath}
\usepackage{amssymb}
\usepackage{cancel}
\usepackage{frcursive}
\usepackage[inline]{asymptote}
\usepackage{tikz}
\usepackage[european,straightvoltages,europeanresistors]{circuitikz}
\usepackage{tikz-cd}
\usepackage{tkz-tab}
\usepackage[b]{esvect}
\usepackage[framemethod=TikZ]{mdframed}
\usepackage{centernot}
\usepackage{diagbox}
\usepackage{dsfont}
\usepackage{fancyhdr}
\usepackage{float}
\usepackage{graphicx}
\usepackage{listings}
\usepackage{multicol}
\usepackage{nicematrix}
\usepackage{pdflscape}
\usepackage{stmaryrd}
\usepackage{xfrac}
\usepackage{hep-math-font}
\usepackage{amsthm}
\usepackage{thmtools}
\usepackage{indentfirst}
\usepackage[framemethod=TikZ]{mdframed}
\usepackage{accents}
\usepackage{soulutf8}
\usepackage{mathtools}
\usepackage{bodegraph}
\usepackage{slashbox}
\usepackage{enumitem}
\usepackage{calligra}
\usepackage{cinzel}
\usepackage{BOONDOX-calo}

% Tikz
\usetikzlibrary{babel}
\usetikzlibrary{positioning}
\usetikzlibrary{calc}

% global settings
\frenchspacing
\reversemarginpar
\setuldepth{a}

%\everymath{\displaystyle}

\frenchbsetup{StandardLists=true}

\def\asydir{asy}

%\sisetup{exponent-product=\cdot,output-decimal-marker={,},separate-uncertainty,range-phrase=\;à\;,locale=FR}

\setlength{\parskip}{1em}

\theoremstyle{definition}

% Changing math
\let\emptyset\varnothing
\let\ge\geqslant
\let\le\leqslant
\let\preceq\preccurlyeq
\let\succeq\succcurlyeq
\let\ds\displaystyle
\let\ts\textstyle

\newcommand{\C}{\mathds{C}}
\newcommand{\R}{\mathds{R}}
\newcommand{\Z}{\mathds{Z}}
\newcommand{\N}{\mathds{N}}
\newcommand{\Q}{\mathds{Q}}

\renewcommand{\O}{\emptyset}

\newcommand\ubar[1]{\underaccent{\bar}{#1}}

\renewcommand\Re{\expandafter\mathfrak{Re}}
\renewcommand\Im{\expandafter\mathfrak{Im}}

\let\slantedpartial\partial
\DeclareRobustCommand{\partial}{\text{\rotatebox[origin=t]{20}{\scalebox{0.95}[1]{$\slantedpartial$}}}\hspace{-1pt}}

% merging two maths characters w/ \charfusion
\makeatletter
\def\moverlay{\mathpalette\mov@rlay}
\def\mov@rlay#1#2{\leavevmode\vtop{%
   \baselineskip\z@skip \lineskiplimit-\maxdimen
   \ialign{\hfil$\m@th#1##$\hfil\cr#2\crcr}}}
\newcommand{\charfusion}[3][\mathord]{
    #1{\ifx#1\mathop\vphantom{#2}\fi
        \mathpalette\mov@rlay{#2\cr#3}
      }
    \ifx#1\mathop\expandafter\displaylimits\fi}
\makeatother

% custom math commands
\newcommand{\T}{{\!\!\,\top}}
\newcommand{\avrt}[1]{\rotatebox{-90}{$#1$}}
\newcommand{\bigcupdot}{\charfusion[\mathop]{\bigcup}{\cdot}}
\newcommand{\cupdot}{\charfusion[\mathbin]{\cup}{\cdot}}
%\newcommand{\danger}{{\large\fontencoding{U}\fontfamily{futs}\selectfont\char 66\relax}\;}
\newcommand{\tendsto}[1]{\xrightarrow[#1]{}}
\newcommand{\vrt}[1]{\rotatebox{90}{$#1$}}
\newcommand{\tsup}[1]{\textsuperscript{\underline{#1}}}
\newcommand{\tsub}[1]{\textsubscript{#1}}

\renewcommand{\mod}[1]{~\left[ #1 \right]}
\renewcommand{\t}{{}^t\!}
\newcommand{\s}{\text{\calligra s}}

% custom units / constants
%\DeclareSIUnit{\litre}{\ell}
\let\hbar\hslash

% header / footer
\pagestyle{fancy}
\fancyhead{} \fancyfoot{}
\fancyfoot[C]{\thepage}

% fonts
\let\sc\scshape
\let\bf\bfseries
\let\it\itshape
\let\sl\slshape

% custom math operators
\let\th\relax
\let\det\relax
\DeclareMathOperator*{\codim}{codim}
\DeclareMathOperator*{\dom}{dom}
\DeclareMathOperator*{\gO}{O}
\DeclareMathOperator*{\po}{\text{\cursive o}}
\DeclareMathOperator*{\sgn}{sgn}
\DeclareMathOperator*{\simi}{\sim}
\DeclareMathOperator{\Arccos}{Arccos}
\DeclareMathOperator{\Arcsin}{Arcsin}
\DeclareMathOperator{\Arctan}{Arctan}
\DeclareMathOperator{\Argsh}{Argsh}
\DeclareMathOperator{\Arg}{Arg}
\DeclareMathOperator{\Aut}{Aut}
\DeclareMathOperator{\Card}{Card}
\DeclareMathOperator{\Cl}{\mathcal{C}\!\ell}
\DeclareMathOperator{\Cov}{Cov}
\DeclareMathOperator{\Ker}{Ker}
\DeclareMathOperator{\Mat}{Mat}
\DeclareMathOperator{\PGCD}{PGCD}
\DeclareMathOperator{\PPCM}{PPCM}
\DeclareMathOperator{\Supp}{Supp}
\DeclareMathOperator{\Vect}{Vect}
\DeclareMathOperator{\argmax}{argmax}
\DeclareMathOperator{\argmin}{argmin}
\DeclareMathOperator{\ch}{ch}
\DeclareMathOperator{\com}{com}
\DeclareMathOperator{\cotan}{cotan}
\DeclareMathOperator{\det}{det}
\DeclareMathOperator{\id}{id}
\DeclareMathOperator{\rg}{rg}
\DeclareMathOperator{\rk}{rk}
\DeclareMathOperator{\sh}{sh}
\DeclareMathOperator{\th}{th}
\DeclareMathOperator{\tr}{tr}

% colors and page style
\definecolor{truewhite}{HTML}{ffffff}
\definecolor{white}{HTML}{faf4ed}
\definecolor{trueblack}{HTML}{000000}
\definecolor{black}{HTML}{575279}
\definecolor{mauve}{HTML}{907aa9}
\definecolor{blue}{HTML}{286983}
\definecolor{red}{HTML}{d7827e}
\definecolor{yellow}{HTML}{ea9d34}
\definecolor{gray}{HTML}{9893a5}
\definecolor{grey}{HTML}{9893a5}
\definecolor{green}{HTML}{a0d971}

\pagecolor{white}
\color{black}

\begin{asydef}
	settings.prc = false;
	settings.render=0;

	white = rgb("faf4ed");
	black = rgb("575279");
	blue = rgb("286983");
	red = rgb("d7827e");
	yellow = rgb("f6c177");
	orange = rgb("ea9d34");
	gray = rgb("9893a5");
	grey = rgb("9893a5");
	deepcyan = rgb("56949f");
	pink = rgb("b4637a");
	magenta = rgb("eb6f92");
	green = rgb("a0d971");
	purple = rgb("907aa9");

	defaultpen(black + fontsize(8pt));

	import three;
	currentlight = nolight;
\end{asydef}

% theorems, proofs, ...

\mdfsetup{skipabove=1em,skipbelow=1em, innertopmargin=6pt, innerbottommargin=6pt,}

\declaretheoremstyle[
	headfont=\normalfont\itshape,
	numbered=no,
	postheadspace=\newline,
	headpunct={:},
	qed=\qedsymbol]{demstyle}

\declaretheorem[style=demstyle, name=Démonstration]{dem}

\newcommand\veczero{\kern-1.2pt\vec{\kern1.2pt 0}} % \vec{0} looks weird since the `0' isn't italicized

\makeatletter
\renewcommand{\title}[2]{
	\AtBeginDocument{
		\begin{titlepage}
			\begin{center}
				\vspace{10cm}
				{\Large \sc Chapitre #1}\\
				\vspace{1cm}
				{\Huge \calligra #2}\\
				\vfill
				Hugo {\sc Salou} MPI${}^{\star}$\\
				{\small Dernière mise à jour le \@date }
			\end{center}
		\end{titlepage}
	}
}

\newcommand{\titletp}[4]{
	\AtBeginDocument{
		\begin{titlepage}
			\begin{center}
				\vspace{10cm}
				{\Large \sc tp #1}\\
				\vspace{1cm}
				{\Huge \textsc{\textit{#2}}}\\
				\vfill
				{#3}\textit{MPI}${}^{\star}$\\
			\end{center}
		\end{titlepage}
	}
	\fancyfoot{}\fancyhead{}
	\fancyfoot[R]{#4 \textit{MPI}${}^{\star}$}
	\fancyhead[C]{{\sc tp #1} : #2}
	\fancyhead[R]{\thepage}
}

\newcommand{\titletd}[2]{
	\AtBeginDocument{
		\begin{titlepage}
			\begin{center}
				\vspace{10cm}
				{\Large \sc td #1}\\
				\vspace{1cm}
				{\Huge \calligra #2}\\
				\vfill
				Hugo {\sc Salou} MPI${}^{\star}$\\
				{\small Dernière mise à jour le \@date }
			\end{center}
		\end{titlepage}
	}
}
\makeatother

\newcommand{\sign}{
	\null
	\vfill
	\begin{center}
		{
			\fontfamily{ccr}\selectfont
			\textit{\textbf{\.{\"i}}}
		}
	\end{center}
	\vfill
	\null
}

\renewcommand{\thefootnote}{\emph{\alph{footnote}}}

% figure support
\usepackage{import}
\usepackage{xifthen}
\pdfminorversion=7
\usepackage{pdfpages}
\usepackage{transparent}
\newcommand{\incfig}[1]{%
	\def\svgwidth{\columnwidth}
	\import{./figures/}{#1.pdf_tex}
}

\pdfsuppresswarningpagegroup=1
\ctikzset{tripoles/european not symbol=circle}

\newcommand{\missingpart}{{\large\color{red} Il manque quelque chose ici\ldots}}

\fancyhead[L]{Hugo {\sc Salou}\/ MPI$^\star$}

\begin{document}
	\begin{center}
		\bfseries\scshape\Huge Khôlle n\tsup{o} 1
	\end{center}

	\paragraph{Exercice 1}
	On considère quatre cas : $a = 1$, $a = -1$, $|a| < 1$\/ et $|a| > 1$.
	\begin{itemize}
		\item[{\sc Cas 1}] On pose $a = 1$. On étudie donc la convergence de la série $\sum \frac{1}{1+n}$. Or, on sait que $\sum \frac{1}{1+n} \sim \sum \frac{1}{n}$, et cette série diverge. On en déduit que, si $a = 1$, la série \fbox{$\sum \frac{a^n}{n + a^{2n}}$\/ diverge.}
		\item[{\sc Cas 2}] On pose $a = -1$. On étudie donc la convergence de la série $\sum \frac{(-1)^n}{n + 1}$. Nous savons que la suite $\left( \frac{1}{n+1} \right)$\/ est décroissante et tend vers 0. D'après le théorème des séries alternées, la série \fbox{$\sum \frac{(-1)^n}{n+1}$\/ converge} donc.
		\item[{\sc Cas 3}] Soit $a \in \R \setminus [-1,1]$.
			On sait que, $n = \po(a^n)$\/ quand $n$\/ tend vers $+\infty$. Et donc, \[
				\frac{a^n}{n + a^{2n}} = \frac{a^n}{a^{2n} + \po(a^n)} = \frac{1}{a^n + \po(1)} = \po_{n\to +\infty}(a^{-n})
			.\] Or, comme $|a| > 1$, $\frac{1}{|a|} < 1$\/ et donc $\sum a^{-n}$\/ converge (car $\sum |a|^{-n}$\/ converge). On en déduit que \fbox{$\sum \frac{a^n}{n + a^{2n}}$ converge.}
		\item[{\sc Cas 4}] Soit $a \in {]-1,1[}$. On sait que, comme $|a| < 1$, $a^n = \po(n^2)$. Ainsi, on a \[
			\frac{a^n}{n + a^{2n}} = \frac{\po(n^2)}{n + \po(n^4)} = \frac{\po(n)}{1 + \po(n^3)} = \frac{1}{\po(n^2)}
		.\] Mais, comme la série $\sum \frac{1}{n^2}$\/ converge, on en déduit que la série \fbox{$\sum \frac{a^n}{n + a^{2n}}$\/ converge.}
	\end{itemize}
	\bigskip
	\bigskip
	\paragraph{Exercice 2}
	\begin{enumerate}
		\item Montrons que $\big(\mathrm{SL}_2(\R), \times\big)$\/ est un sous-groupe de $\mathrm{GL}_2(\R)$.
			On sait que $I_2 \in \mathrm{SL}_2(\R)$\/ car $\det I_2 = 1$.
			Soient $A$\/ et $B$\/ deux matrices de $\mathrm{SL}_2(\R)$.
			Montrons que $A \times B^{-1} \in \mathrm{SL}_2(\R)$.
			Tout d'abord, nous savons que $B^{-1}$\/ existe car $\det B = 1 \neq 0$. Et, $\det(A \times B^{-1}) = \det A \times \frac{1}{\det B} = 1$. On en conclut que $A\times B^{-1} \in \mathrm{SL}_2(\R)$. On a montré \[
				\boxed{\text{$\mathrm{SL}_2(\R)$\/ est un sous-groupe de $\mathrm{GL}_2(\R)$}.}
			\]
		\item D'après l'énoncé, nous savons que \begin{align*}
			\varphi: \mathrm{SL}_2(\R) &\longrightarrow \mathscr{F}(\R,\R) \\
			\begin{pmatrix}
				a&b\\
				c&d
			\end{pmatrix} &\longmapsto \left(x \mapsto \frac{ax + b}{cx + d}\right).
		\end{align*}
		Soient $M$\/ et $N$\/ deux matrices de $\mathrm{SL}_2(\R)$. Montrons que $\varphi(M \cdot N) = \varphi(M) \circ\varphi(N)$.
		On pose $M = {a\: b\choose c\: d}$\/ et $N = {u\: v\choose w\: z}$.
		Calculons $M \cdot N$\/ :
		\[
			M \cdot N = \begin{pmatrix}
				a&b\\
				c&d
			\end{pmatrix} \cdot \begin{pmatrix}
				u&v\\
				w&z
			\end{pmatrix} = \begin{pmatrix}
				au + bw & av + bz\\
				cu + dw & bv + dz
			\end{pmatrix}
		.\] Et, soit $x \in \R$, on a
		\[
			\big(\varphi(M)  \circ \varphi(N)\big)(x) = \frac{a \frac{ux+v}{wx+z} + b}{c \frac{ux+v}{wx+z} + d} = \frac{au\,x + av + bw\,x + bz}{cu\,x + cv + dw\,x+dz} = \varphi \begin{pmatrix}
				au + vw & av + bz\\
				cu + dw & cv + dz
			\end{pmatrix} (x)
		\] On a bien $\varphi(M\cdot N) = \varphi(M)  \circ \varphi(N)$.
		On cherche, à présent, $\Ker \varphi$. On procède par analyse-synthèse.
		\begin{itemize}
			\item[{\sc Analyse}] Soit ${a\:b\choose c\:d} \in \Ker \varphi$. On a donc \[
					\varphi \begin{pmatrix}
						a&b\\
						c&d
					\end{pmatrix} = \left(x \mapsto \frac{ax + b}{cx + d}\right) = (x\mapsto 0)
				.\] D'où, on en déduit que $(x \mapsto ax + b) = (x\mapsto 0)$, autrement dit, $a = b= 0$.
			\item[{\sc Synthèse}] On en déduit que \[
				\boxed{\Ker \varphi = \Vect\left( \begin{pmatrix} 0&0\\0&1\\\end{pmatrix}, \begin{pmatrix} 0&0\\1&0\\\end{pmatrix} \right).}
			\] En effet, soient $\alpha, \beta \in \R$, on calcule $\varphi{0\:0\choose \alpha\:\beta}(x) = \frac{0x + 0}{\alpha x + \beta} = 0$.
		\end{itemize}
	\end{enumerate}
	\bigskip
	\bigskip
	\paragraph{Exercice 3}
	\begin{enumerate}
		\item Montrons que $\big(\sqrt{I}, +\big)$\/ est un sous-groupe de $A$\/ et que $\forall i \in \sqrt{I},\:\forall a \in A,\: i \times a \in \sqrt{I}$.
			\begin{itemize}
				\item $0 \in I$\/ (car c'est un sous-groupe de $(A,+)$) et donc $0 \in \sqrt{I}$\/ (car $x^1 \in I$).
				\item Soient $x$\/ et $y$\/ deux éléments de $\sqrt{I}$. On pose $n$\/ et $m$\/ deux entiers tels que $x^m \in I$\/ et $y^n \in I$. Quitte à intervertir $x$\/ et $y$ ainsi que $m$\/ et $n$, on peut suppose que $n < m$. Ainsi, en posant $k = n + m$, on calcule, à l'aide de la formule du binôme de {\sc Newton} (qui est applicable car $A$\/ est un anneau commutatif), $(x-y)^{k}$\/ : \[
							\sum_{i=0}^n {k \choose i} (-1)^{k-i} x^{i} y^{k-i} = \sum_{i=0}^{m} {k \choose i} (-1)^{k-i} x^i y^{k-i}  + \sum_{i=m + 1}^{m+n} {k \choose i} (-1)^{k-i} x^i y^{k-i}
					.\] Or, avec $i \in \left\llbracket 0,m \right\rrbracket$, on a $k-i \ge n$\/ et donc $y^{k-i} \in I$ (car $I$\/ est un idéal).
					De même, avec $i \in \left\llbracket m+1,m+n \right\rrbracket$, $i \ge m$\/ et donc $x^{i} \in I$\/ (car $I$\/ est un idéal).
					On en déduit, comme $I$\/ est un idéal et que c'est, par conséquent, un sous groupe de $(A,+)$, que \[
						(x-y)^k = \underbrace{\sum_{i=0}^m {k\choose i} (-1)^{k-i} x^i y^{k-i}}_{\in I} + \underbrace{\sum_{i=m+1}^{m+n} {k\choose i} (-1)^{k-i} x^{i} y^{k-i}}_{\in I} \in I
					.\]
					On en déduit que $\big(\sqrt{I}, +\big)$\/ est un sous-groupe de $A$.
				\item En posant $n = 1$, on a $\sqrt{I} \supset \{x \in A  \mid x^n \in I\} = \{x \in I\} = I$. D'où $I \subset \sqrt{I}$.
			\end{itemize}
		\item
			\begin{enumerate}
				\item On sait que $I \cap J$\/ est un sous-groupe de $(A, +)$. Soit $i \in I \cap J$. Soit $a \in A$. Comme $i \in I$, on sait que $a\cdot i \in I$. De même, comme $i \in J$, $a\cdot i \in J$. On en déduit que $a\cdot i \in I \cap J$. On en déduit que $I \cap J$\/ est un idéal de $A$.
				\item On suppose $I \subset J$. Soit $i \in \sqrt{I}$. Soit $n \in \N^*$\/ tel que $i^n \in I \subset J$. D'où $i^n \in J$\/ et donc $i \in \sqrt{J}$.
				\item
					\begin{itemize}
						\item[``$\subset$''] Soit $x \in \sqrt{I \cap J}$. Il existe $n \in \N^*$\/ tel que $x^n \in I \cap J$\/ donc $x^n \in I$\/ et $x^n \in J$. On en déduit que $x \in \sqrt{I}$\/ et $x \in \sqrt{J}$, d'où $x \in \sqrt{I} \cap \sqrt{J}$.
						\item[``$\supset$''] Soit $x \in \sqrt{I} \cap \sqrt{J}$.
							Comme $x \in \sqrt{I}$, soit $n \in \N^*$\/ tel que $x^n \in I$. De même, comme $x \in \sqrt{J}$, soit $m \in \N^*$\/ tel que $x^m \in J$. Ainsi, en posant $N = m + n$, on a $x^{m+n} \in I$\/ (car $I$\/ est un idéal) et $x^{m+n} \in J$\/ (car $J$\/ est un idéal). On en déduit que $x^{m+n} \in I \cap J$\/ et donc $x^{m+n} \in \sqrt{I \cap J}$.
					\end{itemize}
			\end{enumerate}
		\item Si $I$\/ et $J$\/ deux idéaux tels que $I\subset J$, alors $\sqrt{I} \subset \sqrt{J}$. Mais $\sqrt{I}$\/ est un idéal et, $I \subset \sqrt{I}$, d'où $\sqrt{I} \subset \sqrt{\sqrt{I}}$. On montre donc $\sqrt{\sqrt{I}} \subset \sqrt{I}$. Soit $x \in \sqrt{\sqrt{I}}$. Soit $n \in \N^*$\/ tel que $x^n \in \sqrt{I}$. Soit $m \in \N^*$\/ tel que $(x^n)^m \in I$. On a donc $x^{n\cdot m} \in I$\/ et donc $x \in \sqrt{I}$.
	\end{enumerate}
\end{document}
