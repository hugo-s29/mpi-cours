\documentclass[a4paper]{article}

\usepackage[margin=1in]{geometry}
\usepackage[utf8]{inputenc}
\usepackage[T1]{fontenc}
\usepackage{mathrsfs}
\usepackage{textcomp}
\usepackage[french]{babel}
\usepackage{amsmath}
\usepackage{amssymb}
\usepackage{cancel}
\usepackage{frcursive}
\usepackage[inline]{asymptote}
\usepackage{tikz}
\usepackage[european,straightvoltages,europeanresistors]{circuitikz}
\usepackage{tikz-cd}
\usepackage{tkz-tab}
\usepackage[b]{esvect}
\usepackage[framemethod=TikZ]{mdframed}
\usepackage{centernot}
\usepackage{diagbox}
\usepackage{dsfont}
\usepackage{fancyhdr}
\usepackage{float}
\usepackage{graphicx}
\usepackage{listings}
\usepackage{multicol}
\usepackage{nicematrix}
\usepackage{pdflscape}
\usepackage{stmaryrd}
\usepackage{xfrac}
\usepackage{hep-math-font}
\usepackage{amsthm}
\usepackage{thmtools}
\usepackage{indentfirst}
\usepackage[framemethod=TikZ]{mdframed}
\usepackage{accents}
\usepackage{soulutf8}
\usepackage{mathtools}
\usepackage{bodegraph}
\usepackage{slashbox}
\usepackage{enumitem}
\usepackage{calligra}
\usepackage{cinzel}
\usepackage{BOONDOX-calo}

% Tikz
\usetikzlibrary{babel}
\usetikzlibrary{positioning}
\usetikzlibrary{calc}

% global settings
\frenchspacing
\reversemarginpar
\setuldepth{a}

%\everymath{\displaystyle}

\frenchbsetup{StandardLists=true}

\def\asydir{asy}

%\sisetup{exponent-product=\cdot,output-decimal-marker={,},separate-uncertainty,range-phrase=\;à\;,locale=FR}

\setlength{\parskip}{1em}

\theoremstyle{definition}

% Changing math
\let\emptyset\varnothing
\let\ge\geqslant
\let\le\leqslant
\let\preceq\preccurlyeq
\let\succeq\succcurlyeq
\let\ds\displaystyle
\let\ts\textstyle

\newcommand{\C}{\mathds{C}}
\newcommand{\R}{\mathds{R}}
\newcommand{\Z}{\mathds{Z}}
\newcommand{\N}{\mathds{N}}
\newcommand{\Q}{\mathds{Q}}

\renewcommand{\O}{\emptyset}

\newcommand\ubar[1]{\underaccent{\bar}{#1}}

\renewcommand\Re{\expandafter\mathfrak{Re}}
\renewcommand\Im{\expandafter\mathfrak{Im}}

\let\slantedpartial\partial
\DeclareRobustCommand{\partial}{\text{\rotatebox[origin=t]{20}{\scalebox{0.95}[1]{$\slantedpartial$}}}\hspace{-1pt}}

% merging two maths characters w/ \charfusion
\makeatletter
\def\moverlay{\mathpalette\mov@rlay}
\def\mov@rlay#1#2{\leavevmode\vtop{%
   \baselineskip\z@skip \lineskiplimit-\maxdimen
   \ialign{\hfil$\m@th#1##$\hfil\cr#2\crcr}}}
\newcommand{\charfusion}[3][\mathord]{
    #1{\ifx#1\mathop\vphantom{#2}\fi
        \mathpalette\mov@rlay{#2\cr#3}
      }
    \ifx#1\mathop\expandafter\displaylimits\fi}
\makeatother

% custom math commands
\newcommand{\T}{{\!\!\,\top}}
\newcommand{\avrt}[1]{\rotatebox{-90}{$#1$}}
\newcommand{\bigcupdot}{\charfusion[\mathop]{\bigcup}{\cdot}}
\newcommand{\cupdot}{\charfusion[\mathbin]{\cup}{\cdot}}
%\newcommand{\danger}{{\large\fontencoding{U}\fontfamily{futs}\selectfont\char 66\relax}\;}
\newcommand{\tendsto}[1]{\xrightarrow[#1]{}}
\newcommand{\vrt}[1]{\rotatebox{90}{$#1$}}
\newcommand{\tsup}[1]{\textsuperscript{\underline{#1}}}
\newcommand{\tsub}[1]{\textsubscript{#1}}

\renewcommand{\mod}[1]{~\left[ #1 \right]}
\renewcommand{\t}{{}^t\!}
\newcommand{\s}{\text{\calligra s}}

% custom units / constants
%\DeclareSIUnit{\litre}{\ell}
\let\hbar\hslash

% header / footer
\pagestyle{fancy}
\fancyhead{} \fancyfoot{}
\fancyfoot[C]{\thepage}

% fonts
\let\sc\scshape
\let\bf\bfseries
\let\it\itshape
\let\sl\slshape

% custom math operators
\let\th\relax
\let\det\relax
\DeclareMathOperator*{\codim}{codim}
\DeclareMathOperator*{\dom}{dom}
\DeclareMathOperator*{\gO}{O}
\DeclareMathOperator*{\po}{\text{\cursive o}}
\DeclareMathOperator*{\sgn}{sgn}
\DeclareMathOperator*{\simi}{\sim}
\DeclareMathOperator{\Arccos}{Arccos}
\DeclareMathOperator{\Arcsin}{Arcsin}
\DeclareMathOperator{\Arctan}{Arctan}
\DeclareMathOperator{\Argsh}{Argsh}
\DeclareMathOperator{\Arg}{Arg}
\DeclareMathOperator{\Aut}{Aut}
\DeclareMathOperator{\Card}{Card}
\DeclareMathOperator{\Cl}{\mathcal{C}\!\ell}
\DeclareMathOperator{\Cov}{Cov}
\DeclareMathOperator{\Ker}{Ker}
\DeclareMathOperator{\Mat}{Mat}
\DeclareMathOperator{\PGCD}{PGCD}
\DeclareMathOperator{\PPCM}{PPCM}
\DeclareMathOperator{\Supp}{Supp}
\DeclareMathOperator{\Vect}{Vect}
\DeclareMathOperator{\argmax}{argmax}
\DeclareMathOperator{\argmin}{argmin}
\DeclareMathOperator{\ch}{ch}
\DeclareMathOperator{\com}{com}
\DeclareMathOperator{\cotan}{cotan}
\DeclareMathOperator{\det}{det}
\DeclareMathOperator{\id}{id}
\DeclareMathOperator{\rg}{rg}
\DeclareMathOperator{\rk}{rk}
\DeclareMathOperator{\sh}{sh}
\DeclareMathOperator{\th}{th}
\DeclareMathOperator{\tr}{tr}

% colors and page style
\definecolor{truewhite}{HTML}{ffffff}
\definecolor{white}{HTML}{faf4ed}
\definecolor{trueblack}{HTML}{000000}
\definecolor{black}{HTML}{575279}
\definecolor{mauve}{HTML}{907aa9}
\definecolor{blue}{HTML}{286983}
\definecolor{red}{HTML}{d7827e}
\definecolor{yellow}{HTML}{ea9d34}
\definecolor{gray}{HTML}{9893a5}
\definecolor{grey}{HTML}{9893a5}
\definecolor{green}{HTML}{a0d971}

\pagecolor{white}
\color{black}

\begin{asydef}
	settings.prc = false;
	settings.render=0;

	white = rgb("faf4ed");
	black = rgb("575279");
	blue = rgb("286983");
	red = rgb("d7827e");
	yellow = rgb("f6c177");
	orange = rgb("ea9d34");
	gray = rgb("9893a5");
	grey = rgb("9893a5");
	deepcyan = rgb("56949f");
	pink = rgb("b4637a");
	magenta = rgb("eb6f92");
	green = rgb("a0d971");
	purple = rgb("907aa9");

	defaultpen(black + fontsize(8pt));

	import three;
	currentlight = nolight;
\end{asydef}

% theorems, proofs, ...

\mdfsetup{skipabove=1em,skipbelow=1em, innertopmargin=6pt, innerbottommargin=6pt,}

\declaretheoremstyle[
	headfont=\normalfont\itshape,
	numbered=no,
	postheadspace=\newline,
	headpunct={:},
	qed=\qedsymbol]{demstyle}

\declaretheorem[style=demstyle, name=Démonstration]{dem}

\newcommand\veczero{\kern-1.2pt\vec{\kern1.2pt 0}} % \vec{0} looks weird since the `0' isn't italicized

\makeatletter
\renewcommand{\title}[2]{
	\AtBeginDocument{
		\begin{titlepage}
			\begin{center}
				\vspace{10cm}
				{\Large \sc Chapitre #1}\\
				\vspace{1cm}
				{\Huge \calligra #2}\\
				\vfill
				Hugo {\sc Salou} MPI${}^{\star}$\\
				{\small Dernière mise à jour le \@date }
			\end{center}
		\end{titlepage}
	}
}

\newcommand{\titletp}[4]{
	\AtBeginDocument{
		\begin{titlepage}
			\begin{center}
				\vspace{10cm}
				{\Large \sc tp #1}\\
				\vspace{1cm}
				{\Huge \textsc{\textit{#2}}}\\
				\vfill
				{#3}\textit{MPI}${}^{\star}$\\
			\end{center}
		\end{titlepage}
	}
	\fancyfoot{}\fancyhead{}
	\fancyfoot[R]{#4 \textit{MPI}${}^{\star}$}
	\fancyhead[C]{{\sc tp #1} : #2}
	\fancyhead[R]{\thepage}
}

\newcommand{\titletd}[2]{
	\AtBeginDocument{
		\begin{titlepage}
			\begin{center}
				\vspace{10cm}
				{\Large \sc td #1}\\
				\vspace{1cm}
				{\Huge \calligra #2}\\
				\vfill
				Hugo {\sc Salou} MPI${}^{\star}$\\
				{\small Dernière mise à jour le \@date }
			\end{center}
		\end{titlepage}
	}
}
\makeatother

\newcommand{\sign}{
	\null
	\vfill
	\begin{center}
		{
			\fontfamily{ccr}\selectfont
			\textit{\textbf{\.{\"i}}}
		}
	\end{center}
	\vfill
	\null
}

\renewcommand{\thefootnote}{\emph{\alph{footnote}}}

% figure support
\usepackage{import}
\usepackage{xifthen}
\pdfminorversion=7
\usepackage{pdfpages}
\usepackage{transparent}
\newcommand{\incfig}[1]{%
	\def\svgwidth{\columnwidth}
	\import{./figures/}{#1.pdf_tex}
}

\pdfsuppresswarningpagegroup=1
\ctikzset{tripoles/european not symbol=circle}

\newcommand{\missingpart}{{\large\color{red} Il manque quelque chose ici\ldots}}

\fancyhead[L]{Hugo {\sc Salou}\/ MPI$^\star$}
\everymath{\ds}

\begin{document}
	\begin{center}
		\bfseries\scshape\Huge Khôlle n\tsup{o} 5
	\end{center}

	\paragraph{Exercice 1}
	\begin{enumerate}
		\item L'intégrale $\ds I = \int_{1}^{+\infty}  \frac{\Arcsin\big(1 / \sqrt{t}\big)}{t^2}~\mathrm{d}t$\/ est impropre en $+\infty$. Et, on sait que $\forall t \in [-1,1]$, $|\Arcsin t\:| \le \frac{\pi}{2}$. D'où $\int_{1}^{+\infty} \left| \frac{\Arcsin\left( \frac{1}{\sqrt{t}} \right)}{t^2} \right|~\mathrm{d}t \le \frac{\pi}{2} \int_{1}^{+\infty} \frac{1}{t^2}~\mathrm{d}t$\/ qui converge par critère de {\sc Riemann}.
			Ainsi, l'intégrale $I$\/ converge absolument ; elle converge donc.
		\item On a
			\begin{align*}
				\int_{1}^{+\infty} \frac{\Arcsin\left( 1/\sqrt{t} \right)}{t^2}~\mathrm{d}t
				&= \int_{\frac{\pi}{2}}^{0} \frac{x}{\frac{1}{\sin^2 x}} \times \left( -\frac{2\cos x}{\sin^3 x} \right)~\mathrm{d}x \\
				&= \int_{0}^{\frac{\pi}{2}} 2x \cos x \sin x~\mathrm{d}x \\
				&= \int_{0}^{\frac{\pi}{2}} x\sin(2x)~\mathrm{d}x \\
				&= \left[ x\frac{\cos 2x}{2} \right]_0^{\frac{\pi}{2}}  - \int_{0}^{\frac{\pi}{2}} \frac{1}{2}\cos(2x)~\mathrm{d}x \\
				&= \frac{\pi}{4} - \frac{1}{4} \left[ \sin(2x) \right]_0^{\frac{\pi}{2}} \\
				&= \frac{\pi}{4} \\
			\end{align*}
	\end{enumerate}

	\paragraph{Exercice 2}
	\begin{enumerate}
		\item Soit $\lambda \in \C$. On calcule
			\begin{align*}
				-\chi_M(\lambda) = \det(\lambda I_3 - M) &= 
				\begin{vmatrix}
					-\lambda & 0 & a\\
					1 & -\lambda & 0\\
					1 & 1 & -\lambda
				\end{vmatrix}\\
				&= -\lambda
				\begin{vmatrix}
					-\lambda & 0\\
					1 & -\lambda
				\end{vmatrix} + a
				\begin{vmatrix}
					1&-\lambda\\
					1&1
				\end{vmatrix}\\
				&= -\lambda \times \lambda^2 + a (1 + \lambda) \\
				&= -\lambda^3 + a\lambda + a. \\
			\end{align*}
			On en déduit donc que \[
				\chi_M(X) = X^3 - aX - a
			.\]
		\item On suppose $a = 0$. On a donc $\chi_M(X) = X^3$. Comme le polynôme caractéristique est scindé mais pas à racines simples, la matrice $M$\/ n'est pas diagonalisable dans $\mathscr{M}_3(\C)$.
		\item On suppose $a = \frac{1}{2}$. On a donc $\ts\chi_M(X) = X^3 - \frac{1}{2}X^3 - \frac{1}{2}$. On remarque que $1$\/ est racine de ce polynôme. D'où, en utilisant l'algorithme de {\sc Horner}, on trouve une factorisation : \[
				\begin{array}{c|cccc}
					&1&-\sfrac{1}{2}&0&-\sfrac{1}{2}\\ \hline
					1&1&\sfrac{1}{2}&\sfrac{1}{2}&0
				\end{array}\qquad\text{d'où}\qquad \chi_M(X) = (X - 1)\underbrace{\left(X^2 - \frac{1}{2}X - \frac{1}{2}\right)}_{P}
			.\] Le discriminant du trinôme $P$\/ vaut $\ts\Delta = \frac{1}{4} + 2 > 0$. On en déduit que $\chi_M$\/ est scindé et a trois racines simples. D'où $M$\/ est diagonalisable.
		\item On suppose $a = \tfrac{27}{4}$. Ainsi, $\ts\chi_M(X) = X^3 - \frac{27}{4}X - \frac{27}{4}$, et $\ts \chi_M'(X) = 3X^2 - \frac{27}{4}$. On cherche les racines de $\chi_M'(X)$\/ : soit $x \in \C$,
			\begin{align*}
				\chi_M'(x) = 0 \iff& 3x^2 - \frac{27}{4} = 0\\
				\iff& x^2 - \frac{9}{4} = 0\\
				\iff& x = \pm \frac{3}{2}
			\end{align*}
			On remarque que $\tfrac{3}{2}$\/ est également racine de $\chi_M$. On en déduit que la racine $\ts \frac{3}{2}$\/ a une multiplicité supérieure ou égale à 2. Ainsi, $\chi_M$\/ n'est pas un polynôme scindé à racines simples. La matrice $M$\/ n'est donc pas diagonalisable dans $\mathscr{M}_3(\C)$.
		\item
	\end{enumerate}
\end{document}
