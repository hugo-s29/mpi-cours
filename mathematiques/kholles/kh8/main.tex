\documentclass[a4paper,9pt]{article}

\usepackage[margin=1in]{geometry}
\usepackage[utf8]{inputenc}
\usepackage[T1]{fontenc}
\usepackage{mathrsfs}
\usepackage{textcomp}
\usepackage[french]{babel}
\usepackage{amsmath}
\usepackage{amssymb}
\usepackage{cancel}
\usepackage{frcursive}
\usepackage[inline]{asymptote}
\usepackage{tikz}
\usepackage[european,straightvoltages,europeanresistors]{circuitikz}
\usepackage{tikz-cd}
\usepackage{tkz-tab}
\usepackage[b]{esvect}
\usepackage[framemethod=TikZ]{mdframed}
\usepackage{centernot}
\usepackage{diagbox}
\usepackage{dsfont}
\usepackage{fancyhdr}
\usepackage{float}
\usepackage{graphicx}
\usepackage{listings}
\usepackage{multicol}
\usepackage{nicematrix}
\usepackage{pdflscape}
\usepackage{stmaryrd}
\usepackage{xfrac}
\usepackage{hep-math-font}
\usepackage{amsthm}
\usepackage{thmtools}
\usepackage{indentfirst}
\usepackage[framemethod=TikZ]{mdframed}
\usepackage{accents}
\usepackage{soulutf8}
\usepackage{mathtools}
\usepackage{bodegraph}
\usepackage{slashbox}
\usepackage{enumitem}
\usepackage{calligra}
\usepackage{cinzel}
\usepackage{BOONDOX-calo}

% Tikz
\usetikzlibrary{babel}
\usetikzlibrary{positioning}
\usetikzlibrary{calc}

% global settings
\frenchspacing
\reversemarginpar
\setuldepth{a}

%\everymath{\displaystyle}

\frenchbsetup{StandardLists=true}

\def\asydir{asy}

%\sisetup{exponent-product=\cdot,output-decimal-marker={,},separate-uncertainty,range-phrase=\;à\;,locale=FR}

\setlength{\parskip}{1em}

\theoremstyle{definition}

% Changing math
\let\emptyset\varnothing
\let\ge\geqslant
\let\le\leqslant
\let\preceq\preccurlyeq
\let\succeq\succcurlyeq
\let\ds\displaystyle
\let\ts\textstyle

\newcommand{\C}{\mathds{C}}
\newcommand{\R}{\mathds{R}}
\newcommand{\Z}{\mathds{Z}}
\newcommand{\N}{\mathds{N}}
\newcommand{\Q}{\mathds{Q}}

\renewcommand{\O}{\emptyset}

\newcommand\ubar[1]{\underaccent{\bar}{#1}}

\renewcommand\Re{\expandafter\mathfrak{Re}}
\renewcommand\Im{\expandafter\mathfrak{Im}}

\let\slantedpartial\partial
\DeclareRobustCommand{\partial}{\text{\rotatebox[origin=t]{20}{\scalebox{0.95}[1]{$\slantedpartial$}}}\hspace{-1pt}}

% merging two maths characters w/ \charfusion
\makeatletter
\def\moverlay{\mathpalette\mov@rlay}
\def\mov@rlay#1#2{\leavevmode\vtop{%
   \baselineskip\z@skip \lineskiplimit-\maxdimen
   \ialign{\hfil$\m@th#1##$\hfil\cr#2\crcr}}}
\newcommand{\charfusion}[3][\mathord]{
    #1{\ifx#1\mathop\vphantom{#2}\fi
        \mathpalette\mov@rlay{#2\cr#3}
      }
    \ifx#1\mathop\expandafter\displaylimits\fi}
\makeatother

% custom math commands
\newcommand{\T}{{\!\!\,\top}}
\newcommand{\avrt}[1]{\rotatebox{-90}{$#1$}}
\newcommand{\bigcupdot}{\charfusion[\mathop]{\bigcup}{\cdot}}
\newcommand{\cupdot}{\charfusion[\mathbin]{\cup}{\cdot}}
%\newcommand{\danger}{{\large\fontencoding{U}\fontfamily{futs}\selectfont\char 66\relax}\;}
\newcommand{\tendsto}[1]{\xrightarrow[#1]{}}
\newcommand{\vrt}[1]{\rotatebox{90}{$#1$}}
\newcommand{\tsup}[1]{\textsuperscript{\underline{#1}}}
\newcommand{\tsub}[1]{\textsubscript{#1}}

\renewcommand{\mod}[1]{~\left[ #1 \right]}
\renewcommand{\t}{{}^t\!}
\newcommand{\s}{\text{\calligra s}}

% custom units / constants
%\DeclareSIUnit{\litre}{\ell}
\let\hbar\hslash

% header / footer
\pagestyle{fancy}
\fancyhead{} \fancyfoot{}
\fancyfoot[C]{\thepage}

% fonts
\let\sc\scshape
\let\bf\bfseries
\let\it\itshape
\let\sl\slshape

% custom math operators
\let\th\relax
\let\det\relax
\DeclareMathOperator*{\codim}{codim}
\DeclareMathOperator*{\dom}{dom}
\DeclareMathOperator*{\gO}{O}
\DeclareMathOperator*{\po}{\text{\cursive o}}
\DeclareMathOperator*{\sgn}{sgn}
\DeclareMathOperator*{\simi}{\sim}
\DeclareMathOperator{\Arccos}{Arccos}
\DeclareMathOperator{\Arcsin}{Arcsin}
\DeclareMathOperator{\Arctan}{Arctan}
\DeclareMathOperator{\Argsh}{Argsh}
\DeclareMathOperator{\Arg}{Arg}
\DeclareMathOperator{\Aut}{Aut}
\DeclareMathOperator{\Card}{Card}
\DeclareMathOperator{\Cl}{\mathcal{C}\!\ell}
\DeclareMathOperator{\Cov}{Cov}
\DeclareMathOperator{\Ker}{Ker}
\DeclareMathOperator{\Mat}{Mat}
\DeclareMathOperator{\PGCD}{PGCD}
\DeclareMathOperator{\PPCM}{PPCM}
\DeclareMathOperator{\Supp}{Supp}
\DeclareMathOperator{\Vect}{Vect}
\DeclareMathOperator{\argmax}{argmax}
\DeclareMathOperator{\argmin}{argmin}
\DeclareMathOperator{\ch}{ch}
\DeclareMathOperator{\com}{com}
\DeclareMathOperator{\cotan}{cotan}
\DeclareMathOperator{\det}{det}
\DeclareMathOperator{\id}{id}
\DeclareMathOperator{\rg}{rg}
\DeclareMathOperator{\rk}{rk}
\DeclareMathOperator{\sh}{sh}
\DeclareMathOperator{\th}{th}
\DeclareMathOperator{\tr}{tr}

% colors and page style
\definecolor{truewhite}{HTML}{ffffff}
\definecolor{white}{HTML}{faf4ed}
\definecolor{trueblack}{HTML}{000000}
\definecolor{black}{HTML}{575279}
\definecolor{mauve}{HTML}{907aa9}
\definecolor{blue}{HTML}{286983}
\definecolor{red}{HTML}{d7827e}
\definecolor{yellow}{HTML}{ea9d34}
\definecolor{gray}{HTML}{9893a5}
\definecolor{grey}{HTML}{9893a5}
\definecolor{green}{HTML}{a0d971}

\pagecolor{white}
\color{black}

\begin{asydef}
	settings.prc = false;
	settings.render=0;

	white = rgb("faf4ed");
	black = rgb("575279");
	blue = rgb("286983");
	red = rgb("d7827e");
	yellow = rgb("f6c177");
	orange = rgb("ea9d34");
	gray = rgb("9893a5");
	grey = rgb("9893a5");
	deepcyan = rgb("56949f");
	pink = rgb("b4637a");
	magenta = rgb("eb6f92");
	green = rgb("a0d971");
	purple = rgb("907aa9");

	defaultpen(black + fontsize(8pt));

	import three;
	currentlight = nolight;
\end{asydef}

% theorems, proofs, ...

\mdfsetup{skipabove=1em,skipbelow=1em, innertopmargin=6pt, innerbottommargin=6pt,}

\declaretheoremstyle[
	headfont=\normalfont\itshape,
	numbered=no,
	postheadspace=\newline,
	headpunct={:},
	qed=\qedsymbol]{demstyle}

\declaretheorem[style=demstyle, name=Démonstration]{dem}

\newcommand\veczero{\kern-1.2pt\vec{\kern1.2pt 0}} % \vec{0} looks weird since the `0' isn't italicized

\makeatletter
\renewcommand{\title}[2]{
	\AtBeginDocument{
		\begin{titlepage}
			\begin{center}
				\vspace{10cm}
				{\Large \sc Chapitre #1}\\
				\vspace{1cm}
				{\Huge \calligra #2}\\
				\vfill
				Hugo {\sc Salou} MPI${}^{\star}$\\
				{\small Dernière mise à jour le \@date }
			\end{center}
		\end{titlepage}
	}
}

\newcommand{\titletp}[4]{
	\AtBeginDocument{
		\begin{titlepage}
			\begin{center}
				\vspace{10cm}
				{\Large \sc tp #1}\\
				\vspace{1cm}
				{\Huge \textsc{\textit{#2}}}\\
				\vfill
				{#3}\textit{MPI}${}^{\star}$\\
			\end{center}
		\end{titlepage}
	}
	\fancyfoot{}\fancyhead{}
	\fancyfoot[R]{#4 \textit{MPI}${}^{\star}$}
	\fancyhead[C]{{\sc tp #1} : #2}
	\fancyhead[R]{\thepage}
}

\newcommand{\titletd}[2]{
	\AtBeginDocument{
		\begin{titlepage}
			\begin{center}
				\vspace{10cm}
				{\Large \sc td #1}\\
				\vspace{1cm}
				{\Huge \calligra #2}\\
				\vfill
				Hugo {\sc Salou} MPI${}^{\star}$\\
				{\small Dernière mise à jour le \@date }
			\end{center}
		\end{titlepage}
	}
}
\makeatother

\newcommand{\sign}{
	\null
	\vfill
	\begin{center}
		{
			\fontfamily{ccr}\selectfont
			\textit{\textbf{\.{\"i}}}
		}
	\end{center}
	\vfill
	\null
}

\renewcommand{\thefootnote}{\emph{\alph{footnote}}}

% figure support
\usepackage{import}
\usepackage{xifthen}
\pdfminorversion=7
\usepackage{pdfpages}
\usepackage{transparent}
\newcommand{\incfig}[1]{%
	\def\svgwidth{\columnwidth}
	\import{./figures/}{#1.pdf_tex}
}

\pdfsuppresswarningpagegroup=1
\ctikzset{tripoles/european not symbol=circle}

\newcommand{\missingpart}{{\large\color{red} Il manque quelque chose ici\ldots}}

\usepackage{tgschola}
\fancyhead[L]{Hugo {\sc Salou}\/ MPI$^\star$}

\begin{document}
	\begin{center}
		\bfseries\scshape\Huge Khôlle n\tsup{o} 8
	\end{center}

	\paragraph{Exercice 2}
	\begin{enumerate}
		\item Posons $U_k$\/ l'événement \guillemotleft~on choisit l'urne $k$.~\guillemotright\@ Les événements $U_1, U_2, \ldots, U_k$\/ forment une partition de l'univers $\Omega$. Ainsi, pour tout $n \in \N^*$,
			\begin{align*}
				u_{K,n}&=P(B_1 \cap B_2 \cap \cdots \cap B_n)\\
				&= \sum_{k=1}^K P(B_1 \cap B_2 \cap \cdots \cap B_n  \mid U_k) \times P(U_k)\text{ d'après les probabilité totales}\\
				&= \sum_{k=1}^K P(B_1  \mid U_k) \cdot P(B_1  \mid B_2 \cap U_k) \cdot \ldots \cdot P(B_n  \mid B_1 \cap \cdots \cap B_{n-1} \cap U_k) \\
				&= \sum_{k=1}^K P(U_k) \prod_{i=1}^n P(B_i  \mid U_k) \text{ par indépendance} \\
				&= \sum_{k=1}^K P(U_k) \prod_{i=1}^n P(B_1  \mid U_k) \text{ par équiprobabilité} \\
				&= \sum_{k=1}^K P(U_k) P(B_1  \mid U_k)^n \\
				&= \sum_{k=1}^K \frac{1}{K} \times \left( \frac{k}{K} \right)^n \\
				&= \frac{1}{K^{n+1}} \times \sum_{k=1}^K k^n. \\
			\end{align*}
		\item On a, pour $n \in \N^*$, \[
				u_{K,n} = \sum_{k=1}^K \frac{k^n}{K^n} \times \frac{1}{K} = \sum_{k=1}^{K-1} \Big( \underset{\substack{\vrt\in\\ [0,1]}}{\frac{k}{K}}\Big)^n \cdot \frac{1}{K} + \frac{1}{K}
			.\]
			Ainsi, pour tout $k \in \llbracket 1,K-1 \rrbracket$, $\left( \frac{k}{K} \right)^n \tendsto{n\to \infty} 0$, et donc, comme la somme est finie, on a \[
				u_{K,n} = \sum_{k=1}^{K-1} \Big( \frac{k}{K} \Big)^n \cdot \frac{1}{K} + \frac{1}{K} \tendsto{n\to \infty} \frac{1}{K}
			.\]
			Cet événement correspond à \guillemotleft~on tire toujours une boule blanche.~\guillemotright\@ Cet événement est certain uniquement si l'urne tirée est celle contenant $K$\/ boules blanches et $0$\/ boules noires, et la probabilité de choisir cette urne est $P(U_K) = \frac{1}{K}$.
		\item On a, pour $n \in \N^*$, \[
				u_{K,n} = \frac{1}{K} \times \sum_{k=1}^K \left( \frac{k}{K} \right)^n = \frac{1}{K} \sum_{k=1}^K f\left( \frac{k}{K} \right)
			\] avec $f : x \mapsto x^n$, qui est continue par morceaux sur $[0,1]$\/.. Par somme de \textsc{Riemann}, on en déduit que \[
			u_{K,n} \tendsto{K\to \infty} \int_{0}^{1} f(x)~\mathrm{d}x = \left[ \frac{x^{n+1}}{n+1} \right]_0^1 = \frac{1}{n+1}
			.\]
	\end{enumerate}
	\paragraph{Exercice 1}
	\begin{enumerate}
		\item Soit $x > -1$, et soit $k \in \N^*$. On a
			\begin{align*}
				|f_k(x)| = f_k(x) &= \frac{1}{k} - \frac{1}{k+x} \\
				&= \frac{1}{k}\left( 1 - \frac{1}{1+\frac{x}{k}} \right) \\
				&= \frac{1}{k}\left( 1 - 1 + \frac{x}{k} + \po\left( \frac{x}{k} \right) \right) \\
				&= \frac{x}{k^2} + \po_{k\to \infty}\left(\frac{1}{k^2}\right).
			\end{align*}
		\item Soit $x > -1$. La série $\sum \frac{x}{n^2}$\/ converge absolument. Ainsi, d'après la question précédente, on en déduit que la série $\sum f_n(x)$\/ converge, la somme $\sum_{k=1}^{+\infty} \left( \frac{1}{k} - \frac{1}{k+x} \right)$\/ existe donc. On en déduit que $S$\/ est défini pour tout réel $x > -1$.
		\item Soit $\varepsilon > 0$, et soient $k \in \N^*$\/ et $x > -1$. On calcule \[
				f_k(x + \varepsilon) - f_k(x) = \frac{1}{k} - \frac{1}{k} - \frac{1}{k + x + \varepsilon} + \frac{1}{k + x} = \frac{1}{k+x} - \frac{1}{k+x+\varepsilon} \ge 0
			\] car $k + x \le k + x + \varepsilon$. Ainsi, par croissance de la somme et comme les inégalités larges passent à la limite, on en déduit que $S(x+ \varepsilon) \ge S(x)$. On conclut que la fonction $S$\/ est croissante.
		\item Soit $a \ge 0$. D'après la question précédente, pour tout $k \in \N^*$, pour tout $x \in {]-1,a]}$, $f_k(x) \le f_k(a)$, par croissance de $f_k$. De plus, $f_k(x) \ge 0$, car $\frac{1}{k} \ge \frac{1}{k+x}$. On en déduit que \[
				\forall k \in \N^*,\forall x \in {]-1,a]},\quad \big|f_k(x)\big| \le f_k(a)
			.\] Or, comme la suite numérique $\sum f_n(a)$\/ converge (d'après la question 1 car $a > -1$), on en déduit que la série de fonctions $\sum f_k$\/ converge normalement sur $]-1,a]$\/ pour $a \ge  0$. Si $a < 0$, alors la série de fonctions converge toujours normalement car $]-1,a] \subset {]-1,0]}$.
		\item Soit $a > -1$. Les fonctions $f_k$\/ sont continues sur $]-1,a]$. La série de fonctions $\sum f_k$\/ converge normalement donc uniformément sur $]-1,a]$. On sait donc que la fonction $S$\/ est continue sur $]-1,a]$. Ceci étant vrai pour tout $a > -1$. On en déduit que $S$\/ est continue sur $]-1+\infty]$.
		\item Soit $n \in \N^*$, et soit $x > -1$. On calcule
			\begin{align*}
				\sum_{k=1}^n f_k(x + 1) - \sum_{k=1}^n f_k(x)
				&= \sum_{k=1}^n \left( f_k(x + 1) - f_k(x) \right) \\
				&= \sum_{k=1}^n \left( \frac{1}{k+x} - \frac{1}{k+x+1} \right) \\
				&= \sum_{k=1}^n \frac{1}{k+x} - \sum_{k=2}^{n+1} \frac{1}{k+x} \\
				&= \frac{1}{1+x} - \frac{1}{n+1+x}\tendsto{n\to \infty} \frac{1}{1+x} \\
			\end{align*}
			Par unicité de la limite, on a donc bien \[
				S(x+1) - S(x) = \frac{1}{1+x}
			.\]
		\item Soit $x > -1$. D'après la question 6, $S(x) = S(x+1) - \frac{1}{1+x}$. Par continuité de $S$, on a $S(x) = S(0) + \po_{x\to 0}(x)$. Or, $S(0) = \sum_{k=1}^\infty f_k(0) = 0$. Ainsi, on a donc \[
				S(x) = S(x+1) - \frac{1}{1+x} = -\frac{1}{1+x} + \po_{x\to -1}(1+x)
			.\]
		\item En itérant la formule de la question 6, on a \[
				\forall n \in \N^*,\:S(n) = S(n-1) + \frac{1}{1+n} = \cdots = S(0) + \sum_{k=1}^n \frac{1}{1+k} = \sum_{k=2}^{n+1} \frac{1}{k} = \ln n + \po(\ln n)
			\] car il s'agit d'un équivalent de la série harmonique.
		\item On a montré que la fonction $S$\/ est croissante. Mais, comme la suite $\big(S(n)\big)_{n \in \N}$\/ tend vers $+\infty$. On en déduit que la fonction $S$\/ tend vers $+\infty$\/ quand $x \to +\infty$.
	\end{enumerate}
\end{document}
