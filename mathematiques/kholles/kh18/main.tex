\documentclass[a4paper]{article}

\usepackage[margin=1in]{geometry}
\usepackage[utf8]{inputenc}
\usepackage[T1]{fontenc}
\usepackage{mathrsfs}
\usepackage{textcomp}
\usepackage[french]{babel}
\usepackage{amsmath}
\usepackage{amssymb}
\usepackage{cancel}
\usepackage{frcursive}
\usepackage[inline]{asymptote}
\usepackage{tikz}
\usepackage[european,straightvoltages,europeanresistors]{circuitikz}
\usepackage{tikz-cd}
\usepackage{tkz-tab}
\usepackage[b]{esvect}
\usepackage[framemethod=TikZ]{mdframed}
\usepackage{centernot}
\usepackage{diagbox}
\usepackage{dsfont}
\usepackage{fancyhdr}
\usepackage{float}
\usepackage{graphicx}
\usepackage{listings}
\usepackage{multicol}
\usepackage{nicematrix}
\usepackage{pdflscape}
\usepackage{stmaryrd}
\usepackage{xfrac}
\usepackage{hep-math-font}
\usepackage{amsthm}
\usepackage{thmtools}
\usepackage{indentfirst}
\usepackage[framemethod=TikZ]{mdframed}
\usepackage{accents}
\usepackage{soulutf8}
\usepackage{mathtools}
\usepackage{bodegraph}
\usepackage{slashbox}
\usepackage{enumitem}
\usepackage{calligra}
\usepackage{cinzel}
\usepackage{BOONDOX-calo}

% Tikz
\usetikzlibrary{babel}
\usetikzlibrary{positioning}
\usetikzlibrary{calc}

% global settings
\frenchspacing
\reversemarginpar
\setuldepth{a}

%\everymath{\displaystyle}

\frenchbsetup{StandardLists=true}

\def\asydir{asy}

%\sisetup{exponent-product=\cdot,output-decimal-marker={,},separate-uncertainty,range-phrase=\;à\;,locale=FR}

\setlength{\parskip}{1em}

\theoremstyle{definition}

% Changing math
\let\emptyset\varnothing
\let\ge\geqslant
\let\le\leqslant
\let\preceq\preccurlyeq
\let\succeq\succcurlyeq
\let\ds\displaystyle
\let\ts\textstyle

\newcommand{\C}{\mathds{C}}
\newcommand{\R}{\mathds{R}}
\newcommand{\Z}{\mathds{Z}}
\newcommand{\N}{\mathds{N}}
\newcommand{\Q}{\mathds{Q}}

\renewcommand{\O}{\emptyset}

\newcommand\ubar[1]{\underaccent{\bar}{#1}}

\renewcommand\Re{\expandafter\mathfrak{Re}}
\renewcommand\Im{\expandafter\mathfrak{Im}}

\let\slantedpartial\partial
\DeclareRobustCommand{\partial}{\text{\rotatebox[origin=t]{20}{\scalebox{0.95}[1]{$\slantedpartial$}}}\hspace{-1pt}}

% merging two maths characters w/ \charfusion
\makeatletter
\def\moverlay{\mathpalette\mov@rlay}
\def\mov@rlay#1#2{\leavevmode\vtop{%
   \baselineskip\z@skip \lineskiplimit-\maxdimen
   \ialign{\hfil$\m@th#1##$\hfil\cr#2\crcr}}}
\newcommand{\charfusion}[3][\mathord]{
    #1{\ifx#1\mathop\vphantom{#2}\fi
        \mathpalette\mov@rlay{#2\cr#3}
      }
    \ifx#1\mathop\expandafter\displaylimits\fi}
\makeatother

% custom math commands
\newcommand{\T}{{\!\!\,\top}}
\newcommand{\avrt}[1]{\rotatebox{-90}{$#1$}}
\newcommand{\bigcupdot}{\charfusion[\mathop]{\bigcup}{\cdot}}
\newcommand{\cupdot}{\charfusion[\mathbin]{\cup}{\cdot}}
%\newcommand{\danger}{{\large\fontencoding{U}\fontfamily{futs}\selectfont\char 66\relax}\;}
\newcommand{\tendsto}[1]{\xrightarrow[#1]{}}
\newcommand{\vrt}[1]{\rotatebox{90}{$#1$}}
\newcommand{\tsup}[1]{\textsuperscript{\underline{#1}}}
\newcommand{\tsub}[1]{\textsubscript{#1}}

\renewcommand{\mod}[1]{~\left[ #1 \right]}
\renewcommand{\t}{{}^t\!}
\newcommand{\s}{\text{\calligra s}}

% custom units / constants
%\DeclareSIUnit{\litre}{\ell}
\let\hbar\hslash

% header / footer
\pagestyle{fancy}
\fancyhead{} \fancyfoot{}
\fancyfoot[C]{\thepage}

% fonts
\let\sc\scshape
\let\bf\bfseries
\let\it\itshape
\let\sl\slshape

% custom math operators
\let\th\relax
\let\det\relax
\DeclareMathOperator*{\codim}{codim}
\DeclareMathOperator*{\dom}{dom}
\DeclareMathOperator*{\gO}{O}
\DeclareMathOperator*{\po}{\text{\cursive o}}
\DeclareMathOperator*{\sgn}{sgn}
\DeclareMathOperator*{\simi}{\sim}
\DeclareMathOperator{\Arccos}{Arccos}
\DeclareMathOperator{\Arcsin}{Arcsin}
\DeclareMathOperator{\Arctan}{Arctan}
\DeclareMathOperator{\Argsh}{Argsh}
\DeclareMathOperator{\Arg}{Arg}
\DeclareMathOperator{\Aut}{Aut}
\DeclareMathOperator{\Card}{Card}
\DeclareMathOperator{\Cl}{\mathcal{C}\!\ell}
\DeclareMathOperator{\Cov}{Cov}
\DeclareMathOperator{\Ker}{Ker}
\DeclareMathOperator{\Mat}{Mat}
\DeclareMathOperator{\PGCD}{PGCD}
\DeclareMathOperator{\PPCM}{PPCM}
\DeclareMathOperator{\Supp}{Supp}
\DeclareMathOperator{\Vect}{Vect}
\DeclareMathOperator{\argmax}{argmax}
\DeclareMathOperator{\argmin}{argmin}
\DeclareMathOperator{\ch}{ch}
\DeclareMathOperator{\com}{com}
\DeclareMathOperator{\cotan}{cotan}
\DeclareMathOperator{\det}{det}
\DeclareMathOperator{\id}{id}
\DeclareMathOperator{\rg}{rg}
\DeclareMathOperator{\rk}{rk}
\DeclareMathOperator{\sh}{sh}
\DeclareMathOperator{\th}{th}
\DeclareMathOperator{\tr}{tr}

% colors and page style
\definecolor{truewhite}{HTML}{ffffff}
\definecolor{white}{HTML}{faf4ed}
\definecolor{trueblack}{HTML}{000000}
\definecolor{black}{HTML}{575279}
\definecolor{mauve}{HTML}{907aa9}
\definecolor{blue}{HTML}{286983}
\definecolor{red}{HTML}{d7827e}
\definecolor{yellow}{HTML}{ea9d34}
\definecolor{gray}{HTML}{9893a5}
\definecolor{grey}{HTML}{9893a5}
\definecolor{green}{HTML}{a0d971}

\pagecolor{white}
\color{black}

\begin{asydef}
	settings.prc = false;
	settings.render=0;

	white = rgb("faf4ed");
	black = rgb("575279");
	blue = rgb("286983");
	red = rgb("d7827e");
	yellow = rgb("f6c177");
	orange = rgb("ea9d34");
	gray = rgb("9893a5");
	grey = rgb("9893a5");
	deepcyan = rgb("56949f");
	pink = rgb("b4637a");
	magenta = rgb("eb6f92");
	green = rgb("a0d971");
	purple = rgb("907aa9");

	defaultpen(black + fontsize(8pt));

	import three;
	currentlight = nolight;
\end{asydef}

% theorems, proofs, ...

\mdfsetup{skipabove=1em,skipbelow=1em, innertopmargin=6pt, innerbottommargin=6pt,}

\declaretheoremstyle[
	headfont=\normalfont\itshape,
	numbered=no,
	postheadspace=\newline,
	headpunct={:},
	qed=\qedsymbol]{demstyle}

\declaretheorem[style=demstyle, name=Démonstration]{dem}

\newcommand\veczero{\kern-1.2pt\vec{\kern1.2pt 0}} % \vec{0} looks weird since the `0' isn't italicized

\makeatletter
\renewcommand{\title}[2]{
	\AtBeginDocument{
		\begin{titlepage}
			\begin{center}
				\vspace{10cm}
				{\Large \sc Chapitre #1}\\
				\vspace{1cm}
				{\Huge \calligra #2}\\
				\vfill
				Hugo {\sc Salou} MPI${}^{\star}$\\
				{\small Dernière mise à jour le \@date }
			\end{center}
		\end{titlepage}
	}
}

\newcommand{\titletp}[4]{
	\AtBeginDocument{
		\begin{titlepage}
			\begin{center}
				\vspace{10cm}
				{\Large \sc tp #1}\\
				\vspace{1cm}
				{\Huge \textsc{\textit{#2}}}\\
				\vfill
				{#3}\textit{MPI}${}^{\star}$\\
			\end{center}
		\end{titlepage}
	}
	\fancyfoot{}\fancyhead{}
	\fancyfoot[R]{#4 \textit{MPI}${}^{\star}$}
	\fancyhead[C]{{\sc tp #1} : #2}
	\fancyhead[R]{\thepage}
}

\newcommand{\titletd}[2]{
	\AtBeginDocument{
		\begin{titlepage}
			\begin{center}
				\vspace{10cm}
				{\Large \sc td #1}\\
				\vspace{1cm}
				{\Huge \calligra #2}\\
				\vfill
				Hugo {\sc Salou} MPI${}^{\star}$\\
				{\small Dernière mise à jour le \@date }
			\end{center}
		\end{titlepage}
	}
}
\makeatother

\newcommand{\sign}{
	\null
	\vfill
	\begin{center}
		{
			\fontfamily{ccr}\selectfont
			\textit{\textbf{\.{\"i}}}
		}
	\end{center}
	\vfill
	\null
}

\renewcommand{\thefootnote}{\emph{\alph{footnote}}}

% figure support
\usepackage{import}
\usepackage{xifthen}
\pdfminorversion=7
\usepackage{pdfpages}
\usepackage{transparent}
\newcommand{\incfig}[1]{%
	\def\svgwidth{\columnwidth}
	\import{./figures/}{#1.pdf_tex}
}

\pdfsuppresswarningpagegroup=1
\ctikzset{tripoles/european not symbol=circle}

\newcommand{\missingpart}{{\large\color{red} Il manque quelque chose ici\ldots}}


\def\khollenum{18}

\fancyhead[L]{Hugo {\sc Salou}\/ MPI$^\star$}
\fancyhead[R]{\scshape Khôlle n\tsup{o} \khollenum}

\begin{document}
	\begin{center}
		\bfseries\scshape\Huge Khôlle n\tsup{o} \khollenum
	\end{center}

	\paragraph{Exercice 1.}
	\begin{enumerate}
		\item Soient $\mathrm{G}_{X_1}$ et $\mathrm{G}_{X_2}$ les fonctions génératrices de $X_1$ et $X_2$ respectivement.
			On note $\mathrm{G}_{X_1+X_2}$ la fonction génératrice de $X_1 + X_2$.
			On sait que $\mathrm{G}_{X_1+X_2} = \mathrm{G}_{X_1} \cdot \mathrm{G}_{X_2}$, car $X_1$ et $X_2$ sont indépendantes.
			De plus, $X_1 \sim \mathcal{P}(\lambda_1)$ d'où $\mathrm{G}_{X_1}(t) = \mathrm{e}^{\lambda_1(t-1)}$, pour tout réel $t$.
			De même, $\mathrm{G}_{X_2}(t) = \mathrm{e}^{\lambda_2(t-1)}$ pour tout réel $t$ car $X_2 \sim \mathcal{P}(\lambda_2)$.
			On en déduit que, pour tout réel $t$, $\mathrm{G}_{X_1+X_2}(t) = \mathrm{e}^{\lambda_1(t-1)} \cdot \mathrm{e}^{\lambda_2(t-1)} = \mathrm{e}^{(\lambda_1+\lambda_2)(t-1)}$.
			On reconnaît la série génératrice d'une loi de Poisson $\mathcal{P}(\lambda_1 + \lambda_2)$. Or, les termes de la série génératrice permettent de déterminer les probabilités des valeurs prisent par la variable aléatoire.
			D'où, $(X_1+X_2)\sim \mathcal{P}(\lambda_1 + \lambda_2)$.
		\item
			\begin{align*}
				&\mathrel{\phantom=} P(X_1 = j  \mid X_1 + X_2 = k)\\
				&= P\big((X_1 = j \cap (X_1+X_2=k)\big) \cdot P(X_1 + X_2 = k) && \text{ car } P(X_1+X_2 = k) \neq 0,\\
				&= P\big((X_1 = j) \cap (X_2 = k - j)\big) \cdot P(X_1 + X_2 = k) \\
				&= P(X_1 = j) \cdot P(X_2 = k - j) \cdot P(X_1 + X_2 = k) && \text{ par indépendance de $X_1$ et $X_2$}, \\
				&= \mathrm{e}^{-\lambda_1} \frac{\lambda_1^j}{j!} \times \mathrm{e}^{-\lambda_2} \frac{\lambda_2^{k-j}}{(k-j)!} \times \mathrm{e}^{-\lambda_1 - \lambda_2} \frac{(\lambda_1 + \lambda_2)^k}{k!} && \text{ car } (X_1 + X_2) \sim \mathcal{P}(\lambda_1+\lambda_2) \\
				&= \mathrm{e}^{-2(\lambda_1+\lambda_2)} \frac{\big(\lambda_1^{k+1}\lambda_2^{k-j} + \lambda_1^{j} \lambda_2^{k-j+1}\big)^k}{j! \cdot (k-j)! \cdot k!} \\
				&= \frac{\mathrm{e}^{-2(\lambda_1+\lambda_2)}}{2\cdot k!} {k\choose j} \big(\lambda_1^{k+1}\lambda_2^{k-j} + \lambda_1^{j} \lambda_2^{k-j+1}\big)^k \\
			\end{align*}
	\end{enumerate}

	\paragraph{Exercice 2.}
	\begin{enumerate}
		\item Soient $\vec{x}$ et $\vec{y}$ deux vecteurs de $\bar{F}$. Soient $\lambda$ et $\mu$ deux réels. Par la caractérisation séquentielle de l'adhérence, il existe deux suites $(\vec{x}_n)_{n \in \N}$ et $(\vec{y}_n)_{n \in \N}$ de vecteurs de $F$ convergent vers $\vec{x}$ et $\vec{y}$ respectivement.
			On pose $(\vec{z}_n)_{n\in\N}$ la suite de vecteurs de $F$ définie par $\vec{z}_i = \lambda \vec{x}_i + \mu \vec{y}_i$, pour tout entier $i$.
			Ainsi, par somme des limites, la suite $(\vec{z}_n)_{n\in\N}$ converge vers $\lambda \vec{x} + \mu \vec{y}$.
			Par la caractérisation séquentielle de l'adhérence, on en déduit que $\lambda \vec{x} + \mu \vec{y} \in \bar{F}$.
			D'où, $\bar{F}$ est un sous-espace vectoriel de $E$.
		\item 
			\begin{enumerate}
				\item On suppose $\bar{F} \neq F$. Soit $\vec{v} \in \bar{F}$ tel que $\vec{v} \not\in F$.
					Alors, $(\Vect \vec{v}) \cap F = \{\vec{0}\}$ donc $\Vect \vec{v}$ et $F$ sont en somme directe.
					Ainsi, comme $F$ admet un supplémentaire de dimension 1 dans $E$, on en déduit que $(\Vect\vec{v}) \oplus F = E$.
				\item On sait que $\bar{F}$ est un sous-espace vectoriel de $E$.
					Or, $\vec{v} \in \bar{F}$ et $F \subset \bar{F}$, donc $E = (\Vect \vec{v}) \oplus F \subset \bar{F}$. Or, $\bar{F} \subset E$ d'après la question 1.
					On en déduit que $\bar{F} = E$.
			\end{enumerate}
	\end{enumerate}

	\paragraph{Exercice 3.}
	\begin{enumerate}
		\item Pour montrer que $\bar{A} = A$, on utilise la caractérisation séquentielle de l'adhérence.
			Soit $(f_n)_{n\in\N}$ une suite de fonctions de $A$ qui converge pour la norme $\|\cdot\|_\infty$ vers une fonction $f \in \mathcal{C}([0,1],\R)$.
			On sait que la suite de fonctions converge uniformément. En effet, \[
				\sup_{t \in [0,1]} \|f_n(t) - f(t)\| \mathrel{\overset*=} \max_{t \in [0,1]} |f_n(t) - f(t)| = \|f_n - f\|_\infty \tendsto{n\to \infty} 0.
			\]L'égalité $*$ est assurée car les fonctions $f$ et $f_n$ sont continues sur un segment.
			Ainsi, on a $0 = f_n(0) \to f(0)$ quand $n \to \infty$ ; de même, $0 = f_n(1) \to f(1)$.
			On en déduit donc que $f(0) = f(1) = 0$.
			De plus, par interversion limite--intégrale sur un segment, \[
				1 = \int_{0}^{1} f_n(t)~\mathrm{d}t \tendsto{n\to \infty} \int_{0}^{1} f(t)~\mathrm{d}t
				\quad\quad \text{ d'où } \quad\quad \int_{0}^{1} f(t)~\mathrm{d}t = 1
			.\]
			On en déduit donc que $f \in A$.
			On peut en conclure que l'ensemble $A$ est un fermé dans $\mathcal{C}([0,1], \R)$.
		\item On considère la suite de fonctions continues $(f_n)_{n \in \N^*}$ définies comme montré dans la figure ci-dessous.
			\begin{figure}[H]
				\centering
				\begin{tikzpicture}[scale=3]
					\draw[->] (-0.1, 0) -- (1.2, 0);
					\draw[->] (0, -0.1) -- (0, 1.2);
					\draw[red,thick] (0, 0) -- (0.25, 1) -- (0.75, 1) -- (1, 0);
					\draw[red, dotted] (0.25, 1) -- (0.25, 0);
					\draw[red, dotted] (0.75, 1) -- (0.75, 0);
					\draw[<->] (0, -0.15) -- (0.25, -0.15);
					\draw[<->] (0.75, -0.15) -- (1, -0.15);
					\node at (0.125, -0.25){${1\over n}$};
					\node at (0.875, -0.25){${1\over n}$};
					\draw (-0.04, 1) -- (0.04, 1);
					\draw (1, -0.04) -- (1, 0.04);
					\node at (-0.1, 1){$b_n$};
					\node at (1, -0.1){$1$};
				\end{tikzpicture}
				\caption{Suite de fonctions $(f_n)_{n \in \N^*}$}
			\end{figure}
			La suite $(b_n)_{n \in \N}$ est définie par, pour $n \in \N^*$, $b_n = 1 / (1 - \frac{1}{n})$.
			Par construction de la suite de fonctions $(f_n)_{n \in \N^*}$, on a bien $f_n(0) = f_n(1) = 0$, pour tout entier $n \in \N^*$.
			De plus, 
			\[
				\int_{0}^{1} f_n(t)~\mathrm{d}t = b_n \times \left( 1 - \frac{1}{n} \right) = 1.
			\]
			On en déduit que $(f_n)_{n \in \N^*}$ est une suite de fonctions de $A$.
			Par définition de $d$, on a $d(0, A) \le d(0, f_n) = \|f_n\|_\infty = b_n$.
			Or, $b_n \to 1$ quand $n \to \infty$, et $\inf$ est le plus grand minorant, donc $d(0, A) \le 1$.
	\end{enumerate}
\end{document}


