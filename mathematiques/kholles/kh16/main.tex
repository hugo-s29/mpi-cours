\documentclass[a4paper]{article}

\usepackage[margin=1in]{geometry}
\usepackage[utf8]{inputenc}
\usepackage[T1]{fontenc}
\usepackage{mathrsfs}
\usepackage{textcomp}
\usepackage[french]{babel}
\usepackage{amsmath}
\usepackage{amssymb}
\usepackage{cancel}
\usepackage{frcursive}
\usepackage[inline]{asymptote}
\usepackage{tikz}
\usepackage[european,straightvoltages,europeanresistors]{circuitikz}
\usepackage{tikz-cd}
\usepackage{tkz-tab}
\usepackage[b]{esvect}
\usepackage[framemethod=TikZ]{mdframed}
\usepackage{centernot}
\usepackage{diagbox}
\usepackage{dsfont}
\usepackage{fancyhdr}
\usepackage{float}
\usepackage{graphicx}
\usepackage{listings}
\usepackage{multicol}
\usepackage{nicematrix}
\usepackage{pdflscape}
\usepackage{stmaryrd}
\usepackage{xfrac}
\usepackage{hep-math-font}
\usepackage{amsthm}
\usepackage{thmtools}
\usepackage{indentfirst}
\usepackage[framemethod=TikZ]{mdframed}
\usepackage{accents}
\usepackage{soulutf8}
\usepackage{mathtools}
\usepackage{bodegraph}
\usepackage{slashbox}
\usepackage{enumitem}
\usepackage{calligra}
\usepackage{cinzel}
\usepackage{BOONDOX-calo}

% Tikz
\usetikzlibrary{babel}
\usetikzlibrary{positioning}
\usetikzlibrary{calc}

% global settings
\frenchspacing
\reversemarginpar
\setuldepth{a}

%\everymath{\displaystyle}

\frenchbsetup{StandardLists=true}

\def\asydir{asy}

%\sisetup{exponent-product=\cdot,output-decimal-marker={,},separate-uncertainty,range-phrase=\;à\;,locale=FR}

\setlength{\parskip}{1em}

\theoremstyle{definition}

% Changing math
\let\emptyset\varnothing
\let\ge\geqslant
\let\le\leqslant
\let\preceq\preccurlyeq
\let\succeq\succcurlyeq
\let\ds\displaystyle
\let\ts\textstyle

\newcommand{\C}{\mathds{C}}
\newcommand{\R}{\mathds{R}}
\newcommand{\Z}{\mathds{Z}}
\newcommand{\N}{\mathds{N}}
\newcommand{\Q}{\mathds{Q}}

\renewcommand{\O}{\emptyset}

\newcommand\ubar[1]{\underaccent{\bar}{#1}}

\renewcommand\Re{\expandafter\mathfrak{Re}}
\renewcommand\Im{\expandafter\mathfrak{Im}}

\let\slantedpartial\partial
\DeclareRobustCommand{\partial}{\text{\rotatebox[origin=t]{20}{\scalebox{0.95}[1]{$\slantedpartial$}}}\hspace{-1pt}}

% merging two maths characters w/ \charfusion
\makeatletter
\def\moverlay{\mathpalette\mov@rlay}
\def\mov@rlay#1#2{\leavevmode\vtop{%
   \baselineskip\z@skip \lineskiplimit-\maxdimen
   \ialign{\hfil$\m@th#1##$\hfil\cr#2\crcr}}}
\newcommand{\charfusion}[3][\mathord]{
    #1{\ifx#1\mathop\vphantom{#2}\fi
        \mathpalette\mov@rlay{#2\cr#3}
      }
    \ifx#1\mathop\expandafter\displaylimits\fi}
\makeatother

% custom math commands
\newcommand{\T}{{\!\!\,\top}}
\newcommand{\avrt}[1]{\rotatebox{-90}{$#1$}}
\newcommand{\bigcupdot}{\charfusion[\mathop]{\bigcup}{\cdot}}
\newcommand{\cupdot}{\charfusion[\mathbin]{\cup}{\cdot}}
%\newcommand{\danger}{{\large\fontencoding{U}\fontfamily{futs}\selectfont\char 66\relax}\;}
\newcommand{\tendsto}[1]{\xrightarrow[#1]{}}
\newcommand{\vrt}[1]{\rotatebox{90}{$#1$}}
\newcommand{\tsup}[1]{\textsuperscript{\underline{#1}}}
\newcommand{\tsub}[1]{\textsubscript{#1}}

\renewcommand{\mod}[1]{~\left[ #1 \right]}
\renewcommand{\t}{{}^t\!}
\newcommand{\s}{\text{\calligra s}}

% custom units / constants
%\DeclareSIUnit{\litre}{\ell}
\let\hbar\hslash

% header / footer
\pagestyle{fancy}
\fancyhead{} \fancyfoot{}
\fancyfoot[C]{\thepage}

% fonts
\let\sc\scshape
\let\bf\bfseries
\let\it\itshape
\let\sl\slshape

% custom math operators
\let\th\relax
\let\det\relax
\DeclareMathOperator*{\codim}{codim}
\DeclareMathOperator*{\dom}{dom}
\DeclareMathOperator*{\gO}{O}
\DeclareMathOperator*{\po}{\text{\cursive o}}
\DeclareMathOperator*{\sgn}{sgn}
\DeclareMathOperator*{\simi}{\sim}
\DeclareMathOperator{\Arccos}{Arccos}
\DeclareMathOperator{\Arcsin}{Arcsin}
\DeclareMathOperator{\Arctan}{Arctan}
\DeclareMathOperator{\Argsh}{Argsh}
\DeclareMathOperator{\Arg}{Arg}
\DeclareMathOperator{\Aut}{Aut}
\DeclareMathOperator{\Card}{Card}
\DeclareMathOperator{\Cl}{\mathcal{C}\!\ell}
\DeclareMathOperator{\Cov}{Cov}
\DeclareMathOperator{\Ker}{Ker}
\DeclareMathOperator{\Mat}{Mat}
\DeclareMathOperator{\PGCD}{PGCD}
\DeclareMathOperator{\PPCM}{PPCM}
\DeclareMathOperator{\Supp}{Supp}
\DeclareMathOperator{\Vect}{Vect}
\DeclareMathOperator{\argmax}{argmax}
\DeclareMathOperator{\argmin}{argmin}
\DeclareMathOperator{\ch}{ch}
\DeclareMathOperator{\com}{com}
\DeclareMathOperator{\cotan}{cotan}
\DeclareMathOperator{\det}{det}
\DeclareMathOperator{\id}{id}
\DeclareMathOperator{\rg}{rg}
\DeclareMathOperator{\rk}{rk}
\DeclareMathOperator{\sh}{sh}
\DeclareMathOperator{\th}{th}
\DeclareMathOperator{\tr}{tr}

% colors and page style
\definecolor{truewhite}{HTML}{ffffff}
\definecolor{white}{HTML}{faf4ed}
\definecolor{trueblack}{HTML}{000000}
\definecolor{black}{HTML}{575279}
\definecolor{mauve}{HTML}{907aa9}
\definecolor{blue}{HTML}{286983}
\definecolor{red}{HTML}{d7827e}
\definecolor{yellow}{HTML}{ea9d34}
\definecolor{gray}{HTML}{9893a5}
\definecolor{grey}{HTML}{9893a5}
\definecolor{green}{HTML}{a0d971}

\pagecolor{white}
\color{black}

\begin{asydef}
	settings.prc = false;
	settings.render=0;

	white = rgb("faf4ed");
	black = rgb("575279");
	blue = rgb("286983");
	red = rgb("d7827e");
	yellow = rgb("f6c177");
	orange = rgb("ea9d34");
	gray = rgb("9893a5");
	grey = rgb("9893a5");
	deepcyan = rgb("56949f");
	pink = rgb("b4637a");
	magenta = rgb("eb6f92");
	green = rgb("a0d971");
	purple = rgb("907aa9");

	defaultpen(black + fontsize(8pt));

	import three;
	currentlight = nolight;
\end{asydef}

% theorems, proofs, ...

\mdfsetup{skipabove=1em,skipbelow=1em, innertopmargin=6pt, innerbottommargin=6pt,}

\declaretheoremstyle[
	headfont=\normalfont\itshape,
	numbered=no,
	postheadspace=\newline,
	headpunct={:},
	qed=\qedsymbol]{demstyle}

\declaretheorem[style=demstyle, name=Démonstration]{dem}

\newcommand\veczero{\kern-1.2pt\vec{\kern1.2pt 0}} % \vec{0} looks weird since the `0' isn't italicized

\makeatletter
\renewcommand{\title}[2]{
	\AtBeginDocument{
		\begin{titlepage}
			\begin{center}
				\vspace{10cm}
				{\Large \sc Chapitre #1}\\
				\vspace{1cm}
				{\Huge \calligra #2}\\
				\vfill
				Hugo {\sc Salou} MPI${}^{\star}$\\
				{\small Dernière mise à jour le \@date }
			\end{center}
		\end{titlepage}
	}
}

\newcommand{\titletp}[4]{
	\AtBeginDocument{
		\begin{titlepage}
			\begin{center}
				\vspace{10cm}
				{\Large \sc tp #1}\\
				\vspace{1cm}
				{\Huge \textsc{\textit{#2}}}\\
				\vfill
				{#3}\textit{MPI}${}^{\star}$\\
			\end{center}
		\end{titlepage}
	}
	\fancyfoot{}\fancyhead{}
	\fancyfoot[R]{#4 \textit{MPI}${}^{\star}$}
	\fancyhead[C]{{\sc tp #1} : #2}
	\fancyhead[R]{\thepage}
}

\newcommand{\titletd}[2]{
	\AtBeginDocument{
		\begin{titlepage}
			\begin{center}
				\vspace{10cm}
				{\Large \sc td #1}\\
				\vspace{1cm}
				{\Huge \calligra #2}\\
				\vfill
				Hugo {\sc Salou} MPI${}^{\star}$\\
				{\small Dernière mise à jour le \@date }
			\end{center}
		\end{titlepage}
	}
}
\makeatother

\newcommand{\sign}{
	\null
	\vfill
	\begin{center}
		{
			\fontfamily{ccr}\selectfont
			\textit{\textbf{\.{\"i}}}
		}
	\end{center}
	\vfill
	\null
}

\renewcommand{\thefootnote}{\emph{\alph{footnote}}}

% figure support
\usepackage{import}
\usepackage{xifthen}
\pdfminorversion=7
\usepackage{pdfpages}
\usepackage{transparent}
\newcommand{\incfig}[1]{%
	\def\svgwidth{\columnwidth}
	\import{./figures/}{#1.pdf_tex}
}

\pdfsuppresswarningpagegroup=1
\ctikzset{tripoles/european not symbol=circle}

\newcommand{\missingpart}{{\large\color{red} Il manque quelque chose ici\ldots}}

\usepackage{cmbright}

\def\khollenum{16}

\fancyhead[L]{Hugo {\sc Salou}\/ MPI$^\star$}
\fancyhead[R]{\scshape Khôlle n\tsup{o} \khollenum}

\begin{document}
	\begin{center}
		\bfseries\scshape\Huge Khôlle n\tsup{o} \khollenum
	\end{center}

	\paragraph{Exercice 1.}
	\begin{enumerate}
		\item Si $\vec{u} = \vec{0}$, alors $r(\vec{u}) = \vec{0} = \cos \theta \cdot \vec{0} + \sin \theta \cdot \vec{a} \land \vec{0}$.
			On suppose maintenant $\vec{u} \neq \vec{0}$. On pose $\vec{x} = \vec{u} / \|\vec{u}\|$.
			La famille $\mathcal{B} = (\vec{a}, \vec{x}, \vec{a} \land \vec{x})$\/ est une base orthonormée directe de $\R^3$. Ainsi, dans cette base, la matrice de $r$ est \[
				\big[\:r\:\big]_\mathcal{B} = \begin{pNiceMatrix}[last-row,last-col]
					1 & 0 & 0 & \vec{a}\\
					0 & \cos \theta & \sin \theta & \vec{x}\\
					0 & -\sin \theta & \cos \theta & \vec{a} \land \vec{x}\\
					r(\vec{a}) & r(\vec{x}) & r(\vec{a} \land \vec{x})
				\end{pNiceMatrix} 
			.\] Or, $\vec{u} = \vec{x} \cdot \|\vec{x}\|$, et donc \[
				\big[\:r(\vec{u})\:\big]_\mathcal{B} = \begin{pNiceMatrix}[last-col]
					0 & \vec{a}\\
					\|\vec{u}\| \cdot \cos \theta & \vec{x}\\
					-\|\vec{u}\| \cdot \sin \theta & \vec{a} \land \vec{x}
				\end{pNiceMatrix} 
			.\]D'où, \[
				r(\vec{u}) = \|\vec{u}\| \cdot \cos \theta \cdot \vec{x} - \|\vec{u}\| \cdot \sin \theta \cdot \vec{a} \land \vec{x} = \cos \theta \cdot \vec{u} - \sin \theta \cdot \vec{a} \land \vec{u}.
			\]
		\item On pose $f$\/ l'application $f : \vec{v} \mapsto \vec{v} - \left<\vec{v}   \mid \vec{a} \right> \vec{a}$. Soit $\mathcal{B} = (\vec{a}, \vec{x}, \vec{y})$\/ une base orthonormée directe de $\R^3$. Montrons que $\Im f = \Vect(\vec{x}, \vec{y})$. On a
			\begin{gather*}
				f(\vec{a}) = \vec{a} - \left<\vec{a} \mid \vec{a} \right>\vec{a} = \vec{a} - \vec{a} = \vec{0},\\
				f(\vec{x}) = \vec{x} - \left<\vec{x} \mid \vec{a} \right>\vec{a} = \vec{x},\\
				f(\vec{y}) = \vec{y} - \left<\vec{y} \mid \vec{a} \right>\vec{a} = \vec{y},
			\end{gather*}
			car $\vec{x} \perp \vec{a}$ et $\vec{y} \perp \vec{a}$. L'application $f$ est la projection orthogonale sur $\Vect(\vec{x}, \vec{y})$.
			Soit $\vec{v} \in \R^3$. On pose $\vec{u} = f(\vec{v}) \in \Vect(\vec{x},\vec{y})$, et $\vec{w} = \vec{v} - \vec{u} \in \Vect(\vec{a})$.
			Ainsi,
			\begin{align*}
				r(\vec{v}) &= r(\vec{u} + \vec{w})\\
				&= r(\vec{u}) + r(\vec{w})\\
				&= \cos(\theta)\vec{u} + \sin(\theta)\vec{a} \land \vec{u} + \vec{w} \\
			\end{align*}
	\end{enumerate}

	\paragraph{Exercice 2.}
	\begin{enumerate}
		\item On calcule \[
				\left<\vec{u} \mid \vec{v} \right> = \frac{1}{\sqrt{6}} - \frac{1}{\sqrt{6}} = 0
				\quad
				\left<\vec{v} \mid \vec{w} \right> = -\frac{1}{\sqrt{12}} + \frac{1}{\sqrt{12}} = 0
				\quad
				\left<\vec{w}  \mid \vec{u} \right> = \frac{1}{\sqrt{3}} (-1 -1 + 2) = 0
			.\] La famille $\mathcal{B} = (u, v, w)$ est donc une famille orthogonale.
			Comme cette famille orthogonale est composée de trois vecteurs non nuls, $\mathcal{B}$ est une base de $\R^3$.
			De plus, $\|\vec{u}\|^2 = \frac{1}{3} \times 3 = 1$, $\|\vec{v}\|^2 = \frac{1}{2} + \frac{1}{2} = 1$ et $\|\vec{w}\|^2 = \frac{1}{6} + \frac{1}{6} + \frac{4}{6} = 1$. On en déduit que la base $\mathcal{B}$ est orthonormée.
			Et, \[
				\vec{u} \land \vec{v} = \frac{1}{\sqrt{3}} \cdot \frac{1}{\sqrt{2}} (-\vec{\imath} + \vec{\jmath} + 2\vec{k}) = +\vec{w}
			.\] La base orthonormée $\mathcal{B}$ est donc directe.

			Dans cette base $\mathcal{B}$, la matrice de $f$ est \[
				\big[\:f\:\big]_\mathcal{B} = \begin{pNiceMatrix}[last-row,last-col]
					1 & 0 & 0 & \vec{u}\\
					0 & \cos \theta & \sin\theta & \vec{v}\\
					0 & -\sin \theta & \cos \theta & \vec{w}\\
					f(\vec{u}) & f(\vec{v}) & f(\vec{w})
				\end{pNiceMatrix} 
			.\]
		\item La matrice de passage de $\mathcal{B}_0$ vers $\mathcal{B}$ est \[
			P = P_{\mathcal{B}_0 \to \mathcal{B}} =
			\begin{pNiceMatrix}[last-row,last-col]
				\frac{1}{\sqrt{3}} & \frac{1}{\sqrt{2}} & -\frac{1}{\sqrt{6}} & \vec{\imath}\\
				-\frac{1}{\sqrt{3}} & \frac{1}{\sqrt{2}} & \frac{1}{\sqrt{6}} & \vec{\jmath}\\
				\frac{1}{\sqrt{3}} & 0 & \frac{2}{\sqrt{6}} & \vec{k}\\
				\vec{u} & \vec{v} & \vec{w}
			\end{pNiceMatrix}
		.\] La matrice $P$ transforme une base orthonormée directe en une base orthonormée directe, d'où $P \in \mathrm{SO}_3(\R)$. Ainsi, la matrice de passage de $\mathcal{B}$ à $\mathcal{B}_0$ est $P^{-1} = P^\top$ : \[
				P^\top = \begin{pNiceMatrix}[last-row,last-col]
					\frac{1}{\sqrt{3}} & -\frac{1}{\sqrt{3}} & \frac{1}{\sqrt{3}} & \vec{u}\\
					\frac{1}{\sqrt{2}} & \frac{1}{\sqrt{2}} & 0 & \vec{v}\\
					-\frac{1}{\sqrt{6}} & \frac{1}{\sqrt{6}} & \frac{2}{\sqrt{6}} & \vec{w}\\
					\vec{\imath} & \vec{\jmath} & \vec{k}
				\end{pNiceMatrix}
			.\]
		\item On procède par \textsc{Analyse-Synthèse}.
	\end{enumerate}
\end{document}


