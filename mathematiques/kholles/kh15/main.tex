\documentclass[a4paper]{article}

\usepackage[margin=1in]{geometry}
\usepackage[utf8]{inputenc}
\usepackage[T1]{fontenc}
\usepackage{mathrsfs}
\usepackage{textcomp}
\usepackage[french]{babel}
\usepackage{amsmath}
\usepackage{amssymb}
\usepackage{cancel}
\usepackage{frcursive}
\usepackage[inline]{asymptote}
\usepackage{tikz}
\usepackage[european,straightvoltages,europeanresistors]{circuitikz}
\usepackage{tikz-cd}
\usepackage{tkz-tab}
\usepackage[b]{esvect}
\usepackage[framemethod=TikZ]{mdframed}
\usepackage{centernot}
\usepackage{diagbox}
\usepackage{dsfont}
\usepackage{fancyhdr}
\usepackage{float}
\usepackage{graphicx}
\usepackage{listings}
\usepackage{multicol}
\usepackage{nicematrix}
\usepackage{pdflscape}
\usepackage{stmaryrd}
\usepackage{xfrac}
\usepackage{hep-math-font}
\usepackage{amsthm}
\usepackage{thmtools}
\usepackage{indentfirst}
\usepackage[framemethod=TikZ]{mdframed}
\usepackage{accents}
\usepackage{soulutf8}
\usepackage{mathtools}
\usepackage{bodegraph}
\usepackage{slashbox}
\usepackage{enumitem}
\usepackage{calligra}
\usepackage{cinzel}
\usepackage{BOONDOX-calo}

% Tikz
\usetikzlibrary{babel}
\usetikzlibrary{positioning}
\usetikzlibrary{calc}

% global settings
\frenchspacing
\reversemarginpar
\setuldepth{a}

%\everymath{\displaystyle}

\frenchbsetup{StandardLists=true}

\def\asydir{asy}

%\sisetup{exponent-product=\cdot,output-decimal-marker={,},separate-uncertainty,range-phrase=\;à\;,locale=FR}

\setlength{\parskip}{1em}

\theoremstyle{definition}

% Changing math
\let\emptyset\varnothing
\let\ge\geqslant
\let\le\leqslant
\let\preceq\preccurlyeq
\let\succeq\succcurlyeq
\let\ds\displaystyle
\let\ts\textstyle

\newcommand{\C}{\mathds{C}}
\newcommand{\R}{\mathds{R}}
\newcommand{\Z}{\mathds{Z}}
\newcommand{\N}{\mathds{N}}
\newcommand{\Q}{\mathds{Q}}

\renewcommand{\O}{\emptyset}

\newcommand\ubar[1]{\underaccent{\bar}{#1}}

\renewcommand\Re{\expandafter\mathfrak{Re}}
\renewcommand\Im{\expandafter\mathfrak{Im}}

\let\slantedpartial\partial
\DeclareRobustCommand{\partial}{\text{\rotatebox[origin=t]{20}{\scalebox{0.95}[1]{$\slantedpartial$}}}\hspace{-1pt}}

% merging two maths characters w/ \charfusion
\makeatletter
\def\moverlay{\mathpalette\mov@rlay}
\def\mov@rlay#1#2{\leavevmode\vtop{%
   \baselineskip\z@skip \lineskiplimit-\maxdimen
   \ialign{\hfil$\m@th#1##$\hfil\cr#2\crcr}}}
\newcommand{\charfusion}[3][\mathord]{
    #1{\ifx#1\mathop\vphantom{#2}\fi
        \mathpalette\mov@rlay{#2\cr#3}
      }
    \ifx#1\mathop\expandafter\displaylimits\fi}
\makeatother

% custom math commands
\newcommand{\T}{{\!\!\,\top}}
\newcommand{\avrt}[1]{\rotatebox{-90}{$#1$}}
\newcommand{\bigcupdot}{\charfusion[\mathop]{\bigcup}{\cdot}}
\newcommand{\cupdot}{\charfusion[\mathbin]{\cup}{\cdot}}
%\newcommand{\danger}{{\large\fontencoding{U}\fontfamily{futs}\selectfont\char 66\relax}\;}
\newcommand{\tendsto}[1]{\xrightarrow[#1]{}}
\newcommand{\vrt}[1]{\rotatebox{90}{$#1$}}
\newcommand{\tsup}[1]{\textsuperscript{\underline{#1}}}
\newcommand{\tsub}[1]{\textsubscript{#1}}

\renewcommand{\mod}[1]{~\left[ #1 \right]}
\renewcommand{\t}{{}^t\!}
\newcommand{\s}{\text{\calligra s}}

% custom units / constants
%\DeclareSIUnit{\litre}{\ell}
\let\hbar\hslash

% header / footer
\pagestyle{fancy}
\fancyhead{} \fancyfoot{}
\fancyfoot[C]{\thepage}

% fonts
\let\sc\scshape
\let\bf\bfseries
\let\it\itshape
\let\sl\slshape

% custom math operators
\let\th\relax
\let\det\relax
\DeclareMathOperator*{\codim}{codim}
\DeclareMathOperator*{\dom}{dom}
\DeclareMathOperator*{\gO}{O}
\DeclareMathOperator*{\po}{\text{\cursive o}}
\DeclareMathOperator*{\sgn}{sgn}
\DeclareMathOperator*{\simi}{\sim}
\DeclareMathOperator{\Arccos}{Arccos}
\DeclareMathOperator{\Arcsin}{Arcsin}
\DeclareMathOperator{\Arctan}{Arctan}
\DeclareMathOperator{\Argsh}{Argsh}
\DeclareMathOperator{\Arg}{Arg}
\DeclareMathOperator{\Aut}{Aut}
\DeclareMathOperator{\Card}{Card}
\DeclareMathOperator{\Cl}{\mathcal{C}\!\ell}
\DeclareMathOperator{\Cov}{Cov}
\DeclareMathOperator{\Ker}{Ker}
\DeclareMathOperator{\Mat}{Mat}
\DeclareMathOperator{\PGCD}{PGCD}
\DeclareMathOperator{\PPCM}{PPCM}
\DeclareMathOperator{\Supp}{Supp}
\DeclareMathOperator{\Vect}{Vect}
\DeclareMathOperator{\argmax}{argmax}
\DeclareMathOperator{\argmin}{argmin}
\DeclareMathOperator{\ch}{ch}
\DeclareMathOperator{\com}{com}
\DeclareMathOperator{\cotan}{cotan}
\DeclareMathOperator{\det}{det}
\DeclareMathOperator{\id}{id}
\DeclareMathOperator{\rg}{rg}
\DeclareMathOperator{\rk}{rk}
\DeclareMathOperator{\sh}{sh}
\DeclareMathOperator{\th}{th}
\DeclareMathOperator{\tr}{tr}

% colors and page style
\definecolor{truewhite}{HTML}{ffffff}
\definecolor{white}{HTML}{faf4ed}
\definecolor{trueblack}{HTML}{000000}
\definecolor{black}{HTML}{575279}
\definecolor{mauve}{HTML}{907aa9}
\definecolor{blue}{HTML}{286983}
\definecolor{red}{HTML}{d7827e}
\definecolor{yellow}{HTML}{ea9d34}
\definecolor{gray}{HTML}{9893a5}
\definecolor{grey}{HTML}{9893a5}
\definecolor{green}{HTML}{a0d971}

\pagecolor{white}
\color{black}

\begin{asydef}
	settings.prc = false;
	settings.render=0;

	white = rgb("faf4ed");
	black = rgb("575279");
	blue = rgb("286983");
	red = rgb("d7827e");
	yellow = rgb("f6c177");
	orange = rgb("ea9d34");
	gray = rgb("9893a5");
	grey = rgb("9893a5");
	deepcyan = rgb("56949f");
	pink = rgb("b4637a");
	magenta = rgb("eb6f92");
	green = rgb("a0d971");
	purple = rgb("907aa9");

	defaultpen(black + fontsize(8pt));

	import three;
	currentlight = nolight;
\end{asydef}

% theorems, proofs, ...

\mdfsetup{skipabove=1em,skipbelow=1em, innertopmargin=6pt, innerbottommargin=6pt,}

\declaretheoremstyle[
	headfont=\normalfont\itshape,
	numbered=no,
	postheadspace=\newline,
	headpunct={:},
	qed=\qedsymbol]{demstyle}

\declaretheorem[style=demstyle, name=Démonstration]{dem}

\newcommand\veczero{\kern-1.2pt\vec{\kern1.2pt 0}} % \vec{0} looks weird since the `0' isn't italicized

\makeatletter
\renewcommand{\title}[2]{
	\AtBeginDocument{
		\begin{titlepage}
			\begin{center}
				\vspace{10cm}
				{\Large \sc Chapitre #1}\\
				\vspace{1cm}
				{\Huge \calligra #2}\\
				\vfill
				Hugo {\sc Salou} MPI${}^{\star}$\\
				{\small Dernière mise à jour le \@date }
			\end{center}
		\end{titlepage}
	}
}

\newcommand{\titletp}[4]{
	\AtBeginDocument{
		\begin{titlepage}
			\begin{center}
				\vspace{10cm}
				{\Large \sc tp #1}\\
				\vspace{1cm}
				{\Huge \textsc{\textit{#2}}}\\
				\vfill
				{#3}\textit{MPI}${}^{\star}$\\
			\end{center}
		\end{titlepage}
	}
	\fancyfoot{}\fancyhead{}
	\fancyfoot[R]{#4 \textit{MPI}${}^{\star}$}
	\fancyhead[C]{{\sc tp #1} : #2}
	\fancyhead[R]{\thepage}
}

\newcommand{\titletd}[2]{
	\AtBeginDocument{
		\begin{titlepage}
			\begin{center}
				\vspace{10cm}
				{\Large \sc td #1}\\
				\vspace{1cm}
				{\Huge \calligra #2}\\
				\vfill
				Hugo {\sc Salou} MPI${}^{\star}$\\
				{\small Dernière mise à jour le \@date }
			\end{center}
		\end{titlepage}
	}
}
\makeatother

\newcommand{\sign}{
	\null
	\vfill
	\begin{center}
		{
			\fontfamily{ccr}\selectfont
			\textit{\textbf{\.{\"i}}}
		}
	\end{center}
	\vfill
	\null
}

\renewcommand{\thefootnote}{\emph{\alph{footnote}}}

% figure support
\usepackage{import}
\usepackage{xifthen}
\pdfminorversion=7
\usepackage{pdfpages}
\usepackage{transparent}
\newcommand{\incfig}[1]{%
	\def\svgwidth{\columnwidth}
	\import{./figures/}{#1.pdf_tex}
}

\pdfsuppresswarningpagegroup=1
\ctikzset{tripoles/european not symbol=circle}

\newcommand{\missingpart}{{\large\color{red} Il manque quelque chose ici\ldots}}


\def\khollenum{15}

\fancyhead[L]{Hugo {\sc Salou}\/ MPI$^\star$}
\fancyhead[R]{\scshape Khôlle n\tsup{o} \khollenum}

\begin{document}
	\begin{center}
		\bfseries\scshape\Huge Khôlle n\tsup{o} \khollenum
	\end{center}

	\paragraph{Exercice 1.}
	\begin{enumerate}
		\item On pose, pour $i \in \llbracket 1,n \rrbracket$, les événements $F_i$\/ : \guillemotleft~la $i$-ème pièce tombe sur \textsc{Face},~\guillemotright\ $G_i$\/~:~\guillemotleft~le joueur $i$ gagne,~\guillemotright\ et $G$ : \guillemotleft~un joueur gagne la partie.~\guillemotright\@ D'après les règles du jeu, on sait que, pour tout $i \in \llbracket 1,n \rrbracket$, \[
				G_i = \Big( F_i \cap \bigcap_{\substack{j \in \llbracket 1,n \rrbracket\\ j \neq i}} \bar{F}_i \Big) \cup \Big( \bar{F}_i \cap \bigcap_{\substack{j \in \llbracket 1,n \rrbracket\\ j \neq i}} F_i\Big)
			,\] et cette union est disjointe. Ainsi, pour $i \in \llbracket 1,n \rrbracket$,
			\begin{align*}
				P(G_i) &= P\Big(F_i \cap \bigcap_{j \neq i} \bar{F}_i\Big) + P\Big(\bar{F}_i \cap \bigcap_{j \neq i} F_i\Big)\\
				&= P(F_i) \times \prod_{j \neq i} P(\bar{F}_i) + P(\bar{F}_i) \times \prod_{j \neq i} P(F_i) \text{ par indépendance des lancers } \\
				&= \frac{1}{2} \times \prod_{j \neq i} \frac{1}{2} + \frac{1}{2} \times \prod_{j \neq i} \frac{1}{2} \\
				&= \prod_{j \neq i} \frac{1}{2} \\
				&= \left(\frac{1}{2}\right)^{n-1} \\
			\end{align*}
			Et, $G = \bigcup_{i = 1}^n G_i$\/ et cette union est disjointe.
			On en déduit donc que \[
				p = P(G) = \sum_{i=1}^n P(G_i) = \sum_{i=1}^n \left(\frac{1}{2}\right)^{n-1} = \frac{n}{2^{n-1}}
			.\]
		\item Chaque partie jouée représente une épreuve de Bernoulli de probabilité de succès $p$.
			Ces épreuves sont indépendantes.
			Et, la variable $X$\/ représente le temps d'attente d'un premier succès.
			On a donc $X \sim \mathcal{G}(p)$.
			Ainsi, pour tout $k \in \N^*$, $P(X = k) = p\:q^{k-1}$\/ où $q = 1 - p$.
			On en déduit \[
				P(X = k) = \frac{n}{2^{n-1}} \cdot \left( 1 - \frac{n}{2^{n-1}} \right)^{k-1}
			.\]
		\item Comme $X \sim \mathcal{G}(p)$, on a \[
				\mathrm{E}(X) = \frac{1}{p} \quad\quad \text{ et } \quad\quad \mathrm{V}(X) = \frac{q}{p^2}
			.\]
	\end{enumerate}

	\paragraph{Exercice 2.}
	\begin{enumerate}
		\item Soit $a \in \R$, et soit $h \ge 0$. On a $(X \le a) \subset (X \le a + h)$, d'où, par croissance de $P$, $P(X \le a) \le P(X \le a + h)$. Ainsi, $F_X(a) \le F_X(a+h)$.
			On en déduit que $F_X$\/ est croissante.
		\item Soient $a$\/ et $b$\/ deux réels, tels que $a \le b$. On sait que $\overline{(a < X \le b)} = (X \le a) \cup (X > b)$, et cette union est disjointe.
			Ainsi,
			\begin{align*}
				P(a < X \le b) &= 1 - P\big(\overline{a < X \le B}\big)\\
				&= 1 - \big[P(X \le a) + P(X > b)\big] \\
				&= 1 - \big[F_X(a) + [1 - P(X \le b)]\big] \\
				&= 1 - F_X(a) - 1 + F_X(b) \\
				&= F_X(b) - F_X(a) \\
			\end{align*}
		\item On a, pour tout $n \in \N$, $A_{n+1} = (X \le a_{n+1}) \subset (X \le a_n) = A_n$\/ car $a_{n+1} \le a_n$. Ainsi, par continuité décroissante, \[
				P\Big(\bigcap_{n \in \N} A_n \Big) = \lim_{n\to \infty} P(A_n)
			.\] Or, $\bigcap_{n \in \N} A_n = \O$. En effet, par l'absurde, soit $u \in \bigcap_{n \in \N} A_n$, alors $\forall n \in \N$, $u \le a_n$, ce qui est absurde car $a_n$\/ tend vers $-\infty$. On a donc \[
				0 = P\Big(\bigcap_{n\in \N}  A_n\Big) = \lim_{n \to \infty}P(A_n) = \lim_{n\to \infty} F_X(a_n)
			.\] Par la caractérisation séquentielle de la limite, on en déduit que $F_X(x) \tendsto{x\to -\infty} 0$.
		\item Soit $(b_n)_{n\in\N}$\/ une suite tendant vers $+\infty$\/ en croissant.
			On pose, pour tout $n \in \N$, $B_n = (X > b_n) = \overline{(X \le b_n)}$.
			Ainsi, on a $B_{n+1} \supset B_n$. Et donc, par continuité croissante, \[
				0 = P\Big(\bigcup_{n \in \N} B_n\Big) = \lim_{n\to \infty} P(B_n) = 1 - \lim_{n\to \infty} F_X(b_n)
			,\] car l'événement $\bigcup_{n \in \N} B_n$\/ est impossible.
			On en déduit donc que $\lim_{n\to \infty} F_X(b_n) = 1$. Par la caractérisation séquentielle de la limite, on a bien \[
				F_X(x) \tendsto{x\to +\infty} 1
			.\]
	\end{enumerate}

	\paragraph{Exercice 3.}
	\begin{enumerate}
		\item Oui. En effet, soient $\vec{x}$\/ et $\vec{y}$\/ deux vecteurs orthogonaux, alors \[
				\left<\lambda g(\vec{x})  \mid \lambda g(\vec{y}) \right> = \lambda^2 \left<g(\vec{x})  \mid g(\vec{y}) \right> = \lambda^2 \left<\vec{x}  \mid \vec{y} \right> = 0
			.\] L'endomorphisme $\lambda g$\/ conserve donc l'orthogonalité.
		\item
			\begin{enumerate}
				\item La base $\mathcal{B} = (\vec{e}_1, \ldots, \vec{e}_n)$\/ étant orthonormée, on a
					\begin{align*}
					\left<\vec{e}_i + \vec{e}_j  \mid \vec{e}_i - \vec{e}_j \right> = \left<\vec{e}_i  \mid \vec{e}_i \right> + \left< \vec{e}_j  \mid \vec{e}_i \right> - \left<\vec{e}_i  \mid \vec{e}_j \right> - \left<\vec{e}_j  \mid \vec{e}_j \right> = \|\vec{e}_i\|^2 - \|\vec{e}_j\|^2 = 0
					\end{align*}
					Comme $f$\/ conserve l'orthogonalité, on a $\left<f(\vec{e}_i + \vec{e}_j)  \mid f(\vec{e}_i - \vec{e}_j) \right> = 0$.
					Mais,
					\begin{align*}
						&\quad\:\left<f(\vec{e}_i + \vec{e}_j)  \mid f(\vec{e}_i - \vec{e}_j) \right>\\
						&= \left<f(\vec{e}_i) + f(\vec{e}_j)  \mid f(\vec{e}_i - f(\vec{e}_j) \right> \\
						&= \left<f(\vec{e}_i)  \mid f(\vec{e}_i) \right> + \left< f(\vec{e}_i) \mid f(\vec{e}_j) \right> - \left<f(\vec{e}_j)  \mid f(\vec{e}_i) \right> - \left<f(\vec{e}_j)  \mid f(\vec{e}_j) \right> \\
						&= \|f(\vec{e}_i)\|^2  - \|f(\vec{e}_j)\|^2\\
					\end{align*}
					car $\left<\vec{e}_i  \mid \vec{e}_j \right> = 0$\/ et $f$\/ conserve l'orthogonalité. Ainsi, $\|f(\vec{e}_i)\| = \|f(\vec{e}_j)\|$\/ car la norme d'un vecteur est positive.
				\item Soient $\vec{x}$\/ et $\vec{y}$\/ deux vecteurs de $E$.
					On pose $\vec{x} = x_1 \vec{e}_1 + \cdots + x_n \vec{e}_n$, et $\vec{y} = y_1 \vec{e}_1 + \cdots + y_n \vec{e}_n$.
					On a
					\begin{align*}
						\left<f(\vec{x}) \mid f(\vec{y}) \right>
						&= \sum_{i=1}^n x_i \left<f(\vec{e}_i)  \mid f(\vec{y}) \right> \\
						&= \sum_{i=1}^n \sum_{j=1}^n x_i y_j \left<f(\vec{e}_i)  \mid f(\vec{e}_j) \right> \\
						&= \sum_{i=1}^n \sum_{j=1}^n x_i y_j \delta_{i,j} \|f(\vec{e}_i)\| \\
						&= \sum_{i=1}^n x_i y_i \|f(\vec{e}_i)\|^2 \\
						&= \|f(\vec{e}_1)\|^2 \sum_{i=1}^n x_i y_i \\
						&= \|f(\vec{e}_i)\|^2 \left<\vec{x} \mid \vec{y} \right> \\
					\end{align*}
					D'où, $\lambda = \|f(\vec{e}_1)\|$.
			\end{enumerate}
		\item On a montré que, si $f$\/ conserve l'orthogonalité, alors il existe $\lambda \in \R$\/ tel que $\left<f(\vec{x})  \mid f(\vec{y}) \right> = \lambda^2 \left<\vec{x}  \mid \vec{y} \right>$, pour tous vecteurs $\vec{x}$\/ et $\vec{y}$.
			On suppose $\lambda$\/ non nul (le cas $\lambda = 0$\/ est traité après).
			Ainsi, l'endomorphisme $g = f / \lambda$\/ conserve l'orthogonalité par bilinéarité du produit scalaire, et donc \[
				\left<g(\vec{x})  \mid g(\vec{y}) \right> = \left<\vec{x}  \mid \vec{y} \right> \text{ pour tous vecteurs } \vec{x} \text{ et } \vec{y}
			.\]
			On a donc $g \in \mathrm{O}(E)$. Ainsi, on a bien montré qu'il existe une isométrie vectorielle $g$\/ telle que $f = \lambda g$.
			Si $\lambda = 0$, alors $\forall \vec{x},\vec{y} \in E$, $\left<f(\vec{x})  \mid f(\vec{y}) \right> = 0$, en particulier $\|f(\vec{x})\|^2 = 0$, d'où $f : \vec{x} \mapsto \vec{0}$. Ainsi, on pose $g = \id_E$, et on a $f = \lambda g$.
			
			Réciproquement, soient $\lambda \in \R$\/ et $g \in \mathrm{O}(E)$, tels que $f = \lambda g$, alors $f$\/ conserve l'orthogonalité d'après la question 1.
	\end{enumerate}
\end{document}


