\documentclass[a4paper]{article}

\usepackage[margin=1.5in]{geometry}
\usepackage[margin=1in]{geometry}
\usepackage[utf8]{inputenc}
\usepackage[T1]{fontenc}
\usepackage{mathrsfs}
\usepackage{textcomp}
\usepackage[french]{babel}
\usepackage{amsmath}
\usepackage{amssymb}
\usepackage{cancel}
\usepackage{frcursive}
\usepackage[inline]{asymptote}
\usepackage{tikz}
\usepackage[european,straightvoltages,europeanresistors]{circuitikz}
\usepackage{tikz-cd}
\usepackage{tkz-tab}
\usepackage[b]{esvect}
\usepackage[framemethod=TikZ]{mdframed}
\usepackage{centernot}
\usepackage{diagbox}
\usepackage{dsfont}
\usepackage{fancyhdr}
\usepackage{float}
\usepackage{graphicx}
\usepackage{listings}
\usepackage{multicol}
\usepackage{nicematrix}
\usepackage{pdflscape}
\usepackage{stmaryrd}
\usepackage{xfrac}
\usepackage{hep-math-font}
\usepackage{amsthm}
\usepackage{thmtools}
\usepackage{indentfirst}
\usepackage[framemethod=TikZ]{mdframed}
\usepackage{accents}
\usepackage{soulutf8}
\usepackage{mathtools}
\usepackage{bodegraph}
\usepackage{slashbox}
\usepackage{enumitem}
\usepackage{calligra}
\usepackage{cinzel}
\usepackage{BOONDOX-calo}

% Tikz
\usetikzlibrary{babel}
\usetikzlibrary{positioning}
\usetikzlibrary{calc}

% global settings
\frenchspacing
\reversemarginpar
\setuldepth{a}

%\everymath{\displaystyle}

\frenchbsetup{StandardLists=true}

\def\asydir{asy}

%\sisetup{exponent-product=\cdot,output-decimal-marker={,},separate-uncertainty,range-phrase=\;à\;,locale=FR}

\setlength{\parskip}{1em}

\theoremstyle{definition}

% Changing math
\let\emptyset\varnothing
\let\ge\geqslant
\let\le\leqslant
\let\preceq\preccurlyeq
\let\succeq\succcurlyeq
\let\ds\displaystyle
\let\ts\textstyle

\newcommand{\C}{\mathds{C}}
\newcommand{\R}{\mathds{R}}
\newcommand{\Z}{\mathds{Z}}
\newcommand{\N}{\mathds{N}}
\newcommand{\Q}{\mathds{Q}}

\renewcommand{\O}{\emptyset}

\newcommand\ubar[1]{\underaccent{\bar}{#1}}

\renewcommand\Re{\expandafter\mathfrak{Re}}
\renewcommand\Im{\expandafter\mathfrak{Im}}

\let\slantedpartial\partial
\DeclareRobustCommand{\partial}{\text{\rotatebox[origin=t]{20}{\scalebox{0.95}[1]{$\slantedpartial$}}}\hspace{-1pt}}

% merging two maths characters w/ \charfusion
\makeatletter
\def\moverlay{\mathpalette\mov@rlay}
\def\mov@rlay#1#2{\leavevmode\vtop{%
   \baselineskip\z@skip \lineskiplimit-\maxdimen
   \ialign{\hfil$\m@th#1##$\hfil\cr#2\crcr}}}
\newcommand{\charfusion}[3][\mathord]{
    #1{\ifx#1\mathop\vphantom{#2}\fi
        \mathpalette\mov@rlay{#2\cr#3}
      }
    \ifx#1\mathop\expandafter\displaylimits\fi}
\makeatother

% custom math commands
\newcommand{\T}{{\!\!\,\top}}
\newcommand{\avrt}[1]{\rotatebox{-90}{$#1$}}
\newcommand{\bigcupdot}{\charfusion[\mathop]{\bigcup}{\cdot}}
\newcommand{\cupdot}{\charfusion[\mathbin]{\cup}{\cdot}}
%\newcommand{\danger}{{\large\fontencoding{U}\fontfamily{futs}\selectfont\char 66\relax}\;}
\newcommand{\tendsto}[1]{\xrightarrow[#1]{}}
\newcommand{\vrt}[1]{\rotatebox{90}{$#1$}}
\newcommand{\tsup}[1]{\textsuperscript{\underline{#1}}}
\newcommand{\tsub}[1]{\textsubscript{#1}}

\renewcommand{\mod}[1]{~\left[ #1 \right]}
\renewcommand{\t}{{}^t\!}
\newcommand{\s}{\text{\calligra s}}

% custom units / constants
%\DeclareSIUnit{\litre}{\ell}
\let\hbar\hslash

% header / footer
\pagestyle{fancy}
\fancyhead{} \fancyfoot{}
\fancyfoot[C]{\thepage}

% fonts
\let\sc\scshape
\let\bf\bfseries
\let\it\itshape
\let\sl\slshape

% custom math operators
\let\th\relax
\let\det\relax
\DeclareMathOperator*{\codim}{codim}
\DeclareMathOperator*{\dom}{dom}
\DeclareMathOperator*{\gO}{O}
\DeclareMathOperator*{\po}{\text{\cursive o}}
\DeclareMathOperator*{\sgn}{sgn}
\DeclareMathOperator*{\simi}{\sim}
\DeclareMathOperator{\Arccos}{Arccos}
\DeclareMathOperator{\Arcsin}{Arcsin}
\DeclareMathOperator{\Arctan}{Arctan}
\DeclareMathOperator{\Argsh}{Argsh}
\DeclareMathOperator{\Arg}{Arg}
\DeclareMathOperator{\Aut}{Aut}
\DeclareMathOperator{\Card}{Card}
\DeclareMathOperator{\Cl}{\mathcal{C}\!\ell}
\DeclareMathOperator{\Cov}{Cov}
\DeclareMathOperator{\Ker}{Ker}
\DeclareMathOperator{\Mat}{Mat}
\DeclareMathOperator{\PGCD}{PGCD}
\DeclareMathOperator{\PPCM}{PPCM}
\DeclareMathOperator{\Supp}{Supp}
\DeclareMathOperator{\Vect}{Vect}
\DeclareMathOperator{\argmax}{argmax}
\DeclareMathOperator{\argmin}{argmin}
\DeclareMathOperator{\ch}{ch}
\DeclareMathOperator{\com}{com}
\DeclareMathOperator{\cotan}{cotan}
\DeclareMathOperator{\det}{det}
\DeclareMathOperator{\id}{id}
\DeclareMathOperator{\rg}{rg}
\DeclareMathOperator{\rk}{rk}
\DeclareMathOperator{\sh}{sh}
\DeclareMathOperator{\th}{th}
\DeclareMathOperator{\tr}{tr}

% colors and page style
\definecolor{truewhite}{HTML}{ffffff}
\definecolor{white}{HTML}{faf4ed}
\definecolor{trueblack}{HTML}{000000}
\definecolor{black}{HTML}{575279}
\definecolor{mauve}{HTML}{907aa9}
\definecolor{blue}{HTML}{286983}
\definecolor{red}{HTML}{d7827e}
\definecolor{yellow}{HTML}{ea9d34}
\definecolor{gray}{HTML}{9893a5}
\definecolor{grey}{HTML}{9893a5}
\definecolor{green}{HTML}{a0d971}

\pagecolor{white}
\color{black}

\begin{asydef}
	settings.prc = false;
	settings.render=0;

	white = rgb("faf4ed");
	black = rgb("575279");
	blue = rgb("286983");
	red = rgb("d7827e");
	yellow = rgb("f6c177");
	orange = rgb("ea9d34");
	gray = rgb("9893a5");
	grey = rgb("9893a5");
	deepcyan = rgb("56949f");
	pink = rgb("b4637a");
	magenta = rgb("eb6f92");
	green = rgb("a0d971");
	purple = rgb("907aa9");

	defaultpen(black + fontsize(8pt));

	import three;
	currentlight = nolight;
\end{asydef}

% theorems, proofs, ...

\mdfsetup{skipabove=1em,skipbelow=1em, innertopmargin=6pt, innerbottommargin=6pt,}

\declaretheoremstyle[
	headfont=\normalfont\itshape,
	numbered=no,
	postheadspace=\newline,
	headpunct={:},
	qed=\qedsymbol]{demstyle}

\declaretheorem[style=demstyle, name=Démonstration]{dem}

\newcommand\veczero{\kern-1.2pt\vec{\kern1.2pt 0}} % \vec{0} looks weird since the `0' isn't italicized

\makeatletter
\renewcommand{\title}[2]{
	\AtBeginDocument{
		\begin{titlepage}
			\begin{center}
				\vspace{10cm}
				{\Large \sc Chapitre #1}\\
				\vspace{1cm}
				{\Huge \calligra #2}\\
				\vfill
				Hugo {\sc Salou} MPI${}^{\star}$\\
				{\small Dernière mise à jour le \@date }
			\end{center}
		\end{titlepage}
	}
}

\newcommand{\titletp}[4]{
	\AtBeginDocument{
		\begin{titlepage}
			\begin{center}
				\vspace{10cm}
				{\Large \sc tp #1}\\
				\vspace{1cm}
				{\Huge \textsc{\textit{#2}}}\\
				\vfill
				{#3}\textit{MPI}${}^{\star}$\\
			\end{center}
		\end{titlepage}
	}
	\fancyfoot{}\fancyhead{}
	\fancyfoot[R]{#4 \textit{MPI}${}^{\star}$}
	\fancyhead[C]{{\sc tp #1} : #2}
	\fancyhead[R]{\thepage}
}

\newcommand{\titletd}[2]{
	\AtBeginDocument{
		\begin{titlepage}
			\begin{center}
				\vspace{10cm}
				{\Large \sc td #1}\\
				\vspace{1cm}
				{\Huge \calligra #2}\\
				\vfill
				Hugo {\sc Salou} MPI${}^{\star}$\\
				{\small Dernière mise à jour le \@date }
			\end{center}
		\end{titlepage}
	}
}
\makeatother

\newcommand{\sign}{
	\null
	\vfill
	\begin{center}
		{
			\fontfamily{ccr}\selectfont
			\textit{\textbf{\.{\"i}}}
		}
	\end{center}
	\vfill
	\null
}

\renewcommand{\thefootnote}{\emph{\alph{footnote}}}

% figure support
\usepackage{import}
\usepackage{xifthen}
\pdfminorversion=7
\usepackage{pdfpages}
\usepackage{transparent}
\newcommand{\incfig}[1]{%
	\def\svgwidth{\columnwidth}
	\import{./figures/}{#1.pdf_tex}
}

\pdfsuppresswarningpagegroup=1
\ctikzset{tripoles/european not symbol=circle}

\newcommand{\missingpart}{{\large\color{red} Il manque quelque chose ici\ldots}}


\setlist[description]{font=\normalfont\scshape}

\def\khollenum{17}

\fancyhead[L]{Hugo {\sc Salou}\/ MPI$^\star$}
\fancyhead[R]{\scshape Khôlle n\tsup{o} \khollenum}

\begin{document}
	\begin{center}
		\bfseries\scshape\Huge Khôlle n\tsup{o} \khollenum
	\end{center}

	\paragraph{Exercice 1.}
	\begin{enumerate}
		\item
			\begin{itemize}
				\item Soient $u$ et $v$ deux suites de $\ell^1$, et soient $\alpha$ et $\beta$ deux réels.
					Pour $n \in \N$, $0 \le |\alpha u_n + \beta v_n| \le |\alpha| \cdot |u_n| + |\beta| \cdot |v_n|$.
					Or, les séries $\sum |u_n|$ et $\sum|v_n|$ convergent. D'où, $\sum |\alpha u_n + \beta v_n|$ converge.
					Ainsi, $(\alpha u + \beta v) \in \ell^1$.
					De plus, $(0)_{n\in\N} \in \ell^1$, donc~$\ell^1 \neq \O$.
					L'ensemble $\ell^1$ est donc un sous-espace vectoriel de $\R^\N$.
				\item Soient $u$ et $v$ deux suites de $\ell^2$, et soient $\alpha$ et $\beta$ deux réels.
					Pour $n \in \N$, $0 \le (\alpha u_n + \beta v_n)^2 \le \alpha^2 \cdot u_n{}^2 + \beta^2 \cdot v_n{}^2$.
					Or, les séries $\sum u_n{}^2$ et $\sum v_n{}^2$ convergent. D'où, $\sum(\alpha u_n + \beta v_n)^2$ converge.
					Ainsi, $(\alpha u + \beta v) \in \ell^2$.
					De plus, $(0)_{n\in\N} \in \ell^2$, donc $\ell^2 \neq \O$.
					L'ensemble $\ell^2$ est donc un sous-espace vectoriel de $\R^\N$.
				\item Soient $u$ et $v$ deux suites de $\ell^\infty$, et soient $\alpha$ et $\beta$ deux réels.
					Soient $m$ et $m'$ deux réels tels que, pour tout $n \in \N$, $|u_n| \le m$ et $|v_n| \le m'$.
					Or, pour $n \in \N$, d'après l'inégalité triangulaire : \[
						|\alpha u_n + \beta v_n| \le |\alpha| \cdot |u_n| + |\beta| \cdot |v_n| \le |\alpha| \cdot m + |\beta| \cdot m'
					,\] qui est un majorant.
					D'où, $(\alpha u + \beta v) \in \ell^\infty$.
					De plus, $(0)_{n\in\N} \in \ell^\infty$, donc $\ell^\infty \neq \O$.
					L'ensemble $\ell^\infty$ est donc un sous-espace vectoriel de $\R^\N$.
				\item Soit $u$ une suite de $\ell^1$. La série $\sum |u_n|$ converge donc la suite $(|u_n|)_{n \in \N}$ tend vers 0. Il existe donc un certain rang $N$ tel que, pour tout $n \ge N$, $|u_n| \le 1$.
					Ainsi, pour tout $n \ge N$, $|u_n| \ge |u_n|^2 \ge 0$.
					Or, la série $\sum |u_n|$ converge. On en déduit que la série $\sum |u_n|^2 = \sum u_n{}^2$ converge.
					D'où $u \in \ell^2$.
					On en déduit $\ell^1 \subset \ell^2$.
				\item Soit $u$ une suite de $\ell^2$. La série $\sum u_n{}^2$ converge, donc la suite $(u_n{}^2)_{n \in \N}$ tend vers 0, elle est donc majorée. On pose $M > 0$ un majorant et $m = \sqrt{M}$. Ainsi, pour tout $n \in \N$, $|u_n{}^2| \le M$, donc $|u_n|^2 \le m^2$ et donc $0 \le |u_n| \le m$.
					La suite $u$ est donc majorée par $m$, d'où $u \in \ell^\infty$.
					On en déduit $\ell^2 \subset \ell^\infty$.
			\end{itemize}
		\item 
			\begin{enumerate}
				\item Soit $u \in \ell^1$. Montrons que $\|u\|_\infty \le \|u\|_1$.
					La suite $(|u_n|)_{n\in\N}$ tend vers 0.
					Ainsi $\sup_{n \in \N}\: |u_n| = \max_{n \in \N}\: |u_n|$.
					On pose $i \in \N$ tel que $|u_i| = \max_{n \in \N}\: |u_n|$.
					\[
						\|u\|_\infty = |u_i| \le \sum_{k=0}^\infty |u_k| = \|u\|_1
					\] car les termes de la somme de la série $\sum |u_n|$ sont positifs.
					On a donc montré que $\alpha \le 1$.
					Cette valeur de $\alpha$ est la plus petite. En effet, on considère la suite $u \in \ell^1$ définie par $u_0 = 1$, et pour tout $n \in \N^*$, $u_n = 0$. On a $\|u\|_\infty = 1$ et $\|u\|_1 = 1$. D'où, $1 \le \alpha \cdot  1$.
					On en déduit que $\alpha = 1$.
				\item Non, il n'existe pas un réel $\beta$ tel que, pour toute suite $u \in \ell^1$, $\|u\|_\infty \beta\ge \|u\|_1$.
					En effet, par l'absurde, supposons que ce réel $\beta$ existe.
					On considère la suite $u(m)$ définie par, pour tout $i \in \llbracket 1,m \rrbracket$, $u(m)_i = 0$, et pour tout $i > m$, $u(m)_i = 0$.
					Pour tout entier $m$, on a $\|u(m)\|_\infty = 1$, et $\|u(m)\|_1 = m$.
					D'où, pour tout entier $m$, $\beta \ge m$, ce qui est absurde.
				\item Non. En effet, il n'existe pas de réel $\beta$ tel que, pour toute suite $u \in \ell^1$, $\|u\|_1 \le \beta \|u\|_2$.
					On considère la suite $u(m)$ définie à la question précédente.
					On a $\|u(m)\|_1 = m$, et $\|u(m)\|_2 = \sqrt{n}$. D'où, pour tout entier $m$, $m \le \beta \sqrt{m}$, et donc $\beta \ge \sqrt{m}$, pour tout entier $m$, ce qui est absurde.
			\end{enumerate}
	\end{enumerate}

	\paragraph{Exercice 2.}
	Montrons que l'application $f$ est linéaire et continue en 0.
	\begin{itemize}
		\item Soient $P$ et $Q$ deux polynômes, et soient $\alpha$ et $\beta$ deux réels.
			On pose $n = \max(\deg P, \deg Q)$, $P = \sum_{k=0}^n a_k X^k$ et $Q = \sum_{k=0}^n b_k X^k$.
			Ainsi,
			\[
				f(\alpha P + \beta Q) = \sum_{k=0}^n \frac{\alpha a_k + \beta b_k}{k + 1} = \alpha\sum_{k=0}^n \frac{a_k}{k+1} + \beta \sum_{k=0}^n \frac{b_k}{k+1} = \alpha f(P) + \beta f(Q),
			\]l'application $f$ est donc linéaire.
		\item Soit $(P_n)_{n\in\N}$ une suite de polynômes convergent vers $0$.
			Montrons que $|f(P_n)| \to 0$ quand $n \to \infty$.
			On pose, pour tout $n \in \N$, $P_n = \sum_{k=0}^{\deg P_n} a_{k,n} X^k$ ; de plus, pour $k \ge \deg P_n$, on pose $a_{k,n} = 0$.
			Ainsi, $\|P_n\| = \sum_{k=0}^\infty a_{k,n}^2$. Cette somme converge car elle est finie : les termes sont tous nuls à partir d'un certain rang.
			Or, $\|P_n\| \to 0$ quand $n\to \infty$, et la somme n'est composée que de termes positifs ou nuls. On en déduit que, pour tout $k \in \N$, $a_{k,n}^2 \to 0$, et donc $a_{k,n}\to 0$ quand $n\to \infty$.
			Et, pour tout $k \in \N$, $0 \le \big| a_{k,n} / (k + 1) \big| \le |a_{k,n}| \to 0$.
			Par le théorème des gendarmes, chaque terme de la somme \[
				f(P_n) = \sum_{k=0}^{\deg P_n} \frac{a_{k,n}}{k+1}
			\] tend vers 0, donc la somme tend vers 0.
			Ainsi, on a bien $|f(P_n)| \to 0$ quand $n\to \infty$.
			La fonction $f$ est donc continue en $0$.
	\end{itemize}
	On en déduit que la fonction $f$ est continue sur $\R[X]$.

	\paragraph{Exercice 3.}
	\begin{itemize}
		\item Soit $\vec{u} = (x,y) \in \R^2$. Si $N(\vec{u}) = 0$, alors $\sup_{t \in [0,1]}\: |x+ty| = 0$. Or, $|x+ty| \ge 0$. On en déduit que, pour tout $t\in [0,1]$, $x + ty = 0$.
			En particulier, pour $t = 0$, on a $x = 0$ ; puis, pour $t = 1$, on a $x + y = y = 0$.
			Ainsi, $\vec{u} = \vec{0}$.
		\item Soit $\vec{u} = (x,y) \in \R^2$ et soit $\alpha$ un réel. Pour $t \in [0,1]$, $|\alpha x + t \alpha y| = |\alpha| \cdot |x + ty| \le |\alpha| \cdot N(\vec{u})$, qui est un majorant. D'où, $N(\alpha \vec{u}) = \sup_{t \in [0,1]}\: |\alpha| \cdot |x+ty| = |\alpha| \cdot N(\vec{u})$.
		\item Soient $\vec{u} = (a,b) \in \R^2$ et $\vec{v} = (c,d) \in \R^2$. Pour $t \in [0,1]$, on a $|(a+c) + t(b+d)| = |(a + tb) + (c+ td)| \le |a + tb| + |c + td| \le N(\vec{u}) + N(\vec{v})$, qui est un majorant. D'où, $N(\vec{u} + \vec{v}) \le N(\vec{u}) + N(\vec{v})$.
	\end{itemize}
	On en déduit que $N$ est une norme sur $\R^2$.
	Pour dessiner la boule $\bar{B}(\vec{0}, 1)$, on procède par analyse--synthèse.
	\begin{description}
		\item[Analyse] Soit $\vec{u} = (x,y) \in \bar{B}(\vec{0}, 1)$. Ainsi, $N(\vec{u}) = \sup_{t \in [0,1]}\:|x+ty| \le 1$, d'où, pour tout $t \in [0,1]$, $|x+ty| \le 1$.
			En particulier, pour $t = 0$, on a $|x| \le 1$, donc $-1 \le x \le 1$ ; de plus, en $t = 1$, on a $|x+y| \le 1$, d'où $-1 \le x + y \le 1$.
		\item[Synthèse]
			Soit $x \in [-1,1]$, et soit $y \in [-1-x, 1-x]$.
			On pose $\vec{u} = (x,y) \in \R^2$.
			Montrons que $N(\vec{u}) \le 1$.
			La fonction $f: t \mapsto |x+ty|$ est continue sur $[0,1]$, donc $N(\vec{u}) = \max_{t \in [0,1]}\:|x+ty|$.
			Montrons que ce maximum est atteint en $t = 0$ ou $t = 1$.
			Ce résultat est clairement vrai si la fonction $f$ est monotone.
			On suppose maintenant que cette fonction n'est pas monotone.
			Ainsi, la fonction $t \mapsto x + ty$ change de signe sur $[0,1]$.
			Cette fonction s'annule une fois en un point d'abscisse $\alpha \in {]0,1[}$
			Sur $[0,\alpha]$, $f$ est monotone et le maximum est atteint en $0$ ou en $\alpha$ ;
			or, $f$ est positive, et $f(\alpha) = 0$ ; le maximum est donc atteint en $0$, sur cet intervalle.
			Sur $[\alpha,1]$, $f$ est monotone, et le maximum est atteint en $1$ ou en $\alpha$.
			Comme $f$ est positive, on en déduit que le maximum est atteint en $1$ sur cet intervalle.
			Ainsi, le maximum est atteint en $0$ ou en $1$ sur l'intervalle $[0,1]$.
	\end{description}
	\begin{figure}[H]
		\centering
		\begin{tikzpicture}
			\draw[fill,red,fill opacity=0.2,thick] (-1,2)--(1,0)--(1,-2)--(-1,0)--cycle;
			\draw[->] (-3,0)--(3,0);
			\draw[->] (0,-3)--(0,3);
			\foreach \x in {-2,...,2}{
				\draw (\x,-0.1)--(\x,0.1);
				\draw (-0.1,\x)--(0.1,\x);
				\node at (-0.2, \x-0.2){\small$\x$};
				\node at (\x-0.2, -0.2){\small$\x$};
			}
			\node[red] at (1.2,-2.2) {$\bar{B}(\vec{0},1)$};
		\end{tikzpicture}
		\caption{Boule fermée centrée en $\vec{0}$ de rayon 1, pour la norme $N$}
	\end{figure}
\end{document}


