\section{Exercice 1}

On considère la série entière $\sum \tan\left( \frac{n \pi}{7} \right) x^n$. La fonction $\tan$\/ est $\pi$-périodique, d'où la suite $(a_n)_{n\in\N} = \big(\tan \frac{n\pi}{7}\big)_{n \in \N}$\/ est $7$-périodique. D'où, la suite $(a_n)_{n\in\N}$\/ est bornée ; il existe $M \in \R^+$\/ tel que, pour $n \in \N$, $|a_n| \le M$.
Or, le rayon de convergence de la série entière $\sum M x^n = M \sum x^n$\/ vaut 1. D'où, le rayon $R$\/ de convergence est $R \ge 1$.
De plus, \smash{$a_n \centernot{\tendsto{n\to \infty}} 0$}, la série $\sum a_n$\/ diverge grossièrement, donc $R \le 1$. On en déduit que  \[
	\boxed{R = 1.}
\]
