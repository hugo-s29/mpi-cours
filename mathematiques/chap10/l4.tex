\section{(Ne pas) être développable en série entière}

\begin{defn}
	Soit $r \in \R_*^+ \cup \{+\infty\} $. On dit qu'une fonction $f$\/ est \textit{développable en série entière} sur $]-r,r[$\/ s'il existe une série entière $\sum a_n x^n$\/ qui a un rayon de convergence $R \ge r$\/ telle que \[
		\forall x \in {]-r,r[},\quad f(x) = \sum_{n=0}^\infty a_n x^n
	.\]
\end{defn}

\begin{exm}
	$\O$
\end{exm}

\begin{rmk}
	$\O$\/
\end{rmk}

\begin{exo}
	Soit $f : \begin{array}{rcl}
		\R &\longrightarrow& \R \\
		x &\longmapsto& \begin{cases}
			\mathrm{e}^{-1 / x^2}&\text{ si } x \neq 0\\
			0 &\text{ si } x = 0
		\end{cases}
	\end{array}$

	\begin{enumerate}
		\item Montrons que, pour $n \in \N$, $f$\/ est $\mathcal{C}^n$\/ sur $\R^*$, et il existe un polynôme $P_n$\/ tel que $\forall x \neq 0$, $f^{(n)}(x) = P_n(x)\:\frac{1}{x^{3n}}\: \mathrm{e}^{- 1 / x^2}$.
			\begin{itemize}
				\item Pour $n = 0$, alors $f^{(0)}(x) = \mathrm{e}^{- 1 / x^2} = \frac{1}{x^{3\times 0}}\times \mathrm{e}^{-1 / x^2}$, d'où $P_0(X) = 1$.
				\item On suppose $f^{(n)}(x) = \frac{P_n(x)}{x^{3n}} \mathrm{e}^{-1/x^2}$. D'où $f^{(n)}$\/ est de classe $\mathcal{C}^1$\/ et
					\begin{align*}
						f^{(n+1)}(x) &= \frac{\mathrm{d}}{\mathrm{d}x} f^{(n)}(x) \\
						&= \frac{\mathrm{d}}{\mathrm{d}x}\big[P_n(x) x^{-3n} \mathrm{e}^{- 1 / x^2}\big] \\
						&= \frac{\mathrm{d}}{\mathrm{d}x} \big[P_n(x) x^{-3n}\big] + P_n(x) x^{-3n} \times \frac{2}{x^3} \mathrm{e}^{-1/x^2} \\
						&= \Big( P'_n(x) x^{-3n} - 3n P_n(x) x^{-3n-1} + 2P_n(x) x^{-3n-3}\Big)\mathrm{e}^{- 1 / x^2} \\
						&= \frac{x^3 P'_n(x) - 3n x^2P_n(x) + 2P_n(x)}{x^{3n+3}} \mathrm{e}^{ - 1/x^2} \\
						&= \frac{P_{n+1}(x)}{x^{3(n+1)}} \mathrm{e}^{- 1 / x^2} \\
					\end{align*}
			\end{itemize}
		\item Par récurrence :
			\begin{itemize}
				\item $f^{(0)}(0) = f(0) = 0$\/ par définition, et $f(x) = \mathrm{e}^{- 1 / x^2} \tendsto{x\to 0} 0 = f(0)$, donc $f$\/ est $\mathcal{C}^0$.
				\item On suppose $f$\/ $\mathcal{C}^n$\/ sur $\R$\/ et $f^{(n)}(0) = 0$. On sait que, $f^{(n)}$\/ est continue sur $\R$\/ par hypothèse, $f^{(n)}$\/ est dérivable sur $\R^*$\/ car $f$\/ est $\mathcal{C}^\infty$\/ d'après la question 1. Et,
					\begin{align*}
						\frac{\mathrm{d}}{\mathrm{d}x} f^{(n)}(x) &= f^{(n+1)}(x) \\
						&= \frac{P_{n+1}(x)}{x^{3n+3}} \mathrm{e}^{ - 1/x^2} \\
						&\tendsto{x\to 0} 0 \text{ par croissances comparées}.
					\end{align*}
			\end{itemize}
		\item $f$\/ est $\mathcal{C}^\infty$\/ mais pas développable en série entière. Par l'absurde, si $f$\/ est développable en série entière, alors $f(x) = \sum_{n=0}^\infty a_n x^n$\/ et $a_n = \frac{f^{(n)}(0)}{n!} = 0$, d'où $\forall x$, $f(x) = 0$, ce qui est absurde.
	\end{enumerate}
\end{exo}

\begin{rap}[Théorème de la limite de la dérivée]
	Si $f$\/ est continue sur $[a,b]$, dérivable sur $]a,b[$ et
	\begin{enumerate}
		\item $\lim_{x\to a^+} f'(x) = \ell \in \R$, alors $f$\/ est dérivable en $a$\/ et $f'(a) = \ell$.
		\item $\lim_{x\to a^+} f'(x) = \pm \infty$, alors $f$\/ n'est pas dérivable en $a$, et la courbe possède une tangente verticale.
		\item $\lim_{x\to a^+} f'(x)$\/ n'existe pas, alors on ne peut pas conclure.
	\end{enumerate}
\end{rap}

\begin{prop}
	Une fonction $f$\/ est développable en série entière sur $]-r,r[$\/ si, et seulement si
	\begin{enumerate}
		\item $f$\/ est $\mathcal{C}^\infty$\/ sur $]-r,r[$\/
		\item $\forall x \in {]-r,r[}$, $R_N(x) \tendsto{N\to \infty} 0$\/ où $R_N(x) = f(x) - \sum_{n=0}^N \frac{f^{(n)}(0)}{n!} x^n$.
	\end{enumerate}
\end{prop}

\noindent
En effet, on a \[
	f(x) = \sum_{n=0}^N \frac{f^{(n)}(0)}{n!} x^n + R_N(x),
\] ainsi $\sum_{n=0}^N f^{n}(0) / n! \tendsto{N\to \infty} f(x) \iff R_N(x)\tendsto{N\to \infty} 0$.

\begin{rap}[Formules de \textsc{Taylor}]
	\ul{Formule de \textsc{Taylor-Young}} : si $f$\/ est de classe $\mathcal{C}^n$, alors
	\begin{align*}
		f(x) &= f(0) + \po(1)\\
		&= f(0) + x\: f'(0) + \po(x)\\
		&= f(0) + x\: f'(0) + \frac{x^2}{2!}\:f''(0) + \po(x^2)\\
		&= f(0) + x\: f'(0) + \frac{x^2}{2!}\:f''(0) + \cdots + \frac{x^n}{n!}\:f^{(n)}(0) + \po(x^n)
	\end{align*}
	\ul{Formule de \textsc{Taylor} avec reste intégral\footnote{aussi appelé formule de \textsc{Taylor-Laplace}}} : si $f$\/ est de classe $\mathcal{C}^{n+1}$, alors
	\begin{align*}
		f(x) &= f(0) + \int_{0}^{x} f'(t)~\mathrm{d}t \\
		&= f(0) + x\:f'(0) + \int_{0}^{x} (x-t)\: f''(t)~\mathrm{d}t \\
		&= f(0) + x\:f'(0) + \frac{x^2}{2!}\: f''(0) + \int_{0}^{x} \frac{(x-t)^2}{2!}\: f'''(t)~\mathrm{d}t \\
		&= f(0) + x\:f'(0) + \cdots + \frac{x^n}{n!}\: f^{(n)}(0) + \int_{0}^{x} \frac{(x-t)^n}{n!}\: f^{(n+1)}(t)~\mathrm{d}t \\
	\end{align*}
\end{rap}

\begin{exo}
	\begin{enumerate}
		\item Montrons que, pour $x \in \R$, $\cos x = \sum_{n=0}^\infty (-1)^n \frac{x^{2n}}{(2n)!}$.
			Pour $n \in \N$, $R_n(x) = \int_{0}^{x} \frac{(x-t)^n}{n!}\: \cos^{(n+1)}(t)~\mathrm{d}t$\/ car $\cos$\/ est $\mathcal{C}^{n+1}$.
			Soit $x \in \R$. Montrons que $R_n(x) \tendsto{n\to \infty} 0$.
			\begin{align*}
				0 \le |R_n(x)| &= \bigg| \int_{0}^{x} \frac{(x-t)^n}{n!} \cos^{(n+1)}(t)~\mathrm{d}t \bigg|\\
				&\le \bigg|\int_{0}^{x} \left| \frac{(x-t)^n}{n!}\: \cos ^{(n+1)}(t) \right|~\mathrm{d}t\bigg|\\
				&\le \bigg|\int_{0}^{x} \frac{|x-t|^n}{n!}~\mathrm{d}t\bigg|
			\end{align*}
			car $|\cos^{(n+1)}(t)| \le 1$, $\forall t \in [0,x]$.
			D'où, \[
				0 \le |R_n(x)| \le \bigg| \int_{0}^{x} \frac{x^n}{n!}~\mathrm{d}t \bigg| = \frac{|x|^{n+1}}{n!} \tendsto{n\to \infty} 0
			,\] car on sait que la série $\sum \frac{x^n}{n!}$\/ converge, donc le terme général tend donc vers 0.
			Donc $\cos$\/ est développable en série entière sur $\R$.
		\item De même pour $\sin$.
	\end{enumerate}
\end{exo}
