\section*{Annexe I : Bilan de la khôlle n\tsup o2}

\slshape
On considère la matrice \[
	A = \begin{pmatrix}
		2&0&\sfrac{2}{3}\\
		1&x_0&x_0^2\\
		1&-x_0&x_0^2
	\end{pmatrix}
.\]
Préciser si la matrice $A$\/ est inversible, si oui l'inverser, sinon déterminer son noyau.
\upshape

On calcule le déterminant de $A$\/ :
\marginpar{Méthode 1}
\begin{align*}
	\det A = 
	\begin{vmatrix}
		2&0&\sfrac{2}{3}\\
		1&x_0&x_0^2\\
		1&-x_0&x_0^2
	\end{vmatrix} &= 
	\begin{vmatrix}
		2&0&\sfrac{2}{3}\\
		1&x_0&x_0^2\\
		2&0&2x_0^2
	\end{vmatrix} \text{ avec } L_3 \leftarrow L_3 + L_2\\
	&= x_0
	\begin{vmatrix}
		2&\sfrac{2}{3}\\
		2&2x_0^2
	\end{vmatrix}\\
	&= 4x_0\left( x_0^2 - \frac{1}{3} \right) \\
	&= 4x_0 \left( x-\frac{1}{\sqrt{3}} \right)\left( x-\frac{1}{\sqrt{3}} \right). \\
\end{align*}
On remarque qu'il y a 4 cas : si $x = 0$, si $x = \frac{1}{\sqrt{3}}$, si $x = -\frac{1}{\sqrt{3}}$, et sinon.

On peut aussi ne pas utiliser le déterminant : soit $X= \left( \substack{x\\y\\z} \right)\in \mathscr{M}_{3,1}(\R)$.
\marginpar{Méthode 2}
\begin{align*}
	X \in \Ker A \iff AX = 0 \iff& \begin{cases}
		2x + \frac{2}{3} z = 0\\
		x + x_0y + x_0^2 z = 0\\
		x - x_0y + x_0^2 z = 0
	\end{cases}\\
	\iff& \begin{cases}
			x = -\frac{1}{3} z\\
			x_0 y = 0\\
			x + x_0^2 z = 0
	\end{cases} \text{ avec } L_2 \leftarrow \frac{L_1 - L_2}{2} \text{ et } L_3 \leftarrow \frac{L_1 + L_3}{2}\\
	\iff& \begin{cases}
		x = -\frac{1}{3} z\\
		x_0 y = 0\\
		\left( x_0^2 - \frac{1}{3} \right) z = 0
	\end{cases}
\end{align*}
Et, on distingue alors les cas si $x_0 = 0$, si $x = \frac{1}{\sqrt{3}}$, si $x = -\frac{1}{\sqrt{3}}$, et sinon.

