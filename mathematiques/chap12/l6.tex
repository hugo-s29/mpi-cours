\section{Rotations et réflexions}

\begin{defn}
	Soit $E$\/ un $\R$-espace vectoriel de dimension finie.
	On dit que deux bases $\mathcal{B}_0$\/ et $\mathcal{B}$\/ de $E$\/ ont la même \textit{orientation} si le déterminant de la matrice $P$\/ de passage de $\mathcal{B}_0$\/ à $\mathcal{B}$\/ est positif : \[
		\det P = \det_{\mathcal{B}_0}(\mathcal{B}) > 0
	.\]
\end{defn}

\begin{figure}[H]
	\centering
	\begin{asy}
		size(13cm);
		draw((-5, 0)--(-2, 0), Arrow(TeXHead));
		draw((-4, -0.5)--(-4, 1.5), Arrow(TeXHead));
		draw((0, 0)--(3, 0), Arrow(TeXHead));
		draw((1, -0.5)--(1, 1.5), Arrow(TeXHead));
		draw((5, 0)--(8, 0), Arrow(TeXHead));
		draw((6, -0.5)--(6, 1.5), Arrow(TeXHead));
		pair u = (1.5, 1);
		pair v = (0.5, 0.8);
		draw((-4,0)--(-4,0)+u--(-4,0)+u+v--(-4,0)+v--cycle);
		draw((-4, 0)--(-4, 0) + u, deepcyan, Arrow(TeXHead));
		label("$\vec u$", (-4,0) + u/2, deepcyan, align=W);
		draw((-4, 0)--(-4, 0) + v, deepcyan, Arrow(TeXHead));
		label("$\vec v$", (-4,0) + v/2, deepcyan, align=S);
		pair w = (u.x*v.y - v.x*u.y, 0);
		draw((1, 0)--(1,0)+w--(1,0)+v+w--(1,0)+v--cycle);
		draw((6,0)--(6,0)+w--(6,v.y)+w--(6,v.y)--cycle);
	\end{asy}
	\caption{Aire d'un parallélogramme}
\end{figure}

\begin{exm}
	On oriente le plan $\R^2$\/ en décidant que la base canonique $\mathcal{B}_0 = (\vec{\imath}, \vec{\jmath})$\/ est directe, puis on munit $\R^2$\/ du produit canonique : $\mathcal{B}_0$\/ est alors une base orthonormée directe de $\R^2$. Soient deux vecteurs $\vec{u} = a\,\vec{i} + b\,\vec{\jmath}$\/ et $\vec{v} = c\, \vec{\imath} + b\, \vec{\jmath}$.
	Le déterminant \[
		\det(\vec{u}, \vec{v}) = \begin{vmatrix}
			a & c\\
			b & d
		\end{vmatrix} = ad - bc
	\] possède 
	\begin{itemize}
		\item  un signe (s'il est strictement positif, alors $(\vec{u}, \vec{v})$\/ est une base directe, s'il est strictement négatif, alors $(\vec{u}, \vec{v})$\/ est une base indirecte, s'il est nul alors $(\vec{u}, \vec{v})$\/ n'est pas une base car $\vec{u}$\/ et $\vec{v}$\/ sont liés) ;
		\item une valeur absolue, égale à l'aire du parallélogramme construit sur les vecteurs $\vec{u}$\/ et $\vec{v}$. Les trois parallélogrammes de la figure précédente ont la même aire : \[
				\big|{\det (\vec{u}, \vec{v})}\big| = |ad - bc|
		.\]
	\end{itemize}
\end{exm}

\begin{exm}
	On oriente l'espace $\R^3$\/ en décidant que la base canonique $\mathcal{B}_0 = (\vec{\imath}, \vec{\jmath}, \vec{k})$\/ est directe.
\end{exm}

\begin{defn}
	Soit $E$\/ un espace euclidien de dimension $n$. Soit $\mathcal{B}$\/ une base orthonormée de $E$. On dit que
	\begin{itemize}
		\item $f$\/ est une \textit{rotation} si $f$\/ est une isométrie de déterminant $+1$, autrement dit, si~$\big[\:f\:\big]_\mathcal{B} \in \mathrm{SO}_n(\R)$\/ ;
		\item $f$\/ est une \textit{réflexion} si $f$\/ est une symétrie orthogonale par rapport à un hyperplan de $E$.
	\end{itemize}
\end{defn}

\begin{rmk}
	\begin{enumerate}
		\item D'après la définition et les théorèmes précédents, on a les équivalences :
			\begin{align*}
				f \text{ est une rotation } \iff& f \text{ est une isométrie et } \det f = +1 \\
				\iff& f \in \mathrm{O}(E) \text{ et } \det f = +1 \\
				\iff& f \text{ conserve } \begin{cases}
					\text{ le produit scalaire}\\
					\text{ et l'orientation}
				\end{cases} \\
					\iff& f \text{ transforme une base orthonormée directe} \\[-2mm]
						&\quad\quad\text{en une base orthonormée directe}.
			\end{align*}
		\item Les rotations de $E$\/ forment un groupe, noté $\mathrm{SO}(E)$\/ ; c'est un sous-groupe de $\mathrm{O}(E)$\/ des isométries de $E$.
		\item Si $f$\/ est une réflexion par rapport à un hyperplan $H$\/ de $E$, alors sa matrice dans une base adaptée à la somme directe $H \oplus H^\perp$\/ s'écrit \[
			\begin{pmatrix}
				1 & 0 & \ldots & 0\\
				0 & \ddots & \ddots & \vdots\\
				\vdots & \ddots & 1 & 0\\
				0 & \ldots & 0 & -1
			\end{pmatrix} 
		.\] Ainsi, le déterminant d'une réflexion est donc toujours égal à $-1$. Et, la composée de deux réflexions est donc une rotation.
	\end{enumerate}
\end{rmk}

\begin{thm}
	Une matrice appartient à $\mathrm{O}_2(\R)$\/ si, et seulement si elle est de la forme \[
		R_\theta = \begin{pmatrix}
			\cos \theta & -\sin \theta\\
			\sin \theta & \cos \theta
		\end{pmatrix} \quad\quad\text{ ou }\quad\quad S_\theta = \begin{pmatrix}
			\cos \theta & \sin \theta\\
			\sin \theta & -\cos \theta
		\end{pmatrix}
	.\]
	Son déterminant vaut $+1$\/ dans le premier cas, $-1$\/ dans le second.
\end{thm}

La première matrice représente une rotation d'angle $\theta$\/ ; la seconde représente une symétrie par rapport à la droite d'angle $\theta / 2$.

\begin{prv}
	\begin{itemize}
		\item[``$\implies$''] Les colonnes des deux matrices forment une base de $\R^2$. D'où, $R_\theta \in \mathrm{O}_2(\R)$\/ et $S_\theta  \in \mathrm{O}_2(\R)$.
		\item[```$\impliedby$'']
			La matrice $A$\/ est dans le groupe $\mathrm{O}_2(\R)$, d'où \[
				\begin{cases}
					a^2 + b^2 = 1 \quad\quad\quad&(1)\\
					c^2 + d^2 = 1 \quad\quad\quad&(2)\\
					ac + bd = 0. \quad\quad\quad& (3)
				\end{cases}
			\] Il existe donc $\theta \in \R$\/ tel que $(a,b) = (\cos \theta, \sin \theta)$, d'après $(1)$.
			De même, il existe $\varphi \in \R$, tel que $(c,d) = (\cos \varphi, \sin \varphi)$, d'après $(2)$.
			\begin{align*}
				(3) & \text{ d'où } ab + cd = 0\\
						& \text{ d'où } \cos\theta\cos\varphi + \sin\theta\sin\varphi = 0\\
						& \text{ d'où } \cos (\theta - \varphi) = 0\\
						& \text{ d'où } \theta - \varphi \equiv \frac{\pi}{2}\mod \pi\\
						& \text{ d'où } \begin{cases}
							\exists k \in \Z,\: \theta - \varphi =2 k \pi + \frac{\pi}{2}\\
							\text{ ou bien }\\
							\exists k \in \Z,\: \theta- \varphi = 2k\pi - \frac{\pi}{2}\\
						\end{cases}\\
						& \text{ d'où } \begin{cases}
							\exists k \in \Z,\: \theta = \varphi + \frac{\pi}{2} + 2k\pi\\
							\text{ ou bien }\\
							\exists k \in \Z,\: \theta = \varphi - \frac{\pi}{2} + 2k\pi.\\
						\end{cases}\\
						& \text{ d'où } \begin{cases}
							A = R_\theta\\
							\text{ ou bien }\\
							A = S_\theta
						\end{cases}
			\end{align*}
	\end{itemize}
\end{prv}

\begin{crlr}
	Les isométries du plan sont les rotations (autour de l'origine) et les réflexions (par rapport à une droite passant par l'origine).
\end{crlr}

\begin{rmk}
	Pour tout $(\theta, \varphi) \in \R^2$, $R_\theta \cdot R_\varphi = R_{\theta + \varphi} = R_\varphi \cdot R_\theta$.
	Le groupe $\mathrm{SO}_2(\R)$\/ est donc commutatif.
	De plus, l'application \begin{align*}
		\Phi: \R &\longrightarrow \mathrm{SO}_2(\R) \\
		\theta &\longmapsto R_\theta\\
		\varphi &\longmapsto R_\varphi\\
		\theta + \varphi &\longmapsto R_{\theta + \varphi}
	\end{align*}
	est un morphisme du groupe $(\R, +)$\/ vers le groupe $(\mathrm{SO}_2(\R),\:\cdot\:)$.
	Ce morphisme est surjectif, mais pas injectif (car son noyau est $2\pi\:\Z$) : $\Phi(0) = \Phi(2\pi) = I_2$.

	De plus, au lieu de repérer un point du plan par ses coordonnées $(x,y) \in \R^2$, on peut le repérer par son affixe $z = x + iy \in \C$.
	\begin{enumerate}[label=(\alph*)]
		\item Après une rotation, la position du point $M'$\/ sera repérée par les coordonnées \[
					\begin{pmatrix}
						x'\\
						y'
					\end{pmatrix} = S_\theta \cdot \begin{pmatrix}
						x\\ y
					\end{pmatrix} \quad\quad \iff \quad\quad z' = \mathrm{e}^{i\theta} \cdot z
			.\] En effet, $z' = \mathrm{e}^{i\theta} \cdot z \iff x' + iy' = (\cos \theta + i \sin \theta)(x+iy) = (x \cos \theta - y \sin \theta) + i (x\sin \theta + y \cos \theta)$, et \[
				\begin{pmatrix}
					x'\\y'
				\end{pmatrix} = R_\theta \cdot \begin{pmatrix}
					x\\ y
				\end{pmatrix} = \begin{pmatrix}
					\cos \theta & -\sin \theta\\
					\sin \theta & \cos \theta
				\end{pmatrix} \cdot \begin{pmatrix}
					x\\y
				\end{pmatrix} = \begin{pmatrix}
					x \cos \theta - y \sin \theta\\
					x \sin \theta + y \cos \theta
				\end{pmatrix}
			.\]
			Ainsi, l'application \begin{align*}
				\Psi: \mathds{U} &\longrightarrow \mathrm{SO}_2(\R) \\
				\mathrm{e}^{i\theta} &\longmapsto R_\theta
			\end{align*} est un isomorphisme de groupes (\textit{i.e.}\ un morphisme de groupe bijectif).
		\item Après une symétrie, la position du point $M'$\/ sera repérée par les coordonnées \[
					\begin{pmatrix}
						x'\\
						y'
					\end{pmatrix} = R_\theta \cdot \begin{pmatrix}
						x\\ y
					\end{pmatrix} \quad\quad \iff \quad\quad z' = \mathrm{e}^{i\theta} \cdot \bar{z}
			.\] En effet, $z' = \mathrm{e}^{i\theta} \cdot \bar{z} \iff x' + iy' = (\cos \theta + i \sin \theta)(x-iy) = (x \cos \theta + y \sin \theta) + i (x\sin \theta - y \cos \theta)$, et \[
				\begin{pmatrix}
					x'\\y'
				\end{pmatrix} = S_\theta \cdot \begin{pmatrix}
					x\\ y
				\end{pmatrix} = \begin{pmatrix}
					\cos \theta & \sin \theta\\
					\sin \theta & -\cos \theta
				\end{pmatrix} \cdot \begin{pmatrix}
					x\\y
				\end{pmatrix} = \begin{pmatrix}
					x \cos \theta + y \sin \theta\\
					x \sin \theta - y \cos \theta
				\end{pmatrix}
			.\]
			D'où, \[
				S_\theta = \underbrace{\begin{pmatrix}
						\cos \theta & -\sin \theta\\
						\sin \theta & \cos \theta
					\end{pmatrix}}_{R_\theta} \cdot 
					\underbrace{\begin{pmatrix}
							1 & 0\\
							0 & -1
					\end{pmatrix}}_{S_0}
			.\]
	\end{enumerate}
\end{rmk}

\begin{thm}
	Une application $f$\/ est une isométrie de l'espace si, et seulement s'il existe un angle $\theta \in \R$\/ et une base orthonormée directe $\mathcal{B} = (\vec{u}, \vec{v}, \vec{w})$ tels que
	\begin{gather*}
		\big[\:f\:\big]_\mathcal{B} =
		\begin{pNiceMatrix}[last-row,last-col]
			\cos \theta & -\sin \theta & 0 & \vec{u}\\
			\sin \theta & \cos \theta & 0 & \vec{v}\\
			0 & 0 & 1 & \vec{w}\\
			f(\vec{u}) & f(\vec{v}) & f(\vec{w})
		\end{pNiceMatrix}\\
		\text{ ou }\\
		\big[\:f\:\big]_\mathcal{B} =
		\begin{pNiceMatrix}[last-row,last-col]
			\cos \theta & -\sin \theta & 0 & \vec{u}\\
			\sin \theta & \cos \theta & 0 & \vec{v}\\
			0 & 0 & -1 & \vec{w}\\
			f(\vec{u}) & f(\vec{v}) & f(\vec{w})
		\end{pNiceMatrix}.
	\end{gather*}
\end{thm}

