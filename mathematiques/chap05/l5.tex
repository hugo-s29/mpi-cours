L'interpolation n'est pas toujours une bonne approximation.

\begin{exm}[phénomène de Runge]
	La fonction \begin{align*}
		f: \R &\longrightarrow \R \\
		x &\longmapsto \frac{1}{1+x^2}
	\end{align*}
	peut être approché par un polynôme à l'aide des polynômes interpolateur de {\sc Lagrange}\/ : on choisit plusieurs points sur la courbe de $f$, et on interpole entre ces points.
\end{exm}

\begin{figure}[H]
	\centering
	\begin{asy}
		import graph;
		draw((0,-1) -- (0, 5), Arrow(TeXHead));
		draw((-5,0) -- (5, 0), Arrow(TeXHead));
		size(8cm);
		real f(real x) { return 4 / (1 + x * x); }
		draw(graph(f, -5, 5), magenta);
		/*
		pair pt(int i, int m) {

		}
		real a(real x) {
			real p = 1;
			for(int i = 0; i < 10; ++i) {
				pair x = pt(i, 10);
				p *= (
			}
			return p;
		}
		*/
	\end{asy}
	\caption{Graphe de $f : x\mapsto \frac{1}{1+x^2}$}
	\todo{Interpolation de {\sc Lagrange}}
\end{figure}

\begin{thm}[Weierstra\ss]
	Si une fonction $f : [a,b] \to \R$\/ continue sur le segment $[a,b]$, alors il existe une suite de polynômes $P_n$\/ qui converge uniformément vers $f$\/ telle que \[
		\sup_{[a,b]}\:|f - P_n| \tendsto{n\to \infty} 0
	.\]
\end{thm}

\begin{thm}[théorème des moments]
	Soit $f$\/ une fonction continue sur un segment $[a,b]$\/ vers $\R$.
	Montrer que, si \[\forall n \in \N,\:\int_{a}^{b} x^n f(x)~\mathrm{d}x = 0,\] alors la fonction $f$\/ est nulle.
	Soit $P \in \R[X]$, il existe donc $a_0$, $a_1$, \ldots, $a_n \in \R^{n+1}$\/ tels que $P(X) = a_0 + a_1 X + \cdots + a_n X^n$. Alors, \[
		\int_{a}^{b} P(x)\cdot f(x)~\mathrm{d}x = \int_{a}^{b} \Big(\sum_{k=0}^n a_k x^k \Big) \cdot f(x)~\mathrm{d}x = \sum_{k=1}^n a_k \underbrace{\int_{a}^{b} x^k f(x)~\mathrm{d}x}_{0} = 0
	.\] Or, d'après le théorème de {\sc Weierstra\ss}, il existe une suite de polynômes $(P_n)_{n\in\N}$\/ qui convergent uniformément vers $f$. D'où, \[
		\forall n \in \N,\: \int_{a}^{b} P_n(x)\cdot f(x)~\mathrm{d}x
	.\] On va montrer que \[
		\int_{a}^{b} P_n(x) \cdot f(x)~\mathrm{d}x \tendsto{n\to \infty} \int_{a}^{b} f(x)\cdot f(x)~\mathrm{d}x.
	\] Or, d'après le théorème de {\sc Weierstra\ss}, la suite de polynômes $(P_n)_{n\in\N}$\/ converge uniformément vers $f$. D'où,
	\[
		\Big|P_n(x) f(x) - f(x) \cdot f(x)\Big|
		= \underbrace{\Big|P_n(x) - f(x)\Big|}_{\le \sup_{[a,b]}\:|f - P_n|} \times \underbrace{\Big|f(x)\Big|}_{\le M} \\
	\] 
	où $M$\/ est le majorant qui existe car toute fonction continue sur un segment est bornée. D'où, \[
		\Big|P_n(x) f(x) - f(x) \cdot f(x)\Big| \le M \times \sup_{[a,b]}\:|f - P_n|
	.\]
	D'où \[
		0 = \lim_{n\to \infty}\int_{a}^{b} P_n(x)\cdot f(x)~\mathrm{d}x = \int_{a}^{b} \lim_{n\to \infty} P_n(x)\cdot f(x)~\mathrm{d}x
	.\] On en déduit donc que \[
		\int_{a}^{b} f^2(x)~\mathrm{d}x = 0
	.\] Or, si l'intégrale d'une fonction positive {\color{red}et continue} est nulle, alors la fonction est nulle. Donc \[
		\forall x \in [a,b],\quad f(x) = 0
	.\]
\end{thm}

