\documentclass[a4paper]{article}

\usepackage[margin=1in]{geometry}
\usepackage[utf8]{inputenc}
\usepackage[T1]{fontenc}
\usepackage{mathrsfs}
\usepackage{textcomp}
\usepackage[french]{babel}
\usepackage{amsmath}
\usepackage{amssymb}
\usepackage{cancel}
\usepackage{frcursive}
\usepackage[inline]{asymptote}
\usepackage{tikz}
\usepackage[european,straightvoltages,europeanresistors]{circuitikz}
\usepackage{tikz-cd}
\usepackage{tkz-tab}
\usepackage[b]{esvect}
\usepackage[framemethod=TikZ]{mdframed}
\usepackage{centernot}
\usepackage{diagbox}
\usepackage{dsfont}
\usepackage{fancyhdr}
\usepackage{float}
\usepackage{graphicx}
\usepackage{listings}
\usepackage{multicol}
\usepackage{nicematrix}
\usepackage{pdflscape}
\usepackage{stmaryrd}
\usepackage{xfrac}
\usepackage{hep-math-font}
\usepackage{amsthm}
\usepackage{thmtools}
\usepackage{indentfirst}
\usepackage[framemethod=TikZ]{mdframed}
\usepackage{accents}
\usepackage{soulutf8}
\usepackage{mathtools}
\usepackage{bodegraph}
\usepackage{slashbox}
\usepackage{enumitem}
\usepackage{calligra}
\usepackage{cinzel}
\usepackage{BOONDOX-calo}

% Tikz
\usetikzlibrary{babel}
\usetikzlibrary{positioning}
\usetikzlibrary{calc}

% global settings
\frenchspacing
\reversemarginpar
\setuldepth{a}

%\everymath{\displaystyle}

\frenchbsetup{StandardLists=true}

\def\asydir{asy}

%\sisetup{exponent-product=\cdot,output-decimal-marker={,},separate-uncertainty,range-phrase=\;à\;,locale=FR}

\setlength{\parskip}{1em}

\theoremstyle{definition}

% Changing math
\let\emptyset\varnothing
\let\ge\geqslant
\let\le\leqslant
\let\preceq\preccurlyeq
\let\succeq\succcurlyeq
\let\ds\displaystyle
\let\ts\textstyle

\newcommand{\C}{\mathds{C}}
\newcommand{\R}{\mathds{R}}
\newcommand{\Z}{\mathds{Z}}
\newcommand{\N}{\mathds{N}}
\newcommand{\Q}{\mathds{Q}}

\renewcommand{\O}{\emptyset}

\newcommand\ubar[1]{\underaccent{\bar}{#1}}

\renewcommand\Re{\expandafter\mathfrak{Re}}
\renewcommand\Im{\expandafter\mathfrak{Im}}

\let\slantedpartial\partial
\DeclareRobustCommand{\partial}{\text{\rotatebox[origin=t]{20}{\scalebox{0.95}[1]{$\slantedpartial$}}}\hspace{-1pt}}

% merging two maths characters w/ \charfusion
\makeatletter
\def\moverlay{\mathpalette\mov@rlay}
\def\mov@rlay#1#2{\leavevmode\vtop{%
   \baselineskip\z@skip \lineskiplimit-\maxdimen
   \ialign{\hfil$\m@th#1##$\hfil\cr#2\crcr}}}
\newcommand{\charfusion}[3][\mathord]{
    #1{\ifx#1\mathop\vphantom{#2}\fi
        \mathpalette\mov@rlay{#2\cr#3}
      }
    \ifx#1\mathop\expandafter\displaylimits\fi}
\makeatother

% custom math commands
\newcommand{\T}{{\!\!\,\top}}
\newcommand{\avrt}[1]{\rotatebox{-90}{$#1$}}
\newcommand{\bigcupdot}{\charfusion[\mathop]{\bigcup}{\cdot}}
\newcommand{\cupdot}{\charfusion[\mathbin]{\cup}{\cdot}}
%\newcommand{\danger}{{\large\fontencoding{U}\fontfamily{futs}\selectfont\char 66\relax}\;}
\newcommand{\tendsto}[1]{\xrightarrow[#1]{}}
\newcommand{\vrt}[1]{\rotatebox{90}{$#1$}}
\newcommand{\tsup}[1]{\textsuperscript{\underline{#1}}}
\newcommand{\tsub}[1]{\textsubscript{#1}}

\renewcommand{\mod}[1]{~\left[ #1 \right]}
\renewcommand{\t}{{}^t\!}
\newcommand{\s}{\text{\calligra s}}

% custom units / constants
%\DeclareSIUnit{\litre}{\ell}
\let\hbar\hslash

% header / footer
\pagestyle{fancy}
\fancyhead{} \fancyfoot{}
\fancyfoot[C]{\thepage}

% fonts
\let\sc\scshape
\let\bf\bfseries
\let\it\itshape
\let\sl\slshape

% custom math operators
\let\th\relax
\let\det\relax
\DeclareMathOperator*{\codim}{codim}
\DeclareMathOperator*{\dom}{dom}
\DeclareMathOperator*{\gO}{O}
\DeclareMathOperator*{\po}{\text{\cursive o}}
\DeclareMathOperator*{\sgn}{sgn}
\DeclareMathOperator*{\simi}{\sim}
\DeclareMathOperator{\Arccos}{Arccos}
\DeclareMathOperator{\Arcsin}{Arcsin}
\DeclareMathOperator{\Arctan}{Arctan}
\DeclareMathOperator{\Argsh}{Argsh}
\DeclareMathOperator{\Arg}{Arg}
\DeclareMathOperator{\Aut}{Aut}
\DeclareMathOperator{\Card}{Card}
\DeclareMathOperator{\Cl}{\mathcal{C}\!\ell}
\DeclareMathOperator{\Cov}{Cov}
\DeclareMathOperator{\Ker}{Ker}
\DeclareMathOperator{\Mat}{Mat}
\DeclareMathOperator{\PGCD}{PGCD}
\DeclareMathOperator{\PPCM}{PPCM}
\DeclareMathOperator{\Supp}{Supp}
\DeclareMathOperator{\Vect}{Vect}
\DeclareMathOperator{\argmax}{argmax}
\DeclareMathOperator{\argmin}{argmin}
\DeclareMathOperator{\ch}{ch}
\DeclareMathOperator{\com}{com}
\DeclareMathOperator{\cotan}{cotan}
\DeclareMathOperator{\det}{det}
\DeclareMathOperator{\id}{id}
\DeclareMathOperator{\rg}{rg}
\DeclareMathOperator{\rk}{rk}
\DeclareMathOperator{\sh}{sh}
\DeclareMathOperator{\th}{th}
\DeclareMathOperator{\tr}{tr}

% colors and page style
\definecolor{truewhite}{HTML}{ffffff}
\definecolor{white}{HTML}{faf4ed}
\definecolor{trueblack}{HTML}{000000}
\definecolor{black}{HTML}{575279}
\definecolor{mauve}{HTML}{907aa9}
\definecolor{blue}{HTML}{286983}
\definecolor{red}{HTML}{d7827e}
\definecolor{yellow}{HTML}{ea9d34}
\definecolor{gray}{HTML}{9893a5}
\definecolor{grey}{HTML}{9893a5}
\definecolor{green}{HTML}{a0d971}

\pagecolor{white}
\color{black}

\begin{asydef}
	settings.prc = false;
	settings.render=0;

	white = rgb("faf4ed");
	black = rgb("575279");
	blue = rgb("286983");
	red = rgb("d7827e");
	yellow = rgb("f6c177");
	orange = rgb("ea9d34");
	gray = rgb("9893a5");
	grey = rgb("9893a5");
	deepcyan = rgb("56949f");
	pink = rgb("b4637a");
	magenta = rgb("eb6f92");
	green = rgb("a0d971");
	purple = rgb("907aa9");

	defaultpen(black + fontsize(8pt));

	import three;
	currentlight = nolight;
\end{asydef}

% theorems, proofs, ...

\mdfsetup{skipabove=1em,skipbelow=1em, innertopmargin=6pt, innerbottommargin=6pt,}

\declaretheoremstyle[
	headfont=\normalfont\itshape,
	numbered=no,
	postheadspace=\newline,
	headpunct={:},
	qed=\qedsymbol]{demstyle}

\declaretheorem[style=demstyle, name=Démonstration]{dem}

\newcommand\veczero{\kern-1.2pt\vec{\kern1.2pt 0}} % \vec{0} looks weird since the `0' isn't italicized

\makeatletter
\renewcommand{\title}[2]{
	\AtBeginDocument{
		\begin{titlepage}
			\begin{center}
				\vspace{10cm}
				{\Large \sc Chapitre #1}\\
				\vspace{1cm}
				{\Huge \calligra #2}\\
				\vfill
				Hugo {\sc Salou} MPI${}^{\star}$\\
				{\small Dernière mise à jour le \@date }
			\end{center}
		\end{titlepage}
	}
}

\newcommand{\titletp}[4]{
	\AtBeginDocument{
		\begin{titlepage}
			\begin{center}
				\vspace{10cm}
				{\Large \sc tp #1}\\
				\vspace{1cm}
				{\Huge \textsc{\textit{#2}}}\\
				\vfill
				{#3}\textit{MPI}${}^{\star}$\\
			\end{center}
		\end{titlepage}
	}
	\fancyfoot{}\fancyhead{}
	\fancyfoot[R]{#4 \textit{MPI}${}^{\star}$}
	\fancyhead[C]{{\sc tp #1} : #2}
	\fancyhead[R]{\thepage}
}

\newcommand{\titletd}[2]{
	\AtBeginDocument{
		\begin{titlepage}
			\begin{center}
				\vspace{10cm}
				{\Large \sc td #1}\\
				\vspace{1cm}
				{\Huge \calligra #2}\\
				\vfill
				Hugo {\sc Salou} MPI${}^{\star}$\\
				{\small Dernière mise à jour le \@date }
			\end{center}
		\end{titlepage}
	}
}
\makeatother

\newcommand{\sign}{
	\null
	\vfill
	\begin{center}
		{
			\fontfamily{ccr}\selectfont
			\textit{\textbf{\.{\"i}}}
		}
	\end{center}
	\vfill
	\null
}

\renewcommand{\thefootnote}{\emph{\alph{footnote}}}

% figure support
\usepackage{import}
\usepackage{xifthen}
\pdfminorversion=7
\usepackage{pdfpages}
\usepackage{transparent}
\newcommand{\incfig}[1]{%
	\def\svgwidth{\columnwidth}
	\import{./figures/}{#1.pdf_tex}
}

\pdfsuppresswarningpagegroup=1
\ctikzset{tripoles/european not symbol=circle}

\newcommand{\missingpart}{{\large\color{red} Il manque quelque chose ici\ldots}}

\usepackage{pgfornament}
\usepackage{slantsc}

\begin{document}
	\clearpage
	\centerline{\LARGE Cadeaux du 14/09/22}
	\bigskip
	\bigskip
	{\bf Cadeau 1} :\\
	\slshape
	Soit $A$\/ un anneau commutatif et soit $x \in A$. On dit que $x$\/ est {\it nilpotent}\/ (ou {\it nihilpotent}) si \[
		\exists n \in \N,\:x^n = 0_A
	.\]
	\begin{enumerate}
		\item Montrer que, si $x$\/ est nilpotent, alors $x$\/ n'est pas inversible mais $1_A - x$\/ est inversible.
		\item Montrer que l'ensemble des éléments nilpotents de $A$\/ est un idéal de $A$.
	\end{enumerate}
	\upshape
	\bigskip
	\bigskip

	{\bf Réponse du cadeau 1} :
	\begin{enumerate}
		\item On procède par l'absurde. On suppose que $x$\/ est nilpotent. Soit $n \in \N$\/ le plus petit possible tel que $x^n = 0_A$. On suppose qu'il existe $y \in A$\/ tel que $x\cdot y = 1_A$. D'où, $(xy)^n$\/ est, d'une part $x^n \cdot y^n = 0_A \cdot y^n = 0_A$\/ par commutativité, et d'autre part, $(xy)^n = {1_A}^{n} = 1_A \neq 0_A$. Ce qui est absurde.

			On suppose à présent $x \neq 1_A$. On sait que $A \ni \sum_{k=0}^{n-1} x^k = (1-x^n) / (1-x) = 1 / (1-x)$. On a donc trouvé l'inverse de $1-x$.
		\item Soit $x$\/ un élément nilpotent de $A$, et $y$\/ un élément de $A$. Soit $n \in \N$\/ tel que $x^n = 0_A$. $x\cdot y$\/ est aussi un élément nilpotent de $A$. En effet, $(xy)^n = x^n \cdot y^n = 0_A \cdot y^n = 0_A$. On nomme $\mathcal{I}$\/ l'ensemble des éléments nilpotents de $A$. Montrons que $(\mathcal{I},+)$\/ est un sous-groupe additif de $(A, +)$. On a bien $0 \in \mathcal{I}$\/ car $0^{k} = 0$.
			Soient $x$\/ et $y$\/ deux éléments nilpotents. Montrons que $x -y  \in \mathcal{I}$. Soient $n_1$\/ et $n_2 \in \N^*$\/ tels que $x^{n_1} = 0$\/ et $y^{n_2} = 0$. On veut montrer qu'il existe $n \in \N^*$\/ tel que $(x-y)^n = 0$. Soit $n = n_1 + n_2$. On a \[
				(x-y)^n = \sum_{k=0}^n (-1)^{n-k}{n \choose k} x^k y^{n-k}
				= \underbrace{\sum_{k=0}^{n_1} (-1)^{n-k}{n\choose k} x^y y^{n-k}}_{(1)} + \underbrace{\sum_{k=n_1 + 1}^{n_1 + n_2} (-1)^{n-k}{n\choose k} x^y y^{n-k}}_{(2)}
			.\] Or, dans la somme (1), $n-k = n_1 + n_2 - k = n_2 + (n_1 - k) \ge n_2$\/ et, dans la somme (2), $k \ge n_1$.
	\end{enumerate}

	\bigskip
	\bigskip
	
	\centerline{\pgfornament[width=3cm]{88}}

	\bigskip
	\bigskip
	
	{\bf Cadeau 2} :\\
	\slshape
	Soit $F$\/ l'ensemble des matrices de la forme ${x\hfil y\choose -5y\quad x+4y}$\/ où $(x, y) \in \R^2$. On note $J = {-2\quad1\choose -5\quad2}$.
	\begin{enumerate}
		\item Montrer que $F$\/ est un sous-espace vectoriel de $\mathscr{M}_{2,2}(\R)$\/ et que $(I_2, J)$\/ est une base de $F$.
		\item Calculer $J^2$\/ puis $(x\:I_2 + y J)\cdot (x'\:I + y'\:J)$\/ pour tout $(x,y,x',y') \in \R^4$. Qu'en déduire ?
	\end{enumerate}

	\upshape
	\bigskip
	\bigskip

	{\bf Réponse du cadeau 2} :
	\begin{enumerate}
		\item On cherche à trouver $\alpha, \beta \in \R^2$\/ tels que ${x\hfil y\choose -5\quad x+4y} = \alpha I_2 + \beta J$. On a
			\begin{align*}
				\alpha I_2 + \beta J \iff& \begin{pmatrix}
					x &y\\
					-5y&x+4y
				\end{pmatrix} = \begin{pmatrix}
					\alpha & 0\\
					0&\alpha
				\end{pmatrix} + \begin{pmatrix}
					-2\beta & \beta\\
					-5\beta& 2\beta
				\end{pmatrix}\\
				\iff&
				\begin{cases}
					x = \alpha - 2\beta\\
					y = \beta\\
					-5\beta = -5y\\
					2\beta + \alpha = x+ y
				\end{cases}\\
				\iff& \begin{cases}
					\beta = y\\
					\alpha = x + 2y\\
				\end{cases}
			\end{align*}
		\item On a $J^2 = -I_2$\/ et, en calculant minutieusement, on trouve, pour tout $(x,y,x',y') \in \R^4$, $(x\,I_2+ y\,J)\cdot (x'\,I_2 + y'\,J) = \cdots = (xx' - yy')\,I_2 + (x'y + xy')\,J$.
			On remarque que $(F, +, \cdot)$\/ est isomorphe à $(\C, +, \times)$. C'est un isomorphisme d'anneaux. Or, comme l'anneau $(\C, +, \times)$\/ est un corps donc $F$\/ l'est aussi.
	\end{enumerate}
	\clearpage
	\centerline{\LARGE Cadeau du 19/09/22}
	\bigskip
	\bigskip
	{\bf Cadeau} :\\
	\slshape
	Soit $(\vec\imath, \vec\jmath, \vec k)$\/ une base orthonormée de $\R^3$. On pose $f: \R^3 \to \R^3$\/ un endomorphisme défini tel que \[
		A = \begin{bNiceMatrix}[last-row,last-col]
			0&0&1&\vec\imath\\
			1&0&0&\vec\jmath\\
			0&1&0&\vec k\\
			f(\vec\imath)&f(\vec\jmath)&f(\vec k)\\
		\end{bNiceMatrix} = \big[f\big]_{(\vec\imath,\vec\jmath,\vec k)}
	.\]
	Interpréter géométriquement $f$.

	\upshape
	\bigskip
	\bigskip

	{\bf Réponse du cadeau} :\\
	Soit $\mathscr{B}$\/ une base et $A = [f]_\mathscr{B}$, alors $f(\vec\imath) = \vec\jmath$, $f(\vec\jmath) = \vec k$, et $f(\vec k) = \vec\imath$.
	$f$\/ est la rotation d'angle $\sfrac{2\pi}3$\/ autour de $\Vect(\vec \imath + \vec \jmath + \vec k)$.

	On peut également le montrer en décomposant $f = g  \circ h$, où $g$\/ est la symétrie par rapport à $\Vect(\vec{\imath}+\vec{\jmath},\vec{k})$\/ et parallèlement à $\Vect(\vec{\imath},\vec{\jmath})$\/ ; et $h$\/ la symétrie par rapport à $\Vect(\vec{\imath} + \vec{k},\vec{\jmath})$\/ parallèlement à $\Vect(\vec{\imath},\vec{k})$.

	\clearpage
	\centerline{\LARGE Cadeaux du 22/09/22}
	\bigskip
	\bigskip
	{\bf Cadeau 1} :\\
	\slshape
	Soit $(u_n)_{n\in\N}$\/ une suite positive, telle que la suite $S_n = \sum_{k=0}^n u_k$\/ diverge. En calculant $\ln \frac{S_n}{S_{n-1}}$, montrer que la série $\sum \frac{u_n}{S_n}$\/ diverge.
	\upshape

	\bigskip
	\bigskip
	
	\centerline{\pgfornament[width=3cm]{88}}

	\bigskip
	\bigskip
	
	{\bf Cadeau 2} :\\
	\slshape
	On pose \[
		D(x) = 
		\begin{vmatrix}
			7-x&14-x&3-x\\
			8-x&2-x&-x\\
			13-x&-1-x&2-x
		\end{vmatrix}
	.\]
	Montrer qu'il existe deux réels $\alpha$\/ et $\beta$\/ tels que, pour tout $x \in \R$, $D(x) = \alpha x + \beta$. Déterminer $\alpha$\/ et $\beta$.
	\clearpage
	\centerline{\LARGE Cadeau du 23/09/22}
	\bigskip
	\bigskip
	{\bf Cadeau} :\\
	\slshape
	Montrer qu'il n'existe pas $P \in \mathrm{GL}_2(\R)$\/ telle que, $A' = P^{-1}AP$\/ où \[
		A = \begin{bmatrix} 0&7\\0&0 \end{bmatrix}\qquad\text{et}\qquad
		A' = \begin{bmatrix} \lambda&0\\0&\mu \end{bmatrix}
	.\]
	\upshape
	\bigskip
	\bigskip

	{\bf Réponse du cadeau} :\\
	\begin{itemize}
		\item[{\sc Analyse}] On suppose $P^{-1} A P = {\lambda\:0\choose 0\:\mu}$. Alors, $\det A = \det A'$, d'où $0 = \lambda\cdot\mu$.
			Et, $\tr A = \tr A'$, d'où $\lambda + \mu = 0$.
			On en déduit donc que $\lambda = 0 = \mu$.
		\item[{\sc Synthèse}] On a \[
				P^{-1} A P = \begin{bmatrix} 0&0\\0&0 \end{bmatrix}
			.\] D'où, en multipliant à gauche par $P$\/ et à droite par $P^{-1}$, on a \[
				A = \begin{bmatrix} 0&0\\0&0 \end{bmatrix} 
			.\]
	\end{itemize}
	On en conclut que la matrice $A$\/ n'est pas diagonalisable.
	\clearpage
	\centerline{\LARGE Cadeaux du 28/09/22}
	\bigskip
	\bigskip
	{\bf Cadeau 1} :\\
	\slshape
	On considère la matrice \[
		M =
		\begin{pNiceMatrix}[last-row,last-col]
			0&0&1&\vec{\imath}\\
			1&0&0&\vec{\jmath}\\
			0&1&0&\vec{k}\\
			f(\vec{\imath})&f(\vec{\jmath})&f(\vec{k})
		\end{pNiceMatrix} = \Big[\:f\:\Big]_{(\vec{\imath},\vec{\jmath},\vec{k})}
	.\]
	Trouver et interpréter un vecteur propre et une valeur propre de $M$\/ (et, de même, de $f$).\\
	\upshape
	\bigskip
	\bigskip

	{\bfseries Indication :}\\
	On a $M \left( \substack{1\\1\\1} \right) = 1 \times \left( \substack{1\\1\\1} \right)$. Le vecteur $\vec{\imath}+\vec{\jmath}+\vec{k}$\/ est un vecteur directeur de l'axe de rotation. Montrer que les seuls vecteurs propres sont colinéaire au vecteur $\left( \substack{1\\1\\1} \right)$.
	\clearpage
	\centerline{\LARGE Cadeaux du 06/10/22}
	\bigskip
	\bigskip
	{\bf Cadeau} :\\
	\slshape
	On considère la matrice \[
		A = \begin{pmatrix}
			0&1&1\\
			-1&1&1\\
			-1&1&2
		\end{pmatrix}
	.\]
	\begin{enumerate}
		\item Quel est le spectre de la matrice $A$\/ ?
		\item Déterminer une base de chaque sous-espace propre de la matrice $A$.
		\item Montrer que la matrice $A$\/ est trigonalisable mais pas diagonalisable.
		\item Soit $T$\/ la matrice ci-dessous. Déterminer une matrice $P$\/ telle que $P^{-1} \cdot A \cdot P = T$ : \[
				T = \begin{pmatrix}
					1&1&0\\
					0&1&1\\
					0&0&1
				\end{pmatrix}
			\]
		\item Résoudre sur $\R$\/ le système d'équation différentielle $(\Sigma)$\/ ci-dessous : \[
				(\Sigma) : \begin{cases}
					x'(t) = y(t) + z(t)\\
					y'(t) = -x(t) + y(t) + z(t)\\
					z'(t) = -x(t) + y(t) + 2z(t)
				\end{cases}
			\]
	\end{enumerate}
	\upshape
	\bigskip
	\bigskip
	{\bf Réponse au cadeau 1:}\\
	\begin{enumerate}
		\item On calcule
			\begin{align*}
				\chi_A(x) = \det(xI_3 - A) &=
				\begin{vmatrix}
					x & -1 & -1\\
					1 & x - 1 & -1\\
					1 & -1 & x - 2
				\end{vmatrix}\\
				&= x
				\begin{vmatrix}
					x-1&-1\\
					-1&x-2
				\end{vmatrix} -
				\begin{vmatrix}
					-1&-1\\
					-1&x-2
				\end{vmatrix} +
				\begin{vmatrix}
					-1&-1\\
					x-1&-1
				\end{vmatrix}\\
				&= x\big((x-1)(x-2) - 1\big) - (2-x) - 1  + (-1 + x - 1) \\
				&= x(x^2 -3x + 1) -2 + 2x \\
				&= x^3 - 3x^2 + 3x - 1 \\
				&= (x-1)^3 \\
			\end{align*}
			On en déduit que \[
				\boxed{\Sp(A) = \{1\}.}
			\]
		\item On cherche une base de $\mathrm{SEP}(1)$\/ : on cherche $X = \left( \substack{x\\y\\z} \right) \in \mathscr{M}_{3,1}(\R)$, tel que $AX = X$.
			\begin{align*}
				\begin{pmatrix}
					0&1&1\\
					-1&1&1\\
					-1&1&2
				\end{pmatrix} \begin{pmatrix}
					x\\y\\z
				\end{pmatrix} = \begin{pmatrix}
					x\\y\\z
				\end{pmatrix} \iff& \begin{cases}
					y + z = x\\
					-x + y + z = y\\
					-x + y + 2z = z
				\end{cases}\\
				\iff& \begin{cases}
					y = 0\\
					x = z\\
				\end{cases}\\
				\iff& X = x \begin{pmatrix}
					1\\
					0\\
					1
				\end{pmatrix}\\
				\iff& X \in \Vect\begin{pmatrix}
					1\\0\\1
				\end{pmatrix}
			\end{align*}
			Ainsi, la base $\mathscr{B} = \left( \left( \substack{1\\0\\1} \right)\right)$.
		\item La matrice $A$\/ n'est pas diagonalisable : en effet, on a $\dim \R^3 = 3 \neq \dim(\mathrm{SEP}(1)) = 1$. Mais, le polynôme $\chi_A$\/ est scindé donc la matrice $A$\/ est trigonalisable.
		\item
			On cherche $P \in \mathrm{GL}_n(\R)$\/ tel que \[
				P^{-1} \cdot A \cdot P = T = 
				\begin{pNiceMatrix}[last-row,last-col]
					1&1&0&\varepsilon_1\\
					0&1&1&\varepsilon_2\\
					0&0&1&\varepsilon_3\\
					f(\varepsilon_1)&f(\varepsilon_2)&f(\varepsilon_3)
				\end{pNiceMatrix}
			.\]
			On cherche donc $(\varepsilon_1, \varepsilon_2, \varepsilon_3)$\/ une base de $\R^3$\/ tel que \[
				\begin{cases}
					f(\varepsilon_1) = \varepsilon_1 \qquad\qquad&(1)\\
					f(\varepsilon_2) = \varepsilon_1+\varepsilon_2 \qquad\qquad&(2)\\
					f(\varepsilon_3) = \varepsilon_2 + \varepsilon_3 \qquad\qquad&(3).
				\end{cases}
			\]
			On choisit $\varepsilon_1 = \left( \substack{1\\0\\1} \right)$, d'après la question 2.
			Puis, on calcule
			\begin{align*}
				\underbrace{\begin{pmatrix}
					0&1&1\\
					-1&1&1\\
					-1&1&2
				\end{pmatrix}}_{\smash{A}} \begin{pmatrix}
					x\\y\\z
				\end{pmatrix} = \begin{pmatrix}
					1\\0\\1
				\end{pmatrix} + \begin{pmatrix}
					x\\y\\z
				\end{pmatrix} \iff& \begin{cases}
					-x + y + z = 1\\
					-x+z=0\\
					-x + y +z = 1
				\end{cases}\\
				\iff& \begin{cases}
					-x+y+z=1\\
					-x+z = 0
				\end{cases}\\
				\iff& \begin{cases}
					x = z\\
					y = 1
				\end{cases}\\
				\iff& \varepsilon_2 = \begin{pmatrix}
					x\\1\\x
				\end{pmatrix} = \begin{pmatrix}
					0&1&0
				\end{pmatrix} + x \begin{pmatrix}
					1\\0\\1
				\end{pmatrix}
			\end{align*}
			On choisit donc $\varepsilon_2 = \left( \substack{0\\1\\0} \right)$\/ (en choisissant $x = 0$). De même, on choisit $\varepsilon_3 = \left( \substack{0\\-1\\1} \right)$.
			Donc, si \[
				P =
				\begin{pNiceMatrix}[last-row,last-col]
					1&0&0&e_1\\
					0&1&-1&e_2\\
					1&0&1&e_3\\
					\varepsilon_1&\varepsilon_2&\varepsilon_3
				\end{pNiceMatrix}
			\] alors $P^{-1} \cdot A \cdot P$, et on a bien $\det P = 1$.
		\item
			\begin{align*}
				(\Sigma) \iff& X'(t) = A\cdot X(t) \text{ où } X(t) \begin{pmatrix}
					x(t)\\y(t)\\z(t)
				\end{pmatrix}\\
				\iff& U'(t) = T\cdot U(t) \text{ où } U(t) = \begin{pmatrix}
						u(t)\\v(t)\\w(t)
					\end{pmatrix} = P^{-1}\cdot X(t)\\
				\iff& \begin{cases}
					u'(t) = u(t) + v(t)&\qquad(1)\\
					v'(t) = v(t) + w(t)&\qquad(2)\\
					w'(t) = w(t) &\qquad(3)
				\end{cases}
			\end{align*}
			D'où
			\[
				(3) \iff \exists K \in \R,\:\forall t \in \R,\;w(t) = K\mathrm{e}^{t},
			\] et \[
				(2) \iff v'(t) - v(t) = K \mathrm{e}^{t}
			.\]
			On résout l'équation homogène associé : \[
				v'(t) = v(t) \iff \exists L \in \R,\:\forall t \in \R,\:v(t) = L\mathrm{e}^{t}
			.\]
			On pose $v(t) = \ell(t)\:\mathrm{e}^{t}$, et donc
			\begin{align*}
				(2) \iff& \ell'(t)\:\mathrm{e}^{t} + \ell(t)\:\mathrm{e}^{t} - \ell(t)\:\mathrm{e}^{t} = K\mathrm{e}^t\\
				\iff& \ell'(t) = K\\
				\iff& \ell(t) = K\,t + L\\
				\iff& v(t) = (K t + L)\: \mathrm{e}^{t}
			\end{align*}
			Et donc
			\begin{align*}
				(1) \iff& u'(t) = u(t) + v(t) = u(t) + (Kt + L)\:\mathrm{e}^{t}\\
				\iff& u(t) - u(t) = (Kt + L)\:\mathrm{e}^{t}\\
			\end{align*}
			On résout l'équation homogène associée : \[
				u'(t) - u(t) = 0 \iff \exists M \in \R,\:\forall t \in \R\:u(t) = M \mathrm{e}^{t}
			.\]
			On pose $u(t) = m(t)\:\mathrm{e}^{t}$.
			\begin{align*}
				(2) \iff& m'(t)\:\mathrm{e}^{t} - m(t) = (Kt + L)\:\mathrm{e}^{t}\\
				\iff& m'(t) = K t + L\\
				\iff& m(t) = \frac{1}{2} Kt^2 + Lt + M\\
				\iff& u(t) = \left( \frac{1}{2} Kt^2 + L t + M \right)\:\mathrm{e}^{t}
			\end{align*}
			Ainsi, \[
				(\Sigma) \iff \exists (K,L,M) \in \R^3,\:\forall t \in \R,\: \begin{cases}
					u(t) = \left( \frac{1}{2} Kt^2 + Lt + M \right)\:\mathrm{e}^{t}\\
					v(t) = (Kt + L)\:\mathrm{e}^{t}\\
					w(t) = K\:\mathrm{e}^{t}.
				\end{cases}
			\] Or, $X(t) = P \cdot U(t)$, et donc
			\begin{align*}
				(2) \iff& \exists (K,L,M) \in \R^3,\:\forall t \in \R,\: \begin{cases}
					x(t) = u(t)\\
					y(t) = v(t) - w(t)\\
					z(t) = u(t) + w(t)
				\end{cases}\\
				\iff& \exists (K,L,M) \in \R^3,\:\forall t \in \R,\: \begin{cases}
					x(t) = \left( \frac{1}{2} K t^2 + LT + M \right) \mathrm{e}^{t}\\
					y(t) = \left( Kt + L - K \right)\mathrm{e}^{t}\\
					z(t) = \left( \frac{1}{2}Kt^2 + Lt + M + K \right)\mathrm{e}^{t}
				\end{cases}
			\end{align*}
	\end{enumerate}
	\bigskip
	\bigskip
	{\bf Cadeau 2} (matrices stochastiques) :\\
	\slshape
	Soit une matrice carrée $A = (a_{i,j}) \in \mathscr{M}_n(\R)$\/ telle que \[
		\forall i,j \in \left\llbracket 1,n \right\rrbracket,\:a_{i,j} \ge 0\qquad\text{ et }\qquad \forall i \in \left\llbracket 1,n \right\rrbracket,\:\sum_{j=1}^n a_{i,j} = 1
	.\]
	\begin{enumerate}
		\item Montrer que $1 \in \Sp(A)$.
		\item Montrer que, si $\lambda$\/ est une valeur propre de $A$, alors $|\lambda| \le 1$.
	\end{enumerate}
	\upshape

	\bigskip
	\bigskip
	{\bf Cadeau 3} (matrices à diagonale strictement dominante) :\\
	\slshape
	Soit $A = (a_{i,j}) \in \mathscr{M}_{n}(\R)$\/ telle que \[
		\forall  i \in \left\llbracket 1,n \right\rrbracket,\:|a_{i,j}| > \sum_{j \neq i} |a_{i,j}|
	.\] Montrer que $A$\/ est inversible.
	\upshape
	\clearpage
	\centerline{\LARGE Cadeau du 12/10/22}
	\bigskip
	\bigskip
	{\bf Cadeau} :\\
	\slshape
	Soit $A \in \mathscr{M}_2(\R)$ tell que $A^2 = -I_2$. Montrer qu'il existe $P$\/ une matrice inversible telle que \[
		P^{-1}\cdot A\cdot P = \begin{pmatrix}
			0&-1\\
			1&0
		\end{pmatrix} = M
	.\]
	\upshape
	\bigskip
	\bigskip

	{\bf Réponse au cadeau}\/\\
	\begin{comment}
		Le polynôme $Q(X) = X^2 + 1 = (X-i)(X+i)$\/ est annulateur de la matrice~$A$. D'où~$\Sp_\C (A) \subset \{-i,i\}$. Le polynôme~$Q$\/ est scindé à racines simples dans~$\C[X]$, d'où la matrice est diagonalisable dans~$\mathscr{M}_2(\C)$. Or, le spectre est stable par conjugaison car~$A \in \mathscr{M}_2(\R)$, d'où~$\Sp_\C(A) = \{-i,i\}$. De même, on remarque~$M^2 = I_2$, et donc tout ce qui est dit précédemment est aussi vrai pour cette matrice.
		Ainsi, il existe une matrice inversible $P \in \mathrm{GL}_2(\C)$\/ telle que \[
			P^{-1}\cdot A\cdot P = \begin{pmatrix}
				i&0\\
				0&-i
			\end{pmatrix}\qquad\text{et}\qquad \begin{pmatrix}
				0&-1\\
				1&0
			\end{pmatrix} \sim \begin{pmatrix}
				i&0\\
				0&-i
			\end{pmatrix}
		.\]
		D'où $A$\/ et ${0\hfill\:-1\choose1\hfill\:0}$\/ sont semblables dans $\mathscr{M}_2(\C)$.
	\end{comment}

	On remarque que $A \sim {i\hfill\:0\choose0\hfill\:-i}$. D'où, il existe $\varepsilon \in \mathscr{M}_{2,1}(\C)$, tel que $A\cdot \varepsilon = i \varepsilon$. Également, $A \cdot \bar\varepsilon = -i \bar\varepsilon$. Soient $U = \varepsilon + \bar\varepsilon \in \mathscr{M}_{2,1}(\R)$, et $V = i(\varepsilon + \bar\varepsilon) \in \mathscr{M}_{2,1}(\R)$, puis on calcule
	\begin{align*}
		A \cdot U &= A\cdot \varepsilon + A\cdot \bar\varepsilon \\
		&= i \varepsilon - i\bar\varepsilon \\
		&= i(\varepsilon - \bar\varepsilon) \\
		&= V.
	\end{align*}
	\clearpage
	\centerline{\LARGE Cadeau du 19/10/22}
	\bigskip
	\bigskip
	{\bf Cadeau} :\\
	\slshape
	Soit $f$\/ une fonction continue sur $[0, +\infty[$, telle que $\ds\int_{0}^{+\infty} f(t)~\mathrm{d}t$\/ converge. 
	\begin{enumerate}
		\item Montrer que ça n'implique pas que $f(x) \tendsto{x\to +\infty} 0$.
		\item Montrer que, si $f(x) \tendsto{x\to +\infty} \ell \in \R$, alors $\ell = 0$.
		\item Montrer que, si $f$\/ est uniformément continue, alors $f(x) \tendsto{x\to +\infty} 0$.
	\end{enumerate}
	\upshape
	\bigskip
	\bigskip

	{\bf Réponse du cadeau} :
	\begin{enumerate}
		\item c.f.\ remarque 7 du cours
		\item Quitte à remplacer $\ell$\/ par $-\ell$, on suppose $\ell > 0$. Ainsi, il existe $X \ge 0$\/ tel que \[\forall x \ge X,\: f(x) \ge \frac{\ell}{2}.\] Or, l'intégrale \[\int_{X}^{+\infty} \frac{\ell}{2}~\mathrm{d}x\] diverge. D'où \[\int_{X}^{+\infty} f(x)~\mathrm{d}x\] diverge également. Ce qui est absurde.
			On en déduit que $\ell = 0$.
		\item On suppose $f$\/ uniformément continue. Par l'absurde, supposons que $f(x) \centernot{\tendsto{x\to +\infty}} 0$, d'où \[
				\exists \varepsilon > 0,\:\exists (u_n)_{n \in \N} \text{ tendant vers } +\infty,\:\forall n \in \N,\:f(u_n) \ge \varepsilon
			.\] Or, comme $f$\/ est uniformément continue, il existe $\delta > 0$\/ tel que \[
				\forall n \in \N,\:\forall x \in \R^+,\quad |x-u_n| \le \delta \implies |f(x) - f(y)| \le \frac{\varepsilon}{2}
			.\] D'où, \[
			\int_{0}^{+\infty} f(x)~\mathrm{d}x \ge \int_{0}^{+\infty} \frac{\varepsilon}{2}~\mathrm{d}t
			\] qui diverge. Ce qui est absurde.
	\end{enumerate}
	\clearpage
	\centerline{\LARGE Cadeau du 21/10/22}

	\centerline{\textit{Théorème d'interpolation de Lagrange}}
	\bigskip
	\bigskip
	{\bf Cadeau} :\\
	\slshape
	Soient $(a_0, a_1, \ldots, a_n)$ une suite de $n+1$\/ réels distincts deux à deux.
	Soient aussi $(b_0, b_1, \ldots, b_n)$\/ une suite de $n+1$\/ réels (qui peuvent être égales).
	Alors, \[
		\exists ! P \in \R_n[X],\:\forall k \in \left\llbracket 0,n \right\rrbracket,\:P(a_k) = b_k
	.\]
	\upshape
	\bigskip
	\bigskip

	{\bf Réponse du cadeau} :\\
	\begin{itemize}
		\item[\sc Méthode 1] Soient $n+1$\/ réels $a_0$, $a_1$, \ldots, $a_n$\/ distincts deux à deux. L'application \begin{align*}
				f: \R_n[X] &\longrightarrow \R^{n+1} \\
				P &\longmapsto \Big(P(a_0), P(a_1), \ldots, P(a_n)\Big)
			\end{align*}
			est linéaire et la dimension de l'espace vectoriel de départ est égale à la dimension de l'espace vectoriel d'arrivée. Soit $P$\/ un polynôme réel de degré au plus $n$.
			\begin{align*}
				P \in \Ker f \iff& f(P) = (0, 0, \ldots, 0)\\
				\implies& P(a_1) = P(a_2) = \cdots = P(a_n) = 0\\
				\implies& P \text{ a au moins } n + 1 \text{ racines}\\
				\implies& P = 0_{\R_n[X]} \text{ car } \# \text{racines} > \deg(P)
			\end{align*}
			D'où $\Ker f = \{0_{\R_n[X]}\}$.
			On en déduit que $f$\/ est injective. Et, d'après le théorème du rang, $f$\/ est surjective (car $\dim \R_n[X] = \dim \R^{n+1}$).
		\item[\sc Méthode 2] On reprend la fonction $f$\/ de la {\sc méthode 1}.
			Soit $\mathscr{B}$\/ la base canonique de $\R_n[X]$\/ : $\mathscr{B} = (1, X, \ldots, X^n)$\/ ; et, soit $\mathscr{C}$\/ la base canonique de $\R^{n+1}$\/ : $\mathscr{C} = (e_1, e_2, \ldots, e_{n+1})$.
			On a \[
				\big[f\big]_\mathscr{B}^{\mathscr{C}} = \Mat_{\mathscr{B},\mathscr{C}}(f) =
				\begin{pNiceMatrix}[last-row,last-col]
					1 & a_0 & a_0^n & \cdots & a_0^n & e_1\\
					1 & a_1 & a_1^n & \cdots & a_1^n & e_2\\
					1 & a_2 & a_2^n & \cdots & a_2^n & e_3\\
					\vdots & \vdots & \vdots & \ddots & \vdots & \vdots\\
					1 & a_n & a_n^2 & \cdots & a_n^n & e_{n+1}\\
					f(1) & f(X) & f(X^2) & \ldots & f(X^n)
				\end{pNiceMatrix} = V
			.\]
			On reconnaît un déterminant de {\sc Vandermonde}\/ :
			\begin{align*}
				\det V &= (a_n - a_{n-1}) \cdots (a_n - a_1)(a_n - a_0)\\
							 &\times (a_{n-1} - a_{n-2}) \cdots (a_{n-1} - a_{n-2}) \cdots (a_{n-1} - a_0)\\
							 &\times\\
							 &\:\vdots\\
							 &\times (a_2 - a_1) \cdot (a_2 - a_0)\\
							 &\times (a_1 - a_0)\\
							 &= \prod_{i>j} (a_i - a_j)\\
			\end{align*}
			D'où $\det V \neq 0$\/ car les $(a_i)$\/ sont distincts deux à deux. Donc $V$\/ est inversible, et d'où $f$\/ est bijective.
		\item[\sc Méthode 3]
			On va prouver la surjectivité en déterminant {\bf un}\/ polynôme $P$\/ tel que $P(a_0) = b_0$, $P(a_1) = b_1$, \ldots, $P(a_n) = b_n$. Le voilà :
			\begin{align*}
				P &= b_0 \frac{(X-a_1)(X-a_2) \cdots (X- a_n)}{(a_0 - a_1)(a_0 - a_2) \cdots (a_0 - a_n)}\\
				&+ b_1 \frac{(X-a_0)(X-a_1) \cdots (X-a_n)}{(a_1 - a_0)(a_1 - a_2) \cdots (a_1 - a_n)} \\
				&\:\vdots \\
				&+ b_n \frac{(X-a_0)(X-a_1)\cdots(X-a_{n-1})}{(a_n - a_0)(a_n - a_1) \cdots (a_n - a_{n-1})}. \\
			\end{align*}
			Ce polynôme interpole les $n+1$\/ points et $\deg P \le n$.
	\end{itemize}
	\clearpage
	\centerline{\LARGE Cadeau du 08/11/22}

	\centerline{\textit{Dénombrement}}
	\bigskip
	\bigskip
	\textbf{Cadeau} :\\
	\textsl{
	\begin{enumerate}
		\item Calculer, de deux manières, $\sum_{k=0}^n {n\choose k}$. {\color{gray}(binôme et dénombrement)}
		\item (Petite formule de \textsc{Pascal}) Calculer $p {n \choose p} = n {n-1\choose p-1}$.
		\item (Formule de \textsc{Vandermonde}) Montrer, de deux manières, que $\sum_{k=0}^n {a\choose p}{b \choose n-p} = {a+b\choose n}$. {\color{gray}(développement de $(1+x)^{a+b}$\/ et dénombrement)}
		\item Montrer, de deux manières, que $\sum_{k=p}^n {k\choose p} = {n+1\choose p+1}$\/ {\color{gray}(télescope et dénombrement)}
	\end{enumerate}
	}
	\bigskip
	\bigskip

	\textbf{Réponse du cadeau} :
	\begin{enumerate}
		\item On développe, à l'aide du binôme de \textsc{Newton}, $2^n = (1+1)^n = \sum_{k=0}^n {n \choose k}\cdot 1^k\cdot 1^{n-k} = \sum_{k=0}^n {n\choose k}$.

			Autre méthode : $\sum_{k=0}^n {n\choose k}$\/ est le nombre de parties d'un ensemble à $n$\/ éléments. Construire une partie à $n$\/ éléments, c'est : choisir ou non le premier élément (2 manières), choisir ou non le second élément (2 manière), \ldots, etc jusqu'au $n$-ième élément. Il y a donc $2^n$\/ manières.
		\item Soit $E$\/ un ensemble à $n$\/ éléments. Soit $A$\/ une partie de $E$ à $p$\/ éléments, et soit $B$\/ sont complémentaire dans $E$. L'application \begin{align*}
				f: \{X \subset A  \mid |X| = a + b\} &\longrightarrow \bigcupdot_{k=0}^n \{X \subset A  \mid |X| = k\} \cup \{ X \subset B  \mid |X| = n - k\} \\
				X &\longmapsto (X \cap A, X \cap B)
			\end{align*}
			est bijective. (L'application $(X,Y) \mapsto X \cup Y$\/ est sa réciproque.)
			Ainsi, \[|\{X \subset A  \mid |X| = a + b\}| = \sum_{k=0}^n |\{X \subset A  \mid |X| = k\}| \cdot |\{X \subset A  \mid |X| = n - k\}|,\] d'où ${a+b\choose n} = \sum_{k=0}^n {a\choose k}{b \choose n - k}$.

			Moins rigoureusement, choisir $n$\/ éléments de $E$, c'est choisir $0$\/ éléments de $A$\/ et $n$\/ éléments de $B$ (il y en a ${a\choose 0}{b\choose n}$), ou $1$\/ élément de $A$\/ et $n-1$\/ éléments de $B$\/ (il y en a ${a\choose 1}{b\choose n-1}$), ou\ldots\@ On en conclut qu'il y a $\sum_{p=0}^n {a\choose p}{b\choose n-p}$\/ manières de choisir $n$\/ éléments dans $E$. Mais, c'est aussi ${a+b\choose n}$.

			Autre méthode : on a d'une part $(1+X)^{a+b} = \sum_{k=0}^{a+b} {a+b\choose k} X^k$. Et, d'autre part,
			\[
				(1+X)^{a+b} = (1+X)^a \times (1+X)^b = \Big(\sum_{p=0}^a {a\choose p} X^p \Big)\Big(\sum_{q = 0}^b {b\choose q} X^q \Big)
			.\] Or, deux polynômes sont égaux si et seulement si leurs coefficients sont égaux deux-à-deux. D'où \[
				{a+b\choose n} = \sum_{p+q=n} {a\choose p}{b\choose q} = \sum_{p=0}^n {a\choose p}{b\choose n- p}
			.\]
		\item On a \[
				\sum_{k=p}^n {k\choose p} = \sum_{k=p}^n \left({k+1\choose p+1} - {k\choose p+1}\right) = {n+1\choose p+1} - {p\choose p+1} = {n+1\choose p+1}
			.\]
			Autre méthode : dénombrement. Je choisis $p+1$\/ éléments parmi $n+1$. On pose $k+1$\/ le plus grand éléments de ces éléments choisis : $p + 1 \le k+1 \le n+1$, d'où $p \le k \le n$.
			Ainsi, on doit donc choisir encore $p$\/ autres éléments dans $\llbracket 1, k\rrbracket$\/ (car ces éléments sont plus petits que $n$).
	\end{enumerate}
	\clearpage
	\centerline{\LARGE Cadeau du 01/12/22}

	\bigskip
	\bigskip
	\textbf{Cadeau} :\\
	\textsl{Soit $A \in \mathcal{M}_n(\R)$. Montrer que $\Ker(A^\top \cdot A) = \Ker A$.}
	\bigskip
	\bigskip

	\textbf{Réponse du cadeau} :\\
	On procède par double-inclusion.
	\begin{itemize}
		\item[``$\supset$''] Soit $X \in \Ker A$. Alors, $A\cdot X = 0$. Ainsi, $A^\top \cdot A\cdot X = A^\top \cdot 0 = 0$. On a bien $\Ker (A^\top \cdot A) \supset A$.
		\item[``$\subset$''] Soit $X \in \Ker(A^\top \cdot A)$. Alors $A^\top \cdot A \cdot X = 0$, d'où $X^\top \cdot  (A^\top \cdot A \cdot X) = 0$. Ainsi, $(X^\top \cdot A^\top)\cdot (A\cdot X) = 0$, donc $(A\cdot X)^\top \cdot (A\cdot X) = 0$. Mais, avec le produit scalaire canonique sur $\mathcal{M}_{n,1}(\R)$, on a donc $\|A\cdot X\|^2 = 0$. Par le caractère défini du produit scalaire, on a bien $A\cdot X = 0$\/ et donc $X \in \Ker A$. D'où $\Ker(A^\top \cdot A) \subset \Ker A$.
	\end{itemize}
	\clearpage
	\centerline{\LARGE Cadeau du 02/12/22}
	\centerline{\textit{Lemme de \textsc{Borel}-\textsc{Cantelli}}}

	\bigskip
	\bigskip
	\textbf{Cadeau} :\\
	\textsl{Soit $(\Omega, \mathcal{A}, P)$\/ un espace probabilisé, et soit $(E_n)_{n \in \N}$\/ une suite d'événements. On pose $F = \bigcap_{n \in \N} \big(\bigcup_{k\ge n} E_k\big)$.
		\begin{enumerate}
			\item On suppose que la série $\sum P(E_n)$\/ converge. En encadrement judicieusement $P(F)$, montrer que $P(F) = 0$.
			\item On suppose à présent que les événements $E_n$\/ sont indépendants et que la série $\sum P(E_n)$\/ diverge.
				\begin{enumerate}
					\item Soit $n \in \N$. Montrer que, pour tout $N \ge n$, $\ln P\big( \bigcap_{p=n}^N \bar{E}_p \big) \le -\sum_{p=n}^N P(E_p)$.
					\item En déduire $\lim_{N\to \infty} P\big(\bigcap_{p=n}^N \bar{E}_p\big)$.
					\item Montrer que, pour tout $n \in \N$, $P\big(\bigcap_{p \ge n} \bar{E}_p\big) = 0$.
					\item Conclure que $P(F) = 1$.
				\end{enumerate}
	\end{enumerate}}
	\bigskip
	\bigskip

	\textbf{Réponse du cadeau} :\\
	\begin{enumerate}
		\item On pose, pour $k \in \N$, l'événement $A_n = \bigcup_{k \ge n} E_k$.
		On a $A_n = A_{n+1} \cup E_n \supset A_{n+1}$, la suite d'événements $(A_n)_{n \in \N}$\/ est décroissante (pour l'inclusion). Ainsi, par continuité décroissante, \[
			P\Big(\bigcap_{n\in \N} A_n\Big) = \lim_{n\to \infty} P(A_n)
		.\] Or, comme la série $\sum P(E_n)$\/ converge, on a, pour tout $n \in \N$, par $\sigma$-sous-additivité, \[
			0 \le P(A_n) = P\Big( \bigcup_{k \ge n} E_k \Big) \le \sum_{k \ge n} P(E_k),
		\] et il s'agit du reste de la série convergente $\sum P(E_n)$, qui tends vers 0. D'où, par le théorème des gendarmes, $P(A_n)$\/ tend vers 0 quand $n \to \infty$. On en déduit que \[
			\boxed{P(F) = 0.}
		\]
		\item
			\begin{enumerate}
				\item Comme les événements $E_p$\/ sont indépendants, les événements $\bar{E}_p$\/ le sont aussi. Ainsi, $P\big(\bigcap_{p=n}^N \bar{E}_p\big) = \prod_{p=n}^N P(\bar{E}_p)$, et donc
					\begin{align*}
						\ln P\Big(\bigcap_{p=n}^N \bar{E}_p\Big) &= \sum_{p=n}^N \ln P(\bar{E}_p)\\
						&= \sum_{p=n}^N \ln\big(1 - P(E_p)\big)\\
						&\le \sum_{p=n}^N \big(-P(E_p)\big) \text{ car } \ln (1+x) \le x \\
						&\le -\sum_{p=n}^N P(E_p).
					\end{align*}
			\end{enumerate}
			\begin{figure}[H]
				\centering
				\begin{asy}
					import graph;
					size(7cm);
					draw((-1,0)--(4,0), Arrow(TeXHead));
					draw((0,-3)--(0,5), Arrow(TeXHead));
					real f(real x) { return log(x); }
					bool3 checkf(real x) { return x > 0 && -3 <= f(x) && 5 >= f(x); }
					draw(graph(f, 0, 5, checkf), red);
					draw((-1, -2)--(5, 4), magenta, Arrows(TeXHead));
					dot("1", (1, 0), align=SE);
				\end{asy}
				\caption{Concavité de la fonction $\ln$\/ }
			\end{figure}
			\begin{enumerate}[start=2]
				\item Comme, pour tout $p \in \N$, $P(E_p)\ge 0$, donc la suite $(S_N)_{N \in \N} = \big(\sum_{p=n}^N P(E_p)\big)_{N \in \N}$\/ est croissante, elle admet donc une limite. De plus, par hypothèse, la série $\sum P(E_p)$\/ diverge donc $\lim_{N\to \infty} {-S_n} = -\infty$. Par composition des limites, $\lim_{N\to \infty} \ln P\big(\bigcap_{p=n}^N \bar{E}_p\big) = -\infty$. Enfin, comme l'exponentielle est croissante, \[
						0 \le P\Big(\bigcap_{p=n}^N \bar{E}_p\Big) \le \exp\Big(-\sum_{p=n}^N P(E_p) \Big)
					.\] Comme $\lim_{x\to -\infty} \mathrm{e}^{x} = 0$, on en déduit par le théorème des gendarmes, que \[
						\lim_{N\to \infty} P\Big(\bigcap_{p=n}^N \bar{E}_p\Big) = 0
					.\]
				\item Par continuité décroissante, on a donc \[
						P\Big(\bigcap_{n \ge p} \bar{E}_p\Big) = \lim_{N\to \infty} P\Big(\bigcap_{p=n}^N \bar{E}_p\Big) = 0
					.\]
				\item On rappelle que $F = \bigcap_{n \in \N} \bigcup_{p \ge n} E_p$. Or, on sait que $\bigcap_{p \ge n} \bar{E}_p = \overline{\bigcup_{p \ge n} E_p}$. On a doc $P\big(\bigcup_{p\ge N}\big) = 1 - P\big(\bigcap_{p \ge n} \bar{E}_p\big) = 1$. Do,c les événements $\bigcup_{p \ge n} E_p$\/ sont presque certains. Or, l'intersection d'événements presque certains est presque certaine. D'où $P(F) = 1$.
			\end{enumerate}
	\end{enumerate}
	\clearpage
	\centerline{\LARGE Cadeau du 05/12/22}
	\centerline{\textit{Oral CCP PSI 2018}}

	\bigskip
	\bigskip
	\textbf{Cadeau} :\\
	\textsl{Soit $(A_k)_{k \in \llbracket 1,n \rrbracket}$\/ une famille d'événements d'un espace probabilisé $(\Omega, \mathcal{A}, P)$. Montrer que \[
		\sum_{k=1}^n P(A_k) \le P\Big(\bigcap_{k=1} ^n A_k\Big) + n - 1
	.\]}
	\bigskip
	\bigskip

	\textbf{Réponse du cadeau} :\\
	On a
	\begin{align*}
		n - \sum_{k=1}^n P(\bar{A}_k) &= \sum_{k=1}^n \big(1 - P(A_k)\big)\\
		&= \sum_{k=1}^n P(\bar{A}_k) \\
		&\ge P\Big(\bigcup_{k=1}^n \bar{A}_k\Big) \\
		&= 1 - P\Big(\bigcap_{k=1}^n A_k\Big) \\
	\end{align*}
	D'où, \[
		\sum_{k=1}^n P(\bar{A}_k) \le P\Big(\bigcap_{k=1}^n A_k\Big) + n - 1
	.\]

	\clearpage
	\centerline{\LARGE Cadeau du 08/12/22}

	\bigskip
	\bigskip
	\textbf{Cadeau} :\\
	\textsl{La forme \begin{align*}
		\varphi: \R[X] \times \R[X] &\longrightarrow \R \\
		(P,Q) &\longmapsto \int_{0}^{\pi} P(\cos t) Q(\cos t)~\mathrm{d}t
	\end{align*}
	est un produit scalaire ?}

	\clearpage
	\centerline{\LARGE Cadeau du 13/12/22}

	\bigskip
	\bigskip
	\textbf{Cadeau} :\\
	\textsl{\begin{enumerate}
		\item Soit $n \in \N$. En remarquant que $(1+X)^n \cdot (1+X)^n = (1+X)^{2n}$, montrer que \[
				\sum_{k=0}^n {n\choose k}^2 = {2n\choose n}
			.\]
		\item Montrer que \[
				\Big(\sum_{n=0}^\infty  \frac{1}{(n!)^2}\Big)^2 = \sum_{n=0}^\infty \frac{1}{(2n)!} \: {2n\choose n}^2
			.\]
	\end{enumerate}}

	\bigskip
	\bigskip
	\textbf{Réponse du cadeau} :\\
	\begin{enumerate}
		\item Soit $n \in \N$. On sait que
			\begin{align*}
				(1+X)^n \cdot (1+X)^n &= \Big(\sum_{k=0}^n {n\choose k} X^k\Big)\Big(\sum_{k=0}^n {n\choose k} X^k\Big)\\
				&= \sum_{k=0}^{2n} \sum_{p+q=k} {n\choose p}{n\choose q} X^k \\
			\end{align*}
			D'autre part, on sait que \[
				(1+X)^{2n} = \sum_{k=0}^{2n} {2n\choose k} X^k
			.\] Or, deux polynômes sont égaux si, et seulement s'ils ont les mêmes coefficients. On regarde le coefficient $X^n$\/ en posant $k = n$\/ : \[
				{2n\choose n} = \sum_{p+q = n} {n\choose p}{n\choose q} = \sum_{p=0}^n {n\choose p}{n\choose n-p} = \sum_{k=0}^n {n\choose k}^2
			.\]
		\item La série $\sum \frac{1}{(n!)}$\/ converge absolument car $\frac{1}{(n!)^2} = \po\left( \frac{1}{n^2} \right)$. Ainsi, par produit de \textsc{Cauchy},
			\begin{align*}
				\Big(\sum_{n=0}^\infty \frac{1}{(n!)^2}\Big) \Big(\sum_{n=0}^\infty \frac{1}{(n!)^2}\Big)
				&= \sum_{n=0}^\infty \sum_{p+q = n} \frac{1}{(p!)^2}\frac{1}{(q!)^2} \\
				&= \sum_{n=0}^\infty \sum_{p=0}^n \frac{1}{(p!)^2} \frac{1}{(n-p)!^2} \\
				&= \sum_{n=0}^\infty \sum_{p=0}^n {n\choose p}^2 \frac{1}{(n!)^2}\\
				&= \sum_{n=0}^\infty \frac{1}{(n!)^2} \times \Big(\sum_{p=0}^n {n\choose p}^2\Big) \\
				&= \sum_{n=0}^\infty \frac{1}{(n!)^2} \times {2n\choose n} \\
				&= \sum_{n=0}^\infty \frac{1}{(n!)^2} \times \frac{(2n)!}{n! \times n!} \\
				&= \sum_{n=0}^\infty \frac{1}{(2n!)} {2n\choose n}^2.
			\end{align*}
	\end{enumerate}
	\clearpage
	\centerline{\LARGE Cadeaux du 14/12/22}
	\centerline{\textit{CCP PSI 2010, Centrale PSI 2014 \& TPE PSI 2014}}

	\bigskip
	\bigskip
	\textbf{Cadeau 1} :\\
	\textsl{Soient $n \ge 2$\/ et \begin{align*}
		\Phi: \mathcal{M}_n(\R) &\longrightarrow \mathcal{M}_n(\R) \\
		M &\longmapsto M^\top.
	\end{align*}
	Déterminer les éléments propres, la trace et le déterminant de $\Phi$.}

	\bigskip
	\bigskip
	\begin{comment}
		\textbf{Réponse du cadeau 1} :\\
		On a $-1,1 \in \Sp(\Phi)$.
		Montrons qu'il n'y a pas d'autres valeurs propres. Soit $\lambda \in \R$\/ et soit $M \in \mathcal{M}_n(\R)$\/ non nulle telle que $\Phi(M) = \lambda M$, alors, comme $(M^\top)^\top = M$, alors $\Phi^2 = \id_{\mathcal{M}_n(\R)}$\/ et donc $\Phi^2(M) = \lambda^2 M$, d'où $\lambda = -1$\/ ou $\lambda = 1$.
		Les sous-espaces propres de $\Phi$\/ sont $\mathrm{SEP}(1) = \mathcal{S}_n(\R)$\/ et $\mathrm{SEP}(-1) = \mathcal{A}_n(\R)$, où $\mathcal{S}_n$\/ est l'ensemble des matrices symétriques, et $\mathcal{A}_n$\/ est l'ensemble des matrices antisymétriques.
	\end{comment}
	\textbf{Correction du cadeau 1} :\\
	On a $\Phi^2 = \id$\/ donc $X^2 - 1$\/ annule $\Phi$.
	D'où, $\Sp (\Phi) \subset \{-1,1\}$.
	Et, $\forall M \in \mathcal{S}_n(\R)$, $\Phi(M) = M^\top = M = 1 \times M$. De plus, $\forall M \in \mathcal{A}_n(\R)$, $\Phi(M) = M^\top  = -M = -1 \times M$. D'où $\mathcal{S}_n(\R) \subset \mathrm{SEP}(1)$\/ et $\mathcal{A}_n(\R) \subset  \mathrm{SEP}(-1)$.
	Mais, $\dim \mathcal{S}_n(\R) = n(n+1) / 2$\/ et $\dim \mathcal{A}_n(\R) = n(n-1) / 2$. Donc $\mathrm{SEP}(1) = \mathcal{S}_n(\R)$\/ et $\mathrm{SEP}(-1) = \mathcal{A}_n(\R)$.
	L'endomorphisme $\Phi$\/ possède un polynôme annulateur scindé à racines simples, il est donc diagonalisable.
	En se plaçant dans une base $\mathcal{B}$\/ adaptée : la concaténation d'une base de $\mathcal{S}_n$\/ et d'une base de $\mathcal{A}_n$, on a 
	\[
		\big[\Phi\big]_\mathcal{B} =
		\begin{pNiceMatrix}[last-col]
			\begin{array}{ccc}
				1 & & \\
				& \ddots &\\
				& & 1
			\end{array} & 0 & \frac{n(n+1)}{2}\\
			0 & \begin{array}{ccc}
				-1 & & \\
				& \ddots &\\
				& & -1
			\end{array} & \frac{n(n-1)}{2}
		\end{pNiceMatrix}
	.\]
	On a donc $\tr \Phi = \frac{n(n+1)}{2} - \frac{n(n-1)}{2} = \frac{2}{2}n = n$, et $\det \Phi = (-1)^{\frac{n(n-1)}{2}}$.

	\bigskip
	\bigskip

	\centerline{\pgfornament[width=3cm]{88}}

	\bigskip
	\bigskip
	
	\textbf{Cadeau 2} :\\
	\textsl{Soit $V = \big\{u \in \mathcal{L}\big(\mathcal{M}_n(\R)\big)  \mid \forall M \in \mathcal{M}_n(\R),\: u(M^\top) = \big[u(M)\big]^\top\big\}$. Montrer que $V$\/ est un sous-espace vectoriel de $\mathcal{L}\big(\mathcal{M}_n(\R)\big)$\/ et déterminer sa dimension.}

	\bigskip
	\bigskip
	\begin{comment}
		\textbf{Réponse du cadeau 2} :\\
		Soit $M \in \mathcal{M}_n(\R)$. On considère l'application nulle notée $\tilde 0$\/ : $\tilde 0(M^\top) = 0_{\mathcal{M}_n(\R)} = (0_{\mathcal{M}_n(\R)})^\top = [\tilde 0(M)]^\top$. On a donc bien $\tilde 0 \in V$.
		Soient $\lambda,\mu \in \R$, et $u,v \in V$. Soit $M \in \mathcal{M}_n(\R)$. On a $(\lambda u + \mu v)(M^\top) = \lambda u (M^\top) + \mu v(M^\top) = \lambda (u(M))^\top + \mu (v(M))^\top = \big[(\lambda u + \mu v)(M)\big]^\top$, d'où $\lambda u + \mu v \in V$.
		L'ensemble $V$\/ est bien un sous-espace vectoriel de $\mathcal{L}\big(\mathcal{M}_n(\R)\big)$.
		%On se place dans une base adaptée : on considère le sous-espace vectoriel $\mathcal{S}$\/ des endomorphismes de $\mathcal{M}_n(\R)$\/ symétriques, et $\mathcal{A}$\/ le sous-espace vectoriel des endomorphismes de $\mathcal{M}_n(\R)$\/ antisymétriques.
		Soit $u \in V$. Les endomorphismes $u$\/ et $\Phi$\/ commutent.
	\end{comment}
	\textbf{Correction du cadeau 2} :\\
	On a bien $\tilde 0 \in V$\/ donc $V \neq \O$.
	Soient $u,v \in V$\/ et $\lambda, \mu \in \R$. Pour toute matrice $M \in \mathcal{M}_n(\R)$,
	\begin{align*}
		(\lambda u + \mu v)(M^\top)
		&= \lambda u(M^\top) + \mu v(M^\top) \\
		&= \lambda \big(u(M)\big)^\top + \mu \big(v(M)\big)^\top \\
		&= (\lambda u + \mu v)(M)^\top  \\
	\end{align*}
	donc $\lambda u + \mu v \in V$. Ainsi, $V$\/ est un sous-espace vectoriel de $\mathcal{L}\big(\mathcal{M}_n(\R)\big)$.
	\begin{itemize}
		\item Soit $u \in V$. Pour toute matrice antisymétrique $A \in \mathcal{A}_n$, $u(A^\top) = u(A^\top) = -u(A)$, d'où $u(A) \in \mathcal{A}_n$.
			De même, pour toute matrice symétrique $S \in \mathcal{S}_n$, $u(S^\top) = u(S) = u(S)^\top$\/ donc $u(S) \in \mathcal{S}_n$.
		\item Soit $u \in \mathcal{L}\big(\mathcal{M}_n(\R)\big)$\/ un endomorphisme tel que $u(\mathcal{A}_n) \subset \mathcal{A}_n$\/ et $u(\mathcal{S}_n) \subset \mathcal{S}_n$.
			Soit $M \in \mathcal{M}_n(\R)$. On décompose $M$\/ en $M = A + S$\/ avec $A \in \mathcal{A}_n$\/ et $S \in \mathcal{S}_n$.
			\begin{align*}
				u(M^\top) &= u(A^\top + S^\top)\\
				&= u(A^\top) + u(S^\top) \\
				&= -u(A) + u(S) \\
				&= u(A)^\top + u(S)^\top \\
				&= u(M)^\top \\
			\end{align*}
			donc
			\begin{align*}
				\dim V &= \dim^2 \mathcal{A}_n + \dim^2 \mathcal{S}_n\\
				&= \left( \frac{n(n-1)}{2} \right)^2 + \left( \frac{n(n+1)}{2} \right)^2 \\
				&= \frac{n^2}{4}(n^2 - 2n + 1 + n^2 + 2n + 1) \\
				&= \frac{1}{2}n^2(n^2 + 1). \\
			\end{align*}
			En effet, soit $\mathcal{B}$\/ une base adaptée à la somme $\mathcal{A}_n \oplus \mathcal{S}_n = \mathcal{M}_n(\R)$, \[
				[u]_\mathcal{B} = \begin{pmatrix}
					* & 0\\
					0 & *
				\end{pmatrix}
			.\]
	\end{itemize}


	\bigskip
	\bigskip

	\centerline{\pgfornament[width=3cm]{88}}

	\bigskip
	\bigskip
	\textbf{Cadeau 3} :\\
	\textsl{\begin{enumerate}[label=(\alph*)]
		\item Soit $d \in \N^*$, $a \in \C$\/ et \begin{align*}
				u: \C_d[X] &\longrightarrow \C_d[X] \\
				P &\longmapsto (X-a)P'.
			\end{align*}
			Déterminer les éléments propres de $u$.
		\item En déduire l'ensemble des polynômes de $\C[X]$\/ divisibles par leur dérivée.
	\end{enumerate}}

	\bigskip
	\bigskip
	\textbf{Réponse du cadeau 3} :\\
	\begin{enumerate}[label=(\alph*)]
		\item On a $u(k) = 0 = 0 \times k$\/ avec $k \in \C$.
			De même, on remarque que, \[
				\forall k \in \llbracket 1,d \rrbracket,\: u\big((X-a)^k\big) = k (X-a)^{k-1} \cdot (X-a) = k \times (X-a)^{k}
			.\] Ainsi, $(X-a)^k \in \Ker(k \id - u)$.
			L'endomorphisme $u$\/ possède $d+1$\/ valeurs propres distinctes deux à deux. Or, $\dim \C_d[X] = d + 1$, donc $u$\/ est diagonalisable.
			Et, $\forall k \in \llbracket 0,d \rrbracket$, $\Ker(k \id - u) = \Vect\big[(X-a)^k\big]$. Donc $\Sp(u) = \llbracket 0,d \rrbracket$.

			\textsc{Autre solution} : la base $\mathcal{B} = \big((X-a)^0,(X-a)^1,\ldots,(X-a)^d\big)$\/ est adaptée. En effet, dans cette base \[
				[u]_{\mathcal{B}} = \begin{pmatrix}
					0 & & & & \\
					& 1 & & & \\
					& & 2 & & \\
					& & & \ddots & \\
					& & & & d \\
				\end{pmatrix}
			.\]
		\item
			\begin{itemize}
				\item \textsc{Analyse}. Soit $P$\/ un polynôme de degré $d \ge 1$, tel que $P'  \mid P$.
					Alors, $P = \frac{1}{d} (X-a) \times P'$.
					D'où, $u(P) = (X-a) P' = d P$.
				\item \textsc{Synthèse}. \textsc{ok}.
			\end{itemize}
	\end{enumerate}
	\clearpage
	\centerline{\LARGE Cadeaux du 15/12/22}
	\centerline{\textit{Centrale PC 2016, X ESPCI PC 2013 \& X ESPCI PC 2014, 2015}}

	\bigskip
	\bigskip
	\textbf{Cadeau 1} :\\
	\textsl{ Soit $\mathcal{D}_n$\/ l'espace des matrices diagonales et $D = \mathrm{diag}(d_1, \ldots, d_n)$.
	Montrer que $(I_n, D, \ldots, D^{n-1})$\/ est une base de $\mathcal{D}_n$\/ si, et seulement si les réels $d_i$\/ sont distincts deux à deux.}

	\bigskip
	\bigskip
	\textbf{Réponse du cadeau 1} :\\
	\begin{align*}
		&(I_n, D, \ldots, D^{n-1}) \text{ est une base de } \mathcal{D}_n\\
		\iff& \forall (a_0,\ldots,a_{n-1}),\quad a_0 I_n + a_1 D + \cdots + a_{n-1} D^{n-1} = 0 \implies a_0 = a_1 = \cdots = a_{n-1}\\
		\iff& \forall (a_0,\ldots,a_{n-1}),\quad \begin{cases}
			a_0 + a_1 d_1 + \cdots + a_{n-1} d_1^{n-1} &= 0\\
			\quad\quad\vdots & \ddots \vdots\\
			a_0 + a_1 d_n + \cdots + a_{n-1}d^{n-1} &= 0
		\end{cases} \implies a_0 = \cdots = a_{n-1} = 0 \\
		\iff& \forall (a_0,\ldots, a_{n-1}), \begin{pmatrix}
			1 & d_1 & \ldots & d_1^{n-1}\\
			\vdots & \vdots & \ddots & \vdots\\
			1 & d_n & \ldots & d_n^{n-1}
		\end{pmatrix} \begin{pmatrix}
			a_0\\
			\vdots\\
			a_{n-1}
		\end{pmatrix} = 0 \implies a_0 = \cdots = a_{n-1} = 0 \\
		\iff& \text{ les réels $d_i$\/ sont distincts deux à deux}. \\
	\end{align*}
	car on reconnaît le déterminant de \textsc{Vandermonde}.

	\bigskip
	\bigskip

	\centerline{\pgfornament[width=3cm]{88}}

	\bigskip
	\bigskip
	\textbf{Cadeau 2} :\\
	\textsl{Soit $X \in \mathcal{M}_{n,1}(\R)$\/ non nul.
	\begin{enumerate}[label=(\alph*)]
		\item Montrer que $V = \{M \in \mathcal{M}_{n}(\R)  \mid  M\cdot X = 0\}$\/ est un sous-espace vectoriel de $\mathcal{M}_n(\R)$, et déterminer la dimension.
		\item Montrer que l'ensemble $W$\/ des matrices de $\mathcal{M}_n(\R)$\/ dont $X$\/ est un vecteur propre, est un sous-espace vectoriel de $\mathcal{M}_n(\R)$\/ et que $\dim W = ?$.
	\end{enumerate}}

	\bigskip
	\bigskip
	\textbf{Réponse du cadeau 2} :\\
	\begin{enumerate}[label=(\alph*)]
		\item Soit $M = 0_{\mathcal{M}_n(\R)}$\/ la matrice nulle. On a $M \cdot X = 0$, donc $0_{\mathcal{M}_n(\R)} \in V$.
			Soient $M,N \in V$\/ et $\lambda, \mu \in \R$. On a $(\lambda M + \mu N)\cdot X = \lambda M \cdot X + \mu N \cdot X = 0$, donc $\lambda M + \mu N \in V$.
			On en déduit que $V$\/ est un sous-espace vectoriel de $\mathcal{M}_n(\R)$.

			On se place dans une base adaptée $\mathcal{B} = (X, X_2, \ldots, X_n)$\/  où on a complété la famille $X$\/ par des vecteurs $X_2,\ldots,X_n$.
			Soit $f$\/ l'endomorphisme représenté, dans une base, par $M$.
			Ainsi,
			\begin{align*}
				M \in V \iff& M \cdot X = 0 \\
				\iff&  f(x) = 0 \text{ où $X$ est la matrice de $x$ dans la même base} \\
				\iff& \exists P \in \mathrm{GL}_n(\R),\: P^{-1}\cdot M\cdot P = \begin{pmatrix}
					0 & * & \ldots & *\\
					\vdots & \vdots & \ddots & \vdots\\
					0 & * & \ldots & *
				\end{pmatrix} \\
			\end{align*}
			Ainsi, \[
				\boxed{\dim V = n^2 - n.}
			\]
		\item Soit $M = 0_{\mathcal{M}_n(\R)}$\/ la matrice nulle. On a $M \cdot X = 0_{\mathcal{M}_n(\R)} = 0_\R \cdot X$, donc $0_{\mathcal{M}_n(\R)} \in W$.
			Soient $M,N \in W$\/ et $\lambda, \mu \in \R$.
			Soit $\alpha$\/ la valeur propre associée au vecteur propre $X$\/ pour la matrice $M$.
			Soit $\beta$\/ la valeur propre associée au vecteur propre $X$\/ pour la matrice $N$.
			Ainsi, $(\lambda M + \mu N)\cdot X = \lambda M\cdot X + \mu N \cdot X = \lambda \alpha X + \mu \beta X = (\lambda \alpha + \mu \beta) \cdot X$. On en déduit que $\lambda M + \mu N \in W$.
			Ainsi, $W$\/ est un sous-espace vectoriel de $\mathcal{M}_n(\R)$.

			De même, dans la même base adaptée
			\begin{align*}
				M \in W \iff& \exists P \in \mathrm{GL}_n(\R),\: P^{-1}\cdot A\cdot P = \begin{pmatrix}
					* & * & \ldots & \ldots & *\\
					0 & \vdots & \ddots & & \vdots\\
					\vdots & \vdots & & \ddots & \vdots\\
					0 & * & \ldots & \ldots & *
				\end{pmatrix} \\
			\end{align*}
			D'où, \[
				\boxed{\dim W = n^2 - n + 1.}
			\]
	\end{enumerate}

	\clearpage
	\centerline{\LARGE Cadeau du 10/01/23}

	\bigskip
	\bigskip
	\bigskip
	\textbf{Cadeau} :\\
	\textsl{Déterminer le rayon de convergence de la série entière $\sum x^{(n^2)}$.}

	\bigskip
	\bigskip
	\textbf{Réponse du cadeau} :\\
	On pose $(u_n)$\/ le terme général de la série.
	Son rayon de convergence vaut 1. En effet, si $x = 1$, la série $\sum 1$\/ diverge, d'où $R \le 1$.
	De plus, si $|x| < 1$, alors $\forall n \in \N$, $|u_n| \le |x|^n$.
	Or, la série $\sum |x|^n$\/ converge, donc $\sum |u_n|$\/ aussi, d'où $R \ge 1$\/ et donc $R = 1$.
	\clearpage
	\centerline{\LARGE Cadeau du 13/01/23}

	\bigskip
	\bigskip
	\bigskip
	\textbf{Cadeau} :\\
	\textsl{Soient $F$\/ et $G$\/ deux espaces d'un espace préhilbertien\footnote{il n'y a donc aucune hypothèses sur la dimension de $E$.} Montrer que, si $F \oplus G = E$\/ et $F \bot G$, alors $G = F^\top$\/ et $F = G^\top$.}

	\clearpage
	\centerline{\LARGE Cadeau du 18/01/23}

	\bigskip
	\bigskip
	\bigskip
	\textbf{Cadeau} :\\
	\textsl{Soient $\sum a_n z^n$\/ une série entière, et $R$\/ son rayon de convergence. On considère la série entière $\sum {a_n}^n\: z^n$\/ de rayon de convergence $R'$.
	\begin{enumerate}
		\item Montrer que,
			\begin{enumerate}
				\item si $R > 1$, alors $R' = +\infty$\/ ;
				\item si $R < 1$, alors $R' = 0$.
			\end{enumerate}
		\item Que se passe-t-il si $R = 1$\/ ?
	\end{enumerate}}
	\bigskip
	\bigskip
	\textbf{Réponse du cadeau} :\\
	\begin{enumerate}
		\item
			\begin{enumerate}
				\item Soit $x \in {]-R,R[}$. On a $R > 1$, d'où $\sum |a_n| 1^n$\/ converge, d'où $a_n \tendsto{n\to \infty}0$. Ainsi, à partir d'un certain rang, $|a_n x| \le \lambda$, où $0 < \lambda < 1$\/ et donc $|a_n x|^n \le \lambda^n$.
					Or, la série $\sum \lambda^n$\/ converge, d'où $\sum a_n x^n$\/ converge.
					On en déduit $R' = +\infty$.
				\item 
			\end{enumerate}
		\item On considère la série géométrique $\sum x^n$\/ : on pose $a_n = 1$\/ pour tout $n \in \N$. Le rayon de convergence de cette série est $R = 1$. Mais, $\sum {a_n}^n x^n = \sum x^n$, et donc $R' = 1$.\\
			On considère la série $\sum \frac{x^n}{n}$. On pose, pour $n \in \N^*$, $a_n = \frac{1}{n}$. Soit $x \in R$. On pose $u_n = |{a_n}^n x^n|$\/ pour  $n \in \N$.
			Pour tout entier $n \in \N^*$,
			\begin{align*}
				\frac{u_{n+1}}{u_n} &= \left| \frac{a_{n+1}}{a_n} \right|^n \: |a_{n+1}|\:|x| \\
														&= \left( \frac{n}{n+1} \right)^n \cdot \frac{1}{n+1} |x| \tendsto{n\to +\infty} 0 \\
			\end{align*}
			D'où $\sum u_n$\/ converge, et donc $R' = +\infty$.
	\end{enumerate}
	\clearpage
	\centerline{\LARGE Cadeau du 18/01/23}

	\centerline{\textit{Banque PT 2006 Maths C}}

	\bigskip
	\bigskip
	\bigskip
	\textbf{Cadeau} :\\
	\textsl{Trouver les solutions développables en séries entières de l'équation différentielle \[
		16\,(x^2 - x)\,y'' + (16x - 8)\, y' - y = 0
	.\]
	Indication : Il faut montrer que \[
		f : x \mapsto \sum_{n=0}^\infty  a_n x^n \text{ est une solution} \iff \forall n,\: a_{n+1} = \frac{(4n  +1)(4n - 1)}{8(n+1)(2n+1)}\: a_n
	.\]Trouver le rayon de convergence de la série, et calculer $a_n$\/ en fonction de $n$.}

	\bigskip
	\bigskip
	\textbf{Réponse du cadeau} :\\
	Soit $R$\/ le rayon de convergence d'une série entière $\sum a_n x^n$.
	Soit, pour tout $x \in {]-R,R[}$, $f(x) = \sum_{n=0}^\infty a_n x^n$.
On peut dériver terme à terme une série entière sans changer son rayon de convergence, d'où
	\[
		\forall x \in {]-R,R[},\quad \begin{cases}
			f'(x) = \ds\sum_{n=1}^\infty n a_n x^{n-1}\\
			f''(x) = \ds\sum_{n=2}^\infty n(n-1) a_n x^{n-2}\\
		\end{cases}
	.\]
		\begin{align*}
			f \text{ est une solution}
			\iff& \forall x \in {]-R,R[},\: 16 (x^2 - x) \sum_{n=2}^\infty n(n-1) a_n x^{n-2} + (16x - 8) \sum_{n=1}^\infty n a_n x^{n-1} - \sum_{n=0}^\infty a_n x^n = 0\\
			\iff& \forall x \in {]-R,R[},\: \sum_{n=2}^\infty 16n(n-1)a_n x^n - \sum_{n=2}^\infty 16n(n-1)a_n x^{n-1}\\
			&\quad\quad\quad\quad\quad+ \sum_{n=2}^\infty 16na_n x^n + 16x a_1 - \sum_{n=2}^\infty 8n a_n x^{n-1} - \sum_{n=0}^\infty a_n x^n = 0 \\
			\iff& \forall x \in {]-R,R[},\: \sum_{n=2}^\infty 16n(n-1)a_n x^n - \sum_{n=1}^\infty 16n(n+1)a_{n+1} x^n\\
			&\quad\quad\quad\quad\quad+ \sum_{n=2}^\infty 16na_n x^n + 16x a_1 - \sum_{n=1}^\infty 8(n+1) a_{n+1} x^n - \sum_{n=0}^\infty a_n x^n = 0 \\
			\iff& \forall x \in {]-R,R[},\: \sum_{n=2}^\infty \Big(16n(n-1)a_n - 16 n(n+1)a_{n+1} + 16na_n - 8(n+1)a_{n+1} - a_n \Big) x^n\\
			&\quad\quad\quad\quad\quad+ 16a_1x - 32a_2x - 16a_2x - a_1x = 0\\
			\mathllap{\text{(par unicité du DSE)}}\iff& \begin{cases}
				\forall n \ge 2,\quad \big(-16n(n+1) - 8(n+1)\big)a_{n+1} - \big(16n(n-1) + 16n - 1\big)a_n = 0\\
				15 a_1 - 48 a_2 = 0
			\end{cases} \\
			\iff& \begin{cases}
				\ds\forall n \ge 2,\: a_{n+1} = a_n \times \frac{16n(n-1) + 16n - 1}{16n(n+1) + 8(n+1)}\\
				\ds a_2 = \frac{15}{48} a_1
			\end{cases} \\
			\iff& \begin{cases}
				\ds \forall n \ge 2,\: a_{n+1} = a_n \times \frac{16n^2 - 16n + 16n - 1}{16n^2 + 16n + 8n + 8}\\
				\ds a_{2} = \frac{15}{48} a_1
			\end{cases} \\
			\iff& \begin{cases}
				\ds \forall n \ge 2,\: a_{n+1} = a_n \times \frac{(4n - 1)(4n+1)}{8(2n^2 + 3n + 1)}\\
				\ds a_2 = \frac{15}{48} a_1
			\end{cases} \\
			\iff& \begin{cases}
				\ds \forall n \ge 2,\: a_{n+1} = a_n \times \frac{(4n - 1)(4n+1)}{8(n + 1)(2n+1)}\\
				\ds a_2 = \frac{15}{48} a_1
			\end{cases} \\
		\end{align*}

	Soit $x \neq 0$. On a
	\begin{align*}
		\left| \frac{a_{n+1} x^{n+1}}{a_n x^n} \right| = \frac{a_{n+1}}{a_n} |x| = \frac{(4n - 1)(4n+1)}{8(n+1)(2n+1)} \sim \frac{16n^2}{16n^2} |x| \tendsto{n\to \infty} |x|
	\end{align*}
	D'où, par le critère de \textsc{d'Alembert}, $R = 1$.
	On a $a_1 = \frac{1 \times (-1)}{4 \times  2} a_0$, $a_2 = \frac{5 \times  3}{4 \times  4 \times  3} a_1$, et $a_3 = \frac{9 \times 7}{4 \times 6 \times 5} a_2$.
	On devine la formule \[
		\forall n \in \N, \quad a_n = -\frac{(4n -3) \times \cdots \times 9 \times 7 \times 5 \times 3 \times 1}{4^n \times 2n \times \cdots \times 6 \times  5 \times  4 \times  3 \times 2} a_0
	.\] 
	Montrons par récurrence la formule \[
		\forall n \in \N,\quad\quad a_n = - \frac{(4n -2)!}{4^n \times (2n)! \times 2^{2n - 1} \times (2n-1)!}
	.\]
	\clearpage
	\centerline{\LARGE Cadeaux du 31/01/23}

	\bigskip
	\bigskip
	\bigskip
	\textbf{Cadeau 1} :\\
	\begin{slshape}
		Soit $A$\/ la matrice \[
			A = \frac{1}{4} \begin{pmatrix}
				-2 & -\sqrt{6} & \sqrt{6} \\
				\sqrt{6} & 1 & 3\\
				-\sqrt{6} & 3 & 1
			\end{pmatrix}
		.\]
		Interpréter géométriquement l'endomorphisme $f$\/ représenté par $A$\/ dans la base orthonormée $(\vec{\imath}, \vec{\jmath}, \vec{k})$.

	\end{slshape}

	\bigskip

	\textbf{Cadeau 2} (Exercice 6 du \textsc{td}) :\\
	\begin{slshape}
		Écrire la matrice, dans la base orthonormée directe $(\vec{\imath}, \vec{\jmath}, \vec{k})$ de $\R^3$, de la rotation d'angle $\frac{\pi}{6}$\/ autour de l'axe dirigé et orienté $\vec{\imath} + \vec{\jmath}$.
	\end{slshape}

	\bigskip
	\bigskip
	\textbf{Réponse du cadeau 2} :\\
	\begin{figure}[H]
		\centering
		\begin{asy}
			import three;
			draw(O--X, Arrow3(TeXHead2));
			draw(O--Y, Arrow3(TeXHead2));
			draw(O--Z, red, Arrow3(TeXHead2));
			
			label("$\vec\imath$", 1.1X);
			label("$\vec\jmath$", 1.1Y);
			label("$\vec k$", 1.1Z, red);

			triple nz(real x, real y, real z) { triple v = (x,y,z); return v/length(v); }

			pair A = expi(pi/6);
			pair B = expi(pi/2 + pi/6);
			triple U = nz(A.x, A.y, 0);
			triple V = nz(B.x, B.y, 0);

			draw(O--U, red, Arrow3(TeXHead2));
			draw(O--V, red, Arrow3(TeXHead2));

			label("$\vec u$", 1.1U, red);
			label("$\vec v$", 1.1V, red);

			size(8cm);
		\end{asy}
		\caption{Rotation d'angle $\frac{\pi}{6}$\/ autour de l'axe dirigé orienté $\vec{\imath} + \vec{\jmath}$.}
	\end{figure}
	On cherche $\mathcal{B}' = (\vec{k}, \vec{u}, \vec{v})$\/ une base orthonormée directe. On pose $\vec{v} = (\vec{\imath} + \vec{\jmath}) / \sqrt{2}$, et $\vec{u} = \vec{v} \wedge \vec{k}$.
	On a \[
		\vec{u} = \vec{v} \wedge \vec{k} = \begin{pmatrix}
			1 / \sqrt{2}\\ 1 / \sqrt{2}\\ 0
		\end{pmatrix} \wedge \begin{pmatrix}
			0 \\ 0 \\ 1
		\end{pmatrix} = \begin{pmatrix}
			1 / \sqrt{2} \\ - 1 / \sqrt{2} \\ 0
		\end{pmatrix}
	.\] 
	Soit $f$\/ la rotation d'angle $\frac{\pi}{6}$\/ autour de l'axe dirigé et orienté par $(\vec{\imath} + \vec{\jmath}) / \sqrt{2}$.
	Ainsi, \[
		\big[\:f\:\big] =
		\begin{pNiceMatrix}[last-row,last-col]
			\cos \frac{\pi}{6} & - \sin \frac{\pi}{6} & 0 & \vec{k}\\
			\sin \frac{\pi}{6} & \cos \frac{\pi}{6} & 0 & \vec{u}\\
			0 & 0 & 1 & \vec{v}\\
			f(\vec{k}) & f(\vec{u}) & f(\vec{v})
		\end{pNiceMatrix} = A'
	.\] Ainsi, par un changement de base $\mathcal{B} = (\vec{\imath}, \vec{\jmath}, \vec{k})$\/ vers $\mathcal{B}' = (\vec{k}, \vec{u}, \vec{v})$. Ainsi, $A' = P^{-1} \cdot A \cdot P$\/ d'où $A = P \cdot A' \cdot P^{-1}$, et \[
		P =
		\begin{pNiceMatrix}[last-col,last-row]
			0 & \frac{1}{\sqrt{2}} & \frac{1}{\sqrt{2}} & \vec{\imath}\\
			0 & -\frac{1}{\sqrt{2}} & \frac{1}{\sqrt{2}} & \vec{\jmath}\\
			1 & 0 & 0 & \vec{k}\\
			\vec{k} & \vec{u} & \vec{v}
		\end{pNiceMatrix}
	.\]
	De plus, $P^{-1} = P^\top$\/ car les bases $\mathcal{B}$\/ et $\mathcal{B}'$\/ sont orthonormées.

	\bigskip

	\textbf{Réponse du cadeau 1} :\\
	Les colonnes de cette matrice forment une base orthonormée : $\left<C_1  \mid C_2 \right> = \left<C_2  \mid C_3 \right> = \left<C_3  \mid C_1 \right> = 0$, et $\|C_1\|^2 = \|C_2\|^2 = \|C_3\|^2 = \sqrt{(4 + 6 + 6) / 16} = 1$.
	Cette base est directe car $C_1 \land C_2 = C_3$\/ en calculant le produit vectoriel.
	(Ou, on peut aussi calculer $\det A$, et on trouve 1.)
	Ainsi, on a $A \in \mathrm{SO}_3(\R)$. C'est la matrice d'une rotation, d'où, dans une base adaptée, l'endomorphisme est représenté par la matrice \[
		\begin{pmatrix}
			\cos \theta & -\sin \theta & 0\\
			\sin \theta & \cos \theta & 0\\
			0 & 0 & 1
		\end{pmatrix} 
	.\] Comme la trace est un invariant de similitude, $\tr A = 0 = 2\cos \theta + 1$.
	Ainsi, $\cos \theta = -\frac{1}{2}$, donc $\theta \equiv \pm \frac{2\pi}{3}\mod{2\pi}$.
	On résout donc $AX = 1X$\/ avec $X^3$ :
	\begin{align*}
		A\cdot X = 1X \iff& \begin{cases}
			-2x - \sqrt{6}y + \sqrt{6} z = 4x\\
			\sqrt{6}x + y + 3z = 4y\\
			-\sqrt{6}x + 3y + z = 4z
		\end{cases} \\
		\vdots\quad& \\
		\iff& X \in \Vect(\vec{\jmath} + \vec{k}) \\
	\end{align*}
	\begin{itemize}
		\item Soit $\vec{w} = (\vec{\jmath}+\vec{k}) / \sqrt{2}$. Cherchons le signe de $\theta$, modulo $2\pi$.
			Calculons $f(\vec{\imath}) = -\frac{1}{2} \vec{\imath} + \frac{\sqrt{6}}{4} \vec{\jmath} - \frac{\sqrt 6}{4} \vec{k}$.
			Et, on calcule $\vec{\imath} \land f(\vec{\imath})$\/ : \[
				\vec{\imath} \land f(\vec{\imath}) = \begin{pmatrix} 1 \\ 0 \\ 0 \end{pmatrix} \land \frac{1}{4} \begin{pmatrix} -2 \\ \sqrt{6} \\ -\sqrt{6} \end{pmatrix} = \frac{1}{4}\begin{pmatrix} 0\\ \sqrt{6}\\ -\sqrt{6} \end{pmatrix} = \frac{\sqrt{6}}{4} (\vec{\jmath} + \vec{k})
			,\] qui a le même sens que $\vec{w}$, d'où $\theta = +\frac{2\pi}{3}$.
		\item Si $\vec{w} = -(\vec{\jmath} + \vec{k}) / \sqrt{2}$. De la même manière, on a $\theta = - \frac{2\pi}{3}$.
	\end{itemize}

	\clearpage
	\centerline{\LARGE Cadeau du 01/02/23}

	\bigskip
	\bigskip
	\bigskip
	\textbf{Cadeau} :\\
	\begin{slshape}
		Soit $A$\/ la matrice \[
			A = \frac{1}{4} \begin{pmatrix}
				-2 & \sqrt{6} & -\sqrt{6} \\
				\sqrt{6} & 3 & 1\\
				-\sqrt{6} & 1 & 3
			\end{pmatrix}
		.\]
		Interpréter géométriquement l'endomorphisme $f$\/ représenté par $A$\/ dans la base orthonormée $(\vec{\imath}, \vec{\jmath}, \vec{k})$.

	\end{slshape}

	\bigskip
	\bigskip
	\textbf{Réponse du cadeau} :\\
	On remarque que, \[
		B = A \times \underbrace{\begin{pmatrix}
			1 & 0 & 0\\
			0 & 0 & 1\\
			0 & 1 & 0
		\end{pmatrix}}_S.
	\] On pose $s$, $a$ et $b$ les endomorphismes représentés par $S$, $A$\/ et $B$.
	Ainsi $b = a \circ s$.
	Et, $s$\/ est la reflection par rapport au plan $\Vect(\vec{\imath}, \vec{\jmath} + \vec{k})$.

	\textbf{Autre méthode} :\\
	On diagonalise la matrice, et on conclut. Les valeurs propres sont \textit{sympathiques}.
	\clearpage
	\centerline{\LARGE Cadeau du 02/02/23}

	\bigskip
	\bigskip
	\bigskip
	\textbf{Cadeau} :\\
	\begin{slshape}
		Soit $E$\/ un espace euclidien, et soit $\vec{a} \in E$\/ un vecteur non-nul.
		Montrer que l'application \begin{align*}
			\s: E &\longrightarrow E \\
			\vec{x} &\longmapsto \vec{x} - 2 \frac{\left<\vec{a}  \mid \vec{x} \right>}{\|\vec{a}\|^2} \vec{a}
		\end{align*}
		est une réflexion.
	\end{slshape}
	
	\bigskip
	\bigskip
	\textbf{Réponse du cadeau} :\\
	Soit $\mathcal{B} = (\vec{e}_1, \ldots, \vec{e}_{n-1}, \vec{a})$, où $(\vec{e}_1, \ldots, \vec{e}_n)$\/ est une base de $H = [\Vect \vec{a}]^\perp$.
	Ainsi, dans cette base, \[
		\big[\:\s\:\big]_\mathcal{B} = \begin{pNiceMatrix}[last-row, last-col]
			1 & & & & \vec{e}_1\\
			& \ddots & & & \vdots\\
			& & 1 & & \vec{e}_{n-1}\\
			& & & -1 & \vec{a}\\
			\s(\vec{e}_1) & \ldots & \s(\vec{e}_{n-1}) & \s(\vec{a})
		\end{pNiceMatrix} 
	.\] Donc, $\s$\/ est la symétrie orthogonale par rapport à l'hyperplan $H$, (\textit{i.e.}\ c'est une réflexion).
	
	\bigskip
	\bigskip
	\textbf{Corrigé (partiel) de l'exercice 3 du \textsc{td} 13} :\\
	\begin{itemize}
		\item $\ds\|\vec{u}\| = 0 \implies \sup_{n \in \N} |u_n| = 0 \implies u = 0_E$\/ ;
		\item Soit $\alpha \in \R$. $\ds\|\alpha u\| = \sup_{n \in \N} |\alpha u_n| = |\alpha| \sup_{n \in \N} |u_n| = |\alpha|\:\|\vec{u}\|$\/ ;
		\item Soient $u,v \in E$. $\|u + v\| = \sup_{n \in \N} |u_n + v_n|$.
			Or, $\forall n \in \N$, $|u_n + v_n| \le |u_n| + |v_n|$\/ par inégalité triangulaire.
			Et, $\forall n \in \N$, $|u_n| \le \sup_{n \in \N} |u_n|$\/ car le $\sup$\/ est un majorant.
			De même, $\forall n \in \N$, $|v_n| \le \sup_{n \in \N} |v_n|$\/ car le $\sup$\/ est un majorant.
			D'où, $|u_n + v_n| \le \sup_{n \in \N} |u_n| + \sup_{n \in \N} |v_n|$\/ qui est un majorant.
			Or, $\sup_{n \in \N} |u_n + v_n|$\/ est le plus petit majorant.
			On a donc $\sup_{n \in \N} |u_n + v_n| \le \sup_{n \in \N} |u_n| + \sup_{n \in \N} |v_n|$\/ donc $\|u + v\| \le \|u\| + \|v\|$.
	\end{itemize}

	\clearpage
	\centerline{\LARGE Cadeau du 03/02/23}
	\centerline{\textit{Suite du cadeau du 02/02/2023}}

	\bigskip
	\bigskip
	\bigskip
	\textbf{Cadeau} :\\
	\begin{slshape}
		Soient $\vec{u}$\/ et $\vec{v}$\/ deux vecteurs distincts et de même norme. Montrer qu'il existe une unique réflexion $\s$ telle que $\s(\vec{u}) = \vec{v}$.
	\end{slshape}

	\bigskip\bigskip
	\textbf{Réponse du cadeau} :\\
	On procède par Analyse-Synthèse.
	\begin{description}
		\item[Analyse]  On suppose que $\s$ est une réflexion par rapport à un hyperplan $H$, et que $\s(\vec{u}) = \vec{v}$.
			Alors, $\s(\vec{x}) - \vec{x} \perp H$, démontré plus tard.
			En particulier, $\s(\vec{u}) - \vec{u} = \vec{v} - \vec{u} \perp H$, d'où l'unicité de $H$.
		\item[Synthèse] Soient $\vec{a} = \vec{v} - \vec{u}$ et $H \perp \vec{a}$.
			Alors $\s : \vec{x} \mapsto \vec{x} - 2 \left<\vec{a}  \mid \vec{x} \right>\: \vec{a} / \|\vec{a}\|^2$ est une réflexion.
			Il reste à montrer que $\s (\vec{u}) = \vec{v}$ : \[
				\s(\vec{u}) = \vec{u} - \frac{\left<\vec{v} -2 \vec{u}  \mid \vec{u} \right>}{\|\vec{v} - \vec{u}\|} (\vec{v} - \vec{u})
			.\] On veut montrer que $\left<\vec{v}-\vec{u}  \mid \vec{u} \right> / \|\vec{v} - \vec{u}\|^2 = -\frac{1}{2}$.
			On calcule $\left<\vec{u} - \vec{v}  \mid \vec{v} \right> = \left<\vec{u}  \mid \vec{u} \right> - \left<\vec{v}  \mid \vec{u} \right>$. Par ailleurs, $\frac{1}{2} \|\vec{v} - \vec{u}\| = \frac{1}{2} \left<\vec{v} - \vec{u}  \mid \vec{v}- \vec{u} \right> = \frac{1}{2}(\|\vec{v}\|^2 - 2 \left<\vec{u} \mid \vec{v} \right> + \|\vec{u}\|^2) = \|\vec{u}\|^2 - \left<\vec{u} \mid \vec{v} \right>$. Or, par hypothèse $\|\vec{u}\| = \|\vec{v}\|$.
	\end{description}
	Montrons que $\s(\vec{x}) - \vec{x} \perp H$.
	On sait que $H \oplus H^\perp = E$. Pour tout vecteur $\vec{x}$, il existe un unique couple de vecteurs $(a, b) \in H \times H^\perp$\/ tel que $\vec{x} = \vec{a} + \vec{b}$.
	Ainsi, $\s(\vec{x}) = \s(\vec{a}) + \s(\vec{b}) = \vec{a} - \vec{b}$.
	Et, donc $\s(\vec{x}) - \vec{x} = \vec{a} - \vec{b} - \vec{a}- \vec{b} =-2\vec{b} \in H^\perp$.

	\clearpage
	\centerline{\LARGE Cadeau du 03/03/23}
	\centerline{\textit{Cadeau de Corentin en khôlle avec \textsc{JJ.\ Mallet}}}

	\bigskip
	\bigskip
	\bigskip
	\textbf{Cadeau} :\\
	\begin{slshape}
		On considère sur $E = \R[X]$ les normes $N_1$ et $N_2$ ainsi que l'application $\varphi$ définie par \[
			\forall P \in E, \quad N_1(P) = \sup_{t \in [0,1]} |P(t)| \quad\quad N_2 = \sup_{t \in [1,2]} |P(t)| \quad\quad \varphi(P) = P(0)
		.\]
		\begin{enumerate}
			\item Montrer que $\varphi$ définit une application de $(E, N_1)$ vers $\R$, mais discontinue de $(E, N_2)$ vers $\R$. \textit{(On pourra utiliser la suite définie par $P_n(t) = (1 - t/2)^n$.)}
			\item Monter que $\Ker \varphi$ est fermé dans $(E, N_1)$. L'est-il dans $(E, N_2)$ ?
		\end{enumerate}
	\end{slshape}

	\bigskip\bigskip
	\textbf{Réponse du cadeau} :\\
	\begin{enumerate}
		\item L'application $\varphi$ est linéaire. Montrons qu'il existe $M \in \R^+$ tel que, pour tout polynôme $P$, $|\varphi(P)| \le M\, N_1(P)$.
			On a $|\varphi(P)| = |P(0)| \le \sup_{t \in [0,1]}\:|P(t)| = N_1(P)$, car c'est un majorant. D'où, $|\varphi(P)| \le N_1(P)$ pour tout polynôme $P$.

			On pose la suite de polynômes définie comme $P_n(t) = (1 - t / 2)^n$.
			On a $N_2(P_n) = \sup_{t \in [1,2]}\: |(1-t/2)^n| = (1/2)^n$ car la fonction $t \mapsto (1 - t / 2)^n$ est décroissante et positive.
			Et, $|\varphi(P_n)| = 1$.
			Par l'absurde, supposons qu'il existe $K \in \R$ tel que, pour tout polynôme $P$, $|\varphi(P)| \le K \cdot N_2(P)$.
			Ainsi, pour tout $n \in \N$, $1 \le K \left( \frac{1}{2} \right)^n$. Les inégalités larges passent à la limite, d'où $1 \le 0$, ce qui est absurde.
		\item L'ensemble $\{0\}$ est un fermé, $\varphi$ est continue sur $(E, N_1)$, et, par définition de $\Ker$, $\varphi^{-1}(\{0\}) = \Ker \varphi$.
			Ainsi, $\Ker \varphi$ est un fermé (l'image réciproque d'un fermé est un fermé).

			{\color{gray}
			On a $\varphi(P_n) = 1$, donc $P_n \in E \setminus \Ker \varphi$.
			Or, $N_2(P_n) = \frac{1}{2^n} \to 0$, quand $n \to \infty$.
			Et, $0 \in \Ker \varphi$ car $\varphi$ est linéaire.
			Ainsi, $E \setminus \Ker \varphi$ n'est pas un fermé.
			On en déduit que $\Ker \varphi$ n'est pas un ouvert pour $N_2$. }

			Posons $Q_n = P_n - 1$ et alors $\varphi(Q_n) = 0$.
			On a $N_2(Q_n + 1) = N_2(P_n) = \left( \frac{1}{2} \right)^n \to 0$ quand $n\to \infty$.
			Donc, $Q_n \to -1$.
			Donc, pour tout $n \in \N$, $Q_n \in \Ker \varphi$ mais $\lim_{n\to \infty} Q_n \not\in \Ker \varphi$.
			Par caractérisation séquentielle d'un fermé, on en déduit que $\Ker \varphi$ n'est pas un fermé.
	\end{enumerate}

	\clearpage
	\centerline{\LARGE Cadeau du 27/03/23}

	\bigskip
	\bigskip
	\bigskip
	\textbf{Cadeau} :\\
	\begin{slshape}
		On considère la matrice $A$ définie comme \[
			A = \begin{pmatrix}
				3 & 1 & -1\\
				2 & 2 & -1\\
				2 & 2 & 0
			\end{pmatrix}
		.\]
		\begin{enumerate}
			\item Déterminer le polynôme minimal $\mu_A$ de la matrice $A$.
			\item En déduire, pour tout entier $n \in \N$, $A^n$ en fonction de $I_3$, $A$ et $A^2$.
			\item Calculer $\exp A$.
		\end{enumerate}
	\end{slshape}

	\bigskip\bigskip
	\textbf{Réponse du cadeau} :\\
	\begin{enumerate}
		\item On calcule le polynôme caractéristique $\chi_A$, et on trouve $\chi_A = (X - 1) \cdot (X-2)^2$.
			(Vérification : pour un polynôme scindé, $\tr A = \sum_{\lambda \in \Sp A} m_\lambda \cdot \lambda$, et $\det A = \prod_{\lambda \in \Sp A} \lambda^{m_\lambda}$.)
			Le polynôme minimal, $\mu_A$, divise tous les polynômes annulateurs de $A$.
			En particulier, il divise $\chi_A$, d'après le théorème de Cayley \& Hamilton.
			En plus, le spectre de $A$ est l'ensemble des racines de $\chi_A$, mais c'est aussi l'ensemble des racines de $\mu_A$.
			On peut donc calculer $(A-I_3) \cdot (A - 2I_3)$, qui n'est pas nul.
			D'où, $\chi_A = \mu_A$.

			Autre méthode : on trouve $\dim \mathrm{SEP}(2) = 1$, d'où $A$ n'est pas diagonalisable, elle ne possède donc pas de polynôme annulateur scindé à racines simples.
			Ainsi, $\mu_A$ n'est pas scindé à racines simples.
			On en déduit donc que $\mu_A = \chi_A$.

			Autre autre méthode : la famille $(I_3, A, A^2)$ est libre, il n'existe donc pas de polynôme annulateur de degré inférieur ou égal à $2$.
		\item On réalise la division euclidienne $X^n \div \mu_A(X)$, pour obtenir un quotient $Q_n(X)$ et un reste $R_n(X)$.
			Ainsi, $X^n = \mu_A(X) \cdot Q_n(X) + R_n(X)$, et $\deg R_n < \deg \mu_A$.
			Et donc, $A^n = R_n(A)$.
			Or, $R_n(X) = a_n X^2 + b_n X + c_n$.
			Il y a 3 inconnues, on cherche 3 équations.
			On a $1^n = 0 \times Q_n(1) + R_n(1) = a_n + b_n + c_n$.
			De plus, $2^n = 0 \times Q_n(2) + R_n(2) = 4a_n + 2b_n + c_n$.
			Finalement, en dérivant, on trouve $n X^{n-1} + \mu_A' \cdot Q_n  + \mu_A \cdot Q_n' + R'_n$, et donc, pour $X = 2$, on trouve $n 2^{n-1} = 0 \cdot Q_n(2) + 0 \cdot Q'_n(2) + R_n'(2) = 4a_n + b_n$.
			On résout donc le système
			\[
				\begin{cases}
					a_n + b_n + c_n = 1\\
					4a_n + 2b_n + c_n = 2^n\\
					4a_n + b_n = n 2^{n-1}
				\end{cases}
			.\]
			On trouve donc que $a_n = n 2^{n-1} - 2^n + 1$, $b_n = -3n \cdot 2^{n-1} + 2^{n+2} - 4$ et $c_n = n 2^n - 3\cdot 2^n + 4$.
			On en déduit que $A^n = a_n A^2 + b_n A + c_n I_3$.
			Vérification : pour $n = 0$, on a $(a_n, b_n, c_n) = (0, 0, 1)$ ; pour $n = 1$, on a $(a_n, b_n, c_n) = (0, 1, 0)$ ; pour $n = 2$, on a $(a_n, b_n, c_n) = (1, 0, 0)$.
		\item Enfin, on calcule \[
				\exp A = \sum_{n=0}^\infty \frac{A^n}{n!} = \sum_{n=0}^\infty \frac{a_n A^2 + b_n A + c_n I_3}{n!} = \alpha A^2 + \beta A + \gamma I_3
			.\] On calcule $\gamma$, et on procèdera de même pour les autres coefficients.
			\begin{align*}
				\alpha &= \sum_{n=0}^\infty \frac{c_n}{n!} = \sum_{n=0}^\infty \frac{n 2^n - 3 \cdot 2^n + 4}{n!}\\
				&= \sum_{n=1}^\infty \frac{n_2^{n}}{n!} - 3 \sum_{n=0}^\infty \frac{2^n}{n!} + 4 \sum_{n=0 }^\infty \frac{1}{n!} \\
				&= 2\: \mathrm{e}^2 - 3\:\mathrm{e}^2  + 4 \mathrm{e} \\
			\end{align*}
			En effet, $x \sum_{n=1}^\infty n x^{n-1} / n! = x \mathrm{d}\,\mathrm{e}^x/\mathrm{d}x = x \mathrm{e}^x$, car on peut dériver terme à terme une série entière sans changer son rayon de convergence.
	\end{enumerate}
	\clearpage
	\centerline{\LARGE Cadeau du 28/03/23}
	\centerline{\itshape Comment utiliser la trigonalisation pour calculer l'exponentielle d'une matrice ?}

	\bigskip
	\bigskip
	\bigskip
	\textbf{Cadeau} :\\
	\begin{slshape}
		La matrice $B$ est-elle diagonalisable et calculer $\exp B$, où $B$ est la matrice $2 \times 2$ définie comme \[
			B = \begin{pmatrix}
				1 & 2\\
				0 & 1
			\end{pmatrix}
		.\]
	\end{slshape}

	\bigskip\bigskip
	\textbf{Réponse du cadeau} :\\
	\begin{enumerate}
		\item Non. En effet, son unique valeur propre sont $1$, et, si $B$ serait diagonalisable, alors $B$ serait semblable à $I_2$, ce qui est absurde car $B \neq I_2$.
		\item On remarque que $B^2 = {1\:4\choose 0\:1}$. On peut montrer par récurrence que \[
			B^n = \begin{pmatrix}
				1 & 2n\\
				0 & 1
			\end{pmatrix}
		.\]
		Ainsi, pour calculer $\exp B$, on procède pour chaque coefficient de la matrice.
		En effet, \[
			\mathrm{e}^B = \begin{pmatrix}
				\sum_{n=0}^\infty 1/n! & \sum_{n=0}^\infty 2n/n!\\
				0 &\sum_{n=0}^\infty 1/n!\\
			\end{pmatrix} = \begin{pmatrix}
				\mathrm{e} & 2 \mathrm{e}\\
				0 & \mathrm{e}
			\end{pmatrix}
		\] car \[
			\sum_{n=0}^\infty \frac{2n}{n!} = 2 \sum_{n=1}^\infty \frac{1}{(n-1)!} = 2 \mathrm{e}
		.\]
	\end{enumerate}

	\textbf{Autre réponse du cadeau} :\\
	On remarque que $B = I_2 + { 0 \: 2 \choose 0 \: 0 } = I_2 + A$.
	Et, comme $I_2$ et $A$ commutent, alors \[
		\mathrm{e}^B = \mathrm{e}^{I_2} \cdot \mathrm{e}^A = \sum_{n=0}^\infty \frac{I_n^n}{n!} \times \sum_{n=0}^\infty \frac{A^n}{n!} = \sum_{n=0}^\infty \frac{I_n}{n!} \cdot (I_n + A) = \mathrm{e} \cdot B
	\] car la matrice $A$ est nilpotente : $A^2$ est la matrice nulle.
	\clearpage
	\centerline{\LARGE Cadeau du 29/03/23}

	\bigskip
	\bigskip
	\bigskip
	\textbf{Cadeau} :\\
	\begin{slshape}
		Soit $A \in \mathcal{M}_{n,n}(\R)$.
		\begin{enumerate}
			\item Montrer que $A$ est antisymétrique si, et seulement si, $\forall X \in \mathcal{M}_{n,1}(\R)$, $X^\T \cdot A \cdot X = 0$.
			\item Que peut-on dire de $\mathrm{e}^A$ si $A$ est antisymétrique ?
		\end{enumerate}
	\end{slshape}

	\bigskip\bigskip
	\textbf{Réponse du cadeau} :\\
	\begin{enumerate}
		\item Le sens direct a déjà été démontré. Montrons la réciproque.
			On suppose $\forall X \in \mathcal{M}_{n,1}(\R)$, $X^\T AX = 0$.
			Soient $X$ et $Y$ deux vecteurs colonnes de $\mathcal{M}_{n,1}(\R)$.
			On a
			\begin{align*}
				0 &= (X+Y)^\T A (X+Y)\\
					&= X^\T A X + Y^\T A Y + X^\T A Y + Y^\T A X\\
					&= X^\T A Y + Y^\T A Y\\
					&= X^\T A Y + X^\T A^\T Y \text{ car } X^\T A^\T Y \in \R\\
					&= X^\T (A + A^\T) Y \\
			\end{align*}
			Ainsi, en notant $M = A + A^\T$, $\langle X  \mid MY\rangle = X^\T \cdot (MY) = 0$, pour tout $X$.
			D'où, $MY = 0$, et ce, quel que soit $Y$. D'où, $M = 0$.
			La matrice $A$ est donc antisymétrique : $A = - A^\T$.
		\item On sait que $(\exp A)^\T = \exp(A^T) = \exp(-A) = (\exp A)^{-1}$, d'où $\mathrm{e}^A \in \mathrm{O}_n(\R)$.
			De plus, $\det(\exp A) = \exp(\tr A) = 1$ et donc $\mathrm{e}^A \in \mathrm{SO}_n(\R)$.
	\end{enumerate}

	\bigskip\bigskip
	\textbf{Autre réponse du cadeau} :\\
	\begin{enumerate}
		\item On suppose que, pour tout $X \in \mathcal{M}_{n,1}(\R)$, $X^\T A X = 0 \in \R$.
			Donc $X^\T A^\T X = 0$, par application de la transposée.
			D'où, $\forall X \in \mathcal{M}_{n,1}(\R)$, $X^\T (A + A^\T) X = 0$ et donc $A + A^\T \in \mathcal{S}_n(\R)$.
			D'après le théorème spectral, il existe $P \in \mathrm{O}_n(\R)$ tel que $P^\T (A + A^\T) P = D$, où $D$ est diagonale.
			Ainsi, pour tout vecteur $X$, $X^\T P D P^\T = (P^\T X)^\T D (P^\T X) = 0$.
			D'où, par changement de base, pour tout vecteur $U$, $U^\T D U = 0$.
			Ainsi, $D = 0$ et donc $A + A^\T = 0$.
			On en déduit que $A^\T = -A$, la matrice $A$ est antisymétrique.
	\end{enumerate}
	\clearpage
	\centerline{\LARGE Cadeau du 30/03/23}
	\centerline{\itshape polynômes interpolateurs de \textsc{Lagrange}.}

	\bigskip
	\bigskip
	\bigskip
	\textbf{Cadeau} :\\
	\begin{slshape}
		Soit $p \in \N$. Pour tout entier non nul $n \in \N^*$\!, notons $P_n$ l'unique polynôme de $\R_p[X]$ vérifiant, pour tout $i \in \llbracket 0,p \rrbracket$, $P_n(i) = (-1)^n / (n + i)$.
		Montrer que la série de ``vecteurs'' $\sum_{n \ge 1} P_n$ est convergente.
	\end{slshape}

	\bigskip\bigskip
	\textbf{Réponse du cadeau} :\\
	Posons, pour tout $i \in \llbracket 1,p \rrbracket$, $L_i(X)$ le $i$-ème polynôme interpolateur de \textsc{Lagrange} : \[
		L_i(X) = \prod_{\substack{j = 0\\ j \neq i}}^p \frac{X-j}{i-j}
	.\]
	Ainsi, pour tout entier $n$, $P_n(X) = \sum_{i=1}^p (-1)^n L_i(X) / (n +i)$.
	Or, les polynômes interpolateurs de \textsc{Lagrange} forment une base de $\R_p[X]$. Ainsi, $\mathcal{B} = (L_0, \ldots, L_p)$ est une base de $\R_p[X]$.
	Or, $\R_p[X]$ est de dimension finie.
	Et, dans un espace vectoriel normé de dimension finie, une série de vecteurs converge si, et seulement si, les sommes partielles de chaque coefficients convergent.
	Pour tout $i \in \llbracket 1,p \rrbracket$, la série $\sum (-1)^k / (k + 1)$ converge.
	On en déduit que la série de polynômes $\sum P_n$ converge.

	\clearpage
	\centerline{\LARGE Cadeau du 31/03/23}
	\centerline{\itshape (Khôlle 21 \textinterrobang)}

	\bigskip
	\bigskip
	\bigskip
	\textbf{Cadeau A} :\\
	\begin{slshape}
		Soit, pour tout réel $x \in {]0,+\infty[}$, pour tout entier $n \in \N^*$, $f_n(x) = 1 / (n + n^2 x)$.
		\begin{enumerate}
			\item Montrer que la série de fonctions $\sum f_n$ converge simplement sur $]0,+\infty[$.
			\item Soit, pour $x \in {]0,+\infty[}$, $f(x) = \sum_{n=1}^\infty f_n(x)$.
				Montrer que $f$ est continue sur $]0,+\infty[$.
			\item[2{,}5.] Montrer que la convergence de la série $\sum f_n$ n'est pas normale.
			\item La convergence de la série de fonctions $\sum f_n$ est-elle uniforme sur $]0,+\infty[$ ?
			\item Déterminer un équivalent de $f(x)$ quand $x$ tend vers $+\infty$.
		\end{enumerate}
	\end{slshape}

	\bigskip

	\centerline{\pgfornament[width=3cm]{88}}

	\textbf{Cadeau B} :\\
	\begin{slshape}
		Soient $\alpha$ et $\beta$ deux scalaires d'un espace vectoriel $\mathds{K}$ de dimension finie.
		Soient $f$, $u$ et $v$ trois endomorphismes de $\mathds{K}$ tels que
		\[
			\begin{cases}
				f = \alpha u + \beta v,\\
				f^2 = \alpha^2 u + \beta^2 v,\\
				f^3 = \alpha^3 u + \beta^3 v.
			\end{cases}
		\]
		\begin{enumerate}
			\item Déterminer un polynôme annulateur de $f$.
			\item Montrer que $f$ est diagonalisable.
		\end{enumerate}
	\end{slshape}

	\bigskip

	\centerline{\pgfornament[width=3cm]{88}}

	\textbf{Cadeau C} :\\
	\begin{slshape}
		Soit $z \in \C$ tel que $|z| < 1$.
		Montrer que \[
			\sum_{n=1}^\infty \frac{z^{2n-1}}{1-z^{2n-1}} = \sum_{k=1}^\infty \frac{z^k}{1-z^{2k}}
		.\]
	\end{slshape}


	\bigskip\bigskip
	\textbf{Réponse du cadeau A} :\\
	\begin{enumerate}
		\item Soit $x \in {]0,+\infty[}$ : $1 / (n + n^2 x) \sim 1 / n^2 x$, qui ne change pas de signe.
			Or, la série numérique $\sum 1 / n^2 x = (1/x) \sum 1 / n^2$ converge par le critère de \textsc{Riemann}.
			D'où, la série de fonctions $\sum f_n$\/ converge simplement sur $]0,+\infty[$.
		\item Soit $a > 0$. Pour tout entier $n \in \N^*$, pour tout réel $x \in [a, +\infty[$, \[
				\big|f_n(x)\big| = \frac{1}{n + n^2 x} \le \frac{1}{n^2 a}
			,\] et $\sum 1/n^2 a$\/ converge par critère de \textsc{Riemann}.
			Ainsi, la série de fonctions $\sum f_n$\/ converge normalement sur $[a,+\infty[$, donc uniformément sur $[a,+\infty[$.
			Chaque fonction $f_n$ est continue sur $[a,+\infty[$, la fonction $f$ est donc continue sur $[a,+\infty[$.
			Ceci étant vrai pour tout $a$, et la continuité étant une propriété \textit{locale}, on en déduit que la fonction $f$ est continue sur $]0,+\infty[$.
		\item[2{,}5.] Pour tout entier $n \in \N$, $\sup_{x \in {]0,+\infty[}}\:|f_n(x)| = 1 / n$.
			Or, la série numérique $\sum 1 / n$ diverge, donc la série de fonctions $\sum f_n$ ne converge pas normalement sur $]0,+\infty[$.
		\item Supposons, par l'absurde, que la série de fonctions $\sum f_n$ converge uniformément sur $]0,+\infty[$.
			Pour tout $n \in \N^*$, $f_n(x) \to 1 / n$ quand $x \to 0$.
			D'après le théorème de la double limite, on en déduit que la série $\sum 1 / n$ converge, ce qui est absurde.
			D'où, $\sum f_n$ ne converge pas uniformément sur $]0,+\infty[$.
		\item Soit $n \in \N^*$, et $x > 1$. On encadre $f_n(x)$ : \[
				\frac{1}{n^2(x+1)} \le f_n(x) = \frac{1}{n + n^2 x} \le \frac{1}{n^2x}
			.\]D'où, par somme, pour tout entier $N \in \N^*$, \[
				\frac{1}{(x+1)}\sum_{n = 1}^N \frac{1}{n^2} \le \sum_{n=1}^N f_n(x) \le \frac{1}{x}\sum_{n = 1}^N \frac{1}{n^2}
			.\]Or, les inégalités larges passent à la limite, d'où \[
				\frac{\pi^2}{6(x+1)} \le \sum_{n=1}^\infty f_n(x) = f(x) \le \frac{\pi^2}{6x}
			.\]D'où, en divisant par $x + 1$, \[
				\frac{x}{x+1} \le \frac{6x}{\pi^2} f(x) \le 1
			.\]D'après le théorème des gendarmes, on en déduit que $\frac{6x}{\pi^2} f(x) \to 1$ quand $x \to \infty$ car $x / (x+1) \to 1$.
			D'où, $f(x) \sim_{x \to \infty} \pi^2 / 6x$.
	\end{enumerate}

	\bigskip\bigskip
	\textbf{Réponse du cadeau B} :\\
	\begin{enumerate}
		\item On calcule
			\[
				(\alpha + \beta) f^2 = \alpha^3 u + \beta^3 v + \alpha^2 \beta u + \alpha \beta^2 v = f^3 + \alpha \beta f
			.\]
			On en déduit que le polynôme $\mu(X) = X^3 - (\alpha + \beta) X^2 + \alpha \beta X$ annule $f$.
		\item On procède par disjonction de cas.
			\begin{enumerate}
				\item Si $\alpha \neq 0$, $\beta \neq 0$ et $\alpha \neq \beta$, alors $\mu$ est un polynôme scindé à racines simples annulateur de $f$. L'endomorphisme $f$ est donc diagonalisable.
				\item Si $\alpha = 0$ et $\beta = 0$, alors $f = 0$ qui est diagonal.
				\item Si $\alpha = 0$ et $\beta \neq 0$, alors $f = \beta v$ et $f^2 = \beta^2 v$.
					Ainsi, $f^2 - \beta f = 0$ et donc $X^2 - \beta X = X(X-\beta)$ est un polynôme scindé à racines simple annulateur de $f$. D'où, $f$ est diagonalisable.
				\item Si $\alpha \neq 0$ et $\beta = 0$, on procède comme le cas (c).
				\item Si $\alpha \neq 0$, $\beta \neq 0$, et $\alpha = \beta$, alors $f = \alpha (u+v)$ et $f^2 = \alpha^2 (u + v)$. D'où, $f^2 = \alpha f$.
					Ainsi, $X^2 - \alpha X = X(X - \alpha)$ est un polynôme annulateur de $f$, il est scindé à racines simples et donc $f$ est diagonalisable.
			\end{enumerate}
	\end{enumerate}

	\bigskip\bigskip
	\textbf{Réponse \textmd{(partielle)} du cadeau C} :\\
	\begin{align*}
		\sum_{n=1}^\infty \frac{z^{2n-1}}{1-z^{2n-1}}
		&= \sum_{n=1}^\infty z^{2n-1} \Big(\sum_{k=0}^\infty (z^{2n-1})^k\Big) && \text{ car } |z^{2n-1}| = |z|^{2n-1} < 1\\
		&= \sum_{n=1}^\infty \Big( \sum_{k=0}^\infty z^{2n-1} \cdot (z^{2n-1})^k\Big) \\
		&= \sum_{k=0}^\infty \Big(\sum_{n=1}^\infty z^{2n-1}(z^{2n-1})^k\Big) && \text{ car } (\heartsuit) \\
		&= \sum_{k=0}^\infty \sum_{n=1}^\infty (z^{2n-1})^{k+1} \\
		&= \sum_{k=0}^\infty \sum_{n=1}^\infty (z^{k+1})^{2n-1} \\
		&= \ldots \\
	\end{align*}
	$(\heartsuit)$ : la série converge absolument, c'est donc une famille sommable et on applique le théorème de Fubini.

	\clearpage
	\centerline{\LARGE Cadeau du 01/30423}
	\centerline{\itshape (Cadeau ultime)}

	\bigskip
	\bigskip
	\bigskip
	\textbf{Cadeau} :\\
	On considère l'ensemble $\mathcal{S}$ défini comme $\mathcal{S} = \{(x,y,z) \in \R^3  \mid \sqrt{x} + \sqrt{y} + \sqrt{z} = 1\}$.
	\begin{enumerate}
		\item Déterminer trois plans de symétries de la surface $\mathcal{S}$.
			Déterminer une rotation par laquelle $\mathcal{S}$ est stable.
		\item Montrer que, pour tout $(x,y,z) \in \mathcal{S}$, $x + y + z \le 1$.
			En déduire que la surface $\mathcal{S}$ est incluse dans un tétraèdre.
	\end{enumerate}

	\bigskip\bigskip
	\textbf{Réponse au cadeau} :\\
	Rotation de $2\pi / 3$ autour de l'axe orienté par $\vec{u} = (1, 1, 1)$, d'où les plans de symétries.
	Comme $0 \le x \le 1$, $0\le y \le 1$ et $0 \le z \le 1$ donc $x \le \sqrt{x}$, $y \le \sqrt{y}$ et $z \le \sqrt{z}$.
	D'où, par somme, $x + y + z \le 1$. Ainsi, pour tout vecteur $(x,y,z) \in \mathcal{S}$, $x \ge 0$, $y \ge 0$, $z \ge 0$ et $x + y + z \le 1$.
	On en déduit que $\mathcal{S}$ est incluse dans un tétraèdre.
\end{document}
