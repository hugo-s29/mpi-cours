\documentclass[a4paper]{article}

\usepackage[margin=1in]{geometry}
\usepackage[utf8]{inputenc}
\usepackage[T1]{fontenc}
\usepackage{mathrsfs}
\usepackage{textcomp}
\usepackage[french]{babel}
\usepackage{amsmath}
\usepackage{amssymb}
\usepackage{cancel}
\usepackage{frcursive}
\usepackage[inline]{asymptote}
\usepackage{tikz}
\usepackage[european,straightvoltages,europeanresistors]{circuitikz}
\usepackage{tikz-cd}
\usepackage{tkz-tab}
\usepackage[b]{esvect}
\usepackage[framemethod=TikZ]{mdframed}
\usepackage{centernot}
\usepackage{diagbox}
\usepackage{dsfont}
\usepackage{fancyhdr}
\usepackage{float}
\usepackage{graphicx}
\usepackage{listings}
\usepackage{multicol}
\usepackage{nicematrix}
\usepackage{pdflscape}
\usepackage{stmaryrd}
\usepackage{xfrac}
\usepackage{hep-math-font}
\usepackage{amsthm}
\usepackage{thmtools}
\usepackage{indentfirst}
\usepackage[framemethod=TikZ]{mdframed}
\usepackage{accents}
\usepackage{soulutf8}
\usepackage{mathtools}
\usepackage{bodegraph}
\usepackage{slashbox}
\usepackage{enumitem}
\usepackage{calligra}
\usepackage{cinzel}
\usepackage{BOONDOX-calo}

% Tikz
\usetikzlibrary{babel}
\usetikzlibrary{positioning}
\usetikzlibrary{calc}

% global settings
\frenchspacing
\reversemarginpar
\setuldepth{a}

%\everymath{\displaystyle}

\frenchbsetup{StandardLists=true}

\def\asydir{asy}

%\sisetup{exponent-product=\cdot,output-decimal-marker={,},separate-uncertainty,range-phrase=\;à\;,locale=FR}

\setlength{\parskip}{1em}

\theoremstyle{definition}

% Changing math
\let\emptyset\varnothing
\let\ge\geqslant
\let\le\leqslant
\let\preceq\preccurlyeq
\let\succeq\succcurlyeq
\let\ds\displaystyle
\let\ts\textstyle

\newcommand{\C}{\mathds{C}}
\newcommand{\R}{\mathds{R}}
\newcommand{\Z}{\mathds{Z}}
\newcommand{\N}{\mathds{N}}
\newcommand{\Q}{\mathds{Q}}

\renewcommand{\O}{\emptyset}

\newcommand\ubar[1]{\underaccent{\bar}{#1}}

\renewcommand\Re{\expandafter\mathfrak{Re}}
\renewcommand\Im{\expandafter\mathfrak{Im}}

\let\slantedpartial\partial
\DeclareRobustCommand{\partial}{\text{\rotatebox[origin=t]{20}{\scalebox{0.95}[1]{$\slantedpartial$}}}\hspace{-1pt}}

% merging two maths characters w/ \charfusion
\makeatletter
\def\moverlay{\mathpalette\mov@rlay}
\def\mov@rlay#1#2{\leavevmode\vtop{%
   \baselineskip\z@skip \lineskiplimit-\maxdimen
   \ialign{\hfil$\m@th#1##$\hfil\cr#2\crcr}}}
\newcommand{\charfusion}[3][\mathord]{
    #1{\ifx#1\mathop\vphantom{#2}\fi
        \mathpalette\mov@rlay{#2\cr#3}
      }
    \ifx#1\mathop\expandafter\displaylimits\fi}
\makeatother

% custom math commands
\newcommand{\T}{{\!\!\,\top}}
\newcommand{\avrt}[1]{\rotatebox{-90}{$#1$}}
\newcommand{\bigcupdot}{\charfusion[\mathop]{\bigcup}{\cdot}}
\newcommand{\cupdot}{\charfusion[\mathbin]{\cup}{\cdot}}
%\newcommand{\danger}{{\large\fontencoding{U}\fontfamily{futs}\selectfont\char 66\relax}\;}
\newcommand{\tendsto}[1]{\xrightarrow[#1]{}}
\newcommand{\vrt}[1]{\rotatebox{90}{$#1$}}
\newcommand{\tsup}[1]{\textsuperscript{\underline{#1}}}
\newcommand{\tsub}[1]{\textsubscript{#1}}

\renewcommand{\mod}[1]{~\left[ #1 \right]}
\renewcommand{\t}{{}^t\!}
\newcommand{\s}{\text{\calligra s}}

% custom units / constants
%\DeclareSIUnit{\litre}{\ell}
\let\hbar\hslash

% header / footer
\pagestyle{fancy}
\fancyhead{} \fancyfoot{}
\fancyfoot[C]{\thepage}

% fonts
\let\sc\scshape
\let\bf\bfseries
\let\it\itshape
\let\sl\slshape

% custom math operators
\let\th\relax
\let\det\relax
\DeclareMathOperator*{\codim}{codim}
\DeclareMathOperator*{\dom}{dom}
\DeclareMathOperator*{\gO}{O}
\DeclareMathOperator*{\po}{\text{\cursive o}}
\DeclareMathOperator*{\sgn}{sgn}
\DeclareMathOperator*{\simi}{\sim}
\DeclareMathOperator{\Arccos}{Arccos}
\DeclareMathOperator{\Arcsin}{Arcsin}
\DeclareMathOperator{\Arctan}{Arctan}
\DeclareMathOperator{\Argsh}{Argsh}
\DeclareMathOperator{\Arg}{Arg}
\DeclareMathOperator{\Aut}{Aut}
\DeclareMathOperator{\Card}{Card}
\DeclareMathOperator{\Cl}{\mathcal{C}\!\ell}
\DeclareMathOperator{\Cov}{Cov}
\DeclareMathOperator{\Ker}{Ker}
\DeclareMathOperator{\Mat}{Mat}
\DeclareMathOperator{\PGCD}{PGCD}
\DeclareMathOperator{\PPCM}{PPCM}
\DeclareMathOperator{\Supp}{Supp}
\DeclareMathOperator{\Vect}{Vect}
\DeclareMathOperator{\argmax}{argmax}
\DeclareMathOperator{\argmin}{argmin}
\DeclareMathOperator{\ch}{ch}
\DeclareMathOperator{\com}{com}
\DeclareMathOperator{\cotan}{cotan}
\DeclareMathOperator{\det}{det}
\DeclareMathOperator{\id}{id}
\DeclareMathOperator{\rg}{rg}
\DeclareMathOperator{\rk}{rk}
\DeclareMathOperator{\sh}{sh}
\DeclareMathOperator{\th}{th}
\DeclareMathOperator{\tr}{tr}

% colors and page style
\definecolor{truewhite}{HTML}{ffffff}
\definecolor{white}{HTML}{faf4ed}
\definecolor{trueblack}{HTML}{000000}
\definecolor{black}{HTML}{575279}
\definecolor{mauve}{HTML}{907aa9}
\definecolor{blue}{HTML}{286983}
\definecolor{red}{HTML}{d7827e}
\definecolor{yellow}{HTML}{ea9d34}
\definecolor{gray}{HTML}{9893a5}
\definecolor{grey}{HTML}{9893a5}
\definecolor{green}{HTML}{a0d971}

\pagecolor{white}
\color{black}

\begin{asydef}
	settings.prc = false;
	settings.render=0;

	white = rgb("faf4ed");
	black = rgb("575279");
	blue = rgb("286983");
	red = rgb("d7827e");
	yellow = rgb("f6c177");
	orange = rgb("ea9d34");
	gray = rgb("9893a5");
	grey = rgb("9893a5");
	deepcyan = rgb("56949f");
	pink = rgb("b4637a");
	magenta = rgb("eb6f92");
	green = rgb("a0d971");
	purple = rgb("907aa9");

	defaultpen(black + fontsize(8pt));

	import three;
	currentlight = nolight;
\end{asydef}

% theorems, proofs, ...

\mdfsetup{skipabove=1em,skipbelow=1em, innertopmargin=6pt, innerbottommargin=6pt,}

\declaretheoremstyle[
	headfont=\normalfont\itshape,
	numbered=no,
	postheadspace=\newline,
	headpunct={:},
	qed=\qedsymbol]{demstyle}

\declaretheorem[style=demstyle, name=Démonstration]{dem}

\newcommand\veczero{\kern-1.2pt\vec{\kern1.2pt 0}} % \vec{0} looks weird since the `0' isn't italicized

\makeatletter
\renewcommand{\title}[2]{
	\AtBeginDocument{
		\begin{titlepage}
			\begin{center}
				\vspace{10cm}
				{\Large \sc Chapitre #1}\\
				\vspace{1cm}
				{\Huge \calligra #2}\\
				\vfill
				Hugo {\sc Salou} MPI${}^{\star}$\\
				{\small Dernière mise à jour le \@date }
			\end{center}
		\end{titlepage}
	}
}

\newcommand{\titletp}[4]{
	\AtBeginDocument{
		\begin{titlepage}
			\begin{center}
				\vspace{10cm}
				{\Large \sc tp #1}\\
				\vspace{1cm}
				{\Huge \textsc{\textit{#2}}}\\
				\vfill
				{#3}\textit{MPI}${}^{\star}$\\
			\end{center}
		\end{titlepage}
	}
	\fancyfoot{}\fancyhead{}
	\fancyfoot[R]{#4 \textit{MPI}${}^{\star}$}
	\fancyhead[C]{{\sc tp #1} : #2}
	\fancyhead[R]{\thepage}
}

\newcommand{\titletd}[2]{
	\AtBeginDocument{
		\begin{titlepage}
			\begin{center}
				\vspace{10cm}
				{\Large \sc td #1}\\
				\vspace{1cm}
				{\Huge \calligra #2}\\
				\vfill
				Hugo {\sc Salou} MPI${}^{\star}$\\
				{\small Dernière mise à jour le \@date }
			\end{center}
		\end{titlepage}
	}
}
\makeatother

\newcommand{\sign}{
	\null
	\vfill
	\begin{center}
		{
			\fontfamily{ccr}\selectfont
			\textit{\textbf{\.{\"i}}}
		}
	\end{center}
	\vfill
	\null
}

\renewcommand{\thefootnote}{\emph{\alph{footnote}}}

% figure support
\usepackage{import}
\usepackage{xifthen}
\pdfminorversion=7
\usepackage{pdfpages}
\usepackage{transparent}
\newcommand{\incfig}[1]{%
	\def\svgwidth{\columnwidth}
	\import{./figures/}{#1.pdf_tex}
}

\pdfsuppresswarningpagegroup=1
\ctikzset{tripoles/european not symbol=circle}

\newcommand{\missingpart}{{\large\color{red} Il manque quelque chose ici\ldots}}

\usepackage{pgfornament}

\begin{document}
	\clearpage
	\centerline{\LARGE Cadeaux du 14/09/22}
	\bigskip
	\bigskip
	{\bf Cadeau 1} :\\
	\slshape
	Soit $A$\/ un anneau commutatif et soit $x \in A$. On dit que $x$\/ est {\it nilpotent}\/ (ou {\it nihilpotent}) si \[
		\exists n \in \N,\:x^n = 0_A
	.\]
	\begin{enumerate}
		\item Montrer que, si $x$\/ est nilpotent, alors $x$\/ n'est pas inversible mais $1_A - x$\/ est inversible.
		\item Montrer que l'ensemble des éléments nilpotents de $A$\/ est un idéal de $A$.
	\end{enumerate}
	\upshape
	\bigskip
	\bigskip

	{\bf Réponse du cadeau 1} :
	\begin{enumerate}
		\item On procède par l'absurde. On suppose que $x$\/ est nilpotent. Soit $n \in \N$\/ le plus petit possible tel que $x^n = 0_A$. On suppose qu'il existe $y \in A$\/ tel que $x\cdot y = 1_A$. D'où, $(xy)^n$\/ est, d'une part $x^n \cdot y^n = 0_A \cdot y^n = 0_A$\/ par commutativité, et d'autre part, $(xy)^n = {1_A}^{n} = 1_A \neq 0_A$. Ce qui est absurde.

			On suppose à présent $x \neq 1_A$. On sait que $A \ni \sum_{k=0}^{n-1} x^k = (1-x^n) / (1-x) = 1 / (1-x)$. On a donc trouvé l'inverse de $1-x$.
		\item Soit $x$\/ un élément nilpotent de $A$, et $y$\/ un élément de $A$. Soit $n \in \N$\/ tel que $x^n = 0_A$. $x\cdot y$\/ est aussi un élément nilpotent de $A$. En effet, $(xy)^n = x^n \cdot y^n = 0_A \cdot y^n = 0_A$. On nomme $\mathcal{I}$\/ l'ensemble des éléments nilpotents de $A$. Montrons que $(\mathcal{I},+)$\/ est un sous-groupe additif de $(A, +)$. On a bien $0 \in \mathcal{I}$\/ car $0^{k} = 0$.
			Soient $x$\/ et $y$\/ deux éléments nilpotents. Montrons que $x -y  \in \mathcal{I}$. Soient $n_1$\/ et $n_2 \in \N^*$\/ tels que $x^{n_1} = 0$\/ et $y^{n_2} = 0$. On veut montrer qu'il existe $n \in \N^*$\/ tel que $(x-y)^n = 0$. Soit $n = n_1 + n_2$. On a \[
				(x-y)^n = \sum_{k=0}^n (-1)^{n-k}{n \choose k} x^k y^{n-k}
				= \underbrace{\sum_{k=0}^{n_1} (-1)^{n-k}{n\choose k} x^y y^{n-k}}_{(1)} + \underbrace{\sum_{k=n_1 + 1}^{n_1 + n_2} (-1)^{n-k}{n\choose k} x^y y^{n-k}}_{(2)}
			.\] Or, dans la somme (1), $n-k = n_1 + n_2 - k = n_2 + (n_1 - k) \ge n_2$\/ et, dans la somme (2), $k \ge n_1$.
	\end{enumerate}

	\bigskip
	\bigskip
	
	\centerline{\pgfornament[width=3cm]{88}}

	\bigskip
	\bigskip
	
	{\bf Cadeau 2} :\\
	\slshape
	Soit $F$\/ l'ensemble des matrices de la forme ${x\hfil y\choose -5y\quad x+4y}$\/ où $(x, y) \in \R^2$. On note $J = {-2\quad1\choose -5\quad2}$.
	\begin{enumerate}
		\item Montrer que $F$\/ est un sous-espace vectoriel de $\mathscr{M}_{2,2}(\R)$\/ et que $(I_2, J)$\/ est une base de $F$.
		\item Calculer $J^2$\/ puis $(x\:I_2 + y J)\cdot (x'\:I + y'\:J)$\/ pour tout $(x,y,x',y') \in \R^4$. Qu'en déduire ?
	\end{enumerate}

	\upshape
	\bigskip
	\bigskip

	{\bf Réponse du cadeau 2} :
	\begin{enumerate}
		\item On cherche à trouver $\alpha, \beta \in \R^2$\/ tels que ${x\hfil y\choose -5\quad x+4y} = \alpha I_2 + \beta J$. On a
			\begin{align*}
				\alpha I_2 + \beta J \iff& \begin{pmatrix}
					x &y\\
					-5y&x+4y
				\end{pmatrix} = \begin{pmatrix}
					\alpha & 0\\
					0&\alpha
				\end{pmatrix} + \begin{pmatrix}
					-2\beta & \beta\\
					-5\beta& 2\beta
				\end{pmatrix}\\
				\iff&
				\begin{cases}
					x = \alpha - 2\beta\\
					y = \beta\\
					-5\beta = -5y\\
					2\beta + \alpha = x+ y
				\end{cases}\\
				\iff& \begin{cases}
					\beta = y\\
					\alpha = x + 2y\\
				\end{cases}
			\end{align*}
		\item On a $J^2 = -I_2$\/ et, en calculant minutieusement, on trouve, pour tout $(x,y,x',y') \in \R^4$, $(x\,I_2+ y\,J)\cdot (x'\,I_2 + y'\,J) = \cdots = (xx' - yy')\,I_2 + (x'y + xy')\,J$.
			On remarque que $(F, +, \cdot)$\/ est isomorphe à $(\C, +, \times)$. C'est un isomorphisme d'anneaux. Or, comme l'anneau $(\C, +, \times)$\/ est un corps donc $F$\/ l'est aussi.
	\end{enumerate}
	\clearpage
	\centerline{\LARGE Cadeau du 19/09/22}
	\bigskip
	\bigskip
	{\bf Cadeau} :\\
	\slshape
	Soit $(\vec\imath, \vec\jmath, \vec k)$\/ une base orthonormée de $\R^3$. On pose $f: \R^3 \to \R^3$\/ un endomorphisme défini tel que \[
		A = \begin{bNiceMatrix}[last-row,last-col]
			0&0&1&\vec\imath\\
			1&0&0&\vec\jmath\\
			0&1&0&\vec k\\
			f(\vec\imath)&f(\vec\jmath)&f(\vec k)\\
		\end{bNiceMatrix} = \big[f\big]_{(\vec\imath,\vec\jmath,\vec k)}
	.\]
	Interpréter géométriquement $f$.

	\upshape
	\bigskip
	\bigskip

	{\bf Réponse du cadeau} :\\
	Soit $\mathscr{B}$\/ une base et $A = [f]_\mathscr{B}$, alors $f(\vec\imath) = \vec\jmath$, $f(\vec\jmath) = \vec k$, et $f(\vec k) = \vec\imath$.
	$f$\/ est la rotation d'angle $\sfrac{2\pi}3$\/ autour de $\Vect(\vec \imath + \vec \jmath + \vec k)$.

	On peut également le montrer en décomposant $f = g  \circ h$, où $g$\/ est la symétrie par rapport à $\Vect(\vec{\imath}+\vec{\jmath},\vec{k})$\/ et parallèlement à $\Vect(\vec{\imath},\vec{\jmath})$\/ ; et $h$\/ la symétrie par rapport à $\Vect(\vec{\imath} + \vec{k},\vec{\jmath})$\/ parallèlement à $\Vect(\vec{\imath},\vec{k})$.

	\clearpage
	\centerline{\LARGE Cadeaux du 22/09/22}
	\bigskip
	\bigskip
	{\bf Cadeau 1} :\\
	\slshape
	Soit $(u_n)_{n\in\N}$\/ une suite positive, telle que la suite $S_n = \sum_{k=0}^n u_k$\/ diverge. En calculant $\ln \frac{S_n}{S_{n-1}}$, montrer que la série $\sum \frac{u_n}{S_n}$\/ diverge.
	\upshape

	\bigskip
	\bigskip
	
	\centerline{\pgfornament[width=3cm]{88}}

	\bigskip
	\bigskip
	
	{\bf Cadeau 2} :\\
	\slshape
	On pose \[
		D(x) = 
		\begin{vmatrix}
			7-x&14-x&3-x\\
			8-x&2-x&-x\\
			13-x&-1-x&2-x
		\end{vmatrix}
	.\]
	Montrer qu'il existe deux réels $\alpha$\/ et $\beta$\/ tels que, pour tout $x \in \R$, $D(x) = \alpha x + \beta$. Déterminer $\alpha$\/ et $\beta$.
	\clearpage
	\centerline{\LARGE Cadeau du 23/09/22}
	\bigskip
	\bigskip
	{\bf Cadeau} :\\
	\slshape
	Montrer qu'il n'existe pas $P \in \mathrm{GL}_2(\R)$\/ telle que, $A' = P^{-1}AP$\/ où \[
		A = \begin{bmatrix} 0&7\\0&0 \end{bmatrix}\qquad\text{et}\qquad
		A' = \begin{bmatrix} \lambda&0\\0&\mu \end{bmatrix}
	.\]
	\upshape
	\bigskip
	\bigskip

	{\bf Réponse du cadeau} :\\
	\begin{itemize}
		\item[{\sc Analyse}] On suppose $P^{-1} A P = {\lambda\:0\choose 0\:\mu}$. Alors, $\det A = \det A'$, d'où $0 = \lambda\cdot\mu$.
			Et, $\tr A = \tr A'$, d'où $\lambda + \mu = 0$.
			On en déduit donc que $\lambda = 0 = \mu$.
		\item[{\sc Synthèse}] On a \[
				P^{-1} A P = \begin{bmatrix} 0&0\\0&0 \end{bmatrix}
			.\] D'où, en multipliant à gauche par $P$\/ et à droite par $P^{-1}$, on a \[
				A = \begin{bmatrix} 0&0\\0&0 \end{bmatrix} 
			.\]
	\end{itemize}
	On en conclut que la matrice $A$\/ n'est pas diagonalisable.
	\clearpage
	\centerline{\LARGE Cadeaux du 28/09/22}
	\bigskip
	\bigskip
	{\bf Cadeau 1} :\\
	\slshape
	On considère la matrice \[
		M =
		\begin{pNiceMatrix}[last-row,last-col]
			0&0&1&\vec{\imath}\\
			1&0&0&\vec{\jmath}\\
			0&1&0&\vec{k}\\
			f(\vec{\imath})&f(\vec{\jmath})&f(\vec{k})
		\end{pNiceMatrix} = \Big[\:f\:\Big]_{(\vec{\imath},\vec{\jmath},\vec{k})}
	.\]
	Trouver et interpréter un vecteur propre et une valeur propre de $M$\/ (et, de même, de $f$).\\
	\upshape
	\bigskip
	\bigskip

	{\bfseries Indication :}\\
	On a $M \left( \substack{1\\1\\1} \right) = 1 \times \left( \substack{1\\1\\1} \right)$. Le vecteur $\vec{\imath}+\vec{\jmath}+\vec{k}$\/ est un vecteur directeur de l'axe de rotation. Montrer que les seuls vecteurs propres sont colinéaire au vecteur $\left( \substack{1\\1\\1} \right)$.
	\clearpage
	\centerline{\LARGE Cadeaux du 06/10/22}
	\bigskip
	\bigskip
	{\bf Cadeau} :\\
	\slshape
	On considère la matrice \[
		A = \begin{pmatrix}
			0&1&1\\
			-1&1&1\\
			-1&1&2
		\end{pmatrix}
	.\]
	\begin{enumerate}
		\item Quel est le spectre de la matrice $A$\/ ?
		\item Déterminer une base de chaque sous-espace propre de la matrice $A$.
		\item Montrer que la matrice $A$\/ est trigonalisable mais pas diagonalisable.
		\item Soit $T$\/ la matrice ci-dessous. Déterminer une matrice $P$\/ telle que $P^{-1} \cdot A \cdot P = T$ : \[
				T = \begin{pmatrix}
					1&1&0\\
					0&1&1\\
					0&0&1
				\end{pmatrix}
			\]
		\item Résoudre sur $\R$\/ le système d'équation différentielle $(\Sigma)$\/ ci-dessous : \[
				(\Sigma) : \begin{cases}
					x'(t) = y(t) + z(t)\\
					y'(t) = -x(t) + y(t) + z(t)\\
					z'(t) = -x(t) + y(t) + 2z(t)
				\end{cases}
			\]
	\end{enumerate}
	\upshape
	\bigskip
	\bigskip
	{\bf Réponse au cadeau 1:}\\
	\begin{enumerate}
		\item On calcule
			\begin{align*}
				\chi_A(x) = \det(xI_3 - A) &=
				\begin{vmatrix}
					x & -1 & -1\\
					1 & x - 1 & -1\\
					1 & -1 & x - 2
				\end{vmatrix}\\
				&= x
				\begin{vmatrix}
					x-1&-1\\
					-1&x-2
				\end{vmatrix} -
				\begin{vmatrix}
					-1&-1\\
					-1&x-2
				\end{vmatrix} +
				\begin{vmatrix}
					-1&-1\\
					x-1&-1
				\end{vmatrix}\\
				&= x\big((x-1)(x-2) - 1\big) - (2-x) - 1  + (-1 + x - 1) \\
				&= x(x^2 -3x + 1) -2 + 2x \\
				&= x^3 - 3x^2 + 3x - 1 \\
				&= (x-1)^3 \\
			\end{align*}
			On en déduit que \[
				\boxed{\Sp(A) = \{1\}.}
			\]
		\item On cherche une base de $\mathrm{SEP}(1)$\/ : on cherche $X = \left( \substack{x\\y\\z} \right) \in \mathscr{M}_{3,1}(\R)$, tel que $AX = X$.
			\begin{align*}
				\begin{pmatrix}
					0&1&1\\
					-1&1&1\\
					-1&1&2
				\end{pmatrix} \begin{pmatrix}
					x\\y\\z
				\end{pmatrix} = \begin{pmatrix}
					x\\y\\z
				\end{pmatrix} \iff& \begin{cases}
					y + z = x\\
					-x + y + z = y\\
					-x + y + 2z = z
				\end{cases}\\
				\iff& \begin{cases}
					y = 0\\
					x = z\\
				\end{cases}\\
				\iff& X = x \begin{pmatrix}
					1\\
					0\\
					1
				\end{pmatrix}\\
				\iff& X \in \Vect\begin{pmatrix}
					1\\0\\1
				\end{pmatrix}
			\end{align*}
			Ainsi, la base $\mathscr{B} = \left( \left( \substack{1\\0\\1} \right)\right)$.
		\item La matrice $A$\/ n'est pas diagonalisable : en effet, on a $\dim \R^3 = 3 \neq \dim(\mathrm{SEP}(1)) = 1$. Mais, le polynôme $\chi_A$\/ est scindé donc la matrice $A$\/ est trigonalisable.
		\item
			On cherche $P \in \mathrm{GL}_n(\R)$\/ tel que \[
				P^{-1} \cdot A \cdot P = T = 
				\begin{pNiceMatrix}[last-row,last-col]
					1&1&0&\varepsilon_1\\
					0&1&1&\varepsilon_2\\
					0&0&1&\varepsilon_3\\
					f(\varepsilon_1)&f(\varepsilon_2)&f(\varepsilon_3)
				\end{pNiceMatrix}
			.\]
			On cherche donc $(\varepsilon_1, \varepsilon_2, \varepsilon_3)$\/ une base de $\R^3$\/ tel que \[
				\begin{cases}
					f(\varepsilon_1) = \varepsilon_1 \qquad\qquad&(1)\\
					f(\varepsilon_2) = \varepsilon_1+\varepsilon_2 \qquad\qquad&(2)\\
					f(\varepsilon_3) = \varepsilon_2 + \varepsilon_3 \qquad\qquad&(3).
				\end{cases}
			\]
			On choisit $\varepsilon_1 = \left( \substack{1\\0\\1} \right)$, d'après la question 2.
			Puis, on calcule
			\begin{align*}
				\underbrace{\begin{pmatrix}
					0&1&1\\
					-1&1&1\\
					-1&1&2
				\end{pmatrix}}_{\smash{A}} \begin{pmatrix}
					x\\y\\z
				\end{pmatrix} = \begin{pmatrix}
					1\\0\\1
				\end{pmatrix} + \begin{pmatrix}
					x\\y\\z
				\end{pmatrix} \iff& \begin{cases}
					-x + y + z = 1\\
					-x+z=0\\
					-x + y +z = 1
				\end{cases}\\
				\iff& \begin{cases}
					-x+y+z=1\\
					-x+z = 0
				\end{cases}\\
				\iff& \begin{cases}
					x = z\\
					y = 1
				\end{cases}\\
				\iff& \varepsilon_2 = \begin{pmatrix}
					x\\1\\x
				\end{pmatrix} = \begin{pmatrix}
					0&1&0
				\end{pmatrix} + x \begin{pmatrix}
					1\\0\\1
				\end{pmatrix}
			\end{align*}
			On choisit donc $\varepsilon_2 = \left( \substack{0\\1\\0} \right)$\/ (en choisissant $x = 0$). De même, on choisit $\varepsilon_3 = \left( \substack{0\\-1\\1} \right)$.
			Donc, si \[
				P =
				\begin{pNiceMatrix}[last-row,last-col]
					1&0&0&e_1\\
					0&1&-1&e_2\\
					1&0&1&e_3\\
					\varepsilon_1&\varepsilon_2&\varepsilon_3
				\end{pNiceMatrix}
			\] alors $P^{-1} \cdot A \cdot P$, et on a bien $\det P = 1$.
		\item
			\begin{align*}
				(\Sigma) \iff& X'(t) = A\cdot X(t) \text{ où } X(t) \begin{pmatrix}
					x(t)\\y(t)\\z(t)
				\end{pmatrix}\\
				\iff& U'(t) = T\cdot U(t) \text{ où } U(t) = \begin{pmatrix}
						u(t)\\v(t)\\w(t)
					\end{pmatrix} = P^{-1}\cdot X(t)\\
				\iff& \begin{cases}
					u'(t) = u(t) + v(t)&\qquad(1)\\
					v'(t) = v(t) + w(t)&\qquad(2)\\
					w'(t) = w(t) &\qquad(3)
				\end{cases}
			\end{align*}
			D'où
			\[
				(3) \iff \exists K \in \R,\:\forall t \in \R,\;w(t) = K\mathrm{e}^{t},
			\] et \[
				(2) \iff v'(t) - v(t) = K \mathrm{e}^{t}
			.\]
			On résout l'équation homogène associé : \[
				v'(t) = v(t) \iff \exists L \in \R,\:\forall t \in \R,\:v(t) = L\mathrm{e}^{t}
			.\]
			On pose $v(t) = \ell(t)\:\mathrm{e}^{t}$, et donc
			\begin{align*}
				(2) \iff& \ell'(t)\:\mathrm{e}^{t} + \ell(t)\:\mathrm{e}^{t} - \ell(t)\:\mathrm{e}^{t} = K\mathrm{e}^t\\
				\iff& \ell'(t) = K\\
				\iff& \ell(t) = K\,t + L\\
				\iff& v(t) = (K t + L)\: \mathrm{e}^{t}
			\end{align*}
			Et donc
			\begin{align*}
				(1) \iff& u'(t) = u(t) + v(t) = u(t) + (Kt + L)\:\mathrm{e}^{t}\\
				\iff& u(t) - u(t) = (Kt + L)\:\mathrm{e}^{t}\\
			\end{align*}
			On résout l'équation homogène associée : \[
				u'(t) - u(t) = 0 \iff \exists M \in \R,\:\forall t \in \R\:u(t) = M \mathrm{e}^{t}
			.\]
			On pose $u(t) = m(t)\:\mathrm{e}^{t}$.
			\begin{align*}
				(2) \iff& m'(t)\:\mathrm{e}^{t} - m(t) = (Kt + L)\:\mathrm{e}^{t}\\
				\iff& m'(t) = K t + L\\
				\iff& m(t) = \frac{1}{2} Kt^2 + Lt + M\\
				\iff& u(t) = \left( \frac{1}{2} Kt^2 + L t + M \right)\:\mathrm{e}^{t}
			\end{align*}
			Ainsi, \[
				(\Sigma) \iff \exists (K,L,M) \in \R^3,\:\forall t \in \R,\: \begin{cases}
					u(t) = \left( \frac{1}{2} Kt^2 + Lt + M \right)\:\mathrm{e}^{t}\\
					v(t) = (Kt + L)\:\mathrm{e}^{t}\\
					w(t) = K\:\mathrm{e}^{t}.
				\end{cases}
			\] Or, $X(t) = P \cdot U(t)$, et donc
			\begin{align*}
				(2) \iff& \exists (K,L,M) \in \R^3,\:\forall t \in \R,\: \begin{cases}
					x(t) = u(t)\\
					y(t) = v(t) - w(t)\\
					z(t) = u(t) + w(t)
				\end{cases}\\
				\iff& \exists (K,L,M) \in \R^3,\:\forall t \in \R,\: \begin{cases}
					x(t) = \left( \frac{1}{2} K t^2 + LT + M \right) \mathrm{e}^{t}\\
					y(t) = \left( Kt + L - K \right)\mathrm{e}^{t}\\
					z(t) = \left( \frac{1}{2}Kt^2 + Lt + M + K \right)\mathrm{e}^{t}
				\end{cases}
			\end{align*}
	\end{enumerate}
	\bigskip
	\bigskip
	{\bf Cadeau 2} (matrices stochastiques) :\\
	\slshape
	Soit une matrice carrée $A = (a_{i,j}) \in \mathscr{M}_n(\R)$\/ telle que \[
		\forall i,j \in \left\llbracket 1,n \right\rrbracket,\:a_{i,j} \ge 0\qquad\text{ et }\qquad \forall i \in \left\llbracket 1,n \right\rrbracket,\:\sum_{j=1}^n a_{i,j} = 1
	.\]
	\begin{enumerate}
		\item Montrer que $1 \in \Sp(A)$.
		\item Montrer que, si $\lambda$\/ est une valeur propre de $A$, alors $|\lambda| \le 1$.
	\end{enumerate}
	\upshape

	\bigskip
	\bigskip
	{\bf Cadeau 3} (matrices à diagonale strictement dominante) :\\
	\slshape
	Soit $A = (a_{i,j}) \in \mathscr{M}_{n}(\R)$\/ telle que \[
		\forall  i \in \left\llbracket 1,n \right\rrbracket,\:|a_{i,j}| > \sum_{j \neq i} |a_{i,j}|
	.\] Montrer que $A$\/ est inversible.
	\upshape
	\clearpage
	\centerline{\LARGE Cadeau du 12/10/22}
	\bigskip
	\bigskip
	{\bf Cadeau} :\\
	\slshape
	Soit $A \in \mathscr{M}_2(\R)$ tell que $A^2 = -I_2$. Montrer qu'il existe $P$\/ une matrice inversible telle que \[
		P^{-1}\cdot A\cdot P = \begin{pmatrix}
			0&-1\\
			1&0
		\end{pmatrix} = M
	.\]
	\upshape
	\bigskip
	\bigskip

	{\bf Réponse au cadeau}\/\\
	\begin{comment}
		Le polynôme $Q(X) = X^2 + 1 = (X-i)(X+i)$\/ est annulateur de la matrice~$A$. D'où~$\Sp_\C (A) \subset \{-i,i\}$. Le polynôme~$Q$\/ est scindé à racines simples dans~$\C[X]$, d'où la matrice est diagonalisable dans~$\mathscr{M}_2(\C)$. Or, le spectre est stable par conjugaison car~$A \in \mathscr{M}_2(\R)$, d'où~$\Sp_\C(A) = \{-i,i\}$. De même, on remarque~$M^2 = I_2$, et donc tout ce qui est dit précédemment est aussi vrai pour cette matrice.
		Ainsi, il existe une matrice inversible $P \in \mathrm{GL}_2(\C)$\/ telle que \[
			P^{-1}\cdot A\cdot P = \begin{pmatrix}
				i&0\\
				0&-i
			\end{pmatrix}\qquad\text{et}\qquad \begin{pmatrix}
				0&-1\\
				1&0
			\end{pmatrix} \sim \begin{pmatrix}
				i&0\\
				0&-i
			\end{pmatrix}
		.\]
		D'où $A$\/ et ${0\hfill\:-1\choose1\hfill\:0}$\/ sont semblables dans $\mathscr{M}_2(\C)$.
	\end{comment}

	On remarque que $A \sim {i\hfill\:0\choose0\hfill\:-i}$. D'où, il existe $\varepsilon \in \mathscr{M}_{2,1}(\C)$, tel que $A\cdot \varepsilon = i \varepsilon$. Également, $A \cdot \bar\varepsilon = -i \bar\varepsilon$. Soient $U = \varepsilon + \bar\varepsilon \in \mathscr{M}_{2,1}(\R)$, et $V = i(\varepsilon + \bar\varepsilon) \in \mathscr{M}_{2,1}(\R)$, puis on calcule
	\begin{align*}
		A \cdot U &= A\cdot \varepsilon + A\cdot \bar\varepsilon \\
		&= i \varepsilon - i\bar\varepsilon \\
		&= i(\varepsilon - \bar\varepsilon) \\
		&= V.
	\end{align*}
	\clearpage
	\centerline{\LARGE Cadeau du 19/10/22}
	\bigskip
	\bigskip
	{\bf Cadeau} :\\
	\slshape
	Soit $f$\/ une fonction continue sur $[0, +\infty[$, telle que $\ds\int_{0}^{+\infty} f(t)~\mathrm{d}t$\/ converge. 
	\begin{enumerate}
		\item Montrer que ça n'implique pas que $f(x) \tendsto{x\to +\infty} 0$.
		\item Montrer que, si $f(x) \tendsto{x\to +\infty} \ell \in \R$, alors $\ell = 0$.
		\item Montrer que, si $f$\/ est uniformément continue, alors $f(x) \tendsto{x\to +\infty} 0$.
	\end{enumerate}
	\upshape
	\bigskip
	\bigskip

	{\bf Réponse du cadeau} :
	\begin{enumerate}
		\item c.f.\ remarque 7 du cours
		\item Quitte à remplacer $\ell$\/ par $-\ell$, on suppose $\ell > 0$. Ainsi, il existe $X \ge 0$\/ tel que \[\forall x \ge X,\: f(x) \ge \frac{\ell}{2}.\] Or, l'intégrale \[\int_{X}^{+\infty} \frac{\ell}{2}~\mathrm{d}x\] diverge. D'où \[\int_{X}^{+\infty} f(x)~\mathrm{d}x\] diverge également. Ce qui est absurde.
			On en déduit que $\ell = 0$.
		\item On suppose $f$\/ uniformément continue. Par l'absurde, supposons que $f(x) \centernot{\tendsto{x\to +\infty}} 0$, d'où \[
				\exists \varepsilon > 0,\:\exists (u_n)_{n \in \N} \text{ tendant vers } +\infty,\:\forall n \in \N,\:f(u_n) \ge \varepsilon
			.\] Or, comme $f$\/ est uniformément continue, il existe $\delta > 0$\/ tel que \[
				\forall n \in \N,\:\forall x \in \R^+,\quad |x-u_n| \le \delta \implies |f(x) - f(y)| \le \frac{\varepsilon}{2}
			.\] D'où, \[
			\int_{0}^{+\infty} f(x)~\mathrm{d}x \ge \int_{0}^{+\infty} \frac{\varepsilon}{2}~\mathrm{d}t
			\] qui diverge. Ce qui est absurde.
	\end{enumerate}
	\clearpage
	\centerline{\LARGE Cadeau du 21/10/22}

	\centerline{\textit{Théorème d'interpolation de Lagrange}}
	\bigskip
	\bigskip
	{\bf Cadeau} :\\
	\slshape
	Soient $(a_0, a_1, \ldots, a_n)$ une suite de $n+1$\/ réels distincts deux à deux.
	Soient aussi $(b_0, b_1, \ldots, b_n)$\/ une suite de $n+1$\/ réels (qui peuvent être égales).
	Alors, \[
		\exists ! P \in \R_n[X],\:\forall k \in \left\llbracket 0,n \right\rrbracket,\:P(a_k) = b_k
	.\]
	\upshape
	\bigskip
	\bigskip

	{\bf Réponse du cadeau} :\\
	\begin{itemize}
		\item[\sc Méthode 1] Soient $n+1$\/ réels $a_0$, $a_1$, \ldots, $a_n$\/ distincts deux à deux. L'application \begin{align*}
				f: \R_n[X] &\longrightarrow \R^{n+1} \\
				P &\longmapsto \Big(P(a_0), P(a_1), \ldots, P(a_n)\Big)
			\end{align*}
			est linéaire et la dimension de l'espace vectoriel de départ est égale à la dimension de l'espace vectoriel d'arrivée. Soit $P$\/ un polynôme réel de degré au plus $n$.
			\begin{align*}
				P \in \Ker f \iff& f(P) = (0, 0, \ldots, 0)\\
				\implies& P(a_1) = P(a_2) = \cdots = P(a_n) = 0\\
				\implies& P \text{ a au moins } n + 1 \text{ racines}\\
				\implies& P = 0_{\R_n[X]} \text{ car } \# \text{racines} > \deg(P)
			\end{align*}
			D'où $\Ker f = \{0_{\R_n[X]}\}$.
			On en déduit que $f$\/ est injective. Et, d'après le théorème du rang, $f$\/ est surjective (car $\dim \R_n[X] = \dim \R^{n+1}$).
		\item[\sc Méthode 2] On reprend la fonction $f$\/ de la {\sc méthode 1}.
			Soit $\mathscr{B}$\/ la base canonique de $\R_n[X]$\/ : $\mathscr{B} = (1, X, \ldots, X^n)$\/ ; et, soit $\mathscr{C}$\/ la base canonique de $\R^{n+1}$\/ : $\mathscr{C} = (e_1, e_2, \ldots, e_{n+1})$.
			On a \[
				\big[f\big]_\mathscr{B}^{\mathscr{C}} = \Mat_{\mathscr{B},\mathscr{C}}(f) =
				\begin{pNiceMatrix}[last-row,last-col]
					1 & a_0 & a_0^n & \cdots & a_0^n & e_1\\
					1 & a_1 & a_1^n & \cdots & a_1^n & e_2\\
					1 & a_2 & a_2^n & \cdots & a_2^n & e_3\\
					\vdots & \vdots & \vdots & \ddots & \vdots & \vdots\\
					1 & a_n & a_n^2 & \cdots & a_n^n & e_{n+1}\\
					f(1) & f(X) & f(X^2) & \ldots & f(X^n)
				\end{pNiceMatrix} = V
			.\]
			On reconnaît un déterminant de {\sc Vandermonde}\/ :
			\begin{align*}
				\det V &= (a_n - a_{n-1}) \cdots (a_n - a_1)(a_n - a_0)\\
							 &\times (a_{n-1} - a_{n-2}) \cdots (a_{n-1} - a_{n-2}) \cdots (a_{n-1} - a_0)\\
							 &\times\\
							 &\:\vdots\\
							 &\times (a_2 - a_1) \cdot (a_2 - a_0)\\
							 &\times (a_1 - a_0)\\
							 &= \prod_{i>j} (a_i - a_j)\\
			\end{align*}
			D'où $\det V \neq 0$\/ car les $(a_i)$\/ sont distincts deux à deux. Donc $V$\/ est inversible, et d'où $f$\/ est bijective.
		\item[\sc Méthode 3]
			On va prouver la surjectivité en déterminant {\bf un}\/ polynôme $P$\/ tel que $P(a_0) = b_0$, $P(a_1) = b_1$, \ldots, $P(a_n) = b_n$. Le voilà :
			\begin{align*}
				P &= b_0 \frac{(X-a_1)(X-a_2) \cdots (X- a_n)}{(a_0 - a_1)(a_0 - a_2) \cdots (a_0 - a_n)}\\
				&+ b_1 \frac{(X-a_0)(X-a_1) \cdots (X-a_n)}{(a_1 - a_0)(a_1 - a_2) \cdots (a_1 - a_n)} \\
				&\:\vdots \\
				&+ b_n \frac{(X-a_0)(X-a_1)\cdots(X-a_{n-1})}{(a_n - a_0)(a_n - a_1) \cdots (a_n - a_{n-1})}. \\
			\end{align*}
			Ce polynôme interpole les $n+1$\/ points et $\deg P \le n$.
	\end{itemize}
\end{document}
