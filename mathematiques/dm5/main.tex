\documentclass[a4paper]{article}

\usepackage[margin=1in]{geometry}
\usepackage[utf8]{inputenc}
\usepackage[T1]{fontenc}
\usepackage{mathrsfs}
\usepackage{textcomp}
\usepackage[french]{babel}
\usepackage{amsmath}
\usepackage{amssymb}
\usepackage{cancel}
\usepackage{frcursive}
\usepackage[inline]{asymptote}
\usepackage{tikz}
\usepackage[european,straightvoltages,europeanresistors]{circuitikz}
\usepackage{tikz-cd}
\usepackage{tkz-tab}
\usepackage[b]{esvect}
\usepackage[framemethod=TikZ]{mdframed}
\usepackage{centernot}
\usepackage{diagbox}
\usepackage{dsfont}
\usepackage{fancyhdr}
\usepackage{float}
\usepackage{graphicx}
\usepackage{listings}
\usepackage{multicol}
\usepackage{nicematrix}
\usepackage{pdflscape}
\usepackage{stmaryrd}
\usepackage{xfrac}
\usepackage{hep-math-font}
\usepackage{amsthm}
\usepackage{thmtools}
\usepackage{indentfirst}
\usepackage[framemethod=TikZ]{mdframed}
\usepackage{accents}
\usepackage{soulutf8}
\usepackage{mathtools}
\usepackage{bodegraph}
\usepackage{slashbox}
\usepackage{enumitem}
\usepackage{calligra}
\usepackage{cinzel}
\usepackage{BOONDOX-calo}

% Tikz
\usetikzlibrary{babel}
\usetikzlibrary{positioning}
\usetikzlibrary{calc}

% global settings
\frenchspacing
\reversemarginpar
\setuldepth{a}

%\everymath{\displaystyle}

\frenchbsetup{StandardLists=true}

\def\asydir{asy}

%\sisetup{exponent-product=\cdot,output-decimal-marker={,},separate-uncertainty,range-phrase=\;à\;,locale=FR}

\setlength{\parskip}{1em}

\theoremstyle{definition}

% Changing math
\let\emptyset\varnothing
\let\ge\geqslant
\let\le\leqslant
\let\preceq\preccurlyeq
\let\succeq\succcurlyeq
\let\ds\displaystyle
\let\ts\textstyle

\newcommand{\C}{\mathds{C}}
\newcommand{\R}{\mathds{R}}
\newcommand{\Z}{\mathds{Z}}
\newcommand{\N}{\mathds{N}}
\newcommand{\Q}{\mathds{Q}}

\renewcommand{\O}{\emptyset}

\newcommand\ubar[1]{\underaccent{\bar}{#1}}

\renewcommand\Re{\expandafter\mathfrak{Re}}
\renewcommand\Im{\expandafter\mathfrak{Im}}

\let\slantedpartial\partial
\DeclareRobustCommand{\partial}{\text{\rotatebox[origin=t]{20}{\scalebox{0.95}[1]{$\slantedpartial$}}}\hspace{-1pt}}

% merging two maths characters w/ \charfusion
\makeatletter
\def\moverlay{\mathpalette\mov@rlay}
\def\mov@rlay#1#2{\leavevmode\vtop{%
   \baselineskip\z@skip \lineskiplimit-\maxdimen
   \ialign{\hfil$\m@th#1##$\hfil\cr#2\crcr}}}
\newcommand{\charfusion}[3][\mathord]{
    #1{\ifx#1\mathop\vphantom{#2}\fi
        \mathpalette\mov@rlay{#2\cr#3}
      }
    \ifx#1\mathop\expandafter\displaylimits\fi}
\makeatother

% custom math commands
\newcommand{\T}{{\!\!\,\top}}
\newcommand{\avrt}[1]{\rotatebox{-90}{$#1$}}
\newcommand{\bigcupdot}{\charfusion[\mathop]{\bigcup}{\cdot}}
\newcommand{\cupdot}{\charfusion[\mathbin]{\cup}{\cdot}}
%\newcommand{\danger}{{\large\fontencoding{U}\fontfamily{futs}\selectfont\char 66\relax}\;}
\newcommand{\tendsto}[1]{\xrightarrow[#1]{}}
\newcommand{\vrt}[1]{\rotatebox{90}{$#1$}}
\newcommand{\tsup}[1]{\textsuperscript{\underline{#1}}}
\newcommand{\tsub}[1]{\textsubscript{#1}}

\renewcommand{\mod}[1]{~\left[ #1 \right]}
\renewcommand{\t}{{}^t\!}
\newcommand{\s}{\text{\calligra s}}

% custom units / constants
%\DeclareSIUnit{\litre}{\ell}
\let\hbar\hslash

% header / footer
\pagestyle{fancy}
\fancyhead{} \fancyfoot{}
\fancyfoot[C]{\thepage}

% fonts
\let\sc\scshape
\let\bf\bfseries
\let\it\itshape
\let\sl\slshape

% custom math operators
\let\th\relax
\let\det\relax
\DeclareMathOperator*{\codim}{codim}
\DeclareMathOperator*{\dom}{dom}
\DeclareMathOperator*{\gO}{O}
\DeclareMathOperator*{\po}{\text{\cursive o}}
\DeclareMathOperator*{\sgn}{sgn}
\DeclareMathOperator*{\simi}{\sim}
\DeclareMathOperator{\Arccos}{Arccos}
\DeclareMathOperator{\Arcsin}{Arcsin}
\DeclareMathOperator{\Arctan}{Arctan}
\DeclareMathOperator{\Argsh}{Argsh}
\DeclareMathOperator{\Arg}{Arg}
\DeclareMathOperator{\Aut}{Aut}
\DeclareMathOperator{\Card}{Card}
\DeclareMathOperator{\Cl}{\mathcal{C}\!\ell}
\DeclareMathOperator{\Cov}{Cov}
\DeclareMathOperator{\Ker}{Ker}
\DeclareMathOperator{\Mat}{Mat}
\DeclareMathOperator{\PGCD}{PGCD}
\DeclareMathOperator{\PPCM}{PPCM}
\DeclareMathOperator{\Supp}{Supp}
\DeclareMathOperator{\Vect}{Vect}
\DeclareMathOperator{\argmax}{argmax}
\DeclareMathOperator{\argmin}{argmin}
\DeclareMathOperator{\ch}{ch}
\DeclareMathOperator{\com}{com}
\DeclareMathOperator{\cotan}{cotan}
\DeclareMathOperator{\det}{det}
\DeclareMathOperator{\id}{id}
\DeclareMathOperator{\rg}{rg}
\DeclareMathOperator{\rk}{rk}
\DeclareMathOperator{\sh}{sh}
\DeclareMathOperator{\th}{th}
\DeclareMathOperator{\tr}{tr}

% colors and page style
\definecolor{truewhite}{HTML}{ffffff}
\definecolor{white}{HTML}{faf4ed}
\definecolor{trueblack}{HTML}{000000}
\definecolor{black}{HTML}{575279}
\definecolor{mauve}{HTML}{907aa9}
\definecolor{blue}{HTML}{286983}
\definecolor{red}{HTML}{d7827e}
\definecolor{yellow}{HTML}{ea9d34}
\definecolor{gray}{HTML}{9893a5}
\definecolor{grey}{HTML}{9893a5}
\definecolor{green}{HTML}{a0d971}

\pagecolor{white}
\color{black}

\begin{asydef}
	settings.prc = false;
	settings.render=0;

	white = rgb("faf4ed");
	black = rgb("575279");
	blue = rgb("286983");
	red = rgb("d7827e");
	yellow = rgb("f6c177");
	orange = rgb("ea9d34");
	gray = rgb("9893a5");
	grey = rgb("9893a5");
	deepcyan = rgb("56949f");
	pink = rgb("b4637a");
	magenta = rgb("eb6f92");
	green = rgb("a0d971");
	purple = rgb("907aa9");

	defaultpen(black + fontsize(8pt));

	import three;
	currentlight = nolight;
\end{asydef}

% theorems, proofs, ...

\mdfsetup{skipabove=1em,skipbelow=1em, innertopmargin=6pt, innerbottommargin=6pt,}

\declaretheoremstyle[
	headfont=\normalfont\itshape,
	numbered=no,
	postheadspace=\newline,
	headpunct={:},
	qed=\qedsymbol]{demstyle}

\declaretheorem[style=demstyle, name=Démonstration]{dem}

\newcommand\veczero{\kern-1.2pt\vec{\kern1.2pt 0}} % \vec{0} looks weird since the `0' isn't italicized

\makeatletter
\renewcommand{\title}[2]{
	\AtBeginDocument{
		\begin{titlepage}
			\begin{center}
				\vspace{10cm}
				{\Large \sc Chapitre #1}\\
				\vspace{1cm}
				{\Huge \calligra #2}\\
				\vfill
				Hugo {\sc Salou} MPI${}^{\star}$\\
				{\small Dernière mise à jour le \@date }
			\end{center}
		\end{titlepage}
	}
}

\newcommand{\titletp}[4]{
	\AtBeginDocument{
		\begin{titlepage}
			\begin{center}
				\vspace{10cm}
				{\Large \sc tp #1}\\
				\vspace{1cm}
				{\Huge \textsc{\textit{#2}}}\\
				\vfill
				{#3}\textit{MPI}${}^{\star}$\\
			\end{center}
		\end{titlepage}
	}
	\fancyfoot{}\fancyhead{}
	\fancyfoot[R]{#4 \textit{MPI}${}^{\star}$}
	\fancyhead[C]{{\sc tp #1} : #2}
	\fancyhead[R]{\thepage}
}

\newcommand{\titletd}[2]{
	\AtBeginDocument{
		\begin{titlepage}
			\begin{center}
				\vspace{10cm}
				{\Large \sc td #1}\\
				\vspace{1cm}
				{\Huge \calligra #2}\\
				\vfill
				Hugo {\sc Salou} MPI${}^{\star}$\\
				{\small Dernière mise à jour le \@date }
			\end{center}
		\end{titlepage}
	}
}
\makeatother

\newcommand{\sign}{
	\null
	\vfill
	\begin{center}
		{
			\fontfamily{ccr}\selectfont
			\textit{\textbf{\.{\"i}}}
		}
	\end{center}
	\vfill
	\null
}

\renewcommand{\thefootnote}{\emph{\alph{footnote}}}

% figure support
\usepackage{import}
\usepackage{xifthen}
\pdfminorversion=7
\usepackage{pdfpages}
\usepackage{transparent}
\newcommand{\incfig}[1]{%
	\def\svgwidth{\columnwidth}
	\import{./figures/}{#1.pdf_tex}
}

\pdfsuppresswarningpagegroup=1
\ctikzset{tripoles/european not symbol=circle}

\newcommand{\missingpart}{{\large\color{red} Il manque quelque chose ici\ldots}}


\let\thesection\relax

\makeatletter
\newcommand*\longline[1][.4\p@]{%
	\leavevmode
	\leaders \hrule \@height #1\relax \hfill
	\null
}
\makeatother

\begin{document}
	\begin{center}
		\bfseries
		\Huge \textbf{DM\textsubscript5 Mathématiques}
	\end{center}

	{ \noindent
		\longline\quad
		\smash{\LARGE \textsc{Exercice}}
		\quad\longline
	}

	\begin{enumerate}
		\item Soit $n \in \N^*$. Soit $x \in \left]-\frac{\pi}{2}, \frac{\pi}{2} \right[$.
			La fonction tangente est $\mathcal{C}^\infty$ sur cet intervalle.
			On peut donc utiliser la formule de \textsc{Leibniz} : \[
				\forall x \in \left]-\frac{\pi}{2}, \frac{\pi}{2} \right[,\quad
				(\tan^2)^{(n)}(x) = \sum_{k=0}^n {n\choose k} \tan^{(k)}(x) \times \tan^{(n-k)}(x)
			.\]
			Or, on sait que, pour tout $x \in \left]-\frac{\pi}{2}, \frac{\pi}{2} \right[$, $\tan^{(n+1)}(x) = (\tan')^{(n)}(x) = (\tan^2)^{(n)}(x)$\/ car $\tan'x = 1 + \tan^2 x$.
			D'où, \[
				\boxed{\forall x \in \left]-\frac{\pi}{2}, \frac{\pi}{2} \right[,\quad
				\tan^{(n+1)}(x) = \sum_{k=0}^n {n\choose k} \tan^{(k)}(x) \times \tan^{(n-k)}(x).}
			\]
		\item On procède par récurrence forte.
			\begin{itemize}
				\item On a, pour $x \in \left[0, \frac{\pi}{2}\right[$, $\tan^{(0)} x = \tan x \ge 0$.
				\item D'après la question 1, pour $x \in \left[0, \frac{\pi}{2} \right[$, $\tan^{(n+1)} x$\/ est positif, car somme de $n$\/ termes positifs, par hypothèse de récurrence.
			\end{itemize}
			D'où, \[
				\boxed{\forall n \in \N,\:\forall x \in \left[0,\frac{\pi}{2} \right[,\quad \tan^{(n)} x \ge 0.}
			\]
		\item Soit $n \in \N$. Si $I$\/ est un intervalle, et $f : I \to \R$\/ est une fonction de classe $\mathcal{C}^{n+1}$, et $a \in I$, alors \[
				\boxed{\forall x \in I,\quad f(x) = \sum_{k=0}^n \frac{f^{(k)}(a)}{k!}(x-a)^k + \int_{a}^{x} \frac{f^{(n+1)}(x)}{n!}(x-t)^n~\mathrm{d}t.}
			\]
		\item
		\item On a $a_0 = \tan(0) = 0$, et $a_1 = \tan'(0) = 1+\tan^20 = 1$.
			Pour tout $n \in \N^*$,
			\begin{align*}
				(n+1) a_{n+1} &= \frac{(n+1)\: \tan^{(n+1)} 0}{(n+1)!}\\
				&= \frac{1}{n!} \sum_{k=0}^n {n\choose k} \times  \tan^{(k)}0 \times \tan^{(n-k)} 0 \\
				&= \sum_{k=0}^n \frac{n!}{k!\:(n-k)!} \times \frac{1}{n!} \tan^{(k)} 0 \times \tan^{(n-k)} 0  \\
				&= \sum_{k=0}^n \frac{\tan^{(k)} 0}{k!} \times \frac{\tan^{(n-k)} 0}{(n-k)!} \\
				&= \sum_{k=0}^n a_k\:a_{n-k} \\
			\end{align*}
		\item D'après la question 4, la série $\sum a_n x^n$\/ converge pour $x \in \left[0,\frac{\pi}{2}\right[$. Or, la série $\sum a_n x^n$\/ est une série entière, son rayon de convergence $R$ est donc supérieur ou égal à $\frac{\pi}{2}$ : \[
				\ts\forall x \in {]{-R},R[}, \quad \sum a_n x^n \text{ converge}
			.\] D'où, $\forall x \in \left] {-\frac{\pi}{2}}, \frac{\pi}{2} \right[ \subset {]{-R},R[}$, la série $\sum a_n x^n$ converge. La fonction $S$\/ est donc bien définie sur $\left] {-\frac{\pi}{2}}, \frac{\pi}{2} \right[$.
		\item On peut dériver terme à terme la série entière $\sum a_n x^n$\/ sans changer son rayon de convergence. Ainsi, pour $x \in \left]{-\frac{\pi}{2}}, \frac{\pi}{2}\right[$,
			\begin{align*}
				S'(x) = \sum_{n=1}^\infty a_n n x^{n-1} &= \sum_{n=0}^\infty a_{n+1}(n+1) x^n\\
				&= 1+ \sum_{n=1}^\infty x^n\sum_{k=0}^n a_k a_{n-k} \\
				&= 1+\sum_{n=1}^\infty \sum_{k=0}^n (a_k x^k) \cdot(a_{n-k} x^{n-k}) \\
				&= 1+\Big(\!\sum_{n=1}^\infty a_n x^n \Big) \cdot \Big(\!\sum_{n=1}^\infty a_n x^n\Big) \\
				&= 1+\Big(\!\sum_{n=0}^\infty a_n x^n \Big) \cdot \Big(\!\sum_{n=1}^\infty a_n x^n\Big) \\
				&= 1+S^2(x).
			\end{align*}
	\end{enumerate}

	\bigskip
	\bigskip
	\bigskip
	\bigskip

	{ \noindent
		\longline\quad
		\smash{\LARGE \textsc{Problème}}
		\quad\longline
	}

	\section{Partie 1.}

	\begin{enumerate}
		\item Soit $\vec{v} \in E \setminus F$, donc $\vec{v} \neq \vec{0}$.
			Soit $p$\/ l'application $p : \vec{x} \mapsto \vec{x} - p(\vec{x})$, la projection sur $F^\perp$, car $E$\/ est de dimension finie.
			On procède par \textsc{Analyse-Synthèse}, ce qui démontre l'existence et l'unicité.
			\begin{description}
				\item[Analyse] Soient $\vec{u} \in F$\/ et $\lambda \in \R$\/ tels que $\vec{u} + \lambda \vec{v} \in F^\perp$, $\|\vec{u} + \lambda \vec{v}\| = \alpha$\/ et $\left<\vec{u} + \lambda \vec{v}  \mid \vec{v} \right> > 0$. On a
					\begin{align*}
						\vec{u} + \lambda \vec{v} \in F^\perp \quad~&\text{ donc }\quad \left<\vec{u} + \lambda \vec{v}  \mid \vec{u} \right> = 0\\
						& \text{ donc }\quad \left<\vec{u} + \lambda\pi(\vec{v}) \mid \vec{u} \right> + \lambda \left<p(\vec{v})  \mid \vec{u} \right> = 0\\
						&\text{ donc }\quad \left<\vec{u} + \lambda \pi(\vec{v})  \mid \vec{u} \right> = 0\\
						& \text{ donc }\quad \vec{u} + \lambda \pi(\vec{v}) \in F^\perp\\
						& \text{ donc }\quad \vec{u} + \lambda \pi(\vec{v}) = \vec{0}\\
					\end{align*}
					On en déduit $\vec{u} = -\lambda \pi(\vec{v})$.
					On a
					\begin{align*}
						\|\vec{u} + \lambda \vec{v}\| = \alpha &\text{ donc } \|-\lambda \pi(\vec{v}) + \lambda \vec{v}\| = \alpha\\
						&\text{ donc } \|\lambda p(\vec{v})\| = \alpha\\
						& \text{ donc } |\lambda| = \frac{\alpha}{\|p(\vec{v})\|}
					\end{align*}
					car $p(\vec{v}) \neq \vec{0}$ (sinon, $\vec{v} \in F$, ce qui est faux par hypothèse).
					On en déduit $\lambda \in \{\alpha/\|p(\vec{v})\|, -\alpha/\|p(\vec{v})\|\}$.
					Et, on a,
					\begin{align*}
						\left<\vec{u} + \lambda \vec{v}  \mid \vec{v} \right> > 0 \quad~& \text{ si, et seulement si } \quad
						\left< \lambda p(\vec{v})  \mid \vec{v}\right> > 0\\
						&\text{ si, et seulement si }\quad \lambda \left<p(\vec{v}) \mid \vec{v} \right> > 0\\
						& \text{ si, et seulement si }\quad \lambda \left< p(\vec{v})  \mid p(\vec{v}) \right> + \left<p(\vec{v})  \mid \pi(\vec{v}) \right> > 0\\
						&\text{ si, et seulement si }\quad \lambda\:\|p(\vec{v})\|^2 > 0\\
						&\text{ si, et seulement si }\quad \lambda > 0\\
					\end{align*}
					On en déduit $\lambda > 0$.
				\item[Synthèse] On pose $\lambda = \alpha / \|p(\vec{v})\|$, et $\vec{u} = -\lambda \pi(\vec{v})$.
				On a $\vec{u} + \lambda \vec{v} = -\lambda \pi(\vec{v})  + \lambda \vec{v} = \lambda p(\vec{v}) \in F^\perp$. Par l'équivalence de l'analyse, on a $\left<\vec{u} + \lambda \vec{v}  \mid \vec{v} \right> > 0$\/ car $\lambda > 0$.
					Finalement, on a \[
						\left\| \vec{u} + \lambda \vec{v} \right\| = \|\lambda p(\vec{v})\| = \lambda \| p(\vec{v})\| = \alpha
					.\]
			\end{description}
			D'où l'existence et l'unicité.
			De plus, $\vec{u} + \lambda \vec{v} = \lambda p(\vec{v})$\/ où $p$\/ est la projection orthogonale sur $F^\perp$\/ (car $(F^\perp)^\perp = F$, car $E$\/ est de dimension finie).
		\item
			On procède par \textsc{Analyse-Synthèse}.
			\begin{description}
				\item[Analyse] Supposons construit la famille $(\vec{w}_1, \ldots, \vec{w}_n)$ vérifiant les conditions de l'énoncé.
					Ainsi, pour tout $p \in \llbracket 1,n \rrbracket$, $\vec{w}_p \in E_p$.
					Soit alors les réels $x_1, \ldots, x_p$ tels que $\vec{w}_p = x_1 \vec{w}_1 + \cdots + x_{p-1} \vec{w}_{p-1} + x_p \vec{v}_p$.
					De plus, $\vec{w}_p \in E_{p-1}^\perp$, d'où, $\forall q \in \llbracket 1, p - 1 \rrbracket$,
					\begin{align*}
						0 &= \left<\vec{w}_p  \mid \vec{v}_q \right>\\
						&= \left<x_1 \vec{w}_1 + \cdots + x_{p-1}\vec{w}_{p-1} + x_p \vec{w}_p \mid \vec{v}_q \right> \\
						&= x_1 \left<\vec{w}_1  \mid \vec{v}_q \right> + \cdots + x_{p-1} \left< \vec{w}_{p-1}  \mid \vec{v}_q\right> + x_p\left<\vec{v}_p  \mid \vec{v}_q \right>\\
						&= x_q \left<\vec{w}_q  \mid \vec{v}_q \right> + \cdots + x_{p-1} \left<\vec{w}_{p-1}  \mid \vec{v}_q \right> + x_p \|\vec{v}_p\| \\
					\end{align*}
			\end{description}
	\end{enumerate}
\end{document}
