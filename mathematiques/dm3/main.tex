\documentclass[a4paper]{article}

\usepackage[margin=1in]{geometry}
\usepackage[utf8]{inputenc}
\usepackage[T1]{fontenc}
\usepackage{mathrsfs}
\usepackage{textcomp}
\usepackage[french]{babel}
\usepackage{amsmath}
\usepackage{amssymb}
\usepackage{cancel}
\usepackage{frcursive}
\usepackage[inline]{asymptote}
\usepackage{tikz}
\usepackage[european,straightvoltages,europeanresistors]{circuitikz}
\usepackage{tikz-cd}
\usepackage{tkz-tab}
\usepackage[b]{esvect}
\usepackage[framemethod=TikZ]{mdframed}
\usepackage{centernot}
\usepackage{diagbox}
\usepackage{dsfont}
\usepackage{fancyhdr}
\usepackage{float}
\usepackage{graphicx}
\usepackage{listings}
\usepackage{multicol}
\usepackage{nicematrix}
\usepackage{pdflscape}
\usepackage{stmaryrd}
\usepackage{xfrac}
\usepackage{hep-math-font}
\usepackage{amsthm}
\usepackage{thmtools}
\usepackage{indentfirst}
\usepackage[framemethod=TikZ]{mdframed}
\usepackage{accents}
\usepackage{soulutf8}
\usepackage{mathtools}
\usepackage{bodegraph}
\usepackage{slashbox}
\usepackage{enumitem}
\usepackage{calligra}
\usepackage{cinzel}
\usepackage{BOONDOX-calo}

% Tikz
\usetikzlibrary{babel}
\usetikzlibrary{positioning}
\usetikzlibrary{calc}

% global settings
\frenchspacing
\reversemarginpar
\setuldepth{a}

%\everymath{\displaystyle}

\frenchbsetup{StandardLists=true}

\def\asydir{asy}

%\sisetup{exponent-product=\cdot,output-decimal-marker={,},separate-uncertainty,range-phrase=\;à\;,locale=FR}

\setlength{\parskip}{1em}

\theoremstyle{definition}

% Changing math
\let\emptyset\varnothing
\let\ge\geqslant
\let\le\leqslant
\let\preceq\preccurlyeq
\let\succeq\succcurlyeq
\let\ds\displaystyle
\let\ts\textstyle

\newcommand{\C}{\mathds{C}}
\newcommand{\R}{\mathds{R}}
\newcommand{\Z}{\mathds{Z}}
\newcommand{\N}{\mathds{N}}
\newcommand{\Q}{\mathds{Q}}

\renewcommand{\O}{\emptyset}

\newcommand\ubar[1]{\underaccent{\bar}{#1}}

\renewcommand\Re{\expandafter\mathfrak{Re}}
\renewcommand\Im{\expandafter\mathfrak{Im}}

\let\slantedpartial\partial
\DeclareRobustCommand{\partial}{\text{\rotatebox[origin=t]{20}{\scalebox{0.95}[1]{$\slantedpartial$}}}\hspace{-1pt}}

% merging two maths characters w/ \charfusion
\makeatletter
\def\moverlay{\mathpalette\mov@rlay}
\def\mov@rlay#1#2{\leavevmode\vtop{%
   \baselineskip\z@skip \lineskiplimit-\maxdimen
   \ialign{\hfil$\m@th#1##$\hfil\cr#2\crcr}}}
\newcommand{\charfusion}[3][\mathord]{
    #1{\ifx#1\mathop\vphantom{#2}\fi
        \mathpalette\mov@rlay{#2\cr#3}
      }
    \ifx#1\mathop\expandafter\displaylimits\fi}
\makeatother

% custom math commands
\newcommand{\T}{{\!\!\,\top}}
\newcommand{\avrt}[1]{\rotatebox{-90}{$#1$}}
\newcommand{\bigcupdot}{\charfusion[\mathop]{\bigcup}{\cdot}}
\newcommand{\cupdot}{\charfusion[\mathbin]{\cup}{\cdot}}
%\newcommand{\danger}{{\large\fontencoding{U}\fontfamily{futs}\selectfont\char 66\relax}\;}
\newcommand{\tendsto}[1]{\xrightarrow[#1]{}}
\newcommand{\vrt}[1]{\rotatebox{90}{$#1$}}
\newcommand{\tsup}[1]{\textsuperscript{\underline{#1}}}
\newcommand{\tsub}[1]{\textsubscript{#1}}

\renewcommand{\mod}[1]{~\left[ #1 \right]}
\renewcommand{\t}{{}^t\!}
\newcommand{\s}{\text{\calligra s}}

% custom units / constants
%\DeclareSIUnit{\litre}{\ell}
\let\hbar\hslash

% header / footer
\pagestyle{fancy}
\fancyhead{} \fancyfoot{}
\fancyfoot[C]{\thepage}

% fonts
\let\sc\scshape
\let\bf\bfseries
\let\it\itshape
\let\sl\slshape

% custom math operators
\let\th\relax
\let\det\relax
\DeclareMathOperator*{\codim}{codim}
\DeclareMathOperator*{\dom}{dom}
\DeclareMathOperator*{\gO}{O}
\DeclareMathOperator*{\po}{\text{\cursive o}}
\DeclareMathOperator*{\sgn}{sgn}
\DeclareMathOperator*{\simi}{\sim}
\DeclareMathOperator{\Arccos}{Arccos}
\DeclareMathOperator{\Arcsin}{Arcsin}
\DeclareMathOperator{\Arctan}{Arctan}
\DeclareMathOperator{\Argsh}{Argsh}
\DeclareMathOperator{\Arg}{Arg}
\DeclareMathOperator{\Aut}{Aut}
\DeclareMathOperator{\Card}{Card}
\DeclareMathOperator{\Cl}{\mathcal{C}\!\ell}
\DeclareMathOperator{\Cov}{Cov}
\DeclareMathOperator{\Ker}{Ker}
\DeclareMathOperator{\Mat}{Mat}
\DeclareMathOperator{\PGCD}{PGCD}
\DeclareMathOperator{\PPCM}{PPCM}
\DeclareMathOperator{\Supp}{Supp}
\DeclareMathOperator{\Vect}{Vect}
\DeclareMathOperator{\argmax}{argmax}
\DeclareMathOperator{\argmin}{argmin}
\DeclareMathOperator{\ch}{ch}
\DeclareMathOperator{\com}{com}
\DeclareMathOperator{\cotan}{cotan}
\DeclareMathOperator{\det}{det}
\DeclareMathOperator{\id}{id}
\DeclareMathOperator{\rg}{rg}
\DeclareMathOperator{\rk}{rk}
\DeclareMathOperator{\sh}{sh}
\DeclareMathOperator{\th}{th}
\DeclareMathOperator{\tr}{tr}

% colors and page style
\definecolor{truewhite}{HTML}{ffffff}
\definecolor{white}{HTML}{faf4ed}
\definecolor{trueblack}{HTML}{000000}
\definecolor{black}{HTML}{575279}
\definecolor{mauve}{HTML}{907aa9}
\definecolor{blue}{HTML}{286983}
\definecolor{red}{HTML}{d7827e}
\definecolor{yellow}{HTML}{ea9d34}
\definecolor{gray}{HTML}{9893a5}
\definecolor{grey}{HTML}{9893a5}
\definecolor{green}{HTML}{a0d971}

\pagecolor{white}
\color{black}

\begin{asydef}
	settings.prc = false;
	settings.render=0;

	white = rgb("faf4ed");
	black = rgb("575279");
	blue = rgb("286983");
	red = rgb("d7827e");
	yellow = rgb("f6c177");
	orange = rgb("ea9d34");
	gray = rgb("9893a5");
	grey = rgb("9893a5");
	deepcyan = rgb("56949f");
	pink = rgb("b4637a");
	magenta = rgb("eb6f92");
	green = rgb("a0d971");
	purple = rgb("907aa9");

	defaultpen(black + fontsize(8pt));

	import three;
	currentlight = nolight;
\end{asydef}

% theorems, proofs, ...

\mdfsetup{skipabove=1em,skipbelow=1em, innertopmargin=6pt, innerbottommargin=6pt,}

\declaretheoremstyle[
	headfont=\normalfont\itshape,
	numbered=no,
	postheadspace=\newline,
	headpunct={:},
	qed=\qedsymbol]{demstyle}

\declaretheorem[style=demstyle, name=Démonstration]{dem}

\newcommand\veczero{\kern-1.2pt\vec{\kern1.2pt 0}} % \vec{0} looks weird since the `0' isn't italicized

\makeatletter
\renewcommand{\title}[2]{
	\AtBeginDocument{
		\begin{titlepage}
			\begin{center}
				\vspace{10cm}
				{\Large \sc Chapitre #1}\\
				\vspace{1cm}
				{\Huge \calligra #2}\\
				\vfill
				Hugo {\sc Salou} MPI${}^{\star}$\\
				{\small Dernière mise à jour le \@date }
			\end{center}
		\end{titlepage}
	}
}

\newcommand{\titletp}[4]{
	\AtBeginDocument{
		\begin{titlepage}
			\begin{center}
				\vspace{10cm}
				{\Large \sc tp #1}\\
				\vspace{1cm}
				{\Huge \textsc{\textit{#2}}}\\
				\vfill
				{#3}\textit{MPI}${}^{\star}$\\
			\end{center}
		\end{titlepage}
	}
	\fancyfoot{}\fancyhead{}
	\fancyfoot[R]{#4 \textit{MPI}${}^{\star}$}
	\fancyhead[C]{{\sc tp #1} : #2}
	\fancyhead[R]{\thepage}
}

\newcommand{\titletd}[2]{
	\AtBeginDocument{
		\begin{titlepage}
			\begin{center}
				\vspace{10cm}
				{\Large \sc td #1}\\
				\vspace{1cm}
				{\Huge \calligra #2}\\
				\vfill
				Hugo {\sc Salou} MPI${}^{\star}$\\
				{\small Dernière mise à jour le \@date }
			\end{center}
		\end{titlepage}
	}
}
\makeatother

\newcommand{\sign}{
	\null
	\vfill
	\begin{center}
		{
			\fontfamily{ccr}\selectfont
			\textit{\textbf{\.{\"i}}}
		}
	\end{center}
	\vfill
	\null
}

\renewcommand{\thefootnote}{\emph{\alph{footnote}}}

% figure support
\usepackage{import}
\usepackage{xifthen}
\pdfminorversion=7
\usepackage{pdfpages}
\usepackage{transparent}
\newcommand{\incfig}[1]{%
	\def\svgwidth{\columnwidth}
	\import{./figures/}{#1.pdf_tex}
}

\pdfsuppresswarningpagegroup=1
\ctikzset{tripoles/european not symbol=circle}

\newcommand{\missingpart}{{\large\color{red} Il manque quelque chose ici\ldots}}

\usepackage{tgschola}

\fancyhead[L]{Hugo \textsc{Salou} \textit{MPI$^\star$}}
\fancyhead[R]{DM\textsubscript 3 Mathématiques}

\begin{document}
	\begin{center}
		\Huge \textbf{DM\textsubscript3 Mathématiques}
	\end{center}

	\begin{center}
		\LARGE ---\quad\textsc{Exercice}\quad---
	\end{center}

	\begin{enumerate}
		\item
			\begin{enumerate}
				\item Soit $(a_n)_{n\in\N}$\/ une suite complexe vérifiant $(P_1)$. Soit $(u_n)_{n\in\N}$\/ une suite complexe. On suppose que la série $\sum u_n$\/ est convergente. Ainsi, $u_n$\/ tend vers 0 (quand $n \to \infty$), et donc la suite $(u_n)_{n\in\N}$\/ est bornée. D'où, d'après $(P_1)$, la série $\sum a_n u_n$\/ converge.
				\item Soit $(a_n)_{n\in\N}$\/ une suite complexe telle que la série $\sum\:|a_n|$\/ converge. Montrons que la suite vérifie la propriété $(P_1)$. Soit $(u_n)_{n\in\N}$\/ une suite complexe bornée. Soit $M \in \R^+$\/ tel que, pour tout $n \in \N$, $|u_n| \le M$. On a donc $\forall n \in \N$, $0 \le |a_n\:u_n| \le M\:|a_n|$. Comme la série $\sum M\:|a_n| = M \cdot \sum |a_n|$\/ converge, alors la série $\sum |a_n\:u_n|$\/ converge. On en déduit que la série $\sum a_n\:u_n$\/ converge.
			\end{enumerate}
		\item
			\begin{enumerate}
				\item Non. On pose, pour $n \in \N$, $u_n = (-1)^n$. La suite $(u_n)_{n\in\N}$\/ est bornée (par $-1$\/ et $1$). Mais, la série $\sum a_n\:u_n$\/ ne converge pas. En effet, pour $n \in \N$, $a_n\:v_n = (-1)^{2n} \cdot \frac{1}{n} = \frac{1}{n}$, et la série $\sum \frac{1}{n}$\/ diverge par critère de \textsc{Riemann}.
				\item La fonction $x\mapsto x \ln x$\/ est croissante, d'où, par composition avec la fonction inverse, la fonction $\frac{1}{x \ln x}$\/ est décroissante. On compare série et intégrale : \[
						\int_{2}^{N+1} \frac{1}{x \ln x}~\mathrm{d}x \le \sum_{k=2}^N \frac{1}{k \ln k}
					.\] Or, à l'aide du changement de variable $u = \ln x$, on a \[
						\int_{2}^{N+1} \frac{1}{x \ln x}~\mathrm{d}x = \int^{\ln(N+1)}_{\ln 2} \frac{1}{u}~\mathrm{d}u = \ln(\ln(N+1)) - \ln(\ln 2),
					\] qui diverge vers $+\infty$\/ quand $N \to \infty$. On en déduit que la série $\sum \frac{1}{n \ln n}$\/ diverge.
				\item Non. On pose, pour $n \in \N$, $u_n = \frac{(-1)^n}{\ln n}$. La série alternée $\sum u_n$\/ converge ; en effet, la fonction $x \mapsto \frac{1}{\ln x}$\/ est positive et décroissante. On a, pour $n \in \N$, $u_n\:a_n = \frac{1}{n \ln n}$. Et, on a vu dans la question précédente que la série $\sum \frac{1}{n \ln n} = \sum a_n\:u_n$\/ diverge.
			\end{enumerate}
		\item
			\begin{enumerate}
				\item La série $\sum (a_{n+1} - a_n)$\/ converge absolument, donc elle converge simplement. On pose, pour $n \in \N$, $S_n = \sum_{k=1}^{n-1} (a_{k+1} - a_k)$. Or, comme la somme est télescopique, on a, pour tout $n \in \N$, \[
						S_n = \sum_{k=1}^{n-1} (a_{k+1} -a_k) = a_n - a_0
					.\] Or, comme $\sum (a_{n+1} - a_n)$\/ converge, $S_n$\/ admet une limite finie $\ell$. Ainsi, on a $a_n = S_n + a_0 \tendsto{n\to +\infty} \ell + a_0 \in \C$. La suite $(a_n)_{n\in\N}$\/ admet donc une limite finie.
				\item On procède par récurrence.
					\begin{itemize}
						\item On a d'une part $\sum_{n=0}^1 a_n u_n = a_0 u_0 + a_1 u 1$, et d'autre part
							\begin{align*}
								\smash{\sum_{n=0}^0} (a_n- a_{n+1}) U_n + a_1 U_1 &= (a_0 - a_1) U_0 + a_1 U_1\\
								&= a_0\,u_0 - \cancel{a_1\,u_0} + a_1\,u_1 + \cancel{a_1\,u_0} \\
								&= a_0 u_0 + a_1 u_1 \\
							\end{align*}
						\item Soit $N \in \N$. On suppose que \[
								\sum_{n=0}^N a_n u_n = \sum_{n=0}^{N-1}(a_n - a_{n+1}) U_n + a_N U_N
							.\] Ainsi,
							\begin{align*}
								\sum_{n=0}^{N+1} a_n u_n &= \sum_{n=0}^N a_n u_n + a_{N+1}u_{N+1}\\
								&= \sum_{n=0}^{N-1}(a_n - a_{n+1}) U_n + a_N U_N + a_{N+1} (U_{N+1} - U_N)\\
								&= \sum_{n=0}^{N-1}(a_n - a_{n+1}) U_n + (a_N - a_{N+1}) U_N + a_{N+1} U_{N+1}\\
								&= \sum_{n=0}^N (a_n - a_{n+1}) U_n + a_{N+1} U_{N+1}\\
							\end{align*}
					\end{itemize}
					Si $\sum u_n$\/ converge, alors $(U_n)_{n\in\N}$\/ converge, et donc la série $\sum a_n u_n$\/ converge.
			\end{enumerate}
		\item
			\begin{enumerate}
				\item On pose, pour tout $n \in \N$, $r_n = |a_n|$\/ et $\theta_n = \arg(a_n)$. Ainsi, $\forall n \in \N,\: a_n = r_n \mathrm{e}^{i \theta_n}$. On pose, pour tout $n \in \N$, $u_n = \mathrm{e}^{-i\theta_n}$. La suite $(u_n)_{n\in\N}$\/ est composée de complexes de module 1. On a, pour tout $n \in \N$, $a_n\:u_n = r_n\:\mathrm{e}^{i\theta_n}\: \mathrm{e}^{-i\theta_n} = r_n$. Or, comme la série $\sum|a_n| = \sum r_n$\/ diverge, alors la série $\sum a_n u_n$\/ diverge également.
				\item On a montré dans la question (1b) que si la série $\sum |a_n|$\/ converge, alors elle vérifie $(P_1)$. Puis, on a montré en question (4a) que si la série $\sum |a_n|$\/ diverge, alors elle ne vérifie pas $(P_2)$. On en déduit que la suite $(a_n)_{n\in\N}$\/ vérifie $(P_1)$\/ si, et seulement si la série $\sum |a_n|$\/ converge (\textit{i.e.}\ la série $\sum a_n$\/ est absolument convergente).
			\end{enumerate}
	\end{enumerate}

	\begin{center}
		\LARGE ---\quad\textsc{Problème}\quad---
	\end{center}
	\begin{enumerate}
		\item
			\begin{enumerate}
				\item On trouve $\chi_{M(\alpha)}(X) = (X - 2)(X - 1)(X + \alpha - 2)$. En effet, soit $\lambda \in \R$. On calcule
					\begin{align*}
						\det\big(\lambda I_3 - M(\alpha)\big)
						&= 
						\begin{vmatrix}
							\lambda - 1 & 1 & -\alpha\\
							0 & \lambda - 2 & \alpha\\
							-1 & -1 & \lambda - 2 + \alpha\\
						\end{vmatrix}\\
						&= 
						\begin{vmatrix}
							\lambda - 1 & 1 & -\alpha\\
							\lambda - 1 & \lambda-1 & 0\\
							-1 & -1 & \lambda - 2 + \alpha
						\end{vmatrix} \text{ avec } L_2 \gets L_1 + L_2\\
						&= 
						\begin{vmatrix}
							\lambda-2 & 1 & -\alpha\\
							0 & \lambda - 1 & 0\\
							0 & -1 & \lambda - 2 + \alpha
						\end{vmatrix} \text{ avec } C_1 \gets C_1 - C_2\\
						&= (\lambda - 2)
						\begin{vmatrix}
							\lambda - 1 & 0\\
							-1 & \lambda - 2 + \alpha
						\end{vmatrix}\\
						&= (\lambda - 2)(\lambda - 1)(\lambda - 2 + \alpha).
					\end{align*}
				\item On a bien $\prod_{i=1}^3 (X-a_{i,i}) = (X-1)(X-2)(X+\alpha-2) = \chi_{M(\alpha)}(X)$, donc $M(\alpha)$\/ est bien une matrice à diagonale propre.
				\item Comme $\chi_{M(\alpha)}$\/ est scindé, on voit que ses racines sont $1$, $2$\/ et $2-\alpha$. Donc
					\begin{align*}
						M(\alpha) \text{ est diagonalisable }
						\iff& \chi_{M(\alpha)} \text{ est scindé à racines simples }\\
						\iff& 2 - \alpha \neq 1 \text{ et } 2 - \alpha \neq  2\\
						\iff& \alpha \neq 1 \text{ et } \alpha \neq 0\\
						\iff& \alpha \not\in  \{0,1\}.
					\end{align*}
			\end{enumerate}
		\item Soit $\lambda \in \R$. On a
			\begin{align*}
				\det(\lambda I_3 - A) &= 
				\begin{vmatrix}
					\lambda & 0 & 1\\
					0 & \lambda & 0\\
					-1 & 0 & \lambda
				\end{vmatrix}\\
				&= -\lambda
				\begin{vmatrix}
					\lambda & 1\\
					-1 & \lambda
				\end{vmatrix}\\
				&= -\lambda(\lambda^2 + 1) \\
			\end{align*}
			Son polynôme caractéristique est $\chi_A(X) = -X(X^2 + 1)$. Il n'est pas scindé sur $\R$, la matrice $A$\/ n'est donc pas à diagonale propre.
		\item Soit $A = {a\:b\choose c\:d} \in \mathcal{M}_2(\R)$. On a $\chi_A(X) = \det{X - a\quad-b\choose-c\quad X - d} = (X - a)(X-d) - bc$. Ainsi, \[
				A \text{ est à diagonale propre} \iff \chi_A(X) = (X-a)(X-b) \iff b = 0 \text{ ou } c = 0
			.\] On en déduit que $\mathcal{E}_2 = \big\{ {a\:b\choose0\:d}  \mid (a,b,d) \in \R^3 \big\} \cup \big\{ {a\:0\choose c\:d}  \mid (a,c,d) \in \R^3 \big\}$. Autrement dit, $\mathcal{E}_2$\/ est l'ensemble des matrices $2\times 2$\/ triangulaires.
		\item Une matrice est inversible si, et seulement si elle ne possède aucun `\,0\,' sur sa diagonale. En effet, une matrice $A$\/ est inversible $A$\/ si, et seulement si $\det A \neq 0$. Or, $\det A = \prod_{\lambda \in \Sp(A)} \lambda^{m_\lambda}$\/ (où $m_\lambda$\/ correspond à la multiplicité de la racine $\lambda$\/ du polynôme caractéristique). On en déduit que $A$ est inversible si et seulement s'il exist pas $\lambda \in \Sp(A)$\/ avec $\lambda^{m_\lambda} = 0$, donc $\lambda = 0$.

			Par exemple, on pose \[
				A = \begin{pmatrix}
					1&0&0\\
					1&1&0\\
					1&1&1
				\end{pmatrix},
			\]une matrice non diagonale. On a $\chi_A(X) = (X-1)^3$, car c'est un déterminant triangulaire. C'est donc bien une matrice à diagonale propre. Et, comme `\,0\,' n'est pas racine du polynôme caractéristique, on en déduit que $A$\/ est inversible. On a, par méthode du pivot de \textsc{Gauss}, \[
				A^{-1} = \begin{pmatrix}
					1&0&0\\
					-1&1&0\\
					0&-1&1
				\end{pmatrix}
			\] qui est aussi une matrice à diagonale propre. En effet, on a $\chi_{A^{-1}}(X) = (X - 1)^3$, comme c'est un déterminant triangulaire.
		\item On pose \[
				A = \begin{pmatrix}
					a_{11}&a_{12}&a_{13}\\
					a_{21}&a_{22}&a_{23}\\
					a_{31}&a_{32}&a_{33}
				\end{pmatrix} \in \mathcal{M}_3(\R)
			.\] On calcule le polynôme caractéristique : soit $\lambda \in \R$, on a
			\begin{align*}
				\det(\lambda I_3 - A) &=
				\begin{vmatrix}
					\lambda-a_{11}&-a_{12}&-a_{13}\\
					-a_{21}&\lambda-a_{22}&-a_{23}\\
					-a_{31}&-a_{32}&\lambda-a_{33}
				\end{vmatrix}\\
				&= (\lambda - a_{11}) 
				\begin{vmatrix}
					\lambda - a_{22} & -a_{23}\\
					-a_{32}&\lambda-a_{33}
				\end{vmatrix}\\
				&\mathrel{\phantom{=}}{}+ a_{21}
				\begin{vmatrix}
					-a_{12}&-a_{13}\\
					-a_{32}&\lambda-a_{33}
				\end{vmatrix}\\
				&\mathrel{\phantom{=}}{}- a_{31}
				\begin{vmatrix}
					-a_{12} & -a_{13}\\
					\lambda - a_{22} & -a_{23}
				\end{vmatrix}\\
				&= (\lambda - a_{11})\big((\lambda - a_{22})(\lambda-a_{33})-a_{23}a_{32}\big)\\
				&\mathrel{\phantom{=}}{}+ a_{21}\big(a_{12}(a_{33}-\lambda) - a_{13}a_{32} \big) \\
				&\mathrel{\phantom{=}}{}- a_{31}\big(a_{12}a_{23} + a_{13}(\lambda-a_{22})\big)\\
				&= (\lambda-a_{11})(\lambda-a_{22})(\lambda-a_{33}) - a_{23}a_{32}(\lambda - a_{11})\\
				&\mathrel{\phantom{=}}{}+ a_{21}a_{12}(a_{33}-\lambda) - a_{13}a_{32}a_{21} - a_{31}a_{12}a_{23}\\
				&\mathrel{\phantom{=}}{}- a_{31}a_{13}(\lambda-a_{22}) \\
				&= \lambda^3 - \lambda^2(a_{11}+a_{22}+a_{33})\\
				&\mathrel{\phantom{=}}{}+ \lambda(a_{22}a_{33} + a_{11}a_{22} + a_{11}a_{33} - a_{23}a_{32} - a_{21}a_{12} - a_{31}a_{13})\\
				&\mathrel{\phantom{=}}{}-a_{11}a_{22}a_{33} + a_{23}a_{32}a_{11} + a_{21}a_{12}a_{33}\\
				&\mathrel{\phantom{=}}{}- a_{13}a_{32}a_{21}-a_{31}a_{12}a_{23} - a_{31}a_{13}a_{22}
			\end{align*}
			Or, $\prod_{i=1}^3(\lambda-a_{i,i}) = (\lambda-a_{11})(\lambda-a_{22})(\lambda-a_{33}) = \lambda^3 - \lambda^2(a_{11} + a_{22} + a_{33}) + \lambda(a_{11}a_{22} + a_{22}a_{33} + a_{11}a_{33}) - a_{11}a_{22}a_{33}$. Or, deux polynômes (ici d'inconnue $\lambda$) sont égaux si et seulement si leurs coefficients sont égaux. On en déduit que $A$\/ est à diagonale propre si et seulement si $a_{23}a_{32} + a_{21}a_{12}+a_{31}a_{13} = 0$, et $\det A = a_{11}a_{22}a_{33}$\/ (car le coefficient constant vaut $(-1)^3 \det A = - \det A$).
		\item On utilise le résultat trouvé à la question précédente : on a $\det A' = 3 + 4 - 1 = 6$ en développant selon la première ligne, ce qui correspond à $1 \times 1 \times 6$. De plus, on a $3 - 1 - 2 = 0$. On en déduit que $A'$\/ est une matrice à diagonale propre.
		\item On pose $p$\/ la largeur de la matrice $A$, et $q$\/ la largeur de la matrice $C$. On a
			\begin{align*}
				\chi_M(\lambda) &= \det(\lambda I_n - M) =
				\begin{vmatrix}
					\lambda I_p - A & -B\\
					0 & \lambda I_q - C
				\end{vmatrix}\\
				&= \det(\lambda I_p - A) \times \det(\lambda I_q - C) = \chi_A(\lambda) - \chi_C(\lambda)
			\end{align*}
			car le déterminant est triangulaire par blocs. On en déduit que $\chi_M = \chi_A \times \chi_C$.
		\item
			\begin{enumerate}
				\item On pose \[
						M = \left(\begin{array}{ccc|c}
							1&1&1&1\\
							-1&1&1&1\\
							-2&3&6&1\\ \hline
							0&0&0&1
						\end{array}\right)
					.\] D'après les questions précédentes, cette matrice (qui a bien treize coefficients non-nuls) est bien à diagonale propre.
				\item On a
					\begin{align*}
						M \text{ est à diagonale propre } \iff& \chi_M(X) = (X-a)(X-c)(X-e)(X-h)\\
						\iff& \chi_A(X) \times \chi_C(X) = (X-a)(X-c)(X-e)(X-h)\\
						\iff& \begin{cases}
							\chi_A(X) = (X-e)(X-h)\\
							\chi_C(X) = (X-a)(X-c)
						\end{cases}
					\end{align*}
					En effet, s'il y avait un facteur $(X-\beta)$\/ dans la forme factorisée du polynôme caractéristique de $A$\/ (ou respectivement $C$), alors la matrice $A$\/ (ou respectivement $C$) serait diagonale, ce qui est impossible car les coefficients de $A$\/ (respectivement $C$) sont tous non-nuls.

					On pose \[
						M = \left(\begin{array}{cc|cc}
							3&-1&1&1\\
							1&1&1&1\\ \hline
							0&0&2&1\\
							0&0&1&2\\
						\end{array}\right)
					.\]
					En effet, $\chi_{{2\:1\choose1\:2}}(X) = (X-2)^2 - 1 = X^2 -4X + 3 = (X-3)(X-1)$, et $\chi_{{3\:-1\choose 1\:\mathbin{\phantom-}1}}(X) = (X-3)(X-1) - 1 = X^2 - 4X + 2 = (X - 2)^2$. La matrice $M$\/ est donc à diagonale propre.
			\end{enumerate}
		\item Soient $a,b \in \R$. On sait que, si $C$\/ est à diagonale propre, alors $\t C$\/ l'est aussi. (En effet, les coefficients diagonaux sont invariants par transposée, et le polynôme caractéristique aussi.) Or, $\t C = \t(a A) + \t (b I_n) = a \t A + b I_n = C'$. Montrons à présent que $C$\/ est à diagonale propre.
			\begin{itemize}
				\item Si $A$\/ est une matrice à diagonale propre, alors $a A$\/ l'est aussi. En effet, soit $\lambda \in \R$, on a $\chi_{aA}(X) = \det(X I_n - a A) = \det\big(a(\frac{X}{a} I_n - A)\big) = a^n \det(\frac{X}{a} - A) = a^n \chi_A(\frac{X}{a})$, et $\prod_{i=1}^n (X - a a_{i,i}) = \prod_{i=1}^n \big(a (\frac{X}{a} - a_{i,i})\big) = a^n \prod_{i=1}^n (\frac{X}{a} - a_{i,i}) = a^n \chi_A(\frac{X}{a})$. Ainsi, $aA$\/ est une matrice à diagonale propre.
				\item Si $A$\/ est une matrice à diagonale propre, alors $A + I_n$\/ l'est aussi. En effet, $\chi_{A + I_n}(X) = \det(X I_n - A - I_n) = \det\big((X-1) I_n - A\big)$, et $\prod_{i=1}^n \big(X-(a_{i,i}+1)\big) = \prod_{i=1}^n \big((X-1)-a_{i,i}\big)$. Ainsi, $A + I_n$\/ est une matrice à diagonale propre.
			\end{itemize}
			On suppose $b \neq 0$. (On a déjà procédé au cas $b = 0$\/ au premier tiret.) On sait que $C = b\big((\frac{a}{b}A) + I_n\big)$. Si $A$\/ est à diagonale propre, alors $\frac{a}{b}A$\/ l'est aussi, et donc $\frac{a}{b}A + I_n$\/ est aussi une matrice à diagonale propre. On en déduit que $C = b(\frac{a}{b}A + I_n)$\/ est une matrice à diagonale propre.
		\item Soit $A \in \mathcal{E}_n$. Soit $p \in \N$. On pose $C_p = A + \frac{1}{p} I_n$. C'est une matrice à diagonale propre d'après la question (9). La matrice $C_p$\/ est inversible si et seulement si $\det C_p = \det(A - (-\frac{1}{p})I_n \neq 0$, donc si et seulement si $-\frac{1}{p}$\/ n'est pas une valeur propre de $A$. Or, toute matrice à un nombre de valeurs propres fini. Si, pour $p \in \N^*$, $-\frac{1}{p} \not\in \Sp(A)$, alors on pose $p_0 = 1$. Sinon, on pose l'ensemble $P = \{p \in \N^*  \mid -\frac{1}{p} \in \Sp(A)\}$ ; c'est une partie de $\N$\/ non vide, elle admet un maximum. On pose $p_0 = \max(p) + 1$. Par construction, on a bien $p \ge p_0 \implies \det C_p \neq 0$\/ \textit{i.e.} $C_p \in G_n$.
		\item
			\begin{enumerate}
				\item Non. D'après le \textit{théorème spectral}, la matrice $M = {0\:2\choose 2\:0} \in \mathcal{S}_2$\/ est diagonalisable, donc trigonalisable. Mais, $\chi_M(X) = X^2 - 4 \neq (X-0)(X-0)$.
				\item Toute matrice à diagonale propre admet un polynôme caractéristique scindé sur $\R$, elle est donc trigonalisable.
				\item
					\begin{itemize}
						\item Soit $A \in \mathcal{M}_{n}(\R)$. On sait que le polynôme caractéristique est un invariant de similitude. Ainsi, si $A$\/ est semblable à une matrice $B$\/ à diagonale propre, alors $\chi_B = \chi_A$, qui est un polynôme scindé par définition de matrice à diagonale propre.
						\item Soit $A \in \mathcal{M}_n(\R)$\/ une matrice dont le polynôme caractéristique est scindé. Ainsi, elle est trigonalisable : soit $B \in \mathcal{M}_n(\R)$\/ une matrice triangulaire semblable à $A$. Or, une matrice triangulaire est à diagonale propre. En effet, le déterminant $\det(XI_n - B)$\/ est triangulaire, donc égale aux produit des coefficients diagonaux $X - b_{i,i}$.
					\end{itemize}
					On en déduit de cette \textsc{Analyse-Synthèse} qu'une matrice $A$\/ est semblable à une matrice à diagonale propre si et seulement si le polynôme caractéristique de $A$\/ est scindé.
			\end{enumerate}
		\item Soit $A \in \mathcal{M}_n(\R)$. On pose \[
				U = \begin{pNiceMatrix}
					\frac{a_{11}}{2} & 0 & \ldots & 0 \\
					a_{21} & \ddots & \ddots & \vdots\\
					\vdots & \ddots &\ddots & 0\\
					a_{n1}&\ldots&a_{n,n-1}& \frac{a_{nn}}{2}
				\end{pNiceMatrix} \quad\text{ et }\quad
				V = \begin{pNiceMatrix}
					\frac{a_{11}}{2} & a_{12} & \ldots & a_{1n} \\
					0 & \ddots & \ddots & \vdots\\
					\vdots & \ddots &\ddots & a_{n-1,n}\\
					0&\ldots&0& \frac{a_{nn}}{2}
				\end{pNiceMatrix}
			.\] Par construction, on a bien $A = U + V$, et les matrices $U$\/ et $V$\/ sont triangulaires, donc à diagonales propres (\textit{c.f.} question précédente). Ainsi, toute matrice est la somme de deux matrices à diagonales propres.

			Non, $\mathcal{E}_n$\/ n'est pas un sous-espace vectoriel de $\mathcal{M}_n(\R)$. En effet, on a prouvé que toute matrice de $\mathcal{M}_n(\R)$\/ peut être décomposée en somme de deux matrices $U$\/ et $V$\/ de $\mathcal{E}_n$. Or, si $\mathcal{E}_n$\/ est un sous-espace vectoriel, alors $U + V \in \mathcal{E}_n$. Mais, ce n'est pas le cas : il existe des matrices qui ne sont pas à diagonale propre (par exemple ${0\:2\choose 2\:0} \not\in \mathcal{E}_2$).
		\item Soit $A = (a_{i,j})_{i,j \in \left\llbracket 1,n \right\rrbracket} \in \mathcal{M}_n(\R)$. On pose $\t A = (b_{j,k})_{j,k\in\left\llbracket 1,n \right\rrbracket}$, et $\t A A = (c_{i,k})_{i,k\in\left\llbracket 1,n \right\rrbracket}$. On a, pour $i,k \in \left\llbracket 1n \right\rrbracket$, $c_{i,k} = \sum_{j=1}^n a_{i,j}b_{j,k} = \sum_{j=1}^n a_{i,j}a_{k,j}$. Ainsi, \[
				\tr (\t AA) = \sum_{i=1}^n c_{i,i} = \sum_{i=1}^n \sum_{j=1}^n a_{i,j}^2
			.\]
		\item
			\begin{enumerate}
				\item La matrice $A$\/ est semblable à $\mathrm{diag}(\lambda_1, \ldots, \lambda_n)$. De même, la matrice $\t A$\/ est aussi semblable à $\mathrm{diag}(\lambda_1, \ldots, \lambda_n)$\/ (comme c'est sa propre transposée). En effet, soit $P \in \mathrm{GL}_n(\R)$\/ telle que $P^{-1} \cdot A \cdot P = D$\/ où $D = \mathrm{diag}(\lambda_1, \ldots, \lambda_n)$, alors $\t A = \t (P^{-1} \cdot D \cdot P) = (\t P) \cdot {\t D} \cdot (\t P)^{-1}$, donc $\t A$\/ est aussi semblable à $D$. Ainsi, par produit ($P^{-1} \cdot A \cdot P \cdot (\t P) \cdot {\t A} \cdot (\t P) ^{-1} = P^{-1} \cdot A \t A \cdot P = D^2$), la matrice $\t A A$\/ est semblable à $D^2$. D'où, comme la transposée est un invariant de similitude, \[
						\sum_{i=1}^n\sum_{j=1}^n a_{i,j}^2 = \tr(\t AA) = \tr(D^2) = \sum_{i=1}^n \lambda_i^2
					.\]
				\item Si $A$\/ est à diagonale propre, alors, pour $i \in \left\llbracket 1,n \right\rrbracket$, $\lambda_i = a_{i,i}$. D'où \[
					\sum_{i=1}^n \sum_{j=1}^n a_{i,j}^2 = \sum_{i=1}^n a_{i,i}^2 \text{ donc } \sum_{i=1}^n \sum_{\substack{j=1\\j\neq i}}^n a_{i,j}^2 = 0
				.\] Or, il s'agit d'une somme de termes positifs ou nuls, qui est nulle. On en déduit que tous les termes de la sommes sont nuls, \textit{i.e.} $\forall i$, $\forall j \neq i$, $a_{i,j} = 0$. La matrice $A$\/ est donc diagonale. L'inclusion réciproque est vraie comme toute matrice diagonale est une matrice à diagonale propre.
			\end{enumerate}
		\item
			\begin{enumerate}
				\item Comme la matrice $A$\/ est anti-symétrique, sa diagonale est nulle. On en déduit que toutes ses valeurs propres sont nulles. Ainsi, $\chi_A(X) = X^n$. Or, d'après le théorème de \textsc{Cayley} \& \textsc{Hamilton}, le polynôme caractéristique est annulateur de la matrice $A$. D'où $\chi_A(A) = 0$. On en déduit que $A^n = 0$.

					Comme $A$\/ est anti-symétrique, $\t A = -A$, et donc $(\t A A)^n = (- A^2)^n = (-1)^n \cdot A^n \cdot A^n = 0$.
				\item La matrice $\t A A$\/ est symétrique. En effet, $\t(\t A A) = \t A \cdot \t(\t A) = \t A A$. Ainsi, d'après le \textit{théorème spectral}, elle est diagonalisable en $D = \mathrm{diag}(\lambda_1, \ldots, \lambda_n)$. Or, $(\t A A)^n = 0$, donc les valeurs propres sont toutes nulles. On en déduit que $\t A A$\/ est semblable à la matrice nulle, \textit{i.e.} $\t A A = 0$.
				\item On applique la trace : $\tr (\t A A) = \sum_{i=1}^n \sum_{j=1}^n a_{i,j}^2 = 0$. Or, comme c'est une somme de termes positifs ou nuls, ils sont tous nuls, la matrice $A$\/ est donc la matrice nulle.
			\end{enumerate}
		\item On a $\dim \mathcal{A}_n = \frac{n(n-1)}{2}$.
		\item D'après les questions (15abc), $\mathcal{E}_n \cap \mathcal{A}_n = \{0\}$, donc $F \cap \mathcal{A}_n = \{0\}$\/ (car $0 \in F$, car c'est un espace vectoriel). La somme est donc directe. Ainsi, $\dim F + \dim \mathcal{A}_n = \dim(F \oplus \mathcal{A}_n) \le \dim \mathcal{M}_n(\R) = n^2$. Ainsi, $\dim F \le n^2 - \frac{n(n-1)}{2} = \frac{n(n+1)}{2}$. L'espace des matrices triangulaires inférieures ($\mathcal{T}^I_n$) est un sous-espace vectoriel de $\mathcal{M}_n(\R)$. On a bien $\mathcal{T}^I_n \subset \mathcal{E}_n$. En effet, toute matrice triangulaire inférieure est à diagonale propre. Et, on a $\dim \mathcal{T}^I_n = \frac{n(n+1)}{2}$.
		\item Soit $m < n$. On considère l'ensemble $\mathcal{Z}_m$\/ des matrices de la forme $M = {A\:B\choose 0\:D}$ où $A$\/ et $D$\/ sont triangulaires inférieures de tailles respectives $m \times m$\/ et $(n-m)\times (n-m)$\/. Ainsi,  $A$\/ et $D$\/ sont à diagonale propre, donc $M$\/ aussi. L'ensemble $\mathcal{Z}_m$\/ est bien un sous-espace vectoriel, car l'ensemble des matrices triangulaires en forme un. On a $\dim \mathcal{Z}_m = \dim\mathcal{T}^I_m + \dim\mathcal{T}^I_{n-m} +\dim \mathcal{M}_{m,n-m}(\R)$, car la somme est directe. Ainsi,
			\begin{align*}
				2\dim \mathcal{Z}_m &= 2 \dim \mathcal{T}^I_m + 2\dim \mathcal{T}^I_{n-m} + 2\dim \mathcal{M}_{m,n-m}(\R)\\
				&= m(m+1) + (n-m)(n-m+1) + 2m(n-m) \\
				&= \cancel{m^2} + \cancel{m} + n^2 -\cancel{2nm} + \cancel{m^2} + n - \cancel{m} + \cancel{2nm} - \cancel{2m^2}  \\
				&= n^2 + n \\
				&= n(n+1) \\
			\end{align*}
			On en déduit que, pour tout $m \in \left\llbracket 1,n-1 \right\rrbracket$, l'espace $\mathcal{Z}_m$\/ est de dimension $\frac{n(n+1)}{2}$, est composé de matrices à diagonales propres, mais n'est pas composé uniquement de matrices triangulaires : la matrice $M$, ci-dessous, n'est pas triangulaire mais \[
				M = \begin{pmatrix}
					0 & \cdots & \cdots & 0 & 1\\
					\vdots & \ddots & & \vdots & 0\\
					\vdots & & \ddots & \vdots & \vdots\\
					0 & \cdots & \cdots & 0 & 0
				\end{pmatrix} \in \mathcal{Z}_1
			.\]
	\end{enumerate}
\end{document}
