\begin{prv}
	On suppose $\vec{u} \tendsto{n\to \infty} \vec{\ell}$. D'où, $\|\vec{u}_n - \vec{\ell}\| \tendsto{n\to \infty} 0$, et donc \[
		\forall \varepsilon > 0,\: \exists N \in \N,\: \forall n \ge N, \quad \|\vec{u}_n - \vec{\ell}\| \le \varepsilon
	.\] Soit $(\vec{v}_n) = (\vec{u}_{\varphi(n)})$\/ une suite extraite. On veut montrer que $\vec{v}_n \tendsto{n\to \infty} \vec{\ell}$.
	Or, $\varphi(n) \ge n$, pour tout $n \in \N$, d'où $\varphi(n) \ge N$\/ pour $n \ge N$. D'où, $\|\vec{u}_{\varphi(n)} - \vec{\ell}\| \le \varepsilon$\/ et donc $\|\vec{v}_n - \vec{\ell}\| \le \varepsilon$.
\end{prv}

\section{Comparer des normes}

\begin{defn}
	Soient $N$\/ et $\|\cdot\|$\/ deux normes sur un espace vectoriel $E$.
	On dit que ces deux normes sont \textit{équivalentes} s'il existe deux réels $\alpha$\/ et $\beta$\/ tels que \[
		\forall \vec{x} \in E,\quad\quad N(\vec{x}) \le \alpha\cdot \|\vec{x}\|\quad \text{ et }\quad \|\vec{x}\| \le \beta \cdot N(\vec{x})
	.\]
\end{defn}

\begin{rmk}
	\begin{enumerate}
		\item Cette relation entre deux normes est une relation d'équivalence. En effet, elle est réflexive, symétrique et transitive.
		\item Si deux normes sont équivalentes, alors ce qui converge pour l'une, converge aussi pour l'autre : \[
				N(\vec{x} - \vec{a}) \to 0 \quad \iff \quad \|\vec{x} - \vec{a}\| \to 0
			.\] En effet, si $N(\vec{x} - \vec{a})$\/ tend vers 0, alors $\|\vec{x}-\vec{a}\|$\/ aussi : $\|\vec{x} - \vec{a}\| \le \beta\cdot N(\vec{x}-\vec{a})$.
		\item De même, être ou ne pas être borné est indépendant du choix de la norme, si les normes sont équivalentes.
		\item Les trois normes classiques sur $\R^n$\/ sont équivalentes. En effet, on a \[
			\forall \vec{x} \in \R^n, \quad\quad \|\vec{x}\|_\infty \le \|\vec{x}\|_1 \le n \cdot \|\vec{x}\|_\infty \quad \text{et}\quad \|\vec{x}\|_\infty \le \|\vec{x}\|_2 \le \sqrt{n} \|\vec{x}\|_\infty,
		\] d'après l'exercice 7. Mais, ceci est faux dans l'espace $\mathcal{C}([0,1])$, d'après l'exercice 11.
	\end{enumerate}

	Pour des normes équivalentes,
	\begin{itemize}
		\item la nature d'une suite ne dépend pas de la norme,
		\item la limite d'une suite ne dépend pas de la norme,
		\item le caractère borné d'une suite ne dépend pas de la norme.
	\end{itemize}
\end{rmk}

\begin{thm}
	Sur un espace vectoriel de dimension \textbf{finie}, toutes les normes sont équivalentes.
\end{thm}

\begin{prv}
	Voir l'annexe B.
\end{prv}

\begin{exo}
	\textsl{
		\begin{enumerate}
			\item Soit $E$\/ l'espace vectoriel de fonctions de $[0,1]$\/ vers $\R$\/ continues : $E = \mathcal{C}([0,1], \R)$. Déterminer un réel $\alpha$\/ tel que \[
					\forall f \in E,\quad\quad \|f\|_1 \le \alpha \|f\|_\infty
				.\]
			\item Pour tout $n \in \N$, on considère les fonctions $f_n$\/ et $g_n$\/ définies sur $[0,1]$\/ et représentées sur le poly. Étudier $\|f_n\|_1$\/ et $\|f_n\|_\infty$\/ ainsi que $\|g_n\|_1$\/ et $\|g_n\|_\infty$. Conclure.
		\end{enumerate}
	}

	\begin{enumerate}
		\item On sait que $\|f\|_1 = \int_{0}^{1} |f(t)|~\mathrm{d}t$. Et, $\forall t \in [0,1]$, $|f(t)| \le \sup_{t \in [0,1]}\: |f(t)| = \|f\|_\infty$. D'où,
			\begin{align*}
				\|f\|_1 = \int_{0}^{1} |f(t)|~\mathrm{d}t &\le \int_{0}^{1} \|f\|_\infty~\mathrm{d}t\\
				&\le (1-0) \times \|f\|_\infty\\
			\end{align*}
		\item On a $\|f_n\|_\infty = 1$\/ et $\|f_n\|_1 = \frac{1}{2n+2}$.
			De même, on a $\|g_n\|_\infty = 2n + 2$, et $\|g_n\|_1 = 1$.
			On a déjà montré que $\|\cdot\|_1 \le 1 \times \|\cdot \|_\infty$.
			On veut savoir s'il existe un réel $\beta$, tel que $\|\cdot\|_\infty \le \beta \times \|\cdot \|_1$.
			Par l'absurde, supposons qu'il existe un réel $\beta$\/ tel que $\|\cdot\|_\infty \le \beta \times \|\cdot \|_1$.
			En particulier, $\|f_n\|_\infty \le \beta \cdot \|f_n\|_1$, d'où $1 \le \beta \times \frac{1}{2n + 2}$.
			Les inégalités larges passent à la limite, d'où $1 \le 0$, ce qui est absurde.
	\end{enumerate}
\end{exo}

\begin{crlr}[coordonnée par coordonnée]
	Soit $E$\/ un espace vectoriel de dimension \textbf{finie}, et soient $(\vec{u}_n)$\/ une suite de vecteurs de $E$\/ et $\vec{\ell} $\/ un vecteur.
	La suite $(\vec{u}_n)$\/ tend vers $\vec{\ell}$\/ si, et seulement si, chaque coordonnée de $\vec{u}_n$\/ tend vers chaque coordonnée de $\vec{\ell}$.
\end{crlr}

\begin{prv}
	Soit $\mathcal{B} = (\vec{e}_1, \ldots, \vec{e}_d)$\/ une base de $E$. Soit $N$\/ la norme de $E$\/ définie comme \begin{align*}
		N: E &\longrightarrow \R^+ \\
		x_1 \vec{e}_1 + \cdots + x_d \vec{e}_d &\longmapsto \max(|x_1|, \ldots, |x_d|)
	\end{align*}
	On veut montrer que $\vec{u}_n \to \vec{\ell}$\/ si, et seulement si, pour $i \in \llbracket 1,d \rrbracket$, $x_i \to \ell_i$.
	\begin{align*}
		\vec{u}_n \tendsto{n \to +\infty} \vec{\ell} \iff& N(\vec{u}_n - \vec{\ell}) \tendsto{n\to \infty} 0 \\
		\iff& \max(|u_{n,1} - \ell_1|, \ldots,{|u_{n,d} - \ell_d|}) \tendsto{n\to \infty} 0 \\
		\iff& \forall i \in \llbracket 1,d \rrbracket, \quad |u_{n,i} - \ell| \tendsto{n\to \infty} 0 \\
	\end{align*}
\end{prv}

\begin{exo}
	On a $E = \mathcal{M}_{2,2}(\R)$, espace de dimension finie.
	D'où, toutes les normes sur $E$\/ sont équivalentes.
	En particulier, $A_n \to L$\/ si, et seulement si chaque élément de matrice de $A_n$\/ tend vers chaque élément de matrice de $L$.
	Ici, \[
		A_n = \begin{pmatrix}
			1 & -a / n\\
			a / n & 1
		\end{pmatrix}^n
	.\]
	On a \[
		\begin{pmatrix}
			1 & - \frac{a}{n}\\
			\frac{a}{n} & 1
		\end{pmatrix} = \sqrt{1 + \frac{a^2}{n^2}}  \cdot \underbrace{\begin{pmatrix}
			\frac{1}{\sqrt{1 + a^2 / n^2}} & \frac{- a / n}{\sqrt{1 + a^2 / n^2}}\\
			\frac{a / n}{\sqrt{1 + a^2 / n^2}} & \frac{1}{\sqrt{1 + a^2 / n^2}}
		\end{pmatrix}}_{(C_1, C_2) \text{ b.o.n}} = \sqrt{1 + \frac{a^2}{n^2}} \begin{pmatrix}
			\cos \theta_n & -\sin \theta_n\\
			\sin \theta_n & \cos \theta_n
		\end{pmatrix} 
	.\] D'où,
	\begin{align*}
		A_n = \begin{pmatrix}
			1 & - \frac{a}{n}\\
			\frac{a}{n} & 1
		\end{pmatrix}^n &= \left( \sqrt{1 + \frac{a^2}{n^2}} \right)^n \cdot \begin{pmatrix}
			\cos \theta_n & -\sin\theta_n\\
			\sin \theta_n & \cos \theta_n
		\end{pmatrix}^n\\
		&= \left( \sqrt{1 + \frac{a^2}{n^2}} \right)^n \cdot \begin{pmatrix}
			\cos (n\theta_n) & -\sin(n\theta_n)\\
			\sin (n\theta_n) & \cos (n\theta_n)
		\end{pmatrix}\\
	\end{align*}
	On veut montrer que \[
		\begin{cases}
			\left(\sqrt{1 + \frac{a^2}{n^2}}\right)^n \; \cos(n \theta_n) \tendsto{n\to \infty} \cos a,\\
			\left(\sqrt{1 + \frac{a^2}{n^2}}\right)^n \; \sin(n\theta_n) \tendsto{n\to \infty} \sin a.
		\end{cases}
	\]
	On a $\left( \sqrt{1 + \frac{a^2}{n^2}} \right)^n = \exp\left( n \ln \sqrt{1 + \frac{a^2}{n^2}} \right)$.
	Or, \[
		n \ln \sqrt{1 + \frac{a^2}{n^2}}  = \frac{1}{2} n \ln \left(1 + \frac{a^2}{n^2}\right) = \frac{1}{2}n\left( \frac{a^2}{n^2} + \po\left( \frac{a^2}{n^2} \right)\right) = \frac{a^2}{2n} + \po\left( \frac{1}{n} \right) \tendsto{n\to \infty} 0
	.\] D'où, $\left( \sqrt{1 + \frac{a^2}{n^2}}\right)^n \tendsto{n\to \infty} 1$\/ par continuité de l'exponentielle.
	Il nous reste à montrer que $n \theta_n \tendsto{n\to \infty} a$, car, ensuite par continuité du cosinus et du sinus : \[
		\begin{cases}
			\sin(n \theta_n) \tendsto{n\to \infty} \sin a\\
			\cos (n \theta_n) \tendsto{n\to \infty} \cos a
		\end{cases}
	.\]
	On a $\cos \theta_n = 1 / \sqrt{1 + a^2 / n^2}$\/ et $\sin \theta_n = (a / n) / \sqrt{1 + a^2 / n^2}$. On a donc $\sin \theta_n \to 0$, d'où $\sin \theta_n \to 0$\/ par continuité de $\Arcsin$, d'où $\sin \theta_n \sim \theta_n$.
	Et, donc \[
		\theta_n \simi_{n\to \infty} \frac{a/n}{\sqrt{1 + \frac{a^2}{n^2}}}
	.\] D'où, $n \theta_n \sim a / \sqrt{1 + a^2 / n^2} \tendsto{n\to \infty} a$.
\end{exo}

\section{Adhérence}

\begin{defn}
	Soit $A$\/ une partie de $E$, un espace vectoriel. On dit qu'un vecteur $\vec{\ell} \in E$\/ est \textit{adhérent} à $A$, si toute boule centrée en $\vec{\ell}$\/ rencontre $A$\/ : \[
		\forall \varepsilon > 0,\quad\quad B(\vec{\ell},\varepsilon) \cap A \neq \O
	.\] L'\textit{adhérence} de $A$, notée $\bar{A}$, est l'ensemble des vecteurs adhérents à $A$.
\end{defn}

\begin{figure}[H]
	\centering
	\incfig{point-adhérent}
	\caption{Point adhérent}
\end{figure}

On a toujours $A\subset \bar{A}$, mais pas forcément $\bar{A} \subset  A$ (\textit{c.f.}\ exemple ci-après).

\begin{exm}
	\begin{itemize}
		\item Soient $I = [0, 1]$, $J = {]0,1]}$\/ et $K = {]0,+\infty[}$\/ trois intervalles de $\R$.
			On a $\bar{I} = I$, $\bar{J} = I$\/ et $\bar{K} = [0, +\infty[$.
		\item L'adhérence $\overline{B(\vec{a},r)}$\/ d'une boule ouverte $B(\vec{a}, r)$\/ est la boule fermée de même centre $\vec{a}$\/ et de même rayon $r$ : $\bar{B}(\vec{a}, r)$.
	\end{itemize}
\end{exm}

\begin{prop}[Caractérisation séquentielle de l'adhérence]
	Un vecteur $\vec{\ell} \in E$\/ est adhérent à $A \subset E$ si, et seulement si $\vec{\ell}$\/ est la limite d'une suite $(\vec{u}_n)_{n\in\N}$\/ d'éléments de $A$.
\end{prop}

\begin{prv}~\\[-\baselineskip]
	\begin{itemize}
		\item[``$\implies$''] On suppose qu'il existe une suite $(\vec{u}_n)_{n\in\N}$\/ de $A$ telle que $\vec{u}_n \to \vec{a}$. Soit $\varepsilon > 0$.
			Comme $\vec{u}_n \to \vec{a}$ (par hypothèse), il existe un rang $N \in \N$\/ tel que, pour tout $n \ge N$, $\|\vec{u}_, - \vec{a}\| < \varepsilon$.
			D'où, $\vec{u}_n \in B(\vec{a}, \varepsilon)$. Or, par hypothèse, $\vec{u}_n \in A$. On en déduit que \[
				B(\vec{a}, \varepsilon) \cap A \neq \O
			.\]
		\item[``$\impliedby$'']
			Réciproquement, on suppose que, pour tout $\varepsilon > 0$, $B(\vec{a}, \varepsilon) \cap A \neq \O$.
			Soit $n \in \N^*$\/ et soit $\varepsilon = \frac{1}{n}$.
			D'où, par hypothèse, $B(\vec{a}, \varepsilon) \cap A \neq \O$.
			On choisit un élément $\vec{u}_n \in B(\vec{a}, \varepsilon) \cap A$, qui est non vide.
			D'où, $\vec{u}_n \in A$\/ et $\|\vec{u}_n - \vec{a}\| < \frac{1}{n}$.
			Ainsi, pour tout $n \in \N$, $\vec{u}_n \in A$\/ et $\vec{u}_n \to \vec{a}$.
	\end{itemize}
\end{prv}

\begin{defn}
	Soit $A$\/ une partie d'un espace vectoriel $E$. On dit que $A$\/ est \textit{dense} dans $E$\/ si $\bar{A} = E$. Autrement dit, tout vecteur de $E$\/ est adhérent à $A$. Ou encore, tout vecteur de $E$\/ est la limite d'une suite de vecteurs de $A$.
\end{defn}

\begin{rap}[Caractère archimédien de $\R$]
	Pour tout $\varepsilon \in \R^+_*$, il existe un entier $n \in \N$\/ tel que $n \cdot \varepsilon > 1$.
\end{rap}

\begin{rap}[Théorème d'approximation de Weierstrass]
	Si $f \in \mathcal{C}([a,b])$, il existe une suite $(P_n)_{n\in\N}$\/ de polynômes convergent uniformément vers $f$\/ sur $[a,b]$\/ :  \[
		0 \xleftarrow[\infty \gets n]{} \|P_n-f\|_\infty = \sup_{t \in [a,b]}\ \big|P_n(t) - f(t)\big| \tendsto{n\to \infty} 0
	.\]
\end{rap}


\marginpar{\color{cyan}Tartes à la crème}
\begin{exm}
	\begin{enumerate}
		\item $\Q$\/ est dense dans $\R$, et $\R \setminus \Q$\/ est dense dans $\R$.
		\item D'après le théorème d'approximation de Weierstrass, l'ensemble des fonctions polynomiales est dense dans l'espace vectoriel $\mathcal{C}([a,b])$\/ des fonctions continues sur un segment muni de la {\color{red}norme infinie} (par la caractérisation séquentielle de l'adhérence).
	\end{enumerate}
\end{exm}
