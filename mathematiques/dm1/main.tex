\documentclass[a4paper]{article}

\usepackage[margin=1in]{geometry}
\usepackage[utf8]{inputenc}
\usepackage[T1]{fontenc}
\usepackage{mathrsfs}
\usepackage{textcomp}
\usepackage[french]{babel}
\usepackage{amsmath}
\usepackage{amssymb}
\usepackage{cancel}
\usepackage{frcursive}
\usepackage[inline]{asymptote}
\usepackage{tikz}
\usepackage[european,straightvoltages,europeanresistors]{circuitikz}
\usepackage{tikz-cd}
\usepackage{tkz-tab}
\usepackage[b]{esvect}
\usepackage[framemethod=TikZ]{mdframed}
\usepackage{centernot}
\usepackage{diagbox}
\usepackage{dsfont}
\usepackage{fancyhdr}
\usepackage{float}
\usepackage{graphicx}
\usepackage{listings}
\usepackage{multicol}
\usepackage{nicematrix}
\usepackage{pdflscape}
\usepackage{stmaryrd}
\usepackage{xfrac}
\usepackage{hep-math-font}
\usepackage{amsthm}
\usepackage{thmtools}
\usepackage{indentfirst}
\usepackage[framemethod=TikZ]{mdframed}
\usepackage{accents}
\usepackage{soulutf8}
\usepackage{mathtools}
\usepackage{bodegraph}
\usepackage{slashbox}
\usepackage{enumitem}
\usepackage{calligra}
\usepackage{cinzel}
\usepackage{BOONDOX-calo}

% Tikz
\usetikzlibrary{babel}
\usetikzlibrary{positioning}
\usetikzlibrary{calc}

% global settings
\frenchspacing
\reversemarginpar
\setuldepth{a}

%\everymath{\displaystyle}

\frenchbsetup{StandardLists=true}

\def\asydir{asy}

%\sisetup{exponent-product=\cdot,output-decimal-marker={,},separate-uncertainty,range-phrase=\;à\;,locale=FR}

\setlength{\parskip}{1em}

\theoremstyle{definition}

% Changing math
\let\emptyset\varnothing
\let\ge\geqslant
\let\le\leqslant
\let\preceq\preccurlyeq
\let\succeq\succcurlyeq
\let\ds\displaystyle
\let\ts\textstyle

\newcommand{\C}{\mathds{C}}
\newcommand{\R}{\mathds{R}}
\newcommand{\Z}{\mathds{Z}}
\newcommand{\N}{\mathds{N}}
\newcommand{\Q}{\mathds{Q}}

\renewcommand{\O}{\emptyset}

\newcommand\ubar[1]{\underaccent{\bar}{#1}}

\renewcommand\Re{\expandafter\mathfrak{Re}}
\renewcommand\Im{\expandafter\mathfrak{Im}}

\let\slantedpartial\partial
\DeclareRobustCommand{\partial}{\text{\rotatebox[origin=t]{20}{\scalebox{0.95}[1]{$\slantedpartial$}}}\hspace{-1pt}}

% merging two maths characters w/ \charfusion
\makeatletter
\def\moverlay{\mathpalette\mov@rlay}
\def\mov@rlay#1#2{\leavevmode\vtop{%
   \baselineskip\z@skip \lineskiplimit-\maxdimen
   \ialign{\hfil$\m@th#1##$\hfil\cr#2\crcr}}}
\newcommand{\charfusion}[3][\mathord]{
    #1{\ifx#1\mathop\vphantom{#2}\fi
        \mathpalette\mov@rlay{#2\cr#3}
      }
    \ifx#1\mathop\expandafter\displaylimits\fi}
\makeatother

% custom math commands
\newcommand{\T}{{\!\!\,\top}}
\newcommand{\avrt}[1]{\rotatebox{-90}{$#1$}}
\newcommand{\bigcupdot}{\charfusion[\mathop]{\bigcup}{\cdot}}
\newcommand{\cupdot}{\charfusion[\mathbin]{\cup}{\cdot}}
%\newcommand{\danger}{{\large\fontencoding{U}\fontfamily{futs}\selectfont\char 66\relax}\;}
\newcommand{\tendsto}[1]{\xrightarrow[#1]{}}
\newcommand{\vrt}[1]{\rotatebox{90}{$#1$}}
\newcommand{\tsup}[1]{\textsuperscript{\underline{#1}}}
\newcommand{\tsub}[1]{\textsubscript{#1}}

\renewcommand{\mod}[1]{~\left[ #1 \right]}
\renewcommand{\t}{{}^t\!}
\newcommand{\s}{\text{\calligra s}}

% custom units / constants
%\DeclareSIUnit{\litre}{\ell}
\let\hbar\hslash

% header / footer
\pagestyle{fancy}
\fancyhead{} \fancyfoot{}
\fancyfoot[C]{\thepage}

% fonts
\let\sc\scshape
\let\bf\bfseries
\let\it\itshape
\let\sl\slshape

% custom math operators
\let\th\relax
\let\det\relax
\DeclareMathOperator*{\codim}{codim}
\DeclareMathOperator*{\dom}{dom}
\DeclareMathOperator*{\gO}{O}
\DeclareMathOperator*{\po}{\text{\cursive o}}
\DeclareMathOperator*{\sgn}{sgn}
\DeclareMathOperator*{\simi}{\sim}
\DeclareMathOperator{\Arccos}{Arccos}
\DeclareMathOperator{\Arcsin}{Arcsin}
\DeclareMathOperator{\Arctan}{Arctan}
\DeclareMathOperator{\Argsh}{Argsh}
\DeclareMathOperator{\Arg}{Arg}
\DeclareMathOperator{\Aut}{Aut}
\DeclareMathOperator{\Card}{Card}
\DeclareMathOperator{\Cl}{\mathcal{C}\!\ell}
\DeclareMathOperator{\Cov}{Cov}
\DeclareMathOperator{\Ker}{Ker}
\DeclareMathOperator{\Mat}{Mat}
\DeclareMathOperator{\PGCD}{PGCD}
\DeclareMathOperator{\PPCM}{PPCM}
\DeclareMathOperator{\Supp}{Supp}
\DeclareMathOperator{\Vect}{Vect}
\DeclareMathOperator{\argmax}{argmax}
\DeclareMathOperator{\argmin}{argmin}
\DeclareMathOperator{\ch}{ch}
\DeclareMathOperator{\com}{com}
\DeclareMathOperator{\cotan}{cotan}
\DeclareMathOperator{\det}{det}
\DeclareMathOperator{\id}{id}
\DeclareMathOperator{\rg}{rg}
\DeclareMathOperator{\rk}{rk}
\DeclareMathOperator{\sh}{sh}
\DeclareMathOperator{\th}{th}
\DeclareMathOperator{\tr}{tr}

% colors and page style
\definecolor{truewhite}{HTML}{ffffff}
\definecolor{white}{HTML}{faf4ed}
\definecolor{trueblack}{HTML}{000000}
\definecolor{black}{HTML}{575279}
\definecolor{mauve}{HTML}{907aa9}
\definecolor{blue}{HTML}{286983}
\definecolor{red}{HTML}{d7827e}
\definecolor{yellow}{HTML}{ea9d34}
\definecolor{gray}{HTML}{9893a5}
\definecolor{grey}{HTML}{9893a5}
\definecolor{green}{HTML}{a0d971}

\pagecolor{white}
\color{black}

\begin{asydef}
	settings.prc = false;
	settings.render=0;

	white = rgb("faf4ed");
	black = rgb("575279");
	blue = rgb("286983");
	red = rgb("d7827e");
	yellow = rgb("f6c177");
	orange = rgb("ea9d34");
	gray = rgb("9893a5");
	grey = rgb("9893a5");
	deepcyan = rgb("56949f");
	pink = rgb("b4637a");
	magenta = rgb("eb6f92");
	green = rgb("a0d971");
	purple = rgb("907aa9");

	defaultpen(black + fontsize(8pt));

	import three;
	currentlight = nolight;
\end{asydef}

% theorems, proofs, ...

\mdfsetup{skipabove=1em,skipbelow=1em, innertopmargin=6pt, innerbottommargin=6pt,}

\declaretheoremstyle[
	headfont=\normalfont\itshape,
	numbered=no,
	postheadspace=\newline,
	headpunct={:},
	qed=\qedsymbol]{demstyle}

\declaretheorem[style=demstyle, name=Démonstration]{dem}

\newcommand\veczero{\kern-1.2pt\vec{\kern1.2pt 0}} % \vec{0} looks weird since the `0' isn't italicized

\makeatletter
\renewcommand{\title}[2]{
	\AtBeginDocument{
		\begin{titlepage}
			\begin{center}
				\vspace{10cm}
				{\Large \sc Chapitre #1}\\
				\vspace{1cm}
				{\Huge \calligra #2}\\
				\vfill
				Hugo {\sc Salou} MPI${}^{\star}$\\
				{\small Dernière mise à jour le \@date }
			\end{center}
		\end{titlepage}
	}
}

\newcommand{\titletp}[4]{
	\AtBeginDocument{
		\begin{titlepage}
			\begin{center}
				\vspace{10cm}
				{\Large \sc tp #1}\\
				\vspace{1cm}
				{\Huge \textsc{\textit{#2}}}\\
				\vfill
				{#3}\textit{MPI}${}^{\star}$\\
			\end{center}
		\end{titlepage}
	}
	\fancyfoot{}\fancyhead{}
	\fancyfoot[R]{#4 \textit{MPI}${}^{\star}$}
	\fancyhead[C]{{\sc tp #1} : #2}
	\fancyhead[R]{\thepage}
}

\newcommand{\titletd}[2]{
	\AtBeginDocument{
		\begin{titlepage}
			\begin{center}
				\vspace{10cm}
				{\Large \sc td #1}\\
				\vspace{1cm}
				{\Huge \calligra #2}\\
				\vfill
				Hugo {\sc Salou} MPI${}^{\star}$\\
				{\small Dernière mise à jour le \@date }
			\end{center}
		\end{titlepage}
	}
}
\makeatother

\newcommand{\sign}{
	\null
	\vfill
	\begin{center}
		{
			\fontfamily{ccr}\selectfont
			\textit{\textbf{\.{\"i}}}
		}
	\end{center}
	\vfill
	\null
}

\renewcommand{\thefootnote}{\emph{\alph{footnote}}}

% figure support
\usepackage{import}
\usepackage{xifthen}
\pdfminorversion=7
\usepackage{pdfpages}
\usepackage{transparent}
\newcommand{\incfig}[1]{%
	\def\svgwidth{\columnwidth}
	\import{./figures/}{#1.pdf_tex}
}

\pdfsuppresswarningpagegroup=1
\ctikzset{tripoles/european not symbol=circle}

\newcommand{\missingpart}{{\large\color{red} Il manque quelque chose ici\ldots}}


\begin{document}
	\begin{center}
		\Huge $\mathbf{DM_1}$\/ Mathématiques
	\end{center}
	
	\begin{enumerate}
		\item
			\begin{enumerate}
				\item Soit $n \ge 2$. On a \[
						I_n = \int_{0}^{\frac{\pi}{2}} \sin \theta \times \sin^{n-1}\theta ~\mathrm{d}\theta
					\] et donc, par intégration par parties, on obtient
					\begin{align*}
						I_n &= \big[-\cos\theta \times \sin^{n-1}\theta\big]_{0}^{\frac{\pi}{2}} - \int_{0}^{\frac{\pi}{2}} (-\cos\theta) \times (n-1)\sin^{n-2}\theta ~\mathrm{d}\theta\\
						&= \boxed{(n-1)\int_{0}^{\frac{\pi}{2}} \cos^2\thetaù\sin^{n-2}\theta ~\mathrm{d}\theta.} \\
					\end{align*}
				\item Soit $n \ge 2$. D'après le théorème de {\sc Pythagore}, on sait que, pour tout $\theta \in \left[ 0, \frac{\pi}{2} \right]$, on a $\cos^2\theta = 1 - \sin^2\theta$. Ainsi, avec l'expression de $I_n$\/ démontrée dans la question précédente, on obtient
					\begin{align*}
						I_n &= (n-1) \int_{0}^{\frac{\pi}{2}} (1 - \sin^2\theta)\sin^{n-2}\theta~\mathrm{d}\theta\\
						&= (n-1) \bigg(\int_{0}^{\frac{\pi}{2}} \sin^{n-2}\theta ~\mathrm{d}\theta - \int_{0}^{\frac{\pi}{2}} \sin^{n}\theta~\mathrm{d}\theta \bigg) \\
						&= (n-1)(I_{n-2}- I_n). \\
					\end{align*}
					On en déduit que \[
						(n-1)I_{n-2} = (n-1+1) I_n \quad \text{d'où} \quad \boxed{I_n = \frac{n-1}{n}I_{n-2}.}
					\]
				\item Soit~$n \ge 2$ et soit~$\theta \in \left[ 0, \frac{\pi}{2} \right]$. On sait donc que $\sin \theta \in [0, 1]$. Ainsi, \[
						\sin^{n-1}\theta \le \sin^{n}\theta
					\] et, donc \[
						\boxed{I_{n-1} = \int_{0}^{\frac{\pi}{2}} \sin^{n-1}\theta~\mathrm{d}\theta \le \int_{0}^{\frac{\pi}{2}}  \sin^n\theta~\mathrm{d}\theta = I_n.}
					\]
				\item Comme la suite $(I_n)$\/ est décroissante, on a, pour tout $n \in \N^*$, $I_{n-1} \le I_n$\/ et, en composant avec la fonction inverse, et et multipliant par $I_n$, on obtient \[
						\boxed{\frac{I_n}{I_{n-1}} \le \frac{I_n}{I_n} = 1}
					.\]
					Également, d'après la question précédente, pour tout $n \ge 2$, on a \[
						\frac{n-1}{n} = \frac{I_n}{I_{n-2}} \le \frac{I_{n-1}}{I_{n-2}}
					.\] car la suite $(I_n)$\/ est décroissante. On effectue un changement le variables $n\to n+1$\/ et on obtient, pour tout $n \in \N^*$, \[
						\boxed{\frac{n}{n+1} \le \frac{I_n}{I_{n-1}}.}
					\]
				\item Soit $n \ge 2$. On a, d'après la question~(c), \[
						n\,I_n = (n-1)I_{n-2}
					.\] Or, en multipliant des deux côtés par~$I_{n-1}$, on obtient \[
						n\,I_n\,I_{n-1} = (n-1)I_{n-1}I_{n-2}
					.\] Comme ce résultat est vrai pour toute valeur de~$n$\/ supérieure à~2, la suite~$(n I_n I_{n-1})$\/ est constante. Nommons cette constante $\alpha$.

					On pose $n = 2$, ainsi on a $2\,I_2\,I_1 = \alpha$.
					On calcule donc $I_1$\/ : \[
						I_1 = \int_{0}^{\frac{\pi}{2}} \sin \theta ~\mathrm{d}\theta = \big[-\cos\theta\big]_0^{\frac{\pi}{2}} = 1
					.\] Puis, $I_2$\/ : \[
						I_2 = \int_{0}^{\frac{\pi}{2}} \sin^2\theta~\mathrm{d}\theta = \int_{0}^{\frac{\pi}{2}} \frac{1 - \cos(2\theta)}{2}~\mathrm{d}\theta = \frac{1}{2} \bigg(\big[\theta\big]_0^{\frac{\pi}{2}} - \frac{1}{2} \big[\sin 2\theta\big]_0^{\frac{\pi}{2}}\bigg) = \frac{\pi}{4} + 0
					.\]
					On en déduit donc que \[
						\boxed{\alpha = \frac{\pi}{2}.}
					\]
				\item On sait, d'après la question~(d), que, pour tout $n \ge 2$, on a \[
						\frac{n}{n+1} \le \frac{I_n}{I_{n-1}} \le \frac{I_n}{I_n} = 1
					.\] Or, en passant à la limite, pour $n$\/ tendant vers $+\infty$, on obtient, par le théorème des gendarmes \[
						\frac{I_n}{I_{n-1}} \tendsto{n\to +\infty} 1 \quad \text{et donc} \quad I_n \simi_{n\to +\infty} I_{n-1}.
					\] Or, comme la suite $(n\,I_n\,I_{n-1})$\/ est constante et vaut $\frac{\pi}{2}$, on a \[
						\frac{\pi}{2} = n\,I_n\,I_{n-1} \sim n\:{I_n}^2
						\quad\text{d'où}\quad
						{I_n}^2 \sim \frac{\pi}{2n}.
					\]
					On en déduit que \[
						\boxed{I_n \sim \sqrt{\frac{\pi}{2n}}.}
					\]
				\item Soit $k \in \N^*$. D'après la question~(b), on a \[
						I_{2k} = \frac{2k-1}{2k} I_{2k-2} = \frac{2k-1}{2k} \times \frac{2k-3}{2k-2} I_{2k - 4}
					.\] En itérant ce procédé, on obtient
					\begin{align*}
						I_{2k} &= \frac{(2k-1)(2k-3)(2k-5)\cdots3}{(2k)(2k-2)\cdots4} I_2\\
						&= \frac{\prod_{i=1}^{k} (2i+1)}{\prod_{i=2}^k (2i)} I_2 \\
						&= \frac{\prod_{i=1}^{k} (2i+1)}{\prod_{i=1}^k (2i)} 2I_2 \\
						&= \frac{(2k)!}{\Big(\prod_{i=1}^k (2i)\Big)^{\!\!2}} 2I_2 \\
						&= \frac{(2k)!}{(2^k)^2\,(k!)^2} 2I_2\\
						&\!\!\,\boxed{= \frac{(2k)!}{4^k\,(k!)^2} \frac{\pi}{2}.} \\
					\end{align*}
				\item Soit $k \in \N^*$, on a, d'après la question~(f),  \[
						I_{2k} \simi_{k\to +\infty} \sqrt{\frac{\pi}{4k}}
				.\] Or, dans la question~(g), on a également montré que \[
					I_{2k} = \frac{(2k)!}{4^k\,(k!)^2} \cdot \frac{\pi}{2}
				.\] On en déduit, donc que
				\[
					\frac{(2k)!}{(k!)^2} \sim 4^k \frac{2}{\pi} \sqrt{\frac{\pi}{4k\pi}} = 4^k \sqrt{\frac{4\pi}{4k\pi^2}}
				.\] On en conclut que \[
					\boxed{ \frac{(2k)!}{(k!)^2} \simi_{k\to +\infty} \frac{4^k}{\sqrt{k\pi}}.}
				\]
			\end{enumerate}
		\item
			\begin{enumerate}
				\item Soit $n \in \N^*$. On sait, tout d'abord, que $\ln(n!) = \ln 2 + \ln 3 + \cdots + \ln n$\/ et donc, en procédant à l'aide d'une comparaison série-intégrale, on obtient l'inéquation suivante : \[
					\int_{2}^{n+1} \ln x~\mathrm{d}x \ge \sum_{k=1}^n \ln k \ge \int_{1}^{n} \ln x~\mathrm{d}x
				.\] Or, une primitive de $\ln x$\/ est $x \ln x - x$. On en conclut donc l'inéquation devient \[
					(n+1) \ln(n+1) - n - 2 \ln 2 + 1 \ge \ln(n!) \ge n \ln n - n+ 1
				.\] On divise cette inéquation par $n \ln n$\/ (qui ne s'annule pas) et on étudie la limite du membre de droite et celui de gauche :
					\begin{align*}
						&\frac{(n+1) \ln (n+1) - n - 2 \ln 2 + 1}{n \ln n} \ge \frac{\ln(n!)}{n \ln n} \ge  \frac{n \ln n - n + 1}{n \ln n}\\
						\iff& \underbrace{\frac{n+1}{n}}_{\to 1} \times \underbrace{\frac{\ln(n+1)}{\ln n}}_{\to 1} - \underbrace{\frac{n}{n \ln n}}_{\to 0} + \underbrace{\frac{1 - 2\ln 2}{n \ln n}}_{\to 0} \ge \frac{\ln(n!)}{n \ln n} \ge 1 + \underbrace{\frac{1 - n}{n \ln n}}_{\to 0}\\
					\end{align*}
					En passant à la limite, pour $n \to +\infty$, on obtient donc 1 de chaque côtés de l'inéquation. On conclut, par le théorème des gendarmes que, \[
						\frac{\ln(n!)}{n \ln n} \tendsto{n\to +\infty} 1\quad\text{i.e.}\quad\ln(n!) \sim n \ln n\quad\text{i.e.}\quad\boxed{\ln(n!) = n \ln n + \po(n \ln n).}
					\]
				\item Soit $n \ge 3$. On reprend l'inéquation de la question (a) : \[
						(n+1) \ln(n+1) - n - 2 \ln 2 + 1 \ge \ln(n!) \ge n \ln n - n+ 1
					.\]
					On la divise, cette fois, par $n \ln n - n$\/ (qui ne s'annule pas car $n \ge 3$), et on étudie, encore une fois, sa limite :
					\[
						\underbrace{\frac{(n+1) \ln(n+1) - (n + 1)}{n \ln n - n}}_{\to 1} + \underbrace{\frac{2 - 2 \ln 2}{n \ln n - n}}_{\to 0} \ge \frac{\ln (n!)}{n \ln n - n} \ge \underbrace{\frac{n \ln n - n}{n \ln n - n}}_{\to 1} + \underbrace{\frac{1}{n \ln n - n}}_{\to 0}
					.\] Le membre de droite ainsi que le membre de gauche convergent tous deux vers 1. On conclut, à l'aide du théorème des gendarmes, que \[
						\frac{\ln(n!)}{n \ln n - n} \tendsto{n\to +\infty} 1\quad\text{i.e.}\quad \boxed{\ln (n!) = n \ln n - n + \po(n).}
					\]
				\item On définit la suite $(u_n)_{n\in\N}$, pour tout $n \in \N$, $u_n = \ln (n!) - n \ln n + n$. On veut montrer que $u_n \simi_{n\to +\infty} \frac{1}{2} \ln n$. Soit $n \ge 3$. On compare $u_{n+1}$\/ et $u_n$\/ :
					\begin{align*}
						u_{n+1} - u_n &= \ln\big((n+1)!\big) - (n+1)\ln(n+1) + (n+1) - \ln(n!) + n \ln n - n \\
						&= \cancel{\ln(n!)} + \cancel{\ln(n+1)} - (n+\cancel1)\ln(n+1) + 1 - \cancel{\ln(n!)} + n \ln n \\
						&= -n \ln (n+1) + 1 + n \ln n \\
						&= -\cancel{n \ln n} + \cancel{n \ln n} - n \ln\left( 1 + \frac{1}{n} \right) + 1 \\
						&= 1 - 1 + \frac{1}{2n} + \po\left( \frac{1}{n} \right) \\
						&= \frac{1}{2n} + \po\left( \frac{1}{n} \right). \\
					\end{align*}
					Ainsi, en sommant ces termes, on obtient une somme télescopique :
					\begin{align*}
						u_n - u_1 = \sum_{k=1}^{n-1} (u_{k+1} - u_k) &= \sum_{k=1}^{n-1} \left( \frac{1}{2k} + \po\left( \frac{1}{n} \right) \right)\\
						&= \frac{1}{2} \sum_{k=1}^{n-1} \left(\frac{1}{k} + \po\left( \frac{1}{n} \right)\right)\\
						\simi_{n\to +\infty}\!\!\!\!\!\!\!\!&\mathrel{\phantom{=}} \frac{1}{2} \ln n.
					\end{align*}
					On en conclut que \[
						\boxed{\ln (n!) = n \ln n - n + \frac{1}{2} \ln n + \po(\ln n).}
					\]
				\item On pose la suite $(v_n)_{n\in\N^*}$\/ définie, pour tout $n \in \N^*$, comme $v_n = \ln(n!) - n \ln n + n - \frac{1}{2} \ln n$. On veut montrer que la série télescopique $\sum (v_{n+1} - v_n)$\/ converge. Soit $n \in \N^*$.
					\begin{align*}
						v_{n+1} - v_n &= \ln\big((n+1)!\big) - \ln(n!) -(n+1) \ln(n+1) + n \ln n + (n+1) - n - \frac{1}{2} \ln(n+1) - \ln n \\
						&= \ln\left( \frac{(n+1)!}{n!} \right) - (n+1) \ln(n+1) + n \ln n + 1 - \frac{1}{2} \ln\left( 1+ \frac{1}{n} \right) \\
						&= \ln(n+1) - (n+1) \ln(n+1) + n \ln n + 1 - \frac{1}{2} \ln\left(1+\frac{1}{n}\right) \\
						&= -n \ln(n+1) + n \ln n + 1 - \frac{1}{2} \ln\left( 1+ \frac{1}{n} \right)\\
						&= -n \ln\left( 1 + \frac{1}{n} \right) + 1 - \frac{1}{2}\ln\left( 1+ \frac{1}{n} \right) \\
						&= -\left( \frac{1}{2} + n \right) \ln\left( 1+ \frac{1}{n} \right) + 1 \\
						&= -\left( \frac{1}{2} + n \right)\times\left( \frac{1}{n} - \frac{1}{2n^2} + \frac{1}{3n^3} + \po\left( \frac{1}{n^3} \right)  \right) + 1\\
						&= -\frac{1}{2n} - 1 + \frac{1}{4n^2} + \frac{1}{2n} - \frac{1}{3n^2} + \po\left( \frac{1}{n^2} \right) + 1 \\
						&= \frac{1}{3n^2}- \frac{1}{4n^2} + \po\left( \frac{1}{n^2} \right) \\
						&= -\frac{1}{12n^2} + \po\left( \frac{1}{n^2} \right) \\
					\end{align*}
					On remarque que $\sum (v_{n+1} - v_n) \sim \sum -\frac{1}{12n^2} \sim \sum \frac{1}{n^2}$\/ et, comme la série $\sum \frac{1}{n^2}$\/ converge par critère de {\sc Riemann}, alors la série $\sum (v_{n+1} - v_n)$\/ converge également. Or, cette série est télescopique : \[
						\forall n \ge 2,\quad \sum_{k=1}^{n-1} (v_{n+1}- v_n) = v_n - v_1
					.\]
					Le terme $v_1$\/ étant constant, on a démontré que la suite $(v_n)_{n\in\N}$\/ converge. En nommant sa limite $-K$, on a donc l'égalité suivante \[
						v_n = K + \po(1)\qquad\text{i.e.}\qquad \boxed{\ln(n!) = n\ln n - n + \frac{1}{2} \ln n + K + \po(1).}
					\]
				\item
					On pose, pour tout $n \in \N^*$, $w_n = \ln(n!) - n \ln n + n - \frac{1}{2} \ln n - K$. Soit $n \ge 2$. On calcule
					\begin{align*}
						w_{n+1} - w_n &= v_{n+1} - v_n + K - K \\
						&= - \frac{1}{12n^2} + \po\left( \frac{1}{n^2} \right) \\
					\end{align*}
					On en déduit, comme à la question précédente, que la série $\sum (w_{n+1} - w_n)$\/ converge. Or, on a
					\begin{align*}
						\forall n \ge 2,\quad w_n - w_1 = \sum_{k=1}^{n-1} (w_{k+1} - w_k) &= \sum_{k=1}^{n-1} \left( -\frac{1}{12k^2} + \po\left( \frac{1}{n^2} \right) \right)\\
						&= \sum_{k=1}^{n-1} \left( -\frac{1}{12n^2} + \po\left( \frac{1}{n^2} \right) \right)\\
						&= n \times \left( -\frac{1}{12n^2} + \po\left( \frac{1}{n^2} \right) \right)\\
						&= -\frac{1}{12n} + \po\left( \frac{1}{n} \right) \\
					\end{align*}
					On en déduit donc que \[
						\ln(n!) - n \ln n + n - \frac{1}{2} \ln n - K = \frac{1}{12n} + \po\left( \frac{1}{n} \right)
					\] d'où \[
					\boxed{\ln(n!) = n \ln n - n + \frac{1}{2} \ln n + K + \frac{1}{12n} + \po\left( \frac{1}{n} \right).}
					\] 
			\end{enumerate}
		\item En prenant le résultat de la question (2e) et lui appliquant la fonction exponentielle, on obtient \[
			n! = n^n \times \mathrm{e}^{-n} \times \sqrt{n} \times \mathrm{e}^K \times \mathrm{e}^{\po(1)}
		.\] Le terme en $\mathrm{e}^{\po(1)}$\/ tend vers 1 quand $n$\/ tend vers $+\infty$. En factorisant les exposants et en passant à un équivalent, cette expression peut se réécrire sous la forme \[
			n! \simi_{n\to +\infty} \left( \frac{n}{\mathrm{e}} \right)^n \mathrm{e}^K \sqrt{n}
		.\] On s'intéresse maintenant à déterminer $\mathrm{e}^K$. On connaît, d'après la question (1h), un équivalent de $(2n)! / (n!)^2$. On utilise l'équivalent de $n!$\/ trouvé précédemment et on détermine la valeur de $\mathrm{e}^K$.
		\begin{align*}
			\frac{4^n}{\sqrt{n\pi}} \sim \frac{(2n)!}{(n!)^2} \sim& \frac{\left( \frac{2n}{\mathrm{e}} \right)^{2n} \sqrt{2n} \times \mathrm{e}^K}{\left( \frac{n}{\mathrm{e}} \right)^{2n} \times n \times \mathrm{e}^K \times \mathrm{e}^K}\\
			\sim& \left( \frac{2n}{n} \right)^{2n} \frac{\sqrt{2}}{\sqrt{n}\times \mathrm{e}^K}\\
			\sim&\:2^{2n} \times \sqrt{2} \times \frac{1}{\sqrt{n} \times \mathrm{e}^K}
		\end{align*}
		On multiplie par $4^{-n} \times \sqrt{n}$\/ de chaque côtés de l'équivalence pour obtenir \[
			\frac{1}{\sqrt{\pi}} \sim \frac{\sqrt{2}}{\mathrm{e}^K}\qquad\text{i.e.}\qquad \mathrm{e}^K \sim \sqrt{2\pi}
		.\] Or, comme l'équivalent de $\mathrm{e}^K$\/ ne dépend plus de $n$, on en déduit que $\mathrm{e}^K = \sqrt{2\pi}$. On peut donc en déduire la formule de {\sc Stirling}\/ : \[
			\boxed{n! \simi_{n\to +\infty} \left( \frac{n}{\mathrm{e}} \right)^n \sqrt{2\pi\,n}.}
		\]
	\end{enumerate}
\end{document}
