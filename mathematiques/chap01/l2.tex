\begin{exo}
	\begin{align*}
		\ln(n!) &= \ln\big(1 \times 2 \times 3 \times \cdots \times (n-1)\times n\big) \\
		&= \cancel{\ln(1)} + \ln(2) + \ln(3) + \cdots + \ln(n-1) + \ln n \\
	\end{align*}
	Pour calculer cette somme, on utilise la méthode des rectangles. Avec des rectangles à droite on obtient l'inégalité \[
		\ln(n!) \le \int_{2}^{n+1} \ln(x)~\mathrm{d}x
	.\]
	Avec les rectangles à gauche, on obtient \[
		\int_{1}^{n} \ln x~\mathrm{d}x \le \ln(n!)
	.\]

	D'où
	\begin{align*}
		&\big[x \ln x - x\big]_1^n \le \ln(n!) \le \big[x \ln x - x\big]_2^{n+1}\\
		\iff& \frac{\ln n - n + 1}{n \ln n} \le \frac{\ln(n!)}{n\ln n} \le \frac{(n+1) \ln(n+1) - (n+1) - 2\ln 2 + 2}{n \ln n}\\
	\end{align*}
	Or, les deux ``gendarmes'' tendent vers 1, par le théorème des gendarmes, on en déduit que \[
		\frac{\ln (n!)}{n\ln n} \tendsto{n\to +\infty} 1 \quad\text{i.e.}\quad \ln(n!) \sim n \ln n
	.\]
\end{exo}

\begin{prop}
	La série $\sum \frac{1}{n^{\alpha}}$\/ converge si et seulement si $\alpha > 1$.
\end{prop}

\bigskip

\begin{center}
	\large\sc I.3 \quad Le reste d'une série convergente
\end{center}

\bigskip
\[
	\underbrace{u_0 + u_1 + u_2 + \cdots + u_n}_{{\ds\sum_{k = 0}^n u_k = S_n \to \text{ somme partielle}}} + \underbrace{u_{n+1} + u_{n+2} + \cdots}_{\ds\sum_{k=n+1}^{\infty} u_k = R_n \to \text{ reste}}
.\]

Le reste est défini si et seulement si la série $\sum u_n$\/ converge. La somme $S_n + R_n = \sum_{k=0}^{\infty} u_k$\/ est définie si et seulement si la série $\sum u_n$\/ converge. On pose $\R \ni \ell = S_n + R_n$. Ainsi, on a
\begin{gather*}
	S_n \tendsto{n\to +\infty} \ell\\
	\ell - S_n = R_n \tendsto{n\to +\infty} \ell - \ell = 0
\end{gather*}

\begin{exo}
	La série $\sum \frac{1}{n^2}$\/ converge d'après le critère de {\sc Riemann}.
	D'où le reste $R_n = \sum_{k=n+1}^{\infty} \frac{1}{k^2}$\/ est bien défini.
	
	On utilise, encore une fois, la méthode des rectangles : en effet, on a
	\begin{align*}
		&\int_{n+1}^{N+1} \frac{1}{x^2}~\mathrm{d}x \le \sum_{k=n+1}^{N} \frac{1}{k^2} \le \int_{n}^{N} \frac{1}{x^2}~\mathrm{d}x\\
		\iff& \left[ -\frac{1}{x} \right]^{N+1}_{n+1} \le \sum_{k=n+1}^N \frac{1}{k^2} \le \left[ -\frac{1}{x} \right]_n^N\\
		\iff& \frac{1}{n+1} - \frac{1}{N+1} \le \sum_{k=n+1}^N \frac{1}{k^2} \le \frac{1}{n} - \frac{1}{N}.
	\end{align*}
	On n'a pas besoin du théorème des gendarmes ; on utilise le fait que les inégalités larges ``passent à la limite.''
	\[
		\begin{array}{ccccc}
			\ds\frac{1}{n+1} - \frac{1}{N+1} & \le & \ds\sum_{k=n+1}^N & \le & \ds\frac{1}{n} - \frac{1}{N}\\[2mm]
			{\ds \downarrow}\mathrlap{\scriptscriptstyle N\to +\infty} && {\ds \downarrow}\mathrlap{\scriptscriptstyle N\to +\infty} && {\ds \downarrow}\mathrlap{\scriptscriptstyle N\to +\infty}\\[2mm]
			\frac{1}{n+1} & \le & R_n &\le& \frac{1}{n}.
		\end{array}
	\]
\end{exo}

\bigskip

\begin{center}
	\large\sc I.4 \quad Les séries alternées
\end{center}

\bigskip

\begin{thm}[Séries alternées]
	La série $\sum (-1)^k u_k$\/ (où $u_k$\/ tends vers $0$\/ {\color{red}en décroissant}, d'où $u_k \ge 0$) converge. Et, $\ell$\/ a le même signe que le premier terme.

	\begin{align*}
		S_0&= u_0\\
		S_1&= u_0 - u_1 \\
		S_2&= u_0 - u_1 + u_2 \\
		\vdots\:\:&\:\:\quad\vdots\\
		S_n&= u_0 - u_1 + \cdots + (-1)^nu_n \\
	\end{align*}

	\begin{figure}[H]
		\centering
		\todo{Faire figure (fig. 4)}
		\caption{Série alternée}
	\end{figure}
	Le reste $R_n = \ell - S_n$ change de signe une fois sur deux i.e.\ il est alterné. De plus, on a \[
		|R_n| \le u_{n+1}
	.\]
\end{thm}

\begin{exo}[Mines-Ponts]
	On pose, pour tout $n \in \N^*$, \[
		S_n = \sum_{k=0}^{n} \frac{(-1)^{k-1}}{k} = 1 - \frac{1}{2} + \frac{1}{3} - \frac{1}{4} + \cdots + \frac{(-1)^{n-1}}{n}
	.\]

	(Ça ne sert à rien mais\ldots) comme $\frac{1}{k}$\/ tends vers 0 en décroissant, d'où, d'après le théorème des séries alternées, la suite $(S_n)$\/ converge.

	Montrons que \[
		(1)\qquad\ln 2 - S_n = \int_{0}^{1} \frac{(-t)^n}{1+t}~\mathrm{d}t
	\] et $(2)$ en déduire que $S_n$\/ tends vers $\ln 2$.

	\begin{enumerate}
		\item
			On remarque que \[
				\int_{0}^{1} \frac{1}{1+t}~\mathrm{d}t = \big[\ln |1+t|\big]_0^1 = \ln 2
				\quad
				\text{ et }
				\quad
				\frac{1}{k} = \int_{0}^{1} t^{k-1}~\mathrm{d}t = \left[ \frac{t^k}{k} \right]^1_0.
			\]
			D'où, \[
				\sum_{k=1}^n (-1)^{k-1} \int_{0}^{1} t^{k-1}~\mathrm{d}t
			.\]
			Or, par linéarité de l'intégrale, on a \[
				S_n = \int_{0}^{1} \bigg(\sum_{k=1}^{n} (-1)^{k-1} t^{k-1}\bigg)~\mathrm{d}t
			.\]
		\item
			\begin{align*}
				|\ln 2 - S_n| &= \bigg| \int_{0}^{1} \frac{(-1)^n}{1+t}~\mathrm{d}t \bigg| \\
				&\le \int_{0}^{1} \left| \frac{(-t)^n}{1+t} \right| ~\mathrm{d}t\\
				&\le \int_{0}^{1} t^n~\mathrm{d}t\\
				&= \left[ \frac{t^{n+1}}{n + 1} \right]_0^1 = \frac{1}{n+1} \\
			\end{align*}
			par croissance de l'intégrale. D'où, $|\ln 2 - S_n| \tendsto{n\to +\infty} 0$\/ d'après le théorème des gendarmes.
			On en conclut que \[
				\boxed{S_n \tendsto{n\to +\infty} \ln 2.}
			\]
			Or,
			\begin{align*}
				\sum_{k=1}^n (-t)^{k-1} &= (-t)^0 + (-t)^1 + \cdots + (-t)^{n-1} \\
				&= \frac{1 - (-t)^n}{1 - (-t)} \\
			\end{align*}
			car \[
				\sum_{k=0}^{N} q^k = q^0 + q^1 + \cdots + q^N = \frac{1 - q^{N+1}}{1-q} \quad \text{ si} q \neq 1
			.\]
			D'où
			\begin{align*}
				S_n &= \int_{0}^{1} \frac{1 - (-t)^n}{1 + t}~\mathrm{d}t\\
				&= \int_{0}^{1} \frac{1}{1+t}~\mathrm{d}t  - \int_{0}^{1} \frac{(-t)^n}{1+t}~\mathrm{d}t \\
				&= \boxed{\ln 2 - \int_{0}^{1} \frac{(-t)^n}{1+t}~\mathrm{d}t.} \\
			\end{align*}
			On sait que $R_n$\/ est existe car la suite $S_n$\/ converge.

			La série $\sum R_n$\/ converge-t-elle ou diverge-t-elle ?
			On sait que $R_n = (-1)^n\:|R_n|$\/ au signe près.
			On veut montrer que $|R_n|$\/ tend vers 0 en décroissant.
			\begin{align*}
				|R_n| = \int_{0}^{1} \frac{t^n}{1+t}~\mathrm{d}t &\le \int_{0}^{1} t^n~\mathrm{d}t \text{ car } \frac{t^n}{1 + t} \le t^n \forall t \in [0,1]\\
				&\le \left[ \frac{t^{n+1}}{n+1} \right]_0^1 = \frac{1}{n+1}.
			\end{align*}
			D'où $|R_n|$\/ tend vers 0 d'après le théorème des gendarmes.

			\begin{align*}
				|R_{n+1}| - |R_n| &= \int_{0}^{1} \frac{t^{n+1}}{n+1}~\mathrm{d}t - \int_{0}^{1} \frac{t^n}{1 + t}~\mathrm{d}t\\
				&= \int_{0}^{1} \frac{t^{n+1} - t^n}{1+t}~\mathrm{d}t \\
				&= \int_{0}^{1} \frac{(t-1)t^n}{1 + t}~\mathrm{d}t \\
				&\le 0 \\
			\end{align*}

			On aurait aussi très bien pu utiliser le théorème des séries alternées : $|R_n| \le u_{n+1}$.
	\end{enumerate}
\end{exo}

Cadeau du {\it 06/09/2022}\/ : 
calculer les trois limites suivantes (avec un développement limité)
\begin{enumerate}
	\item $\ds \lim_{x\to 0} \frac{\sqrt[3]{8+x} - 2\sqrt{1+x}}{x}$\/ (corrigé) ;
	\item $\ds \lim_{x\to 0} \frac{\tan x - x}{\sin^3 x}$\/ (non corrigé);
	\item $\ds \lim_{x\to 0} \frac{\ln (\cos x)}{x^2}$\/ (corrigé).
\end{enumerate}

\bigskip

\begin{comment}
\begin{enumerate}
	\item[3.]
		\[
			\frac{\ln(\cos x)}{x^2} = \frac{\ln\left( 1 - \frac{x^2}{2} + \po(x^2)\right)}{x^2} = \frac{-\frac{x^2}{2} + \po(x^2)}{x^2} = -\frac{1}{2} + \po(1) \tendsto{n\to +\infty} -\frac{1}{2}
		.\] 
	\item[1.]
		\[
			\frac{\sqrt[3]{8+x} - 2\sqrt{1+x}}{x} = \frac{\sqrt[3]{8} \sqrt[3]{1+\frac{x}{8}} - 2\sqrt{1+x}}{x} = \frac{2\left( 1+ \frac{1}{3}\frac{x}{8} \right) - 2\left( 1+ \frac{1}{2}x \right)}{x}
		.\] 
\end{enumerate}
\end{comment}



