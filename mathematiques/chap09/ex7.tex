\section{Exercice 7}

D'une part, on a \[
	\left<A(X)  \mid X\,A(X) \right> = h\big(X\,A(X)\big) = \big(X\, A(X)\big)(0) = 0
.\] 
Et, d'autre part, \[
	\left<A(X)  \mid X\, A(X) \right> = \int_{0}^{1} t\, A^2(t)~\mathrm{d}t
.\]
Or, comme la fonction $f: t \mapsto t A^2(t)$\/ est continue et positive, et que $\int_{0}^{1} f(t)~\mathrm{d}t = 0$, alors $\forall t \in [0,1]$, $f(t) = 0$.
Le polynôme $A$\/ a donc une infinité de racines, il s'agit du polynôme nul.

Ceci est un \textit{contre-exemple} au théorème de \textsc{Riesz}.


