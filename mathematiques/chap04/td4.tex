\section{Exercice 6}

{\slshape Soient $a$, $b$\/ et $c$\/ trois réels. Soient $A$\/ et $B$\/ deux matrices $3 \times 3$, définies par \[
	A = \begin{pmatrix}
		1&a&1\\
		0&1&0\\
		0&0&a
	\end{pmatrix}\qquad\text{et}\qquad\begin{pmatrix}
		2&b&0\\
		0&c&0\\
		0&0&2
	\end{pmatrix}
.\]
Pour quelles valeurs des réels $a$, $b$\/ et $c$\/ les matrices $A$\/ et $B$\/ sont-elles diagonalisables ?}

Soit $\lambda \in \R$. D'où
\[	\lambda \in \Sp(A) \iff \det(\lambda I_3 - A) = 0\]
Or,
\begin{align*}
	\det(\lambda I_3 - A) = \begin{pmatrix}
		\lambda - 1&-a&-1\\
		0&\lambda-1&0\\
		0&0&\lambda-a
	\end{pmatrix} = (\lambda-1)^2 \cdot (\lambda - a)
\end{align*}

\begin{enumerate}
	\item[\sc 1\tsup{er} cas] $a = 1$\/ : on a donc $\Sp(A) = \{1\}$. Par l'absurde, si $A$\/ est diagonalisable, alors $A \sim I_3$, d'où $A = I_3$, ce qui est absurde. Et donc $A$\/ n'est pas diagonalisable.
	\item[\sc 2\tsup{nd} cas] $a \neq 1$\/ : on a donc $\Sp(A) = \{1,a\}$. D'où $1 \le \dim(\mathrm{SEP}(a)) \le 1$, et $1 \le \dim(\mathrm{SEP}(1)) \le 2$. Or, la matrice $A$\/ est diagonalisable si et seulement si $\dim(\mathrm{SEP}(a)) + \dim(\mathrm{SEP}(1)) = 3$, donc si et seulement si $\dim(\mathrm{SEP}(1)) = 2$. Soit donc $X = \left( \substack{x\\y\\z} \right) \in \mathscr{M}_{3,1}(\R)$.
		\begin{align*}
			X \in \mathrm{SEP}(1) \iff& AX = X\\
			\iff& \begin{pmatrix}
				1&a&1\\
				0&1&0\\
				0&0&a
			\end{pmatrix} \begin{pmatrix}
				x\\y\\z
			\end{pmatrix} = \begin{pmatrix}
				x\\y\\z
			\end{pmatrix}\\
			\iff& \begin{cases}
				ay + z = 0\\
				az = z
			\end{cases}\\
			\iff& \begin{cases}
				ay = 0\\
				z = 0
			\end{cases}\quad \text{ car } a \neq 1\\
		\end{align*}
		\begin{enumerate}
			\item[\itshape 1\tsup{er} sous-cas] $a = 0$\/ : \[
					X \in \mathrm{SEP}(1) \iff z = 0 \iff X = x\begin{pmatrix}
						1\\0\\0
					\end{pmatrix} + y \begin{pmatrix}
						0\\1\\0
					\end{pmatrix}
				.\]
				D'où $\dim(\mathrm{SEP}(1)) = 2$.
			\item[\itshape 2\tsup{nd} sous-cas] $a \neq 0$\/ : \[
					X \in \mathrm{SEP}(1) \iff \begin{cases}
						y = 0\\
						z = 0
					\end{cases} \iff X = x\begin{pmatrix}
						1\\0\\0
					\end{pmatrix} 
				.\]
				D'où $\dim(\mathrm{SEP}(1)) = 1$.
		\end{enumerate}
		Ainsi, $A$\/ est diagonalisable si et seulement si $a = 0$.
\end{enumerate}

