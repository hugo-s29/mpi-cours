\vspace{5mm}

\section{Exercice 9}
On considère la matrice \[
	A = \begin{pmatrix}
		1&1&-1\\
		2&3&-4\\
		4&1&-4
	\end{pmatrix}
.\]
Soit $\lambda \in \R$. On sait que \[
	\lambda \in \Sp(A) \iff \det(\lambda I_3- A) = 0
.\]
On calcule $\det(\lambda I_3 - A)$\/ : 
\begin{align*}
	\det(\lambda I_3 - A) &=
	\begin{vmatrix}
		\lambda-1&-1&1\\
		-2&\lambda-3&4\\
		-4&-1&\lambda + 4
	\end{vmatrix}\\
	&=
	\begin{vmatrix}
		\lambda-1&-1&1\\
		\lambda-1&\lambda-3&4\\
		\lambda-1&-1&\lambda + 4
	\end{vmatrix} \text{ avec le changement } C_1 \leftarrow C_1 + C_2 + C_3\\
	&= (\lambda - 1)
	\begin{vmatrix}
		1&-1&1\\
		1&\lambda-3&4\\
		1&-1&\lambda + 4
	\end{vmatrix}\\
	&= (\lambda - 1)
	\begin{vmatrix}
		1&-1&1\\
		0&\lambda-2&3\\
		0&0&\lambda - 3
	\end{vmatrix}\text{ avec les changements } \begin{cases}
		L_2 \leftarrow L_2 -  L_1\\
		L_3 \leftarrow L_3 - L_1 \\
	\end{cases}\\
	&= (\lambda - 1)(\lambda - 2)(\lambda + 3). \\
\end{align*}
D'où $\Sp(A) = \{1,2,-3\}$, et \[
	1 \le \dim(\mathrm{SEP}(1)) \le 1\qquad
	1 \le \dim(\mathrm{SEP}(2)) \le 1\quad\text{et}\quad
	1 \le \dim(\mathrm{SEP}(-3)) \le 1.\footnote{inutile dans ce cas}
\]
La matrice~$A$\/ est de taille 3 et elle possède 3 valeurs propres distinctes deux à deux. D'où, d'après la proposition 18, on sait donc que $A$\/ est diagonalisable. Diagonalisons-la.


