\documentclass[a4paper]{article}

\usepackage[margin=1in]{geometry}
\usepackage[utf8]{inputenc}
\usepackage[T1]{fontenc}
\usepackage{mathrsfs}
\usepackage{textcomp}
\usepackage[french]{babel}
\usepackage{amsmath}
\usepackage{amssymb}
\usepackage{cancel}
\usepackage{frcursive}
\usepackage[inline]{asymptote}
\usepackage{tikz}
\usepackage[european,straightvoltages,europeanresistors]{circuitikz}
\usepackage{tikz-cd}
\usepackage{tkz-tab}
\usepackage[b]{esvect}
\usepackage[framemethod=TikZ]{mdframed}
\usepackage{centernot}
\usepackage{diagbox}
\usepackage{dsfont}
\usepackage{fancyhdr}
\usepackage{float}
\usepackage{graphicx}
\usepackage{listings}
\usepackage{multicol}
\usepackage{nicematrix}
\usepackage{pdflscape}
\usepackage{stmaryrd}
\usepackage{xfrac}
\usepackage{hep-math-font}
\usepackage{amsthm}
\usepackage{thmtools}
\usepackage{indentfirst}
\usepackage[framemethod=TikZ]{mdframed}
\usepackage{accents}
\usepackage{soulutf8}
\usepackage{mathtools}
\usepackage{bodegraph}
\usepackage{slashbox}
\usepackage{enumitem}
\usepackage{calligra}
\usepackage{cinzel}
\usepackage{BOONDOX-calo}

% Tikz
\usetikzlibrary{babel}
\usetikzlibrary{positioning}
\usetikzlibrary{calc}

% global settings
\frenchspacing
\reversemarginpar
\setuldepth{a}

%\everymath{\displaystyle}

\frenchbsetup{StandardLists=true}

\def\asydir{asy}

%\sisetup{exponent-product=\cdot,output-decimal-marker={,},separate-uncertainty,range-phrase=\;à\;,locale=FR}

\setlength{\parskip}{1em}

\theoremstyle{definition}

% Changing math
\let\emptyset\varnothing
\let\ge\geqslant
\let\le\leqslant
\let\preceq\preccurlyeq
\let\succeq\succcurlyeq
\let\ds\displaystyle
\let\ts\textstyle

\newcommand{\C}{\mathds{C}}
\newcommand{\R}{\mathds{R}}
\newcommand{\Z}{\mathds{Z}}
\newcommand{\N}{\mathds{N}}
\newcommand{\Q}{\mathds{Q}}

\renewcommand{\O}{\emptyset}

\newcommand\ubar[1]{\underaccent{\bar}{#1}}

\renewcommand\Re{\expandafter\mathfrak{Re}}
\renewcommand\Im{\expandafter\mathfrak{Im}}

\let\slantedpartial\partial
\DeclareRobustCommand{\partial}{\text{\rotatebox[origin=t]{20}{\scalebox{0.95}[1]{$\slantedpartial$}}}\hspace{-1pt}}

% merging two maths characters w/ \charfusion
\makeatletter
\def\moverlay{\mathpalette\mov@rlay}
\def\mov@rlay#1#2{\leavevmode\vtop{%
   \baselineskip\z@skip \lineskiplimit-\maxdimen
   \ialign{\hfil$\m@th#1##$\hfil\cr#2\crcr}}}
\newcommand{\charfusion}[3][\mathord]{
    #1{\ifx#1\mathop\vphantom{#2}\fi
        \mathpalette\mov@rlay{#2\cr#3}
      }
    \ifx#1\mathop\expandafter\displaylimits\fi}
\makeatother

% custom math commands
\newcommand{\T}{{\!\!\,\top}}
\newcommand{\avrt}[1]{\rotatebox{-90}{$#1$}}
\newcommand{\bigcupdot}{\charfusion[\mathop]{\bigcup}{\cdot}}
\newcommand{\cupdot}{\charfusion[\mathbin]{\cup}{\cdot}}
%\newcommand{\danger}{{\large\fontencoding{U}\fontfamily{futs}\selectfont\char 66\relax}\;}
\newcommand{\tendsto}[1]{\xrightarrow[#1]{}}
\newcommand{\vrt}[1]{\rotatebox{90}{$#1$}}
\newcommand{\tsup}[1]{\textsuperscript{\underline{#1}}}
\newcommand{\tsub}[1]{\textsubscript{#1}}

\renewcommand{\mod}[1]{~\left[ #1 \right]}
\renewcommand{\t}{{}^t\!}
\newcommand{\s}{\text{\calligra s}}

% custom units / constants
%\DeclareSIUnit{\litre}{\ell}
\let\hbar\hslash

% header / footer
\pagestyle{fancy}
\fancyhead{} \fancyfoot{}
\fancyfoot[C]{\thepage}

% fonts
\let\sc\scshape
\let\bf\bfseries
\let\it\itshape
\let\sl\slshape

% custom math operators
\let\th\relax
\let\det\relax
\DeclareMathOperator*{\codim}{codim}
\DeclareMathOperator*{\dom}{dom}
\DeclareMathOperator*{\gO}{O}
\DeclareMathOperator*{\po}{\text{\cursive o}}
\DeclareMathOperator*{\sgn}{sgn}
\DeclareMathOperator*{\simi}{\sim}
\DeclareMathOperator{\Arccos}{Arccos}
\DeclareMathOperator{\Arcsin}{Arcsin}
\DeclareMathOperator{\Arctan}{Arctan}
\DeclareMathOperator{\Argsh}{Argsh}
\DeclareMathOperator{\Arg}{Arg}
\DeclareMathOperator{\Aut}{Aut}
\DeclareMathOperator{\Card}{Card}
\DeclareMathOperator{\Cl}{\mathcal{C}\!\ell}
\DeclareMathOperator{\Cov}{Cov}
\DeclareMathOperator{\Ker}{Ker}
\DeclareMathOperator{\Mat}{Mat}
\DeclareMathOperator{\PGCD}{PGCD}
\DeclareMathOperator{\PPCM}{PPCM}
\DeclareMathOperator{\Supp}{Supp}
\DeclareMathOperator{\Vect}{Vect}
\DeclareMathOperator{\argmax}{argmax}
\DeclareMathOperator{\argmin}{argmin}
\DeclareMathOperator{\ch}{ch}
\DeclareMathOperator{\com}{com}
\DeclareMathOperator{\cotan}{cotan}
\DeclareMathOperator{\det}{det}
\DeclareMathOperator{\id}{id}
\DeclareMathOperator{\rg}{rg}
\DeclareMathOperator{\rk}{rk}
\DeclareMathOperator{\sh}{sh}
\DeclareMathOperator{\th}{th}
\DeclareMathOperator{\tr}{tr}

% colors and page style
\definecolor{truewhite}{HTML}{ffffff}
\definecolor{white}{HTML}{faf4ed}
\definecolor{trueblack}{HTML}{000000}
\definecolor{black}{HTML}{575279}
\definecolor{mauve}{HTML}{907aa9}
\definecolor{blue}{HTML}{286983}
\definecolor{red}{HTML}{d7827e}
\definecolor{yellow}{HTML}{ea9d34}
\definecolor{gray}{HTML}{9893a5}
\definecolor{grey}{HTML}{9893a5}
\definecolor{green}{HTML}{a0d971}

\pagecolor{white}
\color{black}

\begin{asydef}
	settings.prc = false;
	settings.render=0;

	white = rgb("faf4ed");
	black = rgb("575279");
	blue = rgb("286983");
	red = rgb("d7827e");
	yellow = rgb("f6c177");
	orange = rgb("ea9d34");
	gray = rgb("9893a5");
	grey = rgb("9893a5");
	deepcyan = rgb("56949f");
	pink = rgb("b4637a");
	magenta = rgb("eb6f92");
	green = rgb("a0d971");
	purple = rgb("907aa9");

	defaultpen(black + fontsize(8pt));

	import three;
	currentlight = nolight;
\end{asydef}

% theorems, proofs, ...

\mdfsetup{skipabove=1em,skipbelow=1em, innertopmargin=6pt, innerbottommargin=6pt,}

\declaretheoremstyle[
	headfont=\normalfont\itshape,
	numbered=no,
	postheadspace=\newline,
	headpunct={:},
	qed=\qedsymbol]{demstyle}

\declaretheorem[style=demstyle, name=Démonstration]{dem}

\newcommand\veczero{\kern-1.2pt\vec{\kern1.2pt 0}} % \vec{0} looks weird since the `0' isn't italicized

\makeatletter
\renewcommand{\title}[2]{
	\AtBeginDocument{
		\begin{titlepage}
			\begin{center}
				\vspace{10cm}
				{\Large \sc Chapitre #1}\\
				\vspace{1cm}
				{\Huge \calligra #2}\\
				\vfill
				Hugo {\sc Salou} MPI${}^{\star}$\\
				{\small Dernière mise à jour le \@date }
			\end{center}
		\end{titlepage}
	}
}

\newcommand{\titletp}[4]{
	\AtBeginDocument{
		\begin{titlepage}
			\begin{center}
				\vspace{10cm}
				{\Large \sc tp #1}\\
				\vspace{1cm}
				{\Huge \textsc{\textit{#2}}}\\
				\vfill
				{#3}\textit{MPI}${}^{\star}$\\
			\end{center}
		\end{titlepage}
	}
	\fancyfoot{}\fancyhead{}
	\fancyfoot[R]{#4 \textit{MPI}${}^{\star}$}
	\fancyhead[C]{{\sc tp #1} : #2}
	\fancyhead[R]{\thepage}
}

\newcommand{\titletd}[2]{
	\AtBeginDocument{
		\begin{titlepage}
			\begin{center}
				\vspace{10cm}
				{\Large \sc td #1}\\
				\vspace{1cm}
				{\Huge \calligra #2}\\
				\vfill
				Hugo {\sc Salou} MPI${}^{\star}$\\
				{\small Dernière mise à jour le \@date }
			\end{center}
		\end{titlepage}
	}
}
\makeatother

\newcommand{\sign}{
	\null
	\vfill
	\begin{center}
		{
			\fontfamily{ccr}\selectfont
			\textit{\textbf{\.{\"i}}}
		}
	\end{center}
	\vfill
	\null
}

\renewcommand{\thefootnote}{\emph{\alph{footnote}}}

% figure support
\usepackage{import}
\usepackage{xifthen}
\pdfminorversion=7
\usepackage{pdfpages}
\usepackage{transparent}
\newcommand{\incfig}[1]{%
	\def\svgwidth{\columnwidth}
	\import{./figures/}{#1.pdf_tex}
}

\pdfsuppresswarningpagegroup=1
\ctikzset{tripoles/european not symbol=circle}

\newcommand{\missingpart}{{\large\color{red} Il manque quelque chose ici\ldots}}

\usepackage{tgschola}

\usepackage{subcaption}

\pagecolor{truewhite}
\definecolor{black}{HTML}{000000}
\color{black}

\begin{asydef}
	settings.prc = false;
	settings.render=0;

	white = rgb("ffffff");
	black = rgb("000000");
	blue = rgb("286983");
	red = rgb("d7827e");
	yellow = rgb("f6c177");
	orange = rgb("ea9d34");
	gray = rgb("9893a5");
	grey = rgb("9893a5");
	deepcyan = rgb("56949f");
	pink = rgb("b4637a");
	magenta = rgb("eb6f92");
	green = rgb("a0d971");
	purple = rgb("907aa9");

	defaultpen(black + fontsize(8pt));

	import three;
	currentlight = nolight;
\end{asydef}

\fancyhead{} \fancyfoot{}
\fancyfoot[C]{--\:\thepage\:--}
\fancyhead[L]{Hugo {\sc Salou}, Alan {\sc Le Brech}, Nicolas {\sc Vincent}, {\it MPI}\/$^\star$}
\fancyhead[R]{DM$_1$ Physique}

\def\thefigure{\Roman{figure}}

\begin{document}
	\begin{center}
		\Large\sc Problème n\textsuperscript o\,2 : Étude de circuits
	\end{center}

	\begin{figure}[H]
		\centering
		\begin{subfigure}{0.5\textwidth}
			\centering
			\begin{circuitikz}
				\draw (0,0) to[american, isource, l=$e(t)$] (0,2) to[R=$R_\text{g}$] (0,4) to[short, i=$i$] (1,4) to[american,cute inductors,L=$L$] (3, 4) to[R=$R$,v=$u_R$] (5, 4) to[C=$C$,v=$u_C$] (5, 0) -- (0, 0);
				\draw[-latex] (1.3,3.65)--(2.7,3.65);
				\node at (2, 3.5) {$u_L$};
			\end{circuitikz}
			\caption{Circuit A}
		\end{subfigure}%
		\begin{subfigure}{0.5\textwidth}
			\centering
			\begin{circuitikz}
				\draw (0,0) to[american, isource, l=$e(t)$] (0,2) to[R=$R_\text{g}$] (0,4) to[short, i=$i$] (1,4) to[american,cute inductors,L=$L$] (3, 4) to[C=$C$] (5, 4) -- (5, 3.5) -- (4.5, 3.5) to[R=$R$] (4.5, 0.5) -- (5, 0.5) -- (5, 0) -- (0,0);
				\draw (5,3.5)--(5.5,3.5) to[R=$R$] (5.5, 0.5) -- (5, 0.5);
			\end{circuitikz}
			\caption{Circuit B}
		\end{subfigure}
		\caption{Les circuits A et B comme décrit dans l'énoncé}
	\end{figure}

	\begin{multicols}{2}
		\noindent {\large\bf Question 1}\quad\hrule

		Comme le signal, dans les deux circuits, est sinusoïdal, on détermine la fonction de transfert de chacun des circuit.
		Dans le circuit~A, on réalise un pont diviseur de tension pour~$u_R$, et on a donc
		\begin{align*}
			\ubar{H}_{\text{A},R}(\mathrm{j}\omega) &= \frac{R}{R + R_\text{g} + \frac{1}{\mathrm{j}C\omega} + \mathrm{j}L\omega}\\
			&= \frac{\mathrm{j}RC\omega}{1 + \mathrm{j}C\omega(R + R_\text{g}) - LC\omega^2}.
		\end{align*}
		On peut donc déterminer une expression du facteur de qualité~$Q_{\text{A},R}$\/ et de la pulsation propre~$\omega_{0,\text{A},R}$\/ : \[
			\omega_{0,\text{A}} = \frac{1}{\sqrt{LC}}
		\] et \[
			Q_{\text{A}} = \frac{1}{R + R_\text{g}} \sqrt{\frac{L}{C}}
		.\]

		On s'intéresse à présent au circuit~B.
		La resistance équivalente aux deux résistances en parallèle est~$\sfrac{R}{2}$.
		De même, à l'aide d'un pont diviseur de tension, on a
		\begin{align*}
			\ubar{H}_{\text{B},R_\text{éq}}(\mathrm{j}\omega) &= \frac{\frac{R}{2}}{\frac{R}{2} + R_\text{g} + \frac{1}{\mathrm{j}C\omega} + \mathrm{j}L\omega}\\
			&= \frac{\frac{1}{2}\mathrm{j}RC\omega}{1 + \mathrm{j}C\omega\left( \frac{R}{2} + R_\text{g} \right) - LC\omega^2} \\
		\end{align*}
		D'où \[
			\omega_{0,\text{B}} = \frac{1}{\sqrt{LC}}\quad\text{et}\quad
			Q_{\text{B}} = \frac{1}{\frac{R}{2}+R_\text{g}} \sqrt{\frac{L}{C}}
		.\]

		Comme l'expression de $\omega_{0,\text{A}}$\/ et celle de $\omega_{0,\text{B}}$\/ est identique, pour la suite de l'exercice, on note~$\omega_0$\/ cette pulsation propre.

		Des deux fonctions de transfert précédentes, on en déduit les gains $G_{\text{A},R}$\/ et $G_{\text{B},R}$\/ : pour les circuits~A et~B,\[
			G_{R} = \frac{1}{\sqrt{1 + Q^2\left( \dfrac{\omega}{\omega_0} - \dfrac{\omega_0}{\omega} \right)^2}}
		.\] Comme $Q_\text{B} > Q_\text{A}$, on en déduit $G_{\text{B},R} < G_{\text{A},R}$. La courbe bleue correspond donc à une mesure de la tension aux bornes de la résistance du circuit~A, et la verte à la résistance du circuit~B.
		\bigskip

		En basse fréquence, la tension aux bornes d'une bobine est nulle. Comme ce n'est pas le cas des courbes de la figure~2, on en déduit qu'il s'agit de condensateurs.

		À présent, on détermine les fonctions de transfert aux bornes du condensateur à l'aide d'un pont diviseur de tension :
		\begin{align*}
			\ubar{H}_{\text{A},C}(\mathrm{j}\omega) &= \frac{\frac{1}{\mathrm{j}C\omega}}{R + R_\text{g} + \frac{1}{\mathrm{j}C\omega} + \mathrm{j}L\omega}\\
			&= \frac{1}{1 + \mathrm{j}C\omega(R + R_\text{g}) - LC\omega^2} \\
		\end{align*}
		et, de même \[
			\ubar{H}_{\text{B},C} = \frac{1}{1+\mathrm{j}C\omega\left( \frac{R}{2} + R_\text{g} \right) - LC\omega^2}
		.\]
		On en déduit l'expression du gain $G_{\text{A},C}$\/ et $G_{\text{B},C}$\/ : \[
			G_C = \frac{1}{\sqrt{\left( 1 - \left( \frac{\omega}{\omega_0} \right)^2 \right)^2 + \left( \frac{\omega}{Q\omega_0} \right)^2}}
		.\]
		On remarque que, comme $Q_\text{B} > Q_\text{A}$, $G_{\text{A},C} > G_{\text{B},C}$. Dans la figure~2, la courbe bleue correspond donc au circuit~B et la verte au circuit~A.

		\bigskip
		\noindent {\large\bf Question 2}\hrule

		En basse fréquence, dans le circuit~A, on a~$E = u_L + u_R + u_C$, or,~$u_L$\/ et~$u_R$\/ sont nulles. On en déduit~$E = u_L$\/ en basse fréquence. D'où $E = 15\:\mathrm{V}$.

		À l'aide d'une mesure sur le graphique, on détermine la bande passante des circuits~A ($\mathrm{\Delta}\omega_\text{A}$) et~B ($\mathrm{\Delta}\omega_\text{B}$) sur la figure~1. On a $\smash{\sfrac{u_{\text{A},R,\max}}{\sqrt{2}}} = \smash{5\sqrt{2}\:\mathrm{V}} \cong 7{,}1\:\mathrm{V}$. D'où $\mathrm{\Delta}\omega_{\text{A}} \cong 250\:\mathrm{Hz}$.
		De même, avec le graphe du circuit~B : \[
			\frac{u_{\text{B},R,\max}}{\sqrt{2}} \cong 5{,}3\:\mathrm{V}\quad\text{d'où}\quad \mathrm{\Delta}\omega_\text{B} = 150\:\mathrm{Hz}
		.\] Les mesures de bande passante sont représentées sur la figure II.
	\end{multicols}
	\begin{figure}[H]
		\centering
		\scalebox{0.7}{\incfig{diag1}}
		\caption{Mesure de $\mathrm{\Delta}\omega_\text{A}$, $\mathrm{\Delta}\omega_\text{B}$\/ et $\omega_0$}
	\end{figure}
	\begin{multicols}{2}
		Or, on sait que~$\mathrm{\Delta}\omega_\text{A} = \sfrac{\omega_0}{Q_\text{A}}$\/ et~$\mathrm{\Delta}\omega_\text{B} = \sfrac{\omega_0}{Q_\text{B}}$.
		Après lecture graphique, comme montré sur la figure~II, on a~$\omega_0 \cong 160\:\mathrm{Hz}$.
		On en déduit donc la valeur de $Q_\text{A}$\/ et $Q_\text{B}$\/ : \[
			Q_\text{A} = \frac{\omega_0}{\mathrm{\Delta}\omega_\text{A}} \cong 0{,}64\qquad\text{et}\qquad Q_\text{B} \cong 1{,}07
		.\]

		Nous pouvons désormais déterminer $R_\text{g}$, $R$, $C$, et $L$.
		On sait que $C = \sfrac{1}{L\omega_0^2}$, d'où \[
			Q_\text{A} = \frac{1}{R + R_\text{g}} \sqrt{\frac{L}{C}} = \frac{L\omega_0}{R + R_\text{g}},
		\] d'où $L = \frac{1}{\omega_0}Q_\text{A}(R+R_\text{g})$, et, de même, $L = \frac{1}{\omega_0}\*Q_\text{B}\*\left( \frac{R}{2} + R_\text{g} \right)$.
		Donc $Q_\text{A}\*(R + R_\text{g}) = Q_\text{B}\*\left( \frac{R}{2} + R_\text{g} \right)$. D'où  \[
			R = \frac{Q_\text{B} - Q_\text{A}}{Q_\text{A} - \frac{1}{2}Q_\text{B}} \cdot R_\text{g}
		.\]
		On déduit également que $C = \sfrac{1}{\omega_0Q_\text{A} (R + R_\text{g})}$.
		On utilise la valeur mesurée sur le graphe. On sait que le gain $G_{\text{A},C}$\/ s'écrit de la forme \[
			G_{\text{A},C} = \dfrac{1}{\sqrt{\left( 1 - \left( \dfrac{\omega}{\omega_0} \right)^2 \right)^2 + \left( \dfrac{\omega}{Q_\text{A}\omega_0} \right)^2}}.
		\]

		\smallskip

		Nous n'avons pas réussi à déterminer les valeurs numériques de $R$, $R_\text{g}$, $C$\/ et $L$. Nous avons trouvés 3 équations pour 4 inconnues.
		Les expressions de $R$, $L$\/ et $C$\/ sont toutes en fonction de $R_\text{g}$\/ et des constantes $\omega_0$, $Q_\text{A}$, $Q_\text{B}$\/ : 
		\begin{gather*}
			\boxed{R = \frac{Q_\text{B} - Q_\text{A}}{Q_\text{A} - \frac{1}{2}Q_\text{B}} \cdot R_\text{g}}\ ;\\
			\boxed{C = \frac{1}{\omega_0 Q_\text{A} (R + R_\text{g})}}\ ;\\
			\boxed{L = \frac{1}{\omega_0}Q_\text{A} (R + R_\text{g}).}
		\end{gather*}

		\bigskip
		\noindent {\large\bf Question 3}\hrule

		Afin de réaliser un filtre passe-haut, on utilise la tension aux bornes de la bobine comme tension de sortie. En effet, la fonction de transfert s'écrit de la façon suivante : 
		\begin{align*}
			\ubar{H}_{\text{A},L}(\mathrm{j}\omega) &= \frac{\mathrm{j}L\omega}{R + R_\text{g} + \frac{1}{\mathrm{j}C\omega} + \mathrm{j}L\omega}\\
			&= \frac{-LC\omega^2}{1 + \mathrm{j}C\omega(R + R_\text{g}) - LC\omega^2} \\
		\end{align*}
		et on recoonnaît la forme d'un filtre passe-haut d'ordre 2. On représente son diagramme de {\sc Bode}\/ sur la figure~III.
	\end{multicols}
	\begin{figure}[H]
		\centering
		\begin{asy}
			import graph;
			size(10cm);
			draw((-1,0)--(10,0), Arrow(TeXHead));
			draw((0,-5)--(0,1), Arrow(TeXHead));
			label("$\omega$", (10,0), align=E);
			real w0 = 5;
			real G(real x) {
				if(x < 5) {
					return (x-5)*0.8;
				} else {
					return 0;
				}
			}
			bool3 gcheck(real x) { return G(x) > -5; }
			draw(graph(G,0,10,500,gcheck), magenta);
			label("$\omega_0$", (5,0), align=2*N);
			draw((5,-0.1)--(5,0.1));
			label("$G_\text{dB}$", (0,1), magenta, align=NE);
		\end{asy}

		\bigskip

		\begin{asy}
			import graph;
			size(10cm);
			draw((-1,0)--(10,0), Arrow(TeXHead));
			draw((0,-1)--(0,5), Arrow(TeXHead));
			label("$\omega$", (10,0), align=E);
			draw((0,4)--(5,4)--(5,0)--(10,0), deepcyan);
			label("$\omega_0$", (5,0), align=2*S);
			draw((5,-0.1)--(5,0.1));
			draw((-0.1,4)--(0.1,4));
			label("$\varphi$", (0,5), deepcyan, align=NE);
			label("$\ds\frac\pi2$", (0,4), align=W);
		\end{asy}
		\caption{Forme du diagramme de {\sc Bode}\/ du filtre passe-haut décrit dans la question 3, en gain $G_\text{dB}$\/ et en phase $\varphi$}
	\end{figure}
\end{document}
