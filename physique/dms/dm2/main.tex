\documentclass[a4paper]{article}

\usepackage{tgschola}
\usepackage[margin=1in]{geometry}
\usepackage[utf8]{inputenc}
\usepackage[T1]{fontenc}
\usepackage{mathrsfs}
\usepackage{textcomp}
\usepackage[french]{babel}
\usepackage{amsmath}
\usepackage{amssymb}
\usepackage{cancel}
\usepackage{frcursive}
\usepackage[inline]{asymptote}
\usepackage{tikz}
\usepackage[european,straightvoltages,europeanresistors]{circuitikz}
\usepackage{tikz-cd}
\usepackage{tkz-tab}
\usepackage[b]{esvect}
\usepackage[framemethod=TikZ]{mdframed}
\usepackage{centernot}
\usepackage{diagbox}
\usepackage{dsfont}
\usepackage{fancyhdr}
\usepackage{float}
\usepackage{graphicx}
\usepackage{listings}
\usepackage{multicol}
\usepackage{nicematrix}
\usepackage{pdflscape}
\usepackage{stmaryrd}
\usepackage{xfrac}
\usepackage{hep-math-font}
\usepackage{amsthm}
\usepackage{thmtools}
\usepackage{indentfirst}
\usepackage[framemethod=TikZ]{mdframed}
\usepackage{accents}
\usepackage{soulutf8}
\usepackage{mathtools}
\usepackage{bodegraph}
\usepackage{slashbox}
\usepackage{enumitem}
\usepackage{calligra}
\usepackage{cinzel}
\usepackage{BOONDOX-calo}

% Tikz
\usetikzlibrary{babel}
\usetikzlibrary{positioning}
\usetikzlibrary{calc}

% global settings
\frenchspacing
\reversemarginpar
\setuldepth{a}

%\everymath{\displaystyle}

\frenchbsetup{StandardLists=true}

\def\asydir{asy}

%\sisetup{exponent-product=\cdot,output-decimal-marker={,},separate-uncertainty,range-phrase=\;à\;,locale=FR}

\setlength{\parskip}{1em}

\theoremstyle{definition}

% Changing math
\let\emptyset\varnothing
\let\ge\geqslant
\let\le\leqslant
\let\preceq\preccurlyeq
\let\succeq\succcurlyeq
\let\ds\displaystyle
\let\ts\textstyle

\newcommand{\C}{\mathds{C}}
\newcommand{\R}{\mathds{R}}
\newcommand{\Z}{\mathds{Z}}
\newcommand{\N}{\mathds{N}}
\newcommand{\Q}{\mathds{Q}}

\renewcommand{\O}{\emptyset}

\newcommand\ubar[1]{\underaccent{\bar}{#1}}

\renewcommand\Re{\expandafter\mathfrak{Re}}
\renewcommand\Im{\expandafter\mathfrak{Im}}

\let\slantedpartial\partial
\DeclareRobustCommand{\partial}{\text{\rotatebox[origin=t]{20}{\scalebox{0.95}[1]{$\slantedpartial$}}}\hspace{-1pt}}

% merging two maths characters w/ \charfusion
\makeatletter
\def\moverlay{\mathpalette\mov@rlay}
\def\mov@rlay#1#2{\leavevmode\vtop{%
   \baselineskip\z@skip \lineskiplimit-\maxdimen
   \ialign{\hfil$\m@th#1##$\hfil\cr#2\crcr}}}
\newcommand{\charfusion}[3][\mathord]{
    #1{\ifx#1\mathop\vphantom{#2}\fi
        \mathpalette\mov@rlay{#2\cr#3}
      }
    \ifx#1\mathop\expandafter\displaylimits\fi}
\makeatother

% custom math commands
\newcommand{\T}{{\!\!\,\top}}
\newcommand{\avrt}[1]{\rotatebox{-90}{$#1$}}
\newcommand{\bigcupdot}{\charfusion[\mathop]{\bigcup}{\cdot}}
\newcommand{\cupdot}{\charfusion[\mathbin]{\cup}{\cdot}}
%\newcommand{\danger}{{\large\fontencoding{U}\fontfamily{futs}\selectfont\char 66\relax}\;}
\newcommand{\tendsto}[1]{\xrightarrow[#1]{}}
\newcommand{\vrt}[1]{\rotatebox{90}{$#1$}}
\newcommand{\tsup}[1]{\textsuperscript{\underline{#1}}}
\newcommand{\tsub}[1]{\textsubscript{#1}}

\renewcommand{\mod}[1]{~\left[ #1 \right]}
\renewcommand{\t}{{}^t\!}
\newcommand{\s}{\text{\calligra s}}

% custom units / constants
%\DeclareSIUnit{\litre}{\ell}
\let\hbar\hslash

% header / footer
\pagestyle{fancy}
\fancyhead{} \fancyfoot{}
\fancyfoot[C]{\thepage}

% fonts
\let\sc\scshape
\let\bf\bfseries
\let\it\itshape
\let\sl\slshape

% custom math operators
\let\th\relax
\let\det\relax
\DeclareMathOperator*{\codim}{codim}
\DeclareMathOperator*{\dom}{dom}
\DeclareMathOperator*{\gO}{O}
\DeclareMathOperator*{\po}{\text{\cursive o}}
\DeclareMathOperator*{\sgn}{sgn}
\DeclareMathOperator*{\simi}{\sim}
\DeclareMathOperator{\Arccos}{Arccos}
\DeclareMathOperator{\Arcsin}{Arcsin}
\DeclareMathOperator{\Arctan}{Arctan}
\DeclareMathOperator{\Argsh}{Argsh}
\DeclareMathOperator{\Arg}{Arg}
\DeclareMathOperator{\Aut}{Aut}
\DeclareMathOperator{\Card}{Card}
\DeclareMathOperator{\Cl}{\mathcal{C}\!\ell}
\DeclareMathOperator{\Cov}{Cov}
\DeclareMathOperator{\Ker}{Ker}
\DeclareMathOperator{\Mat}{Mat}
\DeclareMathOperator{\PGCD}{PGCD}
\DeclareMathOperator{\PPCM}{PPCM}
\DeclareMathOperator{\Supp}{Supp}
\DeclareMathOperator{\Vect}{Vect}
\DeclareMathOperator{\argmax}{argmax}
\DeclareMathOperator{\argmin}{argmin}
\DeclareMathOperator{\ch}{ch}
\DeclareMathOperator{\com}{com}
\DeclareMathOperator{\cotan}{cotan}
\DeclareMathOperator{\det}{det}
\DeclareMathOperator{\id}{id}
\DeclareMathOperator{\rg}{rg}
\DeclareMathOperator{\rk}{rk}
\DeclareMathOperator{\sh}{sh}
\DeclareMathOperator{\th}{th}
\DeclareMathOperator{\tr}{tr}

% colors and page style
\definecolor{truewhite}{HTML}{ffffff}
\definecolor{white}{HTML}{faf4ed}
\definecolor{trueblack}{HTML}{000000}
\definecolor{black}{HTML}{575279}
\definecolor{mauve}{HTML}{907aa9}
\definecolor{blue}{HTML}{286983}
\definecolor{red}{HTML}{d7827e}
\definecolor{yellow}{HTML}{ea9d34}
\definecolor{gray}{HTML}{9893a5}
\definecolor{grey}{HTML}{9893a5}
\definecolor{green}{HTML}{a0d971}

\pagecolor{white}
\color{black}

\begin{asydef}
	settings.prc = false;
	settings.render=0;

	white = rgb("faf4ed");
	black = rgb("575279");
	blue = rgb("286983");
	red = rgb("d7827e");
	yellow = rgb("f6c177");
	orange = rgb("ea9d34");
	gray = rgb("9893a5");
	grey = rgb("9893a5");
	deepcyan = rgb("56949f");
	pink = rgb("b4637a");
	magenta = rgb("eb6f92");
	green = rgb("a0d971");
	purple = rgb("907aa9");

	defaultpen(black + fontsize(8pt));

	import three;
	currentlight = nolight;
\end{asydef}

% theorems, proofs, ...

\mdfsetup{skipabove=1em,skipbelow=1em, innertopmargin=6pt, innerbottommargin=6pt,}

\declaretheoremstyle[
	headfont=\normalfont\itshape,
	numbered=no,
	postheadspace=\newline,
	headpunct={:},
	qed=\qedsymbol]{demstyle}

\declaretheorem[style=demstyle, name=Démonstration]{dem}

\newcommand\veczero{\kern-1.2pt\vec{\kern1.2pt 0}} % \vec{0} looks weird since the `0' isn't italicized

\makeatletter
\renewcommand{\title}[2]{
	\AtBeginDocument{
		\begin{titlepage}
			\begin{center}
				\vspace{10cm}
				{\Large \sc Chapitre #1}\\
				\vspace{1cm}
				{\Huge \calligra #2}\\
				\vfill
				Hugo {\sc Salou} MPI${}^{\star}$\\
				{\small Dernière mise à jour le \@date }
			\end{center}
		\end{titlepage}
	}
}

\newcommand{\titletp}[4]{
	\AtBeginDocument{
		\begin{titlepage}
			\begin{center}
				\vspace{10cm}
				{\Large \sc tp #1}\\
				\vspace{1cm}
				{\Huge \textsc{\textit{#2}}}\\
				\vfill
				{#3}\textit{MPI}${}^{\star}$\\
			\end{center}
		\end{titlepage}
	}
	\fancyfoot{}\fancyhead{}
	\fancyfoot[R]{#4 \textit{MPI}${}^{\star}$}
	\fancyhead[C]{{\sc tp #1} : #2}
	\fancyhead[R]{\thepage}
}

\newcommand{\titletd}[2]{
	\AtBeginDocument{
		\begin{titlepage}
			\begin{center}
				\vspace{10cm}
				{\Large \sc td #1}\\
				\vspace{1cm}
				{\Huge \calligra #2}\\
				\vfill
				Hugo {\sc Salou} MPI${}^{\star}$\\
				{\small Dernière mise à jour le \@date }
			\end{center}
		\end{titlepage}
	}
}
\makeatother

\newcommand{\sign}{
	\null
	\vfill
	\begin{center}
		{
			\fontfamily{ccr}\selectfont
			\textit{\textbf{\.{\"i}}}
		}
	\end{center}
	\vfill
	\null
}

\renewcommand{\thefootnote}{\emph{\alph{footnote}}}

% figure support
\usepackage{import}
\usepackage{xifthen}
\pdfminorversion=7
\usepackage{pdfpages}
\usepackage{transparent}
\newcommand{\incfig}[1]{%
	\def\svgwidth{\columnwidth}
	\import{./figures/}{#1.pdf_tex}
}

\pdfsuppresswarningpagegroup=1
\ctikzset{tripoles/european not symbol=circle}

\newcommand{\missingpart}{{\large\color{red} Il manque quelque chose ici\ldots}}


\fancyhead{} \fancyfoot{}
\fancyfoot[C]{--\:\thepage\:--}
\fancyhead[R]{DM$_2$ Physique}

\pagecolor{truewhite}
\definecolor{black}{HTML}{000000}
\color{black}

\begin{document}
	\begin{center}
		\LARGE\sc Problème n\textsuperscript o\,2$^\star$ : Motorisation ultra-sonore\\
		d’un autofocus
	\end{center}

	\begin{multicols}{2}
		\noindent {\large\bfseries Questions III.A.1}\quad\hrule

		\begin{enumerate}[label=({\it\alph*\/})]
			\item Si la barre était infiniment souple, elle vérifierai l'équation d'onde de {\sc d'Alembert} sans pertes : \[
					\frac{\partial^2z}{\partial x^2} - \frac{1}{c^2} \frac{\partial^2 z}{\partial t^2} = 0
				.\]
				Ses solutions générales sont de la forme d'une somme de deux ondes, une progressive et une régressive : \[
					F(x - ct) + G(x + ct)
				.\]

			\item La barre ayant une rigidité importante, elle n'est pas négligeable comme dans le cas d'une corde souple. C'est pour cela que l'équation différentielle ne correspond pas à l'équation de {\sc d'Alembert}.

			\item De l'équation différentielle de l'énoncé, on en déduit que \[
					\left[ \frac{\partial^4 z}{\partial x^4} \right] = \left[ \frac{\partial^2 z}{\partial t^2} \right] \times \big[\gamma^2\big]
				.\]
				D'où, \[
					\big[\gamma^2\big] = \frac{\left[ \frac{\partial^4 z}{\partial x^4} \right]}{\left[ \frac{\partial^2 z}{\partial t^2}  \right]} = \frac{\mathrm{L} \cdot \mathrm{L}^{-4}}{\mathrm{L} \cdot \mathrm{T}^{-2}} = \Big(\frac{\mathrm{T}}{L^2}\Big)^2
				.\]
				On en déduit donc que $[\gamma] = \mathrm{T} \cdot \mathrm{L}^{-2}$. Ainsi, l'unité de $\gamma$\/ serait $\mathrm{s}/\mathrm{m}^2$.
			\item L'onde ayant pour forme \[
					\ubar{z}(x,t) = \ubar{Z}\exp\big(\mathrm{j}(\omega t - \ubar{k} x)\big)\]
					est une onde progressive si $\omega$\/ et $\ubar{k}$\/ sont des réels du même signe, et non nuls.
				\item On utilise l'équation différentielle donnée et la forme de $\ubar{z}$\/ complexe : on a \[
						\Big((\mathrm{j}\ubar{k})^4 + \gamma^2(\mathrm{j} \omega)^2\Big)\:\ubar{z} = 0
				\] d'où, $\ubar{k}^4 - \gamma^2 \omega ^2 = 0$, et donc $\ubar{k}^2 = \gamma \omega$, d'où \[\ubar{k} = \sqrt{\gamma \omega}.\]
				Or, la vitesse de phase $v_\varphi$\/ s'écrit de la forme
				\[
					v_\varphi = \frac{\omega}{\ubar{k}} = \frac{\omega}{\sqrt{\omega \gamma}} = \sqrt{\frac{\omega}{\gamma}}
				\] qui dépend de $\omega$. La barre est donc un milieu dispersif.
			\item La barre reposant, à tout instant $t$, sur les supports à chaque extrémités de celle-ci, on a donc \[
				\forall t,\:z(x=0,t) = z(x=L, t) = 0
			.\]
			Une onde plane progressive ne respecte pas ces contraintes, mais c'est le cas d'une onde stationnaire.
		\end{enumerate}
		\noindent {\large\bfseries Questions III.A.2}\quad\hrule
		\begin{enumerate}[label=({\it\alph*\/})]
			\item Afin de respecter les conditions limites, on doit avoir un ventre à chaque extrémité de la barre. Ainsi, on doit donc avoir $L = n \lambda$\/ avec $n \in \N^*$.
			\item Comme le vecteur d'onde $k = \frac{2\pi}{\lambda}$ est quantifié, on a donc \[k_n = \frac{2\pi n }{L}.\] Et, comme la pulsation $\omega = \frac{k^2}{\gamma}$ est également quantifiée, on en déduit que \[
				\omega_n = \frac{4\pi^2 n^2}{\gamma L^2}
			.\]
		\end{enumerate}
		\noindent {\large\bfseries Questions III.B.1}\quad\hrule
		\begin{enumerate}[label=({\it\alph*\/})]
			\item Les nœuds de vibrations sont situés entre les unités de polarisation. En effet, la polarisation de chaque côtés de ces nœuds est inversée.
			\item On utilise les modes vibratoires : la figure~7 représente 9 nœuds. De plus, la longueur d'arc entre les nœuds est de $\lambda / 2 = R\times \pi/9$ (l'angle est de $20^\circ = \pi/9$) d'après la figure~7. Or, comme $k = 2\pi / \lambda$, on en déduit que $k = R / 9$. D'où \[\omega = \frac{k^2}{\gamma} = \frac{81}{\gamma R^2}.\]
		\end{enumerate}
		\noindent {\large\bfseries Questions III.B.2}\quad\hrule
		\begin{enumerate}[label=({\it\alph*\/})]
			\item On calcule $z_1(s,t) + z_2(s,t)$\/ :
				\begin{align*}
					&z_1(s,t) + z_2(s,t)\\
					=&\: Z\big(\cos(k s + \psi_1)\cos(\omega t + \varphi_1)\\
										&\:\,+ \cos(k s + \psi_2)\cos(\omega t + \varphi_2)\big)
				\end{align*}
				Et, d'autre part, on développe $z(s,t)$\/ :
				\begin{align*}
					z(s,t) &= Z\cos(ks + \omega t + \varphi_1 - \psi_1)\\
					&= Z\big(\cos(k s + \psi_1)\cos(\omega t + \varphi_1)\\
										&\mathrel{\phantom{=}}\:\,+ \sin(k s + \psi_2)\sin(\omega t + \varphi_2)\big)
				\end{align*}
				L'égalité $z = z_1 + z_2$\/ n'est possible que si \[\varphi_2 = \frac{\pi}{2} - \varphi_1\quad\text{et}\quad\psi_2 = \frac{\pi}{2} - \psi_1.\]
			\item Comme il y a un déphasage en temps de $-\frac{\pi}{2}$\/ entre $U_2$\/ et $U_1$, on en déduit qu'il y a un déphasage en longueur de \[
					\frac{-\pi / 2}{2\pi} = -\frac{\lambda}{4}
				.\]
				Ainsi, la longueur de l'arc moyen de l'électrode auxiliaire est de $\lambda / 4$. On en déduit que la longueur de l'arc moyen de l'électrode {\sc gnd}\/ est de $\lambda - (\lambda / 4) = \sfrac{3}{4}\times \lambda$.
			\item Afin d'inverser le sens du rotor, le déphasage entre $U_2$\/ et $U_1$\/ doit être de $\pm\frac{\pi}{2}$. Or, on a vu précédemment que ce déphasage valait $\frac{\pi}{2}$. Ainsi, on calcule le déphasage lié à la fonction de transfert $\ubar{H} = \ubar{U}_2 / \ubar{U}_1$\/ :
				\begin{align*}
					&\mathrel{\phantom=} \Arg(\ubar{H})\\
					&= \Arg(1+\mathrm{j}R'C\omega) - \Arg(1 - \mathrm{j}R'C\omega)\\
					&= \Arctan(R'C\omega) - \Arctan(-R'C\omega) \\
					&= 2\Arctan(R'C\omega)\\
				\end{align*}
				D'où, $\Arctan(R'C\omega) = \frac{\pi}{4}$. On en déduit donc que \[
					R' C \omega = 1
				.\]
			\item Si on modifie $R'$\/ ou $C$\/ de telle sorte que $R' C \omega = 1$, alors le signal $U_2$\/ sera donc inversé. La rotation du stator sera donc inversée : en effet, comme vu dans la question III.B.1.\textit b, on a donc $\varphi_2 = \varphi_1 - \frac{\pi}{2}$, ce qui inverse le sens de propagation du signal $z$.
		\end{enumerate}
	\end{multicols}
\end{document}
