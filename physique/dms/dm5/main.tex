\documentclass[a4paper, 10pt]{article}

\usepackage[margin=1in]{geometry}
\usepackage[utf8]{inputenc}
\usepackage[T1]{fontenc}
\usepackage{mathrsfs}
\usepackage{textcomp}
\usepackage[french]{babel}
\usepackage{amsmath}
\usepackage{amssymb}
\usepackage{cancel}
\usepackage{frcursive}
\usepackage[inline]{asymptote}
\usepackage{tikz}
\usepackage[european,straightvoltages,europeanresistors]{circuitikz}
\usepackage{tikz-cd}
\usepackage{tkz-tab}
\usepackage[b]{esvect}
\usepackage[framemethod=TikZ]{mdframed}
\usepackage{centernot}
\usepackage{diagbox}
\usepackage{dsfont}
\usepackage{fancyhdr}
\usepackage{float}
\usepackage{graphicx}
\usepackage{listings}
\usepackage{multicol}
\usepackage{nicematrix}
\usepackage{pdflscape}
\usepackage{stmaryrd}
\usepackage{xfrac}
\usepackage{hep-math-font}
\usepackage{amsthm}
\usepackage{thmtools}
\usepackage{indentfirst}
\usepackage[framemethod=TikZ]{mdframed}
\usepackage{accents}
\usepackage{soulutf8}
\usepackage{mathtools}
\usepackage{bodegraph}
\usepackage{slashbox}
\usepackage{enumitem}
\usepackage{calligra}
\usepackage{cinzel}
\usepackage{BOONDOX-calo}

% Tikz
\usetikzlibrary{babel}
\usetikzlibrary{positioning}
\usetikzlibrary{calc}

% global settings
\frenchspacing
\reversemarginpar
\setuldepth{a}

%\everymath{\displaystyle}

\frenchbsetup{StandardLists=true}

\def\asydir{asy}

%\sisetup{exponent-product=\cdot,output-decimal-marker={,},separate-uncertainty,range-phrase=\;à\;,locale=FR}

\setlength{\parskip}{1em}

\theoremstyle{definition}

% Changing math
\let\emptyset\varnothing
\let\ge\geqslant
\let\le\leqslant
\let\preceq\preccurlyeq
\let\succeq\succcurlyeq
\let\ds\displaystyle
\let\ts\textstyle

\newcommand{\C}{\mathds{C}}
\newcommand{\R}{\mathds{R}}
\newcommand{\Z}{\mathds{Z}}
\newcommand{\N}{\mathds{N}}
\newcommand{\Q}{\mathds{Q}}

\renewcommand{\O}{\emptyset}

\newcommand\ubar[1]{\underaccent{\bar}{#1}}

\renewcommand\Re{\expandafter\mathfrak{Re}}
\renewcommand\Im{\expandafter\mathfrak{Im}}

\let\slantedpartial\partial
\DeclareRobustCommand{\partial}{\text{\rotatebox[origin=t]{20}{\scalebox{0.95}[1]{$\slantedpartial$}}}\hspace{-1pt}}

% merging two maths characters w/ \charfusion
\makeatletter
\def\moverlay{\mathpalette\mov@rlay}
\def\mov@rlay#1#2{\leavevmode\vtop{%
   \baselineskip\z@skip \lineskiplimit-\maxdimen
   \ialign{\hfil$\m@th#1##$\hfil\cr#2\crcr}}}
\newcommand{\charfusion}[3][\mathord]{
    #1{\ifx#1\mathop\vphantom{#2}\fi
        \mathpalette\mov@rlay{#2\cr#3}
      }
    \ifx#1\mathop\expandafter\displaylimits\fi}
\makeatother

% custom math commands
\newcommand{\T}{{\!\!\,\top}}
\newcommand{\avrt}[1]{\rotatebox{-90}{$#1$}}
\newcommand{\bigcupdot}{\charfusion[\mathop]{\bigcup}{\cdot}}
\newcommand{\cupdot}{\charfusion[\mathbin]{\cup}{\cdot}}
%\newcommand{\danger}{{\large\fontencoding{U}\fontfamily{futs}\selectfont\char 66\relax}\;}
\newcommand{\tendsto}[1]{\xrightarrow[#1]{}}
\newcommand{\vrt}[1]{\rotatebox{90}{$#1$}}
\newcommand{\tsup}[1]{\textsuperscript{\underline{#1}}}
\newcommand{\tsub}[1]{\textsubscript{#1}}

\renewcommand{\mod}[1]{~\left[ #1 \right]}
\renewcommand{\t}{{}^t\!}
\newcommand{\s}{\text{\calligra s}}

% custom units / constants
%\DeclareSIUnit{\litre}{\ell}
\let\hbar\hslash

% header / footer
\pagestyle{fancy}
\fancyhead{} \fancyfoot{}
\fancyfoot[C]{\thepage}

% fonts
\let\sc\scshape
\let\bf\bfseries
\let\it\itshape
\let\sl\slshape

% custom math operators
\let\th\relax
\let\det\relax
\DeclareMathOperator*{\codim}{codim}
\DeclareMathOperator*{\dom}{dom}
\DeclareMathOperator*{\gO}{O}
\DeclareMathOperator*{\po}{\text{\cursive o}}
\DeclareMathOperator*{\sgn}{sgn}
\DeclareMathOperator*{\simi}{\sim}
\DeclareMathOperator{\Arccos}{Arccos}
\DeclareMathOperator{\Arcsin}{Arcsin}
\DeclareMathOperator{\Arctan}{Arctan}
\DeclareMathOperator{\Argsh}{Argsh}
\DeclareMathOperator{\Arg}{Arg}
\DeclareMathOperator{\Aut}{Aut}
\DeclareMathOperator{\Card}{Card}
\DeclareMathOperator{\Cl}{\mathcal{C}\!\ell}
\DeclareMathOperator{\Cov}{Cov}
\DeclareMathOperator{\Ker}{Ker}
\DeclareMathOperator{\Mat}{Mat}
\DeclareMathOperator{\PGCD}{PGCD}
\DeclareMathOperator{\PPCM}{PPCM}
\DeclareMathOperator{\Supp}{Supp}
\DeclareMathOperator{\Vect}{Vect}
\DeclareMathOperator{\argmax}{argmax}
\DeclareMathOperator{\argmin}{argmin}
\DeclareMathOperator{\ch}{ch}
\DeclareMathOperator{\com}{com}
\DeclareMathOperator{\cotan}{cotan}
\DeclareMathOperator{\det}{det}
\DeclareMathOperator{\id}{id}
\DeclareMathOperator{\rg}{rg}
\DeclareMathOperator{\rk}{rk}
\DeclareMathOperator{\sh}{sh}
\DeclareMathOperator{\th}{th}
\DeclareMathOperator{\tr}{tr}

% colors and page style
\definecolor{truewhite}{HTML}{ffffff}
\definecolor{white}{HTML}{faf4ed}
\definecolor{trueblack}{HTML}{000000}
\definecolor{black}{HTML}{575279}
\definecolor{mauve}{HTML}{907aa9}
\definecolor{blue}{HTML}{286983}
\definecolor{red}{HTML}{d7827e}
\definecolor{yellow}{HTML}{ea9d34}
\definecolor{gray}{HTML}{9893a5}
\definecolor{grey}{HTML}{9893a5}
\definecolor{green}{HTML}{a0d971}

\pagecolor{white}
\color{black}

\begin{asydef}
	settings.prc = false;
	settings.render=0;

	white = rgb("faf4ed");
	black = rgb("575279");
	blue = rgb("286983");
	red = rgb("d7827e");
	yellow = rgb("f6c177");
	orange = rgb("ea9d34");
	gray = rgb("9893a5");
	grey = rgb("9893a5");
	deepcyan = rgb("56949f");
	pink = rgb("b4637a");
	magenta = rgb("eb6f92");
	green = rgb("a0d971");
	purple = rgb("907aa9");

	defaultpen(black + fontsize(8pt));

	import three;
	currentlight = nolight;
\end{asydef}

% theorems, proofs, ...

\mdfsetup{skipabove=1em,skipbelow=1em, innertopmargin=6pt, innerbottommargin=6pt,}

\declaretheoremstyle[
	headfont=\normalfont\itshape,
	numbered=no,
	postheadspace=\newline,
	headpunct={:},
	qed=\qedsymbol]{demstyle}

\declaretheorem[style=demstyle, name=Démonstration]{dem}

\newcommand\veczero{\kern-1.2pt\vec{\kern1.2pt 0}} % \vec{0} looks weird since the `0' isn't italicized

\makeatletter
\renewcommand{\title}[2]{
	\AtBeginDocument{
		\begin{titlepage}
			\begin{center}
				\vspace{10cm}
				{\Large \sc Chapitre #1}\\
				\vspace{1cm}
				{\Huge \calligra #2}\\
				\vfill
				Hugo {\sc Salou} MPI${}^{\star}$\\
				{\small Dernière mise à jour le \@date }
			\end{center}
		\end{titlepage}
	}
}

\newcommand{\titletp}[4]{
	\AtBeginDocument{
		\begin{titlepage}
			\begin{center}
				\vspace{10cm}
				{\Large \sc tp #1}\\
				\vspace{1cm}
				{\Huge \textsc{\textit{#2}}}\\
				\vfill
				{#3}\textit{MPI}${}^{\star}$\\
			\end{center}
		\end{titlepage}
	}
	\fancyfoot{}\fancyhead{}
	\fancyfoot[R]{#4 \textit{MPI}${}^{\star}$}
	\fancyhead[C]{{\sc tp #1} : #2}
	\fancyhead[R]{\thepage}
}

\newcommand{\titletd}[2]{
	\AtBeginDocument{
		\begin{titlepage}
			\begin{center}
				\vspace{10cm}
				{\Large \sc td #1}\\
				\vspace{1cm}
				{\Huge \calligra #2}\\
				\vfill
				Hugo {\sc Salou} MPI${}^{\star}$\\
				{\small Dernière mise à jour le \@date }
			\end{center}
		\end{titlepage}
	}
}
\makeatother

\newcommand{\sign}{
	\null
	\vfill
	\begin{center}
		{
			\fontfamily{ccr}\selectfont
			\textit{\textbf{\.{\"i}}}
		}
	\end{center}
	\vfill
	\null
}

\renewcommand{\thefootnote}{\emph{\alph{footnote}}}

% figure support
\usepackage{import}
\usepackage{xifthen}
\pdfminorversion=7
\usepackage{pdfpages}
\usepackage{transparent}
\newcommand{\incfig}[1]{%
	\def\svgwidth{\columnwidth}
	\import{./figures/}{#1.pdf_tex}
}

\pdfsuppresswarningpagegroup=1
\ctikzset{tripoles/european not symbol=circle}

\newcommand{\missingpart}{{\large\color{red} Il manque quelque chose ici\ldots}}

\renewcommand{\baselinestretch}{1.12}
\usepackage{comment}
\usepackage{slantsc}
\usepackage{montserrat}
%\usepackage{concmath}

\fancyhead{} \fancyfoot{}
\fancyfoot[C]{{\sffamily\itshape--\:\thepage\:--}}
\fancyhead[R]{DM\textsubscript5 Physique}
\fancyhead[L]{Hugo \textsc{Salou} \& Noémie \textsc{Combey}, \textit{MPI}$^\star$}

\def\thefigure{\alph{figure}}

\makeatletter
\let\@ref\ref
\def\ref#1{\textsc{\@ref{#1}}}
\makeatother

%\usepackage{bera}
%\usepackage{tgpagella}
%\usepackage{quattrocento}
%\usepackage[t,lf]{spectral}
%\usepackage{tgschola}
%\usepackage{bookman}
%\usepackage[l,nf]{coelacanth}
%\renewcommand{\sfdefault}{cmr}
\begin{comment}
\let\gguillemotleft\guillemotleft
\let\gguillemotright\guillemotright
\def\guillemotleft{\textsf{\gguillemotleft}}
\def\guillemotright{\textsf{\gguillemotright}}
\end{comment}

%\begin{comment}
\pagecolor{truewhite}
\definecolor{black}{HTML}{000000}
\definecolor{white}{HTML}{ffffff}
\color{black}
%\end{comment}

\def\thesection{\Roman{section}.}
\def\thesubsection{\Alph{subsection}.}

\begin{asydef}
	white = rgb("ffffff");
	black = rgb("000000");
	defaultpen(black + fontsize(8pt));
\end{asydef}

\begin{comment}
\makeatletter
\let\@vec\vec
\let\@vv\vv
\renewcommand{\vec}[1]{\ensuremath{\@vec{\mathbf{#1}}}}
\renewcommand{\vv}[1]{\ensuremath{\@vv{\mathbf{#1}}}}
\makeatother
\end{comment}

\def\res#1{{\color{cyan}#1}}

\begin{document}
	\begin{center}
		\LARGE\scshape ---\quad Problème n\textsuperscript o\,1\quad--- \\
		\itshape Traitement des poussières par électrofiltre
	\end{center}

	\begin{multicols}{2}
		\section{Champ électrique dans un électrofiltre}
		\subsection{Champ électrique à vide et tension de seuil}
		\begin{enumerate}
			\item D'après l'équation de \textsc{Poisson}, pour tout point $\mathrm{M}$, \[
					\mathrm{\Delta}V(\mathrm{M}) = - \frac{\rho(\mathrm{M})}{\varepsilon_0}
				.\]Mais, comme l'espace inter-électrode est supposé vide de charge, $\rho(\mathrm{M}) = 0$\/ et l'équation devient donc \[
					\boxed{\mathrm{\Delta}V(\mathrm{M}) = 0.}
				\]
			\item
				\begin{enumerate}
					\item On se place dans le repère cylindrique $(\mathrm{O}, \vec{e}_r, \vec{e}_\theta, \vec{e}_z)$.
						Le système est invariant par révolution (angle $\theta$) autour de l'axe $(\mathrm{O},\vec{e}_z)$, et par translation d'axe $(\mathrm{O},\vec{e}_z)$.
						\begin{figure}[H]
							\centering
							\begin{asy}
								import solids;
								size(3cm);

								revolution r=cylinder(O,1,4,Z);
								draw(r,0,longitudinalpen=nullpen);
								draw(r.silhouette());
								triple M = (0, 1, 1.5);
								triple R = (0, 1, 0);
								triple T = (-1, 0, 0);
								draw(M--M+R, red, Arrow3(TeXHead2));
								draw(M--M+T, red, Arrow3(TeXHead2));
								draw(M--M+Z, red, Arrow3(TeXHead2));
								dot("$\mathrm{M}$", M, align=SW);
								label("$\vec e_r$", M+1.2R, red);
								label("$\vec e_\theta$", M+1.4T, red);
								label("$\vec e_z$", M+1.2Z, red);
								draw(O--4Z, dotted);
								draw(1.5Z--M);

								dot("$\mathrm{O}$", 3Z);
							\end{asy}
							\caption{Invariances du système}
							\label{fig:invariances-elhum}
						\end{figure}
						On en déduit que \[
							\boxed{V(\mathrm{M}) = V(r).}
						\]
						De plus, d'après le calcul du \guillemotleft~laplacien~\guillemotright\ en coordonnées cylindriques, on a \[
							\mathrm{\Delta}(V) = \frac{1}{r} \frac{\partial}{\partial r} \left( r \frac{\partial V}{\partial r} \right) 
						,\] qui est nul car la zone inter-électrode est vide de charge.
						Ainsi, comme la dérivée $\frac{\partial}{\partial r} \left( r \frac{\partial V}{\partial r} \right)$\/ est nulle, \[
							r \frac{\partial V}{\partial r} = A
						,\] où $A$\/ est une constante. On a donc $\frac{\partial V}{\partial r} = \frac{A}{r}$, que l'on peut intégrer en \[
							V(r) = A \ln r + B
						,\] où $B$\/ est aussi une constante. Déterminons les valeurs de ces constantes à l'aide des conditions limites : on a $V(r_\mathrm{c}) = 0$\/ et $V(r_\mathrm{e}) = -U$\/ par hypothèse. Ainsi, d'après la première condition, on trouve \[
							B = - A \ln r_\mathrm{c}
						\] d'où, d'après la seconde condition, \[
							A = \frac{U}{\ln(r_\mathrm{c} / r_\mathrm{e})}
						.\] On en déduit donc l'expression du potentiel \[
							\boxed{V(r) = U\frac{\ln(r/r_\mathrm{c})}{\ln(r_\mathrm{c} / r_\mathrm{e})}.}
						\]
					\item On sait que $\vec{E}(\mathrm{M}) = -\grad V(\mathrm{M})$. D'où, avec l'expression trouvée à la question précédente, on trouve, au contact avec l'émettrice \[
							\boxed{\vec{E}(r_\mathrm{e}) = \frac{U}{\ln(r_\mathrm{c} / r_\mathrm{e})} \cdot \frac{1}{r_\mathrm{e}}\: \vec{e}_r.}
						\]
						Après résolution, en $U$, de l'équation $E(r_\mathrm{e}) = E_0$, on trouve \[
							\boxed{U_0 = E_0\: r_\mathrm{e}\:\ln(r_\mathrm{c} / r_\mathrm{e}).}
						\]
					\item On réalise l'application numérique, et on trouve \res{$U_0 = 2{,}63 \times 10^4\:\mathrm{V}$}.
				\end{enumerate}
			\item
				\begin{enumerate}
					\item On remarque que l'expression de $V(\mathrm{M})$\/ est invariante par symétrie en $y$\/ : $V(x, y, z) = V(x, -y, z)$. Et, d'après la géométrie de l'électrofiltre sec, le potentiel sur les collectrices est nul, ce qui est vrai avec l'expression de $V(\mathrm{M})$\/ :
						\begin{align*}
							V(x, s, z) &= \frac{U}{\Lambda} \sum_{m \in \Z} \ln\!\!\left( \frac{\ch\left( \frac{\pi(x-2md)}{2s} \right)}{\ch\left( \frac{\pi(x-2md)}{2s} \right)} \right)\\
							&= \frac{U}{\Lambda} \sum_{m \in \Z} \ln 1 \\
							&= 0 \\
						\end{align*}
						De plus, l'expression du potentiel $V(\mathrm{M})$\/ est invariante par translation d'axe $(\mathrm{O}z)$, ce qui est compatible avec les hypothèses de l'énoncé : les effets de bords sont négligés, les cylindres sont \textit{infiniment} longs.

						Pour déterminer la valeur de $\Lambda$, on utilise la condition limite $V(r_\mathrm{e}, 0, z) = -U$\/ au niveau de l'émettrice. On pose $\xi = \frac{\pi}{2s}(r_\mathrm{e} - 2md)$, et ainsi, \[
							-U = V(r_\mathrm{e}, 0, z) = \frac{U}{\Lambda} \sum_{m \in \Z} \ln\left( \frac{\ch \xi - 1}{\ch \xi + 1} \right).
						\] On en déduit donc que \[
							\boxed{\Lambda = - \sum_{m \in \Z} \ln\left( \frac{\ch \xi - 1}{\ch \xi + 1} \right).}
						\]
				\end{enumerate}
		\end{enumerate}
	\end{multicols}
	\begin{figure}[H]
		\centering
		\incfig{lignes-de-champ}
		\caption{Lignes de champ de $\vec{E}(\mathrm{M})$\/ pour l'électrofiltre sec}
		\label{fig:lignes-champ}
	\end{figure}
	\begin{multicols}{2}
		\begin{enumerate}[start=4]
			\item[]
				\begin{enumerate}[start=2]
					\item
						On représente les lignes de champ de $\vec{E}(\mathrm{M})$\/ en violet sur la figure \ref{fig:lignes-champ}.
						Les zones de fort champ $E(\mathrm{M})$\/ sont lorsque les lignes de champ se resserrent. Le champ est donc plus fort au niveau des émettrices (représentées en bleu sur la figure).
						Le champ $\vec{E}(\mathrm{M})$\/ s'annule à équidistance de deux émettrices, car les champs individuels de chaque émettrice se compensent. Ces points sont représentés en rouge sur la figure.
					\item Près de l'électrode émettrice, on mesure $|E_y| = 23{,}5\cdot (U/s)$\/ d'après la figure 2 du sujet. Or, la ligne de champ passant par le point $\mathrm{M}(0, r_\mathrm{e}, 0)$\/ est parallèle à l'axe $(\mathrm{O}y)$, d'où $|E_y| = E_0$.
						Ainsi, on a \[
							\boxed{U_0 = \frac{E_0}{23{,}5} s.}
						\]
						Après application numérique, on trouve $\res{U_0 = 2{,}81 \times 10^4\:\mathrm{V}}$. Les valeurs de $U_0$\/ trouvées pour l'électrofiltre sec et humide ont le même ordre de grandeur : $U_0 \sim 10^4\:\mathrm{V}$.
				\end{enumerate}
		\end{enumerate}
		\subsection{Influence des charges d'espace}
		\begin{enumerate}
			\item On sait que, dans les conducteurs ohmiques, $\vec{\jmath}(\mathrm{M}) = \rho(\mathrm{M}) \cdot \vec{v}(\mathrm{M})$. Or, d'après l'énoncé, $\vec{v}(\mathrm{M}) = - b\:\vec{E}(\mathrm{M})$. Et, comme l'espace inter-électrode est \guillemotleft~peuplé~\guillemotright\ d'anions, de charge négative, $\rho(\mathrm{M}) < 0$. Ainsi, $\vec{\jmath}(\mathrm{M})$\/ est colinéaire et va dans le même sens que $\vec{E}(\mathrm{M})$. D'après la figure \textsc{a}, le courant $i$\/ va donc des collectrices aux émettrices. Comme $\vec{\jmath}(\mathrm{M})$\/ et $\vec{E}(\mathrm{M})$\/ sont colinéaires, on en déduit que \[
					\boxed{j(\mathrm{M}) = - b\:\rho(\mathrm{M})\:E(\mathrm{M}).}
				\]
			\item On intègre l'expression trouvée à la question précédente sur une surface $\mathcal{S}$\/ cylindrique de rayon $r$\/ et de hauteur $h$\/ :
				\begin{align*}
					i &= \iint\limits_\mathcal{S} \vec{\jmath}\cdot \mathrm{d}\vec{S}_\text{sortant}\\
					&= \iint\limits_\mathcal{S} j(\mathrm{M})\: \vec{e}_r \cdot \mathrm{d}S\: (-\vec{e}_r) \\
					&= \iint\limits_\mathcal{S} -j(\mathrm{M})~\mathrm{d}S \\
					&= \iint\limits_\mathcal{S} b\:\rho(\mathrm{M})\:E(\mathrm{M})~\mathrm{d}S \\
					&= b\:\rho\:E \iint\limits_\mathcal{S}~\mathrm{d}S\\
					&= b\,\rho\,E \cdot 2\pi\,r\,h. \\
				\end{align*}
				Ainsi, on en déduit que \[
					\boxed{\rho = \frac{i}{2\pi rh b E}.}
				\]
			\item D'après l'équation de \textsc{Maxwell-Gauss}, on a \[
					\boxed{\div \vec{E}(\mathrm{M}) = \frac{\rho(\mathrm{M})}{\varepsilon_0},}
				\] qui exprime localement la modification du champ électrique $\vec{E}(\mathrm{M})$\/ par les charges (\textit{i.e.} les ions).
				On utilise l'expression de $\rho$\/ trouvée à la question précédente, et on a donc \[
					\div \vec{E}(\mathrm{M}) = \frac{i}{2\pi\varepsilon_0 r h b E}
				.\] Mais, en coordonnées cylindriques, la divergence du champ $\vec{E}$\/ est donnée par \[
					\div \vec{E}(r, \theta, z) = \frac{1}{r} \frac{\partial\: (r\,E_r)}{\partial r} + \frac{1}{r} \frac{\partial E_\theta}{\partial \theta} + \frac{\partial E_z}{\partial z}
				.\]
				Par identification avec les deux formules, et comme $\vec{E}$\/ est radial, on trouve \[
					\frac{1}{r} \frac{\partial\, (rE)}{\partial r} = \frac{i}{2\pi\varepsilon_0 h b E}
				,\] d'où l'équation demandée : \[
				\boxed{rE\cdot \frac{\mathrm{d}(rE)}{\mathrm{d}r} = \frac{ri}{2\pi \varepsilon_0 hb}.}
				\]
			\item Avec le changement de variable $u = r E$, on intègre entre $r$\/ et $r_0$\/ pour trouver \[
					\int_{r}^{r_0} u \frac{\mathrm{d}u}{\mathrm{d}r}~\mathrm{d}r = \int_{r}^{r_0} \frac{ri}{2\pi\varepsilon_0 hb}~\mathrm{d}r
				\] d'où \[
				\left[ \frac{u^2}{2} \right]_r^{r_0} = \frac{i}{2\pi\varepsilon_0 hb} \times \left[ \frac{r^2}{2} \right]_r^{r_0},
				\] et donc \[
				\big(r_0{}^2 E_0{}^2 - r^2\:E^2(r)\big) = \frac{i}{2\pi\,\varepsilon_0\, h\, b}\big(r_0{}^2 - r^2)
				.\]
				Après simplification, on trouve \[
					E^2(r) = \frac{1}{r^2} \left( r_0{}^2 - E_0{}^2 - \frac{i(r_0{}^2 - r^2)}{2\pi\, \varepsilon_0\, h\,b} \right)
				.\] On en déduit donc que \[
					\boxed{E(r) = \sqrt{\left(\frac{r_0}{r}\: E_0\right)^2 - \frac{i}{2\pi\,\varepsilon_0\, h\,b} \left( \frac{r_0{}^2}{r^2} - 1 \right) }.}
				\] 
			\item Si $E$\/ devient grossièrement uniforme, alors \[
					rE\frac{\mathrm{d}(rE)}{\mathrm{d}r} = rE \times E \frac{\mathrm{d}r}{\mathrm{d}r} = r\:E^2
				,\] et donc, avec l'expression de la question 3, on trouve alors \[
					r\:E^2 = \frac{r\:i}{2\pi\, \varepsilon_0\, h\, b}
				.\] On conclut donc que, avec cette approximation, \[
					\boxed{E = \sqrt{\frac{i}{2\pi\,\varepsilon_0\, h\,b}}.}
				\] Après application numérique, on trouve \res{$E = 0{,}20\:\mathrm{MJ}$}.
			\item On a $v = \|\vec{v}\| = b\, E = \res{62\:\mathrm{m}/\mathrm{s}}$.
				Et, d'après la question 2, on a \[
					\rho(r_\mathrm{e}) = -\frac{i}{2\pi\, r_\mathrm{e} \,h\,b\,E} = \res{-1{,}4 \times10^{-5}\:\mathrm{C}/\mathrm{m}^3}
				.\] Le nombre d'ions par centimètre cube est
				\begin{align*}
					\frac{\rho(r_\mathrm{e})}{-e} &= \res{89 \times 10^{12}\:\mathrm{ions}/\mathrm{m}^3}\\
					&= \res{89 \times 10^6\:\mathrm{ions}/\mathrm{cm}^3}. \\
				\end{align*}
		\end{enumerate}
		\section{Comportment des poussières dans l'électrofiltre}
		\subsection{Charge d'une particule sphérique : modèle de \textsc{Pauthenier}}
		\begin{enumerate}
			\item
				\begin{enumerate}
					\item~
						\begin{figure}[H]
							\centering
							\def\svgwidth{0.5\textwidth}
							\incfig{particules}
							\caption{Lignes de champ autour d’un grain de poussière pour $Q = 0$ (en haut) et pour $Q < 0$ (en bas). Les anions sont représentés en rouge, le champ $\vec{E}$\/ est représenté en bleu.}
							\label{fig:particule}
						\end{figure}
					\item Le champ $\vec{E}_1(\mathrm{M})$\/ correspond à celui créé par une boule chargée de rayon $a$\/ et de charge uniforme $Q$.
						Analysons les symétries et invariances : les plans $\Pi_1(\mathrm{M}, \vec{u}_r, \vec{u}_\theta)$\/ et $\Pi_2(\mathrm{M}, \vec{u}_r, \vec{u}_z)$\/ sont de symétrie des charges, donc du champ $\vec{E}_1$. De plus, le système est invariant par révolution autour de l'axe $(\mathrm{O}, \vec{u}_r)$\/ et l'axe $(\mathrm{O}, \vec{u}_z)$. On en déduit \[
							\vec{E}_1(\mathrm{M}) = E_1(r)\:\vec{u}_r
						.\]
						On applique le théorème de \textsc{Gauss} sur une sphère $\mathcal{S}$ de rayon $r \ge a$ : \[
							\oiint\limits_\mathcal{S} \vec{E}_1(\mathrm{M}) \cdot \mathrm{d}\vec{S}_\mathrm{sortant} = \frac{Q_\mathrm{int}}{\varepsilon_0} = \frac{Q}{\varepsilon_0}
						.\]
						Or, on a
						\begin{align*}
							\oiint\limits_\mathcal{S} \vec{E}_1(\mathrm{M})\cdot \mathrm{d}\vec{S}_\mathrm{sortant}
							&= \oiint\limits_\mathcal{S} E_1(r)\vec{u}_r \cdot \mathrm{d}S\:\vec{u}_r \\
							&= \oiint\limits_\mathcal{S} E_1(r)~\mathrm{d}S \\
							&= E_1(r) \cdot \oiint\limits_\mathcal{S} \mathrm{d}S \\
							&= E_1(r) \cdot 4\pi r^2. \\
						\end{align*}
						Ainsi, après simplification, on en déduit que \[
							\boxed{\vec{E}_1(r) = \frac{Q}{4\pi r^2 \: \varepsilon_0} \vec{u}_r.}
						\]
						Sur la figure \ref{fig:particule}, on représente en orange la portion de la sphère d'où partent, vers des valeurs croissantes de $r$, les lignes de champ de $\vec{E}(\mathrm{M})$.
						Ainsi, cette portion est plus importante pour $|Q| > 0$ (haut) que pour $Q = 0$ (bas).
						L'accroissement de $|Q|$ a donc tendance à élargir cette portion de sphère.
						Cet accroissement s'oppose ainsi à l'arrivée de nouveaux anions, de charge négative, sur la sphère.
					\item Les lignes de champ de $\vec{E}(\mathrm{M})$\/ sont si distordues qu'aucun anion n'arrive au grain de poussière lorsque \smash{$\vec{E}_\mathrm{t}$} n'a plus de composante radiale, ou elle est strictement négative. Autrement dit, on cherche la valeur maximale de la charge $Q$\/ telle que \smash{$\vec{E}_\mathrm{t}(\mathrm{M}) \cdot \vec{u}_r > 0$}, reste vrai pour certains points $\mathrm{M}$.
						Au vu de la figure \ref{fig:particule}, le dernier point $\mathrm{M}$\/ vérifiant cette condition se situe à $\theta = \pi$\/ et à $r = a$.
						Avec cette condition, on a $\vec{E}(\mathrm{M}) = E \vec{u}_z = -E \vec{u}_r$, et donc {\footnotesize\[
							\vec{E}_\mathrm{t} \cdot \vec{u}_r = -E + 2E \cos \theta \frac{\varepsilon_\mathrm{r} - 1}{\varepsilon_\mathrm{r} + 2}\: \cancel{\frac{a^3}{a^3}} + \frac{Q_\mathrm{lim}}{4\pi \varepsilon_0 a^2}
						\]} qui est nul. D'où, \[
							\boxed{Q_\mathrm{lim} = 4\pi \varepsilon_0 a^2 E \cdot \left( 1 + 2\frac{\varepsilon_\mathrm{r} - 1}{\varepsilon_\mathrm{r} + 2} \right) .}
						\]
					\item On réalise l'application numérique, et on trouve {\footnotesize\[
							\res{Q_\mathrm{lim} = -1{,}39 \times 10^{-16}\:\mathrm{C} = 869 \times (-e)}
						.\]}
				\end{enumerate}
			\item
				\begin{enumerate}
					\item La dimension de $v = bE$\/ est $[v] = [b]\cdot [E] = \textsf{L}\cdot \textsf{T}^{-1}$.
						Et, d'après le théorème de \textsc{Gauss}, on a $\oiint \vec{E} \cdot \mathrm{d}\vec{S} = Q / \varepsilon_0$, d'où {\small\[
							[\varepsilon_0] = \frac{[Q]}{[E] \cdot \textsf{L}^2} = \frac{[Q]\cdot [b]}{[v] \cdot \textsf{L}^2} = \frac{\textsf{T}\cdot [Q] \cdot [b]}{\textsf{L}^3}.
						\]}Également, $[\rho] = [Q] \cdot \textsf{L}^{-3}$.
						Par identification, on trouve $[\varepsilon_0] = [\rho] \cdot [b] \cdot \textsf{T}$.
						Comme la dimension de $\tau_Q$\/ est $[\tau_Q] = \textsf{T}$, on en déduit que \[
							\boxed{\tau_Q = 4\frac{\varepsilon_0}{|\rho| \cdot b}.}
						\]
					\item On cherche une valeur de $t$\/ telle que $t/(t + \tau_Q) = 90\:\%$.
						D'où, \[
							t_{90}(0{,}90 - 1) = -0{,}90\: \tau_Q
						.\] On en déduit \[
							t_{90} = \frac{0{,}90\: \tau_Q}{1 - 0{,}90}
						.\] On trouve, par application numérique, $\tau_Q = 2{,}3\:\mathrm{ms}$, d'où $\res{t_{90} = 21\:\mathrm{ms}}$.
					\item Le temps nécessaire théorique pour qu'un grain de poussière traverse l'électrofiltre serait de $t = 10\:\mathrm{s}$. Or, d'après la question précédente, le temps nécessaire pour que ce grain de poussière soit chargé à 90\:\% de sa charge maximale est de $21\:\mathrm{ms} \ll 10\,\:\mathrm{s}$. Ainsi, les grains de poussières sont, presque instantanément chargés.
				\end{enumerate}
		\end{enumerate}
	\end{multicols}
	\vfill
	\begin{center}
		{
			\fontfamily{ccr}\selectfont
			\textit{\textbf{\.{\"i}}}
		}
	\end{center}
	\vfill
	\null
\end{document}
