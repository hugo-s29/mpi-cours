\section{Distance de visibilité d'une bougie}

Pour trouver la distance maximale de visibilité de la bougie, on analyse l'inégalité \[
	\mathcal{E}_\text{reçue} > 10 \cdot \mathcal{E}_\mathrm{photon}
.\] 

L'énergie par un photon est donnée par $\mathcal{E}_\mathrm{photon} = h \nu$.
La puissance de la bougie est $P = 0{,}1\:\mathrm{W}$, et le temps de réaction est $t_\mathrm{r}  = 0{,}05\:\mathrm{s}$.
L'énergie produite par la bouge est donc $\mathcal{E}_\mathrm{bougie} = P \times t_\mathrm{r}$.

En notant $d_\text{œuil} = 5\:\mathrm{mm}$, on a \[
	\mathcal{E}_\text{reçue} = \frac{\pi \cdot ( d_ \text{œuil} / 2 ) }{4\pi d^2}
.\]
Ainsi, en reprenant l'inégalité précédente, elle est équivalente à \[
	\frac{(d_ \text{œuil} / 2)^2}{4 d^2} \mathcal{E}_\mathrm{bougie} > 10 h \frac{c}{\lambda}
\] \textit{i.e.} \[
	\sqrt{\frac{P \cdot t_\mathrm{r}}{40 h \frac{c}{\lambda}}}  \cdot \left( \frac{d_\text{œuil}}{2} \right) > d
.\]

On choisit $\lambda = 600\:\mathrm{nm}$, on a donc $d < 48\:\mathrm{km}$.

\begin{figure}[H]
	\centering
	\includesvg[width=0.8\linewidth]{figures/bougie.svg}
	\caption{Modèle bougie-œuil}
\end{figure}
