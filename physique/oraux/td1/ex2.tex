\section{Crampes musculaires}

\begin{enumerate}
	\item La concentration en $\mathrm{CO_2}$ provient de la combustion du di-oxygène dans les cellules (respiration cellulaire).

		On calcule le $\mathrm{pH}$ : \[
			\mathrm{pH} = \mathrm{p}K_\mathrm{a} + \log\left( \frac{[\mathrm{HCO_3^{-}}]}{[\mathrm{CO_2}]} \right)
		.\] On trouve $\mathrm{pH} \simeq 6{,}4 + 1 = 7{,}4$.

		Pour le couple $\mathrm{CO_2}/\mathrm{CO_3^{2-}}$, on a $\mathrm{pH} - \mathrm{p}K_\mathrm{a} + \log [\mathrm{HCO_3^-}] = \log [\mathrm{CO_3^{2-}}]$.

		\underline{AN.} $[\mathrm{CO_3^{2-}}] = 2{,}7\cdot 10^{-5}\:\mathrm{mol}$.

		Autre possibilité : réaliser un diagramme de prédominance.
	\item D'après la question précédente, on a montré que la concentration en $\mathrm{CO_3^{2-}}$ est négligeable. De plus, la réaction est favorisée. Ainsi, c'est bien la réaction 1 qui est majoritaire.

		On a
		\begin{align*}
			K^\circ &= \frac{[\mathrm{H_2CO_3}] \cdot [A^-]}{[\mathrm{H}A] \cdot [\mathrm{HCO_3^-}]}\\
							&= \frac{[\mathrm{H_2CO_3}] \cdot [A^-] \cdot [\mathrm{H_3O^+}]}{\mathrm{[HCO_3^-]} \cdot [\mathrm{H}A] \cdot \mathrm{[H_3O^+]}} \\
							&= K_\mathrm{A3} / K_\mathrm{A 1}  \\
		\end{align*}

		\underline{AN.} $K^\circ = 350 \gg 1$.
	\item Lors de l'effort, le $\mathrm{pH}$\/ diminue, puis augmente une fois l'effort terminé.
		
		Hypothèse : épuisement d'un des réactifs, en particulier, l'acide lactique.
		On considère la réaction \[
			\mathrm{H}A = \mathrm{HCO_3^-} = \mathrm{H_2CO_3} + A^-
		.\]
		En supposant la réaction quasi-totale, on a
		\begin{align*}
			n_\mathrm{f}(\mathrm{H}A) \simeq 0 && n_\mathrm{f}(\mathrm{HCO_3^-}) \simeq 19\\
			n_\mathrm{f}(\mathrm{H_2CO_3}) \simeq 5{,}2 && n_\mathrm{f}(A^-) \simeq 3
		\end{align*}
		Ainsi, on calcule le $\mathrm{pH}$ : \[
			\mathrm{pH} = \mathrm{p}K_\mathrm{A 1} + \log(19 / 5{,}2) \simeq 7{,}0
		.\]

	\item On réalise une prise de sang après l'effort, et on réalise un dosage de la base \guillemotleft~lactate~\guillemotright\ par un acide fort.
\end{enumerate}
