\section{Exercice 10}

\begin{enumerate}
	\item On a $u_6 \leadsto \hat{v}_\mathrm{e}$, $u_3 \leadsto \hat{v}_\mathrm{c}$\/ (composante continue), $u_2 \leadsto \hat{v}_\mathrm{d}$\/ (discontinuités), $u_4 \leadsto \hat{v}_\mathrm{a}$\/ (composante continue), $u_5 \leadsto \hat{v}_\mathrm{f}$\/ (nombre de fréquences) et $u_1 \leadsto \hat{v}_\mathrm{b}$.
	\item
		Le filtre donnant le signal $\hat{v}_\mathrm{g}$\/ est un passe-bandes (les hautes et basses fréquences sont éliminées) et c'est un filtre non linéaire (de nouvelles fréquences apparaissent).

		Le filtre donnant le signal $\hat{v}_\mathrm{h}$\/ est un filtre passe-bas.

		Le filtre donnant le signal $\hat{v}_\mathrm{i}$\/ est un filtre passe-haut dont sa fréquence de coupure est inférieure à $1\:\mathrm{kHz}$.
\end{enumerate}

