\section{Exercice 19}

\begin{enumerate}
	\item On a $|\ubar{H}| = G = \left|\frac{1 - \mathrm{j}RC\omega}{1 + \mathrm{j}RC\omega}\right| = 1$, et $\varphi = 2\Arg(1 + \mathrm{j}RC\omega) = -2\Arctan(RC\omega)$. Or, on sait que $\varphi\left( \omega = \frac{1}{RC} \right) = -\frac{\pi}{2}$\/ donc, par lecture graphique, on a $\frac{1}{RC} \circeq 4 \times 10^{4} \:\mathrm{\frac{rad}{s}}$\/ d'où $RC \circeq 2{,}5\times 10^{-5}\:\mathrm{s}$.
	\item Par exemple, en traitement du son, on utilise une ligne de retard pour resynchroniser les signaux.
		Mais aussi, en tant que détecteur de front ascendant et descendant.
	\item Si $\omega \ll \frac{1}{RC}$, le signal est inchangé. Si $\omega \gg \frac{1}{RC}$, le signal est en opposition de phases. Si $\omega \cong \frac{1}{RC}$, le signal est non-carré.
\end{enumerate}

