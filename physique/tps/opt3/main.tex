\documentclass[a4paper,twocolumn,10pt,margin=0.5in]{extreport}

%\usepackage{concmath}
\usepackage{tgschola}
\usepackage[margin=1in]{geometry}
\usepackage[utf8]{inputenc}
\usepackage[T1]{fontenc}
\usepackage{mathrsfs}
\usepackage{textcomp}
\usepackage[french]{babel}
\usepackage{amsmath}
\usepackage{amssymb}
\usepackage{cancel}
\usepackage{frcursive}
\usepackage[inline]{asymptote}
\usepackage{tikz}
\usepackage[european,straightvoltages,europeanresistors]{circuitikz}
\usepackage{tikz-cd}
\usepackage{tkz-tab}
\usepackage[b]{esvect}
\usepackage[framemethod=TikZ]{mdframed}
\usepackage{centernot}
\usepackage{diagbox}
\usepackage{dsfont}
\usepackage{fancyhdr}
\usepackage{float}
\usepackage{graphicx}
\usepackage{listings}
\usepackage{multicol}
\usepackage{nicematrix}
\usepackage{pdflscape}
\usepackage{stmaryrd}
\usepackage{xfrac}
\usepackage{hep-math-font}
\usepackage{amsthm}
\usepackage{thmtools}
\usepackage{indentfirst}
\usepackage[framemethod=TikZ]{mdframed}
\usepackage{accents}
\usepackage{soulutf8}
\usepackage{mathtools}
\usepackage{bodegraph}
\usepackage{slashbox}
\usepackage{enumitem}
\usepackage{calligra}
\usepackage{cinzel}
\usepackage{BOONDOX-calo}

% Tikz
\usetikzlibrary{babel}
\usetikzlibrary{positioning}
\usetikzlibrary{calc}

% global settings
\frenchspacing
\reversemarginpar
\setuldepth{a}

%\everymath{\displaystyle}

\frenchbsetup{StandardLists=true}

\def\asydir{asy}

%\sisetup{exponent-product=\cdot,output-decimal-marker={,},separate-uncertainty,range-phrase=\;à\;,locale=FR}

\setlength{\parskip}{1em}

\theoremstyle{definition}

% Changing math
\let\emptyset\varnothing
\let\ge\geqslant
\let\le\leqslant
\let\preceq\preccurlyeq
\let\succeq\succcurlyeq
\let\ds\displaystyle
\let\ts\textstyle

\newcommand{\C}{\mathds{C}}
\newcommand{\R}{\mathds{R}}
\newcommand{\Z}{\mathds{Z}}
\newcommand{\N}{\mathds{N}}
\newcommand{\Q}{\mathds{Q}}

\renewcommand{\O}{\emptyset}

\newcommand\ubar[1]{\underaccent{\bar}{#1}}

\renewcommand\Re{\expandafter\mathfrak{Re}}
\renewcommand\Im{\expandafter\mathfrak{Im}}

\let\slantedpartial\partial
\DeclareRobustCommand{\partial}{\text{\rotatebox[origin=t]{20}{\scalebox{0.95}[1]{$\slantedpartial$}}}\hspace{-1pt}}

% merging two maths characters w/ \charfusion
\makeatletter
\def\moverlay{\mathpalette\mov@rlay}
\def\mov@rlay#1#2{\leavevmode\vtop{%
   \baselineskip\z@skip \lineskiplimit-\maxdimen
   \ialign{\hfil$\m@th#1##$\hfil\cr#2\crcr}}}
\newcommand{\charfusion}[3][\mathord]{
    #1{\ifx#1\mathop\vphantom{#2}\fi
        \mathpalette\mov@rlay{#2\cr#3}
      }
    \ifx#1\mathop\expandafter\displaylimits\fi}
\makeatother

% custom math commands
\newcommand{\T}{{\!\!\,\top}}
\newcommand{\avrt}[1]{\rotatebox{-90}{$#1$}}
\newcommand{\bigcupdot}{\charfusion[\mathop]{\bigcup}{\cdot}}
\newcommand{\cupdot}{\charfusion[\mathbin]{\cup}{\cdot}}
%\newcommand{\danger}{{\large\fontencoding{U}\fontfamily{futs}\selectfont\char 66\relax}\;}
\newcommand{\tendsto}[1]{\xrightarrow[#1]{}}
\newcommand{\vrt}[1]{\rotatebox{90}{$#1$}}
\newcommand{\tsup}[1]{\textsuperscript{\underline{#1}}}
\newcommand{\tsub}[1]{\textsubscript{#1}}

\renewcommand{\mod}[1]{~\left[ #1 \right]}
\renewcommand{\t}{{}^t\!}
\newcommand{\s}{\text{\calligra s}}

% custom units / constants
%\DeclareSIUnit{\litre}{\ell}
\let\hbar\hslash

% header / footer
\pagestyle{fancy}
\fancyhead{} \fancyfoot{}
\fancyfoot[C]{\thepage}

% fonts
\let\sc\scshape
\let\bf\bfseries
\let\it\itshape
\let\sl\slshape

% custom math operators
\let\th\relax
\let\det\relax
\DeclareMathOperator*{\codim}{codim}
\DeclareMathOperator*{\dom}{dom}
\DeclareMathOperator*{\gO}{O}
\DeclareMathOperator*{\po}{\text{\cursive o}}
\DeclareMathOperator*{\sgn}{sgn}
\DeclareMathOperator*{\simi}{\sim}
\DeclareMathOperator{\Arccos}{Arccos}
\DeclareMathOperator{\Arcsin}{Arcsin}
\DeclareMathOperator{\Arctan}{Arctan}
\DeclareMathOperator{\Argsh}{Argsh}
\DeclareMathOperator{\Arg}{Arg}
\DeclareMathOperator{\Aut}{Aut}
\DeclareMathOperator{\Card}{Card}
\DeclareMathOperator{\Cl}{\mathcal{C}\!\ell}
\DeclareMathOperator{\Cov}{Cov}
\DeclareMathOperator{\Ker}{Ker}
\DeclareMathOperator{\Mat}{Mat}
\DeclareMathOperator{\PGCD}{PGCD}
\DeclareMathOperator{\PPCM}{PPCM}
\DeclareMathOperator{\Supp}{Supp}
\DeclareMathOperator{\Vect}{Vect}
\DeclareMathOperator{\argmax}{argmax}
\DeclareMathOperator{\argmin}{argmin}
\DeclareMathOperator{\ch}{ch}
\DeclareMathOperator{\com}{com}
\DeclareMathOperator{\cotan}{cotan}
\DeclareMathOperator{\det}{det}
\DeclareMathOperator{\id}{id}
\DeclareMathOperator{\rg}{rg}
\DeclareMathOperator{\rk}{rk}
\DeclareMathOperator{\sh}{sh}
\DeclareMathOperator{\th}{th}
\DeclareMathOperator{\tr}{tr}

% colors and page style
\definecolor{truewhite}{HTML}{ffffff}
\definecolor{white}{HTML}{faf4ed}
\definecolor{trueblack}{HTML}{000000}
\definecolor{black}{HTML}{575279}
\definecolor{mauve}{HTML}{907aa9}
\definecolor{blue}{HTML}{286983}
\definecolor{red}{HTML}{d7827e}
\definecolor{yellow}{HTML}{ea9d34}
\definecolor{gray}{HTML}{9893a5}
\definecolor{grey}{HTML}{9893a5}
\definecolor{green}{HTML}{a0d971}

\pagecolor{white}
\color{black}

\begin{asydef}
	settings.prc = false;
	settings.render=0;

	white = rgb("faf4ed");
	black = rgb("575279");
	blue = rgb("286983");
	red = rgb("d7827e");
	yellow = rgb("f6c177");
	orange = rgb("ea9d34");
	gray = rgb("9893a5");
	grey = rgb("9893a5");
	deepcyan = rgb("56949f");
	pink = rgb("b4637a");
	magenta = rgb("eb6f92");
	green = rgb("a0d971");
	purple = rgb("907aa9");

	defaultpen(black + fontsize(8pt));

	import three;
	currentlight = nolight;
\end{asydef}

% theorems, proofs, ...

\mdfsetup{skipabove=1em,skipbelow=1em, innertopmargin=6pt, innerbottommargin=6pt,}

\declaretheoremstyle[
	headfont=\normalfont\itshape,
	numbered=no,
	postheadspace=\newline,
	headpunct={:},
	qed=\qedsymbol]{demstyle}

\declaretheorem[style=demstyle, name=Démonstration]{dem}

\newcommand\veczero{\kern-1.2pt\vec{\kern1.2pt 0}} % \vec{0} looks weird since the `0' isn't italicized

\makeatletter
\renewcommand{\title}[2]{
	\AtBeginDocument{
		\begin{titlepage}
			\begin{center}
				\vspace{10cm}
				{\Large \sc Chapitre #1}\\
				\vspace{1cm}
				{\Huge \calligra #2}\\
				\vfill
				Hugo {\sc Salou} MPI${}^{\star}$\\
				{\small Dernière mise à jour le \@date }
			\end{center}
		\end{titlepage}
	}
}

\newcommand{\titletp}[4]{
	\AtBeginDocument{
		\begin{titlepage}
			\begin{center}
				\vspace{10cm}
				{\Large \sc tp #1}\\
				\vspace{1cm}
				{\Huge \textsc{\textit{#2}}}\\
				\vfill
				{#3}\textit{MPI}${}^{\star}$\\
			\end{center}
		\end{titlepage}
	}
	\fancyfoot{}\fancyhead{}
	\fancyfoot[R]{#4 \textit{MPI}${}^{\star}$}
	\fancyhead[C]{{\sc tp #1} : #2}
	\fancyhead[R]{\thepage}
}

\newcommand{\titletd}[2]{
	\AtBeginDocument{
		\begin{titlepage}
			\begin{center}
				\vspace{10cm}
				{\Large \sc td #1}\\
				\vspace{1cm}
				{\Huge \calligra #2}\\
				\vfill
				Hugo {\sc Salou} MPI${}^{\star}$\\
				{\small Dernière mise à jour le \@date }
			\end{center}
		\end{titlepage}
	}
}
\makeatother

\newcommand{\sign}{
	\null
	\vfill
	\begin{center}
		{
			\fontfamily{ccr}\selectfont
			\textit{\textbf{\.{\"i}}}
		}
	\end{center}
	\vfill
	\null
}

\renewcommand{\thefootnote}{\emph{\alph{footnote}}}

% figure support
\usepackage{import}
\usepackage{xifthen}
\pdfminorversion=7
\usepackage{pdfpages}
\usepackage{transparent}
\newcommand{\incfig}[1]{%
	\def\svgwidth{\columnwidth}
	\import{./figures/}{#1.pdf_tex}
}

\pdfsuppresswarningpagegroup=1
\ctikzset{tripoles/european not symbol=circle}

\newcommand{\missingpart}{{\large\color{red} Il manque quelque chose ici\ldots}}

\usepackage{caption}
\usepackage{subcaption}
\usepackage{comment}
\usepackage{pgfornament}
\usepackage{graphicx}
\usepackage{pgfplots}
\usepackage{multirow}
\usepackage{rotating}

%\begin{comment}

\definecolor{black}{HTML}{000000}
\definecolor{white}{HTML}{ffffff}
\color{black}
\pagecolor{white}

\begin{asydef}
	white = rgb("ffffff");
	black = rgb("000000");
	defaultpen(black + fontsize(8pt));
\end{asydef}
%\end{comment}

\definecolor{BleuClair}{HTML}{286983}
\definecolor{LightBlue}{cmyk}{0.22,0.05,0,0.1}
\definecolor{GrisClair}{HTML}{9893a5}
\def\res#1{{\color{black}#1}}

\titletp{opt\:3}{Interféromètre de Michelson}{\begin{tabular}{c}Hugo \textsc{Salou}\\Noémie \textsc{Combey}\end{tabular}}{Hugo \textsc{Salou} \& Noémie \textsc{Combey}}

\def\thesection{\Roman{section}.}
\def\thesubsection{\Roman{section}.\Alph{subsection}.}

\begin{document}
	L'objectif de ce \textsc{tp} est la mesure d'un spectrogramme, et son analyse, à partir des interférences crées par le Michelson. Plus particulièrement, on s'intéresse à l'effet d'un filtre interférentiel.
	
	\section{Protocole expérimental}

	\begin{figure}[H]
		\centering
		\begin{asy}
			size(5cm);
			draw(arc((0,0), 1, 20, 340));
			label("\begin{tabular}{c}Lumière\\blanche\end{tabular}", (0, -1), align=S);
			draw((3, -1) -- (3, 1), Arrows(TeXHead));
			draw(shift(5.5, 0) * box((-1,-1),(1,1)));
			label("M", (5.5, 0));
			void tri(pair pos, real angle, real r) {
				fill(shift(pos) * rotate(angle) * scale(r) * (expi(2pi/3) -- expi(4pi/3) -- expi(0) -- cycle), pink);
			}
			tri((4.5, 0), 0, 0.5);
			tri((5.5, -1), -90, 0.5);
			fill(circle((0,0), 0.6), yellow);
			for(real t = 0; t < 2pi; t += 2pi/20) {
				pair z = expi(t);
				draw(0.6z -- 0.8z, yellow);
			}

			draw((4.5, -2) -- (6.5, -2), Arrows(TeXHead));

			label("$\mathcal L_{\mathrm p}$", (6.5, -2), align=E);
			label("$\mathcal L_{\mathrm c}$", (3, -1), align=S);
		\end{asy}
		\caption{Configuration initiale du Michelson}
	\end{figure}

	On commence par configurer l'interféromètre de Michelson au contact optique, comme montré dans la figure ci-dessus.
	Le \guillemotleft~M~\guillemotright\ représente l'interféromètre, $\mathcal{L}_\mathrm{c}$ représente le condenseur et $\mathcal{L}_\mathrm{p}$ la lentille de projection.
	On place un filtre interférentiel entre la lumière blanche et le condenseur.
	On ajoute également un récepteur \texttt{Caliens} en sortie du Michelson, après avoir remplacé la lentille de projection par une lentille de focale plus courte ($f = 12{,}5\:\mathrm{cm}$).
	À l'aide d'un moteur, on change l'épaisseur $e$ de la configuration en coin d'air du Michelson. On réalise simultanément l'acquisition de l'intensité $I(e)$ au centre de la figure d'interférence.

	\begin{figure}[H]
		\centering
		\begin{asy}
			import graph;
			size(5cm);
			draw(arc((0,0), 1, 20, 340));
			real f(real x) { return cos(degrees(5*x)) * 0.05; }
			draw(shift(2, 0) * rotate(90) * graph(f, -1, 1));
			label("\begin{tabular}{c}Lumière\\blanche\end{tabular}", (0, -1), align=S);
			draw((3, -1) -- (3, 1), Arrows(TeXHead));
			label("\begin{tabular}{c}Filtre\\interférentiel\end{tabular}", (2, 1), align=N);
			draw(shift(5.5, 0) * box((-1,-1),(1,1)));
			label("M", (5.5, 0));
			void tri(pair pos, real angle, real r) {
				fill(shift(pos) * rotate(angle) * scale(r) * (expi(2pi/3) -- expi(4pi/3) -- expi(0) -- cycle), pink);
			}
			tri((4.5, 0), 0, 0.5);
			tri((5.5, -1), -90, 0.5);
			fill(circle((0,0), 0.6), yellow);
			for(real t = 0; t < 2pi; t += 2pi/20) {
				pair z = expi(t);
				draw(0.6z -- 0.8z, yellow);
			}

			draw((4.5, -2) -- (6.5, -2), Arrows(TeXHead));

			label("$\mathcal L_{\mathrm p}$", (6.5, -2), align=E);
			label("$\mathcal L_{\mathrm c}$", (3, -1), align=S);

			draw(box((4,-3.5),(7,-5.5)));
			label("\begin{tabular}{c}capteur\\\texttt{Caliens}\end{tabular}", (5.5, -4.8));
			tri((5.5, -3.5), -90, 0.5);
		\end{asy}
		\caption{Configuration expérimentale du Michelson}
	\end{figure}

	Les résultats obtenus sont représentés sur la figure ci-dessous. Ils seront analysés dans la section III.

	\begin{figure}[H]
		\centering
		\resizebox{\linewidth}{!}{%% Creator: Matplotlib, PGF backend
%%
%% To include the figure in your LaTeX document, write
%%   \input{<filename>.pgf}
%%
%% Make sure the required packages are loaded in your preamble
%%   \usepackage{pgf}
%%
%% Also ensure that all the required font packages are loaded; for instance,
%% the lmodern package is sometimes necessary when using math font.
%%   \usepackage{lmodern}
%%
%% Figures using additional raster images can only be included by \input if
%% they are in the same directory as the main LaTeX file. For loading figures
%% from other directories you can use the `import` package
%%   \usepackage{import}
%%
%% and then include the figures with
%%   \import{<path to file>}{<filename>.pgf}
%%
%% Matplotlib used the following preamble
%%   
%%   \makeatletter\@ifpackageloaded{underscore}{}{\usepackage[strings]{underscore}}\makeatother
%%
\begingroup%
\makeatletter%
\begin{pgfpicture}%
\pgfpathrectangle{\pgfpointorigin}{\pgfqpoint{6.400000in}{4.800000in}}%
\pgfusepath{use as bounding box, clip}%
\begin{pgfscope}%
\pgfsetbuttcap%
\pgfsetmiterjoin%
\definecolor{currentfill}{rgb}{1.000000,1.000000,1.000000}%
\pgfsetfillcolor{currentfill}%
\pgfsetlinewidth{0.000000pt}%
\definecolor{currentstroke}{rgb}{1.000000,1.000000,1.000000}%
\pgfsetstrokecolor{currentstroke}%
\pgfsetdash{}{0pt}%
\pgfpathmoveto{\pgfqpoint{0.000000in}{0.000000in}}%
\pgfpathlineto{\pgfqpoint{6.400000in}{0.000000in}}%
\pgfpathlineto{\pgfqpoint{6.400000in}{4.800000in}}%
\pgfpathlineto{\pgfqpoint{0.000000in}{4.800000in}}%
\pgfpathlineto{\pgfqpoint{0.000000in}{0.000000in}}%
\pgfpathclose%
\pgfusepath{fill}%
\end{pgfscope}%
\begin{pgfscope}%
\pgfsetbuttcap%
\pgfsetmiterjoin%
\definecolor{currentfill}{rgb}{1.000000,1.000000,1.000000}%
\pgfsetfillcolor{currentfill}%
\pgfsetlinewidth{0.000000pt}%
\definecolor{currentstroke}{rgb}{0.000000,0.000000,0.000000}%
\pgfsetstrokecolor{currentstroke}%
\pgfsetstrokeopacity{0.000000}%
\pgfsetdash{}{0pt}%
\pgfpathmoveto{\pgfqpoint{0.800000in}{0.528000in}}%
\pgfpathlineto{\pgfqpoint{5.760000in}{0.528000in}}%
\pgfpathlineto{\pgfqpoint{5.760000in}{4.224000in}}%
\pgfpathlineto{\pgfqpoint{0.800000in}{4.224000in}}%
\pgfpathlineto{\pgfqpoint{0.800000in}{0.528000in}}%
\pgfpathclose%
\pgfusepath{fill}%
\end{pgfscope}%
\begin{pgfscope}%
\pgfsetbuttcap%
\pgfsetroundjoin%
\definecolor{currentfill}{rgb}{0.000000,0.000000,0.000000}%
\pgfsetfillcolor{currentfill}%
\pgfsetlinewidth{0.803000pt}%
\definecolor{currentstroke}{rgb}{0.000000,0.000000,0.000000}%
\pgfsetstrokecolor{currentstroke}%
\pgfsetdash{}{0pt}%
\pgfsys@defobject{currentmarker}{\pgfqpoint{0.000000in}{-0.048611in}}{\pgfqpoint{0.000000in}{0.000000in}}{%
\pgfpathmoveto{\pgfqpoint{0.000000in}{0.000000in}}%
\pgfpathlineto{\pgfqpoint{0.000000in}{-0.048611in}}%
\pgfusepath{stroke,fill}%
}%
\begin{pgfscope}%
\pgfsys@transformshift{1.025455in}{0.696000in}%
\pgfsys@useobject{currentmarker}{}%
\end{pgfscope}%
\end{pgfscope}%
\begin{pgfscope}%
\definecolor{textcolor}{rgb}{0.000000,0.000000,0.000000}%
\pgfsetstrokecolor{textcolor}%
\pgfsetfillcolor{textcolor}%
\pgftext[x=1.025455in,y=0.598778in,,top]{\color{textcolor}\rmfamily\fontsize{10.000000}{12.000000}\selectfont \(\displaystyle {0}\)}%
\end{pgfscope}%
\begin{pgfscope}%
\pgfsetbuttcap%
\pgfsetroundjoin%
\definecolor{currentfill}{rgb}{0.000000,0.000000,0.000000}%
\pgfsetfillcolor{currentfill}%
\pgfsetlinewidth{0.803000pt}%
\definecolor{currentstroke}{rgb}{0.000000,0.000000,0.000000}%
\pgfsetstrokecolor{currentstroke}%
\pgfsetdash{}{0pt}%
\pgfsys@defobject{currentmarker}{\pgfqpoint{0.000000in}{-0.048611in}}{\pgfqpoint{0.000000in}{0.000000in}}{%
\pgfpathmoveto{\pgfqpoint{0.000000in}{0.000000in}}%
\pgfpathlineto{\pgfqpoint{0.000000in}{-0.048611in}}%
\pgfusepath{stroke,fill}%
}%
\begin{pgfscope}%
\pgfsys@transformshift{2.027697in}{0.696000in}%
\pgfsys@useobject{currentmarker}{}%
\end{pgfscope}%
\end{pgfscope}%
\begin{pgfscope}%
\definecolor{textcolor}{rgb}{0.000000,0.000000,0.000000}%
\pgfsetstrokecolor{textcolor}%
\pgfsetfillcolor{textcolor}%
\pgftext[x=2.027697in,y=0.598778in,,top]{\color{textcolor}\rmfamily\fontsize{10.000000}{12.000000}\selectfont \(\displaystyle {10}\)}%
\end{pgfscope}%
\begin{pgfscope}%
\pgfsetbuttcap%
\pgfsetroundjoin%
\definecolor{currentfill}{rgb}{0.000000,0.000000,0.000000}%
\pgfsetfillcolor{currentfill}%
\pgfsetlinewidth{0.803000pt}%
\definecolor{currentstroke}{rgb}{0.000000,0.000000,0.000000}%
\pgfsetstrokecolor{currentstroke}%
\pgfsetdash{}{0pt}%
\pgfsys@defobject{currentmarker}{\pgfqpoint{0.000000in}{-0.048611in}}{\pgfqpoint{0.000000in}{0.000000in}}{%
\pgfpathmoveto{\pgfqpoint{0.000000in}{0.000000in}}%
\pgfpathlineto{\pgfqpoint{0.000000in}{-0.048611in}}%
\pgfusepath{stroke,fill}%
}%
\begin{pgfscope}%
\pgfsys@transformshift{3.029940in}{0.696000in}%
\pgfsys@useobject{currentmarker}{}%
\end{pgfscope}%
\end{pgfscope}%
\begin{pgfscope}%
\definecolor{textcolor}{rgb}{0.000000,0.000000,0.000000}%
\pgfsetstrokecolor{textcolor}%
\pgfsetfillcolor{textcolor}%
\pgftext[x=3.029940in,y=0.598778in,,top]{\color{textcolor}\rmfamily\fontsize{10.000000}{12.000000}\selectfont \(\displaystyle {20}\)}%
\end{pgfscope}%
\begin{pgfscope}%
\pgfsetbuttcap%
\pgfsetroundjoin%
\definecolor{currentfill}{rgb}{0.000000,0.000000,0.000000}%
\pgfsetfillcolor{currentfill}%
\pgfsetlinewidth{0.803000pt}%
\definecolor{currentstroke}{rgb}{0.000000,0.000000,0.000000}%
\pgfsetstrokecolor{currentstroke}%
\pgfsetdash{}{0pt}%
\pgfsys@defobject{currentmarker}{\pgfqpoint{0.000000in}{-0.048611in}}{\pgfqpoint{0.000000in}{0.000000in}}{%
\pgfpathmoveto{\pgfqpoint{0.000000in}{0.000000in}}%
\pgfpathlineto{\pgfqpoint{0.000000in}{-0.048611in}}%
\pgfusepath{stroke,fill}%
}%
\begin{pgfscope}%
\pgfsys@transformshift{4.032183in}{0.696000in}%
\pgfsys@useobject{currentmarker}{}%
\end{pgfscope}%
\end{pgfscope}%
\begin{pgfscope}%
\definecolor{textcolor}{rgb}{0.000000,0.000000,0.000000}%
\pgfsetstrokecolor{textcolor}%
\pgfsetfillcolor{textcolor}%
\pgftext[x=4.032183in,y=0.598778in,,top]{\color{textcolor}\rmfamily\fontsize{10.000000}{12.000000}\selectfont \(\displaystyle {30}\)}%
\end{pgfscope}%
\begin{pgfscope}%
\pgfsetbuttcap%
\pgfsetroundjoin%
\definecolor{currentfill}{rgb}{0.000000,0.000000,0.000000}%
\pgfsetfillcolor{currentfill}%
\pgfsetlinewidth{0.803000pt}%
\definecolor{currentstroke}{rgb}{0.000000,0.000000,0.000000}%
\pgfsetstrokecolor{currentstroke}%
\pgfsetdash{}{0pt}%
\pgfsys@defobject{currentmarker}{\pgfqpoint{0.000000in}{-0.048611in}}{\pgfqpoint{0.000000in}{0.000000in}}{%
\pgfpathmoveto{\pgfqpoint{0.000000in}{0.000000in}}%
\pgfpathlineto{\pgfqpoint{0.000000in}{-0.048611in}}%
\pgfusepath{stroke,fill}%
}%
\begin{pgfscope}%
\pgfsys@transformshift{5.034426in}{0.696000in}%
\pgfsys@useobject{currentmarker}{}%
\end{pgfscope}%
\end{pgfscope}%
\begin{pgfscope}%
\definecolor{textcolor}{rgb}{0.000000,0.000000,0.000000}%
\pgfsetstrokecolor{textcolor}%
\pgfsetfillcolor{textcolor}%
\pgftext[x=5.034426in,y=0.598778in,,top]{\color{textcolor}\rmfamily\fontsize{10.000000}{12.000000}\selectfont \(\displaystyle {40}\)}%
\end{pgfscope}%
\begin{pgfscope}%
\pgfsetbuttcap%
\pgfsetroundjoin%
\definecolor{currentfill}{rgb}{0.000000,0.000000,0.000000}%
\pgfsetfillcolor{currentfill}%
\pgfsetlinewidth{0.803000pt}%
\definecolor{currentstroke}{rgb}{0.000000,0.000000,0.000000}%
\pgfsetstrokecolor{currentstroke}%
\pgfsetdash{}{0pt}%
\pgfsys@defobject{currentmarker}{\pgfqpoint{-0.048611in}{0.000000in}}{\pgfqpoint{-0.000000in}{0.000000in}}{%
\pgfpathmoveto{\pgfqpoint{-0.000000in}{0.000000in}}%
\pgfpathlineto{\pgfqpoint{-0.048611in}{0.000000in}}%
\pgfusepath{stroke,fill}%
}%
\begin{pgfscope}%
\pgfsys@transformshift{1.025455in}{0.696000in}%
\pgfsys@useobject{currentmarker}{}%
\end{pgfscope}%
\end{pgfscope}%
\begin{pgfscope}%
\definecolor{textcolor}{rgb}{0.000000,0.000000,0.000000}%
\pgfsetstrokecolor{textcolor}%
\pgfsetfillcolor{textcolor}%
\pgftext[x=0.858788in, y=0.647775in, left, base]{\color{textcolor}\rmfamily\fontsize{10.000000}{12.000000}\selectfont \(\displaystyle {0}\)}%
\end{pgfscope}%
\begin{pgfscope}%
\pgfsetbuttcap%
\pgfsetroundjoin%
\definecolor{currentfill}{rgb}{0.000000,0.000000,0.000000}%
\pgfsetfillcolor{currentfill}%
\pgfsetlinewidth{0.803000pt}%
\definecolor{currentstroke}{rgb}{0.000000,0.000000,0.000000}%
\pgfsetstrokecolor{currentstroke}%
\pgfsetdash{}{0pt}%
\pgfsys@defobject{currentmarker}{\pgfqpoint{-0.048611in}{0.000000in}}{\pgfqpoint{-0.000000in}{0.000000in}}{%
\pgfpathmoveto{\pgfqpoint{-0.000000in}{0.000000in}}%
\pgfpathlineto{\pgfqpoint{-0.048611in}{0.000000in}}%
\pgfusepath{stroke,fill}%
}%
\begin{pgfscope}%
\pgfsys@transformshift{1.025455in}{1.377818in}%
\pgfsys@useobject{currentmarker}{}%
\end{pgfscope}%
\end{pgfscope}%
\begin{pgfscope}%
\definecolor{textcolor}{rgb}{0.000000,0.000000,0.000000}%
\pgfsetstrokecolor{textcolor}%
\pgfsetfillcolor{textcolor}%
\pgftext[x=0.719898in, y=1.329593in, left, base]{\color{textcolor}\rmfamily\fontsize{10.000000}{12.000000}\selectfont \(\displaystyle {500}\)}%
\end{pgfscope}%
\begin{pgfscope}%
\pgfsetbuttcap%
\pgfsetroundjoin%
\definecolor{currentfill}{rgb}{0.000000,0.000000,0.000000}%
\pgfsetfillcolor{currentfill}%
\pgfsetlinewidth{0.803000pt}%
\definecolor{currentstroke}{rgb}{0.000000,0.000000,0.000000}%
\pgfsetstrokecolor{currentstroke}%
\pgfsetdash{}{0pt}%
\pgfsys@defobject{currentmarker}{\pgfqpoint{-0.048611in}{0.000000in}}{\pgfqpoint{-0.000000in}{0.000000in}}{%
\pgfpathmoveto{\pgfqpoint{-0.000000in}{0.000000in}}%
\pgfpathlineto{\pgfqpoint{-0.048611in}{0.000000in}}%
\pgfusepath{stroke,fill}%
}%
\begin{pgfscope}%
\pgfsys@transformshift{1.025455in}{2.059636in}%
\pgfsys@useobject{currentmarker}{}%
\end{pgfscope}%
\end{pgfscope}%
\begin{pgfscope}%
\definecolor{textcolor}{rgb}{0.000000,0.000000,0.000000}%
\pgfsetstrokecolor{textcolor}%
\pgfsetfillcolor{textcolor}%
\pgftext[x=0.650454in, y=2.011411in, left, base]{\color{textcolor}\rmfamily\fontsize{10.000000}{12.000000}\selectfont \(\displaystyle {1000}\)}%
\end{pgfscope}%
\begin{pgfscope}%
\pgfsetbuttcap%
\pgfsetroundjoin%
\definecolor{currentfill}{rgb}{0.000000,0.000000,0.000000}%
\pgfsetfillcolor{currentfill}%
\pgfsetlinewidth{0.803000pt}%
\definecolor{currentstroke}{rgb}{0.000000,0.000000,0.000000}%
\pgfsetstrokecolor{currentstroke}%
\pgfsetdash{}{0pt}%
\pgfsys@defobject{currentmarker}{\pgfqpoint{-0.048611in}{0.000000in}}{\pgfqpoint{-0.000000in}{0.000000in}}{%
\pgfpathmoveto{\pgfqpoint{-0.000000in}{0.000000in}}%
\pgfpathlineto{\pgfqpoint{-0.048611in}{0.000000in}}%
\pgfusepath{stroke,fill}%
}%
\begin{pgfscope}%
\pgfsys@transformshift{1.025455in}{2.741455in}%
\pgfsys@useobject{currentmarker}{}%
\end{pgfscope}%
\end{pgfscope}%
\begin{pgfscope}%
\definecolor{textcolor}{rgb}{0.000000,0.000000,0.000000}%
\pgfsetstrokecolor{textcolor}%
\pgfsetfillcolor{textcolor}%
\pgftext[x=0.650454in, y=2.693229in, left, base]{\color{textcolor}\rmfamily\fontsize{10.000000}{12.000000}\selectfont \(\displaystyle {1500}\)}%
\end{pgfscope}%
\begin{pgfscope}%
\pgfsetbuttcap%
\pgfsetroundjoin%
\definecolor{currentfill}{rgb}{0.000000,0.000000,0.000000}%
\pgfsetfillcolor{currentfill}%
\pgfsetlinewidth{0.803000pt}%
\definecolor{currentstroke}{rgb}{0.000000,0.000000,0.000000}%
\pgfsetstrokecolor{currentstroke}%
\pgfsetdash{}{0pt}%
\pgfsys@defobject{currentmarker}{\pgfqpoint{-0.048611in}{0.000000in}}{\pgfqpoint{-0.000000in}{0.000000in}}{%
\pgfpathmoveto{\pgfqpoint{-0.000000in}{0.000000in}}%
\pgfpathlineto{\pgfqpoint{-0.048611in}{0.000000in}}%
\pgfusepath{stroke,fill}%
}%
\begin{pgfscope}%
\pgfsys@transformshift{1.025455in}{3.423273in}%
\pgfsys@useobject{currentmarker}{}%
\end{pgfscope}%
\end{pgfscope}%
\begin{pgfscope}%
\definecolor{textcolor}{rgb}{0.000000,0.000000,0.000000}%
\pgfsetstrokecolor{textcolor}%
\pgfsetfillcolor{textcolor}%
\pgftext[x=0.650454in, y=3.375047in, left, base]{\color{textcolor}\rmfamily\fontsize{10.000000}{12.000000}\selectfont \(\displaystyle {2000}\)}%
\end{pgfscope}%
\begin{pgfscope}%
\pgfsetbuttcap%
\pgfsetroundjoin%
\definecolor{currentfill}{rgb}{0.000000,0.000000,0.000000}%
\pgfsetfillcolor{currentfill}%
\pgfsetlinewidth{0.803000pt}%
\definecolor{currentstroke}{rgb}{0.000000,0.000000,0.000000}%
\pgfsetstrokecolor{currentstroke}%
\pgfsetdash{}{0pt}%
\pgfsys@defobject{currentmarker}{\pgfqpoint{-0.048611in}{0.000000in}}{\pgfqpoint{-0.000000in}{0.000000in}}{%
\pgfpathmoveto{\pgfqpoint{-0.000000in}{0.000000in}}%
\pgfpathlineto{\pgfqpoint{-0.048611in}{0.000000in}}%
\pgfusepath{stroke,fill}%
}%
\begin{pgfscope}%
\pgfsys@transformshift{1.025455in}{4.105091in}%
\pgfsys@useobject{currentmarker}{}%
\end{pgfscope}%
\end{pgfscope}%
\begin{pgfscope}%
\definecolor{textcolor}{rgb}{0.000000,0.000000,0.000000}%
\pgfsetstrokecolor{textcolor}%
\pgfsetfillcolor{textcolor}%
\pgftext[x=0.650454in, y=4.056866in, left, base]{\color{textcolor}\rmfamily\fontsize{10.000000}{12.000000}\selectfont \(\displaystyle {2500}\)}%
\end{pgfscope}%
\begin{pgfscope}%
\pgfpathrectangle{\pgfqpoint{0.800000in}{0.528000in}}{\pgfqpoint{4.960000in}{3.696000in}}%
\pgfusepath{clip}%
\pgfsetrectcap%
\pgfsetroundjoin%
\pgfsetlinewidth{1.505625pt}%
\definecolor{currentstroke}{rgb}{0.564706,0.478431,0.662745}%
\pgfsetstrokecolor{currentstroke}%
\pgfsetdash{}{0pt}%
\pgfpathmoveto{\pgfqpoint{1.025455in}{3.023727in}}%
\pgfpathlineto{\pgfqpoint{1.027459in}{2.992364in}}%
\pgfpathlineto{\pgfqpoint{1.028461in}{2.988273in}}%
\pgfpathlineto{\pgfqpoint{1.029464in}{2.992364in}}%
\pgfpathlineto{\pgfqpoint{1.030466in}{2.984182in}}%
\pgfpathlineto{\pgfqpoint{1.031468in}{2.986909in}}%
\pgfpathlineto{\pgfqpoint{1.032470in}{2.996455in}}%
\pgfpathlineto{\pgfqpoint{1.033472in}{2.993727in}}%
\pgfpathlineto{\pgfqpoint{1.034475in}{2.996455in}}%
\pgfpathlineto{\pgfqpoint{1.035477in}{2.995091in}}%
\pgfpathlineto{\pgfqpoint{1.036479in}{2.984182in}}%
\pgfpathlineto{\pgfqpoint{1.037481in}{2.992364in}}%
\pgfpathlineto{\pgfqpoint{1.038484in}{2.978727in}}%
\pgfpathlineto{\pgfqpoint{1.039486in}{2.982818in}}%
\pgfpathlineto{\pgfqpoint{1.040488in}{2.978727in}}%
\pgfpathlineto{\pgfqpoint{1.041490in}{2.981455in}}%
\pgfpathlineto{\pgfqpoint{1.043495in}{2.967818in}}%
\pgfpathlineto{\pgfqpoint{1.045499in}{2.963727in}}%
\pgfpathlineto{\pgfqpoint{1.048506in}{2.973273in}}%
\pgfpathlineto{\pgfqpoint{1.049508in}{2.974636in}}%
\pgfpathlineto{\pgfqpoint{1.050511in}{2.959636in}}%
\pgfpathlineto{\pgfqpoint{1.051513in}{2.982818in}}%
\pgfpathlineto{\pgfqpoint{1.053517in}{2.966455in}}%
\pgfpathlineto{\pgfqpoint{1.055522in}{2.978727in}}%
\pgfpathlineto{\pgfqpoint{1.056524in}{2.959636in}}%
\pgfpathlineto{\pgfqpoint{1.058529in}{2.978727in}}%
\pgfpathlineto{\pgfqpoint{1.059531in}{2.980091in}}%
\pgfpathlineto{\pgfqpoint{1.063540in}{2.999182in}}%
\pgfpathlineto{\pgfqpoint{1.064542in}{2.978727in}}%
\pgfpathlineto{\pgfqpoint{1.065544in}{2.992364in}}%
\pgfpathlineto{\pgfqpoint{1.066547in}{2.991000in}}%
\pgfpathlineto{\pgfqpoint{1.067549in}{2.995091in}}%
\pgfpathlineto{\pgfqpoint{1.068551in}{2.985545in}}%
\pgfpathlineto{\pgfqpoint{1.069553in}{2.986909in}}%
\pgfpathlineto{\pgfqpoint{1.071558in}{2.999182in}}%
\pgfpathlineto{\pgfqpoint{1.075567in}{2.973273in}}%
\pgfpathlineto{\pgfqpoint{1.076569in}{2.974636in}}%
\pgfpathlineto{\pgfqpoint{1.078573in}{2.969182in}}%
\pgfpathlineto{\pgfqpoint{1.079576in}{2.970545in}}%
\pgfpathlineto{\pgfqpoint{1.081580in}{2.959636in}}%
\pgfpathlineto{\pgfqpoint{1.083585in}{2.969182in}}%
\pgfpathlineto{\pgfqpoint{1.084587in}{2.970545in}}%
\pgfpathlineto{\pgfqpoint{1.086591in}{2.952818in}}%
\pgfpathlineto{\pgfqpoint{1.087594in}{2.981455in}}%
\pgfpathlineto{\pgfqpoint{1.088596in}{2.971909in}}%
\pgfpathlineto{\pgfqpoint{1.091603in}{2.982818in}}%
\pgfpathlineto{\pgfqpoint{1.092605in}{2.980091in}}%
\pgfpathlineto{\pgfqpoint{1.093607in}{2.989636in}}%
\pgfpathlineto{\pgfqpoint{1.094609in}{2.970545in}}%
\pgfpathlineto{\pgfqpoint{1.095612in}{2.986909in}}%
\pgfpathlineto{\pgfqpoint{1.096614in}{2.981455in}}%
\pgfpathlineto{\pgfqpoint{1.099621in}{2.991000in}}%
\pgfpathlineto{\pgfqpoint{1.100623in}{2.977364in}}%
\pgfpathlineto{\pgfqpoint{1.101625in}{2.993727in}}%
\pgfpathlineto{\pgfqpoint{1.102627in}{2.978727in}}%
\pgfpathlineto{\pgfqpoint{1.104632in}{2.985545in}}%
\pgfpathlineto{\pgfqpoint{1.106636in}{2.988273in}}%
\pgfpathlineto{\pgfqpoint{1.108641in}{2.984182in}}%
\pgfpathlineto{\pgfqpoint{1.110645in}{2.973273in}}%
\pgfpathlineto{\pgfqpoint{1.111647in}{2.965091in}}%
\pgfpathlineto{\pgfqpoint{1.113652in}{2.986909in}}%
\pgfpathlineto{\pgfqpoint{1.116659in}{2.963727in}}%
\pgfpathlineto{\pgfqpoint{1.117661in}{2.965091in}}%
\pgfpathlineto{\pgfqpoint{1.119665in}{2.980091in}}%
\pgfpathlineto{\pgfqpoint{1.120668in}{2.965091in}}%
\pgfpathlineto{\pgfqpoint{1.121670in}{2.969182in}}%
\pgfpathlineto{\pgfqpoint{1.122672in}{2.961000in}}%
\pgfpathlineto{\pgfqpoint{1.124677in}{2.974636in}}%
\pgfpathlineto{\pgfqpoint{1.126681in}{2.967818in}}%
\pgfpathlineto{\pgfqpoint{1.129688in}{2.985545in}}%
\pgfpathlineto{\pgfqpoint{1.131692in}{2.982818in}}%
\pgfpathlineto{\pgfqpoint{1.133697in}{2.986909in}}%
\pgfpathlineto{\pgfqpoint{1.134699in}{2.973273in}}%
\pgfpathlineto{\pgfqpoint{1.137706in}{2.996455in}}%
\pgfpathlineto{\pgfqpoint{1.139710in}{2.973273in}}%
\pgfpathlineto{\pgfqpoint{1.142717in}{2.986909in}}%
\pgfpathlineto{\pgfqpoint{1.144721in}{2.977364in}}%
\pgfpathlineto{\pgfqpoint{1.145724in}{2.984182in}}%
\pgfpathlineto{\pgfqpoint{1.147728in}{2.977364in}}%
\pgfpathlineto{\pgfqpoint{1.148730in}{2.980091in}}%
\pgfpathlineto{\pgfqpoint{1.151737in}{2.963727in}}%
\pgfpathlineto{\pgfqpoint{1.152739in}{2.973273in}}%
\pgfpathlineto{\pgfqpoint{1.153742in}{2.961000in}}%
\pgfpathlineto{\pgfqpoint{1.155746in}{2.969182in}}%
\pgfpathlineto{\pgfqpoint{1.156748in}{2.969182in}}%
\pgfpathlineto{\pgfqpoint{1.157751in}{2.981455in}}%
\pgfpathlineto{\pgfqpoint{1.158753in}{2.976000in}}%
\pgfpathlineto{\pgfqpoint{1.160757in}{2.988273in}}%
\pgfpathlineto{\pgfqpoint{1.162762in}{2.980091in}}%
\pgfpathlineto{\pgfqpoint{1.163764in}{2.984182in}}%
\pgfpathlineto{\pgfqpoint{1.164766in}{2.977364in}}%
\pgfpathlineto{\pgfqpoint{1.165769in}{2.991000in}}%
\pgfpathlineto{\pgfqpoint{1.166771in}{2.986909in}}%
\pgfpathlineto{\pgfqpoint{1.167773in}{2.996455in}}%
\pgfpathlineto{\pgfqpoint{1.168775in}{2.988273in}}%
\pgfpathlineto{\pgfqpoint{1.169778in}{2.993727in}}%
\pgfpathlineto{\pgfqpoint{1.170780in}{2.984182in}}%
\pgfpathlineto{\pgfqpoint{1.173786in}{2.989636in}}%
\pgfpathlineto{\pgfqpoint{1.175791in}{2.984182in}}%
\pgfpathlineto{\pgfqpoint{1.176793in}{2.976000in}}%
\pgfpathlineto{\pgfqpoint{1.177795in}{2.995091in}}%
\pgfpathlineto{\pgfqpoint{1.179800in}{2.969182in}}%
\pgfpathlineto{\pgfqpoint{1.180802in}{2.977364in}}%
\pgfpathlineto{\pgfqpoint{1.181804in}{2.973273in}}%
\pgfpathlineto{\pgfqpoint{1.182807in}{2.963727in}}%
\pgfpathlineto{\pgfqpoint{1.183809in}{2.981455in}}%
\pgfpathlineto{\pgfqpoint{1.184811in}{2.969182in}}%
\pgfpathlineto{\pgfqpoint{1.185813in}{2.982818in}}%
\pgfpathlineto{\pgfqpoint{1.187818in}{2.966455in}}%
\pgfpathlineto{\pgfqpoint{1.190825in}{2.977364in}}%
\pgfpathlineto{\pgfqpoint{1.191827in}{2.970545in}}%
\pgfpathlineto{\pgfqpoint{1.193831in}{2.988273in}}%
\pgfpathlineto{\pgfqpoint{1.194834in}{2.985545in}}%
\pgfpathlineto{\pgfqpoint{1.195836in}{2.988273in}}%
\pgfpathlineto{\pgfqpoint{1.196838in}{2.996455in}}%
\pgfpathlineto{\pgfqpoint{1.197840in}{2.993727in}}%
\pgfpathlineto{\pgfqpoint{1.199845in}{2.986909in}}%
\pgfpathlineto{\pgfqpoint{1.200847in}{2.982818in}}%
\pgfpathlineto{\pgfqpoint{1.201849in}{2.989636in}}%
\pgfpathlineto{\pgfqpoint{1.202852in}{2.985545in}}%
\pgfpathlineto{\pgfqpoint{1.203854in}{2.988273in}}%
\pgfpathlineto{\pgfqpoint{1.204856in}{2.984182in}}%
\pgfpathlineto{\pgfqpoint{1.205858in}{2.974636in}}%
\pgfpathlineto{\pgfqpoint{1.206861in}{2.991000in}}%
\pgfpathlineto{\pgfqpoint{1.207863in}{2.989636in}}%
\pgfpathlineto{\pgfqpoint{1.209867in}{2.993727in}}%
\pgfpathlineto{\pgfqpoint{1.210869in}{2.966455in}}%
\pgfpathlineto{\pgfqpoint{1.211872in}{2.970545in}}%
\pgfpathlineto{\pgfqpoint{1.212874in}{2.985545in}}%
\pgfpathlineto{\pgfqpoint{1.214878in}{2.973273in}}%
\pgfpathlineto{\pgfqpoint{1.215881in}{2.977364in}}%
\pgfpathlineto{\pgfqpoint{1.217885in}{2.965091in}}%
\pgfpathlineto{\pgfqpoint{1.219890in}{2.971909in}}%
\pgfpathlineto{\pgfqpoint{1.220892in}{2.974636in}}%
\pgfpathlineto{\pgfqpoint{1.221894in}{2.982818in}}%
\pgfpathlineto{\pgfqpoint{1.222896in}{2.962364in}}%
\pgfpathlineto{\pgfqpoint{1.224901in}{2.982818in}}%
\pgfpathlineto{\pgfqpoint{1.226905in}{2.974636in}}%
\pgfpathlineto{\pgfqpoint{1.227908in}{2.999182in}}%
\pgfpathlineto{\pgfqpoint{1.228910in}{2.971909in}}%
\pgfpathlineto{\pgfqpoint{1.231917in}{2.993727in}}%
\pgfpathlineto{\pgfqpoint{1.232919in}{2.986909in}}%
\pgfpathlineto{\pgfqpoint{1.233921in}{2.997818in}}%
\pgfpathlineto{\pgfqpoint{1.234923in}{2.991000in}}%
\pgfpathlineto{\pgfqpoint{1.235926in}{3.000545in}}%
\pgfpathlineto{\pgfqpoint{1.237930in}{2.986909in}}%
\pgfpathlineto{\pgfqpoint{1.238932in}{2.986909in}}%
\pgfpathlineto{\pgfqpoint{1.239935in}{2.996455in}}%
\pgfpathlineto{\pgfqpoint{1.240937in}{2.984182in}}%
\pgfpathlineto{\pgfqpoint{1.241939in}{2.991000in}}%
\pgfpathlineto{\pgfqpoint{1.242941in}{2.971909in}}%
\pgfpathlineto{\pgfqpoint{1.244946in}{2.988273in}}%
\pgfpathlineto{\pgfqpoint{1.245948in}{2.986909in}}%
\pgfpathlineto{\pgfqpoint{1.247952in}{2.973273in}}%
\pgfpathlineto{\pgfqpoint{1.248955in}{2.976000in}}%
\pgfpathlineto{\pgfqpoint{1.249957in}{2.973273in}}%
\pgfpathlineto{\pgfqpoint{1.250959in}{2.977364in}}%
\pgfpathlineto{\pgfqpoint{1.252964in}{2.969182in}}%
\pgfpathlineto{\pgfqpoint{1.253966in}{2.985545in}}%
\pgfpathlineto{\pgfqpoint{1.254968in}{2.959636in}}%
\pgfpathlineto{\pgfqpoint{1.256973in}{2.974636in}}%
\pgfpathlineto{\pgfqpoint{1.257975in}{2.971909in}}%
\pgfpathlineto{\pgfqpoint{1.258977in}{2.967818in}}%
\pgfpathlineto{\pgfqpoint{1.260982in}{2.974636in}}%
\pgfpathlineto{\pgfqpoint{1.263988in}{2.991000in}}%
\pgfpathlineto{\pgfqpoint{1.264991in}{2.978727in}}%
\pgfpathlineto{\pgfqpoint{1.266995in}{2.997818in}}%
\pgfpathlineto{\pgfqpoint{1.267997in}{2.992364in}}%
\pgfpathlineto{\pgfqpoint{1.269000in}{2.996455in}}%
\pgfpathlineto{\pgfqpoint{1.271004in}{2.981455in}}%
\pgfpathlineto{\pgfqpoint{1.272006in}{2.993727in}}%
\pgfpathlineto{\pgfqpoint{1.274011in}{2.988273in}}%
\pgfpathlineto{\pgfqpoint{1.275013in}{2.993727in}}%
\pgfpathlineto{\pgfqpoint{1.277018in}{2.986909in}}%
\pgfpathlineto{\pgfqpoint{1.278020in}{2.992364in}}%
\pgfpathlineto{\pgfqpoint{1.279022in}{2.989636in}}%
\pgfpathlineto{\pgfqpoint{1.280024in}{2.993727in}}%
\pgfpathlineto{\pgfqpoint{1.281026in}{2.980091in}}%
\pgfpathlineto{\pgfqpoint{1.282029in}{2.981455in}}%
\pgfpathlineto{\pgfqpoint{1.283031in}{2.978727in}}%
\pgfpathlineto{\pgfqpoint{1.284033in}{2.988273in}}%
\pgfpathlineto{\pgfqpoint{1.285035in}{2.986909in}}%
\pgfpathlineto{\pgfqpoint{1.286038in}{2.988273in}}%
\pgfpathlineto{\pgfqpoint{1.287040in}{2.969182in}}%
\pgfpathlineto{\pgfqpoint{1.288042in}{2.989636in}}%
\pgfpathlineto{\pgfqpoint{1.289044in}{2.974636in}}%
\pgfpathlineto{\pgfqpoint{1.290047in}{2.984182in}}%
\pgfpathlineto{\pgfqpoint{1.291049in}{2.976000in}}%
\pgfpathlineto{\pgfqpoint{1.292051in}{2.977364in}}%
\pgfpathlineto{\pgfqpoint{1.293053in}{2.986909in}}%
\pgfpathlineto{\pgfqpoint{1.294056in}{2.974636in}}%
\pgfpathlineto{\pgfqpoint{1.296060in}{2.981455in}}%
\pgfpathlineto{\pgfqpoint{1.297062in}{2.976000in}}%
\pgfpathlineto{\pgfqpoint{1.298065in}{2.988273in}}%
\pgfpathlineto{\pgfqpoint{1.299067in}{2.978727in}}%
\pgfpathlineto{\pgfqpoint{1.302074in}{2.986909in}}%
\pgfpathlineto{\pgfqpoint{1.303076in}{2.988273in}}%
\pgfpathlineto{\pgfqpoint{1.304078in}{2.996455in}}%
\pgfpathlineto{\pgfqpoint{1.306083in}{2.982818in}}%
\pgfpathlineto{\pgfqpoint{1.308087in}{2.999182in}}%
\pgfpathlineto{\pgfqpoint{1.309089in}{2.992364in}}%
\pgfpathlineto{\pgfqpoint{1.310092in}{2.993727in}}%
\pgfpathlineto{\pgfqpoint{1.311094in}{2.982818in}}%
\pgfpathlineto{\pgfqpoint{1.312096in}{2.985545in}}%
\pgfpathlineto{\pgfqpoint{1.313098in}{2.995091in}}%
\pgfpathlineto{\pgfqpoint{1.314101in}{2.984182in}}%
\pgfpathlineto{\pgfqpoint{1.315103in}{2.985545in}}%
\pgfpathlineto{\pgfqpoint{1.316105in}{2.976000in}}%
\pgfpathlineto{\pgfqpoint{1.317107in}{2.978727in}}%
\pgfpathlineto{\pgfqpoint{1.319112in}{2.976000in}}%
\pgfpathlineto{\pgfqpoint{1.320114in}{2.985545in}}%
\pgfpathlineto{\pgfqpoint{1.322118in}{2.974636in}}%
\pgfpathlineto{\pgfqpoint{1.323121in}{2.980091in}}%
\pgfpathlineto{\pgfqpoint{1.324123in}{2.978727in}}%
\pgfpathlineto{\pgfqpoint{1.325125in}{2.967818in}}%
\pgfpathlineto{\pgfqpoint{1.326127in}{2.984182in}}%
\pgfpathlineto{\pgfqpoint{1.329134in}{2.966455in}}%
\pgfpathlineto{\pgfqpoint{1.332141in}{2.973273in}}%
\pgfpathlineto{\pgfqpoint{1.333143in}{2.982818in}}%
\pgfpathlineto{\pgfqpoint{1.334145in}{2.977364in}}%
\pgfpathlineto{\pgfqpoint{1.336150in}{2.989636in}}%
\pgfpathlineto{\pgfqpoint{1.337152in}{2.984182in}}%
\pgfpathlineto{\pgfqpoint{1.338154in}{2.988273in}}%
\pgfpathlineto{\pgfqpoint{1.339157in}{2.981455in}}%
\pgfpathlineto{\pgfqpoint{1.341161in}{2.991000in}}%
\pgfpathlineto{\pgfqpoint{1.342163in}{2.986909in}}%
\pgfpathlineto{\pgfqpoint{1.344168in}{3.003273in}}%
\pgfpathlineto{\pgfqpoint{1.345170in}{2.991000in}}%
\pgfpathlineto{\pgfqpoint{1.346172in}{2.996455in}}%
\pgfpathlineto{\pgfqpoint{1.347175in}{2.995091in}}%
\pgfpathlineto{\pgfqpoint{1.348177in}{3.001909in}}%
\pgfpathlineto{\pgfqpoint{1.350181in}{2.984182in}}%
\pgfpathlineto{\pgfqpoint{1.351183in}{2.992364in}}%
\pgfpathlineto{\pgfqpoint{1.352186in}{2.978727in}}%
\pgfpathlineto{\pgfqpoint{1.354190in}{2.982818in}}%
\pgfpathlineto{\pgfqpoint{1.356195in}{2.977364in}}%
\pgfpathlineto{\pgfqpoint{1.359201in}{2.971909in}}%
\pgfpathlineto{\pgfqpoint{1.360204in}{2.986909in}}%
\pgfpathlineto{\pgfqpoint{1.362208in}{2.974636in}}%
\pgfpathlineto{\pgfqpoint{1.364213in}{2.971909in}}%
\pgfpathlineto{\pgfqpoint{1.365215in}{2.974636in}}%
\pgfpathlineto{\pgfqpoint{1.366217in}{2.970545in}}%
\pgfpathlineto{\pgfqpoint{1.368222in}{2.980091in}}%
\pgfpathlineto{\pgfqpoint{1.369224in}{2.971909in}}%
\pgfpathlineto{\pgfqpoint{1.370226in}{2.977364in}}%
\pgfpathlineto{\pgfqpoint{1.372231in}{2.970545in}}%
\pgfpathlineto{\pgfqpoint{1.374235in}{2.988273in}}%
\pgfpathlineto{\pgfqpoint{1.375237in}{2.973273in}}%
\pgfpathlineto{\pgfqpoint{1.378244in}{2.995091in}}%
\pgfpathlineto{\pgfqpoint{1.380249in}{2.982818in}}%
\pgfpathlineto{\pgfqpoint{1.384258in}{2.996455in}}%
\pgfpathlineto{\pgfqpoint{1.386262in}{2.984182in}}%
\pgfpathlineto{\pgfqpoint{1.387264in}{2.985545in}}%
\pgfpathlineto{\pgfqpoint{1.389269in}{2.995091in}}%
\pgfpathlineto{\pgfqpoint{1.391273in}{2.980091in}}%
\pgfpathlineto{\pgfqpoint{1.392275in}{2.984182in}}%
\pgfpathlineto{\pgfqpoint{1.393278in}{2.965091in}}%
\pgfpathlineto{\pgfqpoint{1.395282in}{2.982818in}}%
\pgfpathlineto{\pgfqpoint{1.400293in}{2.977364in}}%
\pgfpathlineto{\pgfqpoint{1.401296in}{2.980091in}}%
\pgfpathlineto{\pgfqpoint{1.402298in}{2.973273in}}%
\pgfpathlineto{\pgfqpoint{1.403300in}{2.976000in}}%
\pgfpathlineto{\pgfqpoint{1.405305in}{2.969182in}}%
\pgfpathlineto{\pgfqpoint{1.408311in}{2.974636in}}%
\pgfpathlineto{\pgfqpoint{1.409314in}{2.997818in}}%
\pgfpathlineto{\pgfqpoint{1.410316in}{2.969182in}}%
\pgfpathlineto{\pgfqpoint{1.412320in}{2.988273in}}%
\pgfpathlineto{\pgfqpoint{1.413323in}{2.980091in}}%
\pgfpathlineto{\pgfqpoint{1.414325in}{2.988273in}}%
\pgfpathlineto{\pgfqpoint{1.415327in}{2.978727in}}%
\pgfpathlineto{\pgfqpoint{1.416329in}{2.991000in}}%
\pgfpathlineto{\pgfqpoint{1.417332in}{2.973273in}}%
\pgfpathlineto{\pgfqpoint{1.419336in}{2.989636in}}%
\pgfpathlineto{\pgfqpoint{1.420338in}{2.988273in}}%
\pgfpathlineto{\pgfqpoint{1.421340in}{2.999182in}}%
\pgfpathlineto{\pgfqpoint{1.422343in}{2.986909in}}%
\pgfpathlineto{\pgfqpoint{1.423345in}{2.996455in}}%
\pgfpathlineto{\pgfqpoint{1.425349in}{2.984182in}}%
\pgfpathlineto{\pgfqpoint{1.426352in}{2.991000in}}%
\pgfpathlineto{\pgfqpoint{1.427354in}{2.977364in}}%
\pgfpathlineto{\pgfqpoint{1.428356in}{2.991000in}}%
\pgfpathlineto{\pgfqpoint{1.430361in}{2.977364in}}%
\pgfpathlineto{\pgfqpoint{1.431363in}{2.991000in}}%
\pgfpathlineto{\pgfqpoint{1.433367in}{2.967818in}}%
\pgfpathlineto{\pgfqpoint{1.436374in}{2.982818in}}%
\pgfpathlineto{\pgfqpoint{1.437376in}{2.961000in}}%
\pgfpathlineto{\pgfqpoint{1.438379in}{2.974636in}}%
\pgfpathlineto{\pgfqpoint{1.439381in}{2.971909in}}%
\pgfpathlineto{\pgfqpoint{1.440383in}{2.984182in}}%
\pgfpathlineto{\pgfqpoint{1.442388in}{2.970545in}}%
\pgfpathlineto{\pgfqpoint{1.443390in}{2.966455in}}%
\pgfpathlineto{\pgfqpoint{1.444392in}{2.967818in}}%
\pgfpathlineto{\pgfqpoint{1.445394in}{2.980091in}}%
\pgfpathlineto{\pgfqpoint{1.446397in}{2.978727in}}%
\pgfpathlineto{\pgfqpoint{1.448401in}{2.986909in}}%
\pgfpathlineto{\pgfqpoint{1.449403in}{2.969182in}}%
\pgfpathlineto{\pgfqpoint{1.451408in}{2.984182in}}%
\pgfpathlineto{\pgfqpoint{1.452410in}{2.984182in}}%
\pgfpathlineto{\pgfqpoint{1.453412in}{2.999182in}}%
\pgfpathlineto{\pgfqpoint{1.454415in}{2.982818in}}%
\pgfpathlineto{\pgfqpoint{1.455417in}{2.986909in}}%
\pgfpathlineto{\pgfqpoint{1.456419in}{2.981455in}}%
\pgfpathlineto{\pgfqpoint{1.458423in}{3.000545in}}%
\pgfpathlineto{\pgfqpoint{1.459426in}{2.996455in}}%
\pgfpathlineto{\pgfqpoint{1.461430in}{2.967818in}}%
\pgfpathlineto{\pgfqpoint{1.462432in}{3.000545in}}%
\pgfpathlineto{\pgfqpoint{1.463435in}{2.984182in}}%
\pgfpathlineto{\pgfqpoint{1.464437in}{2.988273in}}%
\pgfpathlineto{\pgfqpoint{1.465439in}{2.986909in}}%
\pgfpathlineto{\pgfqpoint{1.466441in}{2.989636in}}%
\pgfpathlineto{\pgfqpoint{1.467444in}{2.988273in}}%
\pgfpathlineto{\pgfqpoint{1.468446in}{2.973273in}}%
\pgfpathlineto{\pgfqpoint{1.470450in}{2.999182in}}%
\pgfpathlineto{\pgfqpoint{1.471453in}{2.974636in}}%
\pgfpathlineto{\pgfqpoint{1.472455in}{2.988273in}}%
\pgfpathlineto{\pgfqpoint{1.473457in}{2.966455in}}%
\pgfpathlineto{\pgfqpoint{1.475462in}{2.981455in}}%
\pgfpathlineto{\pgfqpoint{1.476464in}{2.980091in}}%
\pgfpathlineto{\pgfqpoint{1.478468in}{2.965091in}}%
\pgfpathlineto{\pgfqpoint{1.480473in}{2.982818in}}%
\pgfpathlineto{\pgfqpoint{1.481475in}{2.974636in}}%
\pgfpathlineto{\pgfqpoint{1.482477in}{2.978727in}}%
\pgfpathlineto{\pgfqpoint{1.483480in}{2.959636in}}%
\pgfpathlineto{\pgfqpoint{1.484482in}{2.963727in}}%
\pgfpathlineto{\pgfqpoint{1.486486in}{2.986909in}}%
\pgfpathlineto{\pgfqpoint{1.488491in}{2.976000in}}%
\pgfpathlineto{\pgfqpoint{1.489493in}{2.986909in}}%
\pgfpathlineto{\pgfqpoint{1.490495in}{2.970545in}}%
\pgfpathlineto{\pgfqpoint{1.492500in}{2.992364in}}%
\pgfpathlineto{\pgfqpoint{1.494504in}{2.981455in}}%
\pgfpathlineto{\pgfqpoint{1.495506in}{2.980091in}}%
\pgfpathlineto{\pgfqpoint{1.497511in}{2.982818in}}%
\pgfpathlineto{\pgfqpoint{1.498513in}{2.996455in}}%
\pgfpathlineto{\pgfqpoint{1.499515in}{2.995091in}}%
\pgfpathlineto{\pgfqpoint{1.500518in}{2.980091in}}%
\pgfpathlineto{\pgfqpoint{1.502522in}{3.003273in}}%
\pgfpathlineto{\pgfqpoint{1.504527in}{2.996455in}}%
\pgfpathlineto{\pgfqpoint{1.505529in}{2.977364in}}%
\pgfpathlineto{\pgfqpoint{1.506531in}{3.001909in}}%
\pgfpathlineto{\pgfqpoint{1.507533in}{2.989636in}}%
\pgfpathlineto{\pgfqpoint{1.508536in}{2.991000in}}%
\pgfpathlineto{\pgfqpoint{1.510540in}{2.985545in}}%
\pgfpathlineto{\pgfqpoint{1.512545in}{2.973273in}}%
\pgfpathlineto{\pgfqpoint{1.513547in}{2.986909in}}%
\pgfpathlineto{\pgfqpoint{1.514549in}{2.985545in}}%
\pgfpathlineto{\pgfqpoint{1.515551in}{2.985545in}}%
\pgfpathlineto{\pgfqpoint{1.516554in}{2.984182in}}%
\pgfpathlineto{\pgfqpoint{1.518558in}{2.974636in}}%
\pgfpathlineto{\pgfqpoint{1.520563in}{2.986909in}}%
\pgfpathlineto{\pgfqpoint{1.521565in}{2.967818in}}%
\pgfpathlineto{\pgfqpoint{1.522567in}{2.973273in}}%
\pgfpathlineto{\pgfqpoint{1.523569in}{2.971909in}}%
\pgfpathlineto{\pgfqpoint{1.526576in}{2.976000in}}%
\pgfpathlineto{\pgfqpoint{1.527578in}{2.958273in}}%
\pgfpathlineto{\pgfqpoint{1.528580in}{2.977364in}}%
\pgfpathlineto{\pgfqpoint{1.529583in}{2.974636in}}%
\pgfpathlineto{\pgfqpoint{1.531587in}{2.986909in}}%
\pgfpathlineto{\pgfqpoint{1.532589in}{2.965091in}}%
\pgfpathlineto{\pgfqpoint{1.534594in}{2.980091in}}%
\pgfpathlineto{\pgfqpoint{1.537601in}{2.992364in}}%
\pgfpathlineto{\pgfqpoint{1.538603in}{3.006000in}}%
\pgfpathlineto{\pgfqpoint{1.540607in}{2.997818in}}%
\pgfpathlineto{\pgfqpoint{1.541610in}{2.996455in}}%
\pgfpathlineto{\pgfqpoint{1.543614in}{3.008727in}}%
\pgfpathlineto{\pgfqpoint{1.544616in}{2.995091in}}%
\pgfpathlineto{\pgfqpoint{1.545619in}{2.999182in}}%
\pgfpathlineto{\pgfqpoint{1.546621in}{2.997818in}}%
\pgfpathlineto{\pgfqpoint{1.548625in}{3.014182in}}%
\pgfpathlineto{\pgfqpoint{1.551632in}{2.977364in}}%
\pgfpathlineto{\pgfqpoint{1.552634in}{3.001909in}}%
\pgfpathlineto{\pgfqpoint{1.553637in}{3.000545in}}%
\pgfpathlineto{\pgfqpoint{1.554639in}{2.982818in}}%
\pgfpathlineto{\pgfqpoint{1.555641in}{2.989636in}}%
\pgfpathlineto{\pgfqpoint{1.557646in}{2.981455in}}%
\pgfpathlineto{\pgfqpoint{1.558648in}{2.985545in}}%
\pgfpathlineto{\pgfqpoint{1.559650in}{2.980091in}}%
\pgfpathlineto{\pgfqpoint{1.560652in}{2.992364in}}%
\pgfpathlineto{\pgfqpoint{1.561655in}{2.971909in}}%
\pgfpathlineto{\pgfqpoint{1.562657in}{2.978727in}}%
\pgfpathlineto{\pgfqpoint{1.564661in}{2.967818in}}%
\pgfpathlineto{\pgfqpoint{1.565663in}{2.982818in}}%
\pgfpathlineto{\pgfqpoint{1.567668in}{2.965091in}}%
\pgfpathlineto{\pgfqpoint{1.568670in}{2.973273in}}%
\pgfpathlineto{\pgfqpoint{1.569672in}{2.959636in}}%
\pgfpathlineto{\pgfqpoint{1.570675in}{2.980091in}}%
\pgfpathlineto{\pgfqpoint{1.571677in}{2.970545in}}%
\pgfpathlineto{\pgfqpoint{1.572679in}{2.981455in}}%
\pgfpathlineto{\pgfqpoint{1.573681in}{2.966455in}}%
\pgfpathlineto{\pgfqpoint{1.575686in}{2.982818in}}%
\pgfpathlineto{\pgfqpoint{1.576688in}{2.971909in}}%
\pgfpathlineto{\pgfqpoint{1.577690in}{2.988273in}}%
\pgfpathlineto{\pgfqpoint{1.579695in}{2.966455in}}%
\pgfpathlineto{\pgfqpoint{1.582702in}{2.999182in}}%
\pgfpathlineto{\pgfqpoint{1.584706in}{2.988273in}}%
\pgfpathlineto{\pgfqpoint{1.586711in}{2.986909in}}%
\pgfpathlineto{\pgfqpoint{1.588715in}{3.012818in}}%
\pgfpathlineto{\pgfqpoint{1.589717in}{3.004636in}}%
\pgfpathlineto{\pgfqpoint{1.591722in}{2.982818in}}%
\pgfpathlineto{\pgfqpoint{1.593726in}{3.012818in}}%
\pgfpathlineto{\pgfqpoint{1.594729in}{3.006000in}}%
\pgfpathlineto{\pgfqpoint{1.595731in}{2.978727in}}%
\pgfpathlineto{\pgfqpoint{1.597735in}{3.007364in}}%
\pgfpathlineto{\pgfqpoint{1.598737in}{3.021000in}}%
\pgfpathlineto{\pgfqpoint{1.601744in}{2.980091in}}%
\pgfpathlineto{\pgfqpoint{1.602746in}{2.991000in}}%
\pgfpathlineto{\pgfqpoint{1.603749in}{2.982818in}}%
\pgfpathlineto{\pgfqpoint{1.605753in}{2.988273in}}%
\pgfpathlineto{\pgfqpoint{1.606755in}{2.986909in}}%
\pgfpathlineto{\pgfqpoint{1.608760in}{2.962364in}}%
\pgfpathlineto{\pgfqpoint{1.609762in}{2.958273in}}%
\pgfpathlineto{\pgfqpoint{1.610764in}{2.981455in}}%
\pgfpathlineto{\pgfqpoint{1.611767in}{2.973273in}}%
\pgfpathlineto{\pgfqpoint{1.613771in}{2.950091in}}%
\pgfpathlineto{\pgfqpoint{1.614773in}{2.969182in}}%
\pgfpathlineto{\pgfqpoint{1.615776in}{2.965091in}}%
\pgfpathlineto{\pgfqpoint{1.616778in}{2.984182in}}%
\pgfpathlineto{\pgfqpoint{1.617780in}{2.950091in}}%
\pgfpathlineto{\pgfqpoint{1.620787in}{2.984182in}}%
\pgfpathlineto{\pgfqpoint{1.621789in}{2.978727in}}%
\pgfpathlineto{\pgfqpoint{1.622791in}{2.988273in}}%
\pgfpathlineto{\pgfqpoint{1.623794in}{2.980091in}}%
\pgfpathlineto{\pgfqpoint{1.624796in}{2.982818in}}%
\pgfpathlineto{\pgfqpoint{1.625798in}{2.980091in}}%
\pgfpathlineto{\pgfqpoint{1.628805in}{3.003273in}}%
\pgfpathlineto{\pgfqpoint{1.629807in}{3.001909in}}%
\pgfpathlineto{\pgfqpoint{1.630809in}{2.991000in}}%
\pgfpathlineto{\pgfqpoint{1.631812in}{2.995091in}}%
\pgfpathlineto{\pgfqpoint{1.632814in}{3.012818in}}%
\pgfpathlineto{\pgfqpoint{1.633816in}{3.010091in}}%
\pgfpathlineto{\pgfqpoint{1.635820in}{2.995091in}}%
\pgfpathlineto{\pgfqpoint{1.638827in}{3.018273in}}%
\pgfpathlineto{\pgfqpoint{1.640832in}{2.993727in}}%
\pgfpathlineto{\pgfqpoint{1.643838in}{3.010091in}}%
\pgfpathlineto{\pgfqpoint{1.644841in}{3.004636in}}%
\pgfpathlineto{\pgfqpoint{1.645843in}{2.981455in}}%
\pgfpathlineto{\pgfqpoint{1.646845in}{2.988273in}}%
\pgfpathlineto{\pgfqpoint{1.647847in}{2.977364in}}%
\pgfpathlineto{\pgfqpoint{1.648850in}{2.996455in}}%
\pgfpathlineto{\pgfqpoint{1.650854in}{2.984182in}}%
\pgfpathlineto{\pgfqpoint{1.651856in}{2.965091in}}%
\pgfpathlineto{\pgfqpoint{1.654863in}{2.984182in}}%
\pgfpathlineto{\pgfqpoint{1.655865in}{2.982818in}}%
\pgfpathlineto{\pgfqpoint{1.656868in}{2.980091in}}%
\pgfpathlineto{\pgfqpoint{1.657870in}{2.958273in}}%
\pgfpathlineto{\pgfqpoint{1.658872in}{2.974636in}}%
\pgfpathlineto{\pgfqpoint{1.659874in}{2.971909in}}%
\pgfpathlineto{\pgfqpoint{1.660877in}{2.977364in}}%
\pgfpathlineto{\pgfqpoint{1.661879in}{2.974636in}}%
\pgfpathlineto{\pgfqpoint{1.663883in}{2.948727in}}%
\pgfpathlineto{\pgfqpoint{1.665888in}{2.978727in}}%
\pgfpathlineto{\pgfqpoint{1.666890in}{2.976000in}}%
\pgfpathlineto{\pgfqpoint{1.667892in}{2.969182in}}%
\pgfpathlineto{\pgfqpoint{1.668895in}{2.978727in}}%
\pgfpathlineto{\pgfqpoint{1.669897in}{2.969182in}}%
\pgfpathlineto{\pgfqpoint{1.670899in}{2.986909in}}%
\pgfpathlineto{\pgfqpoint{1.671901in}{2.982818in}}%
\pgfpathlineto{\pgfqpoint{1.672903in}{2.985545in}}%
\pgfpathlineto{\pgfqpoint{1.673906in}{2.992364in}}%
\pgfpathlineto{\pgfqpoint{1.676912in}{2.977364in}}%
\pgfpathlineto{\pgfqpoint{1.678917in}{3.012818in}}%
\pgfpathlineto{\pgfqpoint{1.680921in}{2.985545in}}%
\pgfpathlineto{\pgfqpoint{1.681924in}{2.992364in}}%
\pgfpathlineto{\pgfqpoint{1.682926in}{3.016909in}}%
\pgfpathlineto{\pgfqpoint{1.683928in}{3.011455in}}%
\pgfpathlineto{\pgfqpoint{1.685933in}{2.991000in}}%
\pgfpathlineto{\pgfqpoint{1.687937in}{3.001909in}}%
\pgfpathlineto{\pgfqpoint{1.688939in}{3.018273in}}%
\pgfpathlineto{\pgfqpoint{1.689942in}{2.996455in}}%
\pgfpathlineto{\pgfqpoint{1.690944in}{2.999182in}}%
\pgfpathlineto{\pgfqpoint{1.691946in}{2.984182in}}%
\pgfpathlineto{\pgfqpoint{1.693951in}{3.006000in}}%
\pgfpathlineto{\pgfqpoint{1.694953in}{2.988273in}}%
\pgfpathlineto{\pgfqpoint{1.695955in}{2.996455in}}%
\pgfpathlineto{\pgfqpoint{1.697960in}{2.973273in}}%
\pgfpathlineto{\pgfqpoint{1.698962in}{2.991000in}}%
\pgfpathlineto{\pgfqpoint{1.699964in}{2.984182in}}%
\pgfpathlineto{\pgfqpoint{1.700966in}{2.986909in}}%
\pgfpathlineto{\pgfqpoint{1.702971in}{2.961000in}}%
\pgfpathlineto{\pgfqpoint{1.703973in}{2.954182in}}%
\pgfpathlineto{\pgfqpoint{1.705977in}{2.977364in}}%
\pgfpathlineto{\pgfqpoint{1.707982in}{2.941909in}}%
\pgfpathlineto{\pgfqpoint{1.710989in}{2.974636in}}%
\pgfpathlineto{\pgfqpoint{1.713995in}{2.946000in}}%
\pgfpathlineto{\pgfqpoint{1.716000in}{2.977364in}}%
\pgfpathlineto{\pgfqpoint{1.717002in}{2.973273in}}%
\pgfpathlineto{\pgfqpoint{1.718004in}{3.003273in}}%
\pgfpathlineto{\pgfqpoint{1.719007in}{2.974636in}}%
\pgfpathlineto{\pgfqpoint{1.720009in}{2.977364in}}%
\pgfpathlineto{\pgfqpoint{1.722013in}{2.991000in}}%
\pgfpathlineto{\pgfqpoint{1.724018in}{3.012818in}}%
\pgfpathlineto{\pgfqpoint{1.726022in}{2.991000in}}%
\pgfpathlineto{\pgfqpoint{1.727025in}{3.004636in}}%
\pgfpathlineto{\pgfqpoint{1.728027in}{3.038727in}}%
\pgfpathlineto{\pgfqpoint{1.729029in}{3.029182in}}%
\pgfpathlineto{\pgfqpoint{1.731034in}{3.000545in}}%
\pgfpathlineto{\pgfqpoint{1.733038in}{3.036000in}}%
\pgfpathlineto{\pgfqpoint{1.734040in}{3.031909in}}%
\pgfpathlineto{\pgfqpoint{1.736045in}{2.988273in}}%
\pgfpathlineto{\pgfqpoint{1.738049in}{3.018273in}}%
\pgfpathlineto{\pgfqpoint{1.741056in}{2.981455in}}%
\pgfpathlineto{\pgfqpoint{1.742058in}{2.981455in}}%
\pgfpathlineto{\pgfqpoint{1.743060in}{2.984182in}}%
\pgfpathlineto{\pgfqpoint{1.744063in}{2.956909in}}%
\pgfpathlineto{\pgfqpoint{1.745065in}{2.980091in}}%
\pgfpathlineto{\pgfqpoint{1.746067in}{2.974636in}}%
\pgfpathlineto{\pgfqpoint{1.749074in}{2.937818in}}%
\pgfpathlineto{\pgfqpoint{1.751078in}{2.981455in}}%
\pgfpathlineto{\pgfqpoint{1.754085in}{2.926909in}}%
\pgfpathlineto{\pgfqpoint{1.756090in}{2.966455in}}%
\pgfpathlineto{\pgfqpoint{1.757092in}{2.982818in}}%
\pgfpathlineto{\pgfqpoint{1.759096in}{2.947364in}}%
\pgfpathlineto{\pgfqpoint{1.761101in}{2.976000in}}%
\pgfpathlineto{\pgfqpoint{1.762103in}{2.991000in}}%
\pgfpathlineto{\pgfqpoint{1.763105in}{2.980091in}}%
\pgfpathlineto{\pgfqpoint{1.765110in}{2.996455in}}%
\pgfpathlineto{\pgfqpoint{1.766112in}{2.996455in}}%
\pgfpathlineto{\pgfqpoint{1.769119in}{3.019636in}}%
\pgfpathlineto{\pgfqpoint{1.771123in}{2.989636in}}%
\pgfpathlineto{\pgfqpoint{1.773128in}{3.049636in}}%
\pgfpathlineto{\pgfqpoint{1.776134in}{2.984182in}}%
\pgfpathlineto{\pgfqpoint{1.777137in}{2.997818in}}%
\pgfpathlineto{\pgfqpoint{1.779141in}{3.025091in}}%
\pgfpathlineto{\pgfqpoint{1.781146in}{2.973273in}}%
\pgfpathlineto{\pgfqpoint{1.784152in}{3.006000in}}%
\pgfpathlineto{\pgfqpoint{1.786157in}{2.970545in}}%
\pgfpathlineto{\pgfqpoint{1.787159in}{2.971909in}}%
\pgfpathlineto{\pgfqpoint{1.788161in}{2.951455in}}%
\pgfpathlineto{\pgfqpoint{1.790166in}{2.977364in}}%
\pgfpathlineto{\pgfqpoint{1.792170in}{2.959636in}}%
\pgfpathlineto{\pgfqpoint{1.794175in}{2.928273in}}%
\pgfpathlineto{\pgfqpoint{1.795177in}{2.970545in}}%
\pgfpathlineto{\pgfqpoint{1.796179in}{2.969182in}}%
\pgfpathlineto{\pgfqpoint{1.798184in}{2.931000in}}%
\pgfpathlineto{\pgfqpoint{1.801191in}{2.969182in}}%
\pgfpathlineto{\pgfqpoint{1.802193in}{2.951455in}}%
\pgfpathlineto{\pgfqpoint{1.804197in}{2.965091in}}%
\pgfpathlineto{\pgfqpoint{1.805200in}{2.961000in}}%
\pgfpathlineto{\pgfqpoint{1.806202in}{2.996455in}}%
\pgfpathlineto{\pgfqpoint{1.807204in}{2.978727in}}%
\pgfpathlineto{\pgfqpoint{1.808206in}{2.980091in}}%
\pgfpathlineto{\pgfqpoint{1.809209in}{2.982818in}}%
\pgfpathlineto{\pgfqpoint{1.810211in}{2.967818in}}%
\pgfpathlineto{\pgfqpoint{1.813217in}{3.010091in}}%
\pgfpathlineto{\pgfqpoint{1.814220in}{3.031909in}}%
\pgfpathlineto{\pgfqpoint{1.815222in}{2.993727in}}%
\pgfpathlineto{\pgfqpoint{1.816224in}{2.999182in}}%
\pgfpathlineto{\pgfqpoint{1.817226in}{2.997818in}}%
\pgfpathlineto{\pgfqpoint{1.819231in}{3.034636in}}%
\pgfpathlineto{\pgfqpoint{1.821235in}{2.992364in}}%
\pgfpathlineto{\pgfqpoint{1.822238in}{2.993727in}}%
\pgfpathlineto{\pgfqpoint{1.823240in}{3.023727in}}%
\pgfpathlineto{\pgfqpoint{1.824242in}{3.015545in}}%
\pgfpathlineto{\pgfqpoint{1.825244in}{2.995091in}}%
\pgfpathlineto{\pgfqpoint{1.826247in}{2.997818in}}%
\pgfpathlineto{\pgfqpoint{1.827249in}{2.961000in}}%
\pgfpathlineto{\pgfqpoint{1.828251in}{3.015545in}}%
\pgfpathlineto{\pgfqpoint{1.830256in}{2.959636in}}%
\pgfpathlineto{\pgfqpoint{1.831258in}{2.991000in}}%
\pgfpathlineto{\pgfqpoint{1.832260in}{2.965091in}}%
\pgfpathlineto{\pgfqpoint{1.833262in}{2.974636in}}%
\pgfpathlineto{\pgfqpoint{1.834265in}{2.943273in}}%
\pgfpathlineto{\pgfqpoint{1.836269in}{2.970545in}}%
\pgfpathlineto{\pgfqpoint{1.838274in}{2.924182in}}%
\pgfpathlineto{\pgfqpoint{1.839276in}{2.921455in}}%
\pgfpathlineto{\pgfqpoint{1.841280in}{2.969182in}}%
\pgfpathlineto{\pgfqpoint{1.843285in}{2.903727in}}%
\pgfpathlineto{\pgfqpoint{1.844287in}{2.913273in}}%
\pgfpathlineto{\pgfqpoint{1.846291in}{2.973273in}}%
\pgfpathlineto{\pgfqpoint{1.848296in}{2.941909in}}%
\pgfpathlineto{\pgfqpoint{1.849298in}{2.936455in}}%
\pgfpathlineto{\pgfqpoint{1.851303in}{2.989636in}}%
\pgfpathlineto{\pgfqpoint{1.852305in}{2.993727in}}%
\pgfpathlineto{\pgfqpoint{1.853307in}{2.991000in}}%
\pgfpathlineto{\pgfqpoint{1.854309in}{2.958273in}}%
\pgfpathlineto{\pgfqpoint{1.858318in}{3.038727in}}%
\pgfpathlineto{\pgfqpoint{1.861325in}{2.962364in}}%
\pgfpathlineto{\pgfqpoint{1.863330in}{3.036000in}}%
\pgfpathlineto{\pgfqpoint{1.864332in}{3.027818in}}%
\pgfpathlineto{\pgfqpoint{1.866336in}{2.985545in}}%
\pgfpathlineto{\pgfqpoint{1.868341in}{3.014182in}}%
\pgfpathlineto{\pgfqpoint{1.869343in}{3.026455in}}%
\pgfpathlineto{\pgfqpoint{1.871348in}{2.977364in}}%
\pgfpathlineto{\pgfqpoint{1.873352in}{3.007364in}}%
\pgfpathlineto{\pgfqpoint{1.874354in}{3.004636in}}%
\pgfpathlineto{\pgfqpoint{1.875357in}{2.995091in}}%
\pgfpathlineto{\pgfqpoint{1.876359in}{2.959636in}}%
\pgfpathlineto{\pgfqpoint{1.877361in}{2.991000in}}%
\pgfpathlineto{\pgfqpoint{1.879366in}{2.954182in}}%
\pgfpathlineto{\pgfqpoint{1.880368in}{2.980091in}}%
\pgfpathlineto{\pgfqpoint{1.881370in}{2.959636in}}%
\pgfpathlineto{\pgfqpoint{1.882372in}{2.967818in}}%
\pgfpathlineto{\pgfqpoint{1.884377in}{2.935091in}}%
\pgfpathlineto{\pgfqpoint{1.886381in}{2.974636in}}%
\pgfpathlineto{\pgfqpoint{1.888386in}{2.913273in}}%
\pgfpathlineto{\pgfqpoint{1.889388in}{2.913273in}}%
\pgfpathlineto{\pgfqpoint{1.891392in}{2.970545in}}%
\pgfpathlineto{\pgfqpoint{1.894399in}{2.905091in}}%
\pgfpathlineto{\pgfqpoint{1.896404in}{2.966455in}}%
\pgfpathlineto{\pgfqpoint{1.897406in}{2.966455in}}%
\pgfpathlineto{\pgfqpoint{1.898408in}{2.914636in}}%
\pgfpathlineto{\pgfqpoint{1.900413in}{2.963727in}}%
\pgfpathlineto{\pgfqpoint{1.901415in}{2.969182in}}%
\pgfpathlineto{\pgfqpoint{1.902417in}{2.992364in}}%
\pgfpathlineto{\pgfqpoint{1.903419in}{2.965091in}}%
\pgfpathlineto{\pgfqpoint{1.905424in}{3.000545in}}%
\pgfpathlineto{\pgfqpoint{1.906426in}{2.995091in}}%
\pgfpathlineto{\pgfqpoint{1.907428in}{3.042818in}}%
\pgfpathlineto{\pgfqpoint{1.908431in}{3.029182in}}%
\pgfpathlineto{\pgfqpoint{1.909433in}{3.034636in}}%
\pgfpathlineto{\pgfqpoint{1.911437in}{2.992364in}}%
\pgfpathlineto{\pgfqpoint{1.913442in}{3.056455in}}%
\pgfpathlineto{\pgfqpoint{1.914444in}{3.053727in}}%
\pgfpathlineto{\pgfqpoint{1.915446in}{3.040091in}}%
\pgfpathlineto{\pgfqpoint{1.916449in}{3.007364in}}%
\pgfpathlineto{\pgfqpoint{1.919455in}{3.061909in}}%
\pgfpathlineto{\pgfqpoint{1.921460in}{2.985545in}}%
\pgfpathlineto{\pgfqpoint{1.922462in}{2.993727in}}%
\pgfpathlineto{\pgfqpoint{1.924466in}{3.021000in}}%
\pgfpathlineto{\pgfqpoint{1.926471in}{2.963727in}}%
\pgfpathlineto{\pgfqpoint{1.927473in}{2.962364in}}%
\pgfpathlineto{\pgfqpoint{1.928475in}{2.937818in}}%
\pgfpathlineto{\pgfqpoint{1.929478in}{2.962364in}}%
\pgfpathlineto{\pgfqpoint{1.930480in}{2.961000in}}%
\pgfpathlineto{\pgfqpoint{1.931482in}{2.952818in}}%
\pgfpathlineto{\pgfqpoint{1.933487in}{2.898273in}}%
\pgfpathlineto{\pgfqpoint{1.934489in}{2.902364in}}%
\pgfpathlineto{\pgfqpoint{1.935491in}{2.951455in}}%
\pgfpathlineto{\pgfqpoint{1.938498in}{2.903727in}}%
\pgfpathlineto{\pgfqpoint{1.939500in}{2.921455in}}%
\pgfpathlineto{\pgfqpoint{1.940502in}{2.976000in}}%
\pgfpathlineto{\pgfqpoint{1.941505in}{2.969182in}}%
\pgfpathlineto{\pgfqpoint{1.943509in}{2.926909in}}%
\pgfpathlineto{\pgfqpoint{1.944511in}{2.932364in}}%
\pgfpathlineto{\pgfqpoint{1.945514in}{2.948727in}}%
\pgfpathlineto{\pgfqpoint{1.946516in}{2.986909in}}%
\pgfpathlineto{\pgfqpoint{1.947518in}{2.982818in}}%
\pgfpathlineto{\pgfqpoint{1.948520in}{3.006000in}}%
\pgfpathlineto{\pgfqpoint{1.949523in}{2.992364in}}%
\pgfpathlineto{\pgfqpoint{1.950525in}{3.003273in}}%
\pgfpathlineto{\pgfqpoint{1.951527in}{2.999182in}}%
\pgfpathlineto{\pgfqpoint{1.953531in}{2.955545in}}%
\pgfpathlineto{\pgfqpoint{1.954534in}{3.007364in}}%
\pgfpathlineto{\pgfqpoint{1.955536in}{3.004636in}}%
\pgfpathlineto{\pgfqpoint{1.956538in}{2.974636in}}%
\pgfpathlineto{\pgfqpoint{1.957540in}{3.025091in}}%
\pgfpathlineto{\pgfqpoint{1.958543in}{2.985545in}}%
\pgfpathlineto{\pgfqpoint{1.960547in}{3.010091in}}%
\pgfpathlineto{\pgfqpoint{1.962552in}{2.950091in}}%
\pgfpathlineto{\pgfqpoint{1.964556in}{3.063273in}}%
\pgfpathlineto{\pgfqpoint{1.966561in}{2.980091in}}%
\pgfpathlineto{\pgfqpoint{1.968565in}{3.026455in}}%
\pgfpathlineto{\pgfqpoint{1.970570in}{2.948727in}}%
\pgfpathlineto{\pgfqpoint{1.971572in}{2.947364in}}%
\pgfpathlineto{\pgfqpoint{1.972574in}{2.973273in}}%
\pgfpathlineto{\pgfqpoint{1.974579in}{2.895545in}}%
\pgfpathlineto{\pgfqpoint{1.976583in}{2.976000in}}%
\pgfpathlineto{\pgfqpoint{1.977585in}{2.955545in}}%
\pgfpathlineto{\pgfqpoint{1.978588in}{2.886000in}}%
\pgfpathlineto{\pgfqpoint{1.980592in}{2.932364in}}%
\pgfpathlineto{\pgfqpoint{1.981594in}{2.993727in}}%
\pgfpathlineto{\pgfqpoint{1.983599in}{2.911909in}}%
\pgfpathlineto{\pgfqpoint{1.984601in}{2.903727in}}%
\pgfpathlineto{\pgfqpoint{1.986606in}{2.961000in}}%
\pgfpathlineto{\pgfqpoint{1.987608in}{2.937818in}}%
\pgfpathlineto{\pgfqpoint{1.988610in}{2.965091in}}%
\pgfpathlineto{\pgfqpoint{1.989612in}{2.922818in}}%
\pgfpathlineto{\pgfqpoint{1.991617in}{2.952818in}}%
\pgfpathlineto{\pgfqpoint{1.992619in}{3.019636in}}%
\pgfpathlineto{\pgfqpoint{1.993621in}{3.006000in}}%
\pgfpathlineto{\pgfqpoint{1.994623in}{2.961000in}}%
\pgfpathlineto{\pgfqpoint{1.995626in}{2.962364in}}%
\pgfpathlineto{\pgfqpoint{1.998632in}{3.064636in}}%
\pgfpathlineto{\pgfqpoint{1.999635in}{2.996455in}}%
\pgfpathlineto{\pgfqpoint{2.000637in}{3.003273in}}%
\pgfpathlineto{\pgfqpoint{2.001639in}{2.997818in}}%
\pgfpathlineto{\pgfqpoint{2.003644in}{3.081000in}}%
\pgfpathlineto{\pgfqpoint{2.004646in}{3.071455in}}%
\pgfpathlineto{\pgfqpoint{2.005648in}{3.044182in}}%
\pgfpathlineto{\pgfqpoint{2.006650in}{2.974636in}}%
\pgfpathlineto{\pgfqpoint{2.008655in}{3.121909in}}%
\pgfpathlineto{\pgfqpoint{2.011662in}{2.963727in}}%
\pgfpathlineto{\pgfqpoint{2.013666in}{3.037364in}}%
\pgfpathlineto{\pgfqpoint{2.016673in}{2.961000in}}%
\pgfpathlineto{\pgfqpoint{2.017675in}{2.982818in}}%
\pgfpathlineto{\pgfqpoint{2.019680in}{2.917364in}}%
\pgfpathlineto{\pgfqpoint{2.020682in}{2.946000in}}%
\pgfpathlineto{\pgfqpoint{2.021684in}{2.943273in}}%
\pgfpathlineto{\pgfqpoint{2.024691in}{2.883273in}}%
\pgfpathlineto{\pgfqpoint{2.026695in}{2.943273in}}%
\pgfpathlineto{\pgfqpoint{2.028700in}{2.892818in}}%
\pgfpathlineto{\pgfqpoint{2.029702in}{2.899636in}}%
\pgfpathlineto{\pgfqpoint{2.031706in}{2.958273in}}%
\pgfpathlineto{\pgfqpoint{2.033711in}{2.906455in}}%
\pgfpathlineto{\pgfqpoint{2.036718in}{2.984182in}}%
\pgfpathlineto{\pgfqpoint{2.037720in}{2.981455in}}%
\pgfpathlineto{\pgfqpoint{2.038722in}{2.903727in}}%
\pgfpathlineto{\pgfqpoint{2.040727in}{2.944636in}}%
\pgfpathlineto{\pgfqpoint{2.041729in}{2.977364in}}%
\pgfpathlineto{\pgfqpoint{2.042731in}{3.057818in}}%
\pgfpathlineto{\pgfqpoint{2.043733in}{3.046909in}}%
\pgfpathlineto{\pgfqpoint{2.045738in}{3.007364in}}%
\pgfpathlineto{\pgfqpoint{2.046740in}{2.991000in}}%
\pgfpathlineto{\pgfqpoint{2.048745in}{3.102818in}}%
\pgfpathlineto{\pgfqpoint{2.051751in}{3.018273in}}%
\pgfpathlineto{\pgfqpoint{2.053756in}{3.078273in}}%
\pgfpathlineto{\pgfqpoint{2.054758in}{3.098727in}}%
\pgfpathlineto{\pgfqpoint{2.056763in}{2.943273in}}%
\pgfpathlineto{\pgfqpoint{2.057765in}{2.995091in}}%
\pgfpathlineto{\pgfqpoint{2.058767in}{3.117818in}}%
\pgfpathlineto{\pgfqpoint{2.059769in}{3.113727in}}%
\pgfpathlineto{\pgfqpoint{2.061774in}{2.997818in}}%
\pgfpathlineto{\pgfqpoint{2.062776in}{2.941909in}}%
\pgfpathlineto{\pgfqpoint{2.063778in}{2.950091in}}%
\pgfpathlineto{\pgfqpoint{2.064780in}{3.004636in}}%
\pgfpathlineto{\pgfqpoint{2.067787in}{2.941909in}}%
\pgfpathlineto{\pgfqpoint{2.068789in}{2.865545in}}%
\pgfpathlineto{\pgfqpoint{2.070794in}{2.936455in}}%
\pgfpathlineto{\pgfqpoint{2.071796in}{2.937818in}}%
\pgfpathlineto{\pgfqpoint{2.073801in}{2.854636in}}%
\pgfpathlineto{\pgfqpoint{2.074803in}{2.861455in}}%
\pgfpathlineto{\pgfqpoint{2.076807in}{2.939182in}}%
\pgfpathlineto{\pgfqpoint{2.078812in}{2.836909in}}%
\pgfpathlineto{\pgfqpoint{2.081819in}{2.936455in}}%
\pgfpathlineto{\pgfqpoint{2.083823in}{2.911909in}}%
\pgfpathlineto{\pgfqpoint{2.084825in}{2.913273in}}%
\pgfpathlineto{\pgfqpoint{2.086830in}{2.974636in}}%
\pgfpathlineto{\pgfqpoint{2.087832in}{2.944636in}}%
\pgfpathlineto{\pgfqpoint{2.091841in}{3.027818in}}%
\pgfpathlineto{\pgfqpoint{2.092843in}{3.018273in}}%
\pgfpathlineto{\pgfqpoint{2.093846in}{3.022364in}}%
\pgfpathlineto{\pgfqpoint{2.094848in}{3.090545in}}%
\pgfpathlineto{\pgfqpoint{2.096852in}{3.023727in}}%
\pgfpathlineto{\pgfqpoint{2.099859in}{3.181909in}}%
\pgfpathlineto{\pgfqpoint{2.100861in}{3.022364in}}%
\pgfpathlineto{\pgfqpoint{2.101863in}{3.023727in}}%
\pgfpathlineto{\pgfqpoint{2.103868in}{3.124636in}}%
\pgfpathlineto{\pgfqpoint{2.104870in}{3.115091in}}%
\pgfpathlineto{\pgfqpoint{2.106875in}{3.023727in}}%
\pgfpathlineto{\pgfqpoint{2.107877in}{3.037364in}}%
\pgfpathlineto{\pgfqpoint{2.109881in}{2.980091in}}%
\pgfpathlineto{\pgfqpoint{2.110884in}{2.988273in}}%
\pgfpathlineto{\pgfqpoint{2.111886in}{2.999182in}}%
\pgfpathlineto{\pgfqpoint{2.112888in}{2.939182in}}%
\pgfpathlineto{\pgfqpoint{2.113890in}{2.963727in}}%
\pgfpathlineto{\pgfqpoint{2.115895in}{2.947364in}}%
\pgfpathlineto{\pgfqpoint{2.116897in}{2.941909in}}%
\pgfpathlineto{\pgfqpoint{2.118902in}{2.864182in}}%
\pgfpathlineto{\pgfqpoint{2.120906in}{2.925545in}}%
\pgfpathlineto{\pgfqpoint{2.121908in}{2.925545in}}%
\pgfpathlineto{\pgfqpoint{2.123913in}{2.850545in}}%
\pgfpathlineto{\pgfqpoint{2.124915in}{2.862818in}}%
\pgfpathlineto{\pgfqpoint{2.126920in}{2.921455in}}%
\pgfpathlineto{\pgfqpoint{2.128924in}{2.868273in}}%
\pgfpathlineto{\pgfqpoint{2.130928in}{2.962364in}}%
\pgfpathlineto{\pgfqpoint{2.131931in}{2.959636in}}%
\pgfpathlineto{\pgfqpoint{2.132933in}{2.935091in}}%
\pgfpathlineto{\pgfqpoint{2.134937in}{2.961000in}}%
\pgfpathlineto{\pgfqpoint{2.136942in}{3.018273in}}%
\pgfpathlineto{\pgfqpoint{2.137944in}{3.014182in}}%
\pgfpathlineto{\pgfqpoint{2.138946in}{3.072818in}}%
\pgfpathlineto{\pgfqpoint{2.139949in}{3.063273in}}%
\pgfpathlineto{\pgfqpoint{2.140951in}{3.040091in}}%
\pgfpathlineto{\pgfqpoint{2.142955in}{3.094636in}}%
\pgfpathlineto{\pgfqpoint{2.143958in}{3.304636in}}%
\pgfpathlineto{\pgfqpoint{2.145962in}{3.052364in}}%
\pgfpathlineto{\pgfqpoint{2.146964in}{3.037364in}}%
\pgfpathlineto{\pgfqpoint{2.148969in}{3.258273in}}%
\pgfpathlineto{\pgfqpoint{2.150973in}{3.025091in}}%
\pgfpathlineto{\pgfqpoint{2.151976in}{3.014182in}}%
\pgfpathlineto{\pgfqpoint{2.153980in}{3.079636in}}%
\pgfpathlineto{\pgfqpoint{2.158991in}{2.946000in}}%
\pgfpathlineto{\pgfqpoint{2.159994in}{2.944636in}}%
\pgfpathlineto{\pgfqpoint{2.160996in}{2.958273in}}%
\pgfpathlineto{\pgfqpoint{2.161998in}{2.937818in}}%
\pgfpathlineto{\pgfqpoint{2.164003in}{2.872364in}}%
\pgfpathlineto{\pgfqpoint{2.165005in}{2.880545in}}%
\pgfpathlineto{\pgfqpoint{2.166007in}{2.931000in}}%
\pgfpathlineto{\pgfqpoint{2.167009in}{2.916000in}}%
\pgfpathlineto{\pgfqpoint{2.169014in}{2.830091in}}%
\pgfpathlineto{\pgfqpoint{2.170016in}{2.846455in}}%
\pgfpathlineto{\pgfqpoint{2.172020in}{2.921455in}}%
\pgfpathlineto{\pgfqpoint{2.174025in}{2.851909in}}%
\pgfpathlineto{\pgfqpoint{2.175027in}{2.858727in}}%
\pgfpathlineto{\pgfqpoint{2.177032in}{2.946000in}}%
\pgfpathlineto{\pgfqpoint{2.179036in}{2.896909in}}%
\pgfpathlineto{\pgfqpoint{2.181041in}{2.989636in}}%
\pgfpathlineto{\pgfqpoint{2.182043in}{2.991000in}}%
\pgfpathlineto{\pgfqpoint{2.183045in}{3.004636in}}%
\pgfpathlineto{\pgfqpoint{2.184047in}{2.982818in}}%
\pgfpathlineto{\pgfqpoint{2.186052in}{3.025091in}}%
\pgfpathlineto{\pgfqpoint{2.187054in}{3.019636in}}%
\pgfpathlineto{\pgfqpoint{2.189059in}{3.119182in}}%
\pgfpathlineto{\pgfqpoint{2.192065in}{3.052364in}}%
\pgfpathlineto{\pgfqpoint{2.194070in}{3.382364in}}%
\pgfpathlineto{\pgfqpoint{2.197077in}{3.063273in}}%
\pgfpathlineto{\pgfqpoint{2.199081in}{3.255545in}}%
\pgfpathlineto{\pgfqpoint{2.201085in}{3.044182in}}%
\pgfpathlineto{\pgfqpoint{2.202088in}{3.052364in}}%
\pgfpathlineto{\pgfqpoint{2.203090in}{3.085091in}}%
\pgfpathlineto{\pgfqpoint{2.207099in}{2.963727in}}%
\pgfpathlineto{\pgfqpoint{2.208101in}{2.993727in}}%
\pgfpathlineto{\pgfqpoint{2.209103in}{2.962364in}}%
\pgfpathlineto{\pgfqpoint{2.210106in}{2.976000in}}%
\pgfpathlineto{\pgfqpoint{2.211108in}{2.952818in}}%
\pgfpathlineto{\pgfqpoint{2.214115in}{2.832818in}}%
\pgfpathlineto{\pgfqpoint{2.216119in}{2.922818in}}%
\pgfpathlineto{\pgfqpoint{2.218124in}{2.847818in}}%
\pgfpathlineto{\pgfqpoint{2.219126in}{2.790545in}}%
\pgfpathlineto{\pgfqpoint{2.221130in}{2.907818in}}%
\pgfpathlineto{\pgfqpoint{2.222133in}{2.920091in}}%
\pgfpathlineto{\pgfqpoint{2.224137in}{2.845091in}}%
\pgfpathlineto{\pgfqpoint{2.227144in}{2.951455in}}%
\pgfpathlineto{\pgfqpoint{2.228146in}{2.890091in}}%
\pgfpathlineto{\pgfqpoint{2.229148in}{2.894182in}}%
\pgfpathlineto{\pgfqpoint{2.232155in}{3.003273in}}%
\pgfpathlineto{\pgfqpoint{2.233157in}{3.001909in}}%
\pgfpathlineto{\pgfqpoint{2.234160in}{3.006000in}}%
\pgfpathlineto{\pgfqpoint{2.235162in}{3.034636in}}%
\pgfpathlineto{\pgfqpoint{2.236164in}{3.010091in}}%
\pgfpathlineto{\pgfqpoint{2.237166in}{3.037364in}}%
\pgfpathlineto{\pgfqpoint{2.240173in}{3.251455in}}%
\pgfpathlineto{\pgfqpoint{2.241175in}{3.067364in}}%
\pgfpathlineto{\pgfqpoint{2.242177in}{3.098727in}}%
\pgfpathlineto{\pgfqpoint{2.244182in}{3.461455in}}%
\pgfpathlineto{\pgfqpoint{2.245184in}{3.420545in}}%
\pgfpathlineto{\pgfqpoint{2.247189in}{3.089182in}}%
\pgfpathlineto{\pgfqpoint{2.249193in}{3.524182in}}%
\pgfpathlineto{\pgfqpoint{2.251198in}{3.031909in}}%
\pgfpathlineto{\pgfqpoint{2.253202in}{3.025091in}}%
\pgfpathlineto{\pgfqpoint{2.254204in}{3.115091in}}%
\pgfpathlineto{\pgfqpoint{2.259216in}{2.907818in}}%
\pgfpathlineto{\pgfqpoint{2.261220in}{2.917364in}}%
\pgfpathlineto{\pgfqpoint{2.262222in}{2.947364in}}%
\pgfpathlineto{\pgfqpoint{2.264227in}{2.835545in}}%
\pgfpathlineto{\pgfqpoint{2.267234in}{2.895545in}}%
\pgfpathlineto{\pgfqpoint{2.269238in}{2.813727in}}%
\pgfpathlineto{\pgfqpoint{2.271242in}{2.901000in}}%
\pgfpathlineto{\pgfqpoint{2.272245in}{2.917364in}}%
\pgfpathlineto{\pgfqpoint{2.274249in}{2.836909in}}%
\pgfpathlineto{\pgfqpoint{2.276254in}{2.921455in}}%
\pgfpathlineto{\pgfqpoint{2.277256in}{2.922818in}}%
\pgfpathlineto{\pgfqpoint{2.279260in}{2.898273in}}%
\pgfpathlineto{\pgfqpoint{2.280263in}{2.913273in}}%
\pgfpathlineto{\pgfqpoint{2.282267in}{2.991000in}}%
\pgfpathlineto{\pgfqpoint{2.283269in}{2.966455in}}%
\pgfpathlineto{\pgfqpoint{2.284272in}{3.051000in}}%
\pgfpathlineto{\pgfqpoint{2.286276in}{3.012818in}}%
\pgfpathlineto{\pgfqpoint{2.287278in}{3.044182in}}%
\pgfpathlineto{\pgfqpoint{2.288281in}{3.034636in}}%
\pgfpathlineto{\pgfqpoint{2.289283in}{3.177818in}}%
\pgfpathlineto{\pgfqpoint{2.291287in}{3.070091in}}%
\pgfpathlineto{\pgfqpoint{2.292290in}{3.090545in}}%
\pgfpathlineto{\pgfqpoint{2.293292in}{3.192818in}}%
\pgfpathlineto{\pgfqpoint{2.294294in}{3.503727in}}%
\pgfpathlineto{\pgfqpoint{2.296299in}{3.086455in}}%
\pgfpathlineto{\pgfqpoint{2.297301in}{3.127364in}}%
\pgfpathlineto{\pgfqpoint{2.299305in}{3.529636in}}%
\pgfpathlineto{\pgfqpoint{2.300308in}{3.405545in}}%
\pgfpathlineto{\pgfqpoint{2.302312in}{3.078273in}}%
\pgfpathlineto{\pgfqpoint{2.304317in}{3.173727in}}%
\pgfpathlineto{\pgfqpoint{2.307323in}{2.978727in}}%
\pgfpathlineto{\pgfqpoint{2.308325in}{3.031909in}}%
\pgfpathlineto{\pgfqpoint{2.309328in}{3.021000in}}%
\pgfpathlineto{\pgfqpoint{2.312334in}{2.914636in}}%
\pgfpathlineto{\pgfqpoint{2.313337in}{2.898273in}}%
\pgfpathlineto{\pgfqpoint{2.314339in}{2.924182in}}%
\pgfpathlineto{\pgfqpoint{2.315341in}{2.888727in}}%
\pgfpathlineto{\pgfqpoint{2.316343in}{2.935091in}}%
\pgfpathlineto{\pgfqpoint{2.319350in}{2.816455in}}%
\pgfpathlineto{\pgfqpoint{2.321355in}{2.907818in}}%
\pgfpathlineto{\pgfqpoint{2.322357in}{2.888727in}}%
\pgfpathlineto{\pgfqpoint{2.324361in}{2.820545in}}%
\pgfpathlineto{\pgfqpoint{2.325364in}{2.857364in}}%
\pgfpathlineto{\pgfqpoint{2.326366in}{2.937818in}}%
\pgfpathlineto{\pgfqpoint{2.327368in}{2.914636in}}%
\pgfpathlineto{\pgfqpoint{2.329373in}{2.839636in}}%
\pgfpathlineto{\pgfqpoint{2.331377in}{2.991000in}}%
\pgfpathlineto{\pgfqpoint{2.334384in}{2.905091in}}%
\pgfpathlineto{\pgfqpoint{2.337391in}{3.021000in}}%
\pgfpathlineto{\pgfqpoint{2.338393in}{3.040091in}}%
\pgfpathlineto{\pgfqpoint{2.339395in}{3.022364in}}%
\pgfpathlineto{\pgfqpoint{2.340397in}{3.031909in}}%
\pgfpathlineto{\pgfqpoint{2.341400in}{3.064636in}}%
\pgfpathlineto{\pgfqpoint{2.342402in}{3.055091in}}%
\pgfpathlineto{\pgfqpoint{2.344406in}{3.472364in}}%
\pgfpathlineto{\pgfqpoint{2.346411in}{3.168273in}}%
\pgfpathlineto{\pgfqpoint{2.347413in}{3.184636in}}%
\pgfpathlineto{\pgfqpoint{2.349417in}{3.655091in}}%
\pgfpathlineto{\pgfqpoint{2.352424in}{3.154636in}}%
\pgfpathlineto{\pgfqpoint{2.354429in}{3.601909in}}%
\pgfpathlineto{\pgfqpoint{2.356433in}{3.049636in}}%
\pgfpathlineto{\pgfqpoint{2.357435in}{3.025091in}}%
\pgfpathlineto{\pgfqpoint{2.359440in}{3.104182in}}%
\pgfpathlineto{\pgfqpoint{2.361444in}{2.991000in}}%
\pgfpathlineto{\pgfqpoint{2.365453in}{2.940545in}}%
\pgfpathlineto{\pgfqpoint{2.366456in}{2.941909in}}%
\pgfpathlineto{\pgfqpoint{2.369462in}{2.830091in}}%
\pgfpathlineto{\pgfqpoint{2.371467in}{2.906455in}}%
\pgfpathlineto{\pgfqpoint{2.373471in}{2.800091in}}%
\pgfpathlineto{\pgfqpoint{2.374474in}{2.798727in}}%
\pgfpathlineto{\pgfqpoint{2.375476in}{2.813727in}}%
\pgfpathlineto{\pgfqpoint{2.376478in}{2.890091in}}%
\pgfpathlineto{\pgfqpoint{2.378482in}{2.791909in}}%
\pgfpathlineto{\pgfqpoint{2.379485in}{2.796000in}}%
\pgfpathlineto{\pgfqpoint{2.381489in}{2.921455in}}%
\pgfpathlineto{\pgfqpoint{2.382491in}{2.886000in}}%
\pgfpathlineto{\pgfqpoint{2.383494in}{2.820545in}}%
\pgfpathlineto{\pgfqpoint{2.384496in}{2.847818in}}%
\pgfpathlineto{\pgfqpoint{2.386500in}{2.959636in}}%
\pgfpathlineto{\pgfqpoint{2.387503in}{2.969182in}}%
\pgfpathlineto{\pgfqpoint{2.389507in}{2.916000in}}%
\pgfpathlineto{\pgfqpoint{2.391512in}{3.045545in}}%
\pgfpathlineto{\pgfqpoint{2.392514in}{3.025091in}}%
\pgfpathlineto{\pgfqpoint{2.393516in}{3.131455in}}%
\pgfpathlineto{\pgfqpoint{2.395521in}{3.089182in}}%
\pgfpathlineto{\pgfqpoint{2.396523in}{3.072818in}}%
\pgfpathlineto{\pgfqpoint{2.397525in}{3.134182in}}%
\pgfpathlineto{\pgfqpoint{2.398527in}{3.511909in}}%
\pgfpathlineto{\pgfqpoint{2.399530in}{3.465545in}}%
\pgfpathlineto{\pgfqpoint{2.401534in}{3.085091in}}%
\pgfpathlineto{\pgfqpoint{2.402536in}{3.139636in}}%
\pgfpathlineto{\pgfqpoint{2.404541in}{3.653727in}}%
\pgfpathlineto{\pgfqpoint{2.405543in}{3.510545in}}%
\pgfpathlineto{\pgfqpoint{2.406545in}{3.128727in}}%
\pgfpathlineto{\pgfqpoint{2.407548in}{3.156000in}}%
\pgfpathlineto{\pgfqpoint{2.409552in}{3.642818in}}%
\pgfpathlineto{\pgfqpoint{2.411557in}{3.068727in}}%
\pgfpathlineto{\pgfqpoint{2.412559in}{3.059182in}}%
\pgfpathlineto{\pgfqpoint{2.413561in}{3.016909in}}%
\pgfpathlineto{\pgfqpoint{2.414563in}{3.031909in}}%
\pgfpathlineto{\pgfqpoint{2.417570in}{2.971909in}}%
\pgfpathlineto{\pgfqpoint{2.419574in}{2.839636in}}%
\pgfpathlineto{\pgfqpoint{2.420577in}{2.901000in}}%
\pgfpathlineto{\pgfqpoint{2.421579in}{2.895545in}}%
\pgfpathlineto{\pgfqpoint{2.422581in}{2.890091in}}%
\pgfpathlineto{\pgfqpoint{2.424586in}{2.779636in}}%
\pgfpathlineto{\pgfqpoint{2.426590in}{2.879182in}}%
\pgfpathlineto{\pgfqpoint{2.427592in}{2.860091in}}%
\pgfpathlineto{\pgfqpoint{2.429597in}{2.731909in}}%
\pgfpathlineto{\pgfqpoint{2.431601in}{2.877818in}}%
\pgfpathlineto{\pgfqpoint{2.432604in}{2.857364in}}%
\pgfpathlineto{\pgfqpoint{2.434608in}{2.748273in}}%
\pgfpathlineto{\pgfqpoint{2.436613in}{2.899636in}}%
\pgfpathlineto{\pgfqpoint{2.437615in}{2.888727in}}%
\pgfpathlineto{\pgfqpoint{2.438617in}{2.839636in}}%
\pgfpathlineto{\pgfqpoint{2.441624in}{2.971909in}}%
\pgfpathlineto{\pgfqpoint{2.442626in}{2.970545in}}%
\pgfpathlineto{\pgfqpoint{2.443628in}{2.932364in}}%
\pgfpathlineto{\pgfqpoint{2.449642in}{3.091909in}}%
\pgfpathlineto{\pgfqpoint{2.450644in}{3.108273in}}%
\pgfpathlineto{\pgfqpoint{2.451646in}{3.056455in}}%
\pgfpathlineto{\pgfqpoint{2.454653in}{3.555545in}}%
\pgfpathlineto{\pgfqpoint{2.456657in}{3.106909in}}%
\pgfpathlineto{\pgfqpoint{2.457660in}{3.247364in}}%
\pgfpathlineto{\pgfqpoint{2.459664in}{3.653727in}}%
\pgfpathlineto{\pgfqpoint{2.460666in}{3.543273in}}%
\pgfpathlineto{\pgfqpoint{2.461669in}{3.091909in}}%
\pgfpathlineto{\pgfqpoint{2.462671in}{3.177818in}}%
\pgfpathlineto{\pgfqpoint{2.464675in}{3.595091in}}%
\pgfpathlineto{\pgfqpoint{2.466680in}{3.055091in}}%
\pgfpathlineto{\pgfqpoint{2.468684in}{3.270545in}}%
\pgfpathlineto{\pgfqpoint{2.469687in}{3.195545in}}%
\pgfpathlineto{\pgfqpoint{2.471691in}{2.950091in}}%
\pgfpathlineto{\pgfqpoint{2.473696in}{3.018273in}}%
\pgfpathlineto{\pgfqpoint{2.474698in}{3.004636in}}%
\pgfpathlineto{\pgfqpoint{2.476702in}{2.905091in}}%
\pgfpathlineto{\pgfqpoint{2.478707in}{2.856000in}}%
\pgfpathlineto{\pgfqpoint{2.479709in}{2.881909in}}%
\pgfpathlineto{\pgfqpoint{2.481714in}{2.865545in}}%
\pgfpathlineto{\pgfqpoint{2.482716in}{2.836909in}}%
\pgfpathlineto{\pgfqpoint{2.483718in}{2.748273in}}%
\pgfpathlineto{\pgfqpoint{2.484720in}{2.753727in}}%
\pgfpathlineto{\pgfqpoint{2.486725in}{2.857364in}}%
\pgfpathlineto{\pgfqpoint{2.487727in}{2.820545in}}%
\pgfpathlineto{\pgfqpoint{2.489731in}{2.718273in}}%
\pgfpathlineto{\pgfqpoint{2.491736in}{2.876455in}}%
\pgfpathlineto{\pgfqpoint{2.493740in}{2.752364in}}%
\pgfpathlineto{\pgfqpoint{2.494743in}{2.753727in}}%
\pgfpathlineto{\pgfqpoint{2.496747in}{2.937818in}}%
\pgfpathlineto{\pgfqpoint{2.498752in}{2.849182in}}%
\pgfpathlineto{\pgfqpoint{2.502761in}{3.051000in}}%
\pgfpathlineto{\pgfqpoint{2.503763in}{3.071455in}}%
\pgfpathlineto{\pgfqpoint{2.505767in}{2.969182in}}%
\pgfpathlineto{\pgfqpoint{2.506770in}{2.985545in}}%
\pgfpathlineto{\pgfqpoint{2.507772in}{3.094636in}}%
\pgfpathlineto{\pgfqpoint{2.509776in}{3.667364in}}%
\pgfpathlineto{\pgfqpoint{2.511781in}{3.121909in}}%
\pgfpathlineto{\pgfqpoint{2.513785in}{3.614182in}}%
\pgfpathlineto{\pgfqpoint{2.514788in}{3.656455in}}%
\pgfpathlineto{\pgfqpoint{2.516792in}{3.104182in}}%
\pgfpathlineto{\pgfqpoint{2.518797in}{3.682364in}}%
\pgfpathlineto{\pgfqpoint{2.519799in}{3.749182in}}%
\pgfpathlineto{\pgfqpoint{2.521803in}{3.156000in}}%
\pgfpathlineto{\pgfqpoint{2.522805in}{3.289636in}}%
\pgfpathlineto{\pgfqpoint{2.524810in}{3.636000in}}%
\pgfpathlineto{\pgfqpoint{2.526814in}{3.019636in}}%
\pgfpathlineto{\pgfqpoint{2.528819in}{3.325091in}}%
\pgfpathlineto{\pgfqpoint{2.531826in}{2.931000in}}%
\pgfpathlineto{\pgfqpoint{2.533830in}{3.026455in}}%
\pgfpathlineto{\pgfqpoint{2.534832in}{3.016909in}}%
\pgfpathlineto{\pgfqpoint{2.536837in}{2.896909in}}%
\pgfpathlineto{\pgfqpoint{2.538841in}{2.841000in}}%
\pgfpathlineto{\pgfqpoint{2.539844in}{2.876455in}}%
\pgfpathlineto{\pgfqpoint{2.540846in}{2.839636in}}%
\pgfpathlineto{\pgfqpoint{2.541848in}{2.886000in}}%
\pgfpathlineto{\pgfqpoint{2.542850in}{2.846455in}}%
\pgfpathlineto{\pgfqpoint{2.543853in}{2.712818in}}%
\pgfpathlineto{\pgfqpoint{2.546859in}{2.886000in}}%
\pgfpathlineto{\pgfqpoint{2.548864in}{2.730545in}}%
\pgfpathlineto{\pgfqpoint{2.549866in}{2.744182in}}%
\pgfpathlineto{\pgfqpoint{2.551871in}{2.888727in}}%
\pgfpathlineto{\pgfqpoint{2.552873in}{2.865545in}}%
\pgfpathlineto{\pgfqpoint{2.553875in}{2.736000in}}%
\pgfpathlineto{\pgfqpoint{2.554877in}{2.753727in}}%
\pgfpathlineto{\pgfqpoint{2.556882in}{2.940545in}}%
\pgfpathlineto{\pgfqpoint{2.559888in}{2.850545in}}%
\pgfpathlineto{\pgfqpoint{2.561893in}{2.969182in}}%
\pgfpathlineto{\pgfqpoint{2.562895in}{2.982818in}}%
\pgfpathlineto{\pgfqpoint{2.563897in}{2.973273in}}%
\pgfpathlineto{\pgfqpoint{2.566904in}{3.067364in}}%
\pgfpathlineto{\pgfqpoint{2.567906in}{3.044182in}}%
\pgfpathlineto{\pgfqpoint{2.568909in}{3.091909in}}%
\pgfpathlineto{\pgfqpoint{2.569911in}{3.514636in}}%
\pgfpathlineto{\pgfqpoint{2.571915in}{3.166909in}}%
\pgfpathlineto{\pgfqpoint{2.572918in}{3.258273in}}%
\pgfpathlineto{\pgfqpoint{2.574922in}{3.685091in}}%
\pgfpathlineto{\pgfqpoint{2.576927in}{3.266455in}}%
\pgfpathlineto{\pgfqpoint{2.578931in}{3.745091in}}%
\pgfpathlineto{\pgfqpoint{2.579933in}{3.768273in}}%
\pgfpathlineto{\pgfqpoint{2.580936in}{3.641455in}}%
\pgfpathlineto{\pgfqpoint{2.581938in}{3.344182in}}%
\pgfpathlineto{\pgfqpoint{2.583942in}{3.683727in}}%
\pgfpathlineto{\pgfqpoint{2.584945in}{3.734182in}}%
\pgfpathlineto{\pgfqpoint{2.586949in}{3.066000in}}%
\pgfpathlineto{\pgfqpoint{2.587951in}{3.162818in}}%
\pgfpathlineto{\pgfqpoint{2.588954in}{3.612818in}}%
\pgfpathlineto{\pgfqpoint{2.591960in}{2.986909in}}%
\pgfpathlineto{\pgfqpoint{2.592962in}{3.000545in}}%
\pgfpathlineto{\pgfqpoint{2.593965in}{3.102818in}}%
\pgfpathlineto{\pgfqpoint{2.594967in}{3.070091in}}%
\pgfpathlineto{\pgfqpoint{2.596971in}{2.902364in}}%
\pgfpathlineto{\pgfqpoint{2.597974in}{2.851909in}}%
\pgfpathlineto{\pgfqpoint{2.598976in}{2.890091in}}%
\pgfpathlineto{\pgfqpoint{2.600980in}{2.836909in}}%
\pgfpathlineto{\pgfqpoint{2.601983in}{2.871000in}}%
\pgfpathlineto{\pgfqpoint{2.603987in}{2.716909in}}%
\pgfpathlineto{\pgfqpoint{2.604989in}{2.745545in}}%
\pgfpathlineto{\pgfqpoint{2.606994in}{2.877818in}}%
\pgfpathlineto{\pgfqpoint{2.608998in}{2.708727in}}%
\pgfpathlineto{\pgfqpoint{2.610001in}{2.725091in}}%
\pgfpathlineto{\pgfqpoint{2.612005in}{2.876455in}}%
\pgfpathlineto{\pgfqpoint{2.614010in}{2.755091in}}%
\pgfpathlineto{\pgfqpoint{2.615012in}{2.783727in}}%
\pgfpathlineto{\pgfqpoint{2.617016in}{2.926909in}}%
\pgfpathlineto{\pgfqpoint{2.618019in}{2.913273in}}%
\pgfpathlineto{\pgfqpoint{2.619021in}{2.931000in}}%
\pgfpathlineto{\pgfqpoint{2.620023in}{2.884636in}}%
\pgfpathlineto{\pgfqpoint{2.625034in}{3.145091in}}%
\pgfpathlineto{\pgfqpoint{2.626036in}{3.098727in}}%
\pgfpathlineto{\pgfqpoint{2.628041in}{3.211909in}}%
\pgfpathlineto{\pgfqpoint{2.630045in}{3.711000in}}%
\pgfpathlineto{\pgfqpoint{2.632050in}{3.385091in}}%
\pgfpathlineto{\pgfqpoint{2.634054in}{3.773727in}}%
\pgfpathlineto{\pgfqpoint{2.635057in}{3.783273in}}%
\pgfpathlineto{\pgfqpoint{2.637061in}{3.359182in}}%
\pgfpathlineto{\pgfqpoint{2.639066in}{3.773727in}}%
\pgfpathlineto{\pgfqpoint{2.640068in}{3.719182in}}%
\pgfpathlineto{\pgfqpoint{2.642072in}{3.056455in}}%
\pgfpathlineto{\pgfqpoint{2.643075in}{3.190091in}}%
\pgfpathlineto{\pgfqpoint{2.644077in}{3.513273in}}%
\pgfpathlineto{\pgfqpoint{2.646081in}{3.036000in}}%
\pgfpathlineto{\pgfqpoint{2.648086in}{2.955545in}}%
\pgfpathlineto{\pgfqpoint{2.649088in}{3.033273in}}%
\pgfpathlineto{\pgfqpoint{2.651093in}{2.865545in}}%
\pgfpathlineto{\pgfqpoint{2.652095in}{2.892818in}}%
\pgfpathlineto{\pgfqpoint{2.654099in}{2.789182in}}%
\pgfpathlineto{\pgfqpoint{2.655102in}{2.768727in}}%
\pgfpathlineto{\pgfqpoint{2.656104in}{2.779636in}}%
\pgfpathlineto{\pgfqpoint{2.657106in}{2.871000in}}%
\pgfpathlineto{\pgfqpoint{2.659111in}{2.673273in}}%
\pgfpathlineto{\pgfqpoint{2.660113in}{2.703273in}}%
\pgfpathlineto{\pgfqpoint{2.662117in}{2.873727in}}%
\pgfpathlineto{\pgfqpoint{2.664122in}{2.701909in}}%
\pgfpathlineto{\pgfqpoint{2.665124in}{2.757818in}}%
\pgfpathlineto{\pgfqpoint{2.667128in}{2.950091in}}%
\pgfpathlineto{\pgfqpoint{2.668131in}{2.924182in}}%
\pgfpathlineto{\pgfqpoint{2.670135in}{2.781000in}}%
\pgfpathlineto{\pgfqpoint{2.673142in}{3.142364in}}%
\pgfpathlineto{\pgfqpoint{2.674144in}{3.723273in}}%
\pgfpathlineto{\pgfqpoint{2.675146in}{3.558273in}}%
\pgfpathlineto{\pgfqpoint{2.676149in}{3.059182in}}%
\pgfpathlineto{\pgfqpoint{2.677151in}{3.067364in}}%
\pgfpathlineto{\pgfqpoint{2.678153in}{3.003273in}}%
\pgfpathlineto{\pgfqpoint{2.679155in}{3.690545in}}%
\pgfpathlineto{\pgfqpoint{2.680158in}{3.681000in}}%
\pgfpathlineto{\pgfqpoint{2.682162in}{3.142364in}}%
\pgfpathlineto{\pgfqpoint{2.683164in}{3.132818in}}%
\pgfpathlineto{\pgfqpoint{2.684167in}{3.541909in}}%
\pgfpathlineto{\pgfqpoint{2.685169in}{3.516000in}}%
\pgfpathlineto{\pgfqpoint{2.687173in}{2.961000in}}%
\pgfpathlineto{\pgfqpoint{2.688176in}{3.116455in}}%
\pgfpathlineto{\pgfqpoint{2.689178in}{3.756000in}}%
\pgfpathlineto{\pgfqpoint{2.690180in}{3.719182in}}%
\pgfpathlineto{\pgfqpoint{2.692185in}{2.997818in}}%
\pgfpathlineto{\pgfqpoint{2.693187in}{2.999182in}}%
\pgfpathlineto{\pgfqpoint{2.694189in}{3.004636in}}%
\pgfpathlineto{\pgfqpoint{2.695191in}{3.000545in}}%
\pgfpathlineto{\pgfqpoint{2.697196in}{2.956909in}}%
\pgfpathlineto{\pgfqpoint{2.699200in}{3.036000in}}%
\pgfpathlineto{\pgfqpoint{2.700202in}{2.812364in}}%
\pgfpathlineto{\pgfqpoint{2.701205in}{2.873727in}}%
\pgfpathlineto{\pgfqpoint{2.702207in}{2.864182in}}%
\pgfpathlineto{\pgfqpoint{2.703209in}{2.886000in}}%
\pgfpathlineto{\pgfqpoint{2.704211in}{2.967818in}}%
\pgfpathlineto{\pgfqpoint{2.705214in}{2.858727in}}%
\pgfpathlineto{\pgfqpoint{2.706216in}{2.877818in}}%
\pgfpathlineto{\pgfqpoint{2.707218in}{2.944636in}}%
\pgfpathlineto{\pgfqpoint{2.708220in}{2.871000in}}%
\pgfpathlineto{\pgfqpoint{2.709223in}{2.708727in}}%
\pgfpathlineto{\pgfqpoint{2.710225in}{2.740091in}}%
\pgfpathlineto{\pgfqpoint{2.711227in}{2.913273in}}%
\pgfpathlineto{\pgfqpoint{2.712229in}{2.881909in}}%
\pgfpathlineto{\pgfqpoint{2.714234in}{3.067364in}}%
\pgfpathlineto{\pgfqpoint{2.715236in}{2.839636in}}%
\pgfpathlineto{\pgfqpoint{2.718243in}{2.940545in}}%
\pgfpathlineto{\pgfqpoint{2.719245in}{3.141000in}}%
\pgfpathlineto{\pgfqpoint{2.720247in}{3.580091in}}%
\pgfpathlineto{\pgfqpoint{2.722252in}{3.029182in}}%
\pgfpathlineto{\pgfqpoint{2.723254in}{3.016909in}}%
\pgfpathlineto{\pgfqpoint{2.725259in}{3.217364in}}%
\pgfpathlineto{\pgfqpoint{2.726261in}{3.371455in}}%
\pgfpathlineto{\pgfqpoint{2.727263in}{3.157364in}}%
\pgfpathlineto{\pgfqpoint{2.729268in}{3.730091in}}%
\pgfpathlineto{\pgfqpoint{2.730270in}{3.660545in}}%
\pgfpathlineto{\pgfqpoint{2.732274in}{3.307364in}}%
\pgfpathlineto{\pgfqpoint{2.734279in}{3.779182in}}%
\pgfpathlineto{\pgfqpoint{2.735281in}{3.732818in}}%
\pgfpathlineto{\pgfqpoint{2.737285in}{3.443727in}}%
\pgfpathlineto{\pgfqpoint{2.739290in}{3.726000in}}%
\pgfpathlineto{\pgfqpoint{2.741294in}{2.984182in}}%
\pgfpathlineto{\pgfqpoint{2.742297in}{2.984182in}}%
\pgfpathlineto{\pgfqpoint{2.744301in}{3.640091in}}%
\pgfpathlineto{\pgfqpoint{2.747308in}{2.873727in}}%
\pgfpathlineto{\pgfqpoint{2.750315in}{2.969182in}}%
\pgfpathlineto{\pgfqpoint{2.751317in}{2.958273in}}%
\pgfpathlineto{\pgfqpoint{2.754324in}{2.625545in}}%
\pgfpathlineto{\pgfqpoint{2.756328in}{2.823273in}}%
\pgfpathlineto{\pgfqpoint{2.757330in}{2.806909in}}%
\pgfpathlineto{\pgfqpoint{2.759335in}{2.682818in}}%
\pgfpathlineto{\pgfqpoint{2.760337in}{2.693727in}}%
\pgfpathlineto{\pgfqpoint{2.762342in}{2.817818in}}%
\pgfpathlineto{\pgfqpoint{2.763344in}{2.741455in}}%
\pgfpathlineto{\pgfqpoint{2.764346in}{2.742818in}}%
\pgfpathlineto{\pgfqpoint{2.765348in}{2.789182in}}%
\pgfpathlineto{\pgfqpoint{2.767353in}{2.966455in}}%
\pgfpathlineto{\pgfqpoint{2.769357in}{2.826000in}}%
\pgfpathlineto{\pgfqpoint{2.772364in}{3.022364in}}%
\pgfpathlineto{\pgfqpoint{2.773366in}{3.608727in}}%
\pgfpathlineto{\pgfqpoint{2.775371in}{3.042818in}}%
\pgfpathlineto{\pgfqpoint{2.776373in}{3.056455in}}%
\pgfpathlineto{\pgfqpoint{2.777375in}{3.027818in}}%
\pgfpathlineto{\pgfqpoint{2.779380in}{3.735545in}}%
\pgfpathlineto{\pgfqpoint{2.780382in}{3.772364in}}%
\pgfpathlineto{\pgfqpoint{2.781384in}{3.393273in}}%
\pgfpathlineto{\pgfqpoint{2.785393in}{3.831000in}}%
\pgfpathlineto{\pgfqpoint{2.787398in}{3.619636in}}%
\pgfpathlineto{\pgfqpoint{2.789402in}{3.810545in}}%
\pgfpathlineto{\pgfqpoint{2.790404in}{3.689182in}}%
\pgfpathlineto{\pgfqpoint{2.791407in}{3.121909in}}%
\pgfpathlineto{\pgfqpoint{2.792409in}{3.138273in}}%
\pgfpathlineto{\pgfqpoint{2.794413in}{3.584182in}}%
\pgfpathlineto{\pgfqpoint{2.795416in}{3.413727in}}%
\pgfpathlineto{\pgfqpoint{2.797420in}{2.977364in}}%
\pgfpathlineto{\pgfqpoint{2.798422in}{2.836909in}}%
\pgfpathlineto{\pgfqpoint{2.799425in}{2.881909in}}%
\pgfpathlineto{\pgfqpoint{2.800427in}{2.868273in}}%
\pgfpathlineto{\pgfqpoint{2.801429in}{2.835545in}}%
\pgfpathlineto{\pgfqpoint{2.802431in}{2.845091in}}%
\pgfpathlineto{\pgfqpoint{2.804436in}{2.663727in}}%
\pgfpathlineto{\pgfqpoint{2.807442in}{2.770091in}}%
\pgfpathlineto{\pgfqpoint{2.809447in}{2.614636in}}%
\pgfpathlineto{\pgfqpoint{2.812454in}{2.876455in}}%
\pgfpathlineto{\pgfqpoint{2.814458in}{2.673273in}}%
\pgfpathlineto{\pgfqpoint{2.817465in}{3.003273in}}%
\pgfpathlineto{\pgfqpoint{2.818467in}{2.939182in}}%
\pgfpathlineto{\pgfqpoint{2.820472in}{2.976000in}}%
\pgfpathlineto{\pgfqpoint{2.821474in}{2.992364in}}%
\pgfpathlineto{\pgfqpoint{2.822476in}{3.034636in}}%
\pgfpathlineto{\pgfqpoint{2.823478in}{3.016909in}}%
\pgfpathlineto{\pgfqpoint{2.824481in}{3.657818in}}%
\pgfpathlineto{\pgfqpoint{2.825483in}{3.640091in}}%
\pgfpathlineto{\pgfqpoint{2.826485in}{3.280091in}}%
\pgfpathlineto{\pgfqpoint{2.827487in}{3.349636in}}%
\pgfpathlineto{\pgfqpoint{2.830494in}{3.799636in}}%
\pgfpathlineto{\pgfqpoint{2.831496in}{3.528273in}}%
\pgfpathlineto{\pgfqpoint{2.832499in}{3.533727in}}%
\pgfpathlineto{\pgfqpoint{2.834503in}{3.839182in}}%
\pgfpathlineto{\pgfqpoint{2.835505in}{3.790091in}}%
\pgfpathlineto{\pgfqpoint{2.836508in}{3.619636in}}%
\pgfpathlineto{\pgfqpoint{2.837510in}{3.646909in}}%
\pgfpathlineto{\pgfqpoint{2.839514in}{3.907364in}}%
\pgfpathlineto{\pgfqpoint{2.840516in}{3.705545in}}%
\pgfpathlineto{\pgfqpoint{2.842521in}{3.014182in}}%
\pgfpathlineto{\pgfqpoint{2.844525in}{3.736909in}}%
\pgfpathlineto{\pgfqpoint{2.846530in}{2.991000in}}%
\pgfpathlineto{\pgfqpoint{2.848534in}{2.798727in}}%
\pgfpathlineto{\pgfqpoint{2.850539in}{2.932364in}}%
\pgfpathlineto{\pgfqpoint{2.851541in}{2.969182in}}%
\pgfpathlineto{\pgfqpoint{2.852543in}{2.881909in}}%
\pgfpathlineto{\pgfqpoint{2.854548in}{2.670545in}}%
\pgfpathlineto{\pgfqpoint{2.856552in}{2.869636in}}%
\pgfpathlineto{\pgfqpoint{2.857555in}{2.858727in}}%
\pgfpathlineto{\pgfqpoint{2.859559in}{2.655545in}}%
\pgfpathlineto{\pgfqpoint{2.860561in}{2.697818in}}%
\pgfpathlineto{\pgfqpoint{2.862566in}{2.809636in}}%
\pgfpathlineto{\pgfqpoint{2.864570in}{2.812364in}}%
\pgfpathlineto{\pgfqpoint{2.866575in}{2.916000in}}%
\pgfpathlineto{\pgfqpoint{2.868579in}{2.849182in}}%
\pgfpathlineto{\pgfqpoint{2.870584in}{2.954182in}}%
\pgfpathlineto{\pgfqpoint{2.871586in}{3.113727in}}%
\pgfpathlineto{\pgfqpoint{2.872588in}{3.089182in}}%
\pgfpathlineto{\pgfqpoint{2.873590in}{3.008727in}}%
\pgfpathlineto{\pgfqpoint{2.875595in}{3.278727in}}%
\pgfpathlineto{\pgfqpoint{2.876597in}{3.449182in}}%
\pgfpathlineto{\pgfqpoint{2.877599in}{3.325091in}}%
\pgfpathlineto{\pgfqpoint{2.879604in}{3.708273in}}%
\pgfpathlineto{\pgfqpoint{2.880606in}{3.732818in}}%
\pgfpathlineto{\pgfqpoint{2.882611in}{3.631909in}}%
\pgfpathlineto{\pgfqpoint{2.883613in}{3.691909in}}%
\pgfpathlineto{\pgfqpoint{2.884615in}{3.863727in}}%
\pgfpathlineto{\pgfqpoint{2.885617in}{3.719182in}}%
\pgfpathlineto{\pgfqpoint{2.887622in}{3.239182in}}%
\pgfpathlineto{\pgfqpoint{2.889626in}{3.821455in}}%
\pgfpathlineto{\pgfqpoint{2.891631in}{3.663273in}}%
\pgfpathlineto{\pgfqpoint{2.892633in}{3.082364in}}%
\pgfpathlineto{\pgfqpoint{2.894638in}{3.396000in}}%
\pgfpathlineto{\pgfqpoint{2.898647in}{2.854636in}}%
\pgfpathlineto{\pgfqpoint{2.900651in}{2.961000in}}%
\pgfpathlineto{\pgfqpoint{2.901653in}{2.955545in}}%
\pgfpathlineto{\pgfqpoint{2.903658in}{2.699182in}}%
\pgfpathlineto{\pgfqpoint{2.904660in}{2.669182in}}%
\pgfpathlineto{\pgfqpoint{2.906665in}{2.876455in}}%
\pgfpathlineto{\pgfqpoint{2.907667in}{2.809636in}}%
\pgfpathlineto{\pgfqpoint{2.908669in}{2.666455in}}%
\pgfpathlineto{\pgfqpoint{2.909671in}{2.686909in}}%
\pgfpathlineto{\pgfqpoint{2.910673in}{2.751000in}}%
\pgfpathlineto{\pgfqpoint{2.911676in}{2.913273in}}%
\pgfpathlineto{\pgfqpoint{2.913680in}{2.674636in}}%
\pgfpathlineto{\pgfqpoint{2.914682in}{2.708727in}}%
\pgfpathlineto{\pgfqpoint{2.916687in}{2.997818in}}%
\pgfpathlineto{\pgfqpoint{2.918691in}{2.853273in}}%
\pgfpathlineto{\pgfqpoint{2.919694in}{2.845091in}}%
\pgfpathlineto{\pgfqpoint{2.923703in}{3.126000in}}%
\pgfpathlineto{\pgfqpoint{2.925707in}{3.191455in}}%
\pgfpathlineto{\pgfqpoint{2.926709in}{3.288273in}}%
\pgfpathlineto{\pgfqpoint{2.928714in}{3.775091in}}%
\pgfpathlineto{\pgfqpoint{2.930718in}{3.640091in}}%
\pgfpathlineto{\pgfqpoint{2.931721in}{3.649636in}}%
\pgfpathlineto{\pgfqpoint{2.932723in}{3.711000in}}%
\pgfpathlineto{\pgfqpoint{2.933725in}{3.888273in}}%
\pgfpathlineto{\pgfqpoint{2.934727in}{3.870545in}}%
\pgfpathlineto{\pgfqpoint{2.936732in}{3.618273in}}%
\pgfpathlineto{\pgfqpoint{2.937734in}{3.588273in}}%
\pgfpathlineto{\pgfqpoint{2.939739in}{3.773727in}}%
\pgfpathlineto{\pgfqpoint{2.940741in}{3.716455in}}%
\pgfpathlineto{\pgfqpoint{2.942745in}{3.423273in}}%
\pgfpathlineto{\pgfqpoint{2.943747in}{3.475091in}}%
\pgfpathlineto{\pgfqpoint{2.945752in}{3.003273in}}%
\pgfpathlineto{\pgfqpoint{2.946754in}{2.907818in}}%
\pgfpathlineto{\pgfqpoint{2.948759in}{3.010091in}}%
\pgfpathlineto{\pgfqpoint{2.950763in}{2.763273in}}%
\pgfpathlineto{\pgfqpoint{2.951765in}{2.824636in}}%
\pgfpathlineto{\pgfqpoint{2.954772in}{2.658273in}}%
\pgfpathlineto{\pgfqpoint{2.956777in}{2.787818in}}%
\pgfpathlineto{\pgfqpoint{2.959783in}{2.636455in}}%
\pgfpathlineto{\pgfqpoint{2.962790in}{2.869636in}}%
\pgfpathlineto{\pgfqpoint{2.963792in}{2.775545in}}%
\pgfpathlineto{\pgfqpoint{2.965797in}{2.849182in}}%
\pgfpathlineto{\pgfqpoint{2.967801in}{2.943273in}}%
\pgfpathlineto{\pgfqpoint{2.968804in}{2.928273in}}%
\pgfpathlineto{\pgfqpoint{2.969806in}{2.872364in}}%
\pgfpathlineto{\pgfqpoint{2.971810in}{2.965091in}}%
\pgfpathlineto{\pgfqpoint{2.972813in}{3.149182in}}%
\pgfpathlineto{\pgfqpoint{2.973815in}{3.629182in}}%
\pgfpathlineto{\pgfqpoint{2.975819in}{3.321000in}}%
\pgfpathlineto{\pgfqpoint{2.976822in}{3.220091in}}%
\pgfpathlineto{\pgfqpoint{2.978826in}{3.833727in}}%
\pgfpathlineto{\pgfqpoint{2.981833in}{3.548727in}}%
\pgfpathlineto{\pgfqpoint{2.983837in}{3.953727in}}%
\pgfpathlineto{\pgfqpoint{2.984839in}{3.937364in}}%
\pgfpathlineto{\pgfqpoint{2.986844in}{3.623727in}}%
\pgfpathlineto{\pgfqpoint{2.989851in}{3.851455in}}%
\pgfpathlineto{\pgfqpoint{2.990853in}{3.573273in}}%
\pgfpathlineto{\pgfqpoint{2.991855in}{2.947364in}}%
\pgfpathlineto{\pgfqpoint{2.993860in}{3.277364in}}%
\pgfpathlineto{\pgfqpoint{2.996866in}{2.871000in}}%
\pgfpathlineto{\pgfqpoint{2.997869in}{2.880545in}}%
\pgfpathlineto{\pgfqpoint{2.999873in}{2.836909in}}%
\pgfpathlineto{\pgfqpoint{3.000875in}{2.841000in}}%
\pgfpathlineto{\pgfqpoint{3.004884in}{2.646000in}}%
\pgfpathlineto{\pgfqpoint{3.005887in}{2.854636in}}%
\pgfpathlineto{\pgfqpoint{3.006889in}{2.835545in}}%
\pgfpathlineto{\pgfqpoint{3.009896in}{2.606455in}}%
\pgfpathlineto{\pgfqpoint{3.011900in}{2.865545in}}%
\pgfpathlineto{\pgfqpoint{3.014907in}{2.722364in}}%
\pgfpathlineto{\pgfqpoint{3.017913in}{3.010091in}}%
\pgfpathlineto{\pgfqpoint{3.019918in}{2.876455in}}%
\pgfpathlineto{\pgfqpoint{3.020920in}{2.902364in}}%
\pgfpathlineto{\pgfqpoint{3.021922in}{3.007364in}}%
\pgfpathlineto{\pgfqpoint{3.022925in}{3.469636in}}%
\pgfpathlineto{\pgfqpoint{3.023927in}{3.364636in}}%
\pgfpathlineto{\pgfqpoint{3.024929in}{3.548727in}}%
\pgfpathlineto{\pgfqpoint{3.026934in}{3.060545in}}%
\pgfpathlineto{\pgfqpoint{3.028938in}{3.698727in}}%
\pgfpathlineto{\pgfqpoint{3.029940in}{3.795545in}}%
\pgfpathlineto{\pgfqpoint{3.031945in}{3.543273in}}%
\pgfpathlineto{\pgfqpoint{3.034952in}{4.013727in}}%
\pgfpathlineto{\pgfqpoint{3.036956in}{3.506455in}}%
\pgfpathlineto{\pgfqpoint{3.038961in}{3.869182in}}%
\pgfpathlineto{\pgfqpoint{3.039963in}{3.820091in}}%
\pgfpathlineto{\pgfqpoint{3.041967in}{3.111000in}}%
\pgfpathlineto{\pgfqpoint{3.042970in}{3.548727in}}%
\pgfpathlineto{\pgfqpoint{3.043972in}{3.529636in}}%
\pgfpathlineto{\pgfqpoint{3.044974in}{3.573273in}}%
\pgfpathlineto{\pgfqpoint{3.046979in}{2.901000in}}%
\pgfpathlineto{\pgfqpoint{3.047981in}{2.931000in}}%
\pgfpathlineto{\pgfqpoint{3.048983in}{2.760545in}}%
\pgfpathlineto{\pgfqpoint{3.050987in}{2.847818in}}%
\pgfpathlineto{\pgfqpoint{3.051990in}{2.853273in}}%
\pgfpathlineto{\pgfqpoint{3.053994in}{2.599636in}}%
\pgfpathlineto{\pgfqpoint{3.057001in}{2.781000in}}%
\pgfpathlineto{\pgfqpoint{3.059005in}{2.595545in}}%
\pgfpathlineto{\pgfqpoint{3.061010in}{2.757818in}}%
\pgfpathlineto{\pgfqpoint{3.062012in}{2.892818in}}%
\pgfpathlineto{\pgfqpoint{3.064017in}{2.731909in}}%
\pgfpathlineto{\pgfqpoint{3.065019in}{2.761909in}}%
\pgfpathlineto{\pgfqpoint{3.069028in}{2.939182in}}%
\pgfpathlineto{\pgfqpoint{3.070030in}{2.947364in}}%
\pgfpathlineto{\pgfqpoint{3.071032in}{2.926909in}}%
\pgfpathlineto{\pgfqpoint{3.072035in}{3.072818in}}%
\pgfpathlineto{\pgfqpoint{3.073037in}{3.042818in}}%
\pgfpathlineto{\pgfqpoint{3.074039in}{3.119182in}}%
\pgfpathlineto{\pgfqpoint{3.075041in}{3.622364in}}%
\pgfpathlineto{\pgfqpoint{3.076044in}{3.040091in}}%
\pgfpathlineto{\pgfqpoint{3.078048in}{3.657818in}}%
\pgfpathlineto{\pgfqpoint{3.079050in}{3.686455in}}%
\pgfpathlineto{\pgfqpoint{3.080053in}{3.822818in}}%
\pgfpathlineto{\pgfqpoint{3.082057in}{3.687818in}}%
\pgfpathlineto{\pgfqpoint{3.085064in}{4.056000in}}%
\pgfpathlineto{\pgfqpoint{3.087068in}{3.713727in}}%
\pgfpathlineto{\pgfqpoint{3.088070in}{3.779182in}}%
\pgfpathlineto{\pgfqpoint{3.089073in}{3.942818in}}%
\pgfpathlineto{\pgfqpoint{3.090075in}{3.931909in}}%
\pgfpathlineto{\pgfqpoint{3.093082in}{3.007364in}}%
\pgfpathlineto{\pgfqpoint{3.094084in}{3.551455in}}%
\pgfpathlineto{\pgfqpoint{3.095086in}{3.398727in}}%
\pgfpathlineto{\pgfqpoint{3.096088in}{2.969182in}}%
\pgfpathlineto{\pgfqpoint{3.097091in}{3.018273in}}%
\pgfpathlineto{\pgfqpoint{3.099095in}{2.819182in}}%
\pgfpathlineto{\pgfqpoint{3.101100in}{2.851909in}}%
\pgfpathlineto{\pgfqpoint{3.102102in}{2.980091in}}%
\pgfpathlineto{\pgfqpoint{3.104106in}{2.654182in}}%
\pgfpathlineto{\pgfqpoint{3.105109in}{2.678727in}}%
\pgfpathlineto{\pgfqpoint{3.107113in}{2.872364in}}%
\pgfpathlineto{\pgfqpoint{3.109118in}{2.659636in}}%
\pgfpathlineto{\pgfqpoint{3.110120in}{2.701909in}}%
\pgfpathlineto{\pgfqpoint{3.112124in}{2.890091in}}%
\pgfpathlineto{\pgfqpoint{3.114129in}{2.781000in}}%
\pgfpathlineto{\pgfqpoint{3.115131in}{2.826000in}}%
\pgfpathlineto{\pgfqpoint{3.117136in}{2.993727in}}%
\pgfpathlineto{\pgfqpoint{3.118138in}{2.830091in}}%
\pgfpathlineto{\pgfqpoint{3.119140in}{2.842364in}}%
\pgfpathlineto{\pgfqpoint{3.121144in}{2.969182in}}%
\pgfpathlineto{\pgfqpoint{3.122147in}{3.194182in}}%
\pgfpathlineto{\pgfqpoint{3.123149in}{3.011455in}}%
\pgfpathlineto{\pgfqpoint{3.124151in}{3.149182in}}%
\pgfpathlineto{\pgfqpoint{3.125153in}{3.045545in}}%
\pgfpathlineto{\pgfqpoint{3.126156in}{3.059182in}}%
\pgfpathlineto{\pgfqpoint{3.129162in}{3.713727in}}%
\pgfpathlineto{\pgfqpoint{3.131167in}{3.660545in}}%
\pgfpathlineto{\pgfqpoint{3.132169in}{3.476455in}}%
\pgfpathlineto{\pgfqpoint{3.135176in}{3.976909in}}%
\pgfpathlineto{\pgfqpoint{3.137180in}{3.691909in}}%
\pgfpathlineto{\pgfqpoint{3.139185in}{3.929182in}}%
\pgfpathlineto{\pgfqpoint{3.141189in}{3.784636in}}%
\pgfpathlineto{\pgfqpoint{3.142192in}{3.603273in}}%
\pgfpathlineto{\pgfqpoint{3.143194in}{3.083727in}}%
\pgfpathlineto{\pgfqpoint{3.144196in}{3.705545in}}%
\pgfpathlineto{\pgfqpoint{3.145198in}{3.168273in}}%
\pgfpathlineto{\pgfqpoint{3.146201in}{3.233727in}}%
\pgfpathlineto{\pgfqpoint{3.148205in}{2.830091in}}%
\pgfpathlineto{\pgfqpoint{3.149207in}{2.835545in}}%
\pgfpathlineto{\pgfqpoint{3.150210in}{2.800091in}}%
\pgfpathlineto{\pgfqpoint{3.151212in}{2.911909in}}%
\pgfpathlineto{\pgfqpoint{3.152214in}{2.910545in}}%
\pgfpathlineto{\pgfqpoint{3.154219in}{2.682818in}}%
\pgfpathlineto{\pgfqpoint{3.155221in}{2.663727in}}%
\pgfpathlineto{\pgfqpoint{3.157225in}{2.831455in}}%
\pgfpathlineto{\pgfqpoint{3.159230in}{2.663727in}}%
\pgfpathlineto{\pgfqpoint{3.162236in}{2.937818in}}%
\pgfpathlineto{\pgfqpoint{3.164241in}{2.723727in}}%
\pgfpathlineto{\pgfqpoint{3.165243in}{2.767364in}}%
\pgfpathlineto{\pgfqpoint{3.166245in}{2.895545in}}%
\pgfpathlineto{\pgfqpoint{3.167248in}{2.886000in}}%
\pgfpathlineto{\pgfqpoint{3.168250in}{2.931000in}}%
\pgfpathlineto{\pgfqpoint{3.170254in}{2.881909in}}%
\pgfpathlineto{\pgfqpoint{3.171257in}{3.014182in}}%
\pgfpathlineto{\pgfqpoint{3.172259in}{3.006000in}}%
\pgfpathlineto{\pgfqpoint{3.173261in}{3.076909in}}%
\pgfpathlineto{\pgfqpoint{3.174263in}{3.241909in}}%
\pgfpathlineto{\pgfqpoint{3.175266in}{3.097364in}}%
\pgfpathlineto{\pgfqpoint{3.177270in}{3.202364in}}%
\pgfpathlineto{\pgfqpoint{3.179275in}{3.749182in}}%
\pgfpathlineto{\pgfqpoint{3.180277in}{3.701455in}}%
\pgfpathlineto{\pgfqpoint{3.181279in}{3.539182in}}%
\pgfpathlineto{\pgfqpoint{3.182281in}{3.180545in}}%
\pgfpathlineto{\pgfqpoint{3.184286in}{3.841909in}}%
\pgfpathlineto{\pgfqpoint{3.186290in}{3.743727in}}%
\pgfpathlineto{\pgfqpoint{3.187293in}{3.194182in}}%
\pgfpathlineto{\pgfqpoint{3.189297in}{3.944182in}}%
\pgfpathlineto{\pgfqpoint{3.191302in}{3.843273in}}%
\pgfpathlineto{\pgfqpoint{3.192304in}{3.228273in}}%
\pgfpathlineto{\pgfqpoint{3.193306in}{3.569182in}}%
\pgfpathlineto{\pgfqpoint{3.195310in}{3.029182in}}%
\pgfpathlineto{\pgfqpoint{3.196313in}{3.104182in}}%
\pgfpathlineto{\pgfqpoint{3.197315in}{2.907818in}}%
\pgfpathlineto{\pgfqpoint{3.198317in}{2.961000in}}%
\pgfpathlineto{\pgfqpoint{3.200322in}{2.816455in}}%
\pgfpathlineto{\pgfqpoint{3.201324in}{2.898273in}}%
\pgfpathlineto{\pgfqpoint{3.204331in}{2.742818in}}%
\pgfpathlineto{\pgfqpoint{3.205333in}{2.729182in}}%
\pgfpathlineto{\pgfqpoint{3.206335in}{2.898273in}}%
\pgfpathlineto{\pgfqpoint{3.207337in}{2.860091in}}%
\pgfpathlineto{\pgfqpoint{3.209342in}{2.677364in}}%
\pgfpathlineto{\pgfqpoint{3.211346in}{2.888727in}}%
\pgfpathlineto{\pgfqpoint{3.212349in}{2.835545in}}%
\pgfpathlineto{\pgfqpoint{3.213351in}{2.831455in}}%
\pgfpathlineto{\pgfqpoint{3.214353in}{2.786455in}}%
\pgfpathlineto{\pgfqpoint{3.216358in}{3.006000in}}%
\pgfpathlineto{\pgfqpoint{3.218362in}{2.935091in}}%
\pgfpathlineto{\pgfqpoint{3.219364in}{2.865545in}}%
\pgfpathlineto{\pgfqpoint{3.222371in}{3.093273in}}%
\pgfpathlineto{\pgfqpoint{3.223373in}{3.348273in}}%
\pgfpathlineto{\pgfqpoint{3.225378in}{3.059182in}}%
\pgfpathlineto{\pgfqpoint{3.226380in}{3.078273in}}%
\pgfpathlineto{\pgfqpoint{3.227382in}{3.150545in}}%
\pgfpathlineto{\pgfqpoint{3.229387in}{3.867818in}}%
\pgfpathlineto{\pgfqpoint{3.232393in}{3.623727in}}%
\pgfpathlineto{\pgfqpoint{3.234398in}{3.933273in}}%
\pgfpathlineto{\pgfqpoint{3.235400in}{3.956455in}}%
\pgfpathlineto{\pgfqpoint{3.236402in}{3.670091in}}%
\pgfpathlineto{\pgfqpoint{3.237405in}{3.672818in}}%
\pgfpathlineto{\pgfqpoint{3.239409in}{3.916909in}}%
\pgfpathlineto{\pgfqpoint{3.240411in}{3.955091in}}%
\pgfpathlineto{\pgfqpoint{3.242416in}{3.131455in}}%
\pgfpathlineto{\pgfqpoint{3.243418in}{3.322364in}}%
\pgfpathlineto{\pgfqpoint{3.244420in}{3.102818in}}%
\pgfpathlineto{\pgfqpoint{3.245423in}{3.258273in}}%
\pgfpathlineto{\pgfqpoint{3.248429in}{2.898273in}}%
\pgfpathlineto{\pgfqpoint{3.249432in}{2.808273in}}%
\pgfpathlineto{\pgfqpoint{3.250434in}{2.943273in}}%
\pgfpathlineto{\pgfqpoint{3.253441in}{2.772818in}}%
\pgfpathlineto{\pgfqpoint{3.254443in}{2.650091in}}%
\pgfpathlineto{\pgfqpoint{3.257450in}{2.894182in}}%
\pgfpathlineto{\pgfqpoint{3.259454in}{2.714182in}}%
\pgfpathlineto{\pgfqpoint{3.262461in}{2.941909in}}%
\pgfpathlineto{\pgfqpoint{3.264465in}{2.800091in}}%
\pgfpathlineto{\pgfqpoint{3.267472in}{2.971909in}}%
\pgfpathlineto{\pgfqpoint{3.268474in}{2.931000in}}%
\pgfpathlineto{\pgfqpoint{3.269476in}{2.974636in}}%
\pgfpathlineto{\pgfqpoint{3.271481in}{2.937818in}}%
\pgfpathlineto{\pgfqpoint{3.272483in}{3.051000in}}%
\pgfpathlineto{\pgfqpoint{3.273485in}{3.329182in}}%
\pgfpathlineto{\pgfqpoint{3.274488in}{3.146455in}}%
\pgfpathlineto{\pgfqpoint{3.275490in}{3.236455in}}%
\pgfpathlineto{\pgfqpoint{3.276492in}{3.033273in}}%
\pgfpathlineto{\pgfqpoint{3.278497in}{3.780545in}}%
\pgfpathlineto{\pgfqpoint{3.279499in}{3.871909in}}%
\pgfpathlineto{\pgfqpoint{3.280501in}{3.798273in}}%
\pgfpathlineto{\pgfqpoint{3.281503in}{3.390545in}}%
\pgfpathlineto{\pgfqpoint{3.283508in}{3.809182in}}%
\pgfpathlineto{\pgfqpoint{3.284510in}{3.981000in}}%
\pgfpathlineto{\pgfqpoint{3.286515in}{3.438273in}}%
\pgfpathlineto{\pgfqpoint{3.288519in}{3.938727in}}%
\pgfpathlineto{\pgfqpoint{3.289521in}{4.046455in}}%
\pgfpathlineto{\pgfqpoint{3.290524in}{3.851455in}}%
\pgfpathlineto{\pgfqpoint{3.292528in}{3.100091in}}%
\pgfpathlineto{\pgfqpoint{3.293530in}{3.021000in}}%
\pgfpathlineto{\pgfqpoint{3.294533in}{3.631909in}}%
\pgfpathlineto{\pgfqpoint{3.295535in}{3.471000in}}%
\pgfpathlineto{\pgfqpoint{3.296537in}{3.018273in}}%
\pgfpathlineto{\pgfqpoint{3.297539in}{3.066000in}}%
\pgfpathlineto{\pgfqpoint{3.299544in}{2.806909in}}%
\pgfpathlineto{\pgfqpoint{3.300546in}{2.896909in}}%
\pgfpathlineto{\pgfqpoint{3.301548in}{2.868273in}}%
\pgfpathlineto{\pgfqpoint{3.302550in}{2.941909in}}%
\pgfpathlineto{\pgfqpoint{3.304555in}{2.730545in}}%
\pgfpathlineto{\pgfqpoint{3.306559in}{2.895545in}}%
\pgfpathlineto{\pgfqpoint{3.307562in}{2.861455in}}%
\pgfpathlineto{\pgfqpoint{3.309566in}{2.706000in}}%
\pgfpathlineto{\pgfqpoint{3.312573in}{2.902364in}}%
\pgfpathlineto{\pgfqpoint{3.314577in}{2.808273in}}%
\pgfpathlineto{\pgfqpoint{3.316582in}{2.860091in}}%
\pgfpathlineto{\pgfqpoint{3.319589in}{3.056455in}}%
\pgfpathlineto{\pgfqpoint{3.320591in}{2.939182in}}%
\pgfpathlineto{\pgfqpoint{3.321593in}{2.951455in}}%
\pgfpathlineto{\pgfqpoint{3.322595in}{3.034636in}}%
\pgfpathlineto{\pgfqpoint{3.323598in}{2.978727in}}%
\pgfpathlineto{\pgfqpoint{3.324600in}{3.311455in}}%
\pgfpathlineto{\pgfqpoint{3.326604in}{3.066000in}}%
\pgfpathlineto{\pgfqpoint{3.327607in}{3.360545in}}%
\pgfpathlineto{\pgfqpoint{3.328609in}{3.248727in}}%
\pgfpathlineto{\pgfqpoint{3.329611in}{3.736909in}}%
\pgfpathlineto{\pgfqpoint{3.331616in}{3.559636in}}%
\pgfpathlineto{\pgfqpoint{3.332618in}{3.659182in}}%
\pgfpathlineto{\pgfqpoint{3.334622in}{4.016455in}}%
\pgfpathlineto{\pgfqpoint{3.336627in}{3.610091in}}%
\pgfpathlineto{\pgfqpoint{3.337629in}{3.656455in}}%
\pgfpathlineto{\pgfqpoint{3.339633in}{3.892364in}}%
\pgfpathlineto{\pgfqpoint{3.341638in}{3.104182in}}%
\pgfpathlineto{\pgfqpoint{3.343642in}{3.741000in}}%
\pgfpathlineto{\pgfqpoint{3.344645in}{3.798273in}}%
\pgfpathlineto{\pgfqpoint{3.345647in}{3.012818in}}%
\pgfpathlineto{\pgfqpoint{3.346649in}{3.056455in}}%
\pgfpathlineto{\pgfqpoint{3.347651in}{2.903727in}}%
\pgfpathlineto{\pgfqpoint{3.348654in}{2.905091in}}%
\pgfpathlineto{\pgfqpoint{3.349656in}{3.411000in}}%
\pgfpathlineto{\pgfqpoint{3.350658in}{2.963727in}}%
\pgfpathlineto{\pgfqpoint{3.351660in}{2.985545in}}%
\pgfpathlineto{\pgfqpoint{3.353665in}{2.748273in}}%
\pgfpathlineto{\pgfqpoint{3.356672in}{2.922818in}}%
\pgfpathlineto{\pgfqpoint{3.358676in}{2.782364in}}%
\pgfpathlineto{\pgfqpoint{3.359678in}{2.766000in}}%
\pgfpathlineto{\pgfqpoint{3.361683in}{2.944636in}}%
\pgfpathlineto{\pgfqpoint{3.364690in}{2.786455in}}%
\pgfpathlineto{\pgfqpoint{3.366694in}{2.976000in}}%
\pgfpathlineto{\pgfqpoint{3.367696in}{2.898273in}}%
\pgfpathlineto{\pgfqpoint{3.368698in}{2.902364in}}%
\pgfpathlineto{\pgfqpoint{3.369701in}{2.909182in}}%
\pgfpathlineto{\pgfqpoint{3.370703in}{2.843727in}}%
\pgfpathlineto{\pgfqpoint{3.371705in}{3.018273in}}%
\pgfpathlineto{\pgfqpoint{3.372707in}{2.895545in}}%
\pgfpathlineto{\pgfqpoint{3.374712in}{3.211909in}}%
\pgfpathlineto{\pgfqpoint{3.375714in}{3.019636in}}%
\pgfpathlineto{\pgfqpoint{3.376716in}{3.121909in}}%
\pgfpathlineto{\pgfqpoint{3.377719in}{3.037364in}}%
\pgfpathlineto{\pgfqpoint{3.379723in}{3.749182in}}%
\pgfpathlineto{\pgfqpoint{3.381728in}{3.259636in}}%
\pgfpathlineto{\pgfqpoint{3.384734in}{3.982364in}}%
\pgfpathlineto{\pgfqpoint{3.387741in}{3.686455in}}%
\pgfpathlineto{\pgfqpoint{3.388743in}{3.931909in}}%
\pgfpathlineto{\pgfqpoint{3.389746in}{3.895091in}}%
\pgfpathlineto{\pgfqpoint{3.391750in}{3.736909in}}%
\pgfpathlineto{\pgfqpoint{3.392752in}{3.195545in}}%
\pgfpathlineto{\pgfqpoint{3.393755in}{3.747818in}}%
\pgfpathlineto{\pgfqpoint{3.394757in}{3.682364in}}%
\pgfpathlineto{\pgfqpoint{3.395759in}{3.428727in}}%
\pgfpathlineto{\pgfqpoint{3.396761in}{3.457364in}}%
\pgfpathlineto{\pgfqpoint{3.397764in}{2.950091in}}%
\pgfpathlineto{\pgfqpoint{3.398766in}{3.250091in}}%
\pgfpathlineto{\pgfqpoint{3.399768in}{2.991000in}}%
\pgfpathlineto{\pgfqpoint{3.400770in}{3.011455in}}%
\pgfpathlineto{\pgfqpoint{3.401773in}{3.037364in}}%
\pgfpathlineto{\pgfqpoint{3.403777in}{2.813727in}}%
\pgfpathlineto{\pgfqpoint{3.406784in}{2.969182in}}%
\pgfpathlineto{\pgfqpoint{3.408788in}{2.821909in}}%
\pgfpathlineto{\pgfqpoint{3.409790in}{2.808273in}}%
\pgfpathlineto{\pgfqpoint{3.411795in}{2.917364in}}%
\pgfpathlineto{\pgfqpoint{3.412797in}{2.839636in}}%
\pgfpathlineto{\pgfqpoint{3.413799in}{2.849182in}}%
\pgfpathlineto{\pgfqpoint{3.414802in}{2.790545in}}%
\pgfpathlineto{\pgfqpoint{3.416806in}{2.944636in}}%
\pgfpathlineto{\pgfqpoint{3.419813in}{2.793273in}}%
\pgfpathlineto{\pgfqpoint{3.421817in}{2.967818in}}%
\pgfpathlineto{\pgfqpoint{3.422820in}{2.914636in}}%
\pgfpathlineto{\pgfqpoint{3.423822in}{2.999182in}}%
\pgfpathlineto{\pgfqpoint{3.424824in}{2.887364in}}%
\pgfpathlineto{\pgfqpoint{3.427831in}{3.051000in}}%
\pgfpathlineto{\pgfqpoint{3.428833in}{3.559636in}}%
\pgfpathlineto{\pgfqpoint{3.430838in}{3.104182in}}%
\pgfpathlineto{\pgfqpoint{3.431840in}{3.091909in}}%
\pgfpathlineto{\pgfqpoint{3.433844in}{3.897818in}}%
\pgfpathlineto{\pgfqpoint{3.436851in}{3.701455in}}%
\pgfpathlineto{\pgfqpoint{3.438856in}{4.012364in}}%
\pgfpathlineto{\pgfqpoint{3.441862in}{3.537818in}}%
\pgfpathlineto{\pgfqpoint{3.442864in}{3.816000in}}%
\pgfpathlineto{\pgfqpoint{3.443867in}{3.753273in}}%
\pgfpathlineto{\pgfqpoint{3.444869in}{3.588273in}}%
\pgfpathlineto{\pgfqpoint{3.445871in}{3.611455in}}%
\pgfpathlineto{\pgfqpoint{3.447876in}{2.971909in}}%
\pgfpathlineto{\pgfqpoint{3.448878in}{3.254182in}}%
\pgfpathlineto{\pgfqpoint{3.452887in}{2.888727in}}%
\pgfpathlineto{\pgfqpoint{3.453889in}{2.876455in}}%
\pgfpathlineto{\pgfqpoint{3.454891in}{2.821909in}}%
\pgfpathlineto{\pgfqpoint{3.455894in}{2.914636in}}%
\pgfpathlineto{\pgfqpoint{3.457898in}{2.875091in}}%
\pgfpathlineto{\pgfqpoint{3.459903in}{2.763273in}}%
\pgfpathlineto{\pgfqpoint{3.460905in}{2.909182in}}%
\pgfpathlineto{\pgfqpoint{3.461907in}{2.861455in}}%
\pgfpathlineto{\pgfqpoint{3.462909in}{2.892818in}}%
\pgfpathlineto{\pgfqpoint{3.463912in}{2.805545in}}%
\pgfpathlineto{\pgfqpoint{3.464914in}{2.830091in}}%
\pgfpathlineto{\pgfqpoint{3.465916in}{3.016909in}}%
\pgfpathlineto{\pgfqpoint{3.466918in}{2.905091in}}%
\pgfpathlineto{\pgfqpoint{3.467921in}{2.962364in}}%
\pgfpathlineto{\pgfqpoint{3.469925in}{2.861455in}}%
\pgfpathlineto{\pgfqpoint{3.472932in}{3.027818in}}%
\pgfpathlineto{\pgfqpoint{3.473934in}{3.010091in}}%
\pgfpathlineto{\pgfqpoint{3.474936in}{2.932364in}}%
\pgfpathlineto{\pgfqpoint{3.476941in}{3.011455in}}%
\pgfpathlineto{\pgfqpoint{3.478945in}{3.698727in}}%
\pgfpathlineto{\pgfqpoint{3.481952in}{3.274636in}}%
\pgfpathlineto{\pgfqpoint{3.483956in}{3.833727in}}%
\pgfpathlineto{\pgfqpoint{3.484959in}{3.877364in}}%
\pgfpathlineto{\pgfqpoint{3.486963in}{3.442364in}}%
\pgfpathlineto{\pgfqpoint{3.487965in}{3.847364in}}%
\pgfpathlineto{\pgfqpoint{3.488968in}{3.817364in}}%
\pgfpathlineto{\pgfqpoint{3.489970in}{3.956455in}}%
\pgfpathlineto{\pgfqpoint{3.490972in}{3.719182in}}%
\pgfpathlineto{\pgfqpoint{3.491974in}{3.199636in}}%
\pgfpathlineto{\pgfqpoint{3.492977in}{3.769636in}}%
\pgfpathlineto{\pgfqpoint{3.493979in}{3.510545in}}%
\pgfpathlineto{\pgfqpoint{3.494981in}{3.700091in}}%
\pgfpathlineto{\pgfqpoint{3.495983in}{3.600545in}}%
\pgfpathlineto{\pgfqpoint{3.497988in}{3.045545in}}%
\pgfpathlineto{\pgfqpoint{3.499992in}{2.978727in}}%
\pgfpathlineto{\pgfqpoint{3.500995in}{3.061909in}}%
\pgfpathlineto{\pgfqpoint{3.504001in}{2.830091in}}%
\pgfpathlineto{\pgfqpoint{3.507008in}{2.913273in}}%
\pgfpathlineto{\pgfqpoint{3.509013in}{2.771455in}}%
\pgfpathlineto{\pgfqpoint{3.511017in}{2.876455in}}%
\pgfpathlineto{\pgfqpoint{3.512019in}{2.931000in}}%
\pgfpathlineto{\pgfqpoint{3.514024in}{2.768727in}}%
\pgfpathlineto{\pgfqpoint{3.515026in}{2.902364in}}%
\pgfpathlineto{\pgfqpoint{3.516028in}{2.860091in}}%
\pgfpathlineto{\pgfqpoint{3.518033in}{2.905091in}}%
\pgfpathlineto{\pgfqpoint{3.519035in}{2.790545in}}%
\pgfpathlineto{\pgfqpoint{3.520037in}{2.903727in}}%
\pgfpathlineto{\pgfqpoint{3.521039in}{2.873727in}}%
\pgfpathlineto{\pgfqpoint{3.522042in}{2.898273in}}%
\pgfpathlineto{\pgfqpoint{3.523044in}{3.007364in}}%
\pgfpathlineto{\pgfqpoint{3.524046in}{2.984182in}}%
\pgfpathlineto{\pgfqpoint{3.525048in}{3.010091in}}%
\pgfpathlineto{\pgfqpoint{3.526051in}{2.980091in}}%
\pgfpathlineto{\pgfqpoint{3.527053in}{3.104182in}}%
\pgfpathlineto{\pgfqpoint{3.528055in}{3.524182in}}%
\pgfpathlineto{\pgfqpoint{3.529057in}{3.381000in}}%
\pgfpathlineto{\pgfqpoint{3.530060in}{3.708273in}}%
\pgfpathlineto{\pgfqpoint{3.531062in}{3.156000in}}%
\pgfpathlineto{\pgfqpoint{3.532064in}{3.629182in}}%
\pgfpathlineto{\pgfqpoint{3.533066in}{3.622364in}}%
\pgfpathlineto{\pgfqpoint{3.535071in}{3.948273in}}%
\pgfpathlineto{\pgfqpoint{3.536073in}{3.623727in}}%
\pgfpathlineto{\pgfqpoint{3.540082in}{4.028727in}}%
\pgfpathlineto{\pgfqpoint{3.542087in}{3.522818in}}%
\pgfpathlineto{\pgfqpoint{3.543089in}{3.741000in}}%
\pgfpathlineto{\pgfqpoint{3.544091in}{3.713727in}}%
\pgfpathlineto{\pgfqpoint{3.545093in}{3.859636in}}%
\pgfpathlineto{\pgfqpoint{3.547098in}{3.175091in}}%
\pgfpathlineto{\pgfqpoint{3.549102in}{3.045545in}}%
\pgfpathlineto{\pgfqpoint{3.550104in}{3.255545in}}%
\pgfpathlineto{\pgfqpoint{3.552109in}{3.040091in}}%
\pgfpathlineto{\pgfqpoint{3.554113in}{2.843727in}}%
\pgfpathlineto{\pgfqpoint{3.556118in}{2.970545in}}%
\pgfpathlineto{\pgfqpoint{3.557120in}{2.959636in}}%
\pgfpathlineto{\pgfqpoint{3.558122in}{2.827364in}}%
\pgfpathlineto{\pgfqpoint{3.562131in}{2.903727in}}%
\pgfpathlineto{\pgfqpoint{3.564136in}{2.802818in}}%
\pgfpathlineto{\pgfqpoint{3.565138in}{2.864182in}}%
\pgfpathlineto{\pgfqpoint{3.566140in}{2.839636in}}%
\pgfpathlineto{\pgfqpoint{3.567143in}{2.941909in}}%
\pgfpathlineto{\pgfqpoint{3.569147in}{2.828727in}}%
\pgfpathlineto{\pgfqpoint{3.570149in}{2.910545in}}%
\pgfpathlineto{\pgfqpoint{3.571152in}{2.811000in}}%
\pgfpathlineto{\pgfqpoint{3.572154in}{2.963727in}}%
\pgfpathlineto{\pgfqpoint{3.573156in}{2.876455in}}%
\pgfpathlineto{\pgfqpoint{3.575161in}{2.921455in}}%
\pgfpathlineto{\pgfqpoint{3.576163in}{2.892818in}}%
\pgfpathlineto{\pgfqpoint{3.577165in}{3.034636in}}%
\pgfpathlineto{\pgfqpoint{3.578167in}{2.995091in}}%
\pgfpathlineto{\pgfqpoint{3.579170in}{3.123273in}}%
\pgfpathlineto{\pgfqpoint{3.580172in}{3.119182in}}%
\pgfpathlineto{\pgfqpoint{3.581174in}{3.056455in}}%
\pgfpathlineto{\pgfqpoint{3.583178in}{3.364636in}}%
\pgfpathlineto{\pgfqpoint{3.584181in}{3.777818in}}%
\pgfpathlineto{\pgfqpoint{3.585183in}{3.700091in}}%
\pgfpathlineto{\pgfqpoint{3.586185in}{3.453273in}}%
\pgfpathlineto{\pgfqpoint{3.589192in}{3.931909in}}%
\pgfpathlineto{\pgfqpoint{3.590194in}{3.839182in}}%
\pgfpathlineto{\pgfqpoint{3.591196in}{3.606000in}}%
\pgfpathlineto{\pgfqpoint{3.592199in}{3.859636in}}%
\pgfpathlineto{\pgfqpoint{3.593201in}{3.757364in}}%
\pgfpathlineto{\pgfqpoint{3.594203in}{3.783273in}}%
\pgfpathlineto{\pgfqpoint{3.595205in}{3.732818in}}%
\pgfpathlineto{\pgfqpoint{3.596208in}{3.121909in}}%
\pgfpathlineto{\pgfqpoint{3.597210in}{3.679636in}}%
\pgfpathlineto{\pgfqpoint{3.598212in}{3.217364in}}%
\pgfpathlineto{\pgfqpoint{3.599214in}{3.736909in}}%
\pgfpathlineto{\pgfqpoint{3.601219in}{3.064636in}}%
\pgfpathlineto{\pgfqpoint{3.603223in}{2.933727in}}%
\pgfpathlineto{\pgfqpoint{3.604226in}{3.014182in}}%
\pgfpathlineto{\pgfqpoint{3.606230in}{2.995091in}}%
\pgfpathlineto{\pgfqpoint{3.608235in}{2.832818in}}%
\pgfpathlineto{\pgfqpoint{3.611241in}{2.951455in}}%
\pgfpathlineto{\pgfqpoint{3.612244in}{2.894182in}}%
\pgfpathlineto{\pgfqpoint{3.613246in}{2.771455in}}%
\pgfpathlineto{\pgfqpoint{3.615250in}{2.832818in}}%
\pgfpathlineto{\pgfqpoint{3.617255in}{2.984182in}}%
\pgfpathlineto{\pgfqpoint{3.618257in}{2.768727in}}%
\pgfpathlineto{\pgfqpoint{3.619259in}{2.832818in}}%
\pgfpathlineto{\pgfqpoint{3.620261in}{2.783727in}}%
\pgfpathlineto{\pgfqpoint{3.622266in}{2.989636in}}%
\pgfpathlineto{\pgfqpoint{3.623268in}{2.899636in}}%
\pgfpathlineto{\pgfqpoint{3.624270in}{2.952818in}}%
\pgfpathlineto{\pgfqpoint{3.625273in}{2.824636in}}%
\pgfpathlineto{\pgfqpoint{3.627277in}{2.974636in}}%
\pgfpathlineto{\pgfqpoint{3.628279in}{3.310091in}}%
\pgfpathlineto{\pgfqpoint{3.629282in}{3.225545in}}%
\pgfpathlineto{\pgfqpoint{3.630284in}{3.012818in}}%
\pgfpathlineto{\pgfqpoint{3.632288in}{3.130091in}}%
\pgfpathlineto{\pgfqpoint{3.634293in}{3.866455in}}%
\pgfpathlineto{\pgfqpoint{3.635295in}{3.569182in}}%
\pgfpathlineto{\pgfqpoint{3.636297in}{3.614182in}}%
\pgfpathlineto{\pgfqpoint{3.637300in}{3.150545in}}%
\pgfpathlineto{\pgfqpoint{3.639304in}{3.889636in}}%
\pgfpathlineto{\pgfqpoint{3.642311in}{3.213273in}}%
\pgfpathlineto{\pgfqpoint{3.644315in}{3.810545in}}%
\pgfpathlineto{\pgfqpoint{3.645318in}{3.634636in}}%
\pgfpathlineto{\pgfqpoint{3.646320in}{3.696000in}}%
\pgfpathlineto{\pgfqpoint{3.648324in}{3.160091in}}%
\pgfpathlineto{\pgfqpoint{3.649327in}{3.488727in}}%
\pgfpathlineto{\pgfqpoint{3.650329in}{3.045545in}}%
\pgfpathlineto{\pgfqpoint{3.651331in}{3.134182in}}%
\pgfpathlineto{\pgfqpoint{3.653335in}{2.988273in}}%
\pgfpathlineto{\pgfqpoint{3.654338in}{2.937818in}}%
\pgfpathlineto{\pgfqpoint{3.655340in}{2.951455in}}%
\pgfpathlineto{\pgfqpoint{3.656342in}{3.016909in}}%
\pgfpathlineto{\pgfqpoint{3.658347in}{2.892818in}}%
\pgfpathlineto{\pgfqpoint{3.660351in}{2.811000in}}%
\pgfpathlineto{\pgfqpoint{3.661353in}{2.995091in}}%
\pgfpathlineto{\pgfqpoint{3.663358in}{2.817818in}}%
\pgfpathlineto{\pgfqpoint{3.664360in}{2.826000in}}%
\pgfpathlineto{\pgfqpoint{3.665362in}{2.816455in}}%
\pgfpathlineto{\pgfqpoint{3.666365in}{2.958273in}}%
\pgfpathlineto{\pgfqpoint{3.667367in}{2.811000in}}%
\pgfpathlineto{\pgfqpoint{3.669371in}{2.836909in}}%
\pgfpathlineto{\pgfqpoint{3.670374in}{2.834182in}}%
\pgfpathlineto{\pgfqpoint{3.671376in}{2.895545in}}%
\pgfpathlineto{\pgfqpoint{3.672378in}{2.843727in}}%
\pgfpathlineto{\pgfqpoint{3.674383in}{2.902364in}}%
\pgfpathlineto{\pgfqpoint{3.675385in}{2.868273in}}%
\pgfpathlineto{\pgfqpoint{3.676387in}{2.952818in}}%
\pgfpathlineto{\pgfqpoint{3.677389in}{2.916000in}}%
\pgfpathlineto{\pgfqpoint{3.678392in}{3.068727in}}%
\pgfpathlineto{\pgfqpoint{3.679394in}{3.061909in}}%
\pgfpathlineto{\pgfqpoint{3.680396in}{3.019636in}}%
\pgfpathlineto{\pgfqpoint{3.681398in}{3.094636in}}%
\pgfpathlineto{\pgfqpoint{3.682401in}{3.063273in}}%
\pgfpathlineto{\pgfqpoint{3.683403in}{3.706909in}}%
\pgfpathlineto{\pgfqpoint{3.684405in}{3.698727in}}%
\pgfpathlineto{\pgfqpoint{3.685407in}{3.593727in}}%
\pgfpathlineto{\pgfqpoint{3.686410in}{3.682364in}}%
\pgfpathlineto{\pgfqpoint{3.687412in}{3.344182in}}%
\pgfpathlineto{\pgfqpoint{3.688414in}{3.904636in}}%
\pgfpathlineto{\pgfqpoint{3.691421in}{3.630545in}}%
\pgfpathlineto{\pgfqpoint{3.692423in}{3.087818in}}%
\pgfpathlineto{\pgfqpoint{3.694427in}{3.832364in}}%
\pgfpathlineto{\pgfqpoint{3.695430in}{3.798273in}}%
\pgfpathlineto{\pgfqpoint{3.697434in}{3.419182in}}%
\pgfpathlineto{\pgfqpoint{3.699439in}{3.691909in}}%
\pgfpathlineto{\pgfqpoint{3.702445in}{3.030545in}}%
\pgfpathlineto{\pgfqpoint{3.703448in}{3.124636in}}%
\pgfpathlineto{\pgfqpoint{3.704450in}{3.042818in}}%
\pgfpathlineto{\pgfqpoint{3.705452in}{3.055091in}}%
\pgfpathlineto{\pgfqpoint{3.707457in}{2.884636in}}%
\pgfpathlineto{\pgfqpoint{3.708459in}{2.933727in}}%
\pgfpathlineto{\pgfqpoint{3.709461in}{2.875091in}}%
\pgfpathlineto{\pgfqpoint{3.710463in}{2.944636in}}%
\pgfpathlineto{\pgfqpoint{3.711466in}{2.920091in}}%
\pgfpathlineto{\pgfqpoint{3.712468in}{2.808273in}}%
\pgfpathlineto{\pgfqpoint{3.713470in}{2.873727in}}%
\pgfpathlineto{\pgfqpoint{3.714472in}{2.809636in}}%
\pgfpathlineto{\pgfqpoint{3.716477in}{2.922818in}}%
\pgfpathlineto{\pgfqpoint{3.717479in}{2.786455in}}%
\pgfpathlineto{\pgfqpoint{3.718481in}{2.794636in}}%
\pgfpathlineto{\pgfqpoint{3.719484in}{2.806909in}}%
\pgfpathlineto{\pgfqpoint{3.721488in}{2.944636in}}%
\pgfpathlineto{\pgfqpoint{3.723492in}{2.819182in}}%
\pgfpathlineto{\pgfqpoint{3.724495in}{2.797364in}}%
\pgfpathlineto{\pgfqpoint{3.728504in}{2.971909in}}%
\pgfpathlineto{\pgfqpoint{3.729506in}{2.838273in}}%
\pgfpathlineto{\pgfqpoint{3.731510in}{2.978727in}}%
\pgfpathlineto{\pgfqpoint{3.733515in}{3.595091in}}%
\pgfpathlineto{\pgfqpoint{3.734517in}{3.124636in}}%
\pgfpathlineto{\pgfqpoint{3.735519in}{3.229636in}}%
\pgfpathlineto{\pgfqpoint{3.736522in}{3.027818in}}%
\pgfpathlineto{\pgfqpoint{3.738526in}{3.859636in}}%
\pgfpathlineto{\pgfqpoint{3.739528in}{3.727364in}}%
\pgfpathlineto{\pgfqpoint{3.740531in}{3.760091in}}%
\pgfpathlineto{\pgfqpoint{3.741533in}{3.634636in}}%
\pgfpathlineto{\pgfqpoint{3.742535in}{3.713727in}}%
\pgfpathlineto{\pgfqpoint{3.743537in}{3.931909in}}%
\pgfpathlineto{\pgfqpoint{3.744540in}{3.784636in}}%
\pgfpathlineto{\pgfqpoint{3.745542in}{3.814636in}}%
\pgfpathlineto{\pgfqpoint{3.747546in}{3.607364in}}%
\pgfpathlineto{\pgfqpoint{3.748549in}{3.742364in}}%
\pgfpathlineto{\pgfqpoint{3.750553in}{3.619636in}}%
\pgfpathlineto{\pgfqpoint{3.751555in}{3.486000in}}%
\pgfpathlineto{\pgfqpoint{3.752558in}{3.128727in}}%
\pgfpathlineto{\pgfqpoint{3.753560in}{3.130091in}}%
\pgfpathlineto{\pgfqpoint{3.754562in}{3.034636in}}%
\pgfpathlineto{\pgfqpoint{3.755564in}{3.169636in}}%
\pgfpathlineto{\pgfqpoint{3.756567in}{2.955545in}}%
\pgfpathlineto{\pgfqpoint{3.757569in}{3.025091in}}%
\pgfpathlineto{\pgfqpoint{3.759573in}{2.909182in}}%
\pgfpathlineto{\pgfqpoint{3.760575in}{2.981455in}}%
\pgfpathlineto{\pgfqpoint{3.761578in}{2.903727in}}%
\pgfpathlineto{\pgfqpoint{3.762580in}{2.926909in}}%
\pgfpathlineto{\pgfqpoint{3.764584in}{2.816455in}}%
\pgfpathlineto{\pgfqpoint{3.765587in}{2.914636in}}%
\pgfpathlineto{\pgfqpoint{3.767591in}{2.798727in}}%
\pgfpathlineto{\pgfqpoint{3.768593in}{2.800091in}}%
\pgfpathlineto{\pgfqpoint{3.769596in}{2.749636in}}%
\pgfpathlineto{\pgfqpoint{3.770598in}{2.909182in}}%
\pgfpathlineto{\pgfqpoint{3.772602in}{2.819182in}}%
\pgfpathlineto{\pgfqpoint{3.773605in}{2.816455in}}%
\pgfpathlineto{\pgfqpoint{3.774607in}{2.727818in}}%
\pgfpathlineto{\pgfqpoint{3.775609in}{2.910545in}}%
\pgfpathlineto{\pgfqpoint{3.776611in}{2.905091in}}%
\pgfpathlineto{\pgfqpoint{3.777614in}{2.871000in}}%
\pgfpathlineto{\pgfqpoint{3.778616in}{2.881909in}}%
\pgfpathlineto{\pgfqpoint{3.779618in}{2.875091in}}%
\pgfpathlineto{\pgfqpoint{3.782625in}{3.247364in}}%
\pgfpathlineto{\pgfqpoint{3.783627in}{3.091909in}}%
\pgfpathlineto{\pgfqpoint{3.785632in}{3.243273in}}%
\pgfpathlineto{\pgfqpoint{3.786634in}{3.101455in}}%
\pgfpathlineto{\pgfqpoint{3.788638in}{3.904636in}}%
\pgfpathlineto{\pgfqpoint{3.790643in}{3.772364in}}%
\pgfpathlineto{\pgfqpoint{3.791645in}{3.266455in}}%
\pgfpathlineto{\pgfqpoint{3.793649in}{3.922364in}}%
\pgfpathlineto{\pgfqpoint{3.794652in}{3.761455in}}%
\pgfpathlineto{\pgfqpoint{3.795654in}{3.768273in}}%
\pgfpathlineto{\pgfqpoint{3.796656in}{3.175091in}}%
\pgfpathlineto{\pgfqpoint{3.797658in}{3.705545in}}%
\pgfpathlineto{\pgfqpoint{3.798661in}{3.682364in}}%
\pgfpathlineto{\pgfqpoint{3.799663in}{3.685091in}}%
\pgfpathlineto{\pgfqpoint{3.800665in}{3.701455in}}%
\pgfpathlineto{\pgfqpoint{3.801667in}{3.247364in}}%
\pgfpathlineto{\pgfqpoint{3.802670in}{3.289636in}}%
\pgfpathlineto{\pgfqpoint{3.803672in}{3.113727in}}%
\pgfpathlineto{\pgfqpoint{3.805676in}{3.232364in}}%
\pgfpathlineto{\pgfqpoint{3.807681in}{3.027818in}}%
\pgfpathlineto{\pgfqpoint{3.808683in}{2.926909in}}%
\pgfpathlineto{\pgfqpoint{3.810688in}{2.985545in}}%
\pgfpathlineto{\pgfqpoint{3.812692in}{2.948727in}}%
\pgfpathlineto{\pgfqpoint{3.814697in}{2.856000in}}%
\pgfpathlineto{\pgfqpoint{3.815699in}{2.879182in}}%
\pgfpathlineto{\pgfqpoint{3.816701in}{2.801455in}}%
\pgfpathlineto{\pgfqpoint{3.817703in}{2.933727in}}%
\pgfpathlineto{\pgfqpoint{3.818706in}{2.785091in}}%
\pgfpathlineto{\pgfqpoint{3.819708in}{2.806909in}}%
\pgfpathlineto{\pgfqpoint{3.820710in}{2.879182in}}%
\pgfpathlineto{\pgfqpoint{3.821712in}{2.782364in}}%
\pgfpathlineto{\pgfqpoint{3.822715in}{2.862818in}}%
\pgfpathlineto{\pgfqpoint{3.823717in}{2.736000in}}%
\pgfpathlineto{\pgfqpoint{3.825721in}{2.869636in}}%
\pgfpathlineto{\pgfqpoint{3.826724in}{2.856000in}}%
\pgfpathlineto{\pgfqpoint{3.827726in}{2.879182in}}%
\pgfpathlineto{\pgfqpoint{3.828728in}{2.869636in}}%
\pgfpathlineto{\pgfqpoint{3.829730in}{2.925545in}}%
\pgfpathlineto{\pgfqpoint{3.830732in}{2.917364in}}%
\pgfpathlineto{\pgfqpoint{3.831735in}{2.955545in}}%
\pgfpathlineto{\pgfqpoint{3.832737in}{3.083727in}}%
\pgfpathlineto{\pgfqpoint{3.833739in}{3.033273in}}%
\pgfpathlineto{\pgfqpoint{3.834741in}{3.340091in}}%
\pgfpathlineto{\pgfqpoint{3.835744in}{3.150545in}}%
\pgfpathlineto{\pgfqpoint{3.836746in}{3.259636in}}%
\pgfpathlineto{\pgfqpoint{3.837748in}{3.713727in}}%
\pgfpathlineto{\pgfqpoint{3.838750in}{3.636000in}}%
\pgfpathlineto{\pgfqpoint{3.839753in}{3.754636in}}%
\pgfpathlineto{\pgfqpoint{3.840755in}{3.671455in}}%
\pgfpathlineto{\pgfqpoint{3.841757in}{3.409636in}}%
\pgfpathlineto{\pgfqpoint{3.842759in}{3.855545in}}%
\pgfpathlineto{\pgfqpoint{3.843762in}{3.787364in}}%
\pgfpathlineto{\pgfqpoint{3.844764in}{3.880091in}}%
\pgfpathlineto{\pgfqpoint{3.846768in}{3.360545in}}%
\pgfpathlineto{\pgfqpoint{3.847771in}{3.784636in}}%
\pgfpathlineto{\pgfqpoint{3.848773in}{3.728727in}}%
\pgfpathlineto{\pgfqpoint{3.849775in}{3.783273in}}%
\pgfpathlineto{\pgfqpoint{3.851780in}{3.338727in}}%
\pgfpathlineto{\pgfqpoint{3.852782in}{3.540545in}}%
\pgfpathlineto{\pgfqpoint{3.853784in}{3.491455in}}%
\pgfpathlineto{\pgfqpoint{3.858795in}{2.946000in}}%
\pgfpathlineto{\pgfqpoint{3.859798in}{3.036000in}}%
\pgfpathlineto{\pgfqpoint{3.860800in}{2.956909in}}%
\pgfpathlineto{\pgfqpoint{3.861802in}{2.971909in}}%
\pgfpathlineto{\pgfqpoint{3.863807in}{2.843727in}}%
\pgfpathlineto{\pgfqpoint{3.864809in}{2.903727in}}%
\pgfpathlineto{\pgfqpoint{3.866813in}{2.835545in}}%
\pgfpathlineto{\pgfqpoint{3.867815in}{2.812364in}}%
\pgfpathlineto{\pgfqpoint{3.868818in}{2.731909in}}%
\pgfpathlineto{\pgfqpoint{3.869820in}{2.846455in}}%
\pgfpathlineto{\pgfqpoint{3.870822in}{2.806909in}}%
\pgfpathlineto{\pgfqpoint{3.871824in}{2.845091in}}%
\pgfpathlineto{\pgfqpoint{3.872827in}{2.828727in}}%
\pgfpathlineto{\pgfqpoint{3.873829in}{2.703273in}}%
\pgfpathlineto{\pgfqpoint{3.874831in}{2.787818in}}%
\pgfpathlineto{\pgfqpoint{3.875833in}{2.763273in}}%
\pgfpathlineto{\pgfqpoint{3.876836in}{2.868273in}}%
\pgfpathlineto{\pgfqpoint{3.877838in}{2.841000in}}%
\pgfpathlineto{\pgfqpoint{3.878840in}{2.763273in}}%
\pgfpathlineto{\pgfqpoint{3.882849in}{2.965091in}}%
\pgfpathlineto{\pgfqpoint{3.883851in}{2.886000in}}%
\pgfpathlineto{\pgfqpoint{3.884854in}{3.019636in}}%
\pgfpathlineto{\pgfqpoint{3.885856in}{2.922818in}}%
\pgfpathlineto{\pgfqpoint{3.889865in}{3.652364in}}%
\pgfpathlineto{\pgfqpoint{3.890867in}{3.022364in}}%
\pgfpathlineto{\pgfqpoint{3.892872in}{3.754636in}}%
\pgfpathlineto{\pgfqpoint{3.893874in}{3.837818in}}%
\pgfpathlineto{\pgfqpoint{3.894876in}{3.825545in}}%
\pgfpathlineto{\pgfqpoint{3.895878in}{3.244636in}}%
\pgfpathlineto{\pgfqpoint{3.897883in}{3.754636in}}%
\pgfpathlineto{\pgfqpoint{3.898885in}{3.771000in}}%
\pgfpathlineto{\pgfqpoint{3.899887in}{3.856909in}}%
\pgfpathlineto{\pgfqpoint{3.901892in}{3.416455in}}%
\pgfpathlineto{\pgfqpoint{3.903896in}{3.694636in}}%
\pgfpathlineto{\pgfqpoint{3.904898in}{3.720545in}}%
\pgfpathlineto{\pgfqpoint{3.906903in}{3.117818in}}%
\pgfpathlineto{\pgfqpoint{3.907905in}{3.126000in}}%
\pgfpathlineto{\pgfqpoint{3.908907in}{3.111000in}}%
\pgfpathlineto{\pgfqpoint{3.909910in}{3.307364in}}%
\pgfpathlineto{\pgfqpoint{3.910912in}{3.003273in}}%
\pgfpathlineto{\pgfqpoint{3.911914in}{3.057818in}}%
\pgfpathlineto{\pgfqpoint{3.912916in}{2.914636in}}%
\pgfpathlineto{\pgfqpoint{3.914921in}{2.948727in}}%
\pgfpathlineto{\pgfqpoint{3.915923in}{2.888727in}}%
\pgfpathlineto{\pgfqpoint{3.916925in}{2.906455in}}%
\pgfpathlineto{\pgfqpoint{3.918930in}{2.796000in}}%
\pgfpathlineto{\pgfqpoint{3.919932in}{2.865545in}}%
\pgfpathlineto{\pgfqpoint{3.920934in}{2.838273in}}%
\pgfpathlineto{\pgfqpoint{3.921937in}{2.864182in}}%
\pgfpathlineto{\pgfqpoint{3.923941in}{2.761909in}}%
\pgfpathlineto{\pgfqpoint{3.924943in}{2.834182in}}%
\pgfpathlineto{\pgfqpoint{3.925946in}{2.763273in}}%
\pgfpathlineto{\pgfqpoint{3.926948in}{2.858727in}}%
\pgfpathlineto{\pgfqpoint{3.928952in}{2.768727in}}%
\pgfpathlineto{\pgfqpoint{3.929955in}{2.861455in}}%
\pgfpathlineto{\pgfqpoint{3.930957in}{2.801455in}}%
\pgfpathlineto{\pgfqpoint{3.931959in}{2.920091in}}%
\pgfpathlineto{\pgfqpoint{3.932961in}{2.898273in}}%
\pgfpathlineto{\pgfqpoint{3.934966in}{2.932364in}}%
\pgfpathlineto{\pgfqpoint{3.935968in}{2.921455in}}%
\pgfpathlineto{\pgfqpoint{3.936970in}{3.091909in}}%
\pgfpathlineto{\pgfqpoint{3.937972in}{3.063273in}}%
\pgfpathlineto{\pgfqpoint{3.938975in}{3.416455in}}%
\pgfpathlineto{\pgfqpoint{3.940979in}{3.055091in}}%
\pgfpathlineto{\pgfqpoint{3.942984in}{3.660545in}}%
\pgfpathlineto{\pgfqpoint{3.944988in}{3.777818in}}%
\pgfpathlineto{\pgfqpoint{3.945990in}{3.521455in}}%
\pgfpathlineto{\pgfqpoint{3.947995in}{3.773727in}}%
\pgfpathlineto{\pgfqpoint{3.948997in}{3.843273in}}%
\pgfpathlineto{\pgfqpoint{3.949999in}{3.792818in}}%
\pgfpathlineto{\pgfqpoint{3.951002in}{3.220091in}}%
\pgfpathlineto{\pgfqpoint{3.952004in}{3.700091in}}%
\pgfpathlineto{\pgfqpoint{3.953006in}{3.670091in}}%
\pgfpathlineto{\pgfqpoint{3.954008in}{3.779182in}}%
\pgfpathlineto{\pgfqpoint{3.955011in}{3.607364in}}%
\pgfpathlineto{\pgfqpoint{3.957015in}{3.126000in}}%
\pgfpathlineto{\pgfqpoint{3.958017in}{3.135545in}}%
\pgfpathlineto{\pgfqpoint{3.959020in}{3.535091in}}%
\pgfpathlineto{\pgfqpoint{3.961024in}{3.031909in}}%
\pgfpathlineto{\pgfqpoint{3.962026in}{3.008727in}}%
\pgfpathlineto{\pgfqpoint{3.963029in}{2.901000in}}%
\pgfpathlineto{\pgfqpoint{3.964031in}{2.982818in}}%
\pgfpathlineto{\pgfqpoint{3.967038in}{2.865545in}}%
\pgfpathlineto{\pgfqpoint{3.968040in}{2.763273in}}%
\pgfpathlineto{\pgfqpoint{3.970044in}{2.849182in}}%
\pgfpathlineto{\pgfqpoint{3.971046in}{2.783727in}}%
\pgfpathlineto{\pgfqpoint{3.972049in}{2.816455in}}%
\pgfpathlineto{\pgfqpoint{3.973051in}{2.682818in}}%
\pgfpathlineto{\pgfqpoint{3.975055in}{2.791909in}}%
\pgfpathlineto{\pgfqpoint{3.976058in}{2.811000in}}%
\pgfpathlineto{\pgfqpoint{3.977060in}{2.875091in}}%
\pgfpathlineto{\pgfqpoint{3.978062in}{2.745545in}}%
\pgfpathlineto{\pgfqpoint{3.980067in}{2.816455in}}%
\pgfpathlineto{\pgfqpoint{3.982071in}{2.910545in}}%
\pgfpathlineto{\pgfqpoint{3.983073in}{2.918727in}}%
\pgfpathlineto{\pgfqpoint{3.984076in}{2.943273in}}%
\pgfpathlineto{\pgfqpoint{3.985078in}{2.894182in}}%
\pgfpathlineto{\pgfqpoint{3.986080in}{2.935091in}}%
\pgfpathlineto{\pgfqpoint{3.988085in}{3.106909in}}%
\pgfpathlineto{\pgfqpoint{3.989087in}{3.581455in}}%
\pgfpathlineto{\pgfqpoint{3.990089in}{3.076909in}}%
\pgfpathlineto{\pgfqpoint{3.991091in}{3.097364in}}%
\pgfpathlineto{\pgfqpoint{3.994098in}{3.753273in}}%
\pgfpathlineto{\pgfqpoint{3.995100in}{3.547364in}}%
\pgfpathlineto{\pgfqpoint{3.997105in}{3.741000in}}%
\pgfpathlineto{\pgfqpoint{3.998107in}{3.723273in}}%
\pgfpathlineto{\pgfqpoint{3.999109in}{3.825545in}}%
\pgfpathlineto{\pgfqpoint{4.001114in}{3.558273in}}%
\pgfpathlineto{\pgfqpoint{4.004121in}{3.736909in}}%
\pgfpathlineto{\pgfqpoint{4.007127in}{3.161455in}}%
\pgfpathlineto{\pgfqpoint{4.009132in}{3.405545in}}%
\pgfpathlineto{\pgfqpoint{4.011136in}{3.060545in}}%
\pgfpathlineto{\pgfqpoint{4.013141in}{2.944636in}}%
\pgfpathlineto{\pgfqpoint{4.014143in}{3.025091in}}%
\pgfpathlineto{\pgfqpoint{4.015145in}{2.928273in}}%
\pgfpathlineto{\pgfqpoint{4.016147in}{2.969182in}}%
\pgfpathlineto{\pgfqpoint{4.018152in}{2.772818in}}%
\pgfpathlineto{\pgfqpoint{4.019154in}{2.875091in}}%
\pgfpathlineto{\pgfqpoint{4.020156in}{2.841000in}}%
\pgfpathlineto{\pgfqpoint{4.021159in}{2.922818in}}%
\pgfpathlineto{\pgfqpoint{4.023163in}{2.725091in}}%
\pgfpathlineto{\pgfqpoint{4.024165in}{2.789182in}}%
\pgfpathlineto{\pgfqpoint{4.025168in}{2.771455in}}%
\pgfpathlineto{\pgfqpoint{4.026170in}{2.888727in}}%
\pgfpathlineto{\pgfqpoint{4.028174in}{2.719636in}}%
\pgfpathlineto{\pgfqpoint{4.030179in}{2.767364in}}%
\pgfpathlineto{\pgfqpoint{4.031181in}{2.881909in}}%
\pgfpathlineto{\pgfqpoint{4.032183in}{2.877818in}}%
\pgfpathlineto{\pgfqpoint{4.033186in}{2.835545in}}%
\pgfpathlineto{\pgfqpoint{4.034188in}{2.838273in}}%
\pgfpathlineto{\pgfqpoint{4.035190in}{2.860091in}}%
\pgfpathlineto{\pgfqpoint{4.039199in}{3.127364in}}%
\pgfpathlineto{\pgfqpoint{4.040201in}{2.969182in}}%
\pgfpathlineto{\pgfqpoint{4.041203in}{3.250091in}}%
\pgfpathlineto{\pgfqpoint{4.042206in}{3.083727in}}%
\pgfpathlineto{\pgfqpoint{4.044210in}{3.734182in}}%
\pgfpathlineto{\pgfqpoint{4.045212in}{3.540545in}}%
\pgfpathlineto{\pgfqpoint{4.046215in}{3.661909in}}%
\pgfpathlineto{\pgfqpoint{4.047217in}{3.592364in}}%
\pgfpathlineto{\pgfqpoint{4.049221in}{3.841909in}}%
\pgfpathlineto{\pgfqpoint{4.052228in}{3.619636in}}%
\pgfpathlineto{\pgfqpoint{4.054233in}{3.817364in}}%
\pgfpathlineto{\pgfqpoint{4.057239in}{3.316909in}}%
\pgfpathlineto{\pgfqpoint{4.059244in}{3.649636in}}%
\pgfpathlineto{\pgfqpoint{4.062251in}{2.974636in}}%
\pgfpathlineto{\pgfqpoint{4.064255in}{3.087818in}}%
\pgfpathlineto{\pgfqpoint{4.066260in}{3.022364in}}%
\pgfpathlineto{\pgfqpoint{4.067262in}{2.828727in}}%
\pgfpathlineto{\pgfqpoint{4.069266in}{2.935091in}}%
\pgfpathlineto{\pgfqpoint{4.070269in}{2.935091in}}%
\pgfpathlineto{\pgfqpoint{4.072273in}{2.800091in}}%
\pgfpathlineto{\pgfqpoint{4.074278in}{2.890091in}}%
\pgfpathlineto{\pgfqpoint{4.075280in}{2.857364in}}%
\pgfpathlineto{\pgfqpoint{4.076282in}{2.881909in}}%
\pgfpathlineto{\pgfqpoint{4.077284in}{2.781000in}}%
\pgfpathlineto{\pgfqpoint{4.078286in}{2.798727in}}%
\pgfpathlineto{\pgfqpoint{4.079289in}{2.794636in}}%
\pgfpathlineto{\pgfqpoint{4.080291in}{2.819182in}}%
\pgfpathlineto{\pgfqpoint{4.081293in}{2.946000in}}%
\pgfpathlineto{\pgfqpoint{4.082295in}{2.789182in}}%
\pgfpathlineto{\pgfqpoint{4.084300in}{2.838273in}}%
\pgfpathlineto{\pgfqpoint{4.085302in}{3.081000in}}%
\pgfpathlineto{\pgfqpoint{4.086304in}{2.861455in}}%
\pgfpathlineto{\pgfqpoint{4.087307in}{2.944636in}}%
\pgfpathlineto{\pgfqpoint{4.088309in}{2.823273in}}%
\pgfpathlineto{\pgfqpoint{4.089311in}{3.102818in}}%
\pgfpathlineto{\pgfqpoint{4.090313in}{3.750545in}}%
\pgfpathlineto{\pgfqpoint{4.092318in}{2.967818in}}%
\pgfpathlineto{\pgfqpoint{4.093320in}{3.046909in}}%
\pgfpathlineto{\pgfqpoint{4.095325in}{3.700091in}}%
\pgfpathlineto{\pgfqpoint{4.096327in}{3.664636in}}%
\pgfpathlineto{\pgfqpoint{4.097329in}{3.682364in}}%
\pgfpathlineto{\pgfqpoint{4.098331in}{3.730091in}}%
\pgfpathlineto{\pgfqpoint{4.099334in}{3.604636in}}%
\pgfpathlineto{\pgfqpoint{4.100336in}{3.289636in}}%
\pgfpathlineto{\pgfqpoint{4.101338in}{3.731455in}}%
\pgfpathlineto{\pgfqpoint{4.102340in}{3.698727in}}%
\pgfpathlineto{\pgfqpoint{4.103343in}{3.168273in}}%
\pgfpathlineto{\pgfqpoint{4.104345in}{3.175091in}}%
\pgfpathlineto{\pgfqpoint{4.105347in}{3.150545in}}%
\pgfpathlineto{\pgfqpoint{4.106349in}{3.359182in}}%
\pgfpathlineto{\pgfqpoint{4.107352in}{3.236455in}}%
\pgfpathlineto{\pgfqpoint{4.108354in}{3.281455in}}%
\pgfpathlineto{\pgfqpoint{4.109356in}{3.030545in}}%
\pgfpathlineto{\pgfqpoint{4.110358in}{3.101455in}}%
\pgfpathlineto{\pgfqpoint{4.111361in}{2.969182in}}%
\pgfpathlineto{\pgfqpoint{4.112363in}{2.997818in}}%
\pgfpathlineto{\pgfqpoint{4.114367in}{3.081000in}}%
\pgfpathlineto{\pgfqpoint{4.115369in}{3.081000in}}%
\pgfpathlineto{\pgfqpoint{4.117374in}{2.805545in}}%
\pgfpathlineto{\pgfqpoint{4.120381in}{3.014182in}}%
\pgfpathlineto{\pgfqpoint{4.121383in}{2.928273in}}%
\pgfpathlineto{\pgfqpoint{4.122385in}{2.723727in}}%
\pgfpathlineto{\pgfqpoint{4.124390in}{2.800091in}}%
\pgfpathlineto{\pgfqpoint{4.126394in}{2.971909in}}%
\pgfpathlineto{\pgfqpoint{4.128399in}{2.729182in}}%
\pgfpathlineto{\pgfqpoint{4.131405in}{2.955545in}}%
\pgfpathlineto{\pgfqpoint{4.134412in}{2.768727in}}%
\pgfpathlineto{\pgfqpoint{4.135414in}{3.036000in}}%
\pgfpathlineto{\pgfqpoint{4.136417in}{2.973273in}}%
\pgfpathlineto{\pgfqpoint{4.137419in}{3.031909in}}%
\pgfpathlineto{\pgfqpoint{4.139423in}{2.913273in}}%
\pgfpathlineto{\pgfqpoint{4.140426in}{3.100091in}}%
\pgfpathlineto{\pgfqpoint{4.141428in}{3.034636in}}%
\pgfpathlineto{\pgfqpoint{4.142430in}{3.162818in}}%
\pgfpathlineto{\pgfqpoint{4.144435in}{3.618273in}}%
\pgfpathlineto{\pgfqpoint{4.145437in}{3.663273in}}%
\pgfpathlineto{\pgfqpoint{4.146439in}{3.176455in}}%
\pgfpathlineto{\pgfqpoint{4.148443in}{3.753273in}}%
\pgfpathlineto{\pgfqpoint{4.149446in}{3.743727in}}%
\pgfpathlineto{\pgfqpoint{4.150448in}{3.757364in}}%
\pgfpathlineto{\pgfqpoint{4.151450in}{3.486000in}}%
\pgfpathlineto{\pgfqpoint{4.153455in}{3.799636in}}%
\pgfpathlineto{\pgfqpoint{4.154457in}{3.712364in}}%
\pgfpathlineto{\pgfqpoint{4.155459in}{3.764182in}}%
\pgfpathlineto{\pgfqpoint{4.157464in}{3.368727in}}%
\pgfpathlineto{\pgfqpoint{4.158466in}{3.438273in}}%
\pgfpathlineto{\pgfqpoint{4.159468in}{3.327818in}}%
\pgfpathlineto{\pgfqpoint{4.160470in}{3.599182in}}%
\pgfpathlineto{\pgfqpoint{4.162475in}{3.064636in}}%
\pgfpathlineto{\pgfqpoint{4.163477in}{2.984182in}}%
\pgfpathlineto{\pgfqpoint{4.165482in}{3.106909in}}%
\pgfpathlineto{\pgfqpoint{4.169491in}{2.869636in}}%
\pgfpathlineto{\pgfqpoint{4.170493in}{2.989636in}}%
\pgfpathlineto{\pgfqpoint{4.171495in}{2.909182in}}%
\pgfpathlineto{\pgfqpoint{4.172497in}{2.940545in}}%
\pgfpathlineto{\pgfqpoint{4.174502in}{2.805545in}}%
\pgfpathlineto{\pgfqpoint{4.175504in}{2.898273in}}%
\pgfpathlineto{\pgfqpoint{4.176506in}{2.821909in}}%
\pgfpathlineto{\pgfqpoint{4.177509in}{2.827364in}}%
\pgfpathlineto{\pgfqpoint{4.179513in}{2.764636in}}%
\pgfpathlineto{\pgfqpoint{4.180515in}{2.860091in}}%
\pgfpathlineto{\pgfqpoint{4.181518in}{2.816455in}}%
\pgfpathlineto{\pgfqpoint{4.182520in}{2.841000in}}%
\pgfpathlineto{\pgfqpoint{4.184524in}{2.772818in}}%
\pgfpathlineto{\pgfqpoint{4.186529in}{2.881909in}}%
\pgfpathlineto{\pgfqpoint{4.187531in}{2.866909in}}%
\pgfpathlineto{\pgfqpoint{4.188533in}{2.816455in}}%
\pgfpathlineto{\pgfqpoint{4.189535in}{2.835545in}}%
\pgfpathlineto{\pgfqpoint{4.191540in}{2.965091in}}%
\pgfpathlineto{\pgfqpoint{4.192542in}{3.003273in}}%
\pgfpathlineto{\pgfqpoint{4.194547in}{2.958273in}}%
\pgfpathlineto{\pgfqpoint{4.195549in}{3.090545in}}%
\pgfpathlineto{\pgfqpoint{4.196551in}{3.014182in}}%
\pgfpathlineto{\pgfqpoint{4.197553in}{3.547364in}}%
\pgfpathlineto{\pgfqpoint{4.198556in}{3.378273in}}%
\pgfpathlineto{\pgfqpoint{4.199558in}{3.507818in}}%
\pgfpathlineto{\pgfqpoint{4.200560in}{3.473727in}}%
\pgfpathlineto{\pgfqpoint{4.201562in}{3.124636in}}%
\pgfpathlineto{\pgfqpoint{4.203567in}{3.747818in}}%
\pgfpathlineto{\pgfqpoint{4.205571in}{3.641455in}}%
\pgfpathlineto{\pgfqpoint{4.206574in}{3.203727in}}%
\pgfpathlineto{\pgfqpoint{4.208578in}{3.769636in}}%
\pgfpathlineto{\pgfqpoint{4.210583in}{3.705545in}}%
\pgfpathlineto{\pgfqpoint{4.211585in}{3.454636in}}%
\pgfpathlineto{\pgfqpoint{4.212587in}{3.629182in}}%
\pgfpathlineto{\pgfqpoint{4.213589in}{3.571909in}}%
\pgfpathlineto{\pgfqpoint{4.214592in}{3.608727in}}%
\pgfpathlineto{\pgfqpoint{4.215594in}{3.569182in}}%
\pgfpathlineto{\pgfqpoint{4.217598in}{3.117818in}}%
\pgfpathlineto{\pgfqpoint{4.218600in}{3.086455in}}%
\pgfpathlineto{\pgfqpoint{4.219603in}{3.120545in}}%
\pgfpathlineto{\pgfqpoint{4.220605in}{3.207818in}}%
\pgfpathlineto{\pgfqpoint{4.222609in}{3.001909in}}%
\pgfpathlineto{\pgfqpoint{4.223612in}{2.914636in}}%
\pgfpathlineto{\pgfqpoint{4.224614in}{2.986909in}}%
\pgfpathlineto{\pgfqpoint{4.225616in}{2.974636in}}%
\pgfpathlineto{\pgfqpoint{4.229625in}{2.806909in}}%
\pgfpathlineto{\pgfqpoint{4.230627in}{2.887364in}}%
\pgfpathlineto{\pgfqpoint{4.231630in}{2.881909in}}%
\pgfpathlineto{\pgfqpoint{4.232632in}{2.909182in}}%
\pgfpathlineto{\pgfqpoint{4.234636in}{2.775545in}}%
\pgfpathlineto{\pgfqpoint{4.235639in}{2.836909in}}%
\pgfpathlineto{\pgfqpoint{4.236641in}{2.782364in}}%
\pgfpathlineto{\pgfqpoint{4.237643in}{2.826000in}}%
\pgfpathlineto{\pgfqpoint{4.239648in}{2.782364in}}%
\pgfpathlineto{\pgfqpoint{4.240650in}{2.887364in}}%
\pgfpathlineto{\pgfqpoint{4.241652in}{2.872364in}}%
\pgfpathlineto{\pgfqpoint{4.242654in}{2.877818in}}%
\pgfpathlineto{\pgfqpoint{4.243657in}{2.830091in}}%
\pgfpathlineto{\pgfqpoint{4.245661in}{2.895545in}}%
\pgfpathlineto{\pgfqpoint{4.246663in}{2.933727in}}%
\pgfpathlineto{\pgfqpoint{4.247666in}{3.025091in}}%
\pgfpathlineto{\pgfqpoint{4.248668in}{2.969182in}}%
\pgfpathlineto{\pgfqpoint{4.251675in}{3.057818in}}%
\pgfpathlineto{\pgfqpoint{4.252677in}{3.499636in}}%
\pgfpathlineto{\pgfqpoint{4.253679in}{3.446455in}}%
\pgfpathlineto{\pgfqpoint{4.255683in}{3.222818in}}%
\pgfpathlineto{\pgfqpoint{4.256686in}{3.081000in}}%
\pgfpathlineto{\pgfqpoint{4.258690in}{3.690545in}}%
\pgfpathlineto{\pgfqpoint{4.259692in}{3.664636in}}%
\pgfpathlineto{\pgfqpoint{4.260695in}{3.701455in}}%
\pgfpathlineto{\pgfqpoint{4.261697in}{3.368727in}}%
\pgfpathlineto{\pgfqpoint{4.262699in}{3.735545in}}%
\pgfpathlineto{\pgfqpoint{4.263701in}{3.702818in}}%
\pgfpathlineto{\pgfqpoint{4.265706in}{3.614182in}}%
\pgfpathlineto{\pgfqpoint{4.266708in}{3.396000in}}%
\pgfpathlineto{\pgfqpoint{4.267710in}{3.715091in}}%
\pgfpathlineto{\pgfqpoint{4.268713in}{3.712364in}}%
\pgfpathlineto{\pgfqpoint{4.269715in}{3.675545in}}%
\pgfpathlineto{\pgfqpoint{4.270717in}{3.490091in}}%
\pgfpathlineto{\pgfqpoint{4.271719in}{3.079636in}}%
\pgfpathlineto{\pgfqpoint{4.272722in}{3.169636in}}%
\pgfpathlineto{\pgfqpoint{4.273724in}{3.134182in}}%
\pgfpathlineto{\pgfqpoint{4.274726in}{3.301909in}}%
\pgfpathlineto{\pgfqpoint{4.276731in}{3.019636in}}%
\pgfpathlineto{\pgfqpoint{4.277733in}{3.011455in}}%
\pgfpathlineto{\pgfqpoint{4.278735in}{2.970545in}}%
\pgfpathlineto{\pgfqpoint{4.280740in}{3.006000in}}%
\pgfpathlineto{\pgfqpoint{4.281742in}{2.877818in}}%
\pgfpathlineto{\pgfqpoint{4.282744in}{2.905091in}}%
\pgfpathlineto{\pgfqpoint{4.283746in}{2.819182in}}%
\pgfpathlineto{\pgfqpoint{4.285751in}{2.924182in}}%
\pgfpathlineto{\pgfqpoint{4.286753in}{2.831455in}}%
\pgfpathlineto{\pgfqpoint{4.287755in}{2.877818in}}%
\pgfpathlineto{\pgfqpoint{4.288758in}{2.771455in}}%
\pgfpathlineto{\pgfqpoint{4.290762in}{2.851909in}}%
\pgfpathlineto{\pgfqpoint{4.291764in}{2.862818in}}%
\pgfpathlineto{\pgfqpoint{4.292766in}{2.860091in}}%
\pgfpathlineto{\pgfqpoint{4.293769in}{2.760545in}}%
\pgfpathlineto{\pgfqpoint{4.294771in}{2.782364in}}%
\pgfpathlineto{\pgfqpoint{4.296775in}{2.872364in}}%
\pgfpathlineto{\pgfqpoint{4.297778in}{2.877818in}}%
\pgfpathlineto{\pgfqpoint{4.298780in}{2.805545in}}%
\pgfpathlineto{\pgfqpoint{4.299782in}{2.817818in}}%
\pgfpathlineto{\pgfqpoint{4.302789in}{2.999182in}}%
\pgfpathlineto{\pgfqpoint{4.303791in}{2.920091in}}%
\pgfpathlineto{\pgfqpoint{4.305796in}{2.950091in}}%
\pgfpathlineto{\pgfqpoint{4.306798in}{3.001909in}}%
\pgfpathlineto{\pgfqpoint{4.307800in}{3.498273in}}%
\pgfpathlineto{\pgfqpoint{4.308802in}{3.181909in}}%
\pgfpathlineto{\pgfqpoint{4.309805in}{3.301909in}}%
\pgfpathlineto{\pgfqpoint{4.310807in}{3.075545in}}%
\pgfpathlineto{\pgfqpoint{4.311809in}{3.120545in}}%
\pgfpathlineto{\pgfqpoint{4.312811in}{3.668727in}}%
\pgfpathlineto{\pgfqpoint{4.313814in}{3.615545in}}%
\pgfpathlineto{\pgfqpoint{4.314816in}{3.645545in}}%
\pgfpathlineto{\pgfqpoint{4.316820in}{3.346909in}}%
\pgfpathlineto{\pgfqpoint{4.318825in}{3.711000in}}%
\pgfpathlineto{\pgfqpoint{4.319827in}{3.701455in}}%
\pgfpathlineto{\pgfqpoint{4.321832in}{3.432818in}}%
\pgfpathlineto{\pgfqpoint{4.322834in}{3.697364in}}%
\pgfpathlineto{\pgfqpoint{4.323836in}{3.685091in}}%
\pgfpathlineto{\pgfqpoint{4.324838in}{3.681000in}}%
\pgfpathlineto{\pgfqpoint{4.325840in}{3.521455in}}%
\pgfpathlineto{\pgfqpoint{4.326843in}{3.150545in}}%
\pgfpathlineto{\pgfqpoint{4.327845in}{3.301909in}}%
\pgfpathlineto{\pgfqpoint{4.328847in}{3.246000in}}%
\pgfpathlineto{\pgfqpoint{4.329849in}{3.366000in}}%
\pgfpathlineto{\pgfqpoint{4.331854in}{3.064636in}}%
\pgfpathlineto{\pgfqpoint{4.332856in}{3.056455in}}%
\pgfpathlineto{\pgfqpoint{4.333858in}{2.989636in}}%
\pgfpathlineto{\pgfqpoint{4.334861in}{3.048273in}}%
\pgfpathlineto{\pgfqpoint{4.338870in}{2.866909in}}%
\pgfpathlineto{\pgfqpoint{4.339872in}{2.917364in}}%
\pgfpathlineto{\pgfqpoint{4.340874in}{2.899636in}}%
\pgfpathlineto{\pgfqpoint{4.341876in}{2.907818in}}%
\pgfpathlineto{\pgfqpoint{4.343881in}{2.815091in}}%
\pgfpathlineto{\pgfqpoint{4.344883in}{2.826000in}}%
\pgfpathlineto{\pgfqpoint{4.345885in}{2.864182in}}%
\pgfpathlineto{\pgfqpoint{4.347890in}{2.838273in}}%
\pgfpathlineto{\pgfqpoint{4.348892in}{2.787818in}}%
\pgfpathlineto{\pgfqpoint{4.350897in}{2.890091in}}%
\pgfpathlineto{\pgfqpoint{4.351899in}{2.838273in}}%
\pgfpathlineto{\pgfqpoint{4.352901in}{2.854636in}}%
\pgfpathlineto{\pgfqpoint{4.353903in}{2.789182in}}%
\pgfpathlineto{\pgfqpoint{4.356910in}{2.901000in}}%
\pgfpathlineto{\pgfqpoint{4.357912in}{2.903727in}}%
\pgfpathlineto{\pgfqpoint{4.358915in}{2.841000in}}%
\pgfpathlineto{\pgfqpoint{4.361921in}{3.052364in}}%
\pgfpathlineto{\pgfqpoint{4.362923in}{3.033273in}}%
\pgfpathlineto{\pgfqpoint{4.363926in}{2.980091in}}%
\pgfpathlineto{\pgfqpoint{4.365930in}{3.064636in}}%
\pgfpathlineto{\pgfqpoint{4.366932in}{3.192818in}}%
\pgfpathlineto{\pgfqpoint{4.367935in}{3.589636in}}%
\pgfpathlineto{\pgfqpoint{4.370941in}{3.076909in}}%
\pgfpathlineto{\pgfqpoint{4.373948in}{3.776455in}}%
\pgfpathlineto{\pgfqpoint{4.374950in}{3.762818in}}%
\pgfpathlineto{\pgfqpoint{4.376955in}{3.199636in}}%
\pgfpathlineto{\pgfqpoint{4.378959in}{3.694636in}}%
\pgfpathlineto{\pgfqpoint{4.379962in}{3.720545in}}%
\pgfpathlineto{\pgfqpoint{4.382968in}{3.248727in}}%
\pgfpathlineto{\pgfqpoint{4.383971in}{3.211909in}}%
\pgfpathlineto{\pgfqpoint{4.384973in}{3.312818in}}%
\pgfpathlineto{\pgfqpoint{4.386977in}{3.064636in}}%
\pgfpathlineto{\pgfqpoint{4.388982in}{2.981455in}}%
\pgfpathlineto{\pgfqpoint{4.389984in}{3.045545in}}%
\pgfpathlineto{\pgfqpoint{4.391989in}{2.969182in}}%
\pgfpathlineto{\pgfqpoint{4.392991in}{2.954182in}}%
\pgfpathlineto{\pgfqpoint{4.393993in}{2.903727in}}%
\pgfpathlineto{\pgfqpoint{4.394995in}{2.951455in}}%
\pgfpathlineto{\pgfqpoint{4.395997in}{2.884636in}}%
\pgfpathlineto{\pgfqpoint{4.397000in}{2.898273in}}%
\pgfpathlineto{\pgfqpoint{4.398002in}{2.881909in}}%
\pgfpathlineto{\pgfqpoint{4.399004in}{2.832818in}}%
\pgfpathlineto{\pgfqpoint{4.400006in}{2.913273in}}%
\pgfpathlineto{\pgfqpoint{4.401009in}{2.850545in}}%
\pgfpathlineto{\pgfqpoint{4.402011in}{2.856000in}}%
\pgfpathlineto{\pgfqpoint{4.404015in}{2.791909in}}%
\pgfpathlineto{\pgfqpoint{4.406020in}{2.883273in}}%
\pgfpathlineto{\pgfqpoint{4.407022in}{2.888727in}}%
\pgfpathlineto{\pgfqpoint{4.409027in}{2.823273in}}%
\pgfpathlineto{\pgfqpoint{4.411031in}{2.931000in}}%
\pgfpathlineto{\pgfqpoint{4.412033in}{2.955545in}}%
\pgfpathlineto{\pgfqpoint{4.414038in}{2.910545in}}%
\pgfpathlineto{\pgfqpoint{4.417045in}{3.156000in}}%
\pgfpathlineto{\pgfqpoint{4.418047in}{3.183273in}}%
\pgfpathlineto{\pgfqpoint{4.419049in}{3.090545in}}%
\pgfpathlineto{\pgfqpoint{4.420051in}{3.205091in}}%
\pgfpathlineto{\pgfqpoint{4.421054in}{3.045545in}}%
\pgfpathlineto{\pgfqpoint{4.423058in}{3.634636in}}%
\pgfpathlineto{\pgfqpoint{4.424060in}{3.646909in}}%
\pgfpathlineto{\pgfqpoint{4.425063in}{3.641455in}}%
\pgfpathlineto{\pgfqpoint{4.426065in}{3.158727in}}%
\pgfpathlineto{\pgfqpoint{4.428069in}{3.670091in}}%
\pgfpathlineto{\pgfqpoint{4.429072in}{3.690545in}}%
\pgfpathlineto{\pgfqpoint{4.430074in}{3.668727in}}%
\pgfpathlineto{\pgfqpoint{4.431076in}{3.372818in}}%
\pgfpathlineto{\pgfqpoint{4.432078in}{3.409636in}}%
\pgfpathlineto{\pgfqpoint{4.434083in}{3.648273in}}%
\pgfpathlineto{\pgfqpoint{4.435085in}{3.693273in}}%
\pgfpathlineto{\pgfqpoint{4.437089in}{3.141000in}}%
\pgfpathlineto{\pgfqpoint{4.438092in}{3.136909in}}%
\pgfpathlineto{\pgfqpoint{4.440096in}{3.192818in}}%
\pgfpathlineto{\pgfqpoint{4.441098in}{3.015545in}}%
\pgfpathlineto{\pgfqpoint{4.442101in}{3.036000in}}%
\pgfpathlineto{\pgfqpoint{4.444105in}{2.977364in}}%
\pgfpathlineto{\pgfqpoint{4.445107in}{3.008727in}}%
\pgfpathlineto{\pgfqpoint{4.446110in}{2.948727in}}%
\pgfpathlineto{\pgfqpoint{4.447112in}{2.985545in}}%
\pgfpathlineto{\pgfqpoint{4.449116in}{2.872364in}}%
\pgfpathlineto{\pgfqpoint{4.451121in}{2.909182in}}%
\pgfpathlineto{\pgfqpoint{4.452123in}{2.911909in}}%
\pgfpathlineto{\pgfqpoint{4.454128in}{2.802818in}}%
\pgfpathlineto{\pgfqpoint{4.455130in}{2.886000in}}%
\pgfpathlineto{\pgfqpoint{4.456132in}{2.868273in}}%
\pgfpathlineto{\pgfqpoint{4.457134in}{2.884636in}}%
\pgfpathlineto{\pgfqpoint{4.459139in}{2.802818in}}%
\pgfpathlineto{\pgfqpoint{4.462146in}{2.931000in}}%
\pgfpathlineto{\pgfqpoint{4.464150in}{2.858727in}}%
\pgfpathlineto{\pgfqpoint{4.466154in}{2.921455in}}%
\pgfpathlineto{\pgfqpoint{4.467157in}{3.044182in}}%
\pgfpathlineto{\pgfqpoint{4.469161in}{3.003273in}}%
\pgfpathlineto{\pgfqpoint{4.470163in}{3.011455in}}%
\pgfpathlineto{\pgfqpoint{4.471166in}{3.000545in}}%
\pgfpathlineto{\pgfqpoint{4.474172in}{3.438273in}}%
\pgfpathlineto{\pgfqpoint{4.476177in}{3.071455in}}%
\pgfpathlineto{\pgfqpoint{4.480186in}{3.691909in}}%
\pgfpathlineto{\pgfqpoint{4.481188in}{3.267818in}}%
\pgfpathlineto{\pgfqpoint{4.483193in}{3.611455in}}%
\pgfpathlineto{\pgfqpoint{4.485197in}{3.660545in}}%
\pgfpathlineto{\pgfqpoint{4.486199in}{3.297818in}}%
\pgfpathlineto{\pgfqpoint{4.487202in}{3.442364in}}%
\pgfpathlineto{\pgfqpoint{4.488204in}{3.150545in}}%
\pgfpathlineto{\pgfqpoint{4.489206in}{3.338727in}}%
\pgfpathlineto{\pgfqpoint{4.490208in}{3.277364in}}%
\pgfpathlineto{\pgfqpoint{4.492213in}{3.097364in}}%
\pgfpathlineto{\pgfqpoint{4.493215in}{3.016909in}}%
\pgfpathlineto{\pgfqpoint{4.494217in}{3.042818in}}%
\pgfpathlineto{\pgfqpoint{4.497224in}{2.989636in}}%
\pgfpathlineto{\pgfqpoint{4.498226in}{2.916000in}}%
\pgfpathlineto{\pgfqpoint{4.499229in}{2.932364in}}%
\pgfpathlineto{\pgfqpoint{4.500231in}{2.925545in}}%
\pgfpathlineto{\pgfqpoint{4.501233in}{2.928273in}}%
\pgfpathlineto{\pgfqpoint{4.502235in}{2.913273in}}%
\pgfpathlineto{\pgfqpoint{4.503237in}{2.843727in}}%
\pgfpathlineto{\pgfqpoint{4.504240in}{2.849182in}}%
\pgfpathlineto{\pgfqpoint{4.505242in}{2.892818in}}%
\pgfpathlineto{\pgfqpoint{4.506244in}{2.872364in}}%
\pgfpathlineto{\pgfqpoint{4.507246in}{2.920091in}}%
\pgfpathlineto{\pgfqpoint{4.508249in}{2.812364in}}%
\pgfpathlineto{\pgfqpoint{4.512258in}{2.901000in}}%
\pgfpathlineto{\pgfqpoint{4.513260in}{2.839636in}}%
\pgfpathlineto{\pgfqpoint{4.515264in}{2.879182in}}%
\pgfpathlineto{\pgfqpoint{4.517269in}{2.976000in}}%
\pgfpathlineto{\pgfqpoint{4.518271in}{2.978727in}}%
\pgfpathlineto{\pgfqpoint{4.519273in}{2.985545in}}%
\pgfpathlineto{\pgfqpoint{4.520276in}{2.970545in}}%
\pgfpathlineto{\pgfqpoint{4.522280in}{3.070091in}}%
\pgfpathlineto{\pgfqpoint{4.523282in}{3.053727in}}%
\pgfpathlineto{\pgfqpoint{4.524285in}{3.209182in}}%
\pgfpathlineto{\pgfqpoint{4.525287in}{3.121909in}}%
\pgfpathlineto{\pgfqpoint{4.526289in}{3.132818in}}%
\pgfpathlineto{\pgfqpoint{4.529296in}{3.615545in}}%
\pgfpathlineto{\pgfqpoint{4.531300in}{3.231000in}}%
\pgfpathlineto{\pgfqpoint{4.533305in}{3.551455in}}%
\pgfpathlineto{\pgfqpoint{4.534307in}{3.637364in}}%
\pgfpathlineto{\pgfqpoint{4.536312in}{3.255545in}}%
\pgfpathlineto{\pgfqpoint{4.537314in}{3.295091in}}%
\pgfpathlineto{\pgfqpoint{4.538316in}{3.254182in}}%
\pgfpathlineto{\pgfqpoint{4.539318in}{3.360545in}}%
\pgfpathlineto{\pgfqpoint{4.541323in}{3.094636in}}%
\pgfpathlineto{\pgfqpoint{4.543327in}{3.026455in}}%
\pgfpathlineto{\pgfqpoint{4.544329in}{3.071455in}}%
\pgfpathlineto{\pgfqpoint{4.548338in}{2.909182in}}%
\pgfpathlineto{\pgfqpoint{4.549341in}{2.940545in}}%
\pgfpathlineto{\pgfqpoint{4.550343in}{2.926909in}}%
\pgfpathlineto{\pgfqpoint{4.551345in}{2.956909in}}%
\pgfpathlineto{\pgfqpoint{4.553350in}{2.823273in}}%
\pgfpathlineto{\pgfqpoint{4.554352in}{2.862818in}}%
\pgfpathlineto{\pgfqpoint{4.555354in}{2.853273in}}%
\pgfpathlineto{\pgfqpoint{4.556356in}{2.932364in}}%
\pgfpathlineto{\pgfqpoint{4.558361in}{2.808273in}}%
\pgfpathlineto{\pgfqpoint{4.561368in}{2.914636in}}%
\pgfpathlineto{\pgfqpoint{4.563372in}{2.842364in}}%
\pgfpathlineto{\pgfqpoint{4.567381in}{2.977364in}}%
\pgfpathlineto{\pgfqpoint{4.568383in}{2.963727in}}%
\pgfpathlineto{\pgfqpoint{4.572392in}{3.078273in}}%
\pgfpathlineto{\pgfqpoint{4.574397in}{3.372818in}}%
\pgfpathlineto{\pgfqpoint{4.575399in}{3.075545in}}%
\pgfpathlineto{\pgfqpoint{4.576401in}{3.214636in}}%
\pgfpathlineto{\pgfqpoint{4.577403in}{3.151909in}}%
\pgfpathlineto{\pgfqpoint{4.579408in}{3.599182in}}%
\pgfpathlineto{\pgfqpoint{4.580410in}{3.232364in}}%
\pgfpathlineto{\pgfqpoint{4.581412in}{3.346909in}}%
\pgfpathlineto{\pgfqpoint{4.582415in}{3.232364in}}%
\pgfpathlineto{\pgfqpoint{4.584419in}{3.586909in}}%
\pgfpathlineto{\pgfqpoint{4.587426in}{3.100091in}}%
\pgfpathlineto{\pgfqpoint{4.589430in}{3.292364in}}%
\pgfpathlineto{\pgfqpoint{4.591435in}{3.078273in}}%
\pgfpathlineto{\pgfqpoint{4.592437in}{2.992364in}}%
\pgfpathlineto{\pgfqpoint{4.594442in}{3.006000in}}%
\pgfpathlineto{\pgfqpoint{4.595444in}{2.974636in}}%
\pgfpathlineto{\pgfqpoint{4.596446in}{2.991000in}}%
\pgfpathlineto{\pgfqpoint{4.597448in}{2.921455in}}%
\pgfpathlineto{\pgfqpoint{4.598451in}{2.926909in}}%
\pgfpathlineto{\pgfqpoint{4.600455in}{2.888727in}}%
\pgfpathlineto{\pgfqpoint{4.601457in}{2.941909in}}%
\pgfpathlineto{\pgfqpoint{4.602460in}{2.864182in}}%
\pgfpathlineto{\pgfqpoint{4.604464in}{2.880545in}}%
\pgfpathlineto{\pgfqpoint{4.605466in}{2.846455in}}%
\pgfpathlineto{\pgfqpoint{4.606469in}{2.901000in}}%
\pgfpathlineto{\pgfqpoint{4.607471in}{2.853273in}}%
\pgfpathlineto{\pgfqpoint{4.608473in}{2.864182in}}%
\pgfpathlineto{\pgfqpoint{4.609475in}{2.862818in}}%
\pgfpathlineto{\pgfqpoint{4.610477in}{2.872364in}}%
\pgfpathlineto{\pgfqpoint{4.611480in}{2.910545in}}%
\pgfpathlineto{\pgfqpoint{4.612482in}{2.888727in}}%
\pgfpathlineto{\pgfqpoint{4.613484in}{2.910545in}}%
\pgfpathlineto{\pgfqpoint{4.614486in}{2.891455in}}%
\pgfpathlineto{\pgfqpoint{4.615489in}{2.896909in}}%
\pgfpathlineto{\pgfqpoint{4.618495in}{3.000545in}}%
\pgfpathlineto{\pgfqpoint{4.619498in}{3.000545in}}%
\pgfpathlineto{\pgfqpoint{4.621502in}{3.044182in}}%
\pgfpathlineto{\pgfqpoint{4.622504in}{3.030545in}}%
\pgfpathlineto{\pgfqpoint{4.623507in}{3.175091in}}%
\pgfpathlineto{\pgfqpoint{4.626513in}{3.105545in}}%
\pgfpathlineto{\pgfqpoint{4.627516in}{3.142364in}}%
\pgfpathlineto{\pgfqpoint{4.628518in}{3.555545in}}%
\pgfpathlineto{\pgfqpoint{4.629520in}{3.501000in}}%
\pgfpathlineto{\pgfqpoint{4.631525in}{3.265091in}}%
\pgfpathlineto{\pgfqpoint{4.632527in}{3.263727in}}%
\pgfpathlineto{\pgfqpoint{4.633529in}{3.570545in}}%
\pgfpathlineto{\pgfqpoint{4.634531in}{3.479182in}}%
\pgfpathlineto{\pgfqpoint{4.636536in}{3.169636in}}%
\pgfpathlineto{\pgfqpoint{4.637538in}{3.126000in}}%
\pgfpathlineto{\pgfqpoint{4.638540in}{3.217364in}}%
\pgfpathlineto{\pgfqpoint{4.640545in}{3.066000in}}%
\pgfpathlineto{\pgfqpoint{4.642549in}{2.991000in}}%
\pgfpathlineto{\pgfqpoint{4.643551in}{3.060545in}}%
\pgfpathlineto{\pgfqpoint{4.645556in}{3.001909in}}%
\pgfpathlineto{\pgfqpoint{4.647560in}{2.913273in}}%
\pgfpathlineto{\pgfqpoint{4.648563in}{2.922818in}}%
\pgfpathlineto{\pgfqpoint{4.649565in}{2.903727in}}%
\pgfpathlineto{\pgfqpoint{4.650567in}{2.939182in}}%
\pgfpathlineto{\pgfqpoint{4.652572in}{2.875091in}}%
\pgfpathlineto{\pgfqpoint{4.653574in}{2.875091in}}%
\pgfpathlineto{\pgfqpoint{4.654576in}{2.843727in}}%
\pgfpathlineto{\pgfqpoint{4.656581in}{2.902364in}}%
\pgfpathlineto{\pgfqpoint{4.659587in}{2.854636in}}%
\pgfpathlineto{\pgfqpoint{4.661592in}{2.925545in}}%
\pgfpathlineto{\pgfqpoint{4.662594in}{2.899636in}}%
\pgfpathlineto{\pgfqpoint{4.663596in}{2.921455in}}%
\pgfpathlineto{\pgfqpoint{4.664599in}{2.901000in}}%
\pgfpathlineto{\pgfqpoint{4.666603in}{2.952818in}}%
\pgfpathlineto{\pgfqpoint{4.667605in}{2.954182in}}%
\pgfpathlineto{\pgfqpoint{4.668608in}{2.988273in}}%
\pgfpathlineto{\pgfqpoint{4.669610in}{2.956909in}}%
\pgfpathlineto{\pgfqpoint{4.673619in}{3.158727in}}%
\pgfpathlineto{\pgfqpoint{4.674621in}{3.078273in}}%
\pgfpathlineto{\pgfqpoint{4.675623in}{3.097364in}}%
\pgfpathlineto{\pgfqpoint{4.676626in}{3.090545in}}%
\pgfpathlineto{\pgfqpoint{4.677628in}{3.187364in}}%
\pgfpathlineto{\pgfqpoint{4.678630in}{3.503727in}}%
\pgfpathlineto{\pgfqpoint{4.680634in}{3.191455in}}%
\pgfpathlineto{\pgfqpoint{4.681637in}{3.184636in}}%
\pgfpathlineto{\pgfqpoint{4.682639in}{3.236455in}}%
\pgfpathlineto{\pgfqpoint{4.683641in}{3.533727in}}%
\pgfpathlineto{\pgfqpoint{4.686648in}{3.105545in}}%
\pgfpathlineto{\pgfqpoint{4.687650in}{3.135545in}}%
\pgfpathlineto{\pgfqpoint{4.688652in}{3.211909in}}%
\pgfpathlineto{\pgfqpoint{4.691659in}{3.038727in}}%
\pgfpathlineto{\pgfqpoint{4.692661in}{3.021000in}}%
\pgfpathlineto{\pgfqpoint{4.693664in}{3.059182in}}%
\pgfpathlineto{\pgfqpoint{4.694666in}{3.016909in}}%
\pgfpathlineto{\pgfqpoint{4.695668in}{3.026455in}}%
\pgfpathlineto{\pgfqpoint{4.697673in}{2.950091in}}%
\pgfpathlineto{\pgfqpoint{4.699677in}{2.936455in}}%
\pgfpathlineto{\pgfqpoint{4.700679in}{2.958273in}}%
\pgfpathlineto{\pgfqpoint{4.702684in}{2.894182in}}%
\pgfpathlineto{\pgfqpoint{4.703686in}{2.892818in}}%
\pgfpathlineto{\pgfqpoint{4.704688in}{2.851909in}}%
\pgfpathlineto{\pgfqpoint{4.705691in}{2.920091in}}%
\pgfpathlineto{\pgfqpoint{4.706693in}{2.917364in}}%
\pgfpathlineto{\pgfqpoint{4.709700in}{2.865545in}}%
\pgfpathlineto{\pgfqpoint{4.711704in}{2.933727in}}%
\pgfpathlineto{\pgfqpoint{4.713709in}{2.894182in}}%
\pgfpathlineto{\pgfqpoint{4.714711in}{2.894182in}}%
\pgfpathlineto{\pgfqpoint{4.717717in}{2.982818in}}%
\pgfpathlineto{\pgfqpoint{4.718720in}{2.947364in}}%
\pgfpathlineto{\pgfqpoint{4.719722in}{2.954182in}}%
\pgfpathlineto{\pgfqpoint{4.723731in}{3.078273in}}%
\pgfpathlineto{\pgfqpoint{4.724733in}{3.053727in}}%
\pgfpathlineto{\pgfqpoint{4.725735in}{3.083727in}}%
\pgfpathlineto{\pgfqpoint{4.726738in}{3.051000in}}%
\pgfpathlineto{\pgfqpoint{4.727740in}{3.138273in}}%
\pgfpathlineto{\pgfqpoint{4.728742in}{3.329182in}}%
\pgfpathlineto{\pgfqpoint{4.729744in}{3.218727in}}%
\pgfpathlineto{\pgfqpoint{4.730747in}{3.285545in}}%
\pgfpathlineto{\pgfqpoint{4.731749in}{3.096000in}}%
\pgfpathlineto{\pgfqpoint{4.732751in}{3.179182in}}%
\pgfpathlineto{\pgfqpoint{4.733753in}{3.368727in}}%
\pgfpathlineto{\pgfqpoint{4.736760in}{3.085091in}}%
\pgfpathlineto{\pgfqpoint{4.737762in}{3.116455in}}%
\pgfpathlineto{\pgfqpoint{4.738765in}{3.102818in}}%
\pgfpathlineto{\pgfqpoint{4.740769in}{3.119182in}}%
\pgfpathlineto{\pgfqpoint{4.742774in}{3.027818in}}%
\pgfpathlineto{\pgfqpoint{4.743776in}{2.992364in}}%
\pgfpathlineto{\pgfqpoint{4.745780in}{3.038727in}}%
\pgfpathlineto{\pgfqpoint{4.748787in}{2.925545in}}%
\pgfpathlineto{\pgfqpoint{4.749789in}{2.921455in}}%
\pgfpathlineto{\pgfqpoint{4.750791in}{2.950091in}}%
\pgfpathlineto{\pgfqpoint{4.751794in}{2.940545in}}%
\pgfpathlineto{\pgfqpoint{4.752796in}{2.941909in}}%
\pgfpathlineto{\pgfqpoint{4.754800in}{2.890091in}}%
\pgfpathlineto{\pgfqpoint{4.755803in}{2.931000in}}%
\pgfpathlineto{\pgfqpoint{4.756805in}{2.924182in}}%
\pgfpathlineto{\pgfqpoint{4.757807in}{2.928273in}}%
\pgfpathlineto{\pgfqpoint{4.758809in}{2.884636in}}%
\pgfpathlineto{\pgfqpoint{4.759812in}{2.887364in}}%
\pgfpathlineto{\pgfqpoint{4.760814in}{2.931000in}}%
\pgfpathlineto{\pgfqpoint{4.761816in}{2.918727in}}%
\pgfpathlineto{\pgfqpoint{4.762818in}{2.936455in}}%
\pgfpathlineto{\pgfqpoint{4.763821in}{2.909182in}}%
\pgfpathlineto{\pgfqpoint{4.767830in}{2.973273in}}%
\pgfpathlineto{\pgfqpoint{4.768832in}{2.959636in}}%
\pgfpathlineto{\pgfqpoint{4.772841in}{3.070091in}}%
\pgfpathlineto{\pgfqpoint{4.773843in}{3.026455in}}%
\pgfpathlineto{\pgfqpoint{4.775848in}{3.102818in}}%
\pgfpathlineto{\pgfqpoint{4.776850in}{3.034636in}}%
\pgfpathlineto{\pgfqpoint{4.778854in}{3.168273in}}%
\pgfpathlineto{\pgfqpoint{4.781861in}{3.091909in}}%
\pgfpathlineto{\pgfqpoint{4.783866in}{3.251455in}}%
\pgfpathlineto{\pgfqpoint{4.784868in}{3.207818in}}%
\pgfpathlineto{\pgfqpoint{4.786872in}{3.086455in}}%
\pgfpathlineto{\pgfqpoint{4.787874in}{3.147818in}}%
\pgfpathlineto{\pgfqpoint{4.788877in}{3.091909in}}%
\pgfpathlineto{\pgfqpoint{4.789879in}{3.126000in}}%
\pgfpathlineto{\pgfqpoint{4.791883in}{3.044182in}}%
\pgfpathlineto{\pgfqpoint{4.796895in}{2.981455in}}%
\pgfpathlineto{\pgfqpoint{4.797897in}{2.992364in}}%
\pgfpathlineto{\pgfqpoint{4.798899in}{2.946000in}}%
\pgfpathlineto{\pgfqpoint{4.799901in}{2.988273in}}%
\pgfpathlineto{\pgfqpoint{4.802908in}{2.937818in}}%
\pgfpathlineto{\pgfqpoint{4.803910in}{2.905091in}}%
\pgfpathlineto{\pgfqpoint{4.805915in}{2.932364in}}%
\pgfpathlineto{\pgfqpoint{4.808922in}{2.880545in}}%
\pgfpathlineto{\pgfqpoint{4.809924in}{2.891455in}}%
\pgfpathlineto{\pgfqpoint{4.811928in}{2.950091in}}%
\pgfpathlineto{\pgfqpoint{4.812931in}{2.946000in}}%
\pgfpathlineto{\pgfqpoint{4.813933in}{2.909182in}}%
\pgfpathlineto{\pgfqpoint{4.815937in}{2.946000in}}%
\pgfpathlineto{\pgfqpoint{4.816940in}{2.984182in}}%
\pgfpathlineto{\pgfqpoint{4.818944in}{2.958273in}}%
\pgfpathlineto{\pgfqpoint{4.821951in}{3.014182in}}%
\pgfpathlineto{\pgfqpoint{4.822953in}{3.052364in}}%
\pgfpathlineto{\pgfqpoint{4.823955in}{3.034636in}}%
\pgfpathlineto{\pgfqpoint{4.824957in}{3.051000in}}%
\pgfpathlineto{\pgfqpoint{4.825960in}{3.004636in}}%
\pgfpathlineto{\pgfqpoint{4.826962in}{3.038727in}}%
\pgfpathlineto{\pgfqpoint{4.827964in}{3.134182in}}%
\pgfpathlineto{\pgfqpoint{4.828966in}{3.102818in}}%
\pgfpathlineto{\pgfqpoint{4.829969in}{3.136909in}}%
\pgfpathlineto{\pgfqpoint{4.830971in}{3.037364in}}%
\pgfpathlineto{\pgfqpoint{4.831973in}{3.056455in}}%
\pgfpathlineto{\pgfqpoint{4.834980in}{3.225545in}}%
\pgfpathlineto{\pgfqpoint{4.836984in}{3.078273in}}%
\pgfpathlineto{\pgfqpoint{4.838989in}{3.116455in}}%
\pgfpathlineto{\pgfqpoint{4.839991in}{3.131455in}}%
\pgfpathlineto{\pgfqpoint{4.841996in}{3.037364in}}%
\pgfpathlineto{\pgfqpoint{4.842998in}{3.036000in}}%
\pgfpathlineto{\pgfqpoint{4.844000in}{3.030545in}}%
\pgfpathlineto{\pgfqpoint{4.845002in}{3.041455in}}%
\pgfpathlineto{\pgfqpoint{4.847007in}{3.004636in}}%
\pgfpathlineto{\pgfqpoint{4.848009in}{2.992364in}}%
\pgfpathlineto{\pgfqpoint{4.849011in}{2.962364in}}%
\pgfpathlineto{\pgfqpoint{4.850014in}{2.988273in}}%
\pgfpathlineto{\pgfqpoint{4.851016in}{2.970545in}}%
\pgfpathlineto{\pgfqpoint{4.852018in}{2.992364in}}%
\pgfpathlineto{\pgfqpoint{4.854023in}{2.913273in}}%
\pgfpathlineto{\pgfqpoint{4.857029in}{2.971909in}}%
\pgfpathlineto{\pgfqpoint{4.859034in}{2.901000in}}%
\pgfpathlineto{\pgfqpoint{4.862040in}{2.941909in}}%
\pgfpathlineto{\pgfqpoint{4.863043in}{2.935091in}}%
\pgfpathlineto{\pgfqpoint{4.864045in}{2.907818in}}%
\pgfpathlineto{\pgfqpoint{4.867052in}{2.967818in}}%
\pgfpathlineto{\pgfqpoint{4.869056in}{2.962364in}}%
\pgfpathlineto{\pgfqpoint{4.871061in}{2.982818in}}%
\pgfpathlineto{\pgfqpoint{4.872063in}{3.029182in}}%
\pgfpathlineto{\pgfqpoint{4.873065in}{3.016909in}}%
\pgfpathlineto{\pgfqpoint{4.875070in}{3.045545in}}%
\pgfpathlineto{\pgfqpoint{4.876072in}{3.003273in}}%
\pgfpathlineto{\pgfqpoint{4.878076in}{3.089182in}}%
\pgfpathlineto{\pgfqpoint{4.879079in}{3.089182in}}%
\pgfpathlineto{\pgfqpoint{4.880081in}{3.098727in}}%
\pgfpathlineto{\pgfqpoint{4.881083in}{3.041455in}}%
\pgfpathlineto{\pgfqpoint{4.882085in}{3.059182in}}%
\pgfpathlineto{\pgfqpoint{4.884090in}{3.177818in}}%
\pgfpathlineto{\pgfqpoint{4.886094in}{3.071455in}}%
\pgfpathlineto{\pgfqpoint{4.887097in}{3.074182in}}%
\pgfpathlineto{\pgfqpoint{4.890103in}{3.128727in}}%
\pgfpathlineto{\pgfqpoint{4.892108in}{3.027818in}}%
\pgfpathlineto{\pgfqpoint{4.895114in}{3.056455in}}%
\pgfpathlineto{\pgfqpoint{4.898121in}{2.978727in}}%
\pgfpathlineto{\pgfqpoint{4.899123in}{2.997818in}}%
\pgfpathlineto{\pgfqpoint{4.900126in}{2.996455in}}%
\pgfpathlineto{\pgfqpoint{4.901128in}{2.982818in}}%
\pgfpathlineto{\pgfqpoint{4.902130in}{2.988273in}}%
\pgfpathlineto{\pgfqpoint{4.903132in}{2.947364in}}%
\pgfpathlineto{\pgfqpoint{4.905137in}{2.966455in}}%
\pgfpathlineto{\pgfqpoint{4.906139in}{2.952818in}}%
\pgfpathlineto{\pgfqpoint{4.907141in}{2.969182in}}%
\pgfpathlineto{\pgfqpoint{4.909146in}{2.913273in}}%
\pgfpathlineto{\pgfqpoint{4.912153in}{2.970545in}}%
\pgfpathlineto{\pgfqpoint{4.914157in}{2.932364in}}%
\pgfpathlineto{\pgfqpoint{4.917164in}{2.989636in}}%
\pgfpathlineto{\pgfqpoint{4.918166in}{2.965091in}}%
\pgfpathlineto{\pgfqpoint{4.919168in}{2.988273in}}%
\pgfpathlineto{\pgfqpoint{4.920171in}{2.977364in}}%
\pgfpathlineto{\pgfqpoint{4.921173in}{2.988273in}}%
\pgfpathlineto{\pgfqpoint{4.924180in}{3.060545in}}%
\pgfpathlineto{\pgfqpoint{4.926184in}{3.014182in}}%
\pgfpathlineto{\pgfqpoint{4.929191in}{3.097364in}}%
\pgfpathlineto{\pgfqpoint{4.930193in}{3.082364in}}%
\pgfpathlineto{\pgfqpoint{4.931195in}{3.042818in}}%
\pgfpathlineto{\pgfqpoint{4.934202in}{3.124636in}}%
\pgfpathlineto{\pgfqpoint{4.937209in}{3.056455in}}%
\pgfpathlineto{\pgfqpoint{4.938211in}{3.068727in}}%
\pgfpathlineto{\pgfqpoint{4.939213in}{3.102818in}}%
\pgfpathlineto{\pgfqpoint{4.941218in}{3.033273in}}%
\pgfpathlineto{\pgfqpoint{4.942220in}{3.038727in}}%
\pgfpathlineto{\pgfqpoint{4.943222in}{3.029182in}}%
\pgfpathlineto{\pgfqpoint{4.944224in}{3.057818in}}%
\pgfpathlineto{\pgfqpoint{4.946229in}{3.014182in}}%
\pgfpathlineto{\pgfqpoint{4.948233in}{2.984182in}}%
\pgfpathlineto{\pgfqpoint{4.949236in}{2.995091in}}%
\pgfpathlineto{\pgfqpoint{4.950238in}{2.993727in}}%
\pgfpathlineto{\pgfqpoint{4.952242in}{2.971909in}}%
\pgfpathlineto{\pgfqpoint{4.953245in}{2.939182in}}%
\pgfpathlineto{\pgfqpoint{4.954247in}{2.954182in}}%
\pgfpathlineto{\pgfqpoint{4.955249in}{2.947364in}}%
\pgfpathlineto{\pgfqpoint{4.956251in}{2.969182in}}%
\pgfpathlineto{\pgfqpoint{4.957254in}{2.962364in}}%
\pgfpathlineto{\pgfqpoint{4.958256in}{2.928273in}}%
\pgfpathlineto{\pgfqpoint{4.960260in}{2.937818in}}%
\pgfpathlineto{\pgfqpoint{4.962265in}{2.974636in}}%
\pgfpathlineto{\pgfqpoint{4.964269in}{2.944636in}}%
\pgfpathlineto{\pgfqpoint{4.965271in}{2.962364in}}%
\pgfpathlineto{\pgfqpoint{4.966274in}{3.001909in}}%
\pgfpathlineto{\pgfqpoint{4.968278in}{2.977364in}}%
\pgfpathlineto{\pgfqpoint{4.969280in}{2.989636in}}%
\pgfpathlineto{\pgfqpoint{4.970283in}{2.980091in}}%
\pgfpathlineto{\pgfqpoint{4.973289in}{3.045545in}}%
\pgfpathlineto{\pgfqpoint{4.974292in}{3.041455in}}%
\pgfpathlineto{\pgfqpoint{4.975294in}{3.004636in}}%
\pgfpathlineto{\pgfqpoint{4.976296in}{3.010091in}}%
\pgfpathlineto{\pgfqpoint{4.977298in}{3.026455in}}%
\pgfpathlineto{\pgfqpoint{4.978301in}{3.082364in}}%
\pgfpathlineto{\pgfqpoint{4.979303in}{3.079636in}}%
\pgfpathlineto{\pgfqpoint{4.981307in}{3.012818in}}%
\pgfpathlineto{\pgfqpoint{4.984314in}{3.109636in}}%
\pgfpathlineto{\pgfqpoint{4.986319in}{3.036000in}}%
\pgfpathlineto{\pgfqpoint{4.987321in}{3.036000in}}%
\pgfpathlineto{\pgfqpoint{4.988323in}{3.057818in}}%
\pgfpathlineto{\pgfqpoint{4.989325in}{3.116455in}}%
\pgfpathlineto{\pgfqpoint{4.992332in}{3.006000in}}%
\pgfpathlineto{\pgfqpoint{4.994337in}{3.056455in}}%
\pgfpathlineto{\pgfqpoint{4.997343in}{2.970545in}}%
\pgfpathlineto{\pgfqpoint{4.999348in}{2.980091in}}%
\pgfpathlineto{\pgfqpoint{5.001352in}{3.001909in}}%
\pgfpathlineto{\pgfqpoint{5.003357in}{2.943273in}}%
\pgfpathlineto{\pgfqpoint{5.005361in}{2.970545in}}%
\pgfpathlineto{\pgfqpoint{5.006363in}{3.010091in}}%
\pgfpathlineto{\pgfqpoint{5.008368in}{2.932364in}}%
\pgfpathlineto{\pgfqpoint{5.009370in}{2.939182in}}%
\pgfpathlineto{\pgfqpoint{5.010372in}{2.954182in}}%
\pgfpathlineto{\pgfqpoint{5.011375in}{2.986909in}}%
\pgfpathlineto{\pgfqpoint{5.012377in}{2.977364in}}%
\pgfpathlineto{\pgfqpoint{5.013379in}{2.952818in}}%
\pgfpathlineto{\pgfqpoint{5.014381in}{2.958273in}}%
\pgfpathlineto{\pgfqpoint{5.015384in}{2.954182in}}%
\pgfpathlineto{\pgfqpoint{5.017388in}{2.982818in}}%
\pgfpathlineto{\pgfqpoint{5.018390in}{2.971909in}}%
\pgfpathlineto{\pgfqpoint{5.020395in}{2.986909in}}%
\pgfpathlineto{\pgfqpoint{5.021397in}{2.997818in}}%
\pgfpathlineto{\pgfqpoint{5.022399in}{2.993727in}}%
\pgfpathlineto{\pgfqpoint{5.025406in}{3.022364in}}%
\pgfpathlineto{\pgfqpoint{5.027411in}{3.006000in}}%
\pgfpathlineto{\pgfqpoint{5.030417in}{3.064636in}}%
\pgfpathlineto{\pgfqpoint{5.031420in}{3.040091in}}%
\pgfpathlineto{\pgfqpoint{5.032422in}{2.988273in}}%
\pgfpathlineto{\pgfqpoint{5.034426in}{3.059182in}}%
\pgfpathlineto{\pgfqpoint{5.035428in}{3.066000in}}%
\pgfpathlineto{\pgfqpoint{5.037433in}{3.023727in}}%
\pgfpathlineto{\pgfqpoint{5.039437in}{3.079636in}}%
\pgfpathlineto{\pgfqpoint{5.041442in}{3.046909in}}%
\pgfpathlineto{\pgfqpoint{5.042444in}{3.006000in}}%
\pgfpathlineto{\pgfqpoint{5.043446in}{3.015545in}}%
\pgfpathlineto{\pgfqpoint{5.044449in}{3.042818in}}%
\pgfpathlineto{\pgfqpoint{5.045451in}{3.040091in}}%
\pgfpathlineto{\pgfqpoint{5.046453in}{3.026455in}}%
\pgfpathlineto{\pgfqpoint{5.047455in}{2.982818in}}%
\pgfpathlineto{\pgfqpoint{5.048458in}{2.993727in}}%
\pgfpathlineto{\pgfqpoint{5.049460in}{2.985545in}}%
\pgfpathlineto{\pgfqpoint{5.050462in}{3.022364in}}%
\pgfpathlineto{\pgfqpoint{5.051464in}{3.019636in}}%
\pgfpathlineto{\pgfqpoint{5.053469in}{2.955545in}}%
\pgfpathlineto{\pgfqpoint{5.054471in}{2.961000in}}%
\pgfpathlineto{\pgfqpoint{5.056476in}{3.006000in}}%
\pgfpathlineto{\pgfqpoint{5.058480in}{2.950091in}}%
\pgfpathlineto{\pgfqpoint{5.059482in}{2.955545in}}%
\pgfpathlineto{\pgfqpoint{5.061487in}{3.006000in}}%
\pgfpathlineto{\pgfqpoint{5.063491in}{2.944636in}}%
\pgfpathlineto{\pgfqpoint{5.064494in}{2.952818in}}%
\pgfpathlineto{\pgfqpoint{5.065496in}{2.963727in}}%
\pgfpathlineto{\pgfqpoint{5.066498in}{2.996455in}}%
\pgfpathlineto{\pgfqpoint{5.068502in}{2.973273in}}%
\pgfpathlineto{\pgfqpoint{5.069505in}{2.977364in}}%
\pgfpathlineto{\pgfqpoint{5.070507in}{3.000545in}}%
\pgfpathlineto{\pgfqpoint{5.071509in}{2.980091in}}%
\pgfpathlineto{\pgfqpoint{5.073514in}{3.012818in}}%
\pgfpathlineto{\pgfqpoint{5.074516in}{3.001909in}}%
\pgfpathlineto{\pgfqpoint{5.075518in}{3.036000in}}%
\pgfpathlineto{\pgfqpoint{5.077523in}{3.014182in}}%
\pgfpathlineto{\pgfqpoint{5.078525in}{3.037364in}}%
\pgfpathlineto{\pgfqpoint{5.079527in}{3.033273in}}%
\pgfpathlineto{\pgfqpoint{5.080529in}{3.055091in}}%
\pgfpathlineto{\pgfqpoint{5.082534in}{3.019636in}}%
\pgfpathlineto{\pgfqpoint{5.084538in}{3.067364in}}%
\pgfpathlineto{\pgfqpoint{5.085541in}{3.074182in}}%
\pgfpathlineto{\pgfqpoint{5.087545in}{3.015545in}}%
\pgfpathlineto{\pgfqpoint{5.088547in}{3.053727in}}%
\pgfpathlineto{\pgfqpoint{5.089550in}{3.046909in}}%
\pgfpathlineto{\pgfqpoint{5.090552in}{3.057818in}}%
\pgfpathlineto{\pgfqpoint{5.092556in}{3.011455in}}%
\pgfpathlineto{\pgfqpoint{5.095563in}{3.040091in}}%
\pgfpathlineto{\pgfqpoint{5.098570in}{2.976000in}}%
\pgfpathlineto{\pgfqpoint{5.100574in}{3.012818in}}%
\pgfpathlineto{\pgfqpoint{5.103581in}{2.959636in}}%
\pgfpathlineto{\pgfqpoint{5.104583in}{2.962364in}}%
\pgfpathlineto{\pgfqpoint{5.105585in}{2.992364in}}%
\pgfpathlineto{\pgfqpoint{5.106588in}{2.985545in}}%
\pgfpathlineto{\pgfqpoint{5.109594in}{2.941909in}}%
\pgfpathlineto{\pgfqpoint{5.111599in}{2.993727in}}%
\pgfpathlineto{\pgfqpoint{5.114606in}{2.952818in}}%
\pgfpathlineto{\pgfqpoint{5.116610in}{2.989636in}}%
\pgfpathlineto{\pgfqpoint{5.118615in}{2.978727in}}%
\pgfpathlineto{\pgfqpoint{5.119617in}{2.981455in}}%
\pgfpathlineto{\pgfqpoint{5.121621in}{2.999182in}}%
\pgfpathlineto{\pgfqpoint{5.122624in}{3.010091in}}%
\pgfpathlineto{\pgfqpoint{5.124628in}{3.003273in}}%
\pgfpathlineto{\pgfqpoint{5.125630in}{3.022364in}}%
\pgfpathlineto{\pgfqpoint{5.126633in}{3.012818in}}%
\pgfpathlineto{\pgfqpoint{5.128637in}{3.046909in}}%
\pgfpathlineto{\pgfqpoint{5.130642in}{3.038727in}}%
\pgfpathlineto{\pgfqpoint{5.131644in}{3.004636in}}%
\pgfpathlineto{\pgfqpoint{5.133648in}{3.056455in}}%
\pgfpathlineto{\pgfqpoint{5.134651in}{3.051000in}}%
\pgfpathlineto{\pgfqpoint{5.135653in}{3.053727in}}%
\pgfpathlineto{\pgfqpoint{5.136655in}{3.021000in}}%
\pgfpathlineto{\pgfqpoint{5.137657in}{3.025091in}}%
\pgfpathlineto{\pgfqpoint{5.138660in}{3.053727in}}%
\pgfpathlineto{\pgfqpoint{5.139662in}{3.051000in}}%
\pgfpathlineto{\pgfqpoint{5.140664in}{3.051000in}}%
\pgfpathlineto{\pgfqpoint{5.141666in}{3.006000in}}%
\pgfpathlineto{\pgfqpoint{5.142668in}{3.026455in}}%
\pgfpathlineto{\pgfqpoint{5.143671in}{3.010091in}}%
\pgfpathlineto{\pgfqpoint{5.144673in}{3.042818in}}%
\pgfpathlineto{\pgfqpoint{5.145675in}{3.031909in}}%
\pgfpathlineto{\pgfqpoint{5.147680in}{2.988273in}}%
\pgfpathlineto{\pgfqpoint{5.149684in}{3.007364in}}%
\pgfpathlineto{\pgfqpoint{5.150686in}{3.014182in}}%
\pgfpathlineto{\pgfqpoint{5.151689in}{2.985545in}}%
\pgfpathlineto{\pgfqpoint{5.152691in}{3.001909in}}%
\pgfpathlineto{\pgfqpoint{5.153693in}{2.973273in}}%
\pgfpathlineto{\pgfqpoint{5.154695in}{2.976000in}}%
\pgfpathlineto{\pgfqpoint{5.155698in}{2.993727in}}%
\pgfpathlineto{\pgfqpoint{5.157702in}{2.986909in}}%
\pgfpathlineto{\pgfqpoint{5.159707in}{2.958273in}}%
\pgfpathlineto{\pgfqpoint{5.161711in}{2.993727in}}%
\pgfpathlineto{\pgfqpoint{5.162713in}{2.995091in}}%
\pgfpathlineto{\pgfqpoint{5.164718in}{2.966455in}}%
\pgfpathlineto{\pgfqpoint{5.166722in}{2.988273in}}%
\pgfpathlineto{\pgfqpoint{5.167725in}{2.995091in}}%
\pgfpathlineto{\pgfqpoint{5.169729in}{2.981455in}}%
\pgfpathlineto{\pgfqpoint{5.170731in}{2.982818in}}%
\pgfpathlineto{\pgfqpoint{5.174740in}{3.029182in}}%
\pgfpathlineto{\pgfqpoint{5.175742in}{3.006000in}}%
\pgfpathlineto{\pgfqpoint{5.176745in}{3.007364in}}%
\pgfpathlineto{\pgfqpoint{5.177747in}{3.008727in}}%
\pgfpathlineto{\pgfqpoint{5.178749in}{3.014182in}}%
\pgfpathlineto{\pgfqpoint{5.179751in}{3.036000in}}%
\pgfpathlineto{\pgfqpoint{5.180754in}{3.027818in}}%
\pgfpathlineto{\pgfqpoint{5.181756in}{2.992364in}}%
\pgfpathlineto{\pgfqpoint{5.182758in}{3.026455in}}%
\pgfpathlineto{\pgfqpoint{5.183760in}{3.025091in}}%
\pgfpathlineto{\pgfqpoint{5.185765in}{3.059182in}}%
\pgfpathlineto{\pgfqpoint{5.186767in}{3.015545in}}%
\pgfpathlineto{\pgfqpoint{5.188772in}{3.057818in}}%
\pgfpathlineto{\pgfqpoint{5.189774in}{3.040091in}}%
\pgfpathlineto{\pgfqpoint{5.190776in}{3.057818in}}%
\pgfpathlineto{\pgfqpoint{5.192781in}{3.033273in}}%
\pgfpathlineto{\pgfqpoint{5.193783in}{3.029182in}}%
\pgfpathlineto{\pgfqpoint{5.194785in}{3.049636in}}%
\pgfpathlineto{\pgfqpoint{5.195787in}{3.045545in}}%
\pgfpathlineto{\pgfqpoint{5.196790in}{3.018273in}}%
\pgfpathlineto{\pgfqpoint{5.197792in}{3.027818in}}%
\pgfpathlineto{\pgfqpoint{5.198794in}{3.021000in}}%
\pgfpathlineto{\pgfqpoint{5.199796in}{3.038727in}}%
\pgfpathlineto{\pgfqpoint{5.201801in}{3.010091in}}%
\pgfpathlineto{\pgfqpoint{5.203805in}{2.997818in}}%
\pgfpathlineto{\pgfqpoint{5.204808in}{3.008727in}}%
\pgfpathlineto{\pgfqpoint{5.205810in}{3.007364in}}%
\pgfpathlineto{\pgfqpoint{5.206812in}{2.995091in}}%
\pgfpathlineto{\pgfqpoint{5.207814in}{3.011455in}}%
\pgfpathlineto{\pgfqpoint{5.208817in}{2.986909in}}%
\pgfpathlineto{\pgfqpoint{5.210821in}{2.997818in}}%
\pgfpathlineto{\pgfqpoint{5.211823in}{2.986909in}}%
\pgfpathlineto{\pgfqpoint{5.212825in}{2.997818in}}%
\pgfpathlineto{\pgfqpoint{5.213828in}{2.970545in}}%
\pgfpathlineto{\pgfqpoint{5.215832in}{2.984182in}}%
\pgfpathlineto{\pgfqpoint{5.216834in}{2.981455in}}%
\pgfpathlineto{\pgfqpoint{5.217837in}{2.995091in}}%
\pgfpathlineto{\pgfqpoint{5.218839in}{2.986909in}}%
\pgfpathlineto{\pgfqpoint{5.219841in}{2.988273in}}%
\pgfpathlineto{\pgfqpoint{5.222848in}{2.996455in}}%
\pgfpathlineto{\pgfqpoint{5.223850in}{2.999182in}}%
\pgfpathlineto{\pgfqpoint{5.224852in}{3.015545in}}%
\pgfpathlineto{\pgfqpoint{5.225855in}{2.999182in}}%
\pgfpathlineto{\pgfqpoint{5.227859in}{3.021000in}}%
\pgfpathlineto{\pgfqpoint{5.228861in}{3.018273in}}%
\pgfpathlineto{\pgfqpoint{5.229864in}{3.027818in}}%
\pgfpathlineto{\pgfqpoint{5.230866in}{3.012818in}}%
\pgfpathlineto{\pgfqpoint{5.233873in}{3.037364in}}%
\pgfpathlineto{\pgfqpoint{5.234875in}{3.072818in}}%
\pgfpathlineto{\pgfqpoint{5.235877in}{3.021000in}}%
\pgfpathlineto{\pgfqpoint{5.239886in}{3.052364in}}%
\pgfpathlineto{\pgfqpoint{5.241891in}{3.027818in}}%
\pgfpathlineto{\pgfqpoint{5.243895in}{3.037364in}}%
\pgfpathlineto{\pgfqpoint{5.244897in}{3.038727in}}%
\pgfpathlineto{\pgfqpoint{5.246902in}{3.018273in}}%
\pgfpathlineto{\pgfqpoint{5.247904in}{3.008727in}}%
\pgfpathlineto{\pgfqpoint{5.248906in}{3.012818in}}%
\pgfpathlineto{\pgfqpoint{5.249908in}{3.007364in}}%
\pgfpathlineto{\pgfqpoint{5.250911in}{3.012818in}}%
\pgfpathlineto{\pgfqpoint{5.251913in}{3.011455in}}%
\pgfpathlineto{\pgfqpoint{5.253917in}{2.989636in}}%
\pgfpathlineto{\pgfqpoint{5.254920in}{3.003273in}}%
\pgfpathlineto{\pgfqpoint{5.255922in}{2.996455in}}%
\pgfpathlineto{\pgfqpoint{5.256924in}{3.012818in}}%
\pgfpathlineto{\pgfqpoint{5.258929in}{2.976000in}}%
\pgfpathlineto{\pgfqpoint{5.259931in}{2.993727in}}%
\pgfpathlineto{\pgfqpoint{5.262938in}{2.981455in}}%
\pgfpathlineto{\pgfqpoint{5.263940in}{2.982818in}}%
\pgfpathlineto{\pgfqpoint{5.264942in}{2.981455in}}%
\pgfpathlineto{\pgfqpoint{5.265944in}{2.973273in}}%
\pgfpathlineto{\pgfqpoint{5.267949in}{2.995091in}}%
\pgfpathlineto{\pgfqpoint{5.268951in}{2.989636in}}%
\pgfpathlineto{\pgfqpoint{5.269953in}{3.003273in}}%
\pgfpathlineto{\pgfqpoint{5.270956in}{2.992364in}}%
\pgfpathlineto{\pgfqpoint{5.272960in}{3.006000in}}%
\pgfpathlineto{\pgfqpoint{5.273962in}{3.007364in}}%
\pgfpathlineto{\pgfqpoint{5.274965in}{3.019636in}}%
\pgfpathlineto{\pgfqpoint{5.275967in}{3.010091in}}%
\pgfpathlineto{\pgfqpoint{5.276969in}{3.025091in}}%
\pgfpathlineto{\pgfqpoint{5.277971in}{3.010091in}}%
\pgfpathlineto{\pgfqpoint{5.279976in}{3.034636in}}%
\pgfpathlineto{\pgfqpoint{5.280978in}{3.022364in}}%
\pgfpathlineto{\pgfqpoint{5.281980in}{3.026455in}}%
\pgfpathlineto{\pgfqpoint{5.283985in}{3.044182in}}%
\pgfpathlineto{\pgfqpoint{5.284987in}{3.040091in}}%
\pgfpathlineto{\pgfqpoint{5.285989in}{3.014182in}}%
\pgfpathlineto{\pgfqpoint{5.288996in}{3.036000in}}%
\pgfpathlineto{\pgfqpoint{5.289998in}{3.031909in}}%
\pgfpathlineto{\pgfqpoint{5.291000in}{3.008727in}}%
\pgfpathlineto{\pgfqpoint{5.292003in}{3.012818in}}%
\pgfpathlineto{\pgfqpoint{5.294007in}{3.036000in}}%
\pgfpathlineto{\pgfqpoint{5.297014in}{2.993727in}}%
\pgfpathlineto{\pgfqpoint{5.299018in}{3.019636in}}%
\pgfpathlineto{\pgfqpoint{5.300021in}{3.004636in}}%
\pgfpathlineto{\pgfqpoint{5.301023in}{3.012818in}}%
\pgfpathlineto{\pgfqpoint{5.303027in}{2.991000in}}%
\pgfpathlineto{\pgfqpoint{5.305032in}{3.003273in}}%
\pgfpathlineto{\pgfqpoint{5.306034in}{3.008727in}}%
\pgfpathlineto{\pgfqpoint{5.309041in}{2.982818in}}%
\pgfpathlineto{\pgfqpoint{5.310043in}{2.977364in}}%
\pgfpathlineto{\pgfqpoint{5.312048in}{3.001909in}}%
\pgfpathlineto{\pgfqpoint{5.314052in}{2.974636in}}%
\pgfpathlineto{\pgfqpoint{5.315054in}{2.974636in}}%
\pgfpathlineto{\pgfqpoint{5.317059in}{2.999182in}}%
\pgfpathlineto{\pgfqpoint{5.319063in}{2.986909in}}%
\pgfpathlineto{\pgfqpoint{5.322070in}{3.001909in}}%
\pgfpathlineto{\pgfqpoint{5.323072in}{2.991000in}}%
\pgfpathlineto{\pgfqpoint{5.325077in}{3.034636in}}%
\pgfpathlineto{\pgfqpoint{5.327081in}{2.986909in}}%
\pgfpathlineto{\pgfqpoint{5.328083in}{2.986909in}}%
\pgfpathlineto{\pgfqpoint{5.329086in}{3.036000in}}%
\pgfpathlineto{\pgfqpoint{5.331090in}{3.012818in}}%
\pgfpathlineto{\pgfqpoint{5.332092in}{2.992364in}}%
\pgfpathlineto{\pgfqpoint{5.335099in}{3.041455in}}%
\pgfpathlineto{\pgfqpoint{5.338106in}{3.007364in}}%
\pgfpathlineto{\pgfqpoint{5.341113in}{3.033273in}}%
\pgfpathlineto{\pgfqpoint{5.342115in}{3.004636in}}%
\pgfpathlineto{\pgfqpoint{5.345122in}{3.027818in}}%
\pgfpathlineto{\pgfqpoint{5.347126in}{3.012818in}}%
\pgfpathlineto{\pgfqpoint{5.348128in}{3.015545in}}%
\pgfpathlineto{\pgfqpoint{5.349131in}{3.012818in}}%
\pgfpathlineto{\pgfqpoint{5.350133in}{3.006000in}}%
\pgfpathlineto{\pgfqpoint{5.351135in}{3.010091in}}%
\pgfpathlineto{\pgfqpoint{5.354142in}{2.980091in}}%
\pgfpathlineto{\pgfqpoint{5.356146in}{3.011455in}}%
\pgfpathlineto{\pgfqpoint{5.357148in}{2.991000in}}%
\pgfpathlineto{\pgfqpoint{5.361157in}{3.016909in}}%
\pgfpathlineto{\pgfqpoint{5.362160in}{2.978727in}}%
\pgfpathlineto{\pgfqpoint{5.363162in}{3.022364in}}%
\pgfpathlineto{\pgfqpoint{5.366169in}{2.981455in}}%
\pgfpathlineto{\pgfqpoint{5.367171in}{3.014182in}}%
\pgfpathlineto{\pgfqpoint{5.368173in}{3.011455in}}%
\pgfpathlineto{\pgfqpoint{5.370178in}{2.982818in}}%
\pgfpathlineto{\pgfqpoint{5.371180in}{3.011455in}}%
\pgfpathlineto{\pgfqpoint{5.372182in}{3.000545in}}%
\pgfpathlineto{\pgfqpoint{5.373184in}{3.001909in}}%
\pgfpathlineto{\pgfqpoint{5.375189in}{3.027818in}}%
\pgfpathlineto{\pgfqpoint{5.377193in}{3.001909in}}%
\pgfpathlineto{\pgfqpoint{5.378196in}{3.004636in}}%
\pgfpathlineto{\pgfqpoint{5.379198in}{3.029182in}}%
\pgfpathlineto{\pgfqpoint{5.380200in}{3.027818in}}%
\pgfpathlineto{\pgfqpoint{5.382205in}{2.992364in}}%
\pgfpathlineto{\pgfqpoint{5.384209in}{3.029182in}}%
\pgfpathlineto{\pgfqpoint{5.385211in}{3.034636in}}%
\pgfpathlineto{\pgfqpoint{5.387216in}{3.011455in}}%
\pgfpathlineto{\pgfqpoint{5.388218in}{3.015545in}}%
\pgfpathlineto{\pgfqpoint{5.389220in}{3.034636in}}%
\pgfpathlineto{\pgfqpoint{5.390222in}{3.021000in}}%
\pgfpathlineto{\pgfqpoint{5.391225in}{3.023727in}}%
\pgfpathlineto{\pgfqpoint{5.392227in}{3.006000in}}%
\pgfpathlineto{\pgfqpoint{5.394231in}{3.023727in}}%
\pgfpathlineto{\pgfqpoint{5.395234in}{3.014182in}}%
\pgfpathlineto{\pgfqpoint{5.396236in}{3.016909in}}%
\pgfpathlineto{\pgfqpoint{5.397238in}{3.010091in}}%
\pgfpathlineto{\pgfqpoint{5.398240in}{3.019636in}}%
\pgfpathlineto{\pgfqpoint{5.399243in}{3.012818in}}%
\pgfpathlineto{\pgfqpoint{5.400245in}{3.018273in}}%
\pgfpathlineto{\pgfqpoint{5.401247in}{3.014182in}}%
\pgfpathlineto{\pgfqpoint{5.402249in}{2.999182in}}%
\pgfpathlineto{\pgfqpoint{5.403252in}{3.021000in}}%
\pgfpathlineto{\pgfqpoint{5.404254in}{2.995091in}}%
\pgfpathlineto{\pgfqpoint{5.405256in}{3.007364in}}%
\pgfpathlineto{\pgfqpoint{5.407261in}{3.003273in}}%
\pgfpathlineto{\pgfqpoint{5.408263in}{2.986909in}}%
\pgfpathlineto{\pgfqpoint{5.409265in}{2.993727in}}%
\pgfpathlineto{\pgfqpoint{5.410267in}{2.992364in}}%
\pgfpathlineto{\pgfqpoint{5.411270in}{2.995091in}}%
\pgfpathlineto{\pgfqpoint{5.413274in}{3.012818in}}%
\pgfpathlineto{\pgfqpoint{5.416281in}{2.970545in}}%
\pgfpathlineto{\pgfqpoint{5.417283in}{2.982818in}}%
\pgfpathlineto{\pgfqpoint{5.418285in}{3.018273in}}%
\pgfpathlineto{\pgfqpoint{5.420290in}{2.996455in}}%
\pgfpathlineto{\pgfqpoint{5.421292in}{3.001909in}}%
\pgfpathlineto{\pgfqpoint{5.422294in}{2.986909in}}%
\pgfpathlineto{\pgfqpoint{5.425301in}{3.025091in}}%
\pgfpathlineto{\pgfqpoint{5.426303in}{2.997818in}}%
\pgfpathlineto{\pgfqpoint{5.427305in}{3.001909in}}%
\pgfpathlineto{\pgfqpoint{5.428308in}{3.000545in}}%
\pgfpathlineto{\pgfqpoint{5.429310in}{3.022364in}}%
\pgfpathlineto{\pgfqpoint{5.431314in}{2.996455in}}%
\pgfpathlineto{\pgfqpoint{5.432317in}{2.997818in}}%
\pgfpathlineto{\pgfqpoint{5.433319in}{3.041455in}}%
\pgfpathlineto{\pgfqpoint{5.435323in}{3.004636in}}%
\pgfpathlineto{\pgfqpoint{5.436326in}{3.021000in}}%
\pgfpathlineto{\pgfqpoint{5.437328in}{3.012818in}}%
\pgfpathlineto{\pgfqpoint{5.438330in}{3.025091in}}%
\pgfpathlineto{\pgfqpoint{5.439332in}{3.023727in}}%
\pgfpathlineto{\pgfqpoint{5.440335in}{3.004636in}}%
\pgfpathlineto{\pgfqpoint{5.442339in}{3.023727in}}%
\pgfpathlineto{\pgfqpoint{5.443341in}{3.007364in}}%
\pgfpathlineto{\pgfqpoint{5.444344in}{3.029182in}}%
\pgfpathlineto{\pgfqpoint{5.445346in}{3.025091in}}%
\pgfpathlineto{\pgfqpoint{5.447350in}{3.004636in}}%
\pgfpathlineto{\pgfqpoint{5.449355in}{3.027818in}}%
\pgfpathlineto{\pgfqpoint{5.451359in}{2.999182in}}%
\pgfpathlineto{\pgfqpoint{5.454366in}{3.025091in}}%
\pgfpathlineto{\pgfqpoint{5.455368in}{3.026455in}}%
\pgfpathlineto{\pgfqpoint{5.456371in}{3.010091in}}%
\pgfpathlineto{\pgfqpoint{5.457373in}{3.012818in}}%
\pgfpathlineto{\pgfqpoint{5.458375in}{3.010091in}}%
\pgfpathlineto{\pgfqpoint{5.459377in}{2.997818in}}%
\pgfpathlineto{\pgfqpoint{5.461382in}{3.016909in}}%
\pgfpathlineto{\pgfqpoint{5.463386in}{2.985545in}}%
\pgfpathlineto{\pgfqpoint{5.464388in}{2.988273in}}%
\pgfpathlineto{\pgfqpoint{5.466393in}{3.006000in}}%
\pgfpathlineto{\pgfqpoint{5.467395in}{2.992364in}}%
\pgfpathlineto{\pgfqpoint{5.468397in}{2.993727in}}%
\pgfpathlineto{\pgfqpoint{5.469400in}{2.999182in}}%
\pgfpathlineto{\pgfqpoint{5.470402in}{2.996455in}}%
\pgfpathlineto{\pgfqpoint{5.471404in}{3.003273in}}%
\pgfpathlineto{\pgfqpoint{5.472406in}{2.999182in}}%
\pgfpathlineto{\pgfqpoint{5.473409in}{3.000545in}}%
\pgfpathlineto{\pgfqpoint{5.476415in}{2.985545in}}%
\pgfpathlineto{\pgfqpoint{5.477418in}{3.008727in}}%
\pgfpathlineto{\pgfqpoint{5.479422in}{3.001909in}}%
\pgfpathlineto{\pgfqpoint{5.481427in}{3.014182in}}%
\pgfpathlineto{\pgfqpoint{5.482429in}{2.993727in}}%
\pgfpathlineto{\pgfqpoint{5.485436in}{3.015545in}}%
\pgfpathlineto{\pgfqpoint{5.486438in}{2.991000in}}%
\pgfpathlineto{\pgfqpoint{5.488442in}{3.003273in}}%
\pgfpathlineto{\pgfqpoint{5.489445in}{3.000545in}}%
\pgfpathlineto{\pgfqpoint{5.490447in}{3.042818in}}%
\pgfpathlineto{\pgfqpoint{5.492451in}{3.007364in}}%
\pgfpathlineto{\pgfqpoint{5.493453in}{3.010091in}}%
\pgfpathlineto{\pgfqpoint{5.494456in}{3.008727in}}%
\pgfpathlineto{\pgfqpoint{5.495458in}{3.021000in}}%
\pgfpathlineto{\pgfqpoint{5.496460in}{3.016909in}}%
\pgfpathlineto{\pgfqpoint{5.498465in}{3.006000in}}%
\pgfpathlineto{\pgfqpoint{5.499467in}{3.018273in}}%
\pgfpathlineto{\pgfqpoint{5.500469in}{3.007364in}}%
\pgfpathlineto{\pgfqpoint{5.501471in}{3.021000in}}%
\pgfpathlineto{\pgfqpoint{5.502474in}{3.014182in}}%
\pgfpathlineto{\pgfqpoint{5.504478in}{2.988273in}}%
\pgfpathlineto{\pgfqpoint{5.506483in}{3.010091in}}%
\pgfpathlineto{\pgfqpoint{5.507485in}{3.010091in}}%
\pgfpathlineto{\pgfqpoint{5.508487in}{2.989636in}}%
\pgfpathlineto{\pgfqpoint{5.510492in}{3.006000in}}%
\pgfpathlineto{\pgfqpoint{5.512496in}{2.989636in}}%
\pgfpathlineto{\pgfqpoint{5.514501in}{3.001909in}}%
\pgfpathlineto{\pgfqpoint{5.515503in}{3.001909in}}%
\pgfpathlineto{\pgfqpoint{5.516505in}{2.981455in}}%
\pgfpathlineto{\pgfqpoint{5.517507in}{3.004636in}}%
\pgfpathlineto{\pgfqpoint{5.518510in}{2.997818in}}%
\pgfpathlineto{\pgfqpoint{5.519512in}{2.999182in}}%
\pgfpathlineto{\pgfqpoint{5.522519in}{3.010091in}}%
\pgfpathlineto{\pgfqpoint{5.523521in}{3.014182in}}%
\pgfpathlineto{\pgfqpoint{5.524523in}{3.006000in}}%
\pgfpathlineto{\pgfqpoint{5.525525in}{3.016909in}}%
\pgfpathlineto{\pgfqpoint{5.526528in}{3.000545in}}%
\pgfpathlineto{\pgfqpoint{5.527530in}{3.007364in}}%
\pgfpathlineto{\pgfqpoint{5.528532in}{2.999182in}}%
\pgfpathlineto{\pgfqpoint{5.530536in}{3.008727in}}%
\pgfpathlineto{\pgfqpoint{5.532541in}{3.023727in}}%
\pgfpathlineto{\pgfqpoint{5.533543in}{3.010091in}}%
\pgfpathlineto{\pgfqpoint{5.534545in}{3.016909in}}%
\pgfpathlineto{\pgfqpoint{5.534545in}{3.016909in}}%
\pgfusepath{stroke}%
\end{pgfscope}%
\begin{pgfscope}%
\pgfsetbuttcap%
\pgfsetmiterjoin%
\definecolor{currentfill}{rgb}{0.000000,0.000000,0.000000}%
\pgfsetfillcolor{currentfill}%
\pgfsetlinewidth{1.003750pt}%
\definecolor{currentstroke}{rgb}{0.000000,0.000000,0.000000}%
\pgfsetstrokecolor{currentstroke}%
\pgfsetdash{}{0pt}%
\pgfsys@defobject{currentmarker}{\pgfqpoint{-0.069444in}{-0.069444in}}{\pgfqpoint{0.069444in}{0.069444in}}{%
\pgfpathmoveto{\pgfqpoint{0.069444in}{-0.000000in}}%
\pgfpathlineto{\pgfqpoint{-0.069444in}{0.069444in}}%
\pgfpathlineto{\pgfqpoint{-0.069444in}{-0.069444in}}%
\pgfpathlineto{\pgfqpoint{0.069444in}{-0.000000in}}%
\pgfpathclose%
\pgfusepath{stroke,fill}%
}%
\begin{pgfscope}%
\pgfsys@transformshift{5.760000in}{0.696000in}%
\pgfsys@useobject{currentmarker}{}%
\end{pgfscope}%
\end{pgfscope}%
\begin{pgfscope}%
\pgfsetbuttcap%
\pgfsetmiterjoin%
\definecolor{currentfill}{rgb}{0.000000,0.000000,0.000000}%
\pgfsetfillcolor{currentfill}%
\pgfsetlinewidth{1.003750pt}%
\definecolor{currentstroke}{rgb}{0.000000,0.000000,0.000000}%
\pgfsetstrokecolor{currentstroke}%
\pgfsetdash{}{0pt}%
\pgfsys@defobject{currentmarker}{\pgfqpoint{-0.069444in}{-0.069444in}}{\pgfqpoint{0.069444in}{0.069444in}}{%
\pgfpathmoveto{\pgfqpoint{0.000000in}{0.069444in}}%
\pgfpathlineto{\pgfqpoint{-0.069444in}{-0.069444in}}%
\pgfpathlineto{\pgfqpoint{0.069444in}{-0.069444in}}%
\pgfpathlineto{\pgfqpoint{0.000000in}{0.069444in}}%
\pgfpathclose%
\pgfusepath{stroke,fill}%
}%
\begin{pgfscope}%
\pgfsys@transformshift{1.025455in}{4.224000in}%
\pgfsys@useobject{currentmarker}{}%
\end{pgfscope}%
\end{pgfscope}%
\begin{pgfscope}%
\pgfsetrectcap%
\pgfsetmiterjoin%
\pgfsetlinewidth{0.803000pt}%
\definecolor{currentstroke}{rgb}{0.000000,0.000000,0.000000}%
\pgfsetstrokecolor{currentstroke}%
\pgfsetdash{}{0pt}%
\pgfpathmoveto{\pgfqpoint{1.025455in}{0.528000in}}%
\pgfpathlineto{\pgfqpoint{1.025455in}{4.224000in}}%
\pgfusepath{stroke}%
\end{pgfscope}%
\begin{pgfscope}%
\pgfsetrectcap%
\pgfsetmiterjoin%
\pgfsetlinewidth{0.803000pt}%
\definecolor{currentstroke}{rgb}{0.000000,0.000000,0.000000}%
\pgfsetstrokecolor{currentstroke}%
\pgfsetdash{}{0pt}%
\pgfpathmoveto{\pgfqpoint{0.800000in}{0.696000in}}%
\pgfpathlineto{\pgfqpoint{5.760000in}{0.696000in}}%
\pgfusepath{stroke}%
\end{pgfscope}%
\begin{pgfscope}%
\pgfsetbuttcap%
\pgfsetmiterjoin%
\definecolor{currentfill}{rgb}{1.000000,1.000000,1.000000}%
\pgfsetfillcolor{currentfill}%
\pgfsetfillopacity{0.800000}%
\pgfsetlinewidth{1.003750pt}%
\definecolor{currentstroke}{rgb}{0.800000,0.800000,0.800000}%
\pgfsetstrokecolor{currentstroke}%
\pgfsetstrokeopacity{0.800000}%
\pgfsetdash{}{0pt}%
\pgfpathmoveto{\pgfqpoint{4.807529in}{3.904556in}}%
\pgfpathlineto{\pgfqpoint{5.662778in}{3.904556in}}%
\pgfpathquadraticcurveto{\pgfqpoint{5.690556in}{3.904556in}}{\pgfqpoint{5.690556in}{3.932333in}}%
\pgfpathlineto{\pgfqpoint{5.690556in}{4.126778in}}%
\pgfpathquadraticcurveto{\pgfqpoint{5.690556in}{4.154556in}}{\pgfqpoint{5.662778in}{4.154556in}}%
\pgfpathlineto{\pgfqpoint{4.807529in}{4.154556in}}%
\pgfpathquadraticcurveto{\pgfqpoint{4.779752in}{4.154556in}}{\pgfqpoint{4.779752in}{4.126778in}}%
\pgfpathlineto{\pgfqpoint{4.779752in}{3.932333in}}%
\pgfpathquadraticcurveto{\pgfqpoint{4.779752in}{3.904556in}}{\pgfqpoint{4.807529in}{3.904556in}}%
\pgfpathlineto{\pgfqpoint{4.807529in}{3.904556in}}%
\pgfpathclose%
\pgfusepath{stroke,fill}%
\end{pgfscope}%
\begin{pgfscope}%
\pgfsetrectcap%
\pgfsetroundjoin%
\pgfsetlinewidth{1.505625pt}%
\definecolor{currentstroke}{rgb}{0.564706,0.478431,0.662745}%
\pgfsetstrokecolor{currentstroke}%
\pgfsetdash{}{0pt}%
\pgfpathmoveto{\pgfqpoint{4.835307in}{4.043444in}}%
\pgfpathlineto{\pgfqpoint{4.974196in}{4.043444in}}%
\pgfpathlineto{\pgfqpoint{5.113085in}{4.043444in}}%
\pgfusepath{stroke}%
\end{pgfscope}%
\begin{pgfscope}%
\definecolor{textcolor}{rgb}{0.000000,0.000000,0.000000}%
\pgfsetstrokecolor{textcolor}%
\pgfsetfillcolor{textcolor}%
\pgftext[x=5.224196in,y=3.994833in,left,base]{\color{textcolor}\rmfamily\fontsize{10.000000}{12.000000}\selectfont \(\displaystyle I_{\mathrm{mes}}(t)\)}%
\end{pgfscope}%
\end{pgfpicture}%
\makeatother%
\endgroup%
}
		\caption{Éclairement mesuré $I_\mathrm{mes}(t)$ en fonction du temps $t$ de rotation du moteur}
	\end{figure}

	\section{Étude théorique}

	Dans le cadre de ce \textsc{tp}, on ne s'intéresse qu'à deux types de filtres interférentiels : ceux ayant un profil spectral rectangulaire et ceux ayant un profil spectral gaussien.
	Pour les deux types de filtres, l'éclairement $I(e)$ est donné par la formule \[
		I(e) = I_0 \: \int_{\substack{\text{bande}\\\text{passante}}} F(\sigma) \cdot \big[1+\cos(2\pi\,(2e)\,\sigma)\big]~\mathrm{d}\sigma,
	\]où $\sigma$ est le nombre d'onde d'une radiation.
	Le choix de la fonction $F(\sigma)$ détermine l'effet du filtre sur des interférences, et c'est ce qui est étudié dans les deux sous-sections suivantes.

	\subsection{Filtre à profil spectral rectangulaire}

	\begin{figure}[H]
		\centering
		\begin{tikzpicture}[scale=0.5]
			\draw[->] (-0.5, 0) -- (9, 0);
			\draw[->] (0, -0.5) -- (0, 6);
			\draw[red, thick] (-0.2, 0) -- (2.5, 0) -- (2.5, 6) -- (6.5, 6) -- (6.5, 0) -- (8.8, 0);
			\node[red] at (7.5, 6){$F(\sigma)$};
			\draw[<->, blue] (2.7, 2.5) -- (6.3, 2.5);
			\node[blue] at (4.5, 3){$\mathrm{\Delta}\sigma$};
			\node at (9.5,0){$\sigma$};
			\draw[mauve] (4.5, -0.2)--(4.5, 0.2);
			\node[mauve] at (4.5, -0.5){$\sigma_0$};
		\end{tikzpicture}
		\caption{Fonction $F(\sigma)$ pour le profil spectral rectangulaire}
	\end{figure}

	Avec un profil spectral rectangulaire, la fonction $F(\sigma)$ est de la forme représentée sur la figure ci-avant : $F(\sigma)$ est nul si $\sigma < \sigma_0 - \mathrm{\Delta} \sigma / 2$ ou si $\sigma > \sigma_0 + \mathrm{\Delta} \sigma / 2$, et $F(\sigma)$ vaut $K / \mathrm{\Delta} \sigma$ sinon.

	Cherchons une expression simplifiée de $I(e)$ pour ce filtre. En remplaçant les bornes de la bande passante et la fonction $F(\sigma)$, on trouve \[
		I(e) = I_0 \int_{\sigma_0 - \mathrm{\Delta}\sigma / 2}^{\sigma_0 + \mathrm{\Delta}\sigma / 2} \frac{K}{\mathrm{\Delta}\sigma}\big[1+\cos(2\pi\,(2e)\,\sigma)\big]~\mathrm{d}\sigma
	.\]
	En sortant le facteur $K / \mathrm{\Delta}\sigma$, et à l'aide du changement de variable $\sigma \gets \sigma + \sigma_0$, on trouve 
	\[
		\frac{I_0K}{\mathrm{\Delta}\sigma} + \frac{I_0K}{\mathrm{\Delta}\sigma}  \int_{-\mathrm{\Delta}\sigma / 2}^{\mathrm{\Delta} \sigma / 2} \cos(2\pi\,(2e) \cdot (\sigma + \sigma_0))~\mathrm{d}\sigma 
	.\]
	On applique la formule d'addition des cosinus pour trouver
	\begin{align*}
		I_0 K &+
		\frac{I_0 K}{\mathrm{\Delta}\sigma}
		\int_{-\mathrm{\Delta}\sigma/2}^{\mathrm{\Delta}\sigma / 2} \cos(4\pi e \sigma_0) \cdot \cos(4\pi e \sigma)~\mathrm{d}\sigma\\
		&+
		\frac{I_0 K}{\mathrm{\Delta}\sigma}
		\int_{-\mathrm{\Delta}\sigma/2}^{\mathrm{\Delta}\sigma / 2} \sin(4\pi e \sigma_0) \cdot \sin(4\pi e \sigma)~\mathrm{d}\sigma.
	\end{align*}
	La seconde intégrale s'annule car l'intégrande est une fonction impaire, et l'intervalle est symétrique par rapport à 0.
	À l'aide d'un changement de variables $x \gets 2\sigma / \Delta\sigma$, l'intensité $I(e)$ s'écrit alors comme \[
		I_0K +
		\frac{I_0K}{\mathrm{2\:\Delta}\sigma} \cos(4\pi\,e\,\sigma_0) \int_{-1}^{1} \cos(2\pi\:e\:\mathrm{\Delta}\sigma \:x) ~\mathrm{d}x 
	.\]
	Ce qui donne, après intégration, \[
		I_0K \left( 1+ \frac{1}{2\:\mathrm{\Delta}\sigma} \cdot\cos(4\pi\,e\,\sigma_0) \cdot 2 \cdot \frac{\sin(2\pi \,e\, \mathrm{\Delta}\sigma)}{2\pi\,e\,\mathrm{\Delta}\sigma} \right) 
	,\] que l'on peut simplifier en \[
		I_0K\cdot \left( 1 + \cos(4\pi\,e\,\sigma_0) \cdot \operatorname{sinc}(2\pi\,e\,\mathrm{\Delta}\sigma) / \mathrm{\Delta}\sigma \right)
	.\]Comme le moteur avance à une vitesse constante, on pourra remplacer $e$ par $v\cdot (t - t_0)$.

	\begin{figure}[H]
		\centering
		\resizebox{\linewidth}{!}{%% Creator: Matplotlib, PGF backend
%%
%% To include the figure in your LaTeX document, write
%%   \input{<filename>.pgf}
%%
%% Make sure the required packages are loaded in your preamble
%%   \usepackage{pgf}
%%
%% Also ensure that all the required font packages are loaded; for instance,
%% the lmodern package is sometimes necessary when using math font.
%%   \usepackage{lmodern}
%%
%% Figures using additional raster images can only be included by \input if
%% they are in the same directory as the main LaTeX file. For loading figures
%% from other directories you can use the `import` package
%%   \usepackage{import}
%%
%% and then include the figures with
%%   \import{<path to file>}{<filename>.pgf}
%%
%% Matplotlib used the following preamble
%%   
%%   \makeatletter\@ifpackageloaded{underscore}{}{\usepackage[strings]{underscore}}\makeatother
%%
\begingroup%
\makeatletter%
\begin{pgfpicture}%
\pgfpathrectangle{\pgfpointorigin}{\pgfqpoint{6.400000in}{4.800000in}}%
\pgfusepath{use as bounding box, clip}%
\begin{pgfscope}%
\pgfsetbuttcap%
\pgfsetmiterjoin%
\definecolor{currentfill}{rgb}{1.000000,1.000000,1.000000}%
\pgfsetfillcolor{currentfill}%
\pgfsetlinewidth{0.000000pt}%
\definecolor{currentstroke}{rgb}{1.000000,1.000000,1.000000}%
\pgfsetstrokecolor{currentstroke}%
\pgfsetdash{}{0pt}%
\pgfpathmoveto{\pgfqpoint{0.000000in}{0.000000in}}%
\pgfpathlineto{\pgfqpoint{6.400000in}{0.000000in}}%
\pgfpathlineto{\pgfqpoint{6.400000in}{4.800000in}}%
\pgfpathlineto{\pgfqpoint{0.000000in}{4.800000in}}%
\pgfpathlineto{\pgfqpoint{0.000000in}{0.000000in}}%
\pgfpathclose%
\pgfusepath{fill}%
\end{pgfscope}%
\begin{pgfscope}%
\pgfsetbuttcap%
\pgfsetmiterjoin%
\definecolor{currentfill}{rgb}{1.000000,1.000000,1.000000}%
\pgfsetfillcolor{currentfill}%
\pgfsetlinewidth{0.000000pt}%
\definecolor{currentstroke}{rgb}{0.000000,0.000000,0.000000}%
\pgfsetstrokecolor{currentstroke}%
\pgfsetstrokeopacity{0.000000}%
\pgfsetdash{}{0pt}%
\pgfpathmoveto{\pgfqpoint{0.800000in}{0.528000in}}%
\pgfpathlineto{\pgfqpoint{5.760000in}{0.528000in}}%
\pgfpathlineto{\pgfqpoint{5.760000in}{4.224000in}}%
\pgfpathlineto{\pgfqpoint{0.800000in}{4.224000in}}%
\pgfpathlineto{\pgfqpoint{0.800000in}{0.528000in}}%
\pgfpathclose%
\pgfusepath{fill}%
\end{pgfscope}%
\begin{pgfscope}%
\pgfpathrectangle{\pgfqpoint{0.800000in}{0.528000in}}{\pgfqpoint{4.960000in}{3.696000in}}%
\pgfusepath{clip}%
\pgfsetrectcap%
\pgfsetroundjoin%
\pgfsetlinewidth{1.505625pt}%
\definecolor{currentstroke}{rgb}{0.843137,0.509804,0.494118}%
\pgfsetstrokecolor{currentstroke}%
\pgfsetdash{}{0pt}%
\pgfpathmoveto{\pgfqpoint{1.025455in}{2.549326in}}%
\pgfpathlineto{\pgfqpoint{1.027710in}{2.542832in}}%
\pgfpathlineto{\pgfqpoint{1.029966in}{2.525977in}}%
\pgfpathlineto{\pgfqpoint{1.034477in}{2.466508in}}%
\pgfpathlineto{\pgfqpoint{1.048011in}{2.245017in}}%
\pgfpathlineto{\pgfqpoint{1.050267in}{2.226594in}}%
\pgfpathlineto{\pgfqpoint{1.052523in}{2.217793in}}%
\pgfpathlineto{\pgfqpoint{1.054778in}{2.219022in}}%
\pgfpathlineto{\pgfqpoint{1.057034in}{2.230049in}}%
\pgfpathlineto{\pgfqpoint{1.061545in}{2.277561in}}%
\pgfpathlineto{\pgfqpoint{1.077335in}{2.503194in}}%
\pgfpathlineto{\pgfqpoint{1.079591in}{2.515255in}}%
\pgfpathlineto{\pgfqpoint{1.081846in}{2.518385in}}%
\pgfpathlineto{\pgfqpoint{1.084102in}{2.512556in}}%
\pgfpathlineto{\pgfqpoint{1.086358in}{2.498309in}}%
\pgfpathlineto{\pgfqpoint{1.090869in}{2.449255in}}%
\pgfpathlineto{\pgfqpoint{1.104403in}{2.272564in}}%
\pgfpathlineto{\pgfqpoint{1.106659in}{2.258535in}}%
\pgfpathlineto{\pgfqpoint{1.108914in}{2.252176in}}%
\pgfpathlineto{\pgfqpoint{1.111170in}{2.253704in}}%
\pgfpathlineto{\pgfqpoint{1.113426in}{2.262830in}}%
\pgfpathlineto{\pgfqpoint{1.117937in}{2.300413in}}%
\pgfpathlineto{\pgfqpoint{1.133727in}{2.469993in}}%
\pgfpathlineto{\pgfqpoint{1.135983in}{2.478283in}}%
\pgfpathlineto{\pgfqpoint{1.138238in}{2.479930in}}%
\pgfpathlineto{\pgfqpoint{1.140494in}{2.475037in}}%
\pgfpathlineto{\pgfqpoint{1.145005in}{2.448063in}}%
\pgfpathlineto{\pgfqpoint{1.154028in}{2.358771in}}%
\pgfpathlineto{\pgfqpoint{1.158539in}{2.319629in}}%
\pgfpathlineto{\pgfqpoint{1.163051in}{2.297078in}}%
\pgfpathlineto{\pgfqpoint{1.165306in}{2.293563in}}%
\pgfpathlineto{\pgfqpoint{1.167562in}{2.295348in}}%
\pgfpathlineto{\pgfqpoint{1.169818in}{2.302098in}}%
\pgfpathlineto{\pgfqpoint{1.174329in}{2.327660in}}%
\pgfpathlineto{\pgfqpoint{1.187863in}{2.423151in}}%
\pgfpathlineto{\pgfqpoint{1.192374in}{2.434980in}}%
\pgfpathlineto{\pgfqpoint{1.194630in}{2.434987in}}%
\pgfpathlineto{\pgfqpoint{1.196886in}{2.431279in}}%
\pgfpathlineto{\pgfqpoint{1.201397in}{2.414793in}}%
\pgfpathlineto{\pgfqpoint{1.214931in}{2.349570in}}%
\pgfpathlineto{\pgfqpoint{1.219442in}{2.341147in}}%
\pgfpathlineto{\pgfqpoint{1.221698in}{2.340795in}}%
\pgfpathlineto{\pgfqpoint{1.223954in}{2.342786in}}%
\pgfpathlineto{\pgfqpoint{1.228465in}{2.352181in}}%
\pgfpathlineto{\pgfqpoint{1.239744in}{2.381944in}}%
\pgfpathlineto{\pgfqpoint{1.244255in}{2.387013in}}%
\pgfpathlineto{\pgfqpoint{1.246511in}{2.387423in}}%
\pgfpathlineto{\pgfqpoint{1.251022in}{2.384871in}}%
\pgfpathlineto{\pgfqpoint{1.260045in}{2.376255in}}%
\pgfpathlineto{\pgfqpoint{1.264556in}{2.375764in}}%
\pgfpathlineto{\pgfqpoint{1.269067in}{2.379409in}}%
\pgfpathlineto{\pgfqpoint{1.280346in}{2.394569in}}%
\pgfpathlineto{\pgfqpoint{1.282601in}{2.395406in}}%
\pgfpathlineto{\pgfqpoint{1.284857in}{2.394633in}}%
\pgfpathlineto{\pgfqpoint{1.287113in}{2.392060in}}%
\pgfpathlineto{\pgfqpoint{1.291624in}{2.381534in}}%
\pgfpathlineto{\pgfqpoint{1.307414in}{2.331167in}}%
\pgfpathlineto{\pgfqpoint{1.309669in}{2.330447in}}%
\pgfpathlineto{\pgfqpoint{1.311925in}{2.332858in}}%
\pgfpathlineto{\pgfqpoint{1.314181in}{2.338506in}}%
\pgfpathlineto{\pgfqpoint{1.318692in}{2.358742in}}%
\pgfpathlineto{\pgfqpoint{1.332226in}{2.440285in}}%
\pgfpathlineto{\pgfqpoint{1.334482in}{2.446799in}}%
\pgfpathlineto{\pgfqpoint{1.336737in}{2.449004in}}%
\pgfpathlineto{\pgfqpoint{1.338993in}{2.446485in}}%
\pgfpathlineto{\pgfqpoint{1.341249in}{2.439130in}}%
\pgfpathlineto{\pgfqpoint{1.345760in}{2.411121in}}%
\pgfpathlineto{\pgfqpoint{1.361550in}{2.280993in}}%
\pgfpathlineto{\pgfqpoint{1.363806in}{2.275682in}}%
\pgfpathlineto{\pgfqpoint{1.366061in}{2.276675in}}%
\pgfpathlineto{\pgfqpoint{1.368317in}{2.284185in}}%
\pgfpathlineto{\pgfqpoint{1.372828in}{2.317472in}}%
\pgfpathlineto{\pgfqpoint{1.379595in}{2.397719in}}%
\pgfpathlineto{\pgfqpoint{1.386362in}{2.475162in}}%
\pgfpathlineto{\pgfqpoint{1.390874in}{2.502004in}}%
\pgfpathlineto{\pgfqpoint{1.393129in}{2.504201in}}%
\pgfpathlineto{\pgfqpoint{1.395385in}{2.498203in}}%
\pgfpathlineto{\pgfqpoint{1.397641in}{2.484120in}}%
\pgfpathlineto{\pgfqpoint{1.402152in}{2.434782in}}%
\pgfpathlineto{\pgfqpoint{1.415686in}{2.243474in}}%
\pgfpathlineto{\pgfqpoint{1.417942in}{2.227402in}}%
\pgfpathlineto{\pgfqpoint{1.420197in}{2.220401in}}%
\pgfpathlineto{\pgfqpoint{1.422453in}{2.223181in}}%
\pgfpathlineto{\pgfqpoint{1.424709in}{2.235835in}}%
\pgfpathlineto{\pgfqpoint{1.429220in}{2.287972in}}%
\pgfpathlineto{\pgfqpoint{1.445010in}{2.542726in}}%
\pgfpathlineto{\pgfqpoint{1.447265in}{2.556015in}}%
\pgfpathlineto{\pgfqpoint{1.449521in}{2.558121in}}%
\pgfpathlineto{\pgfqpoint{1.451777in}{2.548646in}}%
\pgfpathlineto{\pgfqpoint{1.456288in}{2.497048in}}%
\pgfpathlineto{\pgfqpoint{1.463055in}{2.363682in}}%
\pgfpathlineto{\pgfqpoint{1.469822in}{2.228452in}}%
\pgfpathlineto{\pgfqpoint{1.474334in}{2.175966in}}%
\pgfpathlineto{\pgfqpoint{1.476589in}{2.167430in}}%
\pgfpathlineto{\pgfqpoint{1.478845in}{2.172007in}}%
\pgfpathlineto{\pgfqpoint{1.481101in}{2.189660in}}%
\pgfpathlineto{\pgfqpoint{1.485612in}{2.259894in}}%
\pgfpathlineto{\pgfqpoint{1.501402in}{2.590395in}}%
\pgfpathlineto{\pgfqpoint{1.503657in}{2.606706in}}%
\pgfpathlineto{\pgfqpoint{1.505913in}{2.608634in}}%
\pgfpathlineto{\pgfqpoint{1.508169in}{2.595815in}}%
\pgfpathlineto{\pgfqpoint{1.512680in}{2.529147in}}%
\pgfpathlineto{\pgfqpoint{1.519447in}{2.360557in}}%
\pgfpathlineto{\pgfqpoint{1.526214in}{2.192626in}}%
\pgfpathlineto{\pgfqpoint{1.530725in}{2.128781in}}%
\pgfpathlineto{\pgfqpoint{1.532981in}{2.118936in}}%
\pgfpathlineto{\pgfqpoint{1.535237in}{2.125253in}}%
\pgfpathlineto{\pgfqpoint{1.537492in}{2.147560in}}%
\pgfpathlineto{\pgfqpoint{1.542004in}{2.234401in}}%
\pgfpathlineto{\pgfqpoint{1.557793in}{2.633013in}}%
\pgfpathlineto{\pgfqpoint{1.560049in}{2.651922in}}%
\pgfpathlineto{\pgfqpoint{1.562305in}{2.653587in}}%
\pgfpathlineto{\pgfqpoint{1.564560in}{2.637693in}}%
\pgfpathlineto{\pgfqpoint{1.569072in}{2.557507in}}%
\pgfpathlineto{\pgfqpoint{1.575839in}{2.357816in}}%
\pgfpathlineto{\pgfqpoint{1.582606in}{2.161451in}}%
\pgfpathlineto{\pgfqpoint{1.587117in}{2.087942in}}%
\pgfpathlineto{\pgfqpoint{1.589373in}{2.077079in}}%
\pgfpathlineto{\pgfqpoint{1.591629in}{2.085011in}}%
\pgfpathlineto{\pgfqpoint{1.593884in}{2.111426in}}%
\pgfpathlineto{\pgfqpoint{1.598396in}{2.212647in}}%
\pgfpathlineto{\pgfqpoint{1.614185in}{2.668595in}}%
\pgfpathlineto{\pgfqpoint{1.616441in}{2.689546in}}%
\pgfpathlineto{\pgfqpoint{1.618697in}{2.690864in}}%
\pgfpathlineto{\pgfqpoint{1.620952in}{2.672300in}}%
\pgfpathlineto{\pgfqpoint{1.625464in}{2.580777in}}%
\pgfpathlineto{\pgfqpoint{1.632231in}{2.355593in}}%
\pgfpathlineto{\pgfqpoint{1.638998in}{2.136458in}}%
\pgfpathlineto{\pgfqpoint{1.643509in}{2.055476in}}%
\pgfpathlineto{\pgfqpoint{1.645765in}{2.043947in}}%
\pgfpathlineto{\pgfqpoint{1.648020in}{2.053298in}}%
\pgfpathlineto{\pgfqpoint{1.650276in}{2.083079in}}%
\pgfpathlineto{\pgfqpoint{1.654787in}{2.195741in}}%
\pgfpathlineto{\pgfqpoint{1.670577in}{2.695247in}}%
\pgfpathlineto{\pgfqpoint{1.672833in}{2.717564in}}%
\pgfpathlineto{\pgfqpoint{1.675088in}{2.718458in}}%
\pgfpathlineto{\pgfqpoint{1.677344in}{2.697761in}}%
\pgfpathlineto{\pgfqpoint{1.681855in}{2.597678in}}%
\pgfpathlineto{\pgfqpoint{1.688622in}{2.354011in}}%
\pgfpathlineto{\pgfqpoint{1.695390in}{2.119081in}}%
\pgfpathlineto{\pgfqpoint{1.699901in}{2.033274in}}%
\pgfpathlineto{\pgfqpoint{1.702157in}{2.021487in}}%
\pgfpathlineto{\pgfqpoint{1.704412in}{2.031994in}}%
\pgfpathlineto{\pgfqpoint{1.706668in}{2.064214in}}%
\pgfpathlineto{\pgfqpoint{1.711179in}{2.184712in}}%
\pgfpathlineto{\pgfqpoint{1.726969in}{2.711237in}}%
\pgfpathlineto{\pgfqpoint{1.729225in}{2.734136in}}%
\pgfpathlineto{\pgfqpoint{1.731480in}{2.734537in}}%
\pgfpathlineto{\pgfqpoint{1.733736in}{2.712365in}}%
\pgfpathlineto{\pgfqpoint{1.738247in}{2.607048in}}%
\pgfpathlineto{\pgfqpoint{1.747270in}{2.263636in}}%
\pgfpathlineto{\pgfqpoint{1.751781in}{2.110611in}}%
\pgfpathlineto{\pgfqpoint{1.756293in}{2.023034in}}%
\pgfpathlineto{\pgfqpoint{1.758548in}{2.011441in}}%
\pgfpathlineto{\pgfqpoint{1.760804in}{2.022777in}}%
\pgfpathlineto{\pgfqpoint{1.763060in}{2.056341in}}%
\pgfpathlineto{\pgfqpoint{1.767571in}{2.180471in}}%
\pgfpathlineto{\pgfqpoint{1.781105in}{2.671098in}}%
\pgfpathlineto{\pgfqpoint{1.785616in}{2.737652in}}%
\pgfpathlineto{\pgfqpoint{1.787872in}{2.737502in}}%
\pgfpathlineto{\pgfqpoint{1.790128in}{2.714624in}}%
\pgfpathlineto{\pgfqpoint{1.794639in}{2.607882in}}%
\pgfpathlineto{\pgfqpoint{1.803662in}{2.263933in}}%
\pgfpathlineto{\pgfqpoint{1.808173in}{2.112149in}}%
\pgfpathlineto{\pgfqpoint{1.812685in}{2.026194in}}%
\pgfpathlineto{\pgfqpoint{1.814940in}{2.015284in}}%
\pgfpathlineto{\pgfqpoint{1.817196in}{2.027062in}}%
\pgfpathlineto{\pgfqpoint{1.819452in}{2.060730in}}%
\pgfpathlineto{\pgfqpoint{1.823963in}{2.183783in}}%
\pgfpathlineto{\pgfqpoint{1.837497in}{2.663217in}}%
\pgfpathlineto{\pgfqpoint{1.839753in}{2.705434in}}%
\pgfpathlineto{\pgfqpoint{1.842008in}{2.726797in}}%
\pgfpathlineto{\pgfqpoint{1.844264in}{2.726049in}}%
\pgfpathlineto{\pgfqpoint{1.846520in}{2.703330in}}%
\pgfpathlineto{\pgfqpoint{1.851031in}{2.599366in}}%
\pgfpathlineto{\pgfqpoint{1.866821in}{2.074624in}}%
\pgfpathlineto{\pgfqpoint{1.869076in}{2.043887in}}%
\pgfpathlineto{\pgfqpoint{1.871332in}{2.034170in}}%
\pgfpathlineto{\pgfqpoint{1.873588in}{2.045956in}}%
\pgfpathlineto{\pgfqpoint{1.875843in}{2.078368in}}%
\pgfpathlineto{\pgfqpoint{1.880355in}{2.195236in}}%
\pgfpathlineto{\pgfqpoint{1.893889in}{2.642875in}}%
\pgfpathlineto{\pgfqpoint{1.896144in}{2.681468in}}%
\pgfpathlineto{\pgfqpoint{1.898400in}{2.700597in}}%
\pgfpathlineto{\pgfqpoint{1.900656in}{2.699220in}}%
\pgfpathlineto{\pgfqpoint{1.902911in}{2.677596in}}%
\pgfpathlineto{\pgfqpoint{1.907423in}{2.580913in}}%
\pgfpathlineto{\pgfqpoint{1.923213in}{2.103917in}}%
\pgfpathlineto{\pgfqpoint{1.925468in}{2.076886in}}%
\pgfpathlineto{\pgfqpoint{1.927724in}{2.068883in}}%
\pgfpathlineto{\pgfqpoint{1.929980in}{2.080202in}}%
\pgfpathlineto{\pgfqpoint{1.932235in}{2.109917in}}%
\pgfpathlineto{\pgfqpoint{1.936747in}{2.215218in}}%
\pgfpathlineto{\pgfqpoint{1.950281in}{2.609563in}}%
\pgfpathlineto{\pgfqpoint{1.952536in}{2.642581in}}%
\pgfpathlineto{\pgfqpoint{1.954792in}{2.658459in}}%
\pgfpathlineto{\pgfqpoint{1.957048in}{2.656438in}}%
\pgfpathlineto{\pgfqpoint{1.959303in}{2.636899in}}%
\pgfpathlineto{\pgfqpoint{1.963815in}{2.552184in}}%
\pgfpathlineto{\pgfqpoint{1.979604in}{2.147401in}}%
\pgfpathlineto{\pgfqpoint{1.981860in}{2.125580in}}%
\pgfpathlineto{\pgfqpoint{1.984116in}{2.119806in}}%
\pgfpathlineto{\pgfqpoint{1.986371in}{2.130155in}}%
\pgfpathlineto{\pgfqpoint{1.988627in}{2.155680in}}%
\pgfpathlineto{\pgfqpoint{1.993138in}{2.243898in}}%
\pgfpathlineto{\pgfqpoint{2.006672in}{2.563111in}}%
\pgfpathlineto{\pgfqpoint{2.008928in}{2.588593in}}%
\pgfpathlineto{\pgfqpoint{2.011184in}{2.600209in}}%
\pgfpathlineto{\pgfqpoint{2.013439in}{2.597546in}}%
\pgfpathlineto{\pgfqpoint{2.015695in}{2.581106in}}%
\pgfpathlineto{\pgfqpoint{2.020206in}{2.513110in}}%
\pgfpathlineto{\pgfqpoint{2.035996in}{2.205065in}}%
\pgfpathlineto{\pgfqpoint{2.038252in}{2.189936in}}%
\pgfpathlineto{\pgfqpoint{2.040508in}{2.186890in}}%
\pgfpathlineto{\pgfqpoint{2.042763in}{2.195750in}}%
\pgfpathlineto{\pgfqpoint{2.047275in}{2.244791in}}%
\pgfpathlineto{\pgfqpoint{2.056297in}{2.407384in}}%
\pgfpathlineto{\pgfqpoint{2.060809in}{2.478680in}}%
\pgfpathlineto{\pgfqpoint{2.065320in}{2.519729in}}%
\pgfpathlineto{\pgfqpoint{2.067576in}{2.526104in}}%
\pgfpathlineto{\pgfqpoint{2.069831in}{2.522819in}}%
\pgfpathlineto{\pgfqpoint{2.072087in}{2.510492in}}%
\pgfpathlineto{\pgfqpoint{2.076598in}{2.463904in}}%
\pgfpathlineto{\pgfqpoint{2.090132in}{2.290646in}}%
\pgfpathlineto{\pgfqpoint{2.092388in}{2.276508in}}%
\pgfpathlineto{\pgfqpoint{2.094644in}{2.269496in}}%
\pgfpathlineto{\pgfqpoint{2.096899in}{2.269642in}}%
\pgfpathlineto{\pgfqpoint{2.099155in}{2.276494in}}%
\pgfpathlineto{\pgfqpoint{2.103666in}{2.306446in}}%
\pgfpathlineto{\pgfqpoint{2.117200in}{2.422467in}}%
\pgfpathlineto{\pgfqpoint{2.121712in}{2.436629in}}%
\pgfpathlineto{\pgfqpoint{2.123967in}{2.436846in}}%
\pgfpathlineto{\pgfqpoint{2.126223in}{2.432975in}}%
\pgfpathlineto{\pgfqpoint{2.130734in}{2.416092in}}%
\pgfpathlineto{\pgfqpoint{2.139757in}{2.373894in}}%
\pgfpathlineto{\pgfqpoint{2.144268in}{2.362385in}}%
\pgfpathlineto{\pgfqpoint{2.146524in}{2.360468in}}%
\pgfpathlineto{\pgfqpoint{2.148780in}{2.360941in}}%
\pgfpathlineto{\pgfqpoint{2.153291in}{2.367127in}}%
\pgfpathlineto{\pgfqpoint{2.160058in}{2.378737in}}%
\pgfpathlineto{\pgfqpoint{2.162314in}{2.380290in}}%
\pgfpathlineto{\pgfqpoint{2.164570in}{2.379795in}}%
\pgfpathlineto{\pgfqpoint{2.166825in}{2.377041in}}%
\pgfpathlineto{\pgfqpoint{2.171337in}{2.365261in}}%
\pgfpathlineto{\pgfqpoint{2.180359in}{2.333569in}}%
\pgfpathlineto{\pgfqpoint{2.182615in}{2.329167in}}%
\pgfpathlineto{\pgfqpoint{2.184871in}{2.327966in}}%
\pgfpathlineto{\pgfqpoint{2.187126in}{2.330583in}}%
\pgfpathlineto{\pgfqpoint{2.189382in}{2.337349in}}%
\pgfpathlineto{\pgfqpoint{2.193893in}{2.362942in}}%
\pgfpathlineto{\pgfqpoint{2.207427in}{2.470251in}}%
\pgfpathlineto{\pgfqpoint{2.209683in}{2.477978in}}%
\pgfpathlineto{\pgfqpoint{2.211939in}{2.479315in}}%
\pgfpathlineto{\pgfqpoint{2.214194in}{2.473597in}}%
\pgfpathlineto{\pgfqpoint{2.216450in}{2.460632in}}%
\pgfpathlineto{\pgfqpoint{2.220961in}{2.414813in}}%
\pgfpathlineto{\pgfqpoint{2.234495in}{2.232309in}}%
\pgfpathlineto{\pgfqpoint{2.236751in}{2.217829in}}%
\pgfpathlineto{\pgfqpoint{2.239007in}{2.212967in}}%
\pgfpathlineto{\pgfqpoint{2.241262in}{2.218638in}}%
\pgfpathlineto{\pgfqpoint{2.243518in}{2.235083in}}%
\pgfpathlineto{\pgfqpoint{2.248029in}{2.297613in}}%
\pgfpathlineto{\pgfqpoint{2.263819in}{2.588436in}}%
\pgfpathlineto{\pgfqpoint{2.266075in}{2.600421in}}%
\pgfpathlineto{\pgfqpoint{2.268331in}{2.598320in}}%
\pgfpathlineto{\pgfqpoint{2.270586in}{2.581631in}}%
\pgfpathlineto{\pgfqpoint{2.275098in}{2.507251in}}%
\pgfpathlineto{\pgfqpoint{2.281865in}{2.327187in}}%
\pgfpathlineto{\pgfqpoint{2.288632in}{2.152569in}}%
\pgfpathlineto{\pgfqpoint{2.293143in}{2.091565in}}%
\pgfpathlineto{\pgfqpoint{2.295399in}{2.086327in}}%
\pgfpathlineto{\pgfqpoint{2.297654in}{2.099606in}}%
\pgfpathlineto{\pgfqpoint{2.299910in}{2.131212in}}%
\pgfpathlineto{\pgfqpoint{2.304421in}{2.242635in}}%
\pgfpathlineto{\pgfqpoint{2.317955in}{2.678740in}}%
\pgfpathlineto{\pgfqpoint{2.320211in}{2.715865in}}%
\pgfpathlineto{\pgfqpoint{2.322467in}{2.732316in}}%
\pgfpathlineto{\pgfqpoint{2.324722in}{2.726389in}}%
\pgfpathlineto{\pgfqpoint{2.326978in}{2.697783in}}%
\pgfpathlineto{\pgfqpoint{2.331489in}{2.578615in}}%
\pgfpathlineto{\pgfqpoint{2.347279in}{1.988544in}}%
\pgfpathlineto{\pgfqpoint{2.349535in}{1.957059in}}%
\pgfpathlineto{\pgfqpoint{2.351790in}{1.951543in}}%
\pgfpathlineto{\pgfqpoint{2.354046in}{1.973036in}}%
\pgfpathlineto{\pgfqpoint{2.356302in}{2.020860in}}%
\pgfpathlineto{\pgfqpoint{2.360813in}{2.184332in}}%
\pgfpathlineto{\pgfqpoint{2.374347in}{2.800041in}}%
\pgfpathlineto{\pgfqpoint{2.376603in}{2.850165in}}%
\pgfpathlineto{\pgfqpoint{2.378859in}{2.871201in}}%
\pgfpathlineto{\pgfqpoint{2.381114in}{2.861128in}}%
\pgfpathlineto{\pgfqpoint{2.383370in}{2.819879in}}%
\pgfpathlineto{\pgfqpoint{2.387881in}{2.653501in}}%
\pgfpathlineto{\pgfqpoint{2.403671in}{1.857341in}}%
\pgfpathlineto{\pgfqpoint{2.405927in}{1.816885in}}%
\pgfpathlineto{\pgfqpoint{2.408182in}{1.811199in}}%
\pgfpathlineto{\pgfqpoint{2.410438in}{1.841355in}}%
\pgfpathlineto{\pgfqpoint{2.414949in}{2.002145in}}%
\pgfpathlineto{\pgfqpoint{2.421716in}{2.413911in}}%
\pgfpathlineto{\pgfqpoint{2.428483in}{2.828725in}}%
\pgfpathlineto{\pgfqpoint{2.432995in}{2.988711in}}%
\pgfpathlineto{\pgfqpoint{2.435250in}{3.014357in}}%
\pgfpathlineto{\pgfqpoint{2.437506in}{2.999895in}}%
\pgfpathlineto{\pgfqpoint{2.439762in}{2.945519in}}%
\pgfpathlineto{\pgfqpoint{2.444273in}{2.730431in}}%
\pgfpathlineto{\pgfqpoint{2.460063in}{1.723339in}}%
\pgfpathlineto{\pgfqpoint{2.462318in}{1.673841in}}%
\pgfpathlineto{\pgfqpoint{2.464574in}{1.668101in}}%
\pgfpathlineto{\pgfqpoint{2.466830in}{1.707201in}}%
\pgfpathlineto{\pgfqpoint{2.471341in}{1.910119in}}%
\pgfpathlineto{\pgfqpoint{2.478108in}{2.422982in}}%
\pgfpathlineto{\pgfqpoint{2.484875in}{2.934069in}}%
\pgfpathlineto{\pgfqpoint{2.489387in}{3.128691in}}%
\pgfpathlineto{\pgfqpoint{2.491642in}{3.158874in}}%
\pgfpathlineto{\pgfqpoint{2.493898in}{3.139863in}}%
\pgfpathlineto{\pgfqpoint{2.496154in}{3.072141in}}%
\pgfpathlineto{\pgfqpoint{2.500665in}{2.807834in}}%
\pgfpathlineto{\pgfqpoint{2.516455in}{1.589303in}}%
\pgfpathlineto{\pgfqpoint{2.518710in}{1.530879in}}%
\pgfpathlineto{\pgfqpoint{2.520966in}{1.525205in}}%
\pgfpathlineto{\pgfqpoint{2.523222in}{1.573351in}}%
\pgfpathlineto{\pgfqpoint{2.527733in}{1.818457in}}%
\pgfpathlineto{\pgfqpoint{2.534500in}{2.431994in}}%
\pgfpathlineto{\pgfqpoint{2.541267in}{3.038465in}}%
\pgfpathlineto{\pgfqpoint{2.545778in}{3.267172in}}%
\pgfpathlineto{\pgfqpoint{2.548034in}{3.301722in}}%
\pgfpathlineto{\pgfqpoint{2.550290in}{3.278097in}}%
\pgfpathlineto{\pgfqpoint{2.552545in}{3.197087in}}%
\pgfpathlineto{\pgfqpoint{2.557057in}{2.884080in}}%
\pgfpathlineto{\pgfqpoint{2.572846in}{1.458073in}}%
\pgfpathlineto{\pgfqpoint{2.575102in}{1.391033in}}%
\pgfpathlineto{\pgfqpoint{2.577358in}{1.385547in}}%
\pgfpathlineto{\pgfqpoint{2.579613in}{1.442649in}}%
\pgfpathlineto{\pgfqpoint{2.584125in}{1.729109in}}%
\pgfpathlineto{\pgfqpoint{2.590892in}{2.440755in}}%
\pgfpathlineto{\pgfqpoint{2.597659in}{3.139677in}}%
\pgfpathlineto{\pgfqpoint{2.602170in}{3.401187in}}%
\pgfpathlineto{\pgfqpoint{2.604426in}{3.439835in}}%
\pgfpathlineto{\pgfqpoint{2.606682in}{3.411627in}}%
\pgfpathlineto{\pgfqpoint{2.608937in}{3.317669in}}%
\pgfpathlineto{\pgfqpoint{2.613449in}{2.957528in}}%
\pgfpathlineto{\pgfqpoint{2.629238in}{1.332491in}}%
\pgfpathlineto{\pgfqpoint{2.631494in}{1.257334in}}%
\pgfpathlineto{\pgfqpoint{2.633750in}{1.252157in}}%
\pgfpathlineto{\pgfqpoint{2.636005in}{1.317933in}}%
\pgfpathlineto{\pgfqpoint{2.640517in}{1.644019in}}%
\pgfpathlineto{\pgfqpoint{2.647284in}{2.449074in}}%
\pgfpathlineto{\pgfqpoint{2.654051in}{3.235495in}}%
\pgfpathlineto{\pgfqpoint{2.658562in}{3.527803in}}%
\pgfpathlineto{\pgfqpoint{2.660818in}{3.570191in}}%
\pgfpathlineto{\pgfqpoint{2.663073in}{3.537527in}}%
\pgfpathlineto{\pgfqpoint{2.665329in}{3.431245in}}%
\pgfpathlineto{\pgfqpoint{2.669840in}{3.026564in}}%
\pgfpathlineto{\pgfqpoint{2.685630in}{1.215326in}}%
\pgfpathlineto{\pgfqpoint{2.687886in}{1.132733in}}%
\pgfpathlineto{\pgfqpoint{2.690141in}{1.127979in}}%
\pgfpathlineto{\pgfqpoint{2.692397in}{1.201958in}}%
\pgfpathlineto{\pgfqpoint{2.696908in}{1.565069in}}%
\pgfpathlineto{\pgfqpoint{2.703675in}{2.456766in}}%
\pgfpathlineto{\pgfqpoint{2.710442in}{3.323791in}}%
\pgfpathlineto{\pgfqpoint{2.714954in}{3.644206in}}%
\pgfpathlineto{\pgfqpoint{2.717210in}{3.689890in}}%
\pgfpathlineto{\pgfqpoint{2.719465in}{3.652997in}}%
\pgfpathlineto{\pgfqpoint{2.721721in}{3.535286in}}%
\pgfpathlineto{\pgfqpoint{2.726232in}{3.089651in}}%
\pgfpathlineto{\pgfqpoint{2.742022in}{1.109201in}}%
\pgfpathlineto{\pgfqpoint{2.744278in}{1.020019in}}%
\pgfpathlineto{\pgfqpoint{2.746533in}{1.015795in}}%
\pgfpathlineto{\pgfqpoint{2.748789in}{1.097325in}}%
\pgfpathlineto{\pgfqpoint{2.753300in}{1.494030in}}%
\pgfpathlineto{\pgfqpoint{2.760067in}{2.463660in}}%
\pgfpathlineto{\pgfqpoint{2.766834in}{3.402578in}}%
\pgfpathlineto{\pgfqpoint{2.771346in}{3.747774in}}%
\pgfpathlineto{\pgfqpoint{2.773601in}{3.796235in}}%
\pgfpathlineto{\pgfqpoint{2.775857in}{3.755432in}}%
\pgfpathlineto{\pgfqpoint{2.778113in}{3.627445in}}%
\pgfpathlineto{\pgfqpoint{2.782624in}{3.145364in}}%
\pgfpathlineto{\pgfqpoint{2.798414in}{1.016518in}}%
\pgfpathlineto{\pgfqpoint{2.800669in}{0.921745in}}%
\pgfpathlineto{\pgfqpoint{2.802925in}{0.918147in}}%
\pgfpathlineto{\pgfqpoint{2.805181in}{1.006403in}}%
\pgfpathlineto{\pgfqpoint{2.809692in}{1.432514in}}%
\pgfpathlineto{\pgfqpoint{2.816459in}{2.469597in}}%
\pgfpathlineto{\pgfqpoint{2.823226in}{3.470066in}}%
\pgfpathlineto{\pgfqpoint{2.827738in}{3.836150in}}%
\pgfpathlineto{\pgfqpoint{2.829993in}{3.886804in}}%
\pgfpathlineto{\pgfqpoint{2.832249in}{3.842501in}}%
\pgfpathlineto{\pgfqpoint{2.834505in}{3.705623in}}%
\pgfpathlineto{\pgfqpoint{2.839016in}{3.192432in}}%
\pgfpathlineto{\pgfqpoint{2.854806in}{0.939396in}}%
\pgfpathlineto{\pgfqpoint{2.857061in}{0.840158in}}%
\pgfpathlineto{\pgfqpoint{2.859317in}{0.837270in}}%
\pgfpathlineto{\pgfqpoint{2.861573in}{0.931275in}}%
\pgfpathlineto{\pgfqpoint{2.866084in}{1.381929in}}%
\pgfpathlineto{\pgfqpoint{2.872851in}{2.474443in}}%
\pgfpathlineto{\pgfqpoint{2.879618in}{3.524705in}}%
\pgfpathlineto{\pgfqpoint{2.884129in}{3.907306in}}%
\pgfpathlineto{\pgfqpoint{2.886385in}{3.959519in}}%
\pgfpathlineto{\pgfqpoint{2.888641in}{3.912203in}}%
\pgfpathlineto{\pgfqpoint{2.890896in}{3.768023in}}%
\pgfpathlineto{\pgfqpoint{2.895408in}{3.229775in}}%
\pgfpathlineto{\pgfqpoint{2.911197in}{0.879610in}}%
\pgfpathlineto{\pgfqpoint{2.913453in}{0.777137in}}%
\pgfpathlineto{\pgfqpoint{2.915709in}{0.775025in}}%
\pgfpathlineto{\pgfqpoint{2.917964in}{0.873670in}}%
\pgfpathlineto{\pgfqpoint{2.922476in}{1.343442in}}%
\pgfpathlineto{\pgfqpoint{2.931498in}{2.878985in}}%
\pgfpathlineto{\pgfqpoint{2.938265in}{3.807181in}}%
\pgfpathlineto{\pgfqpoint{2.940521in}{3.959599in}}%
\pgfpathlineto{\pgfqpoint{2.942777in}{4.012700in}}%
\pgfpathlineto{\pgfqpoint{2.945033in}{3.962928in}}%
\pgfpathlineto{\pgfqpoint{2.947288in}{3.813204in}}%
\pgfpathlineto{\pgfqpoint{2.951800in}{3.256528in}}%
\pgfpathlineto{\pgfqpoint{2.965334in}{1.039334in}}%
\pgfpathlineto{\pgfqpoint{2.969845in}{0.734142in}}%
\pgfpathlineto{\pgfqpoint{2.972101in}{0.732855in}}%
\pgfpathlineto{\pgfqpoint{2.974356in}{0.834925in}}%
\pgfpathlineto{\pgfqpoint{2.978868in}{1.317944in}}%
\pgfpathlineto{\pgfqpoint{2.987890in}{2.890307in}}%
\pgfpathlineto{\pgfqpoint{2.992402in}{3.590716in}}%
\pgfpathlineto{\pgfqpoint{2.996913in}{3.991814in}}%
\pgfpathlineto{\pgfqpoint{2.999169in}{4.045111in}}%
\pgfpathlineto{\pgfqpoint{3.001424in}{3.993499in}}%
\pgfpathlineto{\pgfqpoint{3.003680in}{3.840117in}}%
\pgfpathlineto{\pgfqpoint{3.008191in}{3.272070in}}%
\pgfpathlineto{\pgfqpoint{3.021725in}{1.020027in}}%
\pgfpathlineto{\pgfqpoint{3.026237in}{0.712173in}}%
\pgfpathlineto{\pgfqpoint{3.028492in}{0.711740in}}%
\pgfpathlineto{\pgfqpoint{3.030748in}{0.815939in}}%
\pgfpathlineto{\pgfqpoint{3.035259in}{1.306029in}}%
\pgfpathlineto{\pgfqpoint{3.044282in}{2.895014in}}%
\pgfpathlineto{\pgfqpoint{3.048793in}{3.600553in}}%
\pgfpathlineto{\pgfqpoint{3.053305in}{4.003203in}}%
\pgfpathlineto{\pgfqpoint{3.055561in}{4.056000in}}%
\pgfpathlineto{\pgfqpoint{3.057816in}{4.003203in}}%
\pgfpathlineto{\pgfqpoint{3.060072in}{3.848134in}}%
\pgfpathlineto{\pgfqpoint{3.064583in}{3.276039in}}%
\pgfpathlineto{\pgfqpoint{3.078117in}{1.018263in}}%
\pgfpathlineto{\pgfqpoint{3.082629in}{0.711740in}}%
\pgfpathlineto{\pgfqpoint{3.084884in}{0.712173in}}%
\pgfpathlineto{\pgfqpoint{3.087140in}{0.817155in}}%
\pgfpathlineto{\pgfqpoint{3.091651in}{1.307974in}}%
\pgfpathlineto{\pgfqpoint{3.100674in}{2.892994in}}%
\pgfpathlineto{\pgfqpoint{3.105185in}{3.594518in}}%
\pgfpathlineto{\pgfqpoint{3.109697in}{3.993499in}}%
\pgfpathlineto{\pgfqpoint{3.111952in}{4.045111in}}%
\pgfpathlineto{\pgfqpoint{3.114208in}{3.991814in}}%
\pgfpathlineto{\pgfqpoint{3.116464in}{3.837069in}}%
\pgfpathlineto{\pgfqpoint{3.120975in}{3.268344in}}%
\pgfpathlineto{\pgfqpoint{3.134509in}{1.034085in}}%
\pgfpathlineto{\pgfqpoint{3.139020in}{0.732855in}}%
\pgfpathlineto{\pgfqpoint{3.141276in}{0.734142in}}%
\pgfpathlineto{\pgfqpoint{3.143532in}{0.838544in}}%
\pgfpathlineto{\pgfqpoint{3.148043in}{1.323734in}}%
\pgfpathlineto{\pgfqpoint{3.157066in}{2.884294in}}%
\pgfpathlineto{\pgfqpoint{3.161577in}{3.572751in}}%
\pgfpathlineto{\pgfqpoint{3.166088in}{3.962928in}}%
\pgfpathlineto{\pgfqpoint{3.168344in}{4.012700in}}%
\pgfpathlineto{\pgfqpoint{3.170600in}{3.959599in}}%
\pgfpathlineto{\pgfqpoint{3.172856in}{3.807181in}}%
\pgfpathlineto{\pgfqpoint{3.177367in}{3.249162in}}%
\pgfpathlineto{\pgfqpoint{3.190901in}{1.067123in}}%
\pgfpathlineto{\pgfqpoint{3.195412in}{0.775025in}}%
\pgfpathlineto{\pgfqpoint{3.197668in}{0.777137in}}%
\pgfpathlineto{\pgfqpoint{3.199924in}{0.879610in}}%
\pgfpathlineto{\pgfqpoint{3.204435in}{1.352942in}}%
\pgfpathlineto{\pgfqpoint{3.213458in}{2.869118in}}%
\pgfpathlineto{\pgfqpoint{3.217969in}{3.535760in}}%
\pgfpathlineto{\pgfqpoint{3.222480in}{3.912203in}}%
\pgfpathlineto{\pgfqpoint{3.224736in}{3.959519in}}%
\pgfpathlineto{\pgfqpoint{3.226992in}{3.907306in}}%
\pgfpathlineto{\pgfqpoint{3.229247in}{3.759163in}}%
\pgfpathlineto{\pgfqpoint{3.233759in}{3.218941in}}%
\pgfpathlineto{\pgfqpoint{3.247293in}{1.116609in}}%
\pgfpathlineto{\pgfqpoint{3.249548in}{0.931275in}}%
\pgfpathlineto{\pgfqpoint{3.251804in}{0.837270in}}%
\pgfpathlineto{\pgfqpoint{3.254060in}{0.840158in}}%
\pgfpathlineto{\pgfqpoint{3.256315in}{0.939396in}}%
\pgfpathlineto{\pgfqpoint{3.260827in}{1.394918in}}%
\pgfpathlineto{\pgfqpoint{3.269849in}{2.847818in}}%
\pgfpathlineto{\pgfqpoint{3.274361in}{3.484404in}}%
\pgfpathlineto{\pgfqpoint{3.278872in}{3.842501in}}%
\pgfpathlineto{\pgfqpoint{3.281128in}{3.886804in}}%
\pgfpathlineto{\pgfqpoint{3.283384in}{3.836150in}}%
\pgfpathlineto{\pgfqpoint{3.285639in}{3.694132in}}%
\pgfpathlineto{\pgfqpoint{3.290151in}{3.178381in}}%
\pgfpathlineto{\pgfqpoint{3.303685in}{1.181397in}}%
\pgfpathlineto{\pgfqpoint{3.305940in}{1.006403in}}%
\pgfpathlineto{\pgfqpoint{3.308196in}{0.918147in}}%
\pgfpathlineto{\pgfqpoint{3.310452in}{0.921745in}}%
\pgfpathlineto{\pgfqpoint{3.312707in}{1.016518in}}%
\pgfpathlineto{\pgfqpoint{3.317219in}{1.448693in}}%
\pgfpathlineto{\pgfqpoint{3.333008in}{3.627445in}}%
\pgfpathlineto{\pgfqpoint{3.335264in}{3.755432in}}%
\pgfpathlineto{\pgfqpoint{3.337520in}{3.796235in}}%
\pgfpathlineto{\pgfqpoint{3.339775in}{3.747774in}}%
\pgfpathlineto{\pgfqpoint{3.342031in}{3.613588in}}%
\pgfpathlineto{\pgfqpoint{3.346542in}{3.128419in}}%
\pgfpathlineto{\pgfqpoint{3.360076in}{1.259993in}}%
\pgfpathlineto{\pgfqpoint{3.362332in}{1.097325in}}%
\pgfpathlineto{\pgfqpoint{3.364588in}{1.015795in}}%
\pgfpathlineto{\pgfqpoint{3.366843in}{1.020019in}}%
\pgfpathlineto{\pgfqpoint{3.369099in}{1.109201in}}%
\pgfpathlineto{\pgfqpoint{3.373610in}{1.513026in}}%
\pgfpathlineto{\pgfqpoint{3.389400in}{3.535286in}}%
\pgfpathlineto{\pgfqpoint{3.391656in}{3.652997in}}%
\pgfpathlineto{\pgfqpoint{3.393912in}{3.689890in}}%
\pgfpathlineto{\pgfqpoint{3.396167in}{3.644206in}}%
\pgfpathlineto{\pgfqpoint{3.398423in}{3.519380in}}%
\pgfpathlineto{\pgfqpoint{3.402934in}{3.070200in}}%
\pgfpathlineto{\pgfqpoint{3.416468in}{1.350597in}}%
\pgfpathlineto{\pgfqpoint{3.418724in}{1.201958in}}%
\pgfpathlineto{\pgfqpoint{3.420980in}{1.127979in}}%
\pgfpathlineto{\pgfqpoint{3.423235in}{1.132733in}}%
\pgfpathlineto{\pgfqpoint{3.425491in}{1.215326in}}%
\pgfpathlineto{\pgfqpoint{3.430002in}{1.586451in}}%
\pgfpathlineto{\pgfqpoint{3.445792in}{3.431245in}}%
\pgfpathlineto{\pgfqpoint{3.448048in}{3.537527in}}%
\pgfpathlineto{\pgfqpoint{3.450303in}{3.570191in}}%
\pgfpathlineto{\pgfqpoint{3.452559in}{3.527803in}}%
\pgfpathlineto{\pgfqpoint{3.454815in}{3.413650in}}%
\pgfpathlineto{\pgfqpoint{3.459326in}{3.005048in}}%
\pgfpathlineto{\pgfqpoint{3.472860in}{1.451157in}}%
\pgfpathlineto{\pgfqpoint{3.475116in}{1.317933in}}%
\pgfpathlineto{\pgfqpoint{3.477371in}{1.252157in}}%
\pgfpathlineto{\pgfqpoint{3.479627in}{1.257334in}}%
\pgfpathlineto{\pgfqpoint{3.481883in}{1.332491in}}%
\pgfpathlineto{\pgfqpoint{3.486394in}{1.667304in}}%
\pgfpathlineto{\pgfqpoint{3.502184in}{3.317669in}}%
\pgfpathlineto{\pgfqpoint{3.504439in}{3.411627in}}%
\pgfpathlineto{\pgfqpoint{3.506695in}{3.439835in}}%
\pgfpathlineto{\pgfqpoint{3.508951in}{3.401187in}}%
\pgfpathlineto{\pgfqpoint{3.511207in}{3.298779in}}%
\pgfpathlineto{\pgfqpoint{3.515718in}{2.934429in}}%
\pgfpathlineto{\pgfqpoint{3.529252in}{1.559416in}}%
\pgfpathlineto{\pgfqpoint{3.531508in}{1.442649in}}%
\pgfpathlineto{\pgfqpoint{3.533763in}{1.385547in}}%
\pgfpathlineto{\pgfqpoint{3.536019in}{1.391033in}}%
\pgfpathlineto{\pgfqpoint{3.538275in}{1.458073in}}%
\pgfpathlineto{\pgfqpoint{3.542786in}{1.753780in}}%
\pgfpathlineto{\pgfqpoint{3.558576in}{3.197087in}}%
\pgfpathlineto{\pgfqpoint{3.560831in}{3.278097in}}%
\pgfpathlineto{\pgfqpoint{3.563087in}{3.301722in}}%
\pgfpathlineto{\pgfqpoint{3.565343in}{3.267172in}}%
\pgfpathlineto{\pgfqpoint{3.567598in}{3.177319in}}%
\pgfpathlineto{\pgfqpoint{3.572110in}{2.859908in}}%
\pgfpathlineto{\pgfqpoint{3.585644in}{1.672981in}}%
\pgfpathlineto{\pgfqpoint{3.587899in}{1.573351in}}%
\pgfpathlineto{\pgfqpoint{3.590155in}{1.525205in}}%
\pgfpathlineto{\pgfqpoint{3.592411in}{1.530879in}}%
\pgfpathlineto{\pgfqpoint{3.594666in}{1.589303in}}%
\pgfpathlineto{\pgfqpoint{3.599178in}{1.843973in}}%
\pgfpathlineto{\pgfqpoint{3.614967in}{3.072141in}}%
\pgfpathlineto{\pgfqpoint{3.617223in}{3.139863in}}%
\pgfpathlineto{\pgfqpoint{3.619479in}{3.158874in}}%
\pgfpathlineto{\pgfqpoint{3.621735in}{3.128691in}}%
\pgfpathlineto{\pgfqpoint{3.623990in}{3.051924in}}%
\pgfpathlineto{\pgfqpoint{3.628502in}{2.783112in}}%
\pgfpathlineto{\pgfqpoint{3.642036in}{1.789385in}}%
\pgfpathlineto{\pgfqpoint{3.644291in}{1.707201in}}%
\pgfpathlineto{\pgfqpoint{3.646547in}{1.668101in}}%
\pgfpathlineto{\pgfqpoint{3.648803in}{1.673841in}}%
\pgfpathlineto{\pgfqpoint{3.651058in}{1.723339in}}%
\pgfpathlineto{\pgfqpoint{3.655570in}{1.935933in}}%
\pgfpathlineto{\pgfqpoint{3.671359in}{2.945519in}}%
\pgfpathlineto{\pgfqpoint{3.673615in}{2.999895in}}%
\pgfpathlineto{\pgfqpoint{3.675871in}{3.014357in}}%
\pgfpathlineto{\pgfqpoint{3.678126in}{2.988711in}}%
\pgfpathlineto{\pgfqpoint{3.680382in}{2.925282in}}%
\pgfpathlineto{\pgfqpoint{3.684893in}{2.705685in}}%
\pgfpathlineto{\pgfqpoint{3.698427in}{1.906151in}}%
\pgfpathlineto{\pgfqpoint{3.700683in}{1.841355in}}%
\pgfpathlineto{\pgfqpoint{3.702939in}{1.811199in}}%
\pgfpathlineto{\pgfqpoint{3.705194in}{1.816885in}}%
\pgfpathlineto{\pgfqpoint{3.707450in}{1.857341in}}%
\pgfpathlineto{\pgfqpoint{3.711961in}{2.027715in}}%
\pgfpathlineto{\pgfqpoint{3.727751in}{2.819879in}}%
\pgfpathlineto{\pgfqpoint{3.730007in}{2.861128in}}%
\pgfpathlineto{\pgfqpoint{3.732262in}{2.871201in}}%
\pgfpathlineto{\pgfqpoint{3.734518in}{2.850165in}}%
\pgfpathlineto{\pgfqpoint{3.736774in}{2.800041in}}%
\pgfpathlineto{\pgfqpoint{3.741285in}{2.629242in}}%
\pgfpathlineto{\pgfqpoint{3.754819in}{2.020860in}}%
\pgfpathlineto{\pgfqpoint{3.757075in}{1.973036in}}%
\pgfpathlineto{\pgfqpoint{3.759331in}{1.951543in}}%
\pgfpathlineto{\pgfqpoint{3.761586in}{1.957059in}}%
\pgfpathlineto{\pgfqpoint{3.763842in}{1.988544in}}%
\pgfpathlineto{\pgfqpoint{3.768353in}{2.117431in}}%
\pgfpathlineto{\pgfqpoint{3.784143in}{2.697783in}}%
\pgfpathlineto{\pgfqpoint{3.786399in}{2.726389in}}%
\pgfpathlineto{\pgfqpoint{3.788654in}{2.732316in}}%
\pgfpathlineto{\pgfqpoint{3.790910in}{2.715865in}}%
\pgfpathlineto{\pgfqpoint{3.795421in}{2.623899in}}%
\pgfpathlineto{\pgfqpoint{3.804444in}{2.316306in}}%
\pgfpathlineto{\pgfqpoint{3.811211in}{2.131212in}}%
\pgfpathlineto{\pgfqpoint{3.813467in}{2.099606in}}%
\pgfpathlineto{\pgfqpoint{3.815722in}{2.086327in}}%
\pgfpathlineto{\pgfqpoint{3.817978in}{2.091565in}}%
\pgfpathlineto{\pgfqpoint{3.820234in}{2.114334in}}%
\pgfpathlineto{\pgfqpoint{3.824745in}{2.203298in}}%
\pgfpathlineto{\pgfqpoint{3.838279in}{2.550803in}}%
\pgfpathlineto{\pgfqpoint{3.842790in}{2.598320in}}%
\pgfpathlineto{\pgfqpoint{3.845046in}{2.600421in}}%
\pgfpathlineto{\pgfqpoint{3.847302in}{2.588436in}}%
\pgfpathlineto{\pgfqpoint{3.851813in}{2.528488in}}%
\pgfpathlineto{\pgfqpoint{3.867603in}{2.235083in}}%
\pgfpathlineto{\pgfqpoint{3.869859in}{2.218638in}}%
\pgfpathlineto{\pgfqpoint{3.872114in}{2.212967in}}%
\pgfpathlineto{\pgfqpoint{3.874370in}{2.217829in}}%
\pgfpathlineto{\pgfqpoint{3.876626in}{2.232309in}}%
\pgfpathlineto{\pgfqpoint{3.881137in}{2.283684in}}%
\pgfpathlineto{\pgfqpoint{3.892415in}{2.440750in}}%
\pgfpathlineto{\pgfqpoint{3.896927in}{2.473597in}}%
\pgfpathlineto{\pgfqpoint{3.899182in}{2.479315in}}%
\pgfpathlineto{\pgfqpoint{3.901438in}{2.477978in}}%
\pgfpathlineto{\pgfqpoint{3.903694in}{2.470251in}}%
\pgfpathlineto{\pgfqpoint{3.908205in}{2.440168in}}%
\pgfpathlineto{\pgfqpoint{3.919483in}{2.348260in}}%
\pgfpathlineto{\pgfqpoint{3.923995in}{2.330583in}}%
\pgfpathlineto{\pgfqpoint{3.926250in}{2.327966in}}%
\pgfpathlineto{\pgfqpoint{3.928506in}{2.329167in}}%
\pgfpathlineto{\pgfqpoint{3.933017in}{2.340342in}}%
\pgfpathlineto{\pgfqpoint{3.942040in}{2.372086in}}%
\pgfpathlineto{\pgfqpoint{3.946551in}{2.379795in}}%
\pgfpathlineto{\pgfqpoint{3.948807in}{2.380290in}}%
\pgfpathlineto{\pgfqpoint{3.951063in}{2.378737in}}%
\pgfpathlineto{\pgfqpoint{3.955574in}{2.371462in}}%
\pgfpathlineto{\pgfqpoint{3.960085in}{2.363370in}}%
\pgfpathlineto{\pgfqpoint{3.962341in}{2.360941in}}%
\pgfpathlineto{\pgfqpoint{3.964597in}{2.360468in}}%
\pgfpathlineto{\pgfqpoint{3.966853in}{2.362385in}}%
\pgfpathlineto{\pgfqpoint{3.969108in}{2.366887in}}%
\pgfpathlineto{\pgfqpoint{3.973620in}{2.383042in}}%
\pgfpathlineto{\pgfqpoint{3.984898in}{2.432975in}}%
\pgfpathlineto{\pgfqpoint{3.987154in}{2.436846in}}%
\pgfpathlineto{\pgfqpoint{3.989409in}{2.436629in}}%
\pgfpathlineto{\pgfqpoint{3.991665in}{2.431875in}}%
\pgfpathlineto{\pgfqpoint{3.996176in}{2.408661in}}%
\pgfpathlineto{\pgfqpoint{4.002943in}{2.348806in}}%
\pgfpathlineto{\pgfqpoint{4.009710in}{2.289170in}}%
\pgfpathlineto{\pgfqpoint{4.014222in}{2.269642in}}%
\pgfpathlineto{\pgfqpoint{4.016477in}{2.269496in}}%
\pgfpathlineto{\pgfqpoint{4.018733in}{2.276508in}}%
\pgfpathlineto{\pgfqpoint{4.023244in}{2.311370in}}%
\pgfpathlineto{\pgfqpoint{4.030011in}{2.400559in}}%
\pgfpathlineto{\pgfqpoint{4.036778in}{2.490305in}}%
\pgfpathlineto{\pgfqpoint{4.041290in}{2.522819in}}%
\pgfpathlineto{\pgfqpoint{4.043545in}{2.526104in}}%
\pgfpathlineto{\pgfqpoint{4.045801in}{2.519729in}}%
\pgfpathlineto{\pgfqpoint{4.048057in}{2.503701in}}%
\pgfpathlineto{\pgfqpoint{4.052568in}{2.445955in}}%
\pgfpathlineto{\pgfqpoint{4.066102in}{2.215586in}}%
\pgfpathlineto{\pgfqpoint{4.068358in}{2.195750in}}%
\pgfpathlineto{\pgfqpoint{4.070613in}{2.186890in}}%
\pgfpathlineto{\pgfqpoint{4.072869in}{2.189936in}}%
\pgfpathlineto{\pgfqpoint{4.075125in}{2.205065in}}%
\pgfpathlineto{\pgfqpoint{4.079636in}{2.268344in}}%
\pgfpathlineto{\pgfqpoint{4.095426in}{2.581106in}}%
\pgfpathlineto{\pgfqpoint{4.097682in}{2.597546in}}%
\pgfpathlineto{\pgfqpoint{4.099937in}{2.600209in}}%
\pgfpathlineto{\pgfqpoint{4.102193in}{2.588593in}}%
\pgfpathlineto{\pgfqpoint{4.106704in}{2.525087in}}%
\pgfpathlineto{\pgfqpoint{4.113471in}{2.360833in}}%
\pgfpathlineto{\pgfqpoint{4.120238in}{2.194493in}}%
\pgfpathlineto{\pgfqpoint{4.124750in}{2.130155in}}%
\pgfpathlineto{\pgfqpoint{4.127005in}{2.119806in}}%
\pgfpathlineto{\pgfqpoint{4.129261in}{2.125580in}}%
\pgfpathlineto{\pgfqpoint{4.131517in}{2.147401in}}%
\pgfpathlineto{\pgfqpoint{4.136028in}{2.233773in}}%
\pgfpathlineto{\pgfqpoint{4.151818in}{2.636899in}}%
\pgfpathlineto{\pgfqpoint{4.154073in}{2.656438in}}%
\pgfpathlineto{\pgfqpoint{4.156329in}{2.658459in}}%
\pgfpathlineto{\pgfqpoint{4.158585in}{2.642581in}}%
\pgfpathlineto{\pgfqpoint{4.163096in}{2.561273in}}%
\pgfpathlineto{\pgfqpoint{4.169863in}{2.357390in}}%
\pgfpathlineto{\pgfqpoint{4.176630in}{2.155950in}}%
\pgfpathlineto{\pgfqpoint{4.181141in}{2.080202in}}%
\pgfpathlineto{\pgfqpoint{4.183397in}{2.068883in}}%
\pgfpathlineto{\pgfqpoint{4.185653in}{2.076886in}}%
\pgfpathlineto{\pgfqpoint{4.187908in}{2.103917in}}%
\pgfpathlineto{\pgfqpoint{4.192420in}{2.207881in}}%
\pgfpathlineto{\pgfqpoint{4.208210in}{2.677596in}}%
\pgfpathlineto{\pgfqpoint{4.210465in}{2.699220in}}%
\pgfpathlineto{\pgfqpoint{4.212721in}{2.700597in}}%
\pgfpathlineto{\pgfqpoint{4.214977in}{2.681468in}}%
\pgfpathlineto{\pgfqpoint{4.219488in}{2.587106in}}%
\pgfpathlineto{\pgfqpoint{4.226255in}{2.354967in}}%
\pgfpathlineto{\pgfqpoint{4.233022in}{2.129235in}}%
\pgfpathlineto{\pgfqpoint{4.237533in}{2.045956in}}%
\pgfpathlineto{\pgfqpoint{4.239789in}{2.034170in}}%
\pgfpathlineto{\pgfqpoint{4.242045in}{2.043887in}}%
\pgfpathlineto{\pgfqpoint{4.244300in}{2.074624in}}%
\pgfpathlineto{\pgfqpoint{4.248812in}{2.190658in}}%
\pgfpathlineto{\pgfqpoint{4.264601in}{2.703330in}}%
\pgfpathlineto{\pgfqpoint{4.266857in}{2.726049in}}%
\pgfpathlineto{\pgfqpoint{4.269113in}{2.726797in}}%
\pgfpathlineto{\pgfqpoint{4.271368in}{2.705434in}}%
\pgfpathlineto{\pgfqpoint{4.275880in}{2.602731in}}%
\pgfpathlineto{\pgfqpoint{4.282647in}{2.353547in}}%
\pgfpathlineto{\pgfqpoint{4.289414in}{2.114109in}}%
\pgfpathlineto{\pgfqpoint{4.293925in}{2.027062in}}%
\pgfpathlineto{\pgfqpoint{4.296181in}{2.015284in}}%
\pgfpathlineto{\pgfqpoint{4.298436in}{2.026194in}}%
\pgfpathlineto{\pgfqpoint{4.300692in}{2.059159in}}%
\pgfpathlineto{\pgfqpoint{4.305204in}{2.181862in}}%
\pgfpathlineto{\pgfqpoint{4.318738in}{2.670483in}}%
\pgfpathlineto{\pgfqpoint{4.323249in}{2.737502in}}%
\pgfpathlineto{\pgfqpoint{4.325505in}{2.737652in}}%
\pgfpathlineto{\pgfqpoint{4.327760in}{2.715048in}}%
\pgfpathlineto{\pgfqpoint{4.332272in}{2.608560in}}%
\pgfpathlineto{\pgfqpoint{4.341294in}{2.263226in}}%
\pgfpathlineto{\pgfqpoint{4.345806in}{2.110031in}}%
\pgfpathlineto{\pgfqpoint{4.350317in}{2.022777in}}%
\pgfpathlineto{\pgfqpoint{4.352573in}{2.011441in}}%
\pgfpathlineto{\pgfqpoint{4.354828in}{2.023034in}}%
\pgfpathlineto{\pgfqpoint{4.357084in}{2.056806in}}%
\pgfpathlineto{\pgfqpoint{4.361595in}{2.181039in}}%
\pgfpathlineto{\pgfqpoint{4.375129in}{2.668970in}}%
\pgfpathlineto{\pgfqpoint{4.379641in}{2.734537in}}%
\pgfpathlineto{\pgfqpoint{4.381896in}{2.734136in}}%
\pgfpathlineto{\pgfqpoint{4.384152in}{2.711237in}}%
\pgfpathlineto{\pgfqpoint{4.388663in}{2.605245in}}%
\pgfpathlineto{\pgfqpoint{4.397686in}{2.265506in}}%
\pgfpathlineto{\pgfqpoint{4.402197in}{2.116191in}}%
\pgfpathlineto{\pgfqpoint{4.406709in}{2.031994in}}%
\pgfpathlineto{\pgfqpoint{4.408964in}{2.021487in}}%
\pgfpathlineto{\pgfqpoint{4.411220in}{2.033274in}}%
\pgfpathlineto{\pgfqpoint{4.413476in}{2.066531in}}%
\pgfpathlineto{\pgfqpoint{4.417987in}{2.187544in}}%
\pgfpathlineto{\pgfqpoint{4.431521in}{2.656670in}}%
\pgfpathlineto{\pgfqpoint{4.433777in}{2.697761in}}%
\pgfpathlineto{\pgfqpoint{4.436033in}{2.718458in}}%
\pgfpathlineto{\pgfqpoint{4.438288in}{2.717564in}}%
\pgfpathlineto{\pgfqpoint{4.440544in}{2.695247in}}%
\pgfpathlineto{\pgfqpoint{4.445055in}{2.593658in}}%
\pgfpathlineto{\pgfqpoint{4.460845in}{2.083079in}}%
\pgfpathlineto{\pgfqpoint{4.463101in}{2.053298in}}%
\pgfpathlineto{\pgfqpoint{4.465356in}{2.043947in}}%
\pgfpathlineto{\pgfqpoint{4.467612in}{2.055476in}}%
\pgfpathlineto{\pgfqpoint{4.469868in}{2.087019in}}%
\pgfpathlineto{\pgfqpoint{4.474379in}{2.200559in}}%
\pgfpathlineto{\pgfqpoint{4.487913in}{2.634866in}}%
\pgfpathlineto{\pgfqpoint{4.490169in}{2.672300in}}%
\pgfpathlineto{\pgfqpoint{4.492424in}{2.690864in}}%
\pgfpathlineto{\pgfqpoint{4.494680in}{2.689546in}}%
\pgfpathlineto{\pgfqpoint{4.496936in}{2.668595in}}%
\pgfpathlineto{\pgfqpoint{4.501447in}{2.574850in}}%
\pgfpathlineto{\pgfqpoint{4.517237in}{2.111426in}}%
\pgfpathlineto{\pgfqpoint{4.519492in}{2.085011in}}%
\pgfpathlineto{\pgfqpoint{4.521748in}{2.077079in}}%
\pgfpathlineto{\pgfqpoint{4.524004in}{2.087942in}}%
\pgfpathlineto{\pgfqpoint{4.526259in}{2.116730in}}%
\pgfpathlineto{\pgfqpoint{4.530771in}{2.219133in}}%
\pgfpathlineto{\pgfqpoint{4.544305in}{2.605042in}}%
\pgfpathlineto{\pgfqpoint{4.546561in}{2.637693in}}%
\pgfpathlineto{\pgfqpoint{4.548816in}{2.653587in}}%
\pgfpathlineto{\pgfqpoint{4.551072in}{2.651922in}}%
\pgfpathlineto{\pgfqpoint{4.553328in}{2.633013in}}%
\pgfpathlineto{\pgfqpoint{4.557839in}{2.550022in}}%
\pgfpathlineto{\pgfqpoint{4.573629in}{2.147560in}}%
\pgfpathlineto{\pgfqpoint{4.575884in}{2.125253in}}%
\pgfpathlineto{\pgfqpoint{4.578140in}{2.118936in}}%
\pgfpathlineto{\pgfqpoint{4.580396in}{2.128781in}}%
\pgfpathlineto{\pgfqpoint{4.582651in}{2.153943in}}%
\pgfpathlineto{\pgfqpoint{4.587163in}{2.242206in}}%
\pgfpathlineto{\pgfqpoint{4.600697in}{2.568823in}}%
\pgfpathlineto{\pgfqpoint{4.602952in}{2.595815in}}%
\pgfpathlineto{\pgfqpoint{4.605208in}{2.608634in}}%
\pgfpathlineto{\pgfqpoint{4.607464in}{2.606706in}}%
\pgfpathlineto{\pgfqpoint{4.609719in}{2.590395in}}%
\pgfpathlineto{\pgfqpoint{4.614231in}{2.520477in}}%
\pgfpathlineto{\pgfqpoint{4.630020in}{2.189660in}}%
\pgfpathlineto{\pgfqpoint{4.632276in}{2.172007in}}%
\pgfpathlineto{\pgfqpoint{4.634532in}{2.167430in}}%
\pgfpathlineto{\pgfqpoint{4.636787in}{2.175966in}}%
\pgfpathlineto{\pgfqpoint{4.639043in}{2.196823in}}%
\pgfpathlineto{\pgfqpoint{4.643555in}{2.268653in}}%
\pgfpathlineto{\pgfqpoint{4.657089in}{2.527931in}}%
\pgfpathlineto{\pgfqpoint{4.659344in}{2.548646in}}%
\pgfpathlineto{\pgfqpoint{4.661600in}{2.558121in}}%
\pgfpathlineto{\pgfqpoint{4.663856in}{2.556015in}}%
\pgfpathlineto{\pgfqpoint{4.666111in}{2.542726in}}%
\pgfpathlineto{\pgfqpoint{4.670623in}{2.487579in}}%
\pgfpathlineto{\pgfqpoint{4.686412in}{2.235835in}}%
\pgfpathlineto{\pgfqpoint{4.688668in}{2.223181in}}%
\pgfpathlineto{\pgfqpoint{4.690924in}{2.220401in}}%
\pgfpathlineto{\pgfqpoint{4.693179in}{2.227402in}}%
\pgfpathlineto{\pgfqpoint{4.697691in}{2.267351in}}%
\pgfpathlineto{\pgfqpoint{4.704458in}{2.367055in}}%
\pgfpathlineto{\pgfqpoint{4.711225in}{2.462595in}}%
\pgfpathlineto{\pgfqpoint{4.715736in}{2.498203in}}%
\pgfpathlineto{\pgfqpoint{4.717992in}{2.504201in}}%
\pgfpathlineto{\pgfqpoint{4.720247in}{2.502004in}}%
\pgfpathlineto{\pgfqpoint{4.722503in}{2.492026in}}%
\pgfpathlineto{\pgfqpoint{4.727014in}{2.452714in}}%
\pgfpathlineto{\pgfqpoint{4.740548in}{2.297998in}}%
\pgfpathlineto{\pgfqpoint{4.745060in}{2.276675in}}%
\pgfpathlineto{\pgfqpoint{4.747315in}{2.275682in}}%
\pgfpathlineto{\pgfqpoint{4.749571in}{2.280993in}}%
\pgfpathlineto{\pgfqpoint{4.754082in}{2.307747in}}%
\pgfpathlineto{\pgfqpoint{4.769872in}{2.439130in}}%
\pgfpathlineto{\pgfqpoint{4.772128in}{2.446485in}}%
\pgfpathlineto{\pgfqpoint{4.774384in}{2.449004in}}%
\pgfpathlineto{\pgfqpoint{4.776639in}{2.446799in}}%
\pgfpathlineto{\pgfqpoint{4.778895in}{2.440285in}}%
\pgfpathlineto{\pgfqpoint{4.783406in}{2.417243in}}%
\pgfpathlineto{\pgfqpoint{4.794685in}{2.347258in}}%
\pgfpathlineto{\pgfqpoint{4.799196in}{2.332858in}}%
\pgfpathlineto{\pgfqpoint{4.801452in}{2.330447in}}%
\pgfpathlineto{\pgfqpoint{4.803707in}{2.331167in}}%
\pgfpathlineto{\pgfqpoint{4.805963in}{2.334702in}}%
\pgfpathlineto{\pgfqpoint{4.810474in}{2.348112in}}%
\pgfpathlineto{\pgfqpoint{4.821753in}{2.387651in}}%
\pgfpathlineto{\pgfqpoint{4.826264in}{2.394633in}}%
\pgfpathlineto{\pgfqpoint{4.828520in}{2.395406in}}%
\pgfpathlineto{\pgfqpoint{4.830775in}{2.394569in}}%
\pgfpathlineto{\pgfqpoint{4.835287in}{2.389406in}}%
\pgfpathlineto{\pgfqpoint{4.844309in}{2.377101in}}%
\pgfpathlineto{\pgfqpoint{4.848821in}{2.375494in}}%
\pgfpathlineto{\pgfqpoint{4.853332in}{2.377883in}}%
\pgfpathlineto{\pgfqpoint{4.862355in}{2.386612in}}%
\pgfpathlineto{\pgfqpoint{4.864610in}{2.387423in}}%
\pgfpathlineto{\pgfqpoint{4.866866in}{2.387013in}}%
\pgfpathlineto{\pgfqpoint{4.869122in}{2.385203in}}%
\pgfpathlineto{\pgfqpoint{4.873633in}{2.377337in}}%
\pgfpathlineto{\pgfqpoint{4.887167in}{2.342786in}}%
\pgfpathlineto{\pgfqpoint{4.889423in}{2.340795in}}%
\pgfpathlineto{\pgfqpoint{4.891679in}{2.341147in}}%
\pgfpathlineto{\pgfqpoint{4.893934in}{2.344062in}}%
\pgfpathlineto{\pgfqpoint{4.898446in}{2.357503in}}%
\pgfpathlineto{\pgfqpoint{4.905213in}{2.391244in}}%
\pgfpathlineto{\pgfqpoint{4.911980in}{2.424327in}}%
\pgfpathlineto{\pgfqpoint{4.916491in}{2.434987in}}%
\pgfpathlineto{\pgfqpoint{4.918747in}{2.434980in}}%
\pgfpathlineto{\pgfqpoint{4.921002in}{2.431025in}}%
\pgfpathlineto{\pgfqpoint{4.925514in}{2.411666in}}%
\pgfpathlineto{\pgfqpoint{4.932281in}{2.362480in}}%
\pgfpathlineto{\pgfqpoint{4.939048in}{2.313168in}}%
\pgfpathlineto{\pgfqpoint{4.943559in}{2.295348in}}%
\pgfpathlineto{\pgfqpoint{4.945815in}{2.293563in}}%
\pgfpathlineto{\pgfqpoint{4.948070in}{2.297078in}}%
\pgfpathlineto{\pgfqpoint{4.950326in}{2.305887in}}%
\pgfpathlineto{\pgfqpoint{4.954837in}{2.337596in}}%
\pgfpathlineto{\pgfqpoint{4.968371in}{2.464120in}}%
\pgfpathlineto{\pgfqpoint{4.970627in}{2.475037in}}%
\pgfpathlineto{\pgfqpoint{4.972883in}{2.479930in}}%
\pgfpathlineto{\pgfqpoint{4.975138in}{2.478283in}}%
\pgfpathlineto{\pgfqpoint{4.977394in}{2.469993in}}%
\pgfpathlineto{\pgfqpoint{4.981905in}{2.435235in}}%
\pgfpathlineto{\pgfqpoint{4.997695in}{2.262830in}}%
\pgfpathlineto{\pgfqpoint{4.999951in}{2.253704in}}%
\pgfpathlineto{\pgfqpoint{5.002207in}{2.252176in}}%
\pgfpathlineto{\pgfqpoint{5.004462in}{2.258535in}}%
\pgfpathlineto{\pgfqpoint{5.008974in}{2.293543in}}%
\pgfpathlineto{\pgfqpoint{5.015741in}{2.384402in}}%
\pgfpathlineto{\pgfqpoint{5.022508in}{2.476707in}}%
\pgfpathlineto{\pgfqpoint{5.027019in}{2.512556in}}%
\pgfpathlineto{\pgfqpoint{5.029275in}{2.518385in}}%
\pgfpathlineto{\pgfqpoint{5.031530in}{2.515255in}}%
\pgfpathlineto{\pgfqpoint{5.033786in}{2.503194in}}%
\pgfpathlineto{\pgfqpoint{5.038297in}{2.455229in}}%
\pgfpathlineto{\pgfqpoint{5.054087in}{2.230049in}}%
\pgfpathlineto{\pgfqpoint{5.056343in}{2.219022in}}%
\pgfpathlineto{\pgfqpoint{5.058598in}{2.217793in}}%
\pgfpathlineto{\pgfqpoint{5.060854in}{2.226594in}}%
\pgfpathlineto{\pgfqpoint{5.065365in}{2.272033in}}%
\pgfpathlineto{\pgfqpoint{5.072132in}{2.386463in}}%
\pgfpathlineto{\pgfqpoint{5.078899in}{2.499951in}}%
\pgfpathlineto{\pgfqpoint{5.083411in}{2.542832in}}%
\pgfpathlineto{\pgfqpoint{5.085666in}{2.549326in}}%
\pgfpathlineto{\pgfqpoint{5.087922in}{2.544918in}}%
\pgfpathlineto{\pgfqpoint{5.090178in}{2.529751in}}%
\pgfpathlineto{\pgfqpoint{5.094689in}{2.471124in}}%
\pgfpathlineto{\pgfqpoint{5.110479in}{2.204580in}}%
\pgfpathlineto{\pgfqpoint{5.112735in}{2.192172in}}%
\pgfpathlineto{\pgfqpoint{5.114990in}{2.191269in}}%
\pgfpathlineto{\pgfqpoint{5.117246in}{2.202044in}}%
\pgfpathlineto{\pgfqpoint{5.121757in}{2.255625in}}%
\pgfpathlineto{\pgfqpoint{5.128524in}{2.388016in}}%
\pgfpathlineto{\pgfqpoint{5.135291in}{2.517252in}}%
\pgfpathlineto{\pgfqpoint{5.139803in}{2.565165in}}%
\pgfpathlineto{\pgfqpoint{5.142058in}{2.572046in}}%
\pgfpathlineto{\pgfqpoint{5.144314in}{2.566595in}}%
\pgfpathlineto{\pgfqpoint{5.146570in}{2.549066in}}%
\pgfpathlineto{\pgfqpoint{5.151081in}{2.482569in}}%
\pgfpathlineto{\pgfqpoint{5.166871in}{2.186952in}}%
\pgfpathlineto{\pgfqpoint{5.169126in}{2.173703in}}%
\pgfpathlineto{\pgfqpoint{5.171382in}{2.173143in}}%
\pgfpathlineto{\pgfqpoint{5.173638in}{2.185377in}}%
\pgfpathlineto{\pgfqpoint{5.178149in}{2.244640in}}%
\pgfpathlineto{\pgfqpoint{5.184916in}{2.389033in}}%
\pgfpathlineto{\pgfqpoint{5.191683in}{2.528298in}}%
\pgfpathlineto{\pgfqpoint{5.196194in}{2.579169in}}%
\pgfpathlineto{\pgfqpoint{5.198450in}{2.586157in}}%
\pgfpathlineto{\pgfqpoint{5.200706in}{2.579925in}}%
\pgfpathlineto{\pgfqpoint{5.202961in}{2.560821in}}%
\pgfpathlineto{\pgfqpoint{5.207473in}{2.489384in}}%
\pgfpathlineto{\pgfqpoint{5.223263in}{2.177391in}}%
\pgfpathlineto{\pgfqpoint{5.225518in}{2.163846in}}%
\pgfpathlineto{\pgfqpoint{5.227774in}{2.163631in}}%
\pgfpathlineto{\pgfqpoint{5.230030in}{2.176786in}}%
\pgfpathlineto{\pgfqpoint{5.234541in}{2.239193in}}%
\pgfpathlineto{\pgfqpoint{5.243564in}{2.442498in}}%
\pgfpathlineto{\pgfqpoint{5.248075in}{2.533014in}}%
\pgfpathlineto{\pgfqpoint{5.252586in}{2.584766in}}%
\pgfpathlineto{\pgfqpoint{5.254842in}{2.591591in}}%
\pgfpathlineto{\pgfqpoint{5.257098in}{2.584856in}}%
\pgfpathlineto{\pgfqpoint{5.259353in}{2.564981in}}%
\pgfpathlineto{\pgfqpoint{5.263865in}{2.491560in}}%
\pgfpathlineto{\pgfqpoint{5.277399in}{2.201755in}}%
\pgfpathlineto{\pgfqpoint{5.281910in}{2.162520in}}%
\pgfpathlineto{\pgfqpoint{5.284166in}{2.162640in}}%
\pgfpathlineto{\pgfqpoint{5.286421in}{2.176171in}}%
\pgfpathlineto{\pgfqpoint{5.290933in}{2.239200in}}%
\pgfpathlineto{\pgfqpoint{5.299955in}{2.442081in}}%
\pgfpathlineto{\pgfqpoint{5.304467in}{2.531547in}}%
\pgfpathlineto{\pgfqpoint{5.308978in}{2.582178in}}%
\pgfpathlineto{\pgfqpoint{5.311234in}{2.588590in}}%
\pgfpathlineto{\pgfqpoint{5.313489in}{2.581631in}}%
\pgfpathlineto{\pgfqpoint{5.315745in}{2.561776in}}%
\pgfpathlineto{\pgfqpoint{5.320256in}{2.489251in}}%
\pgfpathlineto{\pgfqpoint{5.333791in}{2.206805in}}%
\pgfpathlineto{\pgfqpoint{5.338302in}{2.169345in}}%
\pgfpathlineto{\pgfqpoint{5.340558in}{2.169780in}}%
\pgfpathlineto{\pgfqpoint{5.342813in}{2.183155in}}%
\pgfpathlineto{\pgfqpoint{5.347325in}{2.244388in}}%
\pgfpathlineto{\pgfqpoint{5.363114in}{2.553738in}}%
\pgfpathlineto{\pgfqpoint{5.365370in}{2.571905in}}%
\pgfpathlineto{\pgfqpoint{5.367626in}{2.577680in}}%
\pgfpathlineto{\pgfqpoint{5.369881in}{2.570771in}}%
\pgfpathlineto{\pgfqpoint{5.372137in}{2.551688in}}%
\pgfpathlineto{\pgfqpoint{5.376648in}{2.482763in}}%
\pgfpathlineto{\pgfqpoint{5.390182in}{2.218024in}}%
\pgfpathlineto{\pgfqpoint{5.392438in}{2.195096in}}%
\pgfpathlineto{\pgfqpoint{5.394694in}{2.183674in}}%
\pgfpathlineto{\pgfqpoint{5.396949in}{2.184391in}}%
\pgfpathlineto{\pgfqpoint{5.399205in}{2.197113in}}%
\pgfpathlineto{\pgfqpoint{5.403716in}{2.254317in}}%
\pgfpathlineto{\pgfqpoint{5.419506in}{2.538402in}}%
\pgfpathlineto{\pgfqpoint{5.421762in}{2.554693in}}%
\pgfpathlineto{\pgfqpoint{5.424017in}{2.559643in}}%
\pgfpathlineto{\pgfqpoint{5.426273in}{2.553044in}}%
\pgfpathlineto{\pgfqpoint{5.428529in}{2.535419in}}%
\pgfpathlineto{\pgfqpoint{5.433040in}{2.472535in}}%
\pgfpathlineto{\pgfqpoint{5.446574in}{2.234705in}}%
\pgfpathlineto{\pgfqpoint{5.448830in}{2.214498in}}%
\pgfpathlineto{\pgfqpoint{5.451086in}{2.204622in}}%
\pgfpathlineto{\pgfqpoint{5.453341in}{2.205582in}}%
\pgfpathlineto{\pgfqpoint{5.455597in}{2.217199in}}%
\pgfpathlineto{\pgfqpoint{5.460108in}{2.268394in}}%
\pgfpathlineto{\pgfqpoint{5.475898in}{2.517621in}}%
\pgfpathlineto{\pgfqpoint{5.478154in}{2.531505in}}%
\pgfpathlineto{\pgfqpoint{5.480409in}{2.535479in}}%
\pgfpathlineto{\pgfqpoint{5.482665in}{2.529425in}}%
\pgfpathlineto{\pgfqpoint{5.484921in}{2.513857in}}%
\pgfpathlineto{\pgfqpoint{5.489432in}{2.459118in}}%
\pgfpathlineto{\pgfqpoint{5.502966in}{2.255978in}}%
\pgfpathlineto{\pgfqpoint{5.505222in}{2.239134in}}%
\pgfpathlineto{\pgfqpoint{5.507477in}{2.231108in}}%
\pgfpathlineto{\pgfqpoint{5.509733in}{2.232266in}}%
\pgfpathlineto{\pgfqpoint{5.511989in}{2.242389in}}%
\pgfpathlineto{\pgfqpoint{5.516500in}{2.285910in}}%
\pgfpathlineto{\pgfqpoint{5.532290in}{2.492411in}}%
\pgfpathlineto{\pgfqpoint{5.534545in}{2.503470in}}%
\pgfpathlineto{\pgfqpoint{5.534545in}{2.503470in}}%
\pgfusepath{stroke}%
\end{pgfscope}%
\begin{pgfscope}%
\pgfpathrectangle{\pgfqpoint{0.800000in}{0.528000in}}{\pgfqpoint{4.960000in}{3.696000in}}%
\pgfusepath{clip}%
\pgfsetrectcap%
\pgfsetroundjoin%
\pgfsetlinewidth{1.505625pt}%
\definecolor{currentstroke}{rgb}{0.156863,0.411765,0.513725}%
\pgfsetstrokecolor{currentstroke}%
\pgfsetdash{}{0pt}%
\pgfpathmoveto{\pgfqpoint{1.025455in}{2.549326in}}%
\pgfpathlineto{\pgfqpoint{1.057034in}{2.532974in}}%
\pgfpathlineto{\pgfqpoint{1.088613in}{2.514149in}}%
\pgfpathlineto{\pgfqpoint{1.122448in}{2.491401in}}%
\pgfpathlineto{\pgfqpoint{1.158539in}{2.464435in}}%
\pgfpathlineto{\pgfqpoint{1.199141in}{2.431148in}}%
\pgfpathlineto{\pgfqpoint{1.244255in}{2.391108in}}%
\pgfpathlineto{\pgfqpoint{1.260045in}{2.376475in}}%
\pgfpathlineto{\pgfqpoint{1.264556in}{2.379755in}}%
\pgfpathlineto{\pgfqpoint{1.327715in}{2.440702in}}%
\pgfpathlineto{\pgfqpoint{1.460799in}{2.570096in}}%
\pgfpathlineto{\pgfqpoint{1.505913in}{2.610483in}}%
\pgfpathlineto{\pgfqpoint{1.544259in}{2.642035in}}%
\pgfpathlineto{\pgfqpoint{1.578095in}{2.667183in}}%
\pgfpathlineto{\pgfqpoint{1.609674in}{2.687959in}}%
\pgfpathlineto{\pgfqpoint{1.638998in}{2.704604in}}%
\pgfpathlineto{\pgfqpoint{1.666066in}{2.717464in}}%
\pgfpathlineto{\pgfqpoint{1.690878in}{2.726962in}}%
\pgfpathlineto{\pgfqpoint{1.715691in}{2.734117in}}%
\pgfpathlineto{\pgfqpoint{1.738247in}{2.738472in}}%
\pgfpathlineto{\pgfqpoint{1.760804in}{2.740681in}}%
\pgfpathlineto{\pgfqpoint{1.783361in}{2.740655in}}%
\pgfpathlineto{\pgfqpoint{1.805918in}{2.738316in}}%
\pgfpathlineto{\pgfqpoint{1.828474in}{2.733591in}}%
\pgfpathlineto{\pgfqpoint{1.851031in}{2.726420in}}%
\pgfpathlineto{\pgfqpoint{1.873588in}{2.716750in}}%
\pgfpathlineto{\pgfqpoint{1.896144in}{2.704539in}}%
\pgfpathlineto{\pgfqpoint{1.918701in}{2.689755in}}%
\pgfpathlineto{\pgfqpoint{1.941258in}{2.672379in}}%
\pgfpathlineto{\pgfqpoint{1.966070in}{2.650259in}}%
\pgfpathlineto{\pgfqpoint{1.990883in}{2.624992in}}%
\pgfpathlineto{\pgfqpoint{2.015695in}{2.596597in}}%
\pgfpathlineto{\pgfqpoint{2.042763in}{2.562096in}}%
\pgfpathlineto{\pgfqpoint{2.069831in}{2.523986in}}%
\pgfpathlineto{\pgfqpoint{2.099155in}{2.478734in}}%
\pgfpathlineto{\pgfqpoint{2.130734in}{2.425557in}}%
\pgfpathlineto{\pgfqpoint{2.157803in}{2.376475in}}%
\pgfpathlineto{\pgfqpoint{2.162314in}{2.384007in}}%
\pgfpathlineto{\pgfqpoint{2.196149in}{2.450243in}}%
\pgfpathlineto{\pgfqpoint{2.232240in}{2.525677in}}%
\pgfpathlineto{\pgfqpoint{2.272842in}{2.615796in}}%
\pgfpathlineto{\pgfqpoint{2.317955in}{2.721473in}}%
\pgfpathlineto{\pgfqpoint{2.372092in}{2.854254in}}%
\pgfpathlineto{\pgfqpoint{2.448784in}{3.049030in}}%
\pgfpathlineto{\pgfqpoint{2.563824in}{3.341002in}}%
\pgfpathlineto{\pgfqpoint{2.617960in}{3.471940in}}%
\pgfpathlineto{\pgfqpoint{2.663073in}{3.575202in}}%
\pgfpathlineto{\pgfqpoint{2.701420in}{3.657609in}}%
\pgfpathlineto{\pgfqpoint{2.737511in}{3.729845in}}%
\pgfpathlineto{\pgfqpoint{2.769090in}{3.788277in}}%
\pgfpathlineto{\pgfqpoint{2.798414in}{3.838160in}}%
\pgfpathlineto{\pgfqpoint{2.825482in}{3.880190in}}%
\pgfpathlineto{\pgfqpoint{2.852550in}{3.918136in}}%
\pgfpathlineto{\pgfqpoint{2.877362in}{3.949159in}}%
\pgfpathlineto{\pgfqpoint{2.899919in}{3.974118in}}%
\pgfpathlineto{\pgfqpoint{2.922476in}{3.995893in}}%
\pgfpathlineto{\pgfqpoint{2.945033in}{4.014401in}}%
\pgfpathlineto{\pgfqpoint{2.965334in}{4.028210in}}%
\pgfpathlineto{\pgfqpoint{2.985635in}{4.039275in}}%
\pgfpathlineto{\pgfqpoint{3.003680in}{4.046781in}}%
\pgfpathlineto{\pgfqpoint{3.021725in}{4.052075in}}%
\pgfpathlineto{\pgfqpoint{3.039771in}{4.055145in}}%
\pgfpathlineto{\pgfqpoint{3.057816in}{4.055983in}}%
\pgfpathlineto{\pgfqpoint{3.075862in}{4.054586in}}%
\pgfpathlineto{\pgfqpoint{3.093907in}{4.050960in}}%
\pgfpathlineto{\pgfqpoint{3.111952in}{4.045111in}}%
\pgfpathlineto{\pgfqpoint{3.129998in}{4.037055in}}%
\pgfpathlineto{\pgfqpoint{3.150299in}{4.025377in}}%
\pgfpathlineto{\pgfqpoint{3.170600in}{4.010964in}}%
\pgfpathlineto{\pgfqpoint{3.190901in}{3.993861in}}%
\pgfpathlineto{\pgfqpoint{3.213458in}{3.971764in}}%
\pgfpathlineto{\pgfqpoint{3.236014in}{3.946491in}}%
\pgfpathlineto{\pgfqpoint{3.260827in}{3.915135in}}%
\pgfpathlineto{\pgfqpoint{3.285639in}{3.880190in}}%
\pgfpathlineto{\pgfqpoint{3.312707in}{3.838160in}}%
\pgfpathlineto{\pgfqpoint{3.342031in}{3.788277in}}%
\pgfpathlineto{\pgfqpoint{3.373610in}{3.729845in}}%
\pgfpathlineto{\pgfqpoint{3.407446in}{3.662282in}}%
\pgfpathlineto{\pgfqpoint{3.443536in}{3.585173in}}%
\pgfpathlineto{\pgfqpoint{3.484138in}{3.493075in}}%
\pgfpathlineto{\pgfqpoint{3.531508in}{3.379844in}}%
\pgfpathlineto{\pgfqpoint{3.590155in}{3.233557in}}%
\pgfpathlineto{\pgfqpoint{3.795421in}{2.716068in}}%
\pgfpathlineto{\pgfqpoint{3.842790in}{2.605531in}}%
\pgfpathlineto{\pgfqpoint{3.885648in}{2.511179in}}%
\pgfpathlineto{\pgfqpoint{3.923995in}{2.432139in}}%
\pgfpathlineto{\pgfqpoint{3.951063in}{2.379755in}}%
\pgfpathlineto{\pgfqpoint{3.953318in}{2.376475in}}%
\pgfpathlineto{\pgfqpoint{3.987154in}{2.437329in}}%
\pgfpathlineto{\pgfqpoint{4.018733in}{2.489536in}}%
\pgfpathlineto{\pgfqpoint{4.048057in}{2.533847in}}%
\pgfpathlineto{\pgfqpoint{4.077381in}{2.574002in}}%
\pgfpathlineto{\pgfqpoint{4.104449in}{2.607282in}}%
\pgfpathlineto{\pgfqpoint{4.129261in}{2.634543in}}%
\pgfpathlineto{\pgfqpoint{4.154073in}{2.658667in}}%
\pgfpathlineto{\pgfqpoint{4.178886in}{2.679642in}}%
\pgfpathlineto{\pgfqpoint{4.201443in}{2.695979in}}%
\pgfpathlineto{\pgfqpoint{4.223999in}{2.709730in}}%
\pgfpathlineto{\pgfqpoint{4.246556in}{2.720921in}}%
\pgfpathlineto{\pgfqpoint{4.269113in}{2.729586in}}%
\pgfpathlineto{\pgfqpoint{4.291669in}{2.735771in}}%
\pgfpathlineto{\pgfqpoint{4.314226in}{2.739534in}}%
\pgfpathlineto{\pgfqpoint{4.336783in}{2.740939in}}%
\pgfpathlineto{\pgfqpoint{4.359340in}{2.740060in}}%
\pgfpathlineto{\pgfqpoint{4.381896in}{2.736982in}}%
\pgfpathlineto{\pgfqpoint{4.404453in}{2.731795in}}%
\pgfpathlineto{\pgfqpoint{4.429266in}{2.723772in}}%
\pgfpathlineto{\pgfqpoint{4.454078in}{2.713458in}}%
\pgfpathlineto{\pgfqpoint{4.481146in}{2.699772in}}%
\pgfpathlineto{\pgfqpoint{4.510470in}{2.682311in}}%
\pgfpathlineto{\pgfqpoint{4.542049in}{2.660751in}}%
\pgfpathlineto{\pgfqpoint{4.575884in}{2.634880in}}%
\pgfpathlineto{\pgfqpoint{4.614231in}{2.602657in}}%
\pgfpathlineto{\pgfqpoint{4.659344in}{2.561685in}}%
\pgfpathlineto{\pgfqpoint{4.720247in}{2.503006in}}%
\pgfpathlineto{\pgfqpoint{4.848821in}{2.377637in}}%
\pgfpathlineto{\pgfqpoint{4.851076in}{2.376475in}}%
\pgfpathlineto{\pgfqpoint{4.900701in}{2.421412in}}%
\pgfpathlineto{\pgfqpoint{4.943559in}{2.457293in}}%
\pgfpathlineto{\pgfqpoint{4.981905in}{2.486549in}}%
\pgfpathlineto{\pgfqpoint{5.017996in}{2.511266in}}%
\pgfpathlineto{\pgfqpoint{5.051831in}{2.531710in}}%
\pgfpathlineto{\pgfqpoint{5.083411in}{2.548243in}}%
\pgfpathlineto{\pgfqpoint{5.114990in}{2.562199in}}%
\pgfpathlineto{\pgfqpoint{5.144314in}{2.572777in}}%
\pgfpathlineto{\pgfqpoint{5.173638in}{2.581020in}}%
\pgfpathlineto{\pgfqpoint{5.202961in}{2.586909in}}%
\pgfpathlineto{\pgfqpoint{5.232285in}{2.590449in}}%
\pgfpathlineto{\pgfqpoint{5.261609in}{2.591668in}}%
\pgfpathlineto{\pgfqpoint{5.290933in}{2.590614in}}%
\pgfpathlineto{\pgfqpoint{5.320256in}{2.587358in}}%
\pgfpathlineto{\pgfqpoint{5.351836in}{2.581491in}}%
\pgfpathlineto{\pgfqpoint{5.383415in}{2.573309in}}%
\pgfpathlineto{\pgfqpoint{5.417250in}{2.562153in}}%
\pgfpathlineto{\pgfqpoint{5.453341in}{2.547772in}}%
\pgfpathlineto{\pgfqpoint{5.491688in}{2.530012in}}%
\pgfpathlineto{\pgfqpoint{5.534545in}{2.507605in}}%
\pgfpathlineto{\pgfqpoint{5.534545in}{2.507605in}}%
\pgfusepath{stroke}%
\end{pgfscope}%
\begin{pgfscope}%
\pgfpathrectangle{\pgfqpoint{0.800000in}{0.528000in}}{\pgfqpoint{4.960000in}{3.696000in}}%
\pgfusepath{clip}%
\pgfsetrectcap%
\pgfsetroundjoin%
\pgfsetlinewidth{1.505625pt}%
\definecolor{currentstroke}{rgb}{0.156863,0.411765,0.513725}%
\pgfsetstrokecolor{currentstroke}%
\pgfsetdash{}{0pt}%
\pgfpathmoveto{\pgfqpoint{1.025455in}{2.202674in}}%
\pgfpathlineto{\pgfqpoint{1.057034in}{2.219026in}}%
\pgfpathlineto{\pgfqpoint{1.088613in}{2.237851in}}%
\pgfpathlineto{\pgfqpoint{1.122448in}{2.260599in}}%
\pgfpathlineto{\pgfqpoint{1.158539in}{2.287565in}}%
\pgfpathlineto{\pgfqpoint{1.199141in}{2.320852in}}%
\pgfpathlineto{\pgfqpoint{1.244255in}{2.360892in}}%
\pgfpathlineto{\pgfqpoint{1.260045in}{2.375525in}}%
\pgfpathlineto{\pgfqpoint{1.264556in}{2.372245in}}%
\pgfpathlineto{\pgfqpoint{1.327715in}{2.311298in}}%
\pgfpathlineto{\pgfqpoint{1.460799in}{2.181904in}}%
\pgfpathlineto{\pgfqpoint{1.505913in}{2.141517in}}%
\pgfpathlineto{\pgfqpoint{1.544259in}{2.109965in}}%
\pgfpathlineto{\pgfqpoint{1.578095in}{2.084817in}}%
\pgfpathlineto{\pgfqpoint{1.609674in}{2.064041in}}%
\pgfpathlineto{\pgfqpoint{1.638998in}{2.047396in}}%
\pgfpathlineto{\pgfqpoint{1.666066in}{2.034536in}}%
\pgfpathlineto{\pgfqpoint{1.690878in}{2.025038in}}%
\pgfpathlineto{\pgfqpoint{1.715691in}{2.017883in}}%
\pgfpathlineto{\pgfqpoint{1.738247in}{2.013528in}}%
\pgfpathlineto{\pgfqpoint{1.760804in}{2.011319in}}%
\pgfpathlineto{\pgfqpoint{1.783361in}{2.011345in}}%
\pgfpathlineto{\pgfqpoint{1.805918in}{2.013684in}}%
\pgfpathlineto{\pgfqpoint{1.828474in}{2.018409in}}%
\pgfpathlineto{\pgfqpoint{1.851031in}{2.025580in}}%
\pgfpathlineto{\pgfqpoint{1.873588in}{2.035250in}}%
\pgfpathlineto{\pgfqpoint{1.896144in}{2.047461in}}%
\pgfpathlineto{\pgfqpoint{1.918701in}{2.062245in}}%
\pgfpathlineto{\pgfqpoint{1.941258in}{2.079621in}}%
\pgfpathlineto{\pgfqpoint{1.966070in}{2.101741in}}%
\pgfpathlineto{\pgfqpoint{1.990883in}{2.127008in}}%
\pgfpathlineto{\pgfqpoint{2.015695in}{2.155403in}}%
\pgfpathlineto{\pgfqpoint{2.042763in}{2.189904in}}%
\pgfpathlineto{\pgfqpoint{2.069831in}{2.228014in}}%
\pgfpathlineto{\pgfqpoint{2.099155in}{2.273266in}}%
\pgfpathlineto{\pgfqpoint{2.130734in}{2.326443in}}%
\pgfpathlineto{\pgfqpoint{2.157803in}{2.375525in}}%
\pgfpathlineto{\pgfqpoint{2.162314in}{2.367993in}}%
\pgfpathlineto{\pgfqpoint{2.196149in}{2.301757in}}%
\pgfpathlineto{\pgfqpoint{2.232240in}{2.226323in}}%
\pgfpathlineto{\pgfqpoint{2.272842in}{2.136204in}}%
\pgfpathlineto{\pgfqpoint{2.317955in}{2.030527in}}%
\pgfpathlineto{\pgfqpoint{2.372092in}{1.897746in}}%
\pgfpathlineto{\pgfqpoint{2.448784in}{1.702970in}}%
\pgfpathlineto{\pgfqpoint{2.563824in}{1.410998in}}%
\pgfpathlineto{\pgfqpoint{2.617960in}{1.280060in}}%
\pgfpathlineto{\pgfqpoint{2.663073in}{1.176798in}}%
\pgfpathlineto{\pgfqpoint{2.701420in}{1.094391in}}%
\pgfpathlineto{\pgfqpoint{2.737511in}{1.022155in}}%
\pgfpathlineto{\pgfqpoint{2.769090in}{0.963723in}}%
\pgfpathlineto{\pgfqpoint{2.798414in}{0.913840in}}%
\pgfpathlineto{\pgfqpoint{2.825482in}{0.871810in}}%
\pgfpathlineto{\pgfqpoint{2.852550in}{0.833864in}}%
\pgfpathlineto{\pgfqpoint{2.877362in}{0.802841in}}%
\pgfpathlineto{\pgfqpoint{2.899919in}{0.777882in}}%
\pgfpathlineto{\pgfqpoint{2.922476in}{0.756107in}}%
\pgfpathlineto{\pgfqpoint{2.945033in}{0.737599in}}%
\pgfpathlineto{\pgfqpoint{2.965334in}{0.723790in}}%
\pgfpathlineto{\pgfqpoint{2.985635in}{0.712725in}}%
\pgfpathlineto{\pgfqpoint{3.003680in}{0.705219in}}%
\pgfpathlineto{\pgfqpoint{3.021725in}{0.699925in}}%
\pgfpathlineto{\pgfqpoint{3.039771in}{0.696855in}}%
\pgfpathlineto{\pgfqpoint{3.057816in}{0.696017in}}%
\pgfpathlineto{\pgfqpoint{3.075862in}{0.697414in}}%
\pgfpathlineto{\pgfqpoint{3.093907in}{0.701040in}}%
\pgfpathlineto{\pgfqpoint{3.111952in}{0.706889in}}%
\pgfpathlineto{\pgfqpoint{3.129998in}{0.714945in}}%
\pgfpathlineto{\pgfqpoint{3.150299in}{0.726623in}}%
\pgfpathlineto{\pgfqpoint{3.170600in}{0.741036in}}%
\pgfpathlineto{\pgfqpoint{3.190901in}{0.758139in}}%
\pgfpathlineto{\pgfqpoint{3.213458in}{0.780236in}}%
\pgfpathlineto{\pgfqpoint{3.236014in}{0.805509in}}%
\pgfpathlineto{\pgfqpoint{3.260827in}{0.836865in}}%
\pgfpathlineto{\pgfqpoint{3.285639in}{0.871810in}}%
\pgfpathlineto{\pgfqpoint{3.312707in}{0.913840in}}%
\pgfpathlineto{\pgfqpoint{3.342031in}{0.963723in}}%
\pgfpathlineto{\pgfqpoint{3.373610in}{1.022155in}}%
\pgfpathlineto{\pgfqpoint{3.407446in}{1.089718in}}%
\pgfpathlineto{\pgfqpoint{3.443536in}{1.166827in}}%
\pgfpathlineto{\pgfqpoint{3.484138in}{1.258925in}}%
\pgfpathlineto{\pgfqpoint{3.531508in}{1.372156in}}%
\pgfpathlineto{\pgfqpoint{3.590155in}{1.518443in}}%
\pgfpathlineto{\pgfqpoint{3.795421in}{2.035932in}}%
\pgfpathlineto{\pgfqpoint{3.842790in}{2.146469in}}%
\pgfpathlineto{\pgfqpoint{3.885648in}{2.240821in}}%
\pgfpathlineto{\pgfqpoint{3.923995in}{2.319861in}}%
\pgfpathlineto{\pgfqpoint{3.951063in}{2.372245in}}%
\pgfpathlineto{\pgfqpoint{3.953318in}{2.375525in}}%
\pgfpathlineto{\pgfqpoint{3.987154in}{2.314671in}}%
\pgfpathlineto{\pgfqpoint{4.018733in}{2.262464in}}%
\pgfpathlineto{\pgfqpoint{4.048057in}{2.218153in}}%
\pgfpathlineto{\pgfqpoint{4.077381in}{2.177998in}}%
\pgfpathlineto{\pgfqpoint{4.104449in}{2.144718in}}%
\pgfpathlineto{\pgfqpoint{4.129261in}{2.117457in}}%
\pgfpathlineto{\pgfqpoint{4.154073in}{2.093333in}}%
\pgfpathlineto{\pgfqpoint{4.178886in}{2.072358in}}%
\pgfpathlineto{\pgfqpoint{4.201443in}{2.056021in}}%
\pgfpathlineto{\pgfqpoint{4.223999in}{2.042270in}}%
\pgfpathlineto{\pgfqpoint{4.246556in}{2.031079in}}%
\pgfpathlineto{\pgfqpoint{4.269113in}{2.022414in}}%
\pgfpathlineto{\pgfqpoint{4.291669in}{2.016229in}}%
\pgfpathlineto{\pgfqpoint{4.314226in}{2.012466in}}%
\pgfpathlineto{\pgfqpoint{4.336783in}{2.011061in}}%
\pgfpathlineto{\pgfqpoint{4.359340in}{2.011940in}}%
\pgfpathlineto{\pgfqpoint{4.381896in}{2.015018in}}%
\pgfpathlineto{\pgfqpoint{4.404453in}{2.020205in}}%
\pgfpathlineto{\pgfqpoint{4.429266in}{2.028228in}}%
\pgfpathlineto{\pgfqpoint{4.454078in}{2.038542in}}%
\pgfpathlineto{\pgfqpoint{4.481146in}{2.052228in}}%
\pgfpathlineto{\pgfqpoint{4.510470in}{2.069689in}}%
\pgfpathlineto{\pgfqpoint{4.542049in}{2.091249in}}%
\pgfpathlineto{\pgfqpoint{4.575884in}{2.117120in}}%
\pgfpathlineto{\pgfqpoint{4.614231in}{2.149343in}}%
\pgfpathlineto{\pgfqpoint{4.659344in}{2.190315in}}%
\pgfpathlineto{\pgfqpoint{4.720247in}{2.248994in}}%
\pgfpathlineto{\pgfqpoint{4.848821in}{2.374363in}}%
\pgfpathlineto{\pgfqpoint{4.851076in}{2.375525in}}%
\pgfpathlineto{\pgfqpoint{4.900701in}{2.330588in}}%
\pgfpathlineto{\pgfqpoint{4.943559in}{2.294707in}}%
\pgfpathlineto{\pgfqpoint{4.981905in}{2.265451in}}%
\pgfpathlineto{\pgfqpoint{5.017996in}{2.240734in}}%
\pgfpathlineto{\pgfqpoint{5.051831in}{2.220290in}}%
\pgfpathlineto{\pgfqpoint{5.083411in}{2.203757in}}%
\pgfpathlineto{\pgfqpoint{5.114990in}{2.189801in}}%
\pgfpathlineto{\pgfqpoint{5.144314in}{2.179223in}}%
\pgfpathlineto{\pgfqpoint{5.173638in}{2.170980in}}%
\pgfpathlineto{\pgfqpoint{5.202961in}{2.165091in}}%
\pgfpathlineto{\pgfqpoint{5.232285in}{2.161551in}}%
\pgfpathlineto{\pgfqpoint{5.261609in}{2.160332in}}%
\pgfpathlineto{\pgfqpoint{5.290933in}{2.161386in}}%
\pgfpathlineto{\pgfqpoint{5.320256in}{2.164642in}}%
\pgfpathlineto{\pgfqpoint{5.351836in}{2.170509in}}%
\pgfpathlineto{\pgfqpoint{5.383415in}{2.178691in}}%
\pgfpathlineto{\pgfqpoint{5.417250in}{2.189847in}}%
\pgfpathlineto{\pgfqpoint{5.453341in}{2.204228in}}%
\pgfpathlineto{\pgfqpoint{5.491688in}{2.221988in}}%
\pgfpathlineto{\pgfqpoint{5.534545in}{2.244395in}}%
\pgfpathlineto{\pgfqpoint{5.534545in}{2.244395in}}%
\pgfusepath{stroke}%
\end{pgfscope}%
\begin{pgfscope}%
\pgfsetbuttcap%
\pgfsetmiterjoin%
\definecolor{currentfill}{rgb}{0.000000,0.000000,0.000000}%
\pgfsetfillcolor{currentfill}%
\pgfsetlinewidth{1.003750pt}%
\definecolor{currentstroke}{rgb}{0.000000,0.000000,0.000000}%
\pgfsetstrokecolor{currentstroke}%
\pgfsetdash{}{0pt}%
\pgfsys@defobject{currentmarker}{\pgfqpoint{-0.069444in}{-0.069444in}}{\pgfqpoint{0.069444in}{0.069444in}}{%
\pgfpathmoveto{\pgfqpoint{0.069444in}{-0.000000in}}%
\pgfpathlineto{\pgfqpoint{-0.069444in}{0.069444in}}%
\pgfpathlineto{\pgfqpoint{-0.069444in}{-0.069444in}}%
\pgfpathlineto{\pgfqpoint{0.069444in}{-0.000000in}}%
\pgfpathclose%
\pgfusepath{stroke,fill}%
}%
\begin{pgfscope}%
\pgfsys@transformshift{5.760000in}{1.032000in}%
\pgfsys@useobject{currentmarker}{}%
\end{pgfscope}%
\end{pgfscope}%
\begin{pgfscope}%
\pgfsetbuttcap%
\pgfsetmiterjoin%
\definecolor{currentfill}{rgb}{0.000000,0.000000,0.000000}%
\pgfsetfillcolor{currentfill}%
\pgfsetlinewidth{1.003750pt}%
\definecolor{currentstroke}{rgb}{0.000000,0.000000,0.000000}%
\pgfsetstrokecolor{currentstroke}%
\pgfsetdash{}{0pt}%
\pgfsys@defobject{currentmarker}{\pgfqpoint{-0.069444in}{-0.069444in}}{\pgfqpoint{0.069444in}{0.069444in}}{%
\pgfpathmoveto{\pgfqpoint{0.000000in}{0.069444in}}%
\pgfpathlineto{\pgfqpoint{-0.069444in}{-0.069444in}}%
\pgfpathlineto{\pgfqpoint{0.069444in}{-0.069444in}}%
\pgfpathlineto{\pgfqpoint{0.000000in}{0.069444in}}%
\pgfpathclose%
\pgfusepath{stroke,fill}%
}%
\begin{pgfscope}%
\pgfsys@transformshift{1.025455in}{4.224000in}%
\pgfsys@useobject{currentmarker}{}%
\end{pgfscope}%
\end{pgfscope}%
\begin{pgfscope}%
\pgfsetrectcap%
\pgfsetmiterjoin%
\pgfsetlinewidth{0.803000pt}%
\definecolor{currentstroke}{rgb}{0.000000,0.000000,0.000000}%
\pgfsetstrokecolor{currentstroke}%
\pgfsetdash{}{0pt}%
\pgfpathmoveto{\pgfqpoint{1.025455in}{0.528000in}}%
\pgfpathlineto{\pgfqpoint{1.025455in}{4.224000in}}%
\pgfusepath{stroke}%
\end{pgfscope}%
\begin{pgfscope}%
\pgfsetrectcap%
\pgfsetmiterjoin%
\pgfsetlinewidth{0.803000pt}%
\definecolor{currentstroke}{rgb}{0.000000,0.000000,0.000000}%
\pgfsetstrokecolor{currentstroke}%
\pgfsetdash{}{0pt}%
\pgfpathmoveto{\pgfqpoint{0.800000in}{1.032000in}}%
\pgfpathlineto{\pgfqpoint{5.760000in}{1.032000in}}%
\pgfusepath{stroke}%
\end{pgfscope}%
\end{pgfpicture}%
\makeatother%
\endgroup%
}
		\caption{Éclairement $I(t)$ pour un filtre à spectre rectangulaire}
	\end{figure}

	Dans l'hypothèse où $\mathrm{\Delta}\sigma \ll \sigma_0$, la courbe $I(e)$ est bornée par $I_0 K(1 \pm \operatorname{sinc}(2\pi\,e\,\mathrm{\Delta}\sigma) / \mathrm{\Delta}\sigma)$, comme le montre la figure ci-dessus.
	En analysant les oscillations de cette enveloppe, et notamment la fréquence, on peut en déduire la valeur de $\mathrm{\Delta}\sigma$ : \[
		\mathrm{\Delta}\sigma = \frac{f}{v}
	,\] où $f$ est la fréquence mesurée de l'enveloppe. En effet, en ré-écrivant les équations des enveloppes en fonction du temps $t$, on trouve $I_0K (1 \pm \operatorname{sinc}(2\pi\,v\,(t-t_0))$, où $t_0$ est l'instant du contact optique.

	\subsection{Filtre à profil spectral gaussien}

	\begin{figure}[H]
		\centering
		\begin{tikzpicture}[scale=0.5]
			\draw[->] (-0.5, 0) -- (9, 0);
			\draw[->] (0, -0.5) -- (0, 6);
			\draw[red, thick] plot [smooth] coordinates {(-1.2, 0.1) (-1.1, 0.12) (-1.0, 0.14) (-0.9, 0.16) (-0.8, 0.18) (-0.7, 0.2) (-0.6, 0.23) (-0.5, 0.26) (-0.4, 0.3) (-0.3, 0.34) (-0.2, 0.38) (-0.1, 0.43) (0.0, 0.48) (0.1, 0.53) (0.2, 0.59) (0.3, 0.66) (0.4, 0.73) (0.5, 0.81) (0.6, 0.9) (0.7, 0.99) (0.8, 1.08) (0.9, 1.19) (1.0, 1.3) (1.1, 1.41) (1.2, 1.54) (1.3, 1.67) (1.4, 1.8) (1.5, 1.95) (1.6, 2.1) (1.7, 2.25) (1.8, 2.41) (1.9, 2.58) (2.0, 2.75) (2.1, 2.92) (2.2, 3.1) (2.3, 3.28) (2.4, 3.46) (2.5, 3.64) (2.6, 3.82) (2.7, 4.0) (2.8, 4.18) (2.9, 4.36) (3.0, 4.53) (3.1, 4.7) (3.2, 4.86) (3.3, 5.01) (3.4, 5.16) (3.5, 5.29) (3.6, 5.42) (3.7, 5.54) (3.8, 5.64) (3.9, 5.74) (4.0, 5.82) (4.1, 5.88) (4.2, 5.93) (4.3, 5.97) (4.4, 5.99) (4.5, 6.0) (4.6, 5.99) (4.7, 5.97) (4.8, 5.93) (4.9, 5.88) (5.0, 5.82) (5.1, 5.74) (5.2, 5.64) (5.3, 5.54) (5.4, 5.42) (5.5, 5.29) (5.6, 5.16) (5.7, 5.01) (5.8, 4.86) (5.9, 4.7) (6.0, 4.53) (6.1, 4.36) (6.2, 4.18) (6.3, 4.0) (6.4, 3.82) (6.5, 3.64) (6.6, 3.46) (6.7, 3.28) (6.8, 3.1) (6.9, 2.92) (7.0, 2.75) (7.1, 2.58) (7.2, 2.41) (7.3, 2.25) (7.4, 2.1) (7.5, 1.95) (7.6, 1.8) (7.7, 1.67) (7.8, 1.54) (7.9, 1.41) (8.0, 1.3) (8.1, 1.19) (8.2, 1.08) (8.3, 0.99) (8.4, 0.9) (8.5, 0.81) (8.6, 0.73) (8.7, 0.66) (8.8, 0.59) (8.9, 0.53) (9.0, 0.48) (9.1, 0.43) (9.2, 0.38) (9.3, 0.34) (9.4, 0.3) (9.5, 0.26) (9.6, 0.23) (9.7, 0.2) (9.8, 0.18) (9.9, 0.16)};
			\node[red] at (7.5, 6){$F(\sigma)$};
			\draw[<->, blue] (2.7, 3) -- (6.3, 3);
			\node[blue] at (4.5, 4.5){$\mathrm{\Delta}\sigma$};
			\node at (9.5,0){$\sigma$};
			\draw[mauve] (4.5, -0.2)--(4.5, 0.2);
			\node[mauve] at (4.5, -0.5){$\sigma_0$};
		\end{tikzpicture}
		\caption{Fonction $F(\sigma)$ pour le profil spectral gaussien}
		\label{fig:gauss}
	\end{figure}

	Avec un profil spectral gaussien, la fonction $F(\sigma)$ est de la forme d'une gaussienne représentée sur la figure ci-avant : \[
		F(\sigma) = \frac{K}{\mathrm{\Delta}\sigma} \cdot \mathrm{e}^{\ds-\left( \frac{\sigma-\sigma_0}{a} \right)^2}
	.\]
	Cette fonction atteint son maximum pour $\sigma = \sigma_0$ avec la valeur $K / \mathrm{\Delta}\sigma$.
	La largeur de la bande passante $\mathrm{\Delta}\sigma$ correspond à la largeur à mi-hauteur de la fonction $F(\sigma)$. Déterminons cette valeur : à mi-hauteur, $\exp(-(\sigma - \sigma_0)^2 / a^2) = 1 / 2$, d'où, par passage au logarithme, \[
		\left( \frac{\sigma - \sigma_0}{a} \right)^2 = \ln 2
	.\]
	La valeur $\sigma - \sigma_0$ correspond à la moitié de la bande passante $\mathrm{\Delta}\sigma / 2$, comme le montre la figure~\ref{fig:gauss}.
	On en déduit donc \[
		\mathrm{\Delta}\sigma = 2a \sqrt{\ln 2}
	.\] 

	Une analyse de Fourier permet de déterminer une expression de $I(e)$ : \[
		I(e) = \frac{I_0K\sqrt{\pi}}{2\sqrt{\ln 2}} \big(1 + \cos(4\pi\,\sigma_0\,e) \cdot \mathrm{e}^{-(2\pi\,a\,e)^2}\big)
	.\]
	Dans la suite, on notera $I_{\max}$ la valeur de~$I_0K \sqrt{\pi} / \sqrt{\ln 2}$, correspondant à l'intensité maximale (atteinte pour $e = 0$).
	On représente la fonction $I(e)$ sur la figure ci-après.
	Cette fonction est bornée par deux exponentielles décroissantes de la forme $I_{\max} (1 \pm \mathrm{e}^{-(2\pi\,a\,e)^2})$.

	\begin{figure}[H]
		\centering
		\resizebox{\linewidth}{!}{%% Creator: Matplotlib, PGF backend
%%
%% To include the figure in your LaTeX document, write
%%   \input{<filename>.pgf}
%%
%% Make sure the required packages are loaded in your preamble
%%   \usepackage{pgf}
%%
%% Also ensure that all the required font packages are loaded; for instance,
%% the lmodern package is sometimes necessary when using math font.
%%   \usepackage{lmodern}
%%
%% Figures using additional raster images can only be included by \input if
%% they are in the same directory as the main LaTeX file. For loading figures
%% from other directories you can use the `import` package
%%   \usepackage{import}
%%
%% and then include the figures with
%%   \import{<path to file>}{<filename>.pgf}
%%
%% Matplotlib used the following preamble
%%   
%%   \makeatletter\@ifpackageloaded{underscore}{}{\usepackage[strings]{underscore}}\makeatother
%%
\begingroup%
\makeatletter%
\begin{pgfpicture}%
\pgfpathrectangle{\pgfpointorigin}{\pgfqpoint{6.400000in}{4.800000in}}%
\pgfusepath{use as bounding box, clip}%
\begin{pgfscope}%
\pgfsetbuttcap%
\pgfsetmiterjoin%
\definecolor{currentfill}{rgb}{1.000000,1.000000,1.000000}%
\pgfsetfillcolor{currentfill}%
\pgfsetlinewidth{0.000000pt}%
\definecolor{currentstroke}{rgb}{1.000000,1.000000,1.000000}%
\pgfsetstrokecolor{currentstroke}%
\pgfsetdash{}{0pt}%
\pgfpathmoveto{\pgfqpoint{0.000000in}{0.000000in}}%
\pgfpathlineto{\pgfqpoint{6.400000in}{0.000000in}}%
\pgfpathlineto{\pgfqpoint{6.400000in}{4.800000in}}%
\pgfpathlineto{\pgfqpoint{0.000000in}{4.800000in}}%
\pgfpathlineto{\pgfqpoint{0.000000in}{0.000000in}}%
\pgfpathclose%
\pgfusepath{fill}%
\end{pgfscope}%
\begin{pgfscope}%
\pgfsetbuttcap%
\pgfsetmiterjoin%
\definecolor{currentfill}{rgb}{1.000000,1.000000,1.000000}%
\pgfsetfillcolor{currentfill}%
\pgfsetlinewidth{0.000000pt}%
\definecolor{currentstroke}{rgb}{0.000000,0.000000,0.000000}%
\pgfsetstrokecolor{currentstroke}%
\pgfsetstrokeopacity{0.000000}%
\pgfsetdash{}{0pt}%
\pgfpathmoveto{\pgfqpoint{0.800000in}{0.528000in}}%
\pgfpathlineto{\pgfqpoint{5.760000in}{0.528000in}}%
\pgfpathlineto{\pgfqpoint{5.760000in}{4.224000in}}%
\pgfpathlineto{\pgfqpoint{0.800000in}{4.224000in}}%
\pgfpathlineto{\pgfqpoint{0.800000in}{0.528000in}}%
\pgfpathclose%
\pgfusepath{fill}%
\end{pgfscope}%
\begin{pgfscope}%
\pgfpathrectangle{\pgfqpoint{0.800000in}{0.528000in}}{\pgfqpoint{4.960000in}{3.696000in}}%
\pgfusepath{clip}%
\pgfsetrectcap%
\pgfsetroundjoin%
\pgfsetlinewidth{1.505625pt}%
\definecolor{currentstroke}{rgb}{0.843137,0.509804,0.494118}%
\pgfsetstrokecolor{currentstroke}%
\pgfsetdash{}{0pt}%
\pgfpathmoveto{\pgfqpoint{1.025455in}{2.470497in}}%
\pgfpathlineto{\pgfqpoint{1.027710in}{2.464425in}}%
\pgfpathlineto{\pgfqpoint{1.029966in}{2.445771in}}%
\pgfpathlineto{\pgfqpoint{1.036733in}{2.345852in}}%
\pgfpathlineto{\pgfqpoint{1.041244in}{2.289415in}}%
\pgfpathlineto{\pgfqpoint{1.043500in}{2.277349in}}%
\pgfpathlineto{\pgfqpoint{1.045756in}{2.279076in}}%
\pgfpathlineto{\pgfqpoint{1.048011in}{2.294530in}}%
\pgfpathlineto{\pgfqpoint{1.052523in}{2.356890in}}%
\pgfpathlineto{\pgfqpoint{1.059290in}{2.460080in}}%
\pgfpathlineto{\pgfqpoint{1.061545in}{2.477298in}}%
\pgfpathlineto{\pgfqpoint{1.063801in}{2.480413in}}%
\pgfpathlineto{\pgfqpoint{1.066057in}{2.468805in}}%
\pgfpathlineto{\pgfqpoint{1.070568in}{2.409139in}}%
\pgfpathlineto{\pgfqpoint{1.077335in}{2.296348in}}%
\pgfpathlineto{\pgfqpoint{1.079591in}{2.273772in}}%
\pgfpathlineto{\pgfqpoint{1.081846in}{2.265365in}}%
\pgfpathlineto{\pgfqpoint{1.084102in}{2.272493in}}%
\pgfpathlineto{\pgfqpoint{1.086358in}{2.294342in}}%
\pgfpathlineto{\pgfqpoint{1.093125in}{2.411265in}}%
\pgfpathlineto{\pgfqpoint{1.097636in}{2.477246in}}%
\pgfpathlineto{\pgfqpoint{1.099892in}{2.491335in}}%
\pgfpathlineto{\pgfqpoint{1.102147in}{2.489295in}}%
\pgfpathlineto{\pgfqpoint{1.104403in}{2.471214in}}%
\pgfpathlineto{\pgfqpoint{1.108914in}{2.398326in}}%
\pgfpathlineto{\pgfqpoint{1.115681in}{2.277823in}}%
\pgfpathlineto{\pgfqpoint{1.117937in}{2.257739in}}%
\pgfpathlineto{\pgfqpoint{1.120193in}{2.254125in}}%
\pgfpathlineto{\pgfqpoint{1.122448in}{2.267693in}}%
\pgfpathlineto{\pgfqpoint{1.126960in}{2.337340in}}%
\pgfpathlineto{\pgfqpoint{1.133727in}{2.468874in}}%
\pgfpathlineto{\pgfqpoint{1.135983in}{2.495177in}}%
\pgfpathlineto{\pgfqpoint{1.138238in}{2.504955in}}%
\pgfpathlineto{\pgfqpoint{1.140494in}{2.496625in}}%
\pgfpathlineto{\pgfqpoint{1.142750in}{2.471145in}}%
\pgfpathlineto{\pgfqpoint{1.149517in}{2.334932in}}%
\pgfpathlineto{\pgfqpoint{1.154028in}{2.258136in}}%
\pgfpathlineto{\pgfqpoint{1.156284in}{2.241758in}}%
\pgfpathlineto{\pgfqpoint{1.158539in}{2.244156in}}%
\pgfpathlineto{\pgfqpoint{1.160795in}{2.265217in}}%
\pgfpathlineto{\pgfqpoint{1.165306in}{2.350033in}}%
\pgfpathlineto{\pgfqpoint{1.172073in}{2.490129in}}%
\pgfpathlineto{\pgfqpoint{1.174329in}{2.513452in}}%
\pgfpathlineto{\pgfqpoint{1.176585in}{2.517628in}}%
\pgfpathlineto{\pgfqpoint{1.178840in}{2.501838in}}%
\pgfpathlineto{\pgfqpoint{1.183352in}{2.420902in}}%
\pgfpathlineto{\pgfqpoint{1.190119in}{2.268188in}}%
\pgfpathlineto{\pgfqpoint{1.192374in}{2.237680in}}%
\pgfpathlineto{\pgfqpoint{1.194630in}{2.226357in}}%
\pgfpathlineto{\pgfqpoint{1.196886in}{2.236048in}}%
\pgfpathlineto{\pgfqpoint{1.199141in}{2.265630in}}%
\pgfpathlineto{\pgfqpoint{1.205908in}{2.423613in}}%
\pgfpathlineto{\pgfqpoint{1.210420in}{2.512602in}}%
\pgfpathlineto{\pgfqpoint{1.212675in}{2.531556in}}%
\pgfpathlineto{\pgfqpoint{1.214931in}{2.528750in}}%
\pgfpathlineto{\pgfqpoint{1.217187in}{2.504327in}}%
\pgfpathlineto{\pgfqpoint{1.221698in}{2.406069in}}%
\pgfpathlineto{\pgfqpoint{1.228465in}{2.243914in}}%
\pgfpathlineto{\pgfqpoint{1.230721in}{2.216950in}}%
\pgfpathlineto{\pgfqpoint{1.232976in}{2.212147in}}%
\pgfpathlineto{\pgfqpoint{1.235232in}{2.230441in}}%
\pgfpathlineto{\pgfqpoint{1.239744in}{2.324079in}}%
\pgfpathlineto{\pgfqpoint{1.246511in}{2.500597in}}%
\pgfpathlineto{\pgfqpoint{1.248766in}{2.535827in}}%
\pgfpathlineto{\pgfqpoint{1.251022in}{2.548879in}}%
\pgfpathlineto{\pgfqpoint{1.253278in}{2.537655in}}%
\pgfpathlineto{\pgfqpoint{1.255533in}{2.503463in}}%
\pgfpathlineto{\pgfqpoint{1.262300in}{2.321042in}}%
\pgfpathlineto{\pgfqpoint{1.266812in}{2.218382in}}%
\pgfpathlineto{\pgfqpoint{1.269067in}{2.196544in}}%
\pgfpathlineto{\pgfqpoint{1.271323in}{2.199813in}}%
\pgfpathlineto{\pgfqpoint{1.273579in}{2.228009in}}%
\pgfpathlineto{\pgfqpoint{1.278090in}{2.341336in}}%
\pgfpathlineto{\pgfqpoint{1.284857in}{2.528190in}}%
\pgfpathlineto{\pgfqpoint{1.287113in}{2.559226in}}%
\pgfpathlineto{\pgfqpoint{1.289368in}{2.564725in}}%
\pgfpathlineto{\pgfqpoint{1.291624in}{2.543625in}}%
\pgfpathlineto{\pgfqpoint{1.296135in}{2.435770in}}%
\pgfpathlineto{\pgfqpoint{1.302902in}{2.232643in}}%
\pgfpathlineto{\pgfqpoint{1.305158in}{2.192140in}}%
\pgfpathlineto{\pgfqpoint{1.307414in}{2.177161in}}%
\pgfpathlineto{\pgfqpoint{1.309669in}{2.190104in}}%
\pgfpathlineto{\pgfqpoint{1.311925in}{2.229450in}}%
\pgfpathlineto{\pgfqpoint{1.318692in}{2.439154in}}%
\pgfpathlineto{\pgfqpoint{1.323203in}{2.557061in}}%
\pgfpathlineto{\pgfqpoint{1.325459in}{2.582110in}}%
\pgfpathlineto{\pgfqpoint{1.327715in}{2.578320in}}%
\pgfpathlineto{\pgfqpoint{1.329970in}{2.545911in}}%
\pgfpathlineto{\pgfqpoint{1.334482in}{2.415784in}}%
\pgfpathlineto{\pgfqpoint{1.341249in}{2.201423in}}%
\pgfpathlineto{\pgfqpoint{1.343504in}{2.165859in}}%
\pgfpathlineto{\pgfqpoint{1.345760in}{2.159590in}}%
\pgfpathlineto{\pgfqpoint{1.348016in}{2.183820in}}%
\pgfpathlineto{\pgfqpoint{1.352527in}{2.307498in}}%
\pgfpathlineto{\pgfqpoint{1.359294in}{2.540212in}}%
\pgfpathlineto{\pgfqpoint{1.361550in}{2.586568in}}%
\pgfpathlineto{\pgfqpoint{1.363806in}{2.603683in}}%
\pgfpathlineto{\pgfqpoint{1.366061in}{2.588825in}}%
\pgfpathlineto{\pgfqpoint{1.368317in}{2.543750in}}%
\pgfpathlineto{\pgfqpoint{1.375084in}{2.303748in}}%
\pgfpathlineto{\pgfqpoint{1.379595in}{2.168932in}}%
\pgfpathlineto{\pgfqpoint{1.381851in}{2.140327in}}%
\pgfpathlineto{\pgfqpoint{1.384107in}{2.144701in}}%
\pgfpathlineto{\pgfqpoint{1.386362in}{2.181786in}}%
\pgfpathlineto{\pgfqpoint{1.390874in}{2.330542in}}%
\pgfpathlineto{\pgfqpoint{1.397641in}{2.575370in}}%
\pgfpathlineto{\pgfqpoint{1.399896in}{2.615942in}}%
\pgfpathlineto{\pgfqpoint{1.402152in}{2.623056in}}%
\pgfpathlineto{\pgfqpoint{1.404408in}{2.595356in}}%
\pgfpathlineto{\pgfqpoint{1.408919in}{2.454161in}}%
\pgfpathlineto{\pgfqpoint{1.415686in}{2.188734in}}%
\pgfpathlineto{\pgfqpoint{1.417942in}{2.135912in}}%
\pgfpathlineto{\pgfqpoint{1.420197in}{2.116444in}}%
\pgfpathlineto{\pgfqpoint{1.422453in}{2.133425in}}%
\pgfpathlineto{\pgfqpoint{1.424709in}{2.184835in}}%
\pgfpathlineto{\pgfqpoint{1.431476in}{2.458293in}}%
\pgfpathlineto{\pgfqpoint{1.435987in}{2.611762in}}%
\pgfpathlineto{\pgfqpoint{1.438243in}{2.644284in}}%
\pgfpathlineto{\pgfqpoint{1.440498in}{2.639257in}}%
\pgfpathlineto{\pgfqpoint{1.442754in}{2.597009in}}%
\pgfpathlineto{\pgfqpoint{1.447265in}{2.427711in}}%
\pgfpathlineto{\pgfqpoint{1.454032in}{2.149325in}}%
\pgfpathlineto{\pgfqpoint{1.456288in}{2.103245in}}%
\pgfpathlineto{\pgfqpoint{1.458544in}{2.095208in}}%
\pgfpathlineto{\pgfqpoint{1.460799in}{2.126735in}}%
\pgfpathlineto{\pgfqpoint{1.465311in}{2.287214in}}%
\pgfpathlineto{\pgfqpoint{1.472078in}{2.588611in}}%
\pgfpathlineto{\pgfqpoint{1.474334in}{2.648534in}}%
\pgfpathlineto{\pgfqpoint{1.476589in}{2.670580in}}%
\pgfpathlineto{\pgfqpoint{1.478845in}{2.651258in}}%
\pgfpathlineto{\pgfqpoint{1.481101in}{2.592884in}}%
\pgfpathlineto{\pgfqpoint{1.487868in}{2.282686in}}%
\pgfpathlineto{\pgfqpoint{1.492379in}{2.108757in}}%
\pgfpathlineto{\pgfqpoint{1.494635in}{2.071947in}}%
\pgfpathlineto{\pgfqpoint{1.496890in}{2.077697in}}%
\pgfpathlineto{\pgfqpoint{1.499146in}{2.125614in}}%
\pgfpathlineto{\pgfqpoint{1.503657in}{2.317436in}}%
\pgfpathlineto{\pgfqpoint{1.510424in}{2.632577in}}%
\pgfpathlineto{\pgfqpoint{1.512680in}{2.684681in}}%
\pgfpathlineto{\pgfqpoint{1.514936in}{2.693720in}}%
\pgfpathlineto{\pgfqpoint{1.517191in}{2.657997in}}%
\pgfpathlineto{\pgfqpoint{1.521703in}{2.476409in}}%
\pgfpathlineto{\pgfqpoint{1.528470in}{2.135684in}}%
\pgfpathlineto{\pgfqpoint{1.530725in}{2.068007in}}%
\pgfpathlineto{\pgfqpoint{1.532981in}{2.043152in}}%
\pgfpathlineto{\pgfqpoint{1.535237in}{2.065038in}}%
\pgfpathlineto{\pgfqpoint{1.537492in}{2.131029in}}%
\pgfpathlineto{\pgfqpoint{1.544259in}{2.481343in}}%
\pgfpathlineto{\pgfqpoint{1.548771in}{2.677585in}}%
\pgfpathlineto{\pgfqpoint{1.551026in}{2.719064in}}%
\pgfpathlineto{\pgfqpoint{1.553282in}{2.712517in}}%
\pgfpathlineto{\pgfqpoint{1.555538in}{2.658411in}}%
\pgfpathlineto{\pgfqpoint{1.560049in}{2.442031in}}%
\pgfpathlineto{\pgfqpoint{1.566816in}{2.086863in}}%
\pgfpathlineto{\pgfqpoint{1.569072in}{2.028209in}}%
\pgfpathlineto{\pgfqpoint{1.571327in}{2.018088in}}%
\pgfpathlineto{\pgfqpoint{1.573583in}{2.058386in}}%
\pgfpathlineto{\pgfqpoint{1.578095in}{2.262949in}}%
\pgfpathlineto{\pgfqpoint{1.584862in}{2.646428in}}%
\pgfpathlineto{\pgfqpoint{1.587117in}{2.722523in}}%
\pgfpathlineto{\pgfqpoint{1.589373in}{2.750422in}}%
\pgfpathlineto{\pgfqpoint{1.591629in}{2.725739in}}%
\pgfpathlineto{\pgfqpoint{1.593884in}{2.651471in}}%
\pgfpathlineto{\pgfqpoint{1.600651in}{2.257605in}}%
\pgfpathlineto{\pgfqpoint{1.605163in}{2.037169in}}%
\pgfpathlineto{\pgfqpoint{1.607418in}{1.990635in}}%
\pgfpathlineto{\pgfqpoint{1.609674in}{1.998057in}}%
\pgfpathlineto{\pgfqpoint{1.611930in}{2.058880in}}%
\pgfpathlineto{\pgfqpoint{1.616441in}{2.301880in}}%
\pgfpathlineto{\pgfqpoint{1.623208in}{2.700385in}}%
\pgfpathlineto{\pgfqpoint{1.625464in}{2.766120in}}%
\pgfpathlineto{\pgfqpoint{1.627719in}{2.777401in}}%
\pgfpathlineto{\pgfqpoint{1.629975in}{2.732143in}}%
\pgfpathlineto{\pgfqpoint{1.634486in}{2.502720in}}%
\pgfpathlineto{\pgfqpoint{1.641253in}{2.073036in}}%
\pgfpathlineto{\pgfqpoint{1.643509in}{1.987855in}}%
\pgfpathlineto{\pgfqpoint{1.645765in}{1.956679in}}%
\pgfpathlineto{\pgfqpoint{1.648020in}{1.984391in}}%
\pgfpathlineto{\pgfqpoint{1.650276in}{2.067606in}}%
\pgfpathlineto{\pgfqpoint{1.657043in}{2.508475in}}%
\pgfpathlineto{\pgfqpoint{1.661554in}{2.754991in}}%
\pgfpathlineto{\pgfqpoint{1.663810in}{2.806963in}}%
\pgfpathlineto{\pgfqpoint{1.666066in}{2.798589in}}%
\pgfpathlineto{\pgfqpoint{1.668321in}{2.730518in}}%
\pgfpathlineto{\pgfqpoint{1.672833in}{2.458832in}}%
\pgfpathlineto{\pgfqpoint{1.679600in}{2.013682in}}%
\pgfpathlineto{\pgfqpoint{1.681855in}{1.940338in}}%
\pgfpathlineto{\pgfqpoint{1.684111in}{1.927820in}}%
\pgfpathlineto{\pgfqpoint{1.686367in}{1.978423in}}%
\pgfpathlineto{\pgfqpoint{1.690878in}{2.234587in}}%
\pgfpathlineto{\pgfqpoint{1.697645in}{2.713911in}}%
\pgfpathlineto{\pgfqpoint{1.699901in}{2.808841in}}%
\pgfpathlineto{\pgfqpoint{1.702157in}{2.843523in}}%
\pgfpathlineto{\pgfqpoint{1.704412in}{2.812548in}}%
\pgfpathlineto{\pgfqpoint{1.706668in}{2.719723in}}%
\pgfpathlineto{\pgfqpoint{1.713435in}{2.228428in}}%
\pgfpathlineto{\pgfqpoint{1.717946in}{1.953968in}}%
\pgfpathlineto{\pgfqpoint{1.720202in}{1.896179in}}%
\pgfpathlineto{\pgfqpoint{1.722458in}{1.905586in}}%
\pgfpathlineto{\pgfqpoint{1.724713in}{1.981431in}}%
\pgfpathlineto{\pgfqpoint{1.729225in}{2.283843in}}%
\pgfpathlineto{\pgfqpoint{1.735992in}{2.778892in}}%
\pgfpathlineto{\pgfqpoint{1.738247in}{2.860364in}}%
\pgfpathlineto{\pgfqpoint{1.740503in}{2.874193in}}%
\pgfpathlineto{\pgfqpoint{1.742759in}{2.817865in}}%
\pgfpathlineto{\pgfqpoint{1.747270in}{2.533109in}}%
\pgfpathlineto{\pgfqpoint{1.754037in}{2.000781in}}%
\pgfpathlineto{\pgfqpoint{1.756293in}{1.895456in}}%
\pgfpathlineto{\pgfqpoint{1.758548in}{1.857044in}}%
\pgfpathlineto{\pgfqpoint{1.760804in}{1.891513in}}%
\pgfpathlineto{\pgfqpoint{1.763060in}{1.994598in}}%
\pgfpathlineto{\pgfqpoint{1.769827in}{2.539662in}}%
\pgfpathlineto{\pgfqpoint{1.774338in}{2.843878in}}%
\pgfpathlineto{\pgfqpoint{1.776594in}{2.907851in}}%
\pgfpathlineto{\pgfqpoint{1.778849in}{2.897331in}}%
\pgfpathlineto{\pgfqpoint{1.781105in}{2.813199in}}%
\pgfpathlineto{\pgfqpoint{1.785616in}{2.478077in}}%
\pgfpathlineto{\pgfqpoint{1.792383in}{1.929975in}}%
\pgfpathlineto{\pgfqpoint{1.794639in}{1.839876in}}%
\pgfpathlineto{\pgfqpoint{1.796895in}{1.824667in}}%
\pgfpathlineto{\pgfqpoint{1.799150in}{1.887091in}}%
\pgfpathlineto{\pgfqpoint{1.803662in}{2.202225in}}%
\pgfpathlineto{\pgfqpoint{1.810429in}{2.790799in}}%
\pgfpathlineto{\pgfqpoint{1.812685in}{2.907140in}}%
\pgfpathlineto{\pgfqpoint{1.814940in}{2.949495in}}%
\pgfpathlineto{\pgfqpoint{1.817196in}{2.911308in}}%
\pgfpathlineto{\pgfqpoint{1.819452in}{2.797335in}}%
\pgfpathlineto{\pgfqpoint{1.826219in}{2.195299in}}%
\pgfpathlineto{\pgfqpoint{1.830730in}{1.859593in}}%
\pgfpathlineto{\pgfqpoint{1.832986in}{1.789090in}}%
\pgfpathlineto{\pgfqpoint{1.835241in}{1.800801in}}%
\pgfpathlineto{\pgfqpoint{1.837497in}{1.893712in}}%
\pgfpathlineto{\pgfqpoint{1.842008in}{2.263435in}}%
\pgfpathlineto{\pgfqpoint{1.848775in}{2.867587in}}%
\pgfpathlineto{\pgfqpoint{1.851031in}{2.966785in}}%
\pgfpathlineto{\pgfqpoint{1.853287in}{2.983437in}}%
\pgfpathlineto{\pgfqpoint{1.855542in}{2.914566in}}%
\pgfpathlineto{\pgfqpoint{1.860054in}{2.567356in}}%
\pgfpathlineto{\pgfqpoint{1.866821in}{1.919478in}}%
\pgfpathlineto{\pgfqpoint{1.869076in}{1.791538in}}%
\pgfpathlineto{\pgfqpoint{1.871332in}{1.745043in}}%
\pgfpathlineto{\pgfqpoint{1.873588in}{1.787160in}}%
\pgfpathlineto{\pgfqpoint{1.875843in}{1.912614in}}%
\pgfpathlineto{\pgfqpoint{1.882610in}{2.574630in}}%
\pgfpathlineto{\pgfqpoint{1.887122in}{2.943442in}}%
\pgfpathlineto{\pgfqpoint{1.889377in}{3.020799in}}%
\pgfpathlineto{\pgfqpoint{1.891633in}{3.007820in}}%
\pgfpathlineto{\pgfqpoint{1.893889in}{2.905670in}}%
\pgfpathlineto{\pgfqpoint{1.898400in}{2.499580in}}%
\pgfpathlineto{\pgfqpoint{1.905167in}{1.836597in}}%
\pgfpathlineto{\pgfqpoint{1.907423in}{1.727865in}}%
\pgfpathlineto{\pgfqpoint{1.909678in}{1.709716in}}%
\pgfpathlineto{\pgfqpoint{1.911934in}{1.785364in}}%
\pgfpathlineto{\pgfqpoint{1.916445in}{2.166217in}}%
\pgfpathlineto{\pgfqpoint{1.923213in}{2.876216in}}%
\pgfpathlineto{\pgfqpoint{1.925468in}{3.016287in}}%
\pgfpathlineto{\pgfqpoint{1.927724in}{3.067100in}}%
\pgfpathlineto{\pgfqpoint{1.929980in}{3.020853in}}%
\pgfpathlineto{\pgfqpoint{1.932235in}{2.883376in}}%
\pgfpathlineto{\pgfqpoint{1.939002in}{2.158630in}}%
\pgfpathlineto{\pgfqpoint{1.943514in}{1.755241in}}%
\pgfpathlineto{\pgfqpoint{1.945769in}{1.670741in}}%
\pgfpathlineto{\pgfqpoint{1.948025in}{1.685060in}}%
\pgfpathlineto{\pgfqpoint{1.950281in}{1.796871in}}%
\pgfpathlineto{\pgfqpoint{1.954792in}{2.240929in}}%
\pgfpathlineto{\pgfqpoint{1.961559in}{2.965247in}}%
\pgfpathlineto{\pgfqpoint{1.963815in}{3.083900in}}%
\pgfpathlineto{\pgfqpoint{1.966070in}{3.103594in}}%
\pgfpathlineto{\pgfqpoint{1.968326in}{3.020870in}}%
\pgfpathlineto{\pgfqpoint{1.972837in}{2.604964in}}%
\pgfpathlineto{\pgfqpoint{1.979604in}{1.830337in}}%
\pgfpathlineto{\pgfqpoint{1.981860in}{1.677664in}}%
\pgfpathlineto{\pgfqpoint{1.984116in}{1.622378in}}%
\pgfpathlineto{\pgfqpoint{1.986371in}{1.672934in}}%
\pgfpathlineto{\pgfqpoint{1.988627in}{1.822920in}}%
\pgfpathlineto{\pgfqpoint{1.995394in}{2.612825in}}%
\pgfpathlineto{\pgfqpoint{1.999905in}{3.052076in}}%
\pgfpathlineto{\pgfqpoint{2.002161in}{3.143969in}}%
\pgfpathlineto{\pgfqpoint{2.004417in}{3.128244in}}%
\pgfpathlineto{\pgfqpoint{2.006672in}{3.006400in}}%
\pgfpathlineto{\pgfqpoint{2.011184in}{2.522977in}}%
\pgfpathlineto{\pgfqpoint{2.017951in}{1.735156in}}%
\pgfpathlineto{\pgfqpoint{2.020206in}{1.606250in}}%
\pgfpathlineto{\pgfqpoint{2.022462in}{1.584975in}}%
\pgfpathlineto{\pgfqpoint{2.024718in}{1.675035in}}%
\pgfpathlineto{\pgfqpoint{2.029229in}{2.127208in}}%
\pgfpathlineto{\pgfqpoint{2.035996in}{2.968601in}}%
\pgfpathlineto{\pgfqpoint{2.038252in}{3.134272in}}%
\pgfpathlineto{\pgfqpoint{2.040508in}{3.194158in}}%
\pgfpathlineto{\pgfqpoint{2.042763in}{3.139137in}}%
\pgfpathlineto{\pgfqpoint{2.045019in}{2.976229in}}%
\pgfpathlineto{\pgfqpoint{2.051786in}{2.119123in}}%
\pgfpathlineto{\pgfqpoint{2.056297in}{1.642941in}}%
\pgfpathlineto{\pgfqpoint{2.058553in}{1.543450in}}%
\pgfpathlineto{\pgfqpoint{2.060809in}{1.560643in}}%
\pgfpathlineto{\pgfqpoint{2.063064in}{1.692831in}}%
\pgfpathlineto{\pgfqpoint{2.067576in}{2.216777in}}%
\pgfpathlineto{\pgfqpoint{2.074343in}{3.069871in}}%
\pgfpathlineto{\pgfqpoint{2.076598in}{3.209296in}}%
\pgfpathlineto{\pgfqpoint{2.078854in}{3.232174in}}%
\pgfpathlineto{\pgfqpoint{2.081110in}{3.134562in}}%
\pgfpathlineto{\pgfqpoint{2.085621in}{2.645139in}}%
\pgfpathlineto{\pgfqpoint{2.092388in}{1.735276in}}%
\pgfpathlineto{\pgfqpoint{2.094644in}{1.556296in}}%
\pgfpathlineto{\pgfqpoint{2.096899in}{1.491716in}}%
\pgfpathlineto{\pgfqpoint{2.099155in}{1.551330in}}%
\pgfpathlineto{\pgfqpoint{2.101411in}{1.727489in}}%
\pgfpathlineto{\pgfqpoint{2.108178in}{2.653392in}}%
\pgfpathlineto{\pgfqpoint{2.112689in}{3.167322in}}%
\pgfpathlineto{\pgfqpoint{2.114945in}{3.274561in}}%
\pgfpathlineto{\pgfqpoint{2.117200in}{3.255849in}}%
\pgfpathlineto{\pgfqpoint{2.119456in}{3.113074in}}%
\pgfpathlineto{\pgfqpoint{2.123967in}{2.547726in}}%
\pgfpathlineto{\pgfqpoint{2.130734in}{1.628044in}}%
\pgfpathlineto{\pgfqpoint{2.132990in}{1.477911in}}%
\pgfpathlineto{\pgfqpoint{2.135246in}{1.453418in}}%
\pgfpathlineto{\pgfqpoint{2.137501in}{1.558746in}}%
\pgfpathlineto{\pgfqpoint{2.142013in}{2.086139in}}%
\pgfpathlineto{\pgfqpoint{2.148780in}{3.065686in}}%
\pgfpathlineto{\pgfqpoint{2.151036in}{3.258186in}}%
\pgfpathlineto{\pgfqpoint{2.153291in}{3.327519in}}%
\pgfpathlineto{\pgfqpoint{2.155547in}{3.263215in}}%
\pgfpathlineto{\pgfqpoint{2.157803in}{3.073572in}}%
\pgfpathlineto{\pgfqpoint{2.164570in}{2.077782in}}%
\pgfpathlineto{\pgfqpoint{2.169081in}{1.525569in}}%
\pgfpathlineto{\pgfqpoint{2.171337in}{1.410491in}}%
\pgfpathlineto{\pgfqpoint{2.173592in}{1.430765in}}%
\pgfpathlineto{\pgfqpoint{2.175848in}{1.584291in}}%
\pgfpathlineto{\pgfqpoint{2.180359in}{2.191611in}}%
\pgfpathlineto{\pgfqpoint{2.187126in}{3.178685in}}%
\pgfpathlineto{\pgfqpoint{2.189382in}{3.339631in}}%
\pgfpathlineto{\pgfqpoint{2.191638in}{3.365736in}}%
\pgfpathlineto{\pgfqpoint{2.193893in}{3.252585in}}%
\pgfpathlineto{\pgfqpoint{2.198405in}{2.686793in}}%
\pgfpathlineto{\pgfqpoint{2.205172in}{1.636901in}}%
\pgfpathlineto{\pgfqpoint{2.207427in}{1.430777in}}%
\pgfpathlineto{\pgfqpoint{2.209683in}{1.356670in}}%
\pgfpathlineto{\pgfqpoint{2.211939in}{1.425726in}}%
\pgfpathlineto{\pgfqpoint{2.214194in}{1.628981in}}%
\pgfpathlineto{\pgfqpoint{2.220961in}{2.695187in}}%
\pgfpathlineto{\pgfqpoint{2.225473in}{3.285905in}}%
\pgfpathlineto{\pgfqpoint{2.227728in}{3.408847in}}%
\pgfpathlineto{\pgfqpoint{2.229984in}{3.386979in}}%
\pgfpathlineto{\pgfqpoint{2.232240in}{3.222625in}}%
\pgfpathlineto{\pgfqpoint{2.236751in}{2.573109in}}%
\pgfpathlineto{\pgfqpoint{2.243518in}{1.518400in}}%
\pgfpathlineto{\pgfqpoint{2.245774in}{1.346626in}}%
\pgfpathlineto{\pgfqpoint{2.248029in}{1.318928in}}%
\pgfpathlineto{\pgfqpoint{2.250285in}{1.439943in}}%
\pgfpathlineto{\pgfqpoint{2.254796in}{2.044239in}}%
\pgfpathlineto{\pgfqpoint{2.261564in}{3.164543in}}%
\pgfpathlineto{\pgfqpoint{2.263819in}{3.384277in}}%
\pgfpathlineto{\pgfqpoint{2.266075in}{3.463133in}}%
\pgfpathlineto{\pgfqpoint{2.268331in}{3.389304in}}%
\pgfpathlineto{\pgfqpoint{2.270586in}{3.172427in}}%
\pgfpathlineto{\pgfqpoint{2.277353in}{2.035884in}}%
\pgfpathlineto{\pgfqpoint{2.281865in}{1.406776in}}%
\pgfpathlineto{\pgfqpoint{2.284120in}{1.276014in}}%
\pgfpathlineto{\pgfqpoint{2.286376in}{1.299495in}}%
\pgfpathlineto{\pgfqpoint{2.288632in}{1.474662in}}%
\pgfpathlineto{\pgfqpoint{2.293143in}{2.166227in}}%
\pgfpathlineto{\pgfqpoint{2.299910in}{3.288211in}}%
\pgfpathlineto{\pgfqpoint{2.302166in}{3.470730in}}%
\pgfpathlineto{\pgfqpoint{2.304421in}{3.499987in}}%
\pgfpathlineto{\pgfqpoint{2.306677in}{3.371134in}}%
\pgfpathlineto{\pgfqpoint{2.311188in}{2.728575in}}%
\pgfpathlineto{\pgfqpoint{2.317955in}{1.538434in}}%
\pgfpathlineto{\pgfqpoint{2.320211in}{1.305230in}}%
\pgfpathlineto{\pgfqpoint{2.322467in}{1.221691in}}%
\pgfpathlineto{\pgfqpoint{2.324722in}{1.300273in}}%
\pgfpathlineto{\pgfqpoint{2.326978in}{1.530661in}}%
\pgfpathlineto{\pgfqpoint{2.333745in}{2.736812in}}%
\pgfpathlineto{\pgfqpoint{2.338256in}{3.403836in}}%
\pgfpathlineto{\pgfqpoint{2.340512in}{3.542297in}}%
\pgfpathlineto{\pgfqpoint{2.342768in}{3.517198in}}%
\pgfpathlineto{\pgfqpoint{2.345023in}{3.331335in}}%
\pgfpathlineto{\pgfqpoint{2.349535in}{2.598260in}}%
\pgfpathlineto{\pgfqpoint{2.356302in}{1.409999in}}%
\pgfpathlineto{\pgfqpoint{2.358557in}{1.216924in}}%
\pgfpathlineto{\pgfqpoint{2.360813in}{1.186159in}}%
\pgfpathlineto{\pgfqpoint{2.363069in}{1.322748in}}%
\pgfpathlineto{\pgfqpoint{2.367580in}{2.002967in}}%
\pgfpathlineto{\pgfqpoint{2.374347in}{3.261694in}}%
\pgfpathlineto{\pgfqpoint{2.376603in}{3.508098in}}%
\pgfpathlineto{\pgfqpoint{2.378859in}{3.596205in}}%
\pgfpathlineto{\pgfqpoint{2.381114in}{3.512935in}}%
\pgfpathlineto{\pgfqpoint{2.383370in}{3.269279in}}%
\pgfpathlineto{\pgfqpoint{2.390137in}{1.994929in}}%
\pgfpathlineto{\pgfqpoint{2.394648in}{1.290839in}}%
\pgfpathlineto{\pgfqpoint{2.396904in}{1.144874in}}%
\pgfpathlineto{\pgfqpoint{2.399160in}{1.171582in}}%
\pgfpathlineto{\pgfqpoint{2.401415in}{1.367921in}}%
\pgfpathlineto{\pgfqpoint{2.405927in}{2.141552in}}%
\pgfpathlineto{\pgfqpoint{2.412694in}{3.394429in}}%
\pgfpathlineto{\pgfqpoint{2.414949in}{3.597766in}}%
\pgfpathlineto{\pgfqpoint{2.417205in}{3.629972in}}%
\pgfpathlineto{\pgfqpoint{2.419461in}{3.485824in}}%
\pgfpathlineto{\pgfqpoint{2.423972in}{2.768930in}}%
\pgfpathlineto{\pgfqpoint{2.430739in}{1.443563in}}%
\pgfpathlineto{\pgfqpoint{2.432995in}{1.184367in}}%
\pgfpathlineto{\pgfqpoint{2.435250in}{1.091854in}}%
\pgfpathlineto{\pgfqpoint{2.437506in}{1.179700in}}%
\pgfpathlineto{\pgfqpoint{2.439762in}{1.436246in}}%
\pgfpathlineto{\pgfqpoint{2.446529in}{2.776684in}}%
\pgfpathlineto{\pgfqpoint{2.451040in}{3.516606in}}%
\pgfpathlineto{\pgfqpoint{2.453296in}{3.669800in}}%
\pgfpathlineto{\pgfqpoint{2.455551in}{3.641507in}}%
\pgfpathlineto{\pgfqpoint{2.457807in}{3.435021in}}%
\pgfpathlineto{\pgfqpoint{2.462318in}{2.622208in}}%
\pgfpathlineto{\pgfqpoint{2.469085in}{1.307055in}}%
\pgfpathlineto{\pgfqpoint{2.471341in}{1.093860in}}%
\pgfpathlineto{\pgfqpoint{2.473597in}{1.060296in}}%
\pgfpathlineto{\pgfqpoint{2.475852in}{1.211748in}}%
\pgfpathlineto{\pgfqpoint{2.480364in}{1.963947in}}%
\pgfpathlineto{\pgfqpoint{2.487131in}{3.353298in}}%
\pgfpathlineto{\pgfqpoint{2.489387in}{3.624742in}}%
\pgfpathlineto{\pgfqpoint{2.491642in}{3.721449in}}%
\pgfpathlineto{\pgfqpoint{2.493898in}{3.629187in}}%
\pgfpathlineto{\pgfqpoint{2.496154in}{3.360267in}}%
\pgfpathlineto{\pgfqpoint{2.502921in}{1.956561in}}%
\pgfpathlineto{\pgfqpoint{2.507432in}{1.182428in}}%
\pgfpathlineto{\pgfqpoint{2.509688in}{1.022362in}}%
\pgfpathlineto{\pgfqpoint{2.511943in}{1.052198in}}%
\pgfpathlineto{\pgfqpoint{2.514199in}{1.268392in}}%
\pgfpathlineto{\pgfqpoint{2.518710in}{2.118587in}}%
\pgfpathlineto{\pgfqpoint{2.525477in}{3.492994in}}%
\pgfpathlineto{\pgfqpoint{2.527733in}{3.715534in}}%
\pgfpathlineto{\pgfqpoint{2.529989in}{3.750356in}}%
\pgfpathlineto{\pgfqpoint{2.532244in}{3.591937in}}%
\pgfpathlineto{\pgfqpoint{2.536756in}{2.806193in}}%
\pgfpathlineto{\pgfqpoint{2.543523in}{1.356223in}}%
\pgfpathlineto{\pgfqpoint{2.545778in}{1.073212in}}%
\pgfpathlineto{\pgfqpoint{2.548034in}{0.972569in}}%
\pgfpathlineto{\pgfqpoint{2.550290in}{1.069039in}}%
\pgfpathlineto{\pgfqpoint{2.552545in}{1.349680in}}%
\pgfpathlineto{\pgfqpoint{2.559312in}{2.813126in}}%
\pgfpathlineto{\pgfqpoint{2.563824in}{3.619462in}}%
\pgfpathlineto{\pgfqpoint{2.566079in}{3.785968in}}%
\pgfpathlineto{\pgfqpoint{2.568335in}{3.754645in}}%
\pgfpathlineto{\pgfqpoint{2.570591in}{3.529290in}}%
\pgfpathlineto{\pgfqpoint{2.575102in}{2.643934in}}%
\pgfpathlineto{\pgfqpoint{2.581869in}{1.213970in}}%
\pgfpathlineto{\pgfqpoint{2.584125in}{0.982705in}}%
\pgfpathlineto{\pgfqpoint{2.586380in}{0.946739in}}%
\pgfpathlineto{\pgfqpoint{2.588636in}{1.111711in}}%
\pgfpathlineto{\pgfqpoint{2.593147in}{1.928860in}}%
\pgfpathlineto{\pgfqpoint{2.599915in}{3.435387in}}%
\pgfpathlineto{\pgfqpoint{2.602170in}{3.729151in}}%
\pgfpathlineto{\pgfqpoint{2.604426in}{3.833425in}}%
\pgfpathlineto{\pgfqpoint{2.606682in}{3.733002in}}%
\pgfpathlineto{\pgfqpoint{2.608937in}{3.441426in}}%
\pgfpathlineto{\pgfqpoint{2.615704in}{1.922459in}}%
\pgfpathlineto{\pgfqpoint{2.620216in}{1.086304in}}%
\pgfpathlineto{\pgfqpoint{2.622471in}{0.913866in}}%
\pgfpathlineto{\pgfqpoint{2.624727in}{0.946601in}}%
\pgfpathlineto{\pgfqpoint{2.626983in}{1.180466in}}%
\pgfpathlineto{\pgfqpoint{2.631494in}{2.098350in}}%
\pgfpathlineto{\pgfqpoint{2.638261in}{3.579525in}}%
\pgfpathlineto{\pgfqpoint{2.640517in}{3.818793in}}%
\pgfpathlineto{\pgfqpoint{2.642772in}{3.855773in}}%
\pgfpathlineto{\pgfqpoint{2.645028in}{3.684738in}}%
\pgfpathlineto{\pgfqpoint{2.649539in}{2.838696in}}%
\pgfpathlineto{\pgfqpoint{2.656306in}{1.280342in}}%
\pgfpathlineto{\pgfqpoint{2.658562in}{0.976768in}}%
\pgfpathlineto{\pgfqpoint{2.660818in}{0.869210in}}%
\pgfpathlineto{\pgfqpoint{2.663073in}{0.973284in}}%
\pgfpathlineto{\pgfqpoint{2.665329in}{1.274878in}}%
\pgfpathlineto{\pgfqpoint{2.672096in}{2.844486in}}%
\pgfpathlineto{\pgfqpoint{2.676607in}{3.707722in}}%
\pgfpathlineto{\pgfqpoint{2.678863in}{3.885511in}}%
\pgfpathlineto{\pgfqpoint{2.681119in}{3.851453in}}%
\pgfpathlineto{\pgfqpoint{2.683374in}{3.609834in}}%
\pgfpathlineto{\pgfqpoint{2.687886in}{2.662443in}}%
\pgfpathlineto{\pgfqpoint{2.694653in}{1.135022in}}%
\pgfpathlineto{\pgfqpoint{2.696908in}{0.888573in}}%
\pgfpathlineto{\pgfqpoint{2.699164in}{0.850719in}}%
\pgfpathlineto{\pgfqpoint{2.701420in}{1.027254in}}%
\pgfpathlineto{\pgfqpoint{2.705931in}{1.899328in}}%
\pgfpathlineto{\pgfqpoint{2.712698in}{3.504150in}}%
\pgfpathlineto{\pgfqpoint{2.714954in}{3.816470in}}%
\pgfpathlineto{\pgfqpoint{2.717210in}{3.926923in}}%
\pgfpathlineto{\pgfqpoint{2.719465in}{3.819544in}}%
\pgfpathlineto{\pgfqpoint{2.721721in}{3.508971in}}%
\pgfpathlineto{\pgfqpoint{2.728488in}{1.894220in}}%
\pgfpathlineto{\pgfqpoint{2.732999in}{1.006976in}}%
\pgfpathlineto{\pgfqpoint{2.735255in}{0.824483in}}%
\pgfpathlineto{\pgfqpoint{2.737511in}{0.859759in}}%
\pgfpathlineto{\pgfqpoint{2.739766in}{1.108282in}}%
\pgfpathlineto{\pgfqpoint{2.744278in}{2.081796in}}%
\pgfpathlineto{\pgfqpoint{2.751045in}{3.649927in}}%
\pgfpathlineto{\pgfqpoint{2.753300in}{3.902648in}}%
\pgfpathlineto{\pgfqpoint{2.755556in}{3.941221in}}%
\pgfpathlineto{\pgfqpoint{2.757812in}{3.759818in}}%
\pgfpathlineto{\pgfqpoint{2.762323in}{2.864893in}}%
\pgfpathlineto{\pgfqpoint{2.769090in}{1.219542in}}%
\pgfpathlineto{\pgfqpoint{2.771346in}{0.899648in}}%
\pgfpathlineto{\pgfqpoint{2.773601in}{0.786726in}}%
\pgfpathlineto{\pgfqpoint{2.775857in}{0.897023in}}%
\pgfpathlineto{\pgfqpoint{2.778113in}{1.215426in}}%
\pgfpathlineto{\pgfqpoint{2.784880in}{2.869255in}}%
\pgfpathlineto{\pgfqpoint{2.789391in}{3.777134in}}%
\pgfpathlineto{\pgfqpoint{2.791647in}{3.963625in}}%
\pgfpathlineto{\pgfqpoint{2.793902in}{3.927253in}}%
\pgfpathlineto{\pgfqpoint{2.796158in}{3.672761in}}%
\pgfpathlineto{\pgfqpoint{2.800669in}{2.676838in}}%
\pgfpathlineto{\pgfqpoint{2.807436in}{1.074046in}}%
\pgfpathlineto{\pgfqpoint{2.809692in}{0.816042in}}%
\pgfpathlineto{\pgfqpoint{2.811948in}{0.776911in}}%
\pgfpathlineto{\pgfqpoint{2.814203in}{0.962490in}}%
\pgfpathlineto{\pgfqpoint{2.818715in}{1.876795in}}%
\pgfpathlineto{\pgfqpoint{2.825482in}{3.556223in}}%
\pgfpathlineto{\pgfqpoint{2.827738in}{3.882423in}}%
\pgfpathlineto{\pgfqpoint{2.829993in}{3.997357in}}%
\pgfpathlineto{\pgfqpoint{2.832249in}{3.884566in}}%
\pgfpathlineto{\pgfqpoint{2.834505in}{3.559582in}}%
\pgfpathlineto{\pgfqpoint{2.841272in}{1.873235in}}%
\pgfpathlineto{\pgfqpoint{2.845783in}{0.948358in}}%
\pgfpathlineto{\pgfqpoint{2.848039in}{0.758625in}}%
\pgfpathlineto{\pgfqpoint{2.850294in}{0.795960in}}%
\pgfpathlineto{\pgfqpoint{2.852550in}{1.055410in}}%
\pgfpathlineto{\pgfqpoint{2.857061in}{2.069743in}}%
\pgfpathlineto{\pgfqpoint{2.863828in}{3.700702in}}%
\pgfpathlineto{\pgfqpoint{2.866084in}{3.962932in}}%
\pgfpathlineto{\pgfqpoint{2.868340in}{4.002451in}}%
\pgfpathlineto{\pgfqpoint{2.870595in}{3.813440in}}%
\pgfpathlineto{\pgfqpoint{2.875107in}{2.883477in}}%
\pgfpathlineto{\pgfqpoint{2.881874in}{1.176862in}}%
\pgfpathlineto{\pgfqpoint{2.884129in}{0.845705in}}%
\pgfpathlineto{\pgfqpoint{2.886385in}{0.729243in}}%
\pgfpathlineto{\pgfqpoint{2.888641in}{0.844073in}}%
\pgfpathlineto{\pgfqpoint{2.890896in}{1.174303in}}%
\pgfpathlineto{\pgfqpoint{2.897663in}{2.886188in}}%
\pgfpathlineto{\pgfqpoint{2.902175in}{3.824206in}}%
\pgfpathlineto{\pgfqpoint{2.904430in}{4.016379in}}%
\pgfpathlineto{\pgfqpoint{2.906686in}{3.978229in}}%
\pgfpathlineto{\pgfqpoint{2.908942in}{3.714898in}}%
\pgfpathlineto{\pgfqpoint{2.913453in}{2.686392in}}%
\pgfpathlineto{\pgfqpoint{2.920220in}{1.034126in}}%
\pgfpathlineto{\pgfqpoint{2.922476in}{0.768782in}}%
\pgfpathlineto{\pgfqpoint{2.924731in}{0.729051in}}%
\pgfpathlineto{\pgfqpoint{2.926987in}{0.920702in}}%
\pgfpathlineto{\pgfqpoint{2.931498in}{1.862401in}}%
\pgfpathlineto{\pgfqpoint{2.938265in}{3.588958in}}%
\pgfpathlineto{\pgfqpoint{2.940521in}{3.923656in}}%
\pgfpathlineto{\pgfqpoint{2.942777in}{4.041143in}}%
\pgfpathlineto{\pgfqpoint{2.945033in}{3.924756in}}%
\pgfpathlineto{\pgfqpoint{2.947288in}{3.590683in}}%
\pgfpathlineto{\pgfqpoint{2.954055in}{1.860573in}}%
\pgfpathlineto{\pgfqpoint{2.958567in}{0.913445in}}%
\pgfpathlineto{\pgfqpoint{2.960822in}{0.719661in}}%
\pgfpathlineto{\pgfqpoint{2.963078in}{0.758470in}}%
\pgfpathlineto{\pgfqpoint{2.965334in}{1.024556in}}%
\pgfpathlineto{\pgfqpoint{2.969845in}{2.062810in}}%
\pgfpathlineto{\pgfqpoint{2.976612in}{3.729246in}}%
\pgfpathlineto{\pgfqpoint{2.978868in}{3.996550in}}%
\pgfpathlineto{\pgfqpoint{2.981123in}{4.036316in}}%
\pgfpathlineto{\pgfqpoint{2.983379in}{3.842849in}}%
\pgfpathlineto{\pgfqpoint{2.987890in}{2.893491in}}%
\pgfpathlineto{\pgfqpoint{2.994657in}{1.154501in}}%
\pgfpathlineto{\pgfqpoint{2.996913in}{0.817723in}}%
\pgfpathlineto{\pgfqpoint{2.999169in}{0.699727in}}%
\pgfpathlineto{\pgfqpoint{3.001424in}{0.817169in}}%
\pgfpathlineto{\pgfqpoint{3.003680in}{1.153632in}}%
\pgfpathlineto{\pgfqpoint{3.010447in}{2.894411in}}%
\pgfpathlineto{\pgfqpoint{3.014958in}{3.846501in}}%
\pgfpathlineto{\pgfqpoint{3.017214in}{4.041042in}}%
\pgfpathlineto{\pgfqpoint{3.019470in}{4.001740in}}%
\pgfpathlineto{\pgfqpoint{3.021725in}{3.734062in}}%
\pgfpathlineto{\pgfqpoint{3.026237in}{2.690612in}}%
\pgfpathlineto{\pgfqpoint{3.033004in}{1.017334in}}%
\pgfpathlineto{\pgfqpoint{3.035259in}{0.749249in}}%
\pgfpathlineto{\pgfqpoint{3.037515in}{0.709626in}}%
\pgfpathlineto{\pgfqpoint{3.039771in}{0.904061in}}%
\pgfpathlineto{\pgfqpoint{3.044282in}{1.856898in}}%
\pgfpathlineto{\pgfqpoint{3.051049in}{3.600650in}}%
\pgfpathlineto{\pgfqpoint{3.053305in}{3.938019in}}%
\pgfpathlineto{\pgfqpoint{3.055561in}{4.056000in}}%
\pgfpathlineto{\pgfqpoint{3.057816in}{3.938019in}}%
\pgfpathlineto{\pgfqpoint{3.060072in}{3.600650in}}%
\pgfpathlineto{\pgfqpoint{3.066839in}{1.856898in}}%
\pgfpathlineto{\pgfqpoint{3.071350in}{0.904061in}}%
\pgfpathlineto{\pgfqpoint{3.073606in}{0.709626in}}%
\pgfpathlineto{\pgfqpoint{3.075862in}{0.749249in}}%
\pgfpathlineto{\pgfqpoint{3.078117in}{1.017334in}}%
\pgfpathlineto{\pgfqpoint{3.082629in}{2.061360in}}%
\pgfpathlineto{\pgfqpoint{3.089396in}{3.734062in}}%
\pgfpathlineto{\pgfqpoint{3.091651in}{4.001740in}}%
\pgfpathlineto{\pgfqpoint{3.093907in}{4.041042in}}%
\pgfpathlineto{\pgfqpoint{3.096163in}{3.846501in}}%
\pgfpathlineto{\pgfqpoint{3.100674in}{2.894411in}}%
\pgfpathlineto{\pgfqpoint{3.107441in}{1.153632in}}%
\pgfpathlineto{\pgfqpoint{3.109697in}{0.817169in}}%
\pgfpathlineto{\pgfqpoint{3.111952in}{0.699727in}}%
\pgfpathlineto{\pgfqpoint{3.114208in}{0.817723in}}%
\pgfpathlineto{\pgfqpoint{3.116464in}{1.154501in}}%
\pgfpathlineto{\pgfqpoint{3.123231in}{2.893491in}}%
\pgfpathlineto{\pgfqpoint{3.127742in}{3.842849in}}%
\pgfpathlineto{\pgfqpoint{3.129998in}{4.036316in}}%
\pgfpathlineto{\pgfqpoint{3.132253in}{3.996550in}}%
\pgfpathlineto{\pgfqpoint{3.134509in}{3.729246in}}%
\pgfpathlineto{\pgfqpoint{3.139020in}{2.689273in}}%
\pgfpathlineto{\pgfqpoint{3.145787in}{1.024556in}}%
\pgfpathlineto{\pgfqpoint{3.148043in}{0.758470in}}%
\pgfpathlineto{\pgfqpoint{3.150299in}{0.719661in}}%
\pgfpathlineto{\pgfqpoint{3.152554in}{0.913445in}}%
\pgfpathlineto{\pgfqpoint{3.157066in}{1.860573in}}%
\pgfpathlineto{\pgfqpoint{3.163833in}{3.590683in}}%
\pgfpathlineto{\pgfqpoint{3.166088in}{3.924756in}}%
\pgfpathlineto{\pgfqpoint{3.168344in}{4.041143in}}%
\pgfpathlineto{\pgfqpoint{3.170600in}{3.923656in}}%
\pgfpathlineto{\pgfqpoint{3.172856in}{3.588958in}}%
\pgfpathlineto{\pgfqpoint{3.179623in}{1.862401in}}%
\pgfpathlineto{\pgfqpoint{3.184134in}{0.920702in}}%
\pgfpathlineto{\pgfqpoint{3.186390in}{0.729051in}}%
\pgfpathlineto{\pgfqpoint{3.188645in}{0.768782in}}%
\pgfpathlineto{\pgfqpoint{3.190901in}{1.034126in}}%
\pgfpathlineto{\pgfqpoint{3.195412in}{2.065470in}}%
\pgfpathlineto{\pgfqpoint{3.202179in}{3.714898in}}%
\pgfpathlineto{\pgfqpoint{3.204435in}{3.978229in}}%
\pgfpathlineto{\pgfqpoint{3.206691in}{4.016379in}}%
\pgfpathlineto{\pgfqpoint{3.208946in}{3.824206in}}%
\pgfpathlineto{\pgfqpoint{3.213458in}{2.886188in}}%
\pgfpathlineto{\pgfqpoint{3.220225in}{1.174303in}}%
\pgfpathlineto{\pgfqpoint{3.222480in}{0.844073in}}%
\pgfpathlineto{\pgfqpoint{3.224736in}{0.729243in}}%
\pgfpathlineto{\pgfqpoint{3.226992in}{0.845705in}}%
\pgfpathlineto{\pgfqpoint{3.229247in}{1.176862in}}%
\pgfpathlineto{\pgfqpoint{3.236014in}{2.883477in}}%
\pgfpathlineto{\pgfqpoint{3.240526in}{3.813440in}}%
\pgfpathlineto{\pgfqpoint{3.242781in}{4.002451in}}%
\pgfpathlineto{\pgfqpoint{3.245037in}{3.962932in}}%
\pgfpathlineto{\pgfqpoint{3.247293in}{3.700702in}}%
\pgfpathlineto{\pgfqpoint{3.251804in}{2.682447in}}%
\pgfpathlineto{\pgfqpoint{3.258571in}{1.055410in}}%
\pgfpathlineto{\pgfqpoint{3.260827in}{0.795960in}}%
\pgfpathlineto{\pgfqpoint{3.263082in}{0.758625in}}%
\pgfpathlineto{\pgfqpoint{3.265338in}{0.948358in}}%
\pgfpathlineto{\pgfqpoint{3.269849in}{1.873235in}}%
\pgfpathlineto{\pgfqpoint{3.276616in}{3.559582in}}%
\pgfpathlineto{\pgfqpoint{3.278872in}{3.884566in}}%
\pgfpathlineto{\pgfqpoint{3.281128in}{3.997357in}}%
\pgfpathlineto{\pgfqpoint{3.283384in}{3.882423in}}%
\pgfpathlineto{\pgfqpoint{3.285639in}{3.556223in}}%
\pgfpathlineto{\pgfqpoint{3.292406in}{1.876795in}}%
\pgfpathlineto{\pgfqpoint{3.296918in}{0.962490in}}%
\pgfpathlineto{\pgfqpoint{3.299173in}{0.776911in}}%
\pgfpathlineto{\pgfqpoint{3.301429in}{0.816042in}}%
\pgfpathlineto{\pgfqpoint{3.303685in}{1.074046in}}%
\pgfpathlineto{\pgfqpoint{3.308196in}{2.074922in}}%
\pgfpathlineto{\pgfqpoint{3.314963in}{3.672761in}}%
\pgfpathlineto{\pgfqpoint{3.317219in}{3.927253in}}%
\pgfpathlineto{\pgfqpoint{3.319474in}{3.963625in}}%
\pgfpathlineto{\pgfqpoint{3.321730in}{3.777134in}}%
\pgfpathlineto{\pgfqpoint{3.326241in}{2.869255in}}%
\pgfpathlineto{\pgfqpoint{3.333008in}{1.215426in}}%
\pgfpathlineto{\pgfqpoint{3.335264in}{0.897023in}}%
\pgfpathlineto{\pgfqpoint{3.337520in}{0.786726in}}%
\pgfpathlineto{\pgfqpoint{3.339775in}{0.899648in}}%
\pgfpathlineto{\pgfqpoint{3.342031in}{1.219542in}}%
\pgfpathlineto{\pgfqpoint{3.348798in}{2.864893in}}%
\pgfpathlineto{\pgfqpoint{3.353309in}{3.759818in}}%
\pgfpathlineto{\pgfqpoint{3.355565in}{3.941221in}}%
\pgfpathlineto{\pgfqpoint{3.357821in}{3.902648in}}%
\pgfpathlineto{\pgfqpoint{3.360076in}{3.649927in}}%
\pgfpathlineto{\pgfqpoint{3.364588in}{2.670492in}}%
\pgfpathlineto{\pgfqpoint{3.371355in}{1.108282in}}%
\pgfpathlineto{\pgfqpoint{3.373610in}{0.859759in}}%
\pgfpathlineto{\pgfqpoint{3.375866in}{0.824483in}}%
\pgfpathlineto{\pgfqpoint{3.378122in}{1.006976in}}%
\pgfpathlineto{\pgfqpoint{3.382633in}{1.894220in}}%
\pgfpathlineto{\pgfqpoint{3.389400in}{3.508971in}}%
\pgfpathlineto{\pgfqpoint{3.391656in}{3.819544in}}%
\pgfpathlineto{\pgfqpoint{3.393912in}{3.926923in}}%
\pgfpathlineto{\pgfqpoint{3.396167in}{3.816470in}}%
\pgfpathlineto{\pgfqpoint{3.398423in}{3.504150in}}%
\pgfpathlineto{\pgfqpoint{3.411957in}{0.850719in}}%
\pgfpathlineto{\pgfqpoint{3.414213in}{0.888573in}}%
\pgfpathlineto{\pgfqpoint{3.416468in}{1.135022in}}%
\pgfpathlineto{\pgfqpoint{3.420980in}{2.089226in}}%
\pgfpathlineto{\pgfqpoint{3.427747in}{3.609834in}}%
\pgfpathlineto{\pgfqpoint{3.430002in}{3.851453in}}%
\pgfpathlineto{\pgfqpoint{3.432258in}{3.885511in}}%
\pgfpathlineto{\pgfqpoint{3.434514in}{3.707722in}}%
\pgfpathlineto{\pgfqpoint{3.439025in}{2.844486in}}%
\pgfpathlineto{\pgfqpoint{3.445792in}{1.274878in}}%
\pgfpathlineto{\pgfqpoint{3.448048in}{0.973284in}}%
\pgfpathlineto{\pgfqpoint{3.450303in}{0.869210in}}%
\pgfpathlineto{\pgfqpoint{3.452559in}{0.976768in}}%
\pgfpathlineto{\pgfqpoint{3.454815in}{1.280342in}}%
\pgfpathlineto{\pgfqpoint{3.468349in}{3.855773in}}%
\pgfpathlineto{\pgfqpoint{3.470604in}{3.818793in}}%
\pgfpathlineto{\pgfqpoint{3.472860in}{3.579525in}}%
\pgfpathlineto{\pgfqpoint{3.477371in}{2.654020in}}%
\pgfpathlineto{\pgfqpoint{3.484138in}{1.180466in}}%
\pgfpathlineto{\pgfqpoint{3.486394in}{0.946601in}}%
\pgfpathlineto{\pgfqpoint{3.488650in}{0.913866in}}%
\pgfpathlineto{\pgfqpoint{3.490905in}{1.086304in}}%
\pgfpathlineto{\pgfqpoint{3.495417in}{1.922459in}}%
\pgfpathlineto{\pgfqpoint{3.502184in}{3.441426in}}%
\pgfpathlineto{\pgfqpoint{3.504439in}{3.733002in}}%
\pgfpathlineto{\pgfqpoint{3.506695in}{3.833425in}}%
\pgfpathlineto{\pgfqpoint{3.508951in}{3.729151in}}%
\pgfpathlineto{\pgfqpoint{3.511207in}{3.435387in}}%
\pgfpathlineto{\pgfqpoint{3.524741in}{0.946739in}}%
\pgfpathlineto{\pgfqpoint{3.526996in}{0.982705in}}%
\pgfpathlineto{\pgfqpoint{3.529252in}{1.213970in}}%
\pgfpathlineto{\pgfqpoint{3.533763in}{2.107661in}}%
\pgfpathlineto{\pgfqpoint{3.540530in}{3.529290in}}%
\pgfpathlineto{\pgfqpoint{3.542786in}{3.754645in}}%
\pgfpathlineto{\pgfqpoint{3.545042in}{3.785968in}}%
\pgfpathlineto{\pgfqpoint{3.547297in}{3.619462in}}%
\pgfpathlineto{\pgfqpoint{3.551809in}{2.813126in}}%
\pgfpathlineto{\pgfqpoint{3.558576in}{1.349680in}}%
\pgfpathlineto{\pgfqpoint{3.560831in}{1.069039in}}%
\pgfpathlineto{\pgfqpoint{3.563087in}{0.972569in}}%
\pgfpathlineto{\pgfqpoint{3.565343in}{1.073212in}}%
\pgfpathlineto{\pgfqpoint{3.567598in}{1.356223in}}%
\pgfpathlineto{\pgfqpoint{3.581132in}{3.750356in}}%
\pgfpathlineto{\pgfqpoint{3.583388in}{3.715534in}}%
\pgfpathlineto{\pgfqpoint{3.585644in}{3.492994in}}%
\pgfpathlineto{\pgfqpoint{3.590155in}{2.633847in}}%
\pgfpathlineto{\pgfqpoint{3.596922in}{1.268392in}}%
\pgfpathlineto{\pgfqpoint{3.599178in}{1.052198in}}%
\pgfpathlineto{\pgfqpoint{3.601433in}{1.022362in}}%
\pgfpathlineto{\pgfqpoint{3.603689in}{1.182428in}}%
\pgfpathlineto{\pgfqpoint{3.608200in}{1.956561in}}%
\pgfpathlineto{\pgfqpoint{3.614967in}{3.360267in}}%
\pgfpathlineto{\pgfqpoint{3.617223in}{3.629187in}}%
\pgfpathlineto{\pgfqpoint{3.619479in}{3.721449in}}%
\pgfpathlineto{\pgfqpoint{3.621735in}{3.624742in}}%
\pgfpathlineto{\pgfqpoint{3.623990in}{3.353298in}}%
\pgfpathlineto{\pgfqpoint{3.637524in}{1.060296in}}%
\pgfpathlineto{\pgfqpoint{3.639780in}{1.093860in}}%
\pgfpathlineto{\pgfqpoint{3.642036in}{1.307055in}}%
\pgfpathlineto{\pgfqpoint{3.646547in}{2.129332in}}%
\pgfpathlineto{\pgfqpoint{3.653314in}{3.435021in}}%
\pgfpathlineto{\pgfqpoint{3.655570in}{3.641507in}}%
\pgfpathlineto{\pgfqpoint{3.657825in}{3.669800in}}%
\pgfpathlineto{\pgfqpoint{3.660081in}{3.516606in}}%
\pgfpathlineto{\pgfqpoint{3.664592in}{2.776684in}}%
\pgfpathlineto{\pgfqpoint{3.671359in}{1.436246in}}%
\pgfpathlineto{\pgfqpoint{3.673615in}{1.179700in}}%
\pgfpathlineto{\pgfqpoint{3.675871in}{1.091854in}}%
\pgfpathlineto{\pgfqpoint{3.678126in}{1.184367in}}%
\pgfpathlineto{\pgfqpoint{3.680382in}{1.443563in}}%
\pgfpathlineto{\pgfqpoint{3.693916in}{3.629972in}}%
\pgfpathlineto{\pgfqpoint{3.696172in}{3.597766in}}%
\pgfpathlineto{\pgfqpoint{3.698427in}{3.394429in}}%
\pgfpathlineto{\pgfqpoint{3.702939in}{2.610928in}}%
\pgfpathlineto{\pgfqpoint{3.709706in}{1.367921in}}%
\pgfpathlineto{\pgfqpoint{3.711961in}{1.171582in}}%
\pgfpathlineto{\pgfqpoint{3.714217in}{1.144874in}}%
\pgfpathlineto{\pgfqpoint{3.716473in}{1.290839in}}%
\pgfpathlineto{\pgfqpoint{3.720984in}{1.994929in}}%
\pgfpathlineto{\pgfqpoint{3.727751in}{3.269279in}}%
\pgfpathlineto{\pgfqpoint{3.730007in}{3.512935in}}%
\pgfpathlineto{\pgfqpoint{3.732262in}{3.596205in}}%
\pgfpathlineto{\pgfqpoint{3.734518in}{3.508098in}}%
\pgfpathlineto{\pgfqpoint{3.736774in}{3.261694in}}%
\pgfpathlineto{\pgfqpoint{3.750308in}{1.186159in}}%
\pgfpathlineto{\pgfqpoint{3.752564in}{1.216924in}}%
\pgfpathlineto{\pgfqpoint{3.754819in}{1.409999in}}%
\pgfpathlineto{\pgfqpoint{3.759331in}{2.153245in}}%
\pgfpathlineto{\pgfqpoint{3.766098in}{3.331335in}}%
\pgfpathlineto{\pgfqpoint{3.768353in}{3.517198in}}%
\pgfpathlineto{\pgfqpoint{3.770609in}{3.542297in}}%
\pgfpathlineto{\pgfqpoint{3.772865in}{3.403836in}}%
\pgfpathlineto{\pgfqpoint{3.777376in}{2.736812in}}%
\pgfpathlineto{\pgfqpoint{3.784143in}{1.530661in}}%
\pgfpathlineto{\pgfqpoint{3.786399in}{1.300273in}}%
\pgfpathlineto{\pgfqpoint{3.788654in}{1.221691in}}%
\pgfpathlineto{\pgfqpoint{3.790910in}{1.305230in}}%
\pgfpathlineto{\pgfqpoint{3.793166in}{1.538434in}}%
\pgfpathlineto{\pgfqpoint{3.806700in}{3.499987in}}%
\pgfpathlineto{\pgfqpoint{3.808955in}{3.470730in}}%
\pgfpathlineto{\pgfqpoint{3.811211in}{3.288211in}}%
\pgfpathlineto{\pgfqpoint{3.815722in}{2.586276in}}%
\pgfpathlineto{\pgfqpoint{3.822489in}{1.474662in}}%
\pgfpathlineto{\pgfqpoint{3.824745in}{1.299495in}}%
\pgfpathlineto{\pgfqpoint{3.827001in}{1.276014in}}%
\pgfpathlineto{\pgfqpoint{3.829256in}{1.406776in}}%
\pgfpathlineto{\pgfqpoint{3.833768in}{2.035884in}}%
\pgfpathlineto{\pgfqpoint{3.840535in}{3.172427in}}%
\pgfpathlineto{\pgfqpoint{3.842790in}{3.389304in}}%
\pgfpathlineto{\pgfqpoint{3.845046in}{3.463133in}}%
\pgfpathlineto{\pgfqpoint{3.847302in}{3.384277in}}%
\pgfpathlineto{\pgfqpoint{3.849558in}{3.164543in}}%
\pgfpathlineto{\pgfqpoint{3.863092in}{1.318928in}}%
\pgfpathlineto{\pgfqpoint{3.865347in}{1.346626in}}%
\pgfpathlineto{\pgfqpoint{3.867603in}{1.518400in}}%
\pgfpathlineto{\pgfqpoint{3.872114in}{2.178383in}}%
\pgfpathlineto{\pgfqpoint{3.878881in}{3.222625in}}%
\pgfpathlineto{\pgfqpoint{3.881137in}{3.386979in}}%
\pgfpathlineto{\pgfqpoint{3.883393in}{3.408847in}}%
\pgfpathlineto{\pgfqpoint{3.885648in}{3.285905in}}%
\pgfpathlineto{\pgfqpoint{3.890160in}{2.695187in}}%
\pgfpathlineto{\pgfqpoint{3.896927in}{1.628981in}}%
\pgfpathlineto{\pgfqpoint{3.899182in}{1.425726in}}%
\pgfpathlineto{\pgfqpoint{3.901438in}{1.356670in}}%
\pgfpathlineto{\pgfqpoint{3.903694in}{1.430777in}}%
\pgfpathlineto{\pgfqpoint{3.905949in}{1.636901in}}%
\pgfpathlineto{\pgfqpoint{3.919483in}{3.365736in}}%
\pgfpathlineto{\pgfqpoint{3.921739in}{3.339631in}}%
\pgfpathlineto{\pgfqpoint{3.923995in}{3.178685in}}%
\pgfpathlineto{\pgfqpoint{3.928506in}{2.560897in}}%
\pgfpathlineto{\pgfqpoint{3.935273in}{1.584291in}}%
\pgfpathlineto{\pgfqpoint{3.937529in}{1.430765in}}%
\pgfpathlineto{\pgfqpoint{3.939784in}{1.410491in}}%
\pgfpathlineto{\pgfqpoint{3.942040in}{1.525569in}}%
\pgfpathlineto{\pgfqpoint{3.946551in}{2.077782in}}%
\pgfpathlineto{\pgfqpoint{3.953318in}{3.073572in}}%
\pgfpathlineto{\pgfqpoint{3.955574in}{3.263215in}}%
\pgfpathlineto{\pgfqpoint{3.957830in}{3.327519in}}%
\pgfpathlineto{\pgfqpoint{3.960085in}{3.258186in}}%
\pgfpathlineto{\pgfqpoint{3.962341in}{3.065686in}}%
\pgfpathlineto{\pgfqpoint{3.975875in}{1.453418in}}%
\pgfpathlineto{\pgfqpoint{3.978131in}{1.477911in}}%
\pgfpathlineto{\pgfqpoint{3.980387in}{1.628044in}}%
\pgfpathlineto{\pgfqpoint{3.984898in}{2.203770in}}%
\pgfpathlineto{\pgfqpoint{3.991665in}{3.113074in}}%
\pgfpathlineto{\pgfqpoint{3.993921in}{3.255849in}}%
\pgfpathlineto{\pgfqpoint{3.996176in}{3.274561in}}%
\pgfpathlineto{\pgfqpoint{3.998432in}{3.167322in}}%
\pgfpathlineto{\pgfqpoint{4.002943in}{2.653392in}}%
\pgfpathlineto{\pgfqpoint{4.009710in}{1.727489in}}%
\pgfpathlineto{\pgfqpoint{4.011966in}{1.551330in}}%
\pgfpathlineto{\pgfqpoint{4.014222in}{1.491716in}}%
\pgfpathlineto{\pgfqpoint{4.016477in}{1.556296in}}%
\pgfpathlineto{\pgfqpoint{4.018733in}{1.735276in}}%
\pgfpathlineto{\pgfqpoint{4.032267in}{3.232174in}}%
\pgfpathlineto{\pgfqpoint{4.034523in}{3.209296in}}%
\pgfpathlineto{\pgfqpoint{4.036778in}{3.069871in}}%
\pgfpathlineto{\pgfqpoint{4.041290in}{2.535719in}}%
\pgfpathlineto{\pgfqpoint{4.048057in}{1.692831in}}%
\pgfpathlineto{\pgfqpoint{4.050312in}{1.560643in}}%
\pgfpathlineto{\pgfqpoint{4.052568in}{1.543450in}}%
\pgfpathlineto{\pgfqpoint{4.054824in}{1.642941in}}%
\pgfpathlineto{\pgfqpoint{4.059335in}{2.119123in}}%
\pgfpathlineto{\pgfqpoint{4.066102in}{2.976229in}}%
\pgfpathlineto{\pgfqpoint{4.068358in}{3.139137in}}%
\pgfpathlineto{\pgfqpoint{4.070613in}{3.194158in}}%
\pgfpathlineto{\pgfqpoint{4.072869in}{3.134272in}}%
\pgfpathlineto{\pgfqpoint{4.075125in}{2.968601in}}%
\pgfpathlineto{\pgfqpoint{4.088659in}{1.584975in}}%
\pgfpathlineto{\pgfqpoint{4.090915in}{1.606250in}}%
\pgfpathlineto{\pgfqpoint{4.093170in}{1.735156in}}%
\pgfpathlineto{\pgfqpoint{4.097682in}{2.228539in}}%
\pgfpathlineto{\pgfqpoint{4.104449in}{3.006400in}}%
\pgfpathlineto{\pgfqpoint{4.106704in}{3.128244in}}%
\pgfpathlineto{\pgfqpoint{4.108960in}{3.143969in}}%
\pgfpathlineto{\pgfqpoint{4.111216in}{3.052076in}}%
\pgfpathlineto{\pgfqpoint{4.115727in}{2.612825in}}%
\pgfpathlineto{\pgfqpoint{4.122494in}{1.822920in}}%
\pgfpathlineto{\pgfqpoint{4.124750in}{1.672934in}}%
\pgfpathlineto{\pgfqpoint{4.127005in}{1.622378in}}%
\pgfpathlineto{\pgfqpoint{4.129261in}{1.677664in}}%
\pgfpathlineto{\pgfqpoint{4.131517in}{1.830337in}}%
\pgfpathlineto{\pgfqpoint{4.145051in}{3.103594in}}%
\pgfpathlineto{\pgfqpoint{4.147306in}{3.083900in}}%
\pgfpathlineto{\pgfqpoint{4.149562in}{2.965247in}}%
\pgfpathlineto{\pgfqpoint{4.154073in}{2.511540in}}%
\pgfpathlineto{\pgfqpoint{4.160840in}{1.796871in}}%
\pgfpathlineto{\pgfqpoint{4.163096in}{1.685060in}}%
\pgfpathlineto{\pgfqpoint{4.165352in}{1.670741in}}%
\pgfpathlineto{\pgfqpoint{4.167607in}{1.755241in}}%
\pgfpathlineto{\pgfqpoint{4.172119in}{2.158630in}}%
\pgfpathlineto{\pgfqpoint{4.178886in}{2.883376in}}%
\pgfpathlineto{\pgfqpoint{4.181141in}{3.020853in}}%
\pgfpathlineto{\pgfqpoint{4.183397in}{3.067100in}}%
\pgfpathlineto{\pgfqpoint{4.185653in}{3.016287in}}%
\pgfpathlineto{\pgfqpoint{4.187908in}{2.876216in}}%
\pgfpathlineto{\pgfqpoint{4.201443in}{1.709716in}}%
\pgfpathlineto{\pgfqpoint{4.203698in}{1.727865in}}%
\pgfpathlineto{\pgfqpoint{4.205954in}{1.836597in}}%
\pgfpathlineto{\pgfqpoint{4.210465in}{2.251970in}}%
\pgfpathlineto{\pgfqpoint{4.217232in}{2.905670in}}%
\pgfpathlineto{\pgfqpoint{4.219488in}{3.007820in}}%
\pgfpathlineto{\pgfqpoint{4.221744in}{3.020799in}}%
\pgfpathlineto{\pgfqpoint{4.223999in}{2.943442in}}%
\pgfpathlineto{\pgfqpoint{4.228511in}{2.574630in}}%
\pgfpathlineto{\pgfqpoint{4.235278in}{1.912614in}}%
\pgfpathlineto{\pgfqpoint{4.237533in}{1.787160in}}%
\pgfpathlineto{\pgfqpoint{4.239789in}{1.745043in}}%
\pgfpathlineto{\pgfqpoint{4.242045in}{1.791538in}}%
\pgfpathlineto{\pgfqpoint{4.246556in}{2.110350in}}%
\pgfpathlineto{\pgfqpoint{4.255579in}{2.914566in}}%
\pgfpathlineto{\pgfqpoint{4.257834in}{2.983437in}}%
\pgfpathlineto{\pgfqpoint{4.260090in}{2.966785in}}%
\pgfpathlineto{\pgfqpoint{4.262346in}{2.867587in}}%
\pgfpathlineto{\pgfqpoint{4.266857in}{2.488995in}}%
\pgfpathlineto{\pgfqpoint{4.273624in}{1.893712in}}%
\pgfpathlineto{\pgfqpoint{4.275880in}{1.800801in}}%
\pgfpathlineto{\pgfqpoint{4.278135in}{1.789090in}}%
\pgfpathlineto{\pgfqpoint{4.280391in}{1.859593in}}%
\pgfpathlineto{\pgfqpoint{4.284902in}{2.195299in}}%
\pgfpathlineto{\pgfqpoint{4.291669in}{2.797335in}}%
\pgfpathlineto{\pgfqpoint{4.293925in}{2.911308in}}%
\pgfpathlineto{\pgfqpoint{4.296181in}{2.949495in}}%
\pgfpathlineto{\pgfqpoint{4.298436in}{2.907140in}}%
\pgfpathlineto{\pgfqpoint{4.302948in}{2.617328in}}%
\pgfpathlineto{\pgfqpoint{4.311971in}{1.887091in}}%
\pgfpathlineto{\pgfqpoint{4.314226in}{1.824667in}}%
\pgfpathlineto{\pgfqpoint{4.316482in}{1.839876in}}%
\pgfpathlineto{\pgfqpoint{4.318738in}{1.929975in}}%
\pgfpathlineto{\pgfqpoint{4.323249in}{2.273514in}}%
\pgfpathlineto{\pgfqpoint{4.330016in}{2.813199in}}%
\pgfpathlineto{\pgfqpoint{4.332272in}{2.897331in}}%
\pgfpathlineto{\pgfqpoint{4.334527in}{2.907851in}}%
\pgfpathlineto{\pgfqpoint{4.336783in}{2.843878in}}%
\pgfpathlineto{\pgfqpoint{4.341294in}{2.539662in}}%
\pgfpathlineto{\pgfqpoint{4.348061in}{1.994598in}}%
\pgfpathlineto{\pgfqpoint{4.350317in}{1.891513in}}%
\pgfpathlineto{\pgfqpoint{4.352573in}{1.857044in}}%
\pgfpathlineto{\pgfqpoint{4.354828in}{1.895456in}}%
\pgfpathlineto{\pgfqpoint{4.359340in}{2.157738in}}%
\pgfpathlineto{\pgfqpoint{4.368362in}{2.817865in}}%
\pgfpathlineto{\pgfqpoint{4.370618in}{2.874193in}}%
\pgfpathlineto{\pgfqpoint{4.372874in}{2.860364in}}%
\pgfpathlineto{\pgfqpoint{4.375129in}{2.778892in}}%
\pgfpathlineto{\pgfqpoint{4.379641in}{2.468542in}}%
\pgfpathlineto{\pgfqpoint{4.386408in}{1.981431in}}%
\pgfpathlineto{\pgfqpoint{4.388663in}{1.905586in}}%
\pgfpathlineto{\pgfqpoint{4.390919in}{1.896179in}}%
\pgfpathlineto{\pgfqpoint{4.393175in}{1.953968in}}%
\pgfpathlineto{\pgfqpoint{4.397686in}{2.228428in}}%
\pgfpathlineto{\pgfqpoint{4.404453in}{2.719723in}}%
\pgfpathlineto{\pgfqpoint{4.406709in}{2.812548in}}%
\pgfpathlineto{\pgfqpoint{4.408964in}{2.843523in}}%
\pgfpathlineto{\pgfqpoint{4.411220in}{2.808841in}}%
\pgfpathlineto{\pgfqpoint{4.415732in}{2.572525in}}%
\pgfpathlineto{\pgfqpoint{4.424754in}{1.978423in}}%
\pgfpathlineto{\pgfqpoint{4.427010in}{1.927820in}}%
\pgfpathlineto{\pgfqpoint{4.429266in}{1.940338in}}%
\pgfpathlineto{\pgfqpoint{4.431521in}{2.013682in}}%
\pgfpathlineto{\pgfqpoint{4.436033in}{2.292807in}}%
\pgfpathlineto{\pgfqpoint{4.442800in}{2.730518in}}%
\pgfpathlineto{\pgfqpoint{4.445055in}{2.798589in}}%
\pgfpathlineto{\pgfqpoint{4.447311in}{2.806963in}}%
\pgfpathlineto{\pgfqpoint{4.449567in}{2.754991in}}%
\pgfpathlineto{\pgfqpoint{4.454078in}{2.508475in}}%
\pgfpathlineto{\pgfqpoint{4.460845in}{2.067606in}}%
\pgfpathlineto{\pgfqpoint{4.463101in}{1.984391in}}%
\pgfpathlineto{\pgfqpoint{4.465356in}{1.956679in}}%
\pgfpathlineto{\pgfqpoint{4.467612in}{1.987855in}}%
\pgfpathlineto{\pgfqpoint{4.472123in}{2.199831in}}%
\pgfpathlineto{\pgfqpoint{4.481146in}{2.732143in}}%
\pgfpathlineto{\pgfqpoint{4.483402in}{2.777401in}}%
\pgfpathlineto{\pgfqpoint{4.485657in}{2.766120in}}%
\pgfpathlineto{\pgfqpoint{4.487913in}{2.700385in}}%
\pgfpathlineto{\pgfqpoint{4.492424in}{2.450457in}}%
\pgfpathlineto{\pgfqpoint{4.499191in}{2.058880in}}%
\pgfpathlineto{\pgfqpoint{4.501447in}{1.998057in}}%
\pgfpathlineto{\pgfqpoint{4.503703in}{1.990635in}}%
\pgfpathlineto{\pgfqpoint{4.505958in}{2.037169in}}%
\pgfpathlineto{\pgfqpoint{4.510470in}{2.257605in}}%
\pgfpathlineto{\pgfqpoint{4.517237in}{2.651471in}}%
\pgfpathlineto{\pgfqpoint{4.519492in}{2.725739in}}%
\pgfpathlineto{\pgfqpoint{4.521748in}{2.750422in}}%
\pgfpathlineto{\pgfqpoint{4.524004in}{2.722523in}}%
\pgfpathlineto{\pgfqpoint{4.528515in}{2.533222in}}%
\pgfpathlineto{\pgfqpoint{4.537538in}{2.058386in}}%
\pgfpathlineto{\pgfqpoint{4.539794in}{2.018088in}}%
\pgfpathlineto{\pgfqpoint{4.542049in}{2.028209in}}%
\pgfpathlineto{\pgfqpoint{4.544305in}{2.086863in}}%
\pgfpathlineto{\pgfqpoint{4.548816in}{2.309657in}}%
\pgfpathlineto{\pgfqpoint{4.555583in}{2.658411in}}%
\pgfpathlineto{\pgfqpoint{4.557839in}{2.712517in}}%
\pgfpathlineto{\pgfqpoint{4.560095in}{2.719064in}}%
\pgfpathlineto{\pgfqpoint{4.562350in}{2.677585in}}%
\pgfpathlineto{\pgfqpoint{4.566862in}{2.481343in}}%
\pgfpathlineto{\pgfqpoint{4.573629in}{2.131029in}}%
\pgfpathlineto{\pgfqpoint{4.575884in}{2.065038in}}%
\pgfpathlineto{\pgfqpoint{4.578140in}{2.043152in}}%
\pgfpathlineto{\pgfqpoint{4.580396in}{2.068007in}}%
\pgfpathlineto{\pgfqpoint{4.584907in}{2.236309in}}%
\pgfpathlineto{\pgfqpoint{4.593930in}{2.657997in}}%
\pgfpathlineto{\pgfqpoint{4.596185in}{2.693720in}}%
\pgfpathlineto{\pgfqpoint{4.598441in}{2.684681in}}%
\pgfpathlineto{\pgfqpoint{4.600697in}{2.632577in}}%
\pgfpathlineto{\pgfqpoint{4.605208in}{2.434851in}}%
\pgfpathlineto{\pgfqpoint{4.611975in}{2.125614in}}%
\pgfpathlineto{\pgfqpoint{4.614231in}{2.077697in}}%
\pgfpathlineto{\pgfqpoint{4.616486in}{2.071947in}}%
\pgfpathlineto{\pgfqpoint{4.618742in}{2.108757in}}%
\pgfpathlineto{\pgfqpoint{4.623253in}{2.282686in}}%
\pgfpathlineto{\pgfqpoint{4.630020in}{2.592884in}}%
\pgfpathlineto{\pgfqpoint{4.632276in}{2.651258in}}%
\pgfpathlineto{\pgfqpoint{4.634532in}{2.670580in}}%
\pgfpathlineto{\pgfqpoint{4.636787in}{2.648534in}}%
\pgfpathlineto{\pgfqpoint{4.641299in}{2.499564in}}%
\pgfpathlineto{\pgfqpoint{4.650322in}{2.126735in}}%
\pgfpathlineto{\pgfqpoint{4.652577in}{2.095208in}}%
\pgfpathlineto{\pgfqpoint{4.654833in}{2.103245in}}%
\pgfpathlineto{\pgfqpoint{4.657089in}{2.149325in}}%
\pgfpathlineto{\pgfqpoint{4.661600in}{2.324026in}}%
\pgfpathlineto{\pgfqpoint{4.668367in}{2.597009in}}%
\pgfpathlineto{\pgfqpoint{4.670623in}{2.639257in}}%
\pgfpathlineto{\pgfqpoint{4.672878in}{2.644284in}}%
\pgfpathlineto{\pgfqpoint{4.675134in}{2.611762in}}%
\pgfpathlineto{\pgfqpoint{4.679645in}{2.458293in}}%
\pgfpathlineto{\pgfqpoint{4.686412in}{2.184835in}}%
\pgfpathlineto{\pgfqpoint{4.688668in}{2.133425in}}%
\pgfpathlineto{\pgfqpoint{4.690924in}{2.116444in}}%
\pgfpathlineto{\pgfqpoint{4.693179in}{2.135912in}}%
\pgfpathlineto{\pgfqpoint{4.697691in}{2.267185in}}%
\pgfpathlineto{\pgfqpoint{4.706713in}{2.595356in}}%
\pgfpathlineto{\pgfqpoint{4.708969in}{2.623056in}}%
\pgfpathlineto{\pgfqpoint{4.711225in}{2.615942in}}%
\pgfpathlineto{\pgfqpoint{4.713480in}{2.575370in}}%
\pgfpathlineto{\pgfqpoint{4.717992in}{2.421697in}}%
\pgfpathlineto{\pgfqpoint{4.724759in}{2.181786in}}%
\pgfpathlineto{\pgfqpoint{4.727014in}{2.144701in}}%
\pgfpathlineto{\pgfqpoint{4.729270in}{2.140327in}}%
\pgfpathlineto{\pgfqpoint{4.731526in}{2.168932in}}%
\pgfpathlineto{\pgfqpoint{4.736037in}{2.303748in}}%
\pgfpathlineto{\pgfqpoint{4.742804in}{2.543750in}}%
\pgfpathlineto{\pgfqpoint{4.745060in}{2.588825in}}%
\pgfpathlineto{\pgfqpoint{4.747315in}{2.603683in}}%
\pgfpathlineto{\pgfqpoint{4.749571in}{2.586568in}}%
\pgfpathlineto{\pgfqpoint{4.754082in}{2.471402in}}%
\pgfpathlineto{\pgfqpoint{4.763105in}{2.183820in}}%
\pgfpathlineto{\pgfqpoint{4.765361in}{2.159590in}}%
\pgfpathlineto{\pgfqpoint{4.767617in}{2.165859in}}%
\pgfpathlineto{\pgfqpoint{4.769872in}{2.201423in}}%
\pgfpathlineto{\pgfqpoint{4.774384in}{2.336000in}}%
\pgfpathlineto{\pgfqpoint{4.781151in}{2.545911in}}%
\pgfpathlineto{\pgfqpoint{4.783406in}{2.578320in}}%
\pgfpathlineto{\pgfqpoint{4.785662in}{2.582110in}}%
\pgfpathlineto{\pgfqpoint{4.787918in}{2.557061in}}%
\pgfpathlineto{\pgfqpoint{4.792429in}{2.439154in}}%
\pgfpathlineto{\pgfqpoint{4.799196in}{2.229450in}}%
\pgfpathlineto{\pgfqpoint{4.801452in}{2.190104in}}%
\pgfpathlineto{\pgfqpoint{4.803707in}{2.177161in}}%
\pgfpathlineto{\pgfqpoint{4.805963in}{2.192140in}}%
\pgfpathlineto{\pgfqpoint{4.810474in}{2.292729in}}%
\pgfpathlineto{\pgfqpoint{4.819497in}{2.543625in}}%
\pgfpathlineto{\pgfqpoint{4.821753in}{2.564725in}}%
\pgfpathlineto{\pgfqpoint{4.824008in}{2.559226in}}%
\pgfpathlineto{\pgfqpoint{4.826264in}{2.528190in}}%
\pgfpathlineto{\pgfqpoint{4.830775in}{2.410858in}}%
\pgfpathlineto{\pgfqpoint{4.837542in}{2.228009in}}%
\pgfpathlineto{\pgfqpoint{4.839798in}{2.199813in}}%
\pgfpathlineto{\pgfqpoint{4.842054in}{2.196544in}}%
\pgfpathlineto{\pgfqpoint{4.844309in}{2.218382in}}%
\pgfpathlineto{\pgfqpoint{4.848821in}{2.321042in}}%
\pgfpathlineto{\pgfqpoint{4.855588in}{2.503463in}}%
\pgfpathlineto{\pgfqpoint{4.857843in}{2.537655in}}%
\pgfpathlineto{\pgfqpoint{4.860099in}{2.548879in}}%
\pgfpathlineto{\pgfqpoint{4.862355in}{2.535827in}}%
\pgfpathlineto{\pgfqpoint{4.866866in}{2.448361in}}%
\pgfpathlineto{\pgfqpoint{4.875889in}{2.230441in}}%
\pgfpathlineto{\pgfqpoint{4.878145in}{2.212147in}}%
\pgfpathlineto{\pgfqpoint{4.880400in}{2.216950in}}%
\pgfpathlineto{\pgfqpoint{4.882656in}{2.243914in}}%
\pgfpathlineto{\pgfqpoint{4.887167in}{2.345757in}}%
\pgfpathlineto{\pgfqpoint{4.893934in}{2.504327in}}%
\pgfpathlineto{\pgfqpoint{4.896190in}{2.528750in}}%
\pgfpathlineto{\pgfqpoint{4.898446in}{2.531556in}}%
\pgfpathlineto{\pgfqpoint{4.900701in}{2.512602in}}%
\pgfpathlineto{\pgfqpoint{4.905213in}{2.423613in}}%
\pgfpathlineto{\pgfqpoint{4.911980in}{2.265630in}}%
\pgfpathlineto{\pgfqpoint{4.914235in}{2.236048in}}%
\pgfpathlineto{\pgfqpoint{4.916491in}{2.226357in}}%
\pgfpathlineto{\pgfqpoint{4.918747in}{2.237680in}}%
\pgfpathlineto{\pgfqpoint{4.923258in}{2.313398in}}%
\pgfpathlineto{\pgfqpoint{4.932281in}{2.501838in}}%
\pgfpathlineto{\pgfqpoint{4.934536in}{2.517628in}}%
\pgfpathlineto{\pgfqpoint{4.936792in}{2.513452in}}%
\pgfpathlineto{\pgfqpoint{4.939048in}{2.490129in}}%
\pgfpathlineto{\pgfqpoint{4.943559in}{2.402122in}}%
\pgfpathlineto{\pgfqpoint{4.950326in}{2.265217in}}%
\pgfpathlineto{\pgfqpoint{4.952582in}{2.244156in}}%
\pgfpathlineto{\pgfqpoint{4.954837in}{2.241758in}}%
\pgfpathlineto{\pgfqpoint{4.957093in}{2.258136in}}%
\pgfpathlineto{\pgfqpoint{4.961604in}{2.334932in}}%
\pgfpathlineto{\pgfqpoint{4.968371in}{2.471145in}}%
\pgfpathlineto{\pgfqpoint{4.970627in}{2.496625in}}%
\pgfpathlineto{\pgfqpoint{4.972883in}{2.504955in}}%
\pgfpathlineto{\pgfqpoint{4.975138in}{2.495177in}}%
\pgfpathlineto{\pgfqpoint{4.979650in}{2.429919in}}%
\pgfpathlineto{\pgfqpoint{4.988673in}{2.267693in}}%
\pgfpathlineto{\pgfqpoint{4.990928in}{2.254125in}}%
\pgfpathlineto{\pgfqpoint{4.993184in}{2.257739in}}%
\pgfpathlineto{\pgfqpoint{4.995440in}{2.277823in}}%
\pgfpathlineto{\pgfqpoint{4.999951in}{2.353537in}}%
\pgfpathlineto{\pgfqpoint{5.006718in}{2.471214in}}%
\pgfpathlineto{\pgfqpoint{5.008974in}{2.489295in}}%
\pgfpathlineto{\pgfqpoint{5.011229in}{2.491335in}}%
\pgfpathlineto{\pgfqpoint{5.013485in}{2.477246in}}%
\pgfpathlineto{\pgfqpoint{5.017996in}{2.411265in}}%
\pgfpathlineto{\pgfqpoint{5.024763in}{2.294342in}}%
\pgfpathlineto{\pgfqpoint{5.027019in}{2.272493in}}%
\pgfpathlineto{\pgfqpoint{5.029275in}{2.265365in}}%
\pgfpathlineto{\pgfqpoint{5.031530in}{2.273772in}}%
\pgfpathlineto{\pgfqpoint{5.036042in}{2.329766in}}%
\pgfpathlineto{\pgfqpoint{5.045064in}{2.468805in}}%
\pgfpathlineto{\pgfqpoint{5.047320in}{2.480413in}}%
\pgfpathlineto{\pgfqpoint{5.049576in}{2.477298in}}%
\pgfpathlineto{\pgfqpoint{5.051831in}{2.460080in}}%
\pgfpathlineto{\pgfqpoint{5.056343in}{2.395231in}}%
\pgfpathlineto{\pgfqpoint{5.063110in}{2.294530in}}%
\pgfpathlineto{\pgfqpoint{5.065365in}{2.279076in}}%
\pgfpathlineto{\pgfqpoint{5.067621in}{2.277349in}}%
\pgfpathlineto{\pgfqpoint{5.069877in}{2.289415in}}%
\pgfpathlineto{\pgfqpoint{5.074388in}{2.345852in}}%
\pgfpathlineto{\pgfqpoint{5.081155in}{2.445771in}}%
\pgfpathlineto{\pgfqpoint{5.083411in}{2.464425in}}%
\pgfpathlineto{\pgfqpoint{5.085666in}{2.470497in}}%
\pgfpathlineto{\pgfqpoint{5.087922in}{2.463301in}}%
\pgfpathlineto{\pgfqpoint{5.092433in}{2.415469in}}%
\pgfpathlineto{\pgfqpoint{5.101456in}{2.296831in}}%
\pgfpathlineto{\pgfqpoint{5.103712in}{2.286944in}}%
\pgfpathlineto{\pgfqpoint{5.105968in}{2.289616in}}%
\pgfpathlineto{\pgfqpoint{5.108223in}{2.304312in}}%
\pgfpathlineto{\pgfqpoint{5.112735in}{2.359609in}}%
\pgfpathlineto{\pgfqpoint{5.119502in}{2.445401in}}%
\pgfpathlineto{\pgfqpoint{5.121757in}{2.458551in}}%
\pgfpathlineto{\pgfqpoint{5.124013in}{2.460007in}}%
\pgfpathlineto{\pgfqpoint{5.126269in}{2.449719in}}%
\pgfpathlineto{\pgfqpoint{5.130780in}{2.401659in}}%
\pgfpathlineto{\pgfqpoint{5.137547in}{2.316649in}}%
\pgfpathlineto{\pgfqpoint{5.139803in}{2.300795in}}%
\pgfpathlineto{\pgfqpoint{5.142058in}{2.295644in}}%
\pgfpathlineto{\pgfqpoint{5.144314in}{2.301777in}}%
\pgfpathlineto{\pgfqpoint{5.148825in}{2.342455in}}%
\pgfpathlineto{\pgfqpoint{5.157848in}{2.443238in}}%
\pgfpathlineto{\pgfqpoint{5.160104in}{2.451621in}}%
\pgfpathlineto{\pgfqpoint{5.162359in}{2.449339in}}%
\pgfpathlineto{\pgfqpoint{5.164615in}{2.436852in}}%
\pgfpathlineto{\pgfqpoint{5.169126in}{2.389908in}}%
\pgfpathlineto{\pgfqpoint{5.175893in}{2.317142in}}%
\pgfpathlineto{\pgfqpoint{5.178149in}{2.306002in}}%
\pgfpathlineto{\pgfqpoint{5.180405in}{2.304780in}}%
\pgfpathlineto{\pgfqpoint{5.182660in}{2.313513in}}%
\pgfpathlineto{\pgfqpoint{5.187172in}{2.354259in}}%
\pgfpathlineto{\pgfqpoint{5.193939in}{2.426263in}}%
\pgfpathlineto{\pgfqpoint{5.196194in}{2.439679in}}%
\pgfpathlineto{\pgfqpoint{5.198450in}{2.444028in}}%
\pgfpathlineto{\pgfqpoint{5.200706in}{2.438825in}}%
\pgfpathlineto{\pgfqpoint{5.205217in}{2.404383in}}%
\pgfpathlineto{\pgfqpoint{5.214240in}{2.319148in}}%
\pgfpathlineto{\pgfqpoint{5.216496in}{2.312072in}}%
\pgfpathlineto{\pgfqpoint{5.218751in}{2.314012in}}%
\pgfpathlineto{\pgfqpoint{5.221007in}{2.324575in}}%
\pgfpathlineto{\pgfqpoint{5.225518in}{2.364251in}}%
\pgfpathlineto{\pgfqpoint{5.232285in}{2.425696in}}%
\pgfpathlineto{\pgfqpoint{5.234541in}{2.435091in}}%
\pgfpathlineto{\pgfqpoint{5.236797in}{2.436112in}}%
\pgfpathlineto{\pgfqpoint{5.239052in}{2.428731in}}%
\pgfpathlineto{\pgfqpoint{5.243564in}{2.394341in}}%
\pgfpathlineto{\pgfqpoint{5.250331in}{2.333621in}}%
\pgfpathlineto{\pgfqpoint{5.252586in}{2.322320in}}%
\pgfpathlineto{\pgfqpoint{5.254842in}{2.318664in}}%
\pgfpathlineto{\pgfqpoint{5.257098in}{2.323059in}}%
\pgfpathlineto{\pgfqpoint{5.261609in}{2.352091in}}%
\pgfpathlineto{\pgfqpoint{5.270632in}{2.423857in}}%
\pgfpathlineto{\pgfqpoint{5.272887in}{2.429804in}}%
\pgfpathlineto{\pgfqpoint{5.275143in}{2.428162in}}%
\pgfpathlineto{\pgfqpoint{5.277399in}{2.419265in}}%
\pgfpathlineto{\pgfqpoint{5.281910in}{2.385881in}}%
\pgfpathlineto{\pgfqpoint{5.288677in}{2.334227in}}%
\pgfpathlineto{\pgfqpoint{5.290933in}{2.326338in}}%
\pgfpathlineto{\pgfqpoint{5.293188in}{2.325489in}}%
\pgfpathlineto{\pgfqpoint{5.295444in}{2.331698in}}%
\pgfpathlineto{\pgfqpoint{5.299955in}{2.360597in}}%
\pgfpathlineto{\pgfqpoint{5.306722in}{2.411572in}}%
\pgfpathlineto{\pgfqpoint{5.308978in}{2.421050in}}%
\pgfpathlineto{\pgfqpoint{5.311234in}{2.424110in}}%
\pgfpathlineto{\pgfqpoint{5.313489in}{2.420415in}}%
\pgfpathlineto{\pgfqpoint{5.318001in}{2.396052in}}%
\pgfpathlineto{\pgfqpoint{5.327024in}{2.335894in}}%
\pgfpathlineto{\pgfqpoint{5.329279in}{2.330917in}}%
\pgfpathlineto{\pgfqpoint{5.331535in}{2.332301in}}%
\pgfpathlineto{\pgfqpoint{5.333791in}{2.339761in}}%
\pgfpathlineto{\pgfqpoint{5.338302in}{2.367726in}}%
\pgfpathlineto{\pgfqpoint{5.345069in}{2.410959in}}%
\pgfpathlineto{\pgfqpoint{5.347325in}{2.417553in}}%
\pgfpathlineto{\pgfqpoint{5.349580in}{2.418256in}}%
\pgfpathlineto{\pgfqpoint{5.351836in}{2.413055in}}%
\pgfpathlineto{\pgfqpoint{5.356347in}{2.388879in}}%
\pgfpathlineto{\pgfqpoint{5.363114in}{2.346273in}}%
\pgfpathlineto{\pgfqpoint{5.365370in}{2.338359in}}%
\pgfpathlineto{\pgfqpoint{5.367626in}{2.335810in}}%
\pgfpathlineto{\pgfqpoint{5.369881in}{2.338904in}}%
\pgfpathlineto{\pgfqpoint{5.374393in}{2.359258in}}%
\pgfpathlineto{\pgfqpoint{5.383415in}{2.409462in}}%
\pgfpathlineto{\pgfqpoint{5.385671in}{2.413608in}}%
\pgfpathlineto{\pgfqpoint{5.387927in}{2.412447in}}%
\pgfpathlineto{\pgfqpoint{5.390182in}{2.406220in}}%
\pgfpathlineto{\pgfqpoint{5.394694in}{2.382897in}}%
\pgfpathlineto{\pgfqpoint{5.401461in}{2.346874in}}%
\pgfpathlineto{\pgfqpoint{5.403716in}{2.341386in}}%
\pgfpathlineto{\pgfqpoint{5.405972in}{2.340807in}}%
\pgfpathlineto{\pgfqpoint{5.408228in}{2.345144in}}%
\pgfpathlineto{\pgfqpoint{5.412739in}{2.365279in}}%
\pgfpathlineto{\pgfqpoint{5.419506in}{2.400732in}}%
\pgfpathlineto{\pgfqpoint{5.421762in}{2.407310in}}%
\pgfpathlineto{\pgfqpoint{5.424017in}{2.409425in}}%
\pgfpathlineto{\pgfqpoint{5.426273in}{2.406847in}}%
\pgfpathlineto{\pgfqpoint{5.430784in}{2.389916in}}%
\pgfpathlineto{\pgfqpoint{5.439807in}{2.348205in}}%
\pgfpathlineto{\pgfqpoint{5.442063in}{2.344767in}}%
\pgfpathlineto{\pgfqpoint{5.444319in}{2.345737in}}%
\pgfpathlineto{\pgfqpoint{5.446574in}{2.350912in}}%
\pgfpathlineto{\pgfqpoint{5.451086in}{2.370276in}}%
\pgfpathlineto{\pgfqpoint{5.457853in}{2.400159in}}%
\pgfpathlineto{\pgfqpoint{5.460108in}{2.404706in}}%
\pgfpathlineto{\pgfqpoint{5.462364in}{2.405181in}}%
\pgfpathlineto{\pgfqpoint{5.464620in}{2.401580in}}%
\pgfpathlineto{\pgfqpoint{5.469131in}{2.384884in}}%
\pgfpathlineto{\pgfqpoint{5.475898in}{2.355515in}}%
\pgfpathlineto{\pgfqpoint{5.478154in}{2.350071in}}%
\pgfpathlineto{\pgfqpoint{5.480409in}{2.348325in}}%
\pgfpathlineto{\pgfqpoint{5.482665in}{2.350464in}}%
\pgfpathlineto{\pgfqpoint{5.487176in}{2.364484in}}%
\pgfpathlineto{\pgfqpoint{5.496199in}{2.398985in}}%
\pgfpathlineto{\pgfqpoint{5.498455in}{2.401824in}}%
\pgfpathlineto{\pgfqpoint{5.500710in}{2.401018in}}%
\pgfpathlineto{\pgfqpoint{5.502966in}{2.396736in}}%
\pgfpathlineto{\pgfqpoint{5.507477in}{2.380729in}}%
\pgfpathlineto{\pgfqpoint{5.514244in}{2.356050in}}%
\pgfpathlineto{\pgfqpoint{5.516500in}{2.352300in}}%
\pgfpathlineto{\pgfqpoint{5.518756in}{2.351911in}}%
\pgfpathlineto{\pgfqpoint{5.521011in}{2.354888in}}%
\pgfpathlineto{\pgfqpoint{5.525523in}{2.368670in}}%
\pgfpathlineto{\pgfqpoint{5.532290in}{2.392892in}}%
\pgfpathlineto{\pgfqpoint{5.534545in}{2.397378in}}%
\pgfpathlineto{\pgfqpoint{5.534545in}{2.397378in}}%
\pgfusepath{stroke}%
\end{pgfscope}%
\begin{pgfscope}%
\pgfpathrectangle{\pgfqpoint{0.800000in}{0.528000in}}{\pgfqpoint{4.960000in}{3.696000in}}%
\pgfusepath{clip}%
\pgfsetrectcap%
\pgfsetroundjoin%
\pgfsetlinewidth{1.505625pt}%
\definecolor{currentstroke}{rgb}{0.156863,0.411765,0.513725}%
\pgfsetstrokecolor{currentstroke}%
\pgfsetdash{}{0pt}%
\pgfpathmoveto{\pgfqpoint{1.025455in}{2.470497in}}%
\pgfpathlineto{\pgfqpoint{1.084102in}{2.487325in}}%
\pgfpathlineto{\pgfqpoint{1.140494in}{2.505736in}}%
\pgfpathlineto{\pgfqpoint{1.196886in}{2.526522in}}%
\pgfpathlineto{\pgfqpoint{1.251022in}{2.548879in}}%
\pgfpathlineto{\pgfqpoint{1.302902in}{2.572658in}}%
\pgfpathlineto{\pgfqpoint{1.354783in}{2.598868in}}%
\pgfpathlineto{\pgfqpoint{1.406663in}{2.627623in}}%
\pgfpathlineto{\pgfqpoint{1.458544in}{2.659023in}}%
\pgfpathlineto{\pgfqpoint{1.510424in}{2.693147in}}%
\pgfpathlineto{\pgfqpoint{1.562305in}{2.730052in}}%
\pgfpathlineto{\pgfqpoint{1.614185in}{2.769768in}}%
\pgfpathlineto{\pgfqpoint{1.666066in}{2.812296in}}%
\pgfpathlineto{\pgfqpoint{1.720202in}{2.859635in}}%
\pgfpathlineto{\pgfqpoint{1.774338in}{2.909921in}}%
\pgfpathlineto{\pgfqpoint{1.830730in}{2.965299in}}%
\pgfpathlineto{\pgfqpoint{1.891633in}{3.028314in}}%
\pgfpathlineto{\pgfqpoint{1.957048in}{3.099336in}}%
\pgfpathlineto{\pgfqpoint{2.029229in}{3.181109in}}%
\pgfpathlineto{\pgfqpoint{2.114945in}{3.281703in}}%
\pgfpathlineto{\pgfqpoint{2.419461in}{3.642479in}}%
\pgfpathlineto{\pgfqpoint{2.482619in}{3.711846in}}%
\pgfpathlineto{\pgfqpoint{2.536756in}{3.768133in}}%
\pgfpathlineto{\pgfqpoint{2.586380in}{3.816621in}}%
\pgfpathlineto{\pgfqpoint{2.631494in}{3.857736in}}%
\pgfpathlineto{\pgfqpoint{2.674352in}{3.893880in}}%
\pgfpathlineto{\pgfqpoint{2.714954in}{3.925265in}}%
\pgfpathlineto{\pgfqpoint{2.753300in}{3.952166in}}%
\pgfpathlineto{\pgfqpoint{2.789391in}{3.974908in}}%
\pgfpathlineto{\pgfqpoint{2.825482in}{3.995031in}}%
\pgfpathlineto{\pgfqpoint{2.859317in}{4.011422in}}%
\pgfpathlineto{\pgfqpoint{2.893152in}{4.025339in}}%
\pgfpathlineto{\pgfqpoint{2.924731in}{4.036039in}}%
\pgfpathlineto{\pgfqpoint{2.956311in}{4.044483in}}%
\pgfpathlineto{\pgfqpoint{2.987890in}{4.050636in}}%
\pgfpathlineto{\pgfqpoint{3.019470in}{4.054473in}}%
\pgfpathlineto{\pgfqpoint{3.051049in}{4.055976in}}%
\pgfpathlineto{\pgfqpoint{3.082629in}{4.055141in}}%
\pgfpathlineto{\pgfqpoint{3.114208in}{4.051970in}}%
\pgfpathlineto{\pgfqpoint{3.145787in}{4.046476in}}%
\pgfpathlineto{\pgfqpoint{3.177367in}{4.038684in}}%
\pgfpathlineto{\pgfqpoint{3.208946in}{4.028624in}}%
\pgfpathlineto{\pgfqpoint{3.240526in}{4.016339in}}%
\pgfpathlineto{\pgfqpoint{3.274361in}{4.000765in}}%
\pgfpathlineto{\pgfqpoint{3.308196in}{3.982767in}}%
\pgfpathlineto{\pgfqpoint{3.344287in}{3.960994in}}%
\pgfpathlineto{\pgfqpoint{3.380377in}{3.936674in}}%
\pgfpathlineto{\pgfqpoint{3.418724in}{3.908185in}}%
\pgfpathlineto{\pgfqpoint{3.459326in}{3.875228in}}%
\pgfpathlineto{\pgfqpoint{3.502184in}{3.837553in}}%
\pgfpathlineto{\pgfqpoint{3.547297in}{3.794980in}}%
\pgfpathlineto{\pgfqpoint{3.596922in}{3.745078in}}%
\pgfpathlineto{\pgfqpoint{3.651058in}{3.687485in}}%
\pgfpathlineto{\pgfqpoint{3.714217in}{3.616911in}}%
\pgfpathlineto{\pgfqpoint{3.790910in}{3.527642in}}%
\pgfpathlineto{\pgfqpoint{3.908205in}{3.387181in}}%
\pgfpathlineto{\pgfqpoint{4.050312in}{3.217804in}}%
\pgfpathlineto{\pgfqpoint{4.131517in}{3.124541in}}%
\pgfpathlineto{\pgfqpoint{4.201443in}{3.047580in}}%
\pgfpathlineto{\pgfqpoint{4.264601in}{2.981328in}}%
\pgfpathlineto{\pgfqpoint{4.323249in}{2.922939in}}%
\pgfpathlineto{\pgfqpoint{4.379641in}{2.869872in}}%
\pgfpathlineto{\pgfqpoint{4.433777in}{2.821910in}}%
\pgfpathlineto{\pgfqpoint{4.487913in}{2.776962in}}%
\pgfpathlineto{\pgfqpoint{4.539794in}{2.736756in}}%
\pgfpathlineto{\pgfqpoint{4.591674in}{2.699364in}}%
\pgfpathlineto{\pgfqpoint{4.643555in}{2.664760in}}%
\pgfpathlineto{\pgfqpoint{4.695435in}{2.632892in}}%
\pgfpathlineto{\pgfqpoint{4.747315in}{2.603683in}}%
\pgfpathlineto{\pgfqpoint{4.799196in}{2.577038in}}%
\pgfpathlineto{\pgfqpoint{4.851076in}{2.552845in}}%
\pgfpathlineto{\pgfqpoint{4.905213in}{2.530080in}}%
\pgfpathlineto{\pgfqpoint{4.959349in}{2.509697in}}%
\pgfpathlineto{\pgfqpoint{5.015741in}{2.490826in}}%
\pgfpathlineto{\pgfqpoint{5.074388in}{2.473559in}}%
\pgfpathlineto{\pgfqpoint{5.135291in}{2.457953in}}%
\pgfpathlineto{\pgfqpoint{5.198450in}{2.444028in}}%
\pgfpathlineto{\pgfqpoint{5.266120in}{2.431379in}}%
\pgfpathlineto{\pgfqpoint{5.338302in}{2.420155in}}%
\pgfpathlineto{\pgfqpoint{5.417250in}{2.410181in}}%
\pgfpathlineto{\pgfqpoint{5.502966in}{2.401631in}}%
\pgfpathlineto{\pgfqpoint{5.534545in}{2.398992in}}%
\pgfpathlineto{\pgfqpoint{5.534545in}{2.398992in}}%
\pgfusepath{stroke}%
\end{pgfscope}%
\begin{pgfscope}%
\pgfpathrectangle{\pgfqpoint{0.800000in}{0.528000in}}{\pgfqpoint{4.960000in}{3.696000in}}%
\pgfusepath{clip}%
\pgfsetrectcap%
\pgfsetroundjoin%
\pgfsetlinewidth{1.505625pt}%
\definecolor{currentstroke}{rgb}{0.156863,0.411765,0.513725}%
\pgfsetstrokecolor{currentstroke}%
\pgfsetdash{}{0pt}%
\pgfpathmoveto{\pgfqpoint{1.025455in}{2.281503in}}%
\pgfpathlineto{\pgfqpoint{1.084102in}{2.264675in}}%
\pgfpathlineto{\pgfqpoint{1.140494in}{2.246264in}}%
\pgfpathlineto{\pgfqpoint{1.196886in}{2.225478in}}%
\pgfpathlineto{\pgfqpoint{1.251022in}{2.203121in}}%
\pgfpathlineto{\pgfqpoint{1.302902in}{2.179342in}}%
\pgfpathlineto{\pgfqpoint{1.354783in}{2.153132in}}%
\pgfpathlineto{\pgfqpoint{1.406663in}{2.124377in}}%
\pgfpathlineto{\pgfqpoint{1.458544in}{2.092977in}}%
\pgfpathlineto{\pgfqpoint{1.510424in}{2.058853in}}%
\pgfpathlineto{\pgfqpoint{1.562305in}{2.021948in}}%
\pgfpathlineto{\pgfqpoint{1.614185in}{1.982232in}}%
\pgfpathlineto{\pgfqpoint{1.666066in}{1.939704in}}%
\pgfpathlineto{\pgfqpoint{1.720202in}{1.892365in}}%
\pgfpathlineto{\pgfqpoint{1.774338in}{1.842079in}}%
\pgfpathlineto{\pgfqpoint{1.830730in}{1.786701in}}%
\pgfpathlineto{\pgfqpoint{1.891633in}{1.723686in}}%
\pgfpathlineto{\pgfqpoint{1.957048in}{1.652664in}}%
\pgfpathlineto{\pgfqpoint{2.029229in}{1.570891in}}%
\pgfpathlineto{\pgfqpoint{2.114945in}{1.470297in}}%
\pgfpathlineto{\pgfqpoint{2.419461in}{1.109521in}}%
\pgfpathlineto{\pgfqpoint{2.482619in}{1.040154in}}%
\pgfpathlineto{\pgfqpoint{2.536756in}{0.983867in}}%
\pgfpathlineto{\pgfqpoint{2.586380in}{0.935379in}}%
\pgfpathlineto{\pgfqpoint{2.631494in}{0.894264in}}%
\pgfpathlineto{\pgfqpoint{2.674352in}{0.858120in}}%
\pgfpathlineto{\pgfqpoint{2.714954in}{0.826735in}}%
\pgfpathlineto{\pgfqpoint{2.753300in}{0.799834in}}%
\pgfpathlineto{\pgfqpoint{2.789391in}{0.777092in}}%
\pgfpathlineto{\pgfqpoint{2.825482in}{0.756969in}}%
\pgfpathlineto{\pgfqpoint{2.859317in}{0.740578in}}%
\pgfpathlineto{\pgfqpoint{2.893152in}{0.726661in}}%
\pgfpathlineto{\pgfqpoint{2.924731in}{0.715961in}}%
\pgfpathlineto{\pgfqpoint{2.956311in}{0.707517in}}%
\pgfpathlineto{\pgfqpoint{2.987890in}{0.701364in}}%
\pgfpathlineto{\pgfqpoint{3.019470in}{0.697527in}}%
\pgfpathlineto{\pgfqpoint{3.051049in}{0.696024in}}%
\pgfpathlineto{\pgfqpoint{3.082629in}{0.696859in}}%
\pgfpathlineto{\pgfqpoint{3.114208in}{0.700030in}}%
\pgfpathlineto{\pgfqpoint{3.145787in}{0.705524in}}%
\pgfpathlineto{\pgfqpoint{3.177367in}{0.713316in}}%
\pgfpathlineto{\pgfqpoint{3.208946in}{0.723376in}}%
\pgfpathlineto{\pgfqpoint{3.240526in}{0.735661in}}%
\pgfpathlineto{\pgfqpoint{3.274361in}{0.751235in}}%
\pgfpathlineto{\pgfqpoint{3.308196in}{0.769233in}}%
\pgfpathlineto{\pgfqpoint{3.344287in}{0.791006in}}%
\pgfpathlineto{\pgfqpoint{3.380377in}{0.815326in}}%
\pgfpathlineto{\pgfqpoint{3.418724in}{0.843815in}}%
\pgfpathlineto{\pgfqpoint{3.459326in}{0.876772in}}%
\pgfpathlineto{\pgfqpoint{3.502184in}{0.914447in}}%
\pgfpathlineto{\pgfqpoint{3.547297in}{0.957020in}}%
\pgfpathlineto{\pgfqpoint{3.596922in}{1.006922in}}%
\pgfpathlineto{\pgfqpoint{3.651058in}{1.064515in}}%
\pgfpathlineto{\pgfqpoint{3.714217in}{1.135089in}}%
\pgfpathlineto{\pgfqpoint{3.790910in}{1.224358in}}%
\pgfpathlineto{\pgfqpoint{3.908205in}{1.364819in}}%
\pgfpathlineto{\pgfqpoint{4.050312in}{1.534196in}}%
\pgfpathlineto{\pgfqpoint{4.131517in}{1.627459in}}%
\pgfpathlineto{\pgfqpoint{4.201443in}{1.704420in}}%
\pgfpathlineto{\pgfqpoint{4.264601in}{1.770672in}}%
\pgfpathlineto{\pgfqpoint{4.323249in}{1.829061in}}%
\pgfpathlineto{\pgfqpoint{4.379641in}{1.882128in}}%
\pgfpathlineto{\pgfqpoint{4.433777in}{1.930090in}}%
\pgfpathlineto{\pgfqpoint{4.487913in}{1.975038in}}%
\pgfpathlineto{\pgfqpoint{4.539794in}{2.015244in}}%
\pgfpathlineto{\pgfqpoint{4.591674in}{2.052636in}}%
\pgfpathlineto{\pgfqpoint{4.643555in}{2.087240in}}%
\pgfpathlineto{\pgfqpoint{4.695435in}{2.119108in}}%
\pgfpathlineto{\pgfqpoint{4.747315in}{2.148317in}}%
\pgfpathlineto{\pgfqpoint{4.799196in}{2.174962in}}%
\pgfpathlineto{\pgfqpoint{4.851076in}{2.199155in}}%
\pgfpathlineto{\pgfqpoint{4.905213in}{2.221920in}}%
\pgfpathlineto{\pgfqpoint{4.959349in}{2.242303in}}%
\pgfpathlineto{\pgfqpoint{5.015741in}{2.261174in}}%
\pgfpathlineto{\pgfqpoint{5.074388in}{2.278441in}}%
\pgfpathlineto{\pgfqpoint{5.135291in}{2.294047in}}%
\pgfpathlineto{\pgfqpoint{5.198450in}{2.307972in}}%
\pgfpathlineto{\pgfqpoint{5.266120in}{2.320621in}}%
\pgfpathlineto{\pgfqpoint{5.338302in}{2.331845in}}%
\pgfpathlineto{\pgfqpoint{5.417250in}{2.341819in}}%
\pgfpathlineto{\pgfqpoint{5.502966in}{2.350369in}}%
\pgfpathlineto{\pgfqpoint{5.534545in}{2.353008in}}%
\pgfpathlineto{\pgfqpoint{5.534545in}{2.353008in}}%
\pgfusepath{stroke}%
\end{pgfscope}%
\begin{pgfscope}%
\pgfsetbuttcap%
\pgfsetmiterjoin%
\definecolor{currentfill}{rgb}{0.000000,0.000000,0.000000}%
\pgfsetfillcolor{currentfill}%
\pgfsetlinewidth{1.003750pt}%
\definecolor{currentstroke}{rgb}{0.000000,0.000000,0.000000}%
\pgfsetstrokecolor{currentstroke}%
\pgfsetdash{}{0pt}%
\pgfsys@defobject{currentmarker}{\pgfqpoint{-0.069444in}{-0.069444in}}{\pgfqpoint{0.069444in}{0.069444in}}{%
\pgfpathmoveto{\pgfqpoint{0.069444in}{-0.000000in}}%
\pgfpathlineto{\pgfqpoint{-0.069444in}{0.069444in}}%
\pgfpathlineto{\pgfqpoint{-0.069444in}{-0.069444in}}%
\pgfpathlineto{\pgfqpoint{0.069444in}{-0.000000in}}%
\pgfpathclose%
\pgfusepath{stroke,fill}%
}%
\begin{pgfscope}%
\pgfsys@transformshift{5.760000in}{0.696000in}%
\pgfsys@useobject{currentmarker}{}%
\end{pgfscope}%
\end{pgfscope}%
\begin{pgfscope}%
\pgfsetbuttcap%
\pgfsetmiterjoin%
\definecolor{currentfill}{rgb}{0.000000,0.000000,0.000000}%
\pgfsetfillcolor{currentfill}%
\pgfsetlinewidth{1.003750pt}%
\definecolor{currentstroke}{rgb}{0.000000,0.000000,0.000000}%
\pgfsetstrokecolor{currentstroke}%
\pgfsetdash{}{0pt}%
\pgfsys@defobject{currentmarker}{\pgfqpoint{-0.069444in}{-0.069444in}}{\pgfqpoint{0.069444in}{0.069444in}}{%
\pgfpathmoveto{\pgfqpoint{0.000000in}{0.069444in}}%
\pgfpathlineto{\pgfqpoint{-0.069444in}{-0.069444in}}%
\pgfpathlineto{\pgfqpoint{0.069444in}{-0.069444in}}%
\pgfpathlineto{\pgfqpoint{0.000000in}{0.069444in}}%
\pgfpathclose%
\pgfusepath{stroke,fill}%
}%
\begin{pgfscope}%
\pgfsys@transformshift{1.025455in}{4.224000in}%
\pgfsys@useobject{currentmarker}{}%
\end{pgfscope}%
\end{pgfscope}%
\begin{pgfscope}%
\pgfsetrectcap%
\pgfsetmiterjoin%
\pgfsetlinewidth{0.803000pt}%
\definecolor{currentstroke}{rgb}{0.000000,0.000000,0.000000}%
\pgfsetstrokecolor{currentstroke}%
\pgfsetdash{}{0pt}%
\pgfpathmoveto{\pgfqpoint{1.025455in}{0.528000in}}%
\pgfpathlineto{\pgfqpoint{1.025455in}{4.224000in}}%
\pgfusepath{stroke}%
\end{pgfscope}%
\begin{pgfscope}%
\pgfsetrectcap%
\pgfsetmiterjoin%
\pgfsetlinewidth{0.803000pt}%
\definecolor{currentstroke}{rgb}{0.000000,0.000000,0.000000}%
\pgfsetstrokecolor{currentstroke}%
\pgfsetdash{}{0pt}%
\pgfpathmoveto{\pgfqpoint{0.800000in}{0.696000in}}%
\pgfpathlineto{\pgfqpoint{5.760000in}{0.696000in}}%
\pgfusepath{stroke}%
\end{pgfscope}%
\end{pgfpicture}%
\makeatother%
\endgroup%
}
		\caption{Éclairement $I(t)$ pour un filtre à spectre gaussien}
	\end{figure}

	\section{Analyse des résultats expérimentaux}

	À l'aide du logiciel \texttt{Regressi}, on réalise une modélisation de l'enveloppe de $I(e)$, sous une des deux formes déterminées dans la section II.

	Pour la modélisation filtre à profil spectral gaussien, l'enveloppe haute est de la forme \[
		I_\mathrm{haut} = M + B \cdot \mathrm{e}^{-(2\pi\,v\,(t-t_0)\,a)^2}
	.\] Après régression, on trouve les valeurs $M \simeq 1\,644$, $B \simeq 810$, $a\simeq 22\,139\:\mathrm{m}^{-1}$ et $t_0 \simeq 24{,}212\:\mathrm{s}$.
	On représente la modélisation sur la figure ci-dessous.

	\begin{figure}[H]
		\centering
		\resizebox{\linewidth}{!}{%% Creator: Matplotlib, PGF backend
%%
%% To include the figure in your LaTeX document, write
%%   \input{<filename>.pgf}
%%
%% Make sure the required packages are loaded in your preamble
%%   \usepackage{pgf}
%%
%% Also ensure that all the required font packages are loaded; for instance,
%% the lmodern package is sometimes necessary when using math font.
%%   \usepackage{lmodern}
%%
%% Figures using additional raster images can only be included by \input if
%% they are in the same directory as the main LaTeX file. For loading figures
%% from other directories you can use the `import` package
%%   \usepackage{import}
%%
%% and then include the figures with
%%   \import{<path to file>}{<filename>.pgf}
%%
%% Matplotlib used the following preamble
%%   
%%   \makeatletter\@ifpackageloaded{underscore}{}{\usepackage[strings]{underscore}}\makeatother
%%
\begingroup%
\makeatletter%
\begin{pgfpicture}%
\pgfpathrectangle{\pgfpointorigin}{\pgfqpoint{6.400000in}{4.800000in}}%
\pgfusepath{use as bounding box, clip}%
\begin{pgfscope}%
\pgfsetbuttcap%
\pgfsetmiterjoin%
\definecolor{currentfill}{rgb}{1.000000,1.000000,1.000000}%
\pgfsetfillcolor{currentfill}%
\pgfsetlinewidth{0.000000pt}%
\definecolor{currentstroke}{rgb}{1.000000,1.000000,1.000000}%
\pgfsetstrokecolor{currentstroke}%
\pgfsetdash{}{0pt}%
\pgfpathmoveto{\pgfqpoint{0.000000in}{0.000000in}}%
\pgfpathlineto{\pgfqpoint{6.400000in}{0.000000in}}%
\pgfpathlineto{\pgfqpoint{6.400000in}{4.800000in}}%
\pgfpathlineto{\pgfqpoint{0.000000in}{4.800000in}}%
\pgfpathlineto{\pgfqpoint{0.000000in}{0.000000in}}%
\pgfpathclose%
\pgfusepath{fill}%
\end{pgfscope}%
\begin{pgfscope}%
\pgfsetbuttcap%
\pgfsetmiterjoin%
\definecolor{currentfill}{rgb}{1.000000,1.000000,1.000000}%
\pgfsetfillcolor{currentfill}%
\pgfsetlinewidth{0.000000pt}%
\definecolor{currentstroke}{rgb}{0.000000,0.000000,0.000000}%
\pgfsetstrokecolor{currentstroke}%
\pgfsetstrokeopacity{0.000000}%
\pgfsetdash{}{0pt}%
\pgfpathmoveto{\pgfqpoint{0.800000in}{0.528000in}}%
\pgfpathlineto{\pgfqpoint{5.760000in}{0.528000in}}%
\pgfpathlineto{\pgfqpoint{5.760000in}{4.224000in}}%
\pgfpathlineto{\pgfqpoint{0.800000in}{4.224000in}}%
\pgfpathlineto{\pgfqpoint{0.800000in}{0.528000in}}%
\pgfpathclose%
\pgfusepath{fill}%
\end{pgfscope}%
\begin{pgfscope}%
\pgfsetbuttcap%
\pgfsetroundjoin%
\definecolor{currentfill}{rgb}{0.000000,0.000000,0.000000}%
\pgfsetfillcolor{currentfill}%
\pgfsetlinewidth{0.803000pt}%
\definecolor{currentstroke}{rgb}{0.000000,0.000000,0.000000}%
\pgfsetstrokecolor{currentstroke}%
\pgfsetdash{}{0pt}%
\pgfsys@defobject{currentmarker}{\pgfqpoint{0.000000in}{-0.048611in}}{\pgfqpoint{0.000000in}{0.000000in}}{%
\pgfpathmoveto{\pgfqpoint{0.000000in}{0.000000in}}%
\pgfpathlineto{\pgfqpoint{0.000000in}{-0.048611in}}%
\pgfusepath{stroke,fill}%
}%
\begin{pgfscope}%
\pgfsys@transformshift{1.025455in}{0.696000in}%
\pgfsys@useobject{currentmarker}{}%
\end{pgfscope}%
\end{pgfscope}%
\begin{pgfscope}%
\definecolor{textcolor}{rgb}{0.000000,0.000000,0.000000}%
\pgfsetstrokecolor{textcolor}%
\pgfsetfillcolor{textcolor}%
\pgftext[x=1.025455in,y=0.598778in,,top]{\color{textcolor}\rmfamily\fontsize{10.000000}{12.000000}\selectfont \(\displaystyle {0}\)}%
\end{pgfscope}%
\begin{pgfscope}%
\pgfsetbuttcap%
\pgfsetroundjoin%
\definecolor{currentfill}{rgb}{0.000000,0.000000,0.000000}%
\pgfsetfillcolor{currentfill}%
\pgfsetlinewidth{0.803000pt}%
\definecolor{currentstroke}{rgb}{0.000000,0.000000,0.000000}%
\pgfsetstrokecolor{currentstroke}%
\pgfsetdash{}{0pt}%
\pgfsys@defobject{currentmarker}{\pgfqpoint{0.000000in}{-0.048611in}}{\pgfqpoint{0.000000in}{0.000000in}}{%
\pgfpathmoveto{\pgfqpoint{0.000000in}{0.000000in}}%
\pgfpathlineto{\pgfqpoint{0.000000in}{-0.048611in}}%
\pgfusepath{stroke,fill}%
}%
\begin{pgfscope}%
\pgfsys@transformshift{2.027697in}{0.696000in}%
\pgfsys@useobject{currentmarker}{}%
\end{pgfscope}%
\end{pgfscope}%
\begin{pgfscope}%
\definecolor{textcolor}{rgb}{0.000000,0.000000,0.000000}%
\pgfsetstrokecolor{textcolor}%
\pgfsetfillcolor{textcolor}%
\pgftext[x=2.027697in,y=0.598778in,,top]{\color{textcolor}\rmfamily\fontsize{10.000000}{12.000000}\selectfont \(\displaystyle {10}\)}%
\end{pgfscope}%
\begin{pgfscope}%
\pgfsetbuttcap%
\pgfsetroundjoin%
\definecolor{currentfill}{rgb}{0.000000,0.000000,0.000000}%
\pgfsetfillcolor{currentfill}%
\pgfsetlinewidth{0.803000pt}%
\definecolor{currentstroke}{rgb}{0.000000,0.000000,0.000000}%
\pgfsetstrokecolor{currentstroke}%
\pgfsetdash{}{0pt}%
\pgfsys@defobject{currentmarker}{\pgfqpoint{0.000000in}{-0.048611in}}{\pgfqpoint{0.000000in}{0.000000in}}{%
\pgfpathmoveto{\pgfqpoint{0.000000in}{0.000000in}}%
\pgfpathlineto{\pgfqpoint{0.000000in}{-0.048611in}}%
\pgfusepath{stroke,fill}%
}%
\begin{pgfscope}%
\pgfsys@transformshift{3.029940in}{0.696000in}%
\pgfsys@useobject{currentmarker}{}%
\end{pgfscope}%
\end{pgfscope}%
\begin{pgfscope}%
\definecolor{textcolor}{rgb}{0.000000,0.000000,0.000000}%
\pgfsetstrokecolor{textcolor}%
\pgfsetfillcolor{textcolor}%
\pgftext[x=3.029940in,y=0.598778in,,top]{\color{textcolor}\rmfamily\fontsize{10.000000}{12.000000}\selectfont \(\displaystyle {20}\)}%
\end{pgfscope}%
\begin{pgfscope}%
\pgfsetbuttcap%
\pgfsetroundjoin%
\definecolor{currentfill}{rgb}{0.000000,0.000000,0.000000}%
\pgfsetfillcolor{currentfill}%
\pgfsetlinewidth{0.803000pt}%
\definecolor{currentstroke}{rgb}{0.000000,0.000000,0.000000}%
\pgfsetstrokecolor{currentstroke}%
\pgfsetdash{}{0pt}%
\pgfsys@defobject{currentmarker}{\pgfqpoint{0.000000in}{-0.048611in}}{\pgfqpoint{0.000000in}{0.000000in}}{%
\pgfpathmoveto{\pgfqpoint{0.000000in}{0.000000in}}%
\pgfpathlineto{\pgfqpoint{0.000000in}{-0.048611in}}%
\pgfusepath{stroke,fill}%
}%
\begin{pgfscope}%
\pgfsys@transformshift{4.032183in}{0.696000in}%
\pgfsys@useobject{currentmarker}{}%
\end{pgfscope}%
\end{pgfscope}%
\begin{pgfscope}%
\definecolor{textcolor}{rgb}{0.000000,0.000000,0.000000}%
\pgfsetstrokecolor{textcolor}%
\pgfsetfillcolor{textcolor}%
\pgftext[x=4.032183in,y=0.598778in,,top]{\color{textcolor}\rmfamily\fontsize{10.000000}{12.000000}\selectfont \(\displaystyle {30}\)}%
\end{pgfscope}%
\begin{pgfscope}%
\pgfsetbuttcap%
\pgfsetroundjoin%
\definecolor{currentfill}{rgb}{0.000000,0.000000,0.000000}%
\pgfsetfillcolor{currentfill}%
\pgfsetlinewidth{0.803000pt}%
\definecolor{currentstroke}{rgb}{0.000000,0.000000,0.000000}%
\pgfsetstrokecolor{currentstroke}%
\pgfsetdash{}{0pt}%
\pgfsys@defobject{currentmarker}{\pgfqpoint{0.000000in}{-0.048611in}}{\pgfqpoint{0.000000in}{0.000000in}}{%
\pgfpathmoveto{\pgfqpoint{0.000000in}{0.000000in}}%
\pgfpathlineto{\pgfqpoint{0.000000in}{-0.048611in}}%
\pgfusepath{stroke,fill}%
}%
\begin{pgfscope}%
\pgfsys@transformshift{5.034426in}{0.696000in}%
\pgfsys@useobject{currentmarker}{}%
\end{pgfscope}%
\end{pgfscope}%
\begin{pgfscope}%
\definecolor{textcolor}{rgb}{0.000000,0.000000,0.000000}%
\pgfsetstrokecolor{textcolor}%
\pgfsetfillcolor{textcolor}%
\pgftext[x=5.034426in,y=0.598778in,,top]{\color{textcolor}\rmfamily\fontsize{10.000000}{12.000000}\selectfont \(\displaystyle {40}\)}%
\end{pgfscope}%
\begin{pgfscope}%
\pgfsetbuttcap%
\pgfsetroundjoin%
\definecolor{currentfill}{rgb}{0.000000,0.000000,0.000000}%
\pgfsetfillcolor{currentfill}%
\pgfsetlinewidth{0.803000pt}%
\definecolor{currentstroke}{rgb}{0.000000,0.000000,0.000000}%
\pgfsetstrokecolor{currentstroke}%
\pgfsetdash{}{0pt}%
\pgfsys@defobject{currentmarker}{\pgfqpoint{-0.048611in}{0.000000in}}{\pgfqpoint{-0.000000in}{0.000000in}}{%
\pgfpathmoveto{\pgfqpoint{-0.000000in}{0.000000in}}%
\pgfpathlineto{\pgfqpoint{-0.048611in}{0.000000in}}%
\pgfusepath{stroke,fill}%
}%
\begin{pgfscope}%
\pgfsys@transformshift{1.025455in}{0.696000in}%
\pgfsys@useobject{currentmarker}{}%
\end{pgfscope}%
\end{pgfscope}%
\begin{pgfscope}%
\definecolor{textcolor}{rgb}{0.000000,0.000000,0.000000}%
\pgfsetstrokecolor{textcolor}%
\pgfsetfillcolor{textcolor}%
\pgftext[x=0.858788in, y=0.647775in, left, base]{\color{textcolor}\rmfamily\fontsize{10.000000}{12.000000}\selectfont \(\displaystyle {0}\)}%
\end{pgfscope}%
\begin{pgfscope}%
\pgfsetbuttcap%
\pgfsetroundjoin%
\definecolor{currentfill}{rgb}{0.000000,0.000000,0.000000}%
\pgfsetfillcolor{currentfill}%
\pgfsetlinewidth{0.803000pt}%
\definecolor{currentstroke}{rgb}{0.000000,0.000000,0.000000}%
\pgfsetstrokecolor{currentstroke}%
\pgfsetdash{}{0pt}%
\pgfsys@defobject{currentmarker}{\pgfqpoint{-0.048611in}{0.000000in}}{\pgfqpoint{-0.000000in}{0.000000in}}{%
\pgfpathmoveto{\pgfqpoint{-0.000000in}{0.000000in}}%
\pgfpathlineto{\pgfqpoint{-0.048611in}{0.000000in}}%
\pgfusepath{stroke,fill}%
}%
\begin{pgfscope}%
\pgfsys@transformshift{1.025455in}{1.377818in}%
\pgfsys@useobject{currentmarker}{}%
\end{pgfscope}%
\end{pgfscope}%
\begin{pgfscope}%
\definecolor{textcolor}{rgb}{0.000000,0.000000,0.000000}%
\pgfsetstrokecolor{textcolor}%
\pgfsetfillcolor{textcolor}%
\pgftext[x=0.719898in, y=1.329593in, left, base]{\color{textcolor}\rmfamily\fontsize{10.000000}{12.000000}\selectfont \(\displaystyle {500}\)}%
\end{pgfscope}%
\begin{pgfscope}%
\pgfsetbuttcap%
\pgfsetroundjoin%
\definecolor{currentfill}{rgb}{0.000000,0.000000,0.000000}%
\pgfsetfillcolor{currentfill}%
\pgfsetlinewidth{0.803000pt}%
\definecolor{currentstroke}{rgb}{0.000000,0.000000,0.000000}%
\pgfsetstrokecolor{currentstroke}%
\pgfsetdash{}{0pt}%
\pgfsys@defobject{currentmarker}{\pgfqpoint{-0.048611in}{0.000000in}}{\pgfqpoint{-0.000000in}{0.000000in}}{%
\pgfpathmoveto{\pgfqpoint{-0.000000in}{0.000000in}}%
\pgfpathlineto{\pgfqpoint{-0.048611in}{0.000000in}}%
\pgfusepath{stroke,fill}%
}%
\begin{pgfscope}%
\pgfsys@transformshift{1.025455in}{2.059636in}%
\pgfsys@useobject{currentmarker}{}%
\end{pgfscope}%
\end{pgfscope}%
\begin{pgfscope}%
\definecolor{textcolor}{rgb}{0.000000,0.000000,0.000000}%
\pgfsetstrokecolor{textcolor}%
\pgfsetfillcolor{textcolor}%
\pgftext[x=0.650454in, y=2.011411in, left, base]{\color{textcolor}\rmfamily\fontsize{10.000000}{12.000000}\selectfont \(\displaystyle {1000}\)}%
\end{pgfscope}%
\begin{pgfscope}%
\pgfsetbuttcap%
\pgfsetroundjoin%
\definecolor{currentfill}{rgb}{0.000000,0.000000,0.000000}%
\pgfsetfillcolor{currentfill}%
\pgfsetlinewidth{0.803000pt}%
\definecolor{currentstroke}{rgb}{0.000000,0.000000,0.000000}%
\pgfsetstrokecolor{currentstroke}%
\pgfsetdash{}{0pt}%
\pgfsys@defobject{currentmarker}{\pgfqpoint{-0.048611in}{0.000000in}}{\pgfqpoint{-0.000000in}{0.000000in}}{%
\pgfpathmoveto{\pgfqpoint{-0.000000in}{0.000000in}}%
\pgfpathlineto{\pgfqpoint{-0.048611in}{0.000000in}}%
\pgfusepath{stroke,fill}%
}%
\begin{pgfscope}%
\pgfsys@transformshift{1.025455in}{2.741455in}%
\pgfsys@useobject{currentmarker}{}%
\end{pgfscope}%
\end{pgfscope}%
\begin{pgfscope}%
\definecolor{textcolor}{rgb}{0.000000,0.000000,0.000000}%
\pgfsetstrokecolor{textcolor}%
\pgfsetfillcolor{textcolor}%
\pgftext[x=0.650454in, y=2.693229in, left, base]{\color{textcolor}\rmfamily\fontsize{10.000000}{12.000000}\selectfont \(\displaystyle {1500}\)}%
\end{pgfscope}%
\begin{pgfscope}%
\pgfsetbuttcap%
\pgfsetroundjoin%
\definecolor{currentfill}{rgb}{0.000000,0.000000,0.000000}%
\pgfsetfillcolor{currentfill}%
\pgfsetlinewidth{0.803000pt}%
\definecolor{currentstroke}{rgb}{0.000000,0.000000,0.000000}%
\pgfsetstrokecolor{currentstroke}%
\pgfsetdash{}{0pt}%
\pgfsys@defobject{currentmarker}{\pgfqpoint{-0.048611in}{0.000000in}}{\pgfqpoint{-0.000000in}{0.000000in}}{%
\pgfpathmoveto{\pgfqpoint{-0.000000in}{0.000000in}}%
\pgfpathlineto{\pgfqpoint{-0.048611in}{0.000000in}}%
\pgfusepath{stroke,fill}%
}%
\begin{pgfscope}%
\pgfsys@transformshift{1.025455in}{3.423273in}%
\pgfsys@useobject{currentmarker}{}%
\end{pgfscope}%
\end{pgfscope}%
\begin{pgfscope}%
\definecolor{textcolor}{rgb}{0.000000,0.000000,0.000000}%
\pgfsetstrokecolor{textcolor}%
\pgfsetfillcolor{textcolor}%
\pgftext[x=0.650454in, y=3.375047in, left, base]{\color{textcolor}\rmfamily\fontsize{10.000000}{12.000000}\selectfont \(\displaystyle {2000}\)}%
\end{pgfscope}%
\begin{pgfscope}%
\pgfsetbuttcap%
\pgfsetroundjoin%
\definecolor{currentfill}{rgb}{0.000000,0.000000,0.000000}%
\pgfsetfillcolor{currentfill}%
\pgfsetlinewidth{0.803000pt}%
\definecolor{currentstroke}{rgb}{0.000000,0.000000,0.000000}%
\pgfsetstrokecolor{currentstroke}%
\pgfsetdash{}{0pt}%
\pgfsys@defobject{currentmarker}{\pgfqpoint{-0.048611in}{0.000000in}}{\pgfqpoint{-0.000000in}{0.000000in}}{%
\pgfpathmoveto{\pgfqpoint{-0.000000in}{0.000000in}}%
\pgfpathlineto{\pgfqpoint{-0.048611in}{0.000000in}}%
\pgfusepath{stroke,fill}%
}%
\begin{pgfscope}%
\pgfsys@transformshift{1.025455in}{4.105091in}%
\pgfsys@useobject{currentmarker}{}%
\end{pgfscope}%
\end{pgfscope}%
\begin{pgfscope}%
\definecolor{textcolor}{rgb}{0.000000,0.000000,0.000000}%
\pgfsetstrokecolor{textcolor}%
\pgfsetfillcolor{textcolor}%
\pgftext[x=0.650454in, y=4.056866in, left, base]{\color{textcolor}\rmfamily\fontsize{10.000000}{12.000000}\selectfont \(\displaystyle {2500}\)}%
\end{pgfscope}%
\begin{pgfscope}%
\pgfpathrectangle{\pgfqpoint{0.800000in}{0.528000in}}{\pgfqpoint{4.960000in}{3.696000in}}%
\pgfusepath{clip}%
\pgfsetrectcap%
\pgfsetroundjoin%
\pgfsetlinewidth{1.505625pt}%
\definecolor{currentstroke}{rgb}{0.564706,0.478431,0.662745}%
\pgfsetstrokecolor{currentstroke}%
\pgfsetdash{}{0pt}%
\pgfpathmoveto{\pgfqpoint{1.025455in}{3.023727in}}%
\pgfpathlineto{\pgfqpoint{1.027459in}{2.992364in}}%
\pgfpathlineto{\pgfqpoint{1.028461in}{2.988273in}}%
\pgfpathlineto{\pgfqpoint{1.029464in}{2.992364in}}%
\pgfpathlineto{\pgfqpoint{1.030466in}{2.984182in}}%
\pgfpathlineto{\pgfqpoint{1.031468in}{2.986909in}}%
\pgfpathlineto{\pgfqpoint{1.032470in}{2.996455in}}%
\pgfpathlineto{\pgfqpoint{1.033472in}{2.993727in}}%
\pgfpathlineto{\pgfqpoint{1.034475in}{2.996455in}}%
\pgfpathlineto{\pgfqpoint{1.035477in}{2.995091in}}%
\pgfpathlineto{\pgfqpoint{1.036479in}{2.984182in}}%
\pgfpathlineto{\pgfqpoint{1.037481in}{2.992364in}}%
\pgfpathlineto{\pgfqpoint{1.038484in}{2.978727in}}%
\pgfpathlineto{\pgfqpoint{1.039486in}{2.982818in}}%
\pgfpathlineto{\pgfqpoint{1.040488in}{2.978727in}}%
\pgfpathlineto{\pgfqpoint{1.041490in}{2.981455in}}%
\pgfpathlineto{\pgfqpoint{1.043495in}{2.967818in}}%
\pgfpathlineto{\pgfqpoint{1.045499in}{2.963727in}}%
\pgfpathlineto{\pgfqpoint{1.048506in}{2.973273in}}%
\pgfpathlineto{\pgfqpoint{1.049508in}{2.974636in}}%
\pgfpathlineto{\pgfqpoint{1.050511in}{2.959636in}}%
\pgfpathlineto{\pgfqpoint{1.051513in}{2.982818in}}%
\pgfpathlineto{\pgfqpoint{1.053517in}{2.966455in}}%
\pgfpathlineto{\pgfqpoint{1.055522in}{2.978727in}}%
\pgfpathlineto{\pgfqpoint{1.056524in}{2.959636in}}%
\pgfpathlineto{\pgfqpoint{1.058529in}{2.978727in}}%
\pgfpathlineto{\pgfqpoint{1.059531in}{2.980091in}}%
\pgfpathlineto{\pgfqpoint{1.063540in}{2.999182in}}%
\pgfpathlineto{\pgfqpoint{1.064542in}{2.978727in}}%
\pgfpathlineto{\pgfqpoint{1.065544in}{2.992364in}}%
\pgfpathlineto{\pgfqpoint{1.066547in}{2.991000in}}%
\pgfpathlineto{\pgfqpoint{1.067549in}{2.995091in}}%
\pgfpathlineto{\pgfqpoint{1.068551in}{2.985545in}}%
\pgfpathlineto{\pgfqpoint{1.069553in}{2.986909in}}%
\pgfpathlineto{\pgfqpoint{1.071558in}{2.999182in}}%
\pgfpathlineto{\pgfqpoint{1.075567in}{2.973273in}}%
\pgfpathlineto{\pgfqpoint{1.076569in}{2.974636in}}%
\pgfpathlineto{\pgfqpoint{1.078573in}{2.969182in}}%
\pgfpathlineto{\pgfqpoint{1.079576in}{2.970545in}}%
\pgfpathlineto{\pgfqpoint{1.081580in}{2.959636in}}%
\pgfpathlineto{\pgfqpoint{1.083585in}{2.969182in}}%
\pgfpathlineto{\pgfqpoint{1.084587in}{2.970545in}}%
\pgfpathlineto{\pgfqpoint{1.086591in}{2.952818in}}%
\pgfpathlineto{\pgfqpoint{1.087594in}{2.981455in}}%
\pgfpathlineto{\pgfqpoint{1.088596in}{2.971909in}}%
\pgfpathlineto{\pgfqpoint{1.091603in}{2.982818in}}%
\pgfpathlineto{\pgfqpoint{1.092605in}{2.980091in}}%
\pgfpathlineto{\pgfqpoint{1.093607in}{2.989636in}}%
\pgfpathlineto{\pgfqpoint{1.094609in}{2.970545in}}%
\pgfpathlineto{\pgfqpoint{1.095612in}{2.986909in}}%
\pgfpathlineto{\pgfqpoint{1.096614in}{2.981455in}}%
\pgfpathlineto{\pgfqpoint{1.099621in}{2.991000in}}%
\pgfpathlineto{\pgfqpoint{1.100623in}{2.977364in}}%
\pgfpathlineto{\pgfqpoint{1.101625in}{2.993727in}}%
\pgfpathlineto{\pgfqpoint{1.102627in}{2.978727in}}%
\pgfpathlineto{\pgfqpoint{1.104632in}{2.985545in}}%
\pgfpathlineto{\pgfqpoint{1.106636in}{2.988273in}}%
\pgfpathlineto{\pgfqpoint{1.108641in}{2.984182in}}%
\pgfpathlineto{\pgfqpoint{1.110645in}{2.973273in}}%
\pgfpathlineto{\pgfqpoint{1.111647in}{2.965091in}}%
\pgfpathlineto{\pgfqpoint{1.113652in}{2.986909in}}%
\pgfpathlineto{\pgfqpoint{1.116659in}{2.963727in}}%
\pgfpathlineto{\pgfqpoint{1.117661in}{2.965091in}}%
\pgfpathlineto{\pgfqpoint{1.119665in}{2.980091in}}%
\pgfpathlineto{\pgfqpoint{1.120668in}{2.965091in}}%
\pgfpathlineto{\pgfqpoint{1.121670in}{2.969182in}}%
\pgfpathlineto{\pgfqpoint{1.122672in}{2.961000in}}%
\pgfpathlineto{\pgfqpoint{1.124677in}{2.974636in}}%
\pgfpathlineto{\pgfqpoint{1.126681in}{2.967818in}}%
\pgfpathlineto{\pgfqpoint{1.129688in}{2.985545in}}%
\pgfpathlineto{\pgfqpoint{1.131692in}{2.982818in}}%
\pgfpathlineto{\pgfqpoint{1.133697in}{2.986909in}}%
\pgfpathlineto{\pgfqpoint{1.134699in}{2.973273in}}%
\pgfpathlineto{\pgfqpoint{1.137706in}{2.996455in}}%
\pgfpathlineto{\pgfqpoint{1.139710in}{2.973273in}}%
\pgfpathlineto{\pgfqpoint{1.142717in}{2.986909in}}%
\pgfpathlineto{\pgfqpoint{1.144721in}{2.977364in}}%
\pgfpathlineto{\pgfqpoint{1.145724in}{2.984182in}}%
\pgfpathlineto{\pgfqpoint{1.147728in}{2.977364in}}%
\pgfpathlineto{\pgfqpoint{1.148730in}{2.980091in}}%
\pgfpathlineto{\pgfqpoint{1.151737in}{2.963727in}}%
\pgfpathlineto{\pgfqpoint{1.152739in}{2.973273in}}%
\pgfpathlineto{\pgfqpoint{1.153742in}{2.961000in}}%
\pgfpathlineto{\pgfqpoint{1.155746in}{2.969182in}}%
\pgfpathlineto{\pgfqpoint{1.156748in}{2.969182in}}%
\pgfpathlineto{\pgfqpoint{1.157751in}{2.981455in}}%
\pgfpathlineto{\pgfqpoint{1.158753in}{2.976000in}}%
\pgfpathlineto{\pgfqpoint{1.160757in}{2.988273in}}%
\pgfpathlineto{\pgfqpoint{1.162762in}{2.980091in}}%
\pgfpathlineto{\pgfqpoint{1.163764in}{2.984182in}}%
\pgfpathlineto{\pgfqpoint{1.164766in}{2.977364in}}%
\pgfpathlineto{\pgfqpoint{1.165769in}{2.991000in}}%
\pgfpathlineto{\pgfqpoint{1.166771in}{2.986909in}}%
\pgfpathlineto{\pgfqpoint{1.167773in}{2.996455in}}%
\pgfpathlineto{\pgfqpoint{1.168775in}{2.988273in}}%
\pgfpathlineto{\pgfqpoint{1.169778in}{2.993727in}}%
\pgfpathlineto{\pgfqpoint{1.170780in}{2.984182in}}%
\pgfpathlineto{\pgfqpoint{1.173786in}{2.989636in}}%
\pgfpathlineto{\pgfqpoint{1.175791in}{2.984182in}}%
\pgfpathlineto{\pgfqpoint{1.176793in}{2.976000in}}%
\pgfpathlineto{\pgfqpoint{1.177795in}{2.995091in}}%
\pgfpathlineto{\pgfqpoint{1.179800in}{2.969182in}}%
\pgfpathlineto{\pgfqpoint{1.180802in}{2.977364in}}%
\pgfpathlineto{\pgfqpoint{1.181804in}{2.973273in}}%
\pgfpathlineto{\pgfqpoint{1.182807in}{2.963727in}}%
\pgfpathlineto{\pgfqpoint{1.183809in}{2.981455in}}%
\pgfpathlineto{\pgfqpoint{1.184811in}{2.969182in}}%
\pgfpathlineto{\pgfqpoint{1.185813in}{2.982818in}}%
\pgfpathlineto{\pgfqpoint{1.187818in}{2.966455in}}%
\pgfpathlineto{\pgfqpoint{1.190825in}{2.977364in}}%
\pgfpathlineto{\pgfqpoint{1.191827in}{2.970545in}}%
\pgfpathlineto{\pgfqpoint{1.193831in}{2.988273in}}%
\pgfpathlineto{\pgfqpoint{1.194834in}{2.985545in}}%
\pgfpathlineto{\pgfqpoint{1.195836in}{2.988273in}}%
\pgfpathlineto{\pgfqpoint{1.196838in}{2.996455in}}%
\pgfpathlineto{\pgfqpoint{1.197840in}{2.993727in}}%
\pgfpathlineto{\pgfqpoint{1.199845in}{2.986909in}}%
\pgfpathlineto{\pgfqpoint{1.200847in}{2.982818in}}%
\pgfpathlineto{\pgfqpoint{1.201849in}{2.989636in}}%
\pgfpathlineto{\pgfqpoint{1.202852in}{2.985545in}}%
\pgfpathlineto{\pgfqpoint{1.203854in}{2.988273in}}%
\pgfpathlineto{\pgfqpoint{1.204856in}{2.984182in}}%
\pgfpathlineto{\pgfqpoint{1.205858in}{2.974636in}}%
\pgfpathlineto{\pgfqpoint{1.206861in}{2.991000in}}%
\pgfpathlineto{\pgfqpoint{1.207863in}{2.989636in}}%
\pgfpathlineto{\pgfqpoint{1.209867in}{2.993727in}}%
\pgfpathlineto{\pgfqpoint{1.210869in}{2.966455in}}%
\pgfpathlineto{\pgfqpoint{1.211872in}{2.970545in}}%
\pgfpathlineto{\pgfqpoint{1.212874in}{2.985545in}}%
\pgfpathlineto{\pgfqpoint{1.214878in}{2.973273in}}%
\pgfpathlineto{\pgfqpoint{1.215881in}{2.977364in}}%
\pgfpathlineto{\pgfqpoint{1.217885in}{2.965091in}}%
\pgfpathlineto{\pgfqpoint{1.219890in}{2.971909in}}%
\pgfpathlineto{\pgfqpoint{1.220892in}{2.974636in}}%
\pgfpathlineto{\pgfqpoint{1.221894in}{2.982818in}}%
\pgfpathlineto{\pgfqpoint{1.222896in}{2.962364in}}%
\pgfpathlineto{\pgfqpoint{1.224901in}{2.982818in}}%
\pgfpathlineto{\pgfqpoint{1.226905in}{2.974636in}}%
\pgfpathlineto{\pgfqpoint{1.227908in}{2.999182in}}%
\pgfpathlineto{\pgfqpoint{1.228910in}{2.971909in}}%
\pgfpathlineto{\pgfqpoint{1.231917in}{2.993727in}}%
\pgfpathlineto{\pgfqpoint{1.232919in}{2.986909in}}%
\pgfpathlineto{\pgfqpoint{1.233921in}{2.997818in}}%
\pgfpathlineto{\pgfqpoint{1.234923in}{2.991000in}}%
\pgfpathlineto{\pgfqpoint{1.235926in}{3.000545in}}%
\pgfpathlineto{\pgfqpoint{1.237930in}{2.986909in}}%
\pgfpathlineto{\pgfqpoint{1.238932in}{2.986909in}}%
\pgfpathlineto{\pgfqpoint{1.239935in}{2.996455in}}%
\pgfpathlineto{\pgfqpoint{1.240937in}{2.984182in}}%
\pgfpathlineto{\pgfqpoint{1.241939in}{2.991000in}}%
\pgfpathlineto{\pgfqpoint{1.242941in}{2.971909in}}%
\pgfpathlineto{\pgfqpoint{1.244946in}{2.988273in}}%
\pgfpathlineto{\pgfqpoint{1.245948in}{2.986909in}}%
\pgfpathlineto{\pgfqpoint{1.247952in}{2.973273in}}%
\pgfpathlineto{\pgfqpoint{1.248955in}{2.976000in}}%
\pgfpathlineto{\pgfqpoint{1.249957in}{2.973273in}}%
\pgfpathlineto{\pgfqpoint{1.250959in}{2.977364in}}%
\pgfpathlineto{\pgfqpoint{1.252964in}{2.969182in}}%
\pgfpathlineto{\pgfqpoint{1.253966in}{2.985545in}}%
\pgfpathlineto{\pgfqpoint{1.254968in}{2.959636in}}%
\pgfpathlineto{\pgfqpoint{1.256973in}{2.974636in}}%
\pgfpathlineto{\pgfqpoint{1.257975in}{2.971909in}}%
\pgfpathlineto{\pgfqpoint{1.258977in}{2.967818in}}%
\pgfpathlineto{\pgfqpoint{1.260982in}{2.974636in}}%
\pgfpathlineto{\pgfqpoint{1.263988in}{2.991000in}}%
\pgfpathlineto{\pgfqpoint{1.264991in}{2.978727in}}%
\pgfpathlineto{\pgfqpoint{1.266995in}{2.997818in}}%
\pgfpathlineto{\pgfqpoint{1.267997in}{2.992364in}}%
\pgfpathlineto{\pgfqpoint{1.269000in}{2.996455in}}%
\pgfpathlineto{\pgfqpoint{1.271004in}{2.981455in}}%
\pgfpathlineto{\pgfqpoint{1.272006in}{2.993727in}}%
\pgfpathlineto{\pgfqpoint{1.274011in}{2.988273in}}%
\pgfpathlineto{\pgfqpoint{1.275013in}{2.993727in}}%
\pgfpathlineto{\pgfqpoint{1.277018in}{2.986909in}}%
\pgfpathlineto{\pgfqpoint{1.278020in}{2.992364in}}%
\pgfpathlineto{\pgfqpoint{1.279022in}{2.989636in}}%
\pgfpathlineto{\pgfqpoint{1.280024in}{2.993727in}}%
\pgfpathlineto{\pgfqpoint{1.281026in}{2.980091in}}%
\pgfpathlineto{\pgfqpoint{1.282029in}{2.981455in}}%
\pgfpathlineto{\pgfqpoint{1.283031in}{2.978727in}}%
\pgfpathlineto{\pgfqpoint{1.284033in}{2.988273in}}%
\pgfpathlineto{\pgfqpoint{1.285035in}{2.986909in}}%
\pgfpathlineto{\pgfqpoint{1.286038in}{2.988273in}}%
\pgfpathlineto{\pgfqpoint{1.287040in}{2.969182in}}%
\pgfpathlineto{\pgfqpoint{1.288042in}{2.989636in}}%
\pgfpathlineto{\pgfqpoint{1.289044in}{2.974636in}}%
\pgfpathlineto{\pgfqpoint{1.290047in}{2.984182in}}%
\pgfpathlineto{\pgfqpoint{1.291049in}{2.976000in}}%
\pgfpathlineto{\pgfqpoint{1.292051in}{2.977364in}}%
\pgfpathlineto{\pgfqpoint{1.293053in}{2.986909in}}%
\pgfpathlineto{\pgfqpoint{1.294056in}{2.974636in}}%
\pgfpathlineto{\pgfqpoint{1.296060in}{2.981455in}}%
\pgfpathlineto{\pgfqpoint{1.297062in}{2.976000in}}%
\pgfpathlineto{\pgfqpoint{1.298065in}{2.988273in}}%
\pgfpathlineto{\pgfqpoint{1.299067in}{2.978727in}}%
\pgfpathlineto{\pgfqpoint{1.302074in}{2.986909in}}%
\pgfpathlineto{\pgfqpoint{1.303076in}{2.988273in}}%
\pgfpathlineto{\pgfqpoint{1.304078in}{2.996455in}}%
\pgfpathlineto{\pgfqpoint{1.306083in}{2.982818in}}%
\pgfpathlineto{\pgfqpoint{1.308087in}{2.999182in}}%
\pgfpathlineto{\pgfqpoint{1.309089in}{2.992364in}}%
\pgfpathlineto{\pgfqpoint{1.310092in}{2.993727in}}%
\pgfpathlineto{\pgfqpoint{1.311094in}{2.982818in}}%
\pgfpathlineto{\pgfqpoint{1.312096in}{2.985545in}}%
\pgfpathlineto{\pgfqpoint{1.313098in}{2.995091in}}%
\pgfpathlineto{\pgfqpoint{1.314101in}{2.984182in}}%
\pgfpathlineto{\pgfqpoint{1.315103in}{2.985545in}}%
\pgfpathlineto{\pgfqpoint{1.316105in}{2.976000in}}%
\pgfpathlineto{\pgfqpoint{1.317107in}{2.978727in}}%
\pgfpathlineto{\pgfqpoint{1.319112in}{2.976000in}}%
\pgfpathlineto{\pgfqpoint{1.320114in}{2.985545in}}%
\pgfpathlineto{\pgfqpoint{1.322118in}{2.974636in}}%
\pgfpathlineto{\pgfqpoint{1.323121in}{2.980091in}}%
\pgfpathlineto{\pgfqpoint{1.324123in}{2.978727in}}%
\pgfpathlineto{\pgfqpoint{1.325125in}{2.967818in}}%
\pgfpathlineto{\pgfqpoint{1.326127in}{2.984182in}}%
\pgfpathlineto{\pgfqpoint{1.329134in}{2.966455in}}%
\pgfpathlineto{\pgfqpoint{1.332141in}{2.973273in}}%
\pgfpathlineto{\pgfqpoint{1.333143in}{2.982818in}}%
\pgfpathlineto{\pgfqpoint{1.334145in}{2.977364in}}%
\pgfpathlineto{\pgfqpoint{1.336150in}{2.989636in}}%
\pgfpathlineto{\pgfqpoint{1.337152in}{2.984182in}}%
\pgfpathlineto{\pgfqpoint{1.338154in}{2.988273in}}%
\pgfpathlineto{\pgfqpoint{1.339157in}{2.981455in}}%
\pgfpathlineto{\pgfqpoint{1.341161in}{2.991000in}}%
\pgfpathlineto{\pgfqpoint{1.342163in}{2.986909in}}%
\pgfpathlineto{\pgfqpoint{1.344168in}{3.003273in}}%
\pgfpathlineto{\pgfqpoint{1.345170in}{2.991000in}}%
\pgfpathlineto{\pgfqpoint{1.346172in}{2.996455in}}%
\pgfpathlineto{\pgfqpoint{1.347175in}{2.995091in}}%
\pgfpathlineto{\pgfqpoint{1.348177in}{3.001909in}}%
\pgfpathlineto{\pgfqpoint{1.350181in}{2.984182in}}%
\pgfpathlineto{\pgfqpoint{1.351183in}{2.992364in}}%
\pgfpathlineto{\pgfqpoint{1.352186in}{2.978727in}}%
\pgfpathlineto{\pgfqpoint{1.354190in}{2.982818in}}%
\pgfpathlineto{\pgfqpoint{1.356195in}{2.977364in}}%
\pgfpathlineto{\pgfqpoint{1.359201in}{2.971909in}}%
\pgfpathlineto{\pgfqpoint{1.360204in}{2.986909in}}%
\pgfpathlineto{\pgfqpoint{1.362208in}{2.974636in}}%
\pgfpathlineto{\pgfqpoint{1.364213in}{2.971909in}}%
\pgfpathlineto{\pgfqpoint{1.365215in}{2.974636in}}%
\pgfpathlineto{\pgfqpoint{1.366217in}{2.970545in}}%
\pgfpathlineto{\pgfqpoint{1.368222in}{2.980091in}}%
\pgfpathlineto{\pgfqpoint{1.369224in}{2.971909in}}%
\pgfpathlineto{\pgfqpoint{1.370226in}{2.977364in}}%
\pgfpathlineto{\pgfqpoint{1.372231in}{2.970545in}}%
\pgfpathlineto{\pgfqpoint{1.374235in}{2.988273in}}%
\pgfpathlineto{\pgfqpoint{1.375237in}{2.973273in}}%
\pgfpathlineto{\pgfqpoint{1.378244in}{2.995091in}}%
\pgfpathlineto{\pgfqpoint{1.380249in}{2.982818in}}%
\pgfpathlineto{\pgfqpoint{1.384258in}{2.996455in}}%
\pgfpathlineto{\pgfqpoint{1.386262in}{2.984182in}}%
\pgfpathlineto{\pgfqpoint{1.387264in}{2.985545in}}%
\pgfpathlineto{\pgfqpoint{1.389269in}{2.995091in}}%
\pgfpathlineto{\pgfqpoint{1.391273in}{2.980091in}}%
\pgfpathlineto{\pgfqpoint{1.392275in}{2.984182in}}%
\pgfpathlineto{\pgfqpoint{1.393278in}{2.965091in}}%
\pgfpathlineto{\pgfqpoint{1.395282in}{2.982818in}}%
\pgfpathlineto{\pgfqpoint{1.400293in}{2.977364in}}%
\pgfpathlineto{\pgfqpoint{1.401296in}{2.980091in}}%
\pgfpathlineto{\pgfqpoint{1.402298in}{2.973273in}}%
\pgfpathlineto{\pgfqpoint{1.403300in}{2.976000in}}%
\pgfpathlineto{\pgfqpoint{1.405305in}{2.969182in}}%
\pgfpathlineto{\pgfqpoint{1.408311in}{2.974636in}}%
\pgfpathlineto{\pgfqpoint{1.409314in}{2.997818in}}%
\pgfpathlineto{\pgfqpoint{1.410316in}{2.969182in}}%
\pgfpathlineto{\pgfqpoint{1.412320in}{2.988273in}}%
\pgfpathlineto{\pgfqpoint{1.413323in}{2.980091in}}%
\pgfpathlineto{\pgfqpoint{1.414325in}{2.988273in}}%
\pgfpathlineto{\pgfqpoint{1.415327in}{2.978727in}}%
\pgfpathlineto{\pgfqpoint{1.416329in}{2.991000in}}%
\pgfpathlineto{\pgfqpoint{1.417332in}{2.973273in}}%
\pgfpathlineto{\pgfqpoint{1.419336in}{2.989636in}}%
\pgfpathlineto{\pgfqpoint{1.420338in}{2.988273in}}%
\pgfpathlineto{\pgfqpoint{1.421340in}{2.999182in}}%
\pgfpathlineto{\pgfqpoint{1.422343in}{2.986909in}}%
\pgfpathlineto{\pgfqpoint{1.423345in}{2.996455in}}%
\pgfpathlineto{\pgfqpoint{1.425349in}{2.984182in}}%
\pgfpathlineto{\pgfqpoint{1.426352in}{2.991000in}}%
\pgfpathlineto{\pgfqpoint{1.427354in}{2.977364in}}%
\pgfpathlineto{\pgfqpoint{1.428356in}{2.991000in}}%
\pgfpathlineto{\pgfqpoint{1.430361in}{2.977364in}}%
\pgfpathlineto{\pgfqpoint{1.431363in}{2.991000in}}%
\pgfpathlineto{\pgfqpoint{1.433367in}{2.967818in}}%
\pgfpathlineto{\pgfqpoint{1.436374in}{2.982818in}}%
\pgfpathlineto{\pgfqpoint{1.437376in}{2.961000in}}%
\pgfpathlineto{\pgfqpoint{1.438379in}{2.974636in}}%
\pgfpathlineto{\pgfqpoint{1.439381in}{2.971909in}}%
\pgfpathlineto{\pgfqpoint{1.440383in}{2.984182in}}%
\pgfpathlineto{\pgfqpoint{1.442388in}{2.970545in}}%
\pgfpathlineto{\pgfqpoint{1.443390in}{2.966455in}}%
\pgfpathlineto{\pgfqpoint{1.444392in}{2.967818in}}%
\pgfpathlineto{\pgfqpoint{1.445394in}{2.980091in}}%
\pgfpathlineto{\pgfqpoint{1.446397in}{2.978727in}}%
\pgfpathlineto{\pgfqpoint{1.448401in}{2.986909in}}%
\pgfpathlineto{\pgfqpoint{1.449403in}{2.969182in}}%
\pgfpathlineto{\pgfqpoint{1.451408in}{2.984182in}}%
\pgfpathlineto{\pgfqpoint{1.452410in}{2.984182in}}%
\pgfpathlineto{\pgfqpoint{1.453412in}{2.999182in}}%
\pgfpathlineto{\pgfqpoint{1.454415in}{2.982818in}}%
\pgfpathlineto{\pgfqpoint{1.455417in}{2.986909in}}%
\pgfpathlineto{\pgfqpoint{1.456419in}{2.981455in}}%
\pgfpathlineto{\pgfqpoint{1.458423in}{3.000545in}}%
\pgfpathlineto{\pgfqpoint{1.459426in}{2.996455in}}%
\pgfpathlineto{\pgfqpoint{1.461430in}{2.967818in}}%
\pgfpathlineto{\pgfqpoint{1.462432in}{3.000545in}}%
\pgfpathlineto{\pgfqpoint{1.463435in}{2.984182in}}%
\pgfpathlineto{\pgfqpoint{1.464437in}{2.988273in}}%
\pgfpathlineto{\pgfqpoint{1.465439in}{2.986909in}}%
\pgfpathlineto{\pgfqpoint{1.466441in}{2.989636in}}%
\pgfpathlineto{\pgfqpoint{1.467444in}{2.988273in}}%
\pgfpathlineto{\pgfqpoint{1.468446in}{2.973273in}}%
\pgfpathlineto{\pgfqpoint{1.470450in}{2.999182in}}%
\pgfpathlineto{\pgfqpoint{1.471453in}{2.974636in}}%
\pgfpathlineto{\pgfqpoint{1.472455in}{2.988273in}}%
\pgfpathlineto{\pgfqpoint{1.473457in}{2.966455in}}%
\pgfpathlineto{\pgfqpoint{1.475462in}{2.981455in}}%
\pgfpathlineto{\pgfqpoint{1.476464in}{2.980091in}}%
\pgfpathlineto{\pgfqpoint{1.478468in}{2.965091in}}%
\pgfpathlineto{\pgfqpoint{1.480473in}{2.982818in}}%
\pgfpathlineto{\pgfqpoint{1.481475in}{2.974636in}}%
\pgfpathlineto{\pgfqpoint{1.482477in}{2.978727in}}%
\pgfpathlineto{\pgfqpoint{1.483480in}{2.959636in}}%
\pgfpathlineto{\pgfqpoint{1.484482in}{2.963727in}}%
\pgfpathlineto{\pgfqpoint{1.486486in}{2.986909in}}%
\pgfpathlineto{\pgfqpoint{1.488491in}{2.976000in}}%
\pgfpathlineto{\pgfqpoint{1.489493in}{2.986909in}}%
\pgfpathlineto{\pgfqpoint{1.490495in}{2.970545in}}%
\pgfpathlineto{\pgfqpoint{1.492500in}{2.992364in}}%
\pgfpathlineto{\pgfqpoint{1.494504in}{2.981455in}}%
\pgfpathlineto{\pgfqpoint{1.495506in}{2.980091in}}%
\pgfpathlineto{\pgfqpoint{1.497511in}{2.982818in}}%
\pgfpathlineto{\pgfqpoint{1.498513in}{2.996455in}}%
\pgfpathlineto{\pgfqpoint{1.499515in}{2.995091in}}%
\pgfpathlineto{\pgfqpoint{1.500518in}{2.980091in}}%
\pgfpathlineto{\pgfqpoint{1.502522in}{3.003273in}}%
\pgfpathlineto{\pgfqpoint{1.504527in}{2.996455in}}%
\pgfpathlineto{\pgfqpoint{1.505529in}{2.977364in}}%
\pgfpathlineto{\pgfqpoint{1.506531in}{3.001909in}}%
\pgfpathlineto{\pgfqpoint{1.507533in}{2.989636in}}%
\pgfpathlineto{\pgfqpoint{1.508536in}{2.991000in}}%
\pgfpathlineto{\pgfqpoint{1.510540in}{2.985545in}}%
\pgfpathlineto{\pgfqpoint{1.512545in}{2.973273in}}%
\pgfpathlineto{\pgfqpoint{1.513547in}{2.986909in}}%
\pgfpathlineto{\pgfqpoint{1.514549in}{2.985545in}}%
\pgfpathlineto{\pgfqpoint{1.515551in}{2.985545in}}%
\pgfpathlineto{\pgfqpoint{1.516554in}{2.984182in}}%
\pgfpathlineto{\pgfqpoint{1.518558in}{2.974636in}}%
\pgfpathlineto{\pgfqpoint{1.520563in}{2.986909in}}%
\pgfpathlineto{\pgfqpoint{1.521565in}{2.967818in}}%
\pgfpathlineto{\pgfqpoint{1.522567in}{2.973273in}}%
\pgfpathlineto{\pgfqpoint{1.523569in}{2.971909in}}%
\pgfpathlineto{\pgfqpoint{1.526576in}{2.976000in}}%
\pgfpathlineto{\pgfqpoint{1.527578in}{2.958273in}}%
\pgfpathlineto{\pgfqpoint{1.528580in}{2.977364in}}%
\pgfpathlineto{\pgfqpoint{1.529583in}{2.974636in}}%
\pgfpathlineto{\pgfqpoint{1.531587in}{2.986909in}}%
\pgfpathlineto{\pgfqpoint{1.532589in}{2.965091in}}%
\pgfpathlineto{\pgfqpoint{1.534594in}{2.980091in}}%
\pgfpathlineto{\pgfqpoint{1.537601in}{2.992364in}}%
\pgfpathlineto{\pgfqpoint{1.538603in}{3.006000in}}%
\pgfpathlineto{\pgfqpoint{1.540607in}{2.997818in}}%
\pgfpathlineto{\pgfqpoint{1.541610in}{2.996455in}}%
\pgfpathlineto{\pgfqpoint{1.543614in}{3.008727in}}%
\pgfpathlineto{\pgfqpoint{1.544616in}{2.995091in}}%
\pgfpathlineto{\pgfqpoint{1.545619in}{2.999182in}}%
\pgfpathlineto{\pgfqpoint{1.546621in}{2.997818in}}%
\pgfpathlineto{\pgfqpoint{1.548625in}{3.014182in}}%
\pgfpathlineto{\pgfqpoint{1.551632in}{2.977364in}}%
\pgfpathlineto{\pgfqpoint{1.552634in}{3.001909in}}%
\pgfpathlineto{\pgfqpoint{1.553637in}{3.000545in}}%
\pgfpathlineto{\pgfqpoint{1.554639in}{2.982818in}}%
\pgfpathlineto{\pgfqpoint{1.555641in}{2.989636in}}%
\pgfpathlineto{\pgfqpoint{1.557646in}{2.981455in}}%
\pgfpathlineto{\pgfqpoint{1.558648in}{2.985545in}}%
\pgfpathlineto{\pgfqpoint{1.559650in}{2.980091in}}%
\pgfpathlineto{\pgfqpoint{1.560652in}{2.992364in}}%
\pgfpathlineto{\pgfqpoint{1.561655in}{2.971909in}}%
\pgfpathlineto{\pgfqpoint{1.562657in}{2.978727in}}%
\pgfpathlineto{\pgfqpoint{1.564661in}{2.967818in}}%
\pgfpathlineto{\pgfqpoint{1.565663in}{2.982818in}}%
\pgfpathlineto{\pgfqpoint{1.567668in}{2.965091in}}%
\pgfpathlineto{\pgfqpoint{1.568670in}{2.973273in}}%
\pgfpathlineto{\pgfqpoint{1.569672in}{2.959636in}}%
\pgfpathlineto{\pgfqpoint{1.570675in}{2.980091in}}%
\pgfpathlineto{\pgfqpoint{1.571677in}{2.970545in}}%
\pgfpathlineto{\pgfqpoint{1.572679in}{2.981455in}}%
\pgfpathlineto{\pgfqpoint{1.573681in}{2.966455in}}%
\pgfpathlineto{\pgfqpoint{1.575686in}{2.982818in}}%
\pgfpathlineto{\pgfqpoint{1.576688in}{2.971909in}}%
\pgfpathlineto{\pgfqpoint{1.577690in}{2.988273in}}%
\pgfpathlineto{\pgfqpoint{1.579695in}{2.966455in}}%
\pgfpathlineto{\pgfqpoint{1.582702in}{2.999182in}}%
\pgfpathlineto{\pgfqpoint{1.584706in}{2.988273in}}%
\pgfpathlineto{\pgfqpoint{1.586711in}{2.986909in}}%
\pgfpathlineto{\pgfqpoint{1.588715in}{3.012818in}}%
\pgfpathlineto{\pgfqpoint{1.589717in}{3.004636in}}%
\pgfpathlineto{\pgfqpoint{1.591722in}{2.982818in}}%
\pgfpathlineto{\pgfqpoint{1.593726in}{3.012818in}}%
\pgfpathlineto{\pgfqpoint{1.594729in}{3.006000in}}%
\pgfpathlineto{\pgfqpoint{1.595731in}{2.978727in}}%
\pgfpathlineto{\pgfqpoint{1.597735in}{3.007364in}}%
\pgfpathlineto{\pgfqpoint{1.598737in}{3.021000in}}%
\pgfpathlineto{\pgfqpoint{1.601744in}{2.980091in}}%
\pgfpathlineto{\pgfqpoint{1.602746in}{2.991000in}}%
\pgfpathlineto{\pgfqpoint{1.603749in}{2.982818in}}%
\pgfpathlineto{\pgfqpoint{1.605753in}{2.988273in}}%
\pgfpathlineto{\pgfqpoint{1.606755in}{2.986909in}}%
\pgfpathlineto{\pgfqpoint{1.608760in}{2.962364in}}%
\pgfpathlineto{\pgfqpoint{1.609762in}{2.958273in}}%
\pgfpathlineto{\pgfqpoint{1.610764in}{2.981455in}}%
\pgfpathlineto{\pgfqpoint{1.611767in}{2.973273in}}%
\pgfpathlineto{\pgfqpoint{1.613771in}{2.950091in}}%
\pgfpathlineto{\pgfqpoint{1.614773in}{2.969182in}}%
\pgfpathlineto{\pgfqpoint{1.615776in}{2.965091in}}%
\pgfpathlineto{\pgfqpoint{1.616778in}{2.984182in}}%
\pgfpathlineto{\pgfqpoint{1.617780in}{2.950091in}}%
\pgfpathlineto{\pgfqpoint{1.620787in}{2.984182in}}%
\pgfpathlineto{\pgfqpoint{1.621789in}{2.978727in}}%
\pgfpathlineto{\pgfqpoint{1.622791in}{2.988273in}}%
\pgfpathlineto{\pgfqpoint{1.623794in}{2.980091in}}%
\pgfpathlineto{\pgfqpoint{1.624796in}{2.982818in}}%
\pgfpathlineto{\pgfqpoint{1.625798in}{2.980091in}}%
\pgfpathlineto{\pgfqpoint{1.628805in}{3.003273in}}%
\pgfpathlineto{\pgfqpoint{1.629807in}{3.001909in}}%
\pgfpathlineto{\pgfqpoint{1.630809in}{2.991000in}}%
\pgfpathlineto{\pgfqpoint{1.631812in}{2.995091in}}%
\pgfpathlineto{\pgfqpoint{1.632814in}{3.012818in}}%
\pgfpathlineto{\pgfqpoint{1.633816in}{3.010091in}}%
\pgfpathlineto{\pgfqpoint{1.635820in}{2.995091in}}%
\pgfpathlineto{\pgfqpoint{1.638827in}{3.018273in}}%
\pgfpathlineto{\pgfqpoint{1.640832in}{2.993727in}}%
\pgfpathlineto{\pgfqpoint{1.643838in}{3.010091in}}%
\pgfpathlineto{\pgfqpoint{1.644841in}{3.004636in}}%
\pgfpathlineto{\pgfqpoint{1.645843in}{2.981455in}}%
\pgfpathlineto{\pgfqpoint{1.646845in}{2.988273in}}%
\pgfpathlineto{\pgfqpoint{1.647847in}{2.977364in}}%
\pgfpathlineto{\pgfqpoint{1.648850in}{2.996455in}}%
\pgfpathlineto{\pgfqpoint{1.650854in}{2.984182in}}%
\pgfpathlineto{\pgfqpoint{1.651856in}{2.965091in}}%
\pgfpathlineto{\pgfqpoint{1.654863in}{2.984182in}}%
\pgfpathlineto{\pgfqpoint{1.655865in}{2.982818in}}%
\pgfpathlineto{\pgfqpoint{1.656868in}{2.980091in}}%
\pgfpathlineto{\pgfqpoint{1.657870in}{2.958273in}}%
\pgfpathlineto{\pgfqpoint{1.658872in}{2.974636in}}%
\pgfpathlineto{\pgfqpoint{1.659874in}{2.971909in}}%
\pgfpathlineto{\pgfqpoint{1.660877in}{2.977364in}}%
\pgfpathlineto{\pgfqpoint{1.661879in}{2.974636in}}%
\pgfpathlineto{\pgfqpoint{1.663883in}{2.948727in}}%
\pgfpathlineto{\pgfqpoint{1.665888in}{2.978727in}}%
\pgfpathlineto{\pgfqpoint{1.666890in}{2.976000in}}%
\pgfpathlineto{\pgfqpoint{1.667892in}{2.969182in}}%
\pgfpathlineto{\pgfqpoint{1.668895in}{2.978727in}}%
\pgfpathlineto{\pgfqpoint{1.669897in}{2.969182in}}%
\pgfpathlineto{\pgfqpoint{1.670899in}{2.986909in}}%
\pgfpathlineto{\pgfqpoint{1.671901in}{2.982818in}}%
\pgfpathlineto{\pgfqpoint{1.672903in}{2.985545in}}%
\pgfpathlineto{\pgfqpoint{1.673906in}{2.992364in}}%
\pgfpathlineto{\pgfqpoint{1.676912in}{2.977364in}}%
\pgfpathlineto{\pgfqpoint{1.678917in}{3.012818in}}%
\pgfpathlineto{\pgfqpoint{1.680921in}{2.985545in}}%
\pgfpathlineto{\pgfqpoint{1.681924in}{2.992364in}}%
\pgfpathlineto{\pgfqpoint{1.682926in}{3.016909in}}%
\pgfpathlineto{\pgfqpoint{1.683928in}{3.011455in}}%
\pgfpathlineto{\pgfqpoint{1.685933in}{2.991000in}}%
\pgfpathlineto{\pgfqpoint{1.687937in}{3.001909in}}%
\pgfpathlineto{\pgfqpoint{1.688939in}{3.018273in}}%
\pgfpathlineto{\pgfqpoint{1.689942in}{2.996455in}}%
\pgfpathlineto{\pgfqpoint{1.690944in}{2.999182in}}%
\pgfpathlineto{\pgfqpoint{1.691946in}{2.984182in}}%
\pgfpathlineto{\pgfqpoint{1.693951in}{3.006000in}}%
\pgfpathlineto{\pgfqpoint{1.694953in}{2.988273in}}%
\pgfpathlineto{\pgfqpoint{1.695955in}{2.996455in}}%
\pgfpathlineto{\pgfqpoint{1.697960in}{2.973273in}}%
\pgfpathlineto{\pgfqpoint{1.698962in}{2.991000in}}%
\pgfpathlineto{\pgfqpoint{1.699964in}{2.984182in}}%
\pgfpathlineto{\pgfqpoint{1.700966in}{2.986909in}}%
\pgfpathlineto{\pgfqpoint{1.702971in}{2.961000in}}%
\pgfpathlineto{\pgfqpoint{1.703973in}{2.954182in}}%
\pgfpathlineto{\pgfqpoint{1.705977in}{2.977364in}}%
\pgfpathlineto{\pgfqpoint{1.707982in}{2.941909in}}%
\pgfpathlineto{\pgfqpoint{1.710989in}{2.974636in}}%
\pgfpathlineto{\pgfqpoint{1.713995in}{2.946000in}}%
\pgfpathlineto{\pgfqpoint{1.716000in}{2.977364in}}%
\pgfpathlineto{\pgfqpoint{1.717002in}{2.973273in}}%
\pgfpathlineto{\pgfqpoint{1.718004in}{3.003273in}}%
\pgfpathlineto{\pgfqpoint{1.719007in}{2.974636in}}%
\pgfpathlineto{\pgfqpoint{1.720009in}{2.977364in}}%
\pgfpathlineto{\pgfqpoint{1.722013in}{2.991000in}}%
\pgfpathlineto{\pgfqpoint{1.724018in}{3.012818in}}%
\pgfpathlineto{\pgfqpoint{1.726022in}{2.991000in}}%
\pgfpathlineto{\pgfqpoint{1.727025in}{3.004636in}}%
\pgfpathlineto{\pgfqpoint{1.728027in}{3.038727in}}%
\pgfpathlineto{\pgfqpoint{1.729029in}{3.029182in}}%
\pgfpathlineto{\pgfqpoint{1.731034in}{3.000545in}}%
\pgfpathlineto{\pgfqpoint{1.733038in}{3.036000in}}%
\pgfpathlineto{\pgfqpoint{1.734040in}{3.031909in}}%
\pgfpathlineto{\pgfqpoint{1.736045in}{2.988273in}}%
\pgfpathlineto{\pgfqpoint{1.738049in}{3.018273in}}%
\pgfpathlineto{\pgfqpoint{1.741056in}{2.981455in}}%
\pgfpathlineto{\pgfqpoint{1.742058in}{2.981455in}}%
\pgfpathlineto{\pgfqpoint{1.743060in}{2.984182in}}%
\pgfpathlineto{\pgfqpoint{1.744063in}{2.956909in}}%
\pgfpathlineto{\pgfqpoint{1.745065in}{2.980091in}}%
\pgfpathlineto{\pgfqpoint{1.746067in}{2.974636in}}%
\pgfpathlineto{\pgfqpoint{1.749074in}{2.937818in}}%
\pgfpathlineto{\pgfqpoint{1.751078in}{2.981455in}}%
\pgfpathlineto{\pgfqpoint{1.754085in}{2.926909in}}%
\pgfpathlineto{\pgfqpoint{1.756090in}{2.966455in}}%
\pgfpathlineto{\pgfqpoint{1.757092in}{2.982818in}}%
\pgfpathlineto{\pgfqpoint{1.759096in}{2.947364in}}%
\pgfpathlineto{\pgfqpoint{1.761101in}{2.976000in}}%
\pgfpathlineto{\pgfqpoint{1.762103in}{2.991000in}}%
\pgfpathlineto{\pgfqpoint{1.763105in}{2.980091in}}%
\pgfpathlineto{\pgfqpoint{1.765110in}{2.996455in}}%
\pgfpathlineto{\pgfqpoint{1.766112in}{2.996455in}}%
\pgfpathlineto{\pgfqpoint{1.769119in}{3.019636in}}%
\pgfpathlineto{\pgfqpoint{1.771123in}{2.989636in}}%
\pgfpathlineto{\pgfqpoint{1.773128in}{3.049636in}}%
\pgfpathlineto{\pgfqpoint{1.776134in}{2.984182in}}%
\pgfpathlineto{\pgfqpoint{1.777137in}{2.997818in}}%
\pgfpathlineto{\pgfqpoint{1.779141in}{3.025091in}}%
\pgfpathlineto{\pgfqpoint{1.781146in}{2.973273in}}%
\pgfpathlineto{\pgfqpoint{1.784152in}{3.006000in}}%
\pgfpathlineto{\pgfqpoint{1.786157in}{2.970545in}}%
\pgfpathlineto{\pgfqpoint{1.787159in}{2.971909in}}%
\pgfpathlineto{\pgfqpoint{1.788161in}{2.951455in}}%
\pgfpathlineto{\pgfqpoint{1.790166in}{2.977364in}}%
\pgfpathlineto{\pgfqpoint{1.792170in}{2.959636in}}%
\pgfpathlineto{\pgfqpoint{1.794175in}{2.928273in}}%
\pgfpathlineto{\pgfqpoint{1.795177in}{2.970545in}}%
\pgfpathlineto{\pgfqpoint{1.796179in}{2.969182in}}%
\pgfpathlineto{\pgfqpoint{1.798184in}{2.931000in}}%
\pgfpathlineto{\pgfqpoint{1.801191in}{2.969182in}}%
\pgfpathlineto{\pgfqpoint{1.802193in}{2.951455in}}%
\pgfpathlineto{\pgfqpoint{1.804197in}{2.965091in}}%
\pgfpathlineto{\pgfqpoint{1.805200in}{2.961000in}}%
\pgfpathlineto{\pgfqpoint{1.806202in}{2.996455in}}%
\pgfpathlineto{\pgfqpoint{1.807204in}{2.978727in}}%
\pgfpathlineto{\pgfqpoint{1.808206in}{2.980091in}}%
\pgfpathlineto{\pgfqpoint{1.809209in}{2.982818in}}%
\pgfpathlineto{\pgfqpoint{1.810211in}{2.967818in}}%
\pgfpathlineto{\pgfqpoint{1.813217in}{3.010091in}}%
\pgfpathlineto{\pgfqpoint{1.814220in}{3.031909in}}%
\pgfpathlineto{\pgfqpoint{1.815222in}{2.993727in}}%
\pgfpathlineto{\pgfqpoint{1.816224in}{2.999182in}}%
\pgfpathlineto{\pgfqpoint{1.817226in}{2.997818in}}%
\pgfpathlineto{\pgfqpoint{1.819231in}{3.034636in}}%
\pgfpathlineto{\pgfqpoint{1.821235in}{2.992364in}}%
\pgfpathlineto{\pgfqpoint{1.822238in}{2.993727in}}%
\pgfpathlineto{\pgfqpoint{1.823240in}{3.023727in}}%
\pgfpathlineto{\pgfqpoint{1.824242in}{3.015545in}}%
\pgfpathlineto{\pgfqpoint{1.825244in}{2.995091in}}%
\pgfpathlineto{\pgfqpoint{1.826247in}{2.997818in}}%
\pgfpathlineto{\pgfqpoint{1.827249in}{2.961000in}}%
\pgfpathlineto{\pgfqpoint{1.828251in}{3.015545in}}%
\pgfpathlineto{\pgfqpoint{1.830256in}{2.959636in}}%
\pgfpathlineto{\pgfqpoint{1.831258in}{2.991000in}}%
\pgfpathlineto{\pgfqpoint{1.832260in}{2.965091in}}%
\pgfpathlineto{\pgfqpoint{1.833262in}{2.974636in}}%
\pgfpathlineto{\pgfqpoint{1.834265in}{2.943273in}}%
\pgfpathlineto{\pgfqpoint{1.836269in}{2.970545in}}%
\pgfpathlineto{\pgfqpoint{1.838274in}{2.924182in}}%
\pgfpathlineto{\pgfqpoint{1.839276in}{2.921455in}}%
\pgfpathlineto{\pgfqpoint{1.841280in}{2.969182in}}%
\pgfpathlineto{\pgfqpoint{1.843285in}{2.903727in}}%
\pgfpathlineto{\pgfqpoint{1.844287in}{2.913273in}}%
\pgfpathlineto{\pgfqpoint{1.846291in}{2.973273in}}%
\pgfpathlineto{\pgfqpoint{1.848296in}{2.941909in}}%
\pgfpathlineto{\pgfqpoint{1.849298in}{2.936455in}}%
\pgfpathlineto{\pgfqpoint{1.851303in}{2.989636in}}%
\pgfpathlineto{\pgfqpoint{1.852305in}{2.993727in}}%
\pgfpathlineto{\pgfqpoint{1.853307in}{2.991000in}}%
\pgfpathlineto{\pgfqpoint{1.854309in}{2.958273in}}%
\pgfpathlineto{\pgfqpoint{1.858318in}{3.038727in}}%
\pgfpathlineto{\pgfqpoint{1.861325in}{2.962364in}}%
\pgfpathlineto{\pgfqpoint{1.863330in}{3.036000in}}%
\pgfpathlineto{\pgfqpoint{1.864332in}{3.027818in}}%
\pgfpathlineto{\pgfqpoint{1.866336in}{2.985545in}}%
\pgfpathlineto{\pgfqpoint{1.868341in}{3.014182in}}%
\pgfpathlineto{\pgfqpoint{1.869343in}{3.026455in}}%
\pgfpathlineto{\pgfqpoint{1.871348in}{2.977364in}}%
\pgfpathlineto{\pgfqpoint{1.873352in}{3.007364in}}%
\pgfpathlineto{\pgfqpoint{1.874354in}{3.004636in}}%
\pgfpathlineto{\pgfqpoint{1.875357in}{2.995091in}}%
\pgfpathlineto{\pgfqpoint{1.876359in}{2.959636in}}%
\pgfpathlineto{\pgfqpoint{1.877361in}{2.991000in}}%
\pgfpathlineto{\pgfqpoint{1.879366in}{2.954182in}}%
\pgfpathlineto{\pgfqpoint{1.880368in}{2.980091in}}%
\pgfpathlineto{\pgfqpoint{1.881370in}{2.959636in}}%
\pgfpathlineto{\pgfqpoint{1.882372in}{2.967818in}}%
\pgfpathlineto{\pgfqpoint{1.884377in}{2.935091in}}%
\pgfpathlineto{\pgfqpoint{1.886381in}{2.974636in}}%
\pgfpathlineto{\pgfqpoint{1.888386in}{2.913273in}}%
\pgfpathlineto{\pgfqpoint{1.889388in}{2.913273in}}%
\pgfpathlineto{\pgfqpoint{1.891392in}{2.970545in}}%
\pgfpathlineto{\pgfqpoint{1.894399in}{2.905091in}}%
\pgfpathlineto{\pgfqpoint{1.896404in}{2.966455in}}%
\pgfpathlineto{\pgfqpoint{1.897406in}{2.966455in}}%
\pgfpathlineto{\pgfqpoint{1.898408in}{2.914636in}}%
\pgfpathlineto{\pgfqpoint{1.900413in}{2.963727in}}%
\pgfpathlineto{\pgfqpoint{1.901415in}{2.969182in}}%
\pgfpathlineto{\pgfqpoint{1.902417in}{2.992364in}}%
\pgfpathlineto{\pgfqpoint{1.903419in}{2.965091in}}%
\pgfpathlineto{\pgfqpoint{1.905424in}{3.000545in}}%
\pgfpathlineto{\pgfqpoint{1.906426in}{2.995091in}}%
\pgfpathlineto{\pgfqpoint{1.907428in}{3.042818in}}%
\pgfpathlineto{\pgfqpoint{1.908431in}{3.029182in}}%
\pgfpathlineto{\pgfqpoint{1.909433in}{3.034636in}}%
\pgfpathlineto{\pgfqpoint{1.911437in}{2.992364in}}%
\pgfpathlineto{\pgfqpoint{1.913442in}{3.056455in}}%
\pgfpathlineto{\pgfqpoint{1.914444in}{3.053727in}}%
\pgfpathlineto{\pgfqpoint{1.915446in}{3.040091in}}%
\pgfpathlineto{\pgfqpoint{1.916449in}{3.007364in}}%
\pgfpathlineto{\pgfqpoint{1.919455in}{3.061909in}}%
\pgfpathlineto{\pgfqpoint{1.921460in}{2.985545in}}%
\pgfpathlineto{\pgfqpoint{1.922462in}{2.993727in}}%
\pgfpathlineto{\pgfqpoint{1.924466in}{3.021000in}}%
\pgfpathlineto{\pgfqpoint{1.926471in}{2.963727in}}%
\pgfpathlineto{\pgfqpoint{1.927473in}{2.962364in}}%
\pgfpathlineto{\pgfqpoint{1.928475in}{2.937818in}}%
\pgfpathlineto{\pgfqpoint{1.929478in}{2.962364in}}%
\pgfpathlineto{\pgfqpoint{1.930480in}{2.961000in}}%
\pgfpathlineto{\pgfqpoint{1.931482in}{2.952818in}}%
\pgfpathlineto{\pgfqpoint{1.933487in}{2.898273in}}%
\pgfpathlineto{\pgfqpoint{1.934489in}{2.902364in}}%
\pgfpathlineto{\pgfqpoint{1.935491in}{2.951455in}}%
\pgfpathlineto{\pgfqpoint{1.938498in}{2.903727in}}%
\pgfpathlineto{\pgfqpoint{1.939500in}{2.921455in}}%
\pgfpathlineto{\pgfqpoint{1.940502in}{2.976000in}}%
\pgfpathlineto{\pgfqpoint{1.941505in}{2.969182in}}%
\pgfpathlineto{\pgfqpoint{1.943509in}{2.926909in}}%
\pgfpathlineto{\pgfqpoint{1.944511in}{2.932364in}}%
\pgfpathlineto{\pgfqpoint{1.945514in}{2.948727in}}%
\pgfpathlineto{\pgfqpoint{1.946516in}{2.986909in}}%
\pgfpathlineto{\pgfqpoint{1.947518in}{2.982818in}}%
\pgfpathlineto{\pgfqpoint{1.948520in}{3.006000in}}%
\pgfpathlineto{\pgfqpoint{1.949523in}{2.992364in}}%
\pgfpathlineto{\pgfqpoint{1.950525in}{3.003273in}}%
\pgfpathlineto{\pgfqpoint{1.951527in}{2.999182in}}%
\pgfpathlineto{\pgfqpoint{1.953531in}{2.955545in}}%
\pgfpathlineto{\pgfqpoint{1.954534in}{3.007364in}}%
\pgfpathlineto{\pgfqpoint{1.955536in}{3.004636in}}%
\pgfpathlineto{\pgfqpoint{1.956538in}{2.974636in}}%
\pgfpathlineto{\pgfqpoint{1.957540in}{3.025091in}}%
\pgfpathlineto{\pgfqpoint{1.958543in}{2.985545in}}%
\pgfpathlineto{\pgfqpoint{1.960547in}{3.010091in}}%
\pgfpathlineto{\pgfqpoint{1.962552in}{2.950091in}}%
\pgfpathlineto{\pgfqpoint{1.964556in}{3.063273in}}%
\pgfpathlineto{\pgfqpoint{1.966561in}{2.980091in}}%
\pgfpathlineto{\pgfqpoint{1.968565in}{3.026455in}}%
\pgfpathlineto{\pgfqpoint{1.970570in}{2.948727in}}%
\pgfpathlineto{\pgfqpoint{1.971572in}{2.947364in}}%
\pgfpathlineto{\pgfqpoint{1.972574in}{2.973273in}}%
\pgfpathlineto{\pgfqpoint{1.974579in}{2.895545in}}%
\pgfpathlineto{\pgfqpoint{1.976583in}{2.976000in}}%
\pgfpathlineto{\pgfqpoint{1.977585in}{2.955545in}}%
\pgfpathlineto{\pgfqpoint{1.978588in}{2.886000in}}%
\pgfpathlineto{\pgfqpoint{1.980592in}{2.932364in}}%
\pgfpathlineto{\pgfqpoint{1.981594in}{2.993727in}}%
\pgfpathlineto{\pgfqpoint{1.983599in}{2.911909in}}%
\pgfpathlineto{\pgfqpoint{1.984601in}{2.903727in}}%
\pgfpathlineto{\pgfqpoint{1.986606in}{2.961000in}}%
\pgfpathlineto{\pgfqpoint{1.987608in}{2.937818in}}%
\pgfpathlineto{\pgfqpoint{1.988610in}{2.965091in}}%
\pgfpathlineto{\pgfqpoint{1.989612in}{2.922818in}}%
\pgfpathlineto{\pgfqpoint{1.991617in}{2.952818in}}%
\pgfpathlineto{\pgfqpoint{1.992619in}{3.019636in}}%
\pgfpathlineto{\pgfqpoint{1.993621in}{3.006000in}}%
\pgfpathlineto{\pgfqpoint{1.994623in}{2.961000in}}%
\pgfpathlineto{\pgfqpoint{1.995626in}{2.962364in}}%
\pgfpathlineto{\pgfqpoint{1.998632in}{3.064636in}}%
\pgfpathlineto{\pgfqpoint{1.999635in}{2.996455in}}%
\pgfpathlineto{\pgfqpoint{2.000637in}{3.003273in}}%
\pgfpathlineto{\pgfqpoint{2.001639in}{2.997818in}}%
\pgfpathlineto{\pgfqpoint{2.003644in}{3.081000in}}%
\pgfpathlineto{\pgfqpoint{2.004646in}{3.071455in}}%
\pgfpathlineto{\pgfqpoint{2.005648in}{3.044182in}}%
\pgfpathlineto{\pgfqpoint{2.006650in}{2.974636in}}%
\pgfpathlineto{\pgfqpoint{2.008655in}{3.121909in}}%
\pgfpathlineto{\pgfqpoint{2.011662in}{2.963727in}}%
\pgfpathlineto{\pgfqpoint{2.013666in}{3.037364in}}%
\pgfpathlineto{\pgfqpoint{2.016673in}{2.961000in}}%
\pgfpathlineto{\pgfqpoint{2.017675in}{2.982818in}}%
\pgfpathlineto{\pgfqpoint{2.019680in}{2.917364in}}%
\pgfpathlineto{\pgfqpoint{2.020682in}{2.946000in}}%
\pgfpathlineto{\pgfqpoint{2.021684in}{2.943273in}}%
\pgfpathlineto{\pgfqpoint{2.024691in}{2.883273in}}%
\pgfpathlineto{\pgfqpoint{2.026695in}{2.943273in}}%
\pgfpathlineto{\pgfqpoint{2.028700in}{2.892818in}}%
\pgfpathlineto{\pgfqpoint{2.029702in}{2.899636in}}%
\pgfpathlineto{\pgfqpoint{2.031706in}{2.958273in}}%
\pgfpathlineto{\pgfqpoint{2.033711in}{2.906455in}}%
\pgfpathlineto{\pgfqpoint{2.036718in}{2.984182in}}%
\pgfpathlineto{\pgfqpoint{2.037720in}{2.981455in}}%
\pgfpathlineto{\pgfqpoint{2.038722in}{2.903727in}}%
\pgfpathlineto{\pgfqpoint{2.040727in}{2.944636in}}%
\pgfpathlineto{\pgfqpoint{2.041729in}{2.977364in}}%
\pgfpathlineto{\pgfqpoint{2.042731in}{3.057818in}}%
\pgfpathlineto{\pgfqpoint{2.043733in}{3.046909in}}%
\pgfpathlineto{\pgfqpoint{2.045738in}{3.007364in}}%
\pgfpathlineto{\pgfqpoint{2.046740in}{2.991000in}}%
\pgfpathlineto{\pgfqpoint{2.048745in}{3.102818in}}%
\pgfpathlineto{\pgfqpoint{2.051751in}{3.018273in}}%
\pgfpathlineto{\pgfqpoint{2.053756in}{3.078273in}}%
\pgfpathlineto{\pgfqpoint{2.054758in}{3.098727in}}%
\pgfpathlineto{\pgfqpoint{2.056763in}{2.943273in}}%
\pgfpathlineto{\pgfqpoint{2.057765in}{2.995091in}}%
\pgfpathlineto{\pgfqpoint{2.058767in}{3.117818in}}%
\pgfpathlineto{\pgfqpoint{2.059769in}{3.113727in}}%
\pgfpathlineto{\pgfqpoint{2.061774in}{2.997818in}}%
\pgfpathlineto{\pgfqpoint{2.062776in}{2.941909in}}%
\pgfpathlineto{\pgfqpoint{2.063778in}{2.950091in}}%
\pgfpathlineto{\pgfqpoint{2.064780in}{3.004636in}}%
\pgfpathlineto{\pgfqpoint{2.067787in}{2.941909in}}%
\pgfpathlineto{\pgfqpoint{2.068789in}{2.865545in}}%
\pgfpathlineto{\pgfqpoint{2.070794in}{2.936455in}}%
\pgfpathlineto{\pgfqpoint{2.071796in}{2.937818in}}%
\pgfpathlineto{\pgfqpoint{2.073801in}{2.854636in}}%
\pgfpathlineto{\pgfqpoint{2.074803in}{2.861455in}}%
\pgfpathlineto{\pgfqpoint{2.076807in}{2.939182in}}%
\pgfpathlineto{\pgfqpoint{2.078812in}{2.836909in}}%
\pgfpathlineto{\pgfqpoint{2.081819in}{2.936455in}}%
\pgfpathlineto{\pgfqpoint{2.083823in}{2.911909in}}%
\pgfpathlineto{\pgfqpoint{2.084825in}{2.913273in}}%
\pgfpathlineto{\pgfqpoint{2.086830in}{2.974636in}}%
\pgfpathlineto{\pgfqpoint{2.087832in}{2.944636in}}%
\pgfpathlineto{\pgfqpoint{2.091841in}{3.027818in}}%
\pgfpathlineto{\pgfqpoint{2.092843in}{3.018273in}}%
\pgfpathlineto{\pgfqpoint{2.093846in}{3.022364in}}%
\pgfpathlineto{\pgfqpoint{2.094848in}{3.090545in}}%
\pgfpathlineto{\pgfqpoint{2.096852in}{3.023727in}}%
\pgfpathlineto{\pgfqpoint{2.099859in}{3.181909in}}%
\pgfpathlineto{\pgfqpoint{2.100861in}{3.022364in}}%
\pgfpathlineto{\pgfqpoint{2.101863in}{3.023727in}}%
\pgfpathlineto{\pgfqpoint{2.103868in}{3.124636in}}%
\pgfpathlineto{\pgfqpoint{2.104870in}{3.115091in}}%
\pgfpathlineto{\pgfqpoint{2.106875in}{3.023727in}}%
\pgfpathlineto{\pgfqpoint{2.107877in}{3.037364in}}%
\pgfpathlineto{\pgfqpoint{2.109881in}{2.980091in}}%
\pgfpathlineto{\pgfqpoint{2.110884in}{2.988273in}}%
\pgfpathlineto{\pgfqpoint{2.111886in}{2.999182in}}%
\pgfpathlineto{\pgfqpoint{2.112888in}{2.939182in}}%
\pgfpathlineto{\pgfqpoint{2.113890in}{2.963727in}}%
\pgfpathlineto{\pgfqpoint{2.115895in}{2.947364in}}%
\pgfpathlineto{\pgfqpoint{2.116897in}{2.941909in}}%
\pgfpathlineto{\pgfqpoint{2.118902in}{2.864182in}}%
\pgfpathlineto{\pgfqpoint{2.120906in}{2.925545in}}%
\pgfpathlineto{\pgfqpoint{2.121908in}{2.925545in}}%
\pgfpathlineto{\pgfqpoint{2.123913in}{2.850545in}}%
\pgfpathlineto{\pgfqpoint{2.124915in}{2.862818in}}%
\pgfpathlineto{\pgfqpoint{2.126920in}{2.921455in}}%
\pgfpathlineto{\pgfqpoint{2.128924in}{2.868273in}}%
\pgfpathlineto{\pgfqpoint{2.130928in}{2.962364in}}%
\pgfpathlineto{\pgfqpoint{2.131931in}{2.959636in}}%
\pgfpathlineto{\pgfqpoint{2.132933in}{2.935091in}}%
\pgfpathlineto{\pgfqpoint{2.134937in}{2.961000in}}%
\pgfpathlineto{\pgfqpoint{2.136942in}{3.018273in}}%
\pgfpathlineto{\pgfqpoint{2.137944in}{3.014182in}}%
\pgfpathlineto{\pgfqpoint{2.138946in}{3.072818in}}%
\pgfpathlineto{\pgfqpoint{2.139949in}{3.063273in}}%
\pgfpathlineto{\pgfqpoint{2.140951in}{3.040091in}}%
\pgfpathlineto{\pgfqpoint{2.142955in}{3.094636in}}%
\pgfpathlineto{\pgfqpoint{2.143958in}{3.304636in}}%
\pgfpathlineto{\pgfqpoint{2.145962in}{3.052364in}}%
\pgfpathlineto{\pgfqpoint{2.146964in}{3.037364in}}%
\pgfpathlineto{\pgfqpoint{2.148969in}{3.258273in}}%
\pgfpathlineto{\pgfqpoint{2.150973in}{3.025091in}}%
\pgfpathlineto{\pgfqpoint{2.151976in}{3.014182in}}%
\pgfpathlineto{\pgfqpoint{2.153980in}{3.079636in}}%
\pgfpathlineto{\pgfqpoint{2.158991in}{2.946000in}}%
\pgfpathlineto{\pgfqpoint{2.159994in}{2.944636in}}%
\pgfpathlineto{\pgfqpoint{2.160996in}{2.958273in}}%
\pgfpathlineto{\pgfqpoint{2.161998in}{2.937818in}}%
\pgfpathlineto{\pgfqpoint{2.164003in}{2.872364in}}%
\pgfpathlineto{\pgfqpoint{2.165005in}{2.880545in}}%
\pgfpathlineto{\pgfqpoint{2.166007in}{2.931000in}}%
\pgfpathlineto{\pgfqpoint{2.167009in}{2.916000in}}%
\pgfpathlineto{\pgfqpoint{2.169014in}{2.830091in}}%
\pgfpathlineto{\pgfqpoint{2.170016in}{2.846455in}}%
\pgfpathlineto{\pgfqpoint{2.172020in}{2.921455in}}%
\pgfpathlineto{\pgfqpoint{2.174025in}{2.851909in}}%
\pgfpathlineto{\pgfqpoint{2.175027in}{2.858727in}}%
\pgfpathlineto{\pgfqpoint{2.177032in}{2.946000in}}%
\pgfpathlineto{\pgfqpoint{2.179036in}{2.896909in}}%
\pgfpathlineto{\pgfqpoint{2.181041in}{2.989636in}}%
\pgfpathlineto{\pgfqpoint{2.182043in}{2.991000in}}%
\pgfpathlineto{\pgfqpoint{2.183045in}{3.004636in}}%
\pgfpathlineto{\pgfqpoint{2.184047in}{2.982818in}}%
\pgfpathlineto{\pgfqpoint{2.186052in}{3.025091in}}%
\pgfpathlineto{\pgfqpoint{2.187054in}{3.019636in}}%
\pgfpathlineto{\pgfqpoint{2.189059in}{3.119182in}}%
\pgfpathlineto{\pgfqpoint{2.192065in}{3.052364in}}%
\pgfpathlineto{\pgfqpoint{2.194070in}{3.382364in}}%
\pgfpathlineto{\pgfqpoint{2.197077in}{3.063273in}}%
\pgfpathlineto{\pgfqpoint{2.199081in}{3.255545in}}%
\pgfpathlineto{\pgfqpoint{2.201085in}{3.044182in}}%
\pgfpathlineto{\pgfqpoint{2.202088in}{3.052364in}}%
\pgfpathlineto{\pgfqpoint{2.203090in}{3.085091in}}%
\pgfpathlineto{\pgfqpoint{2.207099in}{2.963727in}}%
\pgfpathlineto{\pgfqpoint{2.208101in}{2.993727in}}%
\pgfpathlineto{\pgfqpoint{2.209103in}{2.962364in}}%
\pgfpathlineto{\pgfqpoint{2.210106in}{2.976000in}}%
\pgfpathlineto{\pgfqpoint{2.211108in}{2.952818in}}%
\pgfpathlineto{\pgfqpoint{2.214115in}{2.832818in}}%
\pgfpathlineto{\pgfqpoint{2.216119in}{2.922818in}}%
\pgfpathlineto{\pgfqpoint{2.218124in}{2.847818in}}%
\pgfpathlineto{\pgfqpoint{2.219126in}{2.790545in}}%
\pgfpathlineto{\pgfqpoint{2.221130in}{2.907818in}}%
\pgfpathlineto{\pgfqpoint{2.222133in}{2.920091in}}%
\pgfpathlineto{\pgfqpoint{2.224137in}{2.845091in}}%
\pgfpathlineto{\pgfqpoint{2.227144in}{2.951455in}}%
\pgfpathlineto{\pgfqpoint{2.228146in}{2.890091in}}%
\pgfpathlineto{\pgfqpoint{2.229148in}{2.894182in}}%
\pgfpathlineto{\pgfqpoint{2.232155in}{3.003273in}}%
\pgfpathlineto{\pgfqpoint{2.233157in}{3.001909in}}%
\pgfpathlineto{\pgfqpoint{2.234160in}{3.006000in}}%
\pgfpathlineto{\pgfqpoint{2.235162in}{3.034636in}}%
\pgfpathlineto{\pgfqpoint{2.236164in}{3.010091in}}%
\pgfpathlineto{\pgfqpoint{2.237166in}{3.037364in}}%
\pgfpathlineto{\pgfqpoint{2.240173in}{3.251455in}}%
\pgfpathlineto{\pgfqpoint{2.241175in}{3.067364in}}%
\pgfpathlineto{\pgfqpoint{2.242177in}{3.098727in}}%
\pgfpathlineto{\pgfqpoint{2.244182in}{3.461455in}}%
\pgfpathlineto{\pgfqpoint{2.245184in}{3.420545in}}%
\pgfpathlineto{\pgfqpoint{2.247189in}{3.089182in}}%
\pgfpathlineto{\pgfqpoint{2.249193in}{3.524182in}}%
\pgfpathlineto{\pgfqpoint{2.251198in}{3.031909in}}%
\pgfpathlineto{\pgfqpoint{2.253202in}{3.025091in}}%
\pgfpathlineto{\pgfqpoint{2.254204in}{3.115091in}}%
\pgfpathlineto{\pgfqpoint{2.259216in}{2.907818in}}%
\pgfpathlineto{\pgfqpoint{2.261220in}{2.917364in}}%
\pgfpathlineto{\pgfqpoint{2.262222in}{2.947364in}}%
\pgfpathlineto{\pgfqpoint{2.264227in}{2.835545in}}%
\pgfpathlineto{\pgfqpoint{2.267234in}{2.895545in}}%
\pgfpathlineto{\pgfqpoint{2.269238in}{2.813727in}}%
\pgfpathlineto{\pgfqpoint{2.271242in}{2.901000in}}%
\pgfpathlineto{\pgfqpoint{2.272245in}{2.917364in}}%
\pgfpathlineto{\pgfqpoint{2.274249in}{2.836909in}}%
\pgfpathlineto{\pgfqpoint{2.276254in}{2.921455in}}%
\pgfpathlineto{\pgfqpoint{2.277256in}{2.922818in}}%
\pgfpathlineto{\pgfqpoint{2.279260in}{2.898273in}}%
\pgfpathlineto{\pgfqpoint{2.280263in}{2.913273in}}%
\pgfpathlineto{\pgfqpoint{2.282267in}{2.991000in}}%
\pgfpathlineto{\pgfqpoint{2.283269in}{2.966455in}}%
\pgfpathlineto{\pgfqpoint{2.284272in}{3.051000in}}%
\pgfpathlineto{\pgfqpoint{2.286276in}{3.012818in}}%
\pgfpathlineto{\pgfqpoint{2.287278in}{3.044182in}}%
\pgfpathlineto{\pgfqpoint{2.288281in}{3.034636in}}%
\pgfpathlineto{\pgfqpoint{2.289283in}{3.177818in}}%
\pgfpathlineto{\pgfqpoint{2.291287in}{3.070091in}}%
\pgfpathlineto{\pgfqpoint{2.292290in}{3.090545in}}%
\pgfpathlineto{\pgfqpoint{2.293292in}{3.192818in}}%
\pgfpathlineto{\pgfqpoint{2.294294in}{3.503727in}}%
\pgfpathlineto{\pgfqpoint{2.296299in}{3.086455in}}%
\pgfpathlineto{\pgfqpoint{2.297301in}{3.127364in}}%
\pgfpathlineto{\pgfqpoint{2.299305in}{3.529636in}}%
\pgfpathlineto{\pgfqpoint{2.300308in}{3.405545in}}%
\pgfpathlineto{\pgfqpoint{2.302312in}{3.078273in}}%
\pgfpathlineto{\pgfqpoint{2.304317in}{3.173727in}}%
\pgfpathlineto{\pgfqpoint{2.307323in}{2.978727in}}%
\pgfpathlineto{\pgfqpoint{2.308325in}{3.031909in}}%
\pgfpathlineto{\pgfqpoint{2.309328in}{3.021000in}}%
\pgfpathlineto{\pgfqpoint{2.312334in}{2.914636in}}%
\pgfpathlineto{\pgfqpoint{2.313337in}{2.898273in}}%
\pgfpathlineto{\pgfqpoint{2.314339in}{2.924182in}}%
\pgfpathlineto{\pgfqpoint{2.315341in}{2.888727in}}%
\pgfpathlineto{\pgfqpoint{2.316343in}{2.935091in}}%
\pgfpathlineto{\pgfqpoint{2.319350in}{2.816455in}}%
\pgfpathlineto{\pgfqpoint{2.321355in}{2.907818in}}%
\pgfpathlineto{\pgfqpoint{2.322357in}{2.888727in}}%
\pgfpathlineto{\pgfqpoint{2.324361in}{2.820545in}}%
\pgfpathlineto{\pgfqpoint{2.325364in}{2.857364in}}%
\pgfpathlineto{\pgfqpoint{2.326366in}{2.937818in}}%
\pgfpathlineto{\pgfqpoint{2.327368in}{2.914636in}}%
\pgfpathlineto{\pgfqpoint{2.329373in}{2.839636in}}%
\pgfpathlineto{\pgfqpoint{2.331377in}{2.991000in}}%
\pgfpathlineto{\pgfqpoint{2.334384in}{2.905091in}}%
\pgfpathlineto{\pgfqpoint{2.337391in}{3.021000in}}%
\pgfpathlineto{\pgfqpoint{2.338393in}{3.040091in}}%
\pgfpathlineto{\pgfqpoint{2.339395in}{3.022364in}}%
\pgfpathlineto{\pgfqpoint{2.340397in}{3.031909in}}%
\pgfpathlineto{\pgfqpoint{2.341400in}{3.064636in}}%
\pgfpathlineto{\pgfqpoint{2.342402in}{3.055091in}}%
\pgfpathlineto{\pgfqpoint{2.344406in}{3.472364in}}%
\pgfpathlineto{\pgfqpoint{2.346411in}{3.168273in}}%
\pgfpathlineto{\pgfqpoint{2.347413in}{3.184636in}}%
\pgfpathlineto{\pgfqpoint{2.349417in}{3.655091in}}%
\pgfpathlineto{\pgfqpoint{2.352424in}{3.154636in}}%
\pgfpathlineto{\pgfqpoint{2.354429in}{3.601909in}}%
\pgfpathlineto{\pgfqpoint{2.356433in}{3.049636in}}%
\pgfpathlineto{\pgfqpoint{2.357435in}{3.025091in}}%
\pgfpathlineto{\pgfqpoint{2.359440in}{3.104182in}}%
\pgfpathlineto{\pgfqpoint{2.361444in}{2.991000in}}%
\pgfpathlineto{\pgfqpoint{2.365453in}{2.940545in}}%
\pgfpathlineto{\pgfqpoint{2.366456in}{2.941909in}}%
\pgfpathlineto{\pgfqpoint{2.369462in}{2.830091in}}%
\pgfpathlineto{\pgfqpoint{2.371467in}{2.906455in}}%
\pgfpathlineto{\pgfqpoint{2.373471in}{2.800091in}}%
\pgfpathlineto{\pgfqpoint{2.374474in}{2.798727in}}%
\pgfpathlineto{\pgfqpoint{2.375476in}{2.813727in}}%
\pgfpathlineto{\pgfqpoint{2.376478in}{2.890091in}}%
\pgfpathlineto{\pgfqpoint{2.378482in}{2.791909in}}%
\pgfpathlineto{\pgfqpoint{2.379485in}{2.796000in}}%
\pgfpathlineto{\pgfqpoint{2.381489in}{2.921455in}}%
\pgfpathlineto{\pgfqpoint{2.382491in}{2.886000in}}%
\pgfpathlineto{\pgfqpoint{2.383494in}{2.820545in}}%
\pgfpathlineto{\pgfqpoint{2.384496in}{2.847818in}}%
\pgfpathlineto{\pgfqpoint{2.386500in}{2.959636in}}%
\pgfpathlineto{\pgfqpoint{2.387503in}{2.969182in}}%
\pgfpathlineto{\pgfqpoint{2.389507in}{2.916000in}}%
\pgfpathlineto{\pgfqpoint{2.391512in}{3.045545in}}%
\pgfpathlineto{\pgfqpoint{2.392514in}{3.025091in}}%
\pgfpathlineto{\pgfqpoint{2.393516in}{3.131455in}}%
\pgfpathlineto{\pgfqpoint{2.395521in}{3.089182in}}%
\pgfpathlineto{\pgfqpoint{2.396523in}{3.072818in}}%
\pgfpathlineto{\pgfqpoint{2.397525in}{3.134182in}}%
\pgfpathlineto{\pgfqpoint{2.398527in}{3.511909in}}%
\pgfpathlineto{\pgfqpoint{2.399530in}{3.465545in}}%
\pgfpathlineto{\pgfqpoint{2.401534in}{3.085091in}}%
\pgfpathlineto{\pgfqpoint{2.402536in}{3.139636in}}%
\pgfpathlineto{\pgfqpoint{2.404541in}{3.653727in}}%
\pgfpathlineto{\pgfqpoint{2.405543in}{3.510545in}}%
\pgfpathlineto{\pgfqpoint{2.406545in}{3.128727in}}%
\pgfpathlineto{\pgfqpoint{2.407548in}{3.156000in}}%
\pgfpathlineto{\pgfqpoint{2.409552in}{3.642818in}}%
\pgfpathlineto{\pgfqpoint{2.411557in}{3.068727in}}%
\pgfpathlineto{\pgfqpoint{2.412559in}{3.059182in}}%
\pgfpathlineto{\pgfqpoint{2.413561in}{3.016909in}}%
\pgfpathlineto{\pgfqpoint{2.414563in}{3.031909in}}%
\pgfpathlineto{\pgfqpoint{2.417570in}{2.971909in}}%
\pgfpathlineto{\pgfqpoint{2.419574in}{2.839636in}}%
\pgfpathlineto{\pgfqpoint{2.420577in}{2.901000in}}%
\pgfpathlineto{\pgfqpoint{2.421579in}{2.895545in}}%
\pgfpathlineto{\pgfqpoint{2.422581in}{2.890091in}}%
\pgfpathlineto{\pgfqpoint{2.424586in}{2.779636in}}%
\pgfpathlineto{\pgfqpoint{2.426590in}{2.879182in}}%
\pgfpathlineto{\pgfqpoint{2.427592in}{2.860091in}}%
\pgfpathlineto{\pgfqpoint{2.429597in}{2.731909in}}%
\pgfpathlineto{\pgfqpoint{2.431601in}{2.877818in}}%
\pgfpathlineto{\pgfqpoint{2.432604in}{2.857364in}}%
\pgfpathlineto{\pgfqpoint{2.434608in}{2.748273in}}%
\pgfpathlineto{\pgfqpoint{2.436613in}{2.899636in}}%
\pgfpathlineto{\pgfqpoint{2.437615in}{2.888727in}}%
\pgfpathlineto{\pgfqpoint{2.438617in}{2.839636in}}%
\pgfpathlineto{\pgfqpoint{2.441624in}{2.971909in}}%
\pgfpathlineto{\pgfqpoint{2.442626in}{2.970545in}}%
\pgfpathlineto{\pgfqpoint{2.443628in}{2.932364in}}%
\pgfpathlineto{\pgfqpoint{2.449642in}{3.091909in}}%
\pgfpathlineto{\pgfqpoint{2.450644in}{3.108273in}}%
\pgfpathlineto{\pgfqpoint{2.451646in}{3.056455in}}%
\pgfpathlineto{\pgfqpoint{2.454653in}{3.555545in}}%
\pgfpathlineto{\pgfqpoint{2.456657in}{3.106909in}}%
\pgfpathlineto{\pgfqpoint{2.457660in}{3.247364in}}%
\pgfpathlineto{\pgfqpoint{2.459664in}{3.653727in}}%
\pgfpathlineto{\pgfqpoint{2.460666in}{3.543273in}}%
\pgfpathlineto{\pgfqpoint{2.461669in}{3.091909in}}%
\pgfpathlineto{\pgfqpoint{2.462671in}{3.177818in}}%
\pgfpathlineto{\pgfqpoint{2.464675in}{3.595091in}}%
\pgfpathlineto{\pgfqpoint{2.466680in}{3.055091in}}%
\pgfpathlineto{\pgfqpoint{2.468684in}{3.270545in}}%
\pgfpathlineto{\pgfqpoint{2.469687in}{3.195545in}}%
\pgfpathlineto{\pgfqpoint{2.471691in}{2.950091in}}%
\pgfpathlineto{\pgfqpoint{2.473696in}{3.018273in}}%
\pgfpathlineto{\pgfqpoint{2.474698in}{3.004636in}}%
\pgfpathlineto{\pgfqpoint{2.476702in}{2.905091in}}%
\pgfpathlineto{\pgfqpoint{2.478707in}{2.856000in}}%
\pgfpathlineto{\pgfqpoint{2.479709in}{2.881909in}}%
\pgfpathlineto{\pgfqpoint{2.481714in}{2.865545in}}%
\pgfpathlineto{\pgfqpoint{2.482716in}{2.836909in}}%
\pgfpathlineto{\pgfqpoint{2.483718in}{2.748273in}}%
\pgfpathlineto{\pgfqpoint{2.484720in}{2.753727in}}%
\pgfpathlineto{\pgfqpoint{2.486725in}{2.857364in}}%
\pgfpathlineto{\pgfqpoint{2.487727in}{2.820545in}}%
\pgfpathlineto{\pgfqpoint{2.489731in}{2.718273in}}%
\pgfpathlineto{\pgfqpoint{2.491736in}{2.876455in}}%
\pgfpathlineto{\pgfqpoint{2.493740in}{2.752364in}}%
\pgfpathlineto{\pgfqpoint{2.494743in}{2.753727in}}%
\pgfpathlineto{\pgfqpoint{2.496747in}{2.937818in}}%
\pgfpathlineto{\pgfqpoint{2.498752in}{2.849182in}}%
\pgfpathlineto{\pgfqpoint{2.502761in}{3.051000in}}%
\pgfpathlineto{\pgfqpoint{2.503763in}{3.071455in}}%
\pgfpathlineto{\pgfqpoint{2.505767in}{2.969182in}}%
\pgfpathlineto{\pgfqpoint{2.506770in}{2.985545in}}%
\pgfpathlineto{\pgfqpoint{2.507772in}{3.094636in}}%
\pgfpathlineto{\pgfqpoint{2.509776in}{3.667364in}}%
\pgfpathlineto{\pgfqpoint{2.511781in}{3.121909in}}%
\pgfpathlineto{\pgfqpoint{2.513785in}{3.614182in}}%
\pgfpathlineto{\pgfqpoint{2.514788in}{3.656455in}}%
\pgfpathlineto{\pgfqpoint{2.516792in}{3.104182in}}%
\pgfpathlineto{\pgfqpoint{2.518797in}{3.682364in}}%
\pgfpathlineto{\pgfqpoint{2.519799in}{3.749182in}}%
\pgfpathlineto{\pgfqpoint{2.521803in}{3.156000in}}%
\pgfpathlineto{\pgfqpoint{2.522805in}{3.289636in}}%
\pgfpathlineto{\pgfqpoint{2.524810in}{3.636000in}}%
\pgfpathlineto{\pgfqpoint{2.526814in}{3.019636in}}%
\pgfpathlineto{\pgfqpoint{2.528819in}{3.325091in}}%
\pgfpathlineto{\pgfqpoint{2.531826in}{2.931000in}}%
\pgfpathlineto{\pgfqpoint{2.533830in}{3.026455in}}%
\pgfpathlineto{\pgfqpoint{2.534832in}{3.016909in}}%
\pgfpathlineto{\pgfqpoint{2.536837in}{2.896909in}}%
\pgfpathlineto{\pgfqpoint{2.538841in}{2.841000in}}%
\pgfpathlineto{\pgfqpoint{2.539844in}{2.876455in}}%
\pgfpathlineto{\pgfqpoint{2.540846in}{2.839636in}}%
\pgfpathlineto{\pgfqpoint{2.541848in}{2.886000in}}%
\pgfpathlineto{\pgfqpoint{2.542850in}{2.846455in}}%
\pgfpathlineto{\pgfqpoint{2.543853in}{2.712818in}}%
\pgfpathlineto{\pgfqpoint{2.546859in}{2.886000in}}%
\pgfpathlineto{\pgfqpoint{2.548864in}{2.730545in}}%
\pgfpathlineto{\pgfqpoint{2.549866in}{2.744182in}}%
\pgfpathlineto{\pgfqpoint{2.551871in}{2.888727in}}%
\pgfpathlineto{\pgfqpoint{2.552873in}{2.865545in}}%
\pgfpathlineto{\pgfqpoint{2.553875in}{2.736000in}}%
\pgfpathlineto{\pgfqpoint{2.554877in}{2.753727in}}%
\pgfpathlineto{\pgfqpoint{2.556882in}{2.940545in}}%
\pgfpathlineto{\pgfqpoint{2.559888in}{2.850545in}}%
\pgfpathlineto{\pgfqpoint{2.561893in}{2.969182in}}%
\pgfpathlineto{\pgfqpoint{2.562895in}{2.982818in}}%
\pgfpathlineto{\pgfqpoint{2.563897in}{2.973273in}}%
\pgfpathlineto{\pgfqpoint{2.566904in}{3.067364in}}%
\pgfpathlineto{\pgfqpoint{2.567906in}{3.044182in}}%
\pgfpathlineto{\pgfqpoint{2.568909in}{3.091909in}}%
\pgfpathlineto{\pgfqpoint{2.569911in}{3.514636in}}%
\pgfpathlineto{\pgfqpoint{2.571915in}{3.166909in}}%
\pgfpathlineto{\pgfqpoint{2.572918in}{3.258273in}}%
\pgfpathlineto{\pgfqpoint{2.574922in}{3.685091in}}%
\pgfpathlineto{\pgfqpoint{2.576927in}{3.266455in}}%
\pgfpathlineto{\pgfqpoint{2.578931in}{3.745091in}}%
\pgfpathlineto{\pgfqpoint{2.579933in}{3.768273in}}%
\pgfpathlineto{\pgfqpoint{2.580936in}{3.641455in}}%
\pgfpathlineto{\pgfqpoint{2.581938in}{3.344182in}}%
\pgfpathlineto{\pgfqpoint{2.583942in}{3.683727in}}%
\pgfpathlineto{\pgfqpoint{2.584945in}{3.734182in}}%
\pgfpathlineto{\pgfqpoint{2.586949in}{3.066000in}}%
\pgfpathlineto{\pgfqpoint{2.587951in}{3.162818in}}%
\pgfpathlineto{\pgfqpoint{2.588954in}{3.612818in}}%
\pgfpathlineto{\pgfqpoint{2.591960in}{2.986909in}}%
\pgfpathlineto{\pgfqpoint{2.592962in}{3.000545in}}%
\pgfpathlineto{\pgfqpoint{2.593965in}{3.102818in}}%
\pgfpathlineto{\pgfqpoint{2.594967in}{3.070091in}}%
\pgfpathlineto{\pgfqpoint{2.596971in}{2.902364in}}%
\pgfpathlineto{\pgfqpoint{2.597974in}{2.851909in}}%
\pgfpathlineto{\pgfqpoint{2.598976in}{2.890091in}}%
\pgfpathlineto{\pgfqpoint{2.600980in}{2.836909in}}%
\pgfpathlineto{\pgfqpoint{2.601983in}{2.871000in}}%
\pgfpathlineto{\pgfqpoint{2.603987in}{2.716909in}}%
\pgfpathlineto{\pgfqpoint{2.604989in}{2.745545in}}%
\pgfpathlineto{\pgfqpoint{2.606994in}{2.877818in}}%
\pgfpathlineto{\pgfqpoint{2.608998in}{2.708727in}}%
\pgfpathlineto{\pgfqpoint{2.610001in}{2.725091in}}%
\pgfpathlineto{\pgfqpoint{2.612005in}{2.876455in}}%
\pgfpathlineto{\pgfqpoint{2.614010in}{2.755091in}}%
\pgfpathlineto{\pgfqpoint{2.615012in}{2.783727in}}%
\pgfpathlineto{\pgfqpoint{2.617016in}{2.926909in}}%
\pgfpathlineto{\pgfqpoint{2.618019in}{2.913273in}}%
\pgfpathlineto{\pgfqpoint{2.619021in}{2.931000in}}%
\pgfpathlineto{\pgfqpoint{2.620023in}{2.884636in}}%
\pgfpathlineto{\pgfqpoint{2.625034in}{3.145091in}}%
\pgfpathlineto{\pgfqpoint{2.626036in}{3.098727in}}%
\pgfpathlineto{\pgfqpoint{2.628041in}{3.211909in}}%
\pgfpathlineto{\pgfqpoint{2.630045in}{3.711000in}}%
\pgfpathlineto{\pgfqpoint{2.632050in}{3.385091in}}%
\pgfpathlineto{\pgfqpoint{2.634054in}{3.773727in}}%
\pgfpathlineto{\pgfqpoint{2.635057in}{3.783273in}}%
\pgfpathlineto{\pgfqpoint{2.637061in}{3.359182in}}%
\pgfpathlineto{\pgfqpoint{2.639066in}{3.773727in}}%
\pgfpathlineto{\pgfqpoint{2.640068in}{3.719182in}}%
\pgfpathlineto{\pgfqpoint{2.642072in}{3.056455in}}%
\pgfpathlineto{\pgfqpoint{2.643075in}{3.190091in}}%
\pgfpathlineto{\pgfqpoint{2.644077in}{3.513273in}}%
\pgfpathlineto{\pgfqpoint{2.646081in}{3.036000in}}%
\pgfpathlineto{\pgfqpoint{2.648086in}{2.955545in}}%
\pgfpathlineto{\pgfqpoint{2.649088in}{3.033273in}}%
\pgfpathlineto{\pgfqpoint{2.651093in}{2.865545in}}%
\pgfpathlineto{\pgfqpoint{2.652095in}{2.892818in}}%
\pgfpathlineto{\pgfqpoint{2.654099in}{2.789182in}}%
\pgfpathlineto{\pgfqpoint{2.655102in}{2.768727in}}%
\pgfpathlineto{\pgfqpoint{2.656104in}{2.779636in}}%
\pgfpathlineto{\pgfqpoint{2.657106in}{2.871000in}}%
\pgfpathlineto{\pgfqpoint{2.659111in}{2.673273in}}%
\pgfpathlineto{\pgfqpoint{2.660113in}{2.703273in}}%
\pgfpathlineto{\pgfqpoint{2.662117in}{2.873727in}}%
\pgfpathlineto{\pgfqpoint{2.664122in}{2.701909in}}%
\pgfpathlineto{\pgfqpoint{2.665124in}{2.757818in}}%
\pgfpathlineto{\pgfqpoint{2.667128in}{2.950091in}}%
\pgfpathlineto{\pgfqpoint{2.668131in}{2.924182in}}%
\pgfpathlineto{\pgfqpoint{2.670135in}{2.781000in}}%
\pgfpathlineto{\pgfqpoint{2.673142in}{3.142364in}}%
\pgfpathlineto{\pgfqpoint{2.674144in}{3.723273in}}%
\pgfpathlineto{\pgfqpoint{2.675146in}{3.558273in}}%
\pgfpathlineto{\pgfqpoint{2.676149in}{3.059182in}}%
\pgfpathlineto{\pgfqpoint{2.677151in}{3.067364in}}%
\pgfpathlineto{\pgfqpoint{2.678153in}{3.003273in}}%
\pgfpathlineto{\pgfqpoint{2.679155in}{3.690545in}}%
\pgfpathlineto{\pgfqpoint{2.680158in}{3.681000in}}%
\pgfpathlineto{\pgfqpoint{2.682162in}{3.142364in}}%
\pgfpathlineto{\pgfqpoint{2.683164in}{3.132818in}}%
\pgfpathlineto{\pgfqpoint{2.684167in}{3.541909in}}%
\pgfpathlineto{\pgfqpoint{2.685169in}{3.516000in}}%
\pgfpathlineto{\pgfqpoint{2.687173in}{2.961000in}}%
\pgfpathlineto{\pgfqpoint{2.688176in}{3.116455in}}%
\pgfpathlineto{\pgfqpoint{2.689178in}{3.756000in}}%
\pgfpathlineto{\pgfqpoint{2.690180in}{3.719182in}}%
\pgfpathlineto{\pgfqpoint{2.692185in}{2.997818in}}%
\pgfpathlineto{\pgfqpoint{2.693187in}{2.999182in}}%
\pgfpathlineto{\pgfqpoint{2.694189in}{3.004636in}}%
\pgfpathlineto{\pgfqpoint{2.695191in}{3.000545in}}%
\pgfpathlineto{\pgfqpoint{2.697196in}{2.956909in}}%
\pgfpathlineto{\pgfqpoint{2.699200in}{3.036000in}}%
\pgfpathlineto{\pgfqpoint{2.700202in}{2.812364in}}%
\pgfpathlineto{\pgfqpoint{2.701205in}{2.873727in}}%
\pgfpathlineto{\pgfqpoint{2.702207in}{2.864182in}}%
\pgfpathlineto{\pgfqpoint{2.703209in}{2.886000in}}%
\pgfpathlineto{\pgfqpoint{2.704211in}{2.967818in}}%
\pgfpathlineto{\pgfqpoint{2.705214in}{2.858727in}}%
\pgfpathlineto{\pgfqpoint{2.706216in}{2.877818in}}%
\pgfpathlineto{\pgfqpoint{2.707218in}{2.944636in}}%
\pgfpathlineto{\pgfqpoint{2.708220in}{2.871000in}}%
\pgfpathlineto{\pgfqpoint{2.709223in}{2.708727in}}%
\pgfpathlineto{\pgfqpoint{2.710225in}{2.740091in}}%
\pgfpathlineto{\pgfqpoint{2.711227in}{2.913273in}}%
\pgfpathlineto{\pgfqpoint{2.712229in}{2.881909in}}%
\pgfpathlineto{\pgfqpoint{2.714234in}{3.067364in}}%
\pgfpathlineto{\pgfqpoint{2.715236in}{2.839636in}}%
\pgfpathlineto{\pgfqpoint{2.718243in}{2.940545in}}%
\pgfpathlineto{\pgfqpoint{2.719245in}{3.141000in}}%
\pgfpathlineto{\pgfqpoint{2.720247in}{3.580091in}}%
\pgfpathlineto{\pgfqpoint{2.722252in}{3.029182in}}%
\pgfpathlineto{\pgfqpoint{2.723254in}{3.016909in}}%
\pgfpathlineto{\pgfqpoint{2.725259in}{3.217364in}}%
\pgfpathlineto{\pgfqpoint{2.726261in}{3.371455in}}%
\pgfpathlineto{\pgfqpoint{2.727263in}{3.157364in}}%
\pgfpathlineto{\pgfqpoint{2.729268in}{3.730091in}}%
\pgfpathlineto{\pgfqpoint{2.730270in}{3.660545in}}%
\pgfpathlineto{\pgfqpoint{2.732274in}{3.307364in}}%
\pgfpathlineto{\pgfqpoint{2.734279in}{3.779182in}}%
\pgfpathlineto{\pgfqpoint{2.735281in}{3.732818in}}%
\pgfpathlineto{\pgfqpoint{2.737285in}{3.443727in}}%
\pgfpathlineto{\pgfqpoint{2.739290in}{3.726000in}}%
\pgfpathlineto{\pgfqpoint{2.741294in}{2.984182in}}%
\pgfpathlineto{\pgfqpoint{2.742297in}{2.984182in}}%
\pgfpathlineto{\pgfqpoint{2.744301in}{3.640091in}}%
\pgfpathlineto{\pgfqpoint{2.747308in}{2.873727in}}%
\pgfpathlineto{\pgfqpoint{2.750315in}{2.969182in}}%
\pgfpathlineto{\pgfqpoint{2.751317in}{2.958273in}}%
\pgfpathlineto{\pgfqpoint{2.754324in}{2.625545in}}%
\pgfpathlineto{\pgfqpoint{2.756328in}{2.823273in}}%
\pgfpathlineto{\pgfqpoint{2.757330in}{2.806909in}}%
\pgfpathlineto{\pgfqpoint{2.759335in}{2.682818in}}%
\pgfpathlineto{\pgfqpoint{2.760337in}{2.693727in}}%
\pgfpathlineto{\pgfqpoint{2.762342in}{2.817818in}}%
\pgfpathlineto{\pgfqpoint{2.763344in}{2.741455in}}%
\pgfpathlineto{\pgfqpoint{2.764346in}{2.742818in}}%
\pgfpathlineto{\pgfqpoint{2.765348in}{2.789182in}}%
\pgfpathlineto{\pgfqpoint{2.767353in}{2.966455in}}%
\pgfpathlineto{\pgfqpoint{2.769357in}{2.826000in}}%
\pgfpathlineto{\pgfqpoint{2.772364in}{3.022364in}}%
\pgfpathlineto{\pgfqpoint{2.773366in}{3.608727in}}%
\pgfpathlineto{\pgfqpoint{2.775371in}{3.042818in}}%
\pgfpathlineto{\pgfqpoint{2.776373in}{3.056455in}}%
\pgfpathlineto{\pgfqpoint{2.777375in}{3.027818in}}%
\pgfpathlineto{\pgfqpoint{2.779380in}{3.735545in}}%
\pgfpathlineto{\pgfqpoint{2.780382in}{3.772364in}}%
\pgfpathlineto{\pgfqpoint{2.781384in}{3.393273in}}%
\pgfpathlineto{\pgfqpoint{2.785393in}{3.831000in}}%
\pgfpathlineto{\pgfqpoint{2.787398in}{3.619636in}}%
\pgfpathlineto{\pgfqpoint{2.789402in}{3.810545in}}%
\pgfpathlineto{\pgfqpoint{2.790404in}{3.689182in}}%
\pgfpathlineto{\pgfqpoint{2.791407in}{3.121909in}}%
\pgfpathlineto{\pgfqpoint{2.792409in}{3.138273in}}%
\pgfpathlineto{\pgfqpoint{2.794413in}{3.584182in}}%
\pgfpathlineto{\pgfqpoint{2.795416in}{3.413727in}}%
\pgfpathlineto{\pgfqpoint{2.797420in}{2.977364in}}%
\pgfpathlineto{\pgfqpoint{2.798422in}{2.836909in}}%
\pgfpathlineto{\pgfqpoint{2.799425in}{2.881909in}}%
\pgfpathlineto{\pgfqpoint{2.800427in}{2.868273in}}%
\pgfpathlineto{\pgfqpoint{2.801429in}{2.835545in}}%
\pgfpathlineto{\pgfqpoint{2.802431in}{2.845091in}}%
\pgfpathlineto{\pgfqpoint{2.804436in}{2.663727in}}%
\pgfpathlineto{\pgfqpoint{2.807442in}{2.770091in}}%
\pgfpathlineto{\pgfqpoint{2.809447in}{2.614636in}}%
\pgfpathlineto{\pgfqpoint{2.812454in}{2.876455in}}%
\pgfpathlineto{\pgfqpoint{2.814458in}{2.673273in}}%
\pgfpathlineto{\pgfqpoint{2.817465in}{3.003273in}}%
\pgfpathlineto{\pgfqpoint{2.818467in}{2.939182in}}%
\pgfpathlineto{\pgfqpoint{2.820472in}{2.976000in}}%
\pgfpathlineto{\pgfqpoint{2.821474in}{2.992364in}}%
\pgfpathlineto{\pgfqpoint{2.822476in}{3.034636in}}%
\pgfpathlineto{\pgfqpoint{2.823478in}{3.016909in}}%
\pgfpathlineto{\pgfqpoint{2.824481in}{3.657818in}}%
\pgfpathlineto{\pgfqpoint{2.825483in}{3.640091in}}%
\pgfpathlineto{\pgfqpoint{2.826485in}{3.280091in}}%
\pgfpathlineto{\pgfqpoint{2.827487in}{3.349636in}}%
\pgfpathlineto{\pgfqpoint{2.830494in}{3.799636in}}%
\pgfpathlineto{\pgfqpoint{2.831496in}{3.528273in}}%
\pgfpathlineto{\pgfqpoint{2.832499in}{3.533727in}}%
\pgfpathlineto{\pgfqpoint{2.834503in}{3.839182in}}%
\pgfpathlineto{\pgfqpoint{2.835505in}{3.790091in}}%
\pgfpathlineto{\pgfqpoint{2.836508in}{3.619636in}}%
\pgfpathlineto{\pgfqpoint{2.837510in}{3.646909in}}%
\pgfpathlineto{\pgfqpoint{2.839514in}{3.907364in}}%
\pgfpathlineto{\pgfqpoint{2.840516in}{3.705545in}}%
\pgfpathlineto{\pgfqpoint{2.842521in}{3.014182in}}%
\pgfpathlineto{\pgfqpoint{2.844525in}{3.736909in}}%
\pgfpathlineto{\pgfqpoint{2.846530in}{2.991000in}}%
\pgfpathlineto{\pgfqpoint{2.848534in}{2.798727in}}%
\pgfpathlineto{\pgfqpoint{2.850539in}{2.932364in}}%
\pgfpathlineto{\pgfqpoint{2.851541in}{2.969182in}}%
\pgfpathlineto{\pgfqpoint{2.852543in}{2.881909in}}%
\pgfpathlineto{\pgfqpoint{2.854548in}{2.670545in}}%
\pgfpathlineto{\pgfqpoint{2.856552in}{2.869636in}}%
\pgfpathlineto{\pgfqpoint{2.857555in}{2.858727in}}%
\pgfpathlineto{\pgfqpoint{2.859559in}{2.655545in}}%
\pgfpathlineto{\pgfqpoint{2.860561in}{2.697818in}}%
\pgfpathlineto{\pgfqpoint{2.862566in}{2.809636in}}%
\pgfpathlineto{\pgfqpoint{2.864570in}{2.812364in}}%
\pgfpathlineto{\pgfqpoint{2.866575in}{2.916000in}}%
\pgfpathlineto{\pgfqpoint{2.868579in}{2.849182in}}%
\pgfpathlineto{\pgfqpoint{2.870584in}{2.954182in}}%
\pgfpathlineto{\pgfqpoint{2.871586in}{3.113727in}}%
\pgfpathlineto{\pgfqpoint{2.872588in}{3.089182in}}%
\pgfpathlineto{\pgfqpoint{2.873590in}{3.008727in}}%
\pgfpathlineto{\pgfqpoint{2.875595in}{3.278727in}}%
\pgfpathlineto{\pgfqpoint{2.876597in}{3.449182in}}%
\pgfpathlineto{\pgfqpoint{2.877599in}{3.325091in}}%
\pgfpathlineto{\pgfqpoint{2.879604in}{3.708273in}}%
\pgfpathlineto{\pgfqpoint{2.880606in}{3.732818in}}%
\pgfpathlineto{\pgfqpoint{2.882611in}{3.631909in}}%
\pgfpathlineto{\pgfqpoint{2.883613in}{3.691909in}}%
\pgfpathlineto{\pgfqpoint{2.884615in}{3.863727in}}%
\pgfpathlineto{\pgfqpoint{2.885617in}{3.719182in}}%
\pgfpathlineto{\pgfqpoint{2.887622in}{3.239182in}}%
\pgfpathlineto{\pgfqpoint{2.889626in}{3.821455in}}%
\pgfpathlineto{\pgfqpoint{2.891631in}{3.663273in}}%
\pgfpathlineto{\pgfqpoint{2.892633in}{3.082364in}}%
\pgfpathlineto{\pgfqpoint{2.894638in}{3.396000in}}%
\pgfpathlineto{\pgfqpoint{2.898647in}{2.854636in}}%
\pgfpathlineto{\pgfqpoint{2.900651in}{2.961000in}}%
\pgfpathlineto{\pgfqpoint{2.901653in}{2.955545in}}%
\pgfpathlineto{\pgfqpoint{2.903658in}{2.699182in}}%
\pgfpathlineto{\pgfqpoint{2.904660in}{2.669182in}}%
\pgfpathlineto{\pgfqpoint{2.906665in}{2.876455in}}%
\pgfpathlineto{\pgfqpoint{2.907667in}{2.809636in}}%
\pgfpathlineto{\pgfqpoint{2.908669in}{2.666455in}}%
\pgfpathlineto{\pgfqpoint{2.909671in}{2.686909in}}%
\pgfpathlineto{\pgfqpoint{2.910673in}{2.751000in}}%
\pgfpathlineto{\pgfqpoint{2.911676in}{2.913273in}}%
\pgfpathlineto{\pgfqpoint{2.913680in}{2.674636in}}%
\pgfpathlineto{\pgfqpoint{2.914682in}{2.708727in}}%
\pgfpathlineto{\pgfqpoint{2.916687in}{2.997818in}}%
\pgfpathlineto{\pgfqpoint{2.918691in}{2.853273in}}%
\pgfpathlineto{\pgfqpoint{2.919694in}{2.845091in}}%
\pgfpathlineto{\pgfqpoint{2.923703in}{3.126000in}}%
\pgfpathlineto{\pgfqpoint{2.925707in}{3.191455in}}%
\pgfpathlineto{\pgfqpoint{2.926709in}{3.288273in}}%
\pgfpathlineto{\pgfqpoint{2.928714in}{3.775091in}}%
\pgfpathlineto{\pgfqpoint{2.930718in}{3.640091in}}%
\pgfpathlineto{\pgfqpoint{2.931721in}{3.649636in}}%
\pgfpathlineto{\pgfqpoint{2.932723in}{3.711000in}}%
\pgfpathlineto{\pgfqpoint{2.933725in}{3.888273in}}%
\pgfpathlineto{\pgfqpoint{2.934727in}{3.870545in}}%
\pgfpathlineto{\pgfqpoint{2.936732in}{3.618273in}}%
\pgfpathlineto{\pgfqpoint{2.937734in}{3.588273in}}%
\pgfpathlineto{\pgfqpoint{2.939739in}{3.773727in}}%
\pgfpathlineto{\pgfqpoint{2.940741in}{3.716455in}}%
\pgfpathlineto{\pgfqpoint{2.942745in}{3.423273in}}%
\pgfpathlineto{\pgfqpoint{2.943747in}{3.475091in}}%
\pgfpathlineto{\pgfqpoint{2.945752in}{3.003273in}}%
\pgfpathlineto{\pgfqpoint{2.946754in}{2.907818in}}%
\pgfpathlineto{\pgfqpoint{2.948759in}{3.010091in}}%
\pgfpathlineto{\pgfqpoint{2.950763in}{2.763273in}}%
\pgfpathlineto{\pgfqpoint{2.951765in}{2.824636in}}%
\pgfpathlineto{\pgfqpoint{2.954772in}{2.658273in}}%
\pgfpathlineto{\pgfqpoint{2.956777in}{2.787818in}}%
\pgfpathlineto{\pgfqpoint{2.959783in}{2.636455in}}%
\pgfpathlineto{\pgfqpoint{2.962790in}{2.869636in}}%
\pgfpathlineto{\pgfqpoint{2.963792in}{2.775545in}}%
\pgfpathlineto{\pgfqpoint{2.965797in}{2.849182in}}%
\pgfpathlineto{\pgfqpoint{2.967801in}{2.943273in}}%
\pgfpathlineto{\pgfqpoint{2.968804in}{2.928273in}}%
\pgfpathlineto{\pgfqpoint{2.969806in}{2.872364in}}%
\pgfpathlineto{\pgfqpoint{2.971810in}{2.965091in}}%
\pgfpathlineto{\pgfqpoint{2.972813in}{3.149182in}}%
\pgfpathlineto{\pgfqpoint{2.973815in}{3.629182in}}%
\pgfpathlineto{\pgfqpoint{2.975819in}{3.321000in}}%
\pgfpathlineto{\pgfqpoint{2.976822in}{3.220091in}}%
\pgfpathlineto{\pgfqpoint{2.978826in}{3.833727in}}%
\pgfpathlineto{\pgfqpoint{2.981833in}{3.548727in}}%
\pgfpathlineto{\pgfqpoint{2.983837in}{3.953727in}}%
\pgfpathlineto{\pgfqpoint{2.984839in}{3.937364in}}%
\pgfpathlineto{\pgfqpoint{2.986844in}{3.623727in}}%
\pgfpathlineto{\pgfqpoint{2.989851in}{3.851455in}}%
\pgfpathlineto{\pgfqpoint{2.990853in}{3.573273in}}%
\pgfpathlineto{\pgfqpoint{2.991855in}{2.947364in}}%
\pgfpathlineto{\pgfqpoint{2.993860in}{3.277364in}}%
\pgfpathlineto{\pgfqpoint{2.996866in}{2.871000in}}%
\pgfpathlineto{\pgfqpoint{2.997869in}{2.880545in}}%
\pgfpathlineto{\pgfqpoint{2.999873in}{2.836909in}}%
\pgfpathlineto{\pgfqpoint{3.000875in}{2.841000in}}%
\pgfpathlineto{\pgfqpoint{3.004884in}{2.646000in}}%
\pgfpathlineto{\pgfqpoint{3.005887in}{2.854636in}}%
\pgfpathlineto{\pgfqpoint{3.006889in}{2.835545in}}%
\pgfpathlineto{\pgfqpoint{3.009896in}{2.606455in}}%
\pgfpathlineto{\pgfqpoint{3.011900in}{2.865545in}}%
\pgfpathlineto{\pgfqpoint{3.014907in}{2.722364in}}%
\pgfpathlineto{\pgfqpoint{3.017913in}{3.010091in}}%
\pgfpathlineto{\pgfqpoint{3.019918in}{2.876455in}}%
\pgfpathlineto{\pgfqpoint{3.020920in}{2.902364in}}%
\pgfpathlineto{\pgfqpoint{3.021922in}{3.007364in}}%
\pgfpathlineto{\pgfqpoint{3.022925in}{3.469636in}}%
\pgfpathlineto{\pgfqpoint{3.023927in}{3.364636in}}%
\pgfpathlineto{\pgfqpoint{3.024929in}{3.548727in}}%
\pgfpathlineto{\pgfqpoint{3.026934in}{3.060545in}}%
\pgfpathlineto{\pgfqpoint{3.028938in}{3.698727in}}%
\pgfpathlineto{\pgfqpoint{3.029940in}{3.795545in}}%
\pgfpathlineto{\pgfqpoint{3.031945in}{3.543273in}}%
\pgfpathlineto{\pgfqpoint{3.034952in}{4.013727in}}%
\pgfpathlineto{\pgfqpoint{3.036956in}{3.506455in}}%
\pgfpathlineto{\pgfqpoint{3.038961in}{3.869182in}}%
\pgfpathlineto{\pgfqpoint{3.039963in}{3.820091in}}%
\pgfpathlineto{\pgfqpoint{3.041967in}{3.111000in}}%
\pgfpathlineto{\pgfqpoint{3.042970in}{3.548727in}}%
\pgfpathlineto{\pgfqpoint{3.043972in}{3.529636in}}%
\pgfpathlineto{\pgfqpoint{3.044974in}{3.573273in}}%
\pgfpathlineto{\pgfqpoint{3.046979in}{2.901000in}}%
\pgfpathlineto{\pgfqpoint{3.047981in}{2.931000in}}%
\pgfpathlineto{\pgfqpoint{3.048983in}{2.760545in}}%
\pgfpathlineto{\pgfqpoint{3.050987in}{2.847818in}}%
\pgfpathlineto{\pgfqpoint{3.051990in}{2.853273in}}%
\pgfpathlineto{\pgfqpoint{3.053994in}{2.599636in}}%
\pgfpathlineto{\pgfqpoint{3.057001in}{2.781000in}}%
\pgfpathlineto{\pgfqpoint{3.059005in}{2.595545in}}%
\pgfpathlineto{\pgfqpoint{3.061010in}{2.757818in}}%
\pgfpathlineto{\pgfqpoint{3.062012in}{2.892818in}}%
\pgfpathlineto{\pgfqpoint{3.064017in}{2.731909in}}%
\pgfpathlineto{\pgfqpoint{3.065019in}{2.761909in}}%
\pgfpathlineto{\pgfqpoint{3.069028in}{2.939182in}}%
\pgfpathlineto{\pgfqpoint{3.070030in}{2.947364in}}%
\pgfpathlineto{\pgfqpoint{3.071032in}{2.926909in}}%
\pgfpathlineto{\pgfqpoint{3.072035in}{3.072818in}}%
\pgfpathlineto{\pgfqpoint{3.073037in}{3.042818in}}%
\pgfpathlineto{\pgfqpoint{3.074039in}{3.119182in}}%
\pgfpathlineto{\pgfqpoint{3.075041in}{3.622364in}}%
\pgfpathlineto{\pgfqpoint{3.076044in}{3.040091in}}%
\pgfpathlineto{\pgfqpoint{3.078048in}{3.657818in}}%
\pgfpathlineto{\pgfqpoint{3.079050in}{3.686455in}}%
\pgfpathlineto{\pgfqpoint{3.080053in}{3.822818in}}%
\pgfpathlineto{\pgfqpoint{3.082057in}{3.687818in}}%
\pgfpathlineto{\pgfqpoint{3.085064in}{4.056000in}}%
\pgfpathlineto{\pgfqpoint{3.087068in}{3.713727in}}%
\pgfpathlineto{\pgfqpoint{3.088070in}{3.779182in}}%
\pgfpathlineto{\pgfqpoint{3.089073in}{3.942818in}}%
\pgfpathlineto{\pgfqpoint{3.090075in}{3.931909in}}%
\pgfpathlineto{\pgfqpoint{3.093082in}{3.007364in}}%
\pgfpathlineto{\pgfqpoint{3.094084in}{3.551455in}}%
\pgfpathlineto{\pgfqpoint{3.095086in}{3.398727in}}%
\pgfpathlineto{\pgfqpoint{3.096088in}{2.969182in}}%
\pgfpathlineto{\pgfqpoint{3.097091in}{3.018273in}}%
\pgfpathlineto{\pgfqpoint{3.099095in}{2.819182in}}%
\pgfpathlineto{\pgfqpoint{3.101100in}{2.851909in}}%
\pgfpathlineto{\pgfqpoint{3.102102in}{2.980091in}}%
\pgfpathlineto{\pgfqpoint{3.104106in}{2.654182in}}%
\pgfpathlineto{\pgfqpoint{3.105109in}{2.678727in}}%
\pgfpathlineto{\pgfqpoint{3.107113in}{2.872364in}}%
\pgfpathlineto{\pgfqpoint{3.109118in}{2.659636in}}%
\pgfpathlineto{\pgfqpoint{3.110120in}{2.701909in}}%
\pgfpathlineto{\pgfqpoint{3.112124in}{2.890091in}}%
\pgfpathlineto{\pgfqpoint{3.114129in}{2.781000in}}%
\pgfpathlineto{\pgfqpoint{3.115131in}{2.826000in}}%
\pgfpathlineto{\pgfqpoint{3.117136in}{2.993727in}}%
\pgfpathlineto{\pgfqpoint{3.118138in}{2.830091in}}%
\pgfpathlineto{\pgfqpoint{3.119140in}{2.842364in}}%
\pgfpathlineto{\pgfqpoint{3.121144in}{2.969182in}}%
\pgfpathlineto{\pgfqpoint{3.122147in}{3.194182in}}%
\pgfpathlineto{\pgfqpoint{3.123149in}{3.011455in}}%
\pgfpathlineto{\pgfqpoint{3.124151in}{3.149182in}}%
\pgfpathlineto{\pgfqpoint{3.125153in}{3.045545in}}%
\pgfpathlineto{\pgfqpoint{3.126156in}{3.059182in}}%
\pgfpathlineto{\pgfqpoint{3.129162in}{3.713727in}}%
\pgfpathlineto{\pgfqpoint{3.131167in}{3.660545in}}%
\pgfpathlineto{\pgfqpoint{3.132169in}{3.476455in}}%
\pgfpathlineto{\pgfqpoint{3.135176in}{3.976909in}}%
\pgfpathlineto{\pgfqpoint{3.137180in}{3.691909in}}%
\pgfpathlineto{\pgfqpoint{3.139185in}{3.929182in}}%
\pgfpathlineto{\pgfqpoint{3.141189in}{3.784636in}}%
\pgfpathlineto{\pgfqpoint{3.142192in}{3.603273in}}%
\pgfpathlineto{\pgfqpoint{3.143194in}{3.083727in}}%
\pgfpathlineto{\pgfqpoint{3.144196in}{3.705545in}}%
\pgfpathlineto{\pgfqpoint{3.145198in}{3.168273in}}%
\pgfpathlineto{\pgfqpoint{3.146201in}{3.233727in}}%
\pgfpathlineto{\pgfqpoint{3.148205in}{2.830091in}}%
\pgfpathlineto{\pgfqpoint{3.149207in}{2.835545in}}%
\pgfpathlineto{\pgfqpoint{3.150210in}{2.800091in}}%
\pgfpathlineto{\pgfqpoint{3.151212in}{2.911909in}}%
\pgfpathlineto{\pgfqpoint{3.152214in}{2.910545in}}%
\pgfpathlineto{\pgfqpoint{3.154219in}{2.682818in}}%
\pgfpathlineto{\pgfqpoint{3.155221in}{2.663727in}}%
\pgfpathlineto{\pgfqpoint{3.157225in}{2.831455in}}%
\pgfpathlineto{\pgfqpoint{3.159230in}{2.663727in}}%
\pgfpathlineto{\pgfqpoint{3.162236in}{2.937818in}}%
\pgfpathlineto{\pgfqpoint{3.164241in}{2.723727in}}%
\pgfpathlineto{\pgfqpoint{3.165243in}{2.767364in}}%
\pgfpathlineto{\pgfqpoint{3.166245in}{2.895545in}}%
\pgfpathlineto{\pgfqpoint{3.167248in}{2.886000in}}%
\pgfpathlineto{\pgfqpoint{3.168250in}{2.931000in}}%
\pgfpathlineto{\pgfqpoint{3.170254in}{2.881909in}}%
\pgfpathlineto{\pgfqpoint{3.171257in}{3.014182in}}%
\pgfpathlineto{\pgfqpoint{3.172259in}{3.006000in}}%
\pgfpathlineto{\pgfqpoint{3.173261in}{3.076909in}}%
\pgfpathlineto{\pgfqpoint{3.174263in}{3.241909in}}%
\pgfpathlineto{\pgfqpoint{3.175266in}{3.097364in}}%
\pgfpathlineto{\pgfqpoint{3.177270in}{3.202364in}}%
\pgfpathlineto{\pgfqpoint{3.179275in}{3.749182in}}%
\pgfpathlineto{\pgfqpoint{3.180277in}{3.701455in}}%
\pgfpathlineto{\pgfqpoint{3.181279in}{3.539182in}}%
\pgfpathlineto{\pgfqpoint{3.182281in}{3.180545in}}%
\pgfpathlineto{\pgfqpoint{3.184286in}{3.841909in}}%
\pgfpathlineto{\pgfqpoint{3.186290in}{3.743727in}}%
\pgfpathlineto{\pgfqpoint{3.187293in}{3.194182in}}%
\pgfpathlineto{\pgfqpoint{3.189297in}{3.944182in}}%
\pgfpathlineto{\pgfqpoint{3.191302in}{3.843273in}}%
\pgfpathlineto{\pgfqpoint{3.192304in}{3.228273in}}%
\pgfpathlineto{\pgfqpoint{3.193306in}{3.569182in}}%
\pgfpathlineto{\pgfqpoint{3.195310in}{3.029182in}}%
\pgfpathlineto{\pgfqpoint{3.196313in}{3.104182in}}%
\pgfpathlineto{\pgfqpoint{3.197315in}{2.907818in}}%
\pgfpathlineto{\pgfqpoint{3.198317in}{2.961000in}}%
\pgfpathlineto{\pgfqpoint{3.200322in}{2.816455in}}%
\pgfpathlineto{\pgfqpoint{3.201324in}{2.898273in}}%
\pgfpathlineto{\pgfqpoint{3.204331in}{2.742818in}}%
\pgfpathlineto{\pgfqpoint{3.205333in}{2.729182in}}%
\pgfpathlineto{\pgfqpoint{3.206335in}{2.898273in}}%
\pgfpathlineto{\pgfqpoint{3.207337in}{2.860091in}}%
\pgfpathlineto{\pgfqpoint{3.209342in}{2.677364in}}%
\pgfpathlineto{\pgfqpoint{3.211346in}{2.888727in}}%
\pgfpathlineto{\pgfqpoint{3.212349in}{2.835545in}}%
\pgfpathlineto{\pgfqpoint{3.213351in}{2.831455in}}%
\pgfpathlineto{\pgfqpoint{3.214353in}{2.786455in}}%
\pgfpathlineto{\pgfqpoint{3.216358in}{3.006000in}}%
\pgfpathlineto{\pgfqpoint{3.218362in}{2.935091in}}%
\pgfpathlineto{\pgfqpoint{3.219364in}{2.865545in}}%
\pgfpathlineto{\pgfqpoint{3.222371in}{3.093273in}}%
\pgfpathlineto{\pgfqpoint{3.223373in}{3.348273in}}%
\pgfpathlineto{\pgfqpoint{3.225378in}{3.059182in}}%
\pgfpathlineto{\pgfqpoint{3.226380in}{3.078273in}}%
\pgfpathlineto{\pgfqpoint{3.227382in}{3.150545in}}%
\pgfpathlineto{\pgfqpoint{3.229387in}{3.867818in}}%
\pgfpathlineto{\pgfqpoint{3.232393in}{3.623727in}}%
\pgfpathlineto{\pgfqpoint{3.234398in}{3.933273in}}%
\pgfpathlineto{\pgfqpoint{3.235400in}{3.956455in}}%
\pgfpathlineto{\pgfqpoint{3.236402in}{3.670091in}}%
\pgfpathlineto{\pgfqpoint{3.237405in}{3.672818in}}%
\pgfpathlineto{\pgfqpoint{3.239409in}{3.916909in}}%
\pgfpathlineto{\pgfqpoint{3.240411in}{3.955091in}}%
\pgfpathlineto{\pgfqpoint{3.242416in}{3.131455in}}%
\pgfpathlineto{\pgfqpoint{3.243418in}{3.322364in}}%
\pgfpathlineto{\pgfqpoint{3.244420in}{3.102818in}}%
\pgfpathlineto{\pgfqpoint{3.245423in}{3.258273in}}%
\pgfpathlineto{\pgfqpoint{3.248429in}{2.898273in}}%
\pgfpathlineto{\pgfqpoint{3.249432in}{2.808273in}}%
\pgfpathlineto{\pgfqpoint{3.250434in}{2.943273in}}%
\pgfpathlineto{\pgfqpoint{3.253441in}{2.772818in}}%
\pgfpathlineto{\pgfqpoint{3.254443in}{2.650091in}}%
\pgfpathlineto{\pgfqpoint{3.257450in}{2.894182in}}%
\pgfpathlineto{\pgfqpoint{3.259454in}{2.714182in}}%
\pgfpathlineto{\pgfqpoint{3.262461in}{2.941909in}}%
\pgfpathlineto{\pgfqpoint{3.264465in}{2.800091in}}%
\pgfpathlineto{\pgfqpoint{3.267472in}{2.971909in}}%
\pgfpathlineto{\pgfqpoint{3.268474in}{2.931000in}}%
\pgfpathlineto{\pgfqpoint{3.269476in}{2.974636in}}%
\pgfpathlineto{\pgfqpoint{3.271481in}{2.937818in}}%
\pgfpathlineto{\pgfqpoint{3.272483in}{3.051000in}}%
\pgfpathlineto{\pgfqpoint{3.273485in}{3.329182in}}%
\pgfpathlineto{\pgfqpoint{3.274488in}{3.146455in}}%
\pgfpathlineto{\pgfqpoint{3.275490in}{3.236455in}}%
\pgfpathlineto{\pgfqpoint{3.276492in}{3.033273in}}%
\pgfpathlineto{\pgfqpoint{3.278497in}{3.780545in}}%
\pgfpathlineto{\pgfqpoint{3.279499in}{3.871909in}}%
\pgfpathlineto{\pgfqpoint{3.280501in}{3.798273in}}%
\pgfpathlineto{\pgfqpoint{3.281503in}{3.390545in}}%
\pgfpathlineto{\pgfqpoint{3.283508in}{3.809182in}}%
\pgfpathlineto{\pgfqpoint{3.284510in}{3.981000in}}%
\pgfpathlineto{\pgfqpoint{3.286515in}{3.438273in}}%
\pgfpathlineto{\pgfqpoint{3.288519in}{3.938727in}}%
\pgfpathlineto{\pgfqpoint{3.289521in}{4.046455in}}%
\pgfpathlineto{\pgfqpoint{3.290524in}{3.851455in}}%
\pgfpathlineto{\pgfqpoint{3.292528in}{3.100091in}}%
\pgfpathlineto{\pgfqpoint{3.293530in}{3.021000in}}%
\pgfpathlineto{\pgfqpoint{3.294533in}{3.631909in}}%
\pgfpathlineto{\pgfqpoint{3.295535in}{3.471000in}}%
\pgfpathlineto{\pgfqpoint{3.296537in}{3.018273in}}%
\pgfpathlineto{\pgfqpoint{3.297539in}{3.066000in}}%
\pgfpathlineto{\pgfqpoint{3.299544in}{2.806909in}}%
\pgfpathlineto{\pgfqpoint{3.300546in}{2.896909in}}%
\pgfpathlineto{\pgfqpoint{3.301548in}{2.868273in}}%
\pgfpathlineto{\pgfqpoint{3.302550in}{2.941909in}}%
\pgfpathlineto{\pgfqpoint{3.304555in}{2.730545in}}%
\pgfpathlineto{\pgfqpoint{3.306559in}{2.895545in}}%
\pgfpathlineto{\pgfqpoint{3.307562in}{2.861455in}}%
\pgfpathlineto{\pgfqpoint{3.309566in}{2.706000in}}%
\pgfpathlineto{\pgfqpoint{3.312573in}{2.902364in}}%
\pgfpathlineto{\pgfqpoint{3.314577in}{2.808273in}}%
\pgfpathlineto{\pgfqpoint{3.316582in}{2.860091in}}%
\pgfpathlineto{\pgfqpoint{3.319589in}{3.056455in}}%
\pgfpathlineto{\pgfqpoint{3.320591in}{2.939182in}}%
\pgfpathlineto{\pgfqpoint{3.321593in}{2.951455in}}%
\pgfpathlineto{\pgfqpoint{3.322595in}{3.034636in}}%
\pgfpathlineto{\pgfqpoint{3.323598in}{2.978727in}}%
\pgfpathlineto{\pgfqpoint{3.324600in}{3.311455in}}%
\pgfpathlineto{\pgfqpoint{3.326604in}{3.066000in}}%
\pgfpathlineto{\pgfqpoint{3.327607in}{3.360545in}}%
\pgfpathlineto{\pgfqpoint{3.328609in}{3.248727in}}%
\pgfpathlineto{\pgfqpoint{3.329611in}{3.736909in}}%
\pgfpathlineto{\pgfqpoint{3.331616in}{3.559636in}}%
\pgfpathlineto{\pgfqpoint{3.332618in}{3.659182in}}%
\pgfpathlineto{\pgfqpoint{3.334622in}{4.016455in}}%
\pgfpathlineto{\pgfqpoint{3.336627in}{3.610091in}}%
\pgfpathlineto{\pgfqpoint{3.337629in}{3.656455in}}%
\pgfpathlineto{\pgfqpoint{3.339633in}{3.892364in}}%
\pgfpathlineto{\pgfqpoint{3.341638in}{3.104182in}}%
\pgfpathlineto{\pgfqpoint{3.343642in}{3.741000in}}%
\pgfpathlineto{\pgfqpoint{3.344645in}{3.798273in}}%
\pgfpathlineto{\pgfqpoint{3.345647in}{3.012818in}}%
\pgfpathlineto{\pgfqpoint{3.346649in}{3.056455in}}%
\pgfpathlineto{\pgfqpoint{3.347651in}{2.903727in}}%
\pgfpathlineto{\pgfqpoint{3.348654in}{2.905091in}}%
\pgfpathlineto{\pgfqpoint{3.349656in}{3.411000in}}%
\pgfpathlineto{\pgfqpoint{3.350658in}{2.963727in}}%
\pgfpathlineto{\pgfqpoint{3.351660in}{2.985545in}}%
\pgfpathlineto{\pgfqpoint{3.353665in}{2.748273in}}%
\pgfpathlineto{\pgfqpoint{3.356672in}{2.922818in}}%
\pgfpathlineto{\pgfqpoint{3.358676in}{2.782364in}}%
\pgfpathlineto{\pgfqpoint{3.359678in}{2.766000in}}%
\pgfpathlineto{\pgfqpoint{3.361683in}{2.944636in}}%
\pgfpathlineto{\pgfqpoint{3.364690in}{2.786455in}}%
\pgfpathlineto{\pgfqpoint{3.366694in}{2.976000in}}%
\pgfpathlineto{\pgfqpoint{3.367696in}{2.898273in}}%
\pgfpathlineto{\pgfqpoint{3.368698in}{2.902364in}}%
\pgfpathlineto{\pgfqpoint{3.369701in}{2.909182in}}%
\pgfpathlineto{\pgfqpoint{3.370703in}{2.843727in}}%
\pgfpathlineto{\pgfqpoint{3.371705in}{3.018273in}}%
\pgfpathlineto{\pgfqpoint{3.372707in}{2.895545in}}%
\pgfpathlineto{\pgfqpoint{3.374712in}{3.211909in}}%
\pgfpathlineto{\pgfqpoint{3.375714in}{3.019636in}}%
\pgfpathlineto{\pgfqpoint{3.376716in}{3.121909in}}%
\pgfpathlineto{\pgfqpoint{3.377719in}{3.037364in}}%
\pgfpathlineto{\pgfqpoint{3.379723in}{3.749182in}}%
\pgfpathlineto{\pgfqpoint{3.381728in}{3.259636in}}%
\pgfpathlineto{\pgfqpoint{3.384734in}{3.982364in}}%
\pgfpathlineto{\pgfqpoint{3.387741in}{3.686455in}}%
\pgfpathlineto{\pgfqpoint{3.388743in}{3.931909in}}%
\pgfpathlineto{\pgfqpoint{3.389746in}{3.895091in}}%
\pgfpathlineto{\pgfqpoint{3.391750in}{3.736909in}}%
\pgfpathlineto{\pgfqpoint{3.392752in}{3.195545in}}%
\pgfpathlineto{\pgfqpoint{3.393755in}{3.747818in}}%
\pgfpathlineto{\pgfqpoint{3.394757in}{3.682364in}}%
\pgfpathlineto{\pgfqpoint{3.395759in}{3.428727in}}%
\pgfpathlineto{\pgfqpoint{3.396761in}{3.457364in}}%
\pgfpathlineto{\pgfqpoint{3.397764in}{2.950091in}}%
\pgfpathlineto{\pgfqpoint{3.398766in}{3.250091in}}%
\pgfpathlineto{\pgfqpoint{3.399768in}{2.991000in}}%
\pgfpathlineto{\pgfqpoint{3.400770in}{3.011455in}}%
\pgfpathlineto{\pgfqpoint{3.401773in}{3.037364in}}%
\pgfpathlineto{\pgfqpoint{3.403777in}{2.813727in}}%
\pgfpathlineto{\pgfqpoint{3.406784in}{2.969182in}}%
\pgfpathlineto{\pgfqpoint{3.408788in}{2.821909in}}%
\pgfpathlineto{\pgfqpoint{3.409790in}{2.808273in}}%
\pgfpathlineto{\pgfqpoint{3.411795in}{2.917364in}}%
\pgfpathlineto{\pgfqpoint{3.412797in}{2.839636in}}%
\pgfpathlineto{\pgfqpoint{3.413799in}{2.849182in}}%
\pgfpathlineto{\pgfqpoint{3.414802in}{2.790545in}}%
\pgfpathlineto{\pgfqpoint{3.416806in}{2.944636in}}%
\pgfpathlineto{\pgfqpoint{3.419813in}{2.793273in}}%
\pgfpathlineto{\pgfqpoint{3.421817in}{2.967818in}}%
\pgfpathlineto{\pgfqpoint{3.422820in}{2.914636in}}%
\pgfpathlineto{\pgfqpoint{3.423822in}{2.999182in}}%
\pgfpathlineto{\pgfqpoint{3.424824in}{2.887364in}}%
\pgfpathlineto{\pgfqpoint{3.427831in}{3.051000in}}%
\pgfpathlineto{\pgfqpoint{3.428833in}{3.559636in}}%
\pgfpathlineto{\pgfqpoint{3.430838in}{3.104182in}}%
\pgfpathlineto{\pgfqpoint{3.431840in}{3.091909in}}%
\pgfpathlineto{\pgfqpoint{3.433844in}{3.897818in}}%
\pgfpathlineto{\pgfqpoint{3.436851in}{3.701455in}}%
\pgfpathlineto{\pgfqpoint{3.438856in}{4.012364in}}%
\pgfpathlineto{\pgfqpoint{3.441862in}{3.537818in}}%
\pgfpathlineto{\pgfqpoint{3.442864in}{3.816000in}}%
\pgfpathlineto{\pgfqpoint{3.443867in}{3.753273in}}%
\pgfpathlineto{\pgfqpoint{3.444869in}{3.588273in}}%
\pgfpathlineto{\pgfqpoint{3.445871in}{3.611455in}}%
\pgfpathlineto{\pgfqpoint{3.447876in}{2.971909in}}%
\pgfpathlineto{\pgfqpoint{3.448878in}{3.254182in}}%
\pgfpathlineto{\pgfqpoint{3.452887in}{2.888727in}}%
\pgfpathlineto{\pgfqpoint{3.453889in}{2.876455in}}%
\pgfpathlineto{\pgfqpoint{3.454891in}{2.821909in}}%
\pgfpathlineto{\pgfqpoint{3.455894in}{2.914636in}}%
\pgfpathlineto{\pgfqpoint{3.457898in}{2.875091in}}%
\pgfpathlineto{\pgfqpoint{3.459903in}{2.763273in}}%
\pgfpathlineto{\pgfqpoint{3.460905in}{2.909182in}}%
\pgfpathlineto{\pgfqpoint{3.461907in}{2.861455in}}%
\pgfpathlineto{\pgfqpoint{3.462909in}{2.892818in}}%
\pgfpathlineto{\pgfqpoint{3.463912in}{2.805545in}}%
\pgfpathlineto{\pgfqpoint{3.464914in}{2.830091in}}%
\pgfpathlineto{\pgfqpoint{3.465916in}{3.016909in}}%
\pgfpathlineto{\pgfqpoint{3.466918in}{2.905091in}}%
\pgfpathlineto{\pgfqpoint{3.467921in}{2.962364in}}%
\pgfpathlineto{\pgfqpoint{3.469925in}{2.861455in}}%
\pgfpathlineto{\pgfqpoint{3.472932in}{3.027818in}}%
\pgfpathlineto{\pgfqpoint{3.473934in}{3.010091in}}%
\pgfpathlineto{\pgfqpoint{3.474936in}{2.932364in}}%
\pgfpathlineto{\pgfqpoint{3.476941in}{3.011455in}}%
\pgfpathlineto{\pgfqpoint{3.478945in}{3.698727in}}%
\pgfpathlineto{\pgfqpoint{3.481952in}{3.274636in}}%
\pgfpathlineto{\pgfqpoint{3.483956in}{3.833727in}}%
\pgfpathlineto{\pgfqpoint{3.484959in}{3.877364in}}%
\pgfpathlineto{\pgfqpoint{3.486963in}{3.442364in}}%
\pgfpathlineto{\pgfqpoint{3.487965in}{3.847364in}}%
\pgfpathlineto{\pgfqpoint{3.488968in}{3.817364in}}%
\pgfpathlineto{\pgfqpoint{3.489970in}{3.956455in}}%
\pgfpathlineto{\pgfqpoint{3.490972in}{3.719182in}}%
\pgfpathlineto{\pgfqpoint{3.491974in}{3.199636in}}%
\pgfpathlineto{\pgfqpoint{3.492977in}{3.769636in}}%
\pgfpathlineto{\pgfqpoint{3.493979in}{3.510545in}}%
\pgfpathlineto{\pgfqpoint{3.494981in}{3.700091in}}%
\pgfpathlineto{\pgfqpoint{3.495983in}{3.600545in}}%
\pgfpathlineto{\pgfqpoint{3.497988in}{3.045545in}}%
\pgfpathlineto{\pgfqpoint{3.499992in}{2.978727in}}%
\pgfpathlineto{\pgfqpoint{3.500995in}{3.061909in}}%
\pgfpathlineto{\pgfqpoint{3.504001in}{2.830091in}}%
\pgfpathlineto{\pgfqpoint{3.507008in}{2.913273in}}%
\pgfpathlineto{\pgfqpoint{3.509013in}{2.771455in}}%
\pgfpathlineto{\pgfqpoint{3.511017in}{2.876455in}}%
\pgfpathlineto{\pgfqpoint{3.512019in}{2.931000in}}%
\pgfpathlineto{\pgfqpoint{3.514024in}{2.768727in}}%
\pgfpathlineto{\pgfqpoint{3.515026in}{2.902364in}}%
\pgfpathlineto{\pgfqpoint{3.516028in}{2.860091in}}%
\pgfpathlineto{\pgfqpoint{3.518033in}{2.905091in}}%
\pgfpathlineto{\pgfqpoint{3.519035in}{2.790545in}}%
\pgfpathlineto{\pgfqpoint{3.520037in}{2.903727in}}%
\pgfpathlineto{\pgfqpoint{3.521039in}{2.873727in}}%
\pgfpathlineto{\pgfqpoint{3.522042in}{2.898273in}}%
\pgfpathlineto{\pgfqpoint{3.523044in}{3.007364in}}%
\pgfpathlineto{\pgfqpoint{3.524046in}{2.984182in}}%
\pgfpathlineto{\pgfqpoint{3.525048in}{3.010091in}}%
\pgfpathlineto{\pgfqpoint{3.526051in}{2.980091in}}%
\pgfpathlineto{\pgfqpoint{3.527053in}{3.104182in}}%
\pgfpathlineto{\pgfqpoint{3.528055in}{3.524182in}}%
\pgfpathlineto{\pgfqpoint{3.529057in}{3.381000in}}%
\pgfpathlineto{\pgfqpoint{3.530060in}{3.708273in}}%
\pgfpathlineto{\pgfqpoint{3.531062in}{3.156000in}}%
\pgfpathlineto{\pgfqpoint{3.532064in}{3.629182in}}%
\pgfpathlineto{\pgfqpoint{3.533066in}{3.622364in}}%
\pgfpathlineto{\pgfqpoint{3.535071in}{3.948273in}}%
\pgfpathlineto{\pgfqpoint{3.536073in}{3.623727in}}%
\pgfpathlineto{\pgfqpoint{3.540082in}{4.028727in}}%
\pgfpathlineto{\pgfqpoint{3.542087in}{3.522818in}}%
\pgfpathlineto{\pgfqpoint{3.543089in}{3.741000in}}%
\pgfpathlineto{\pgfqpoint{3.544091in}{3.713727in}}%
\pgfpathlineto{\pgfqpoint{3.545093in}{3.859636in}}%
\pgfpathlineto{\pgfqpoint{3.547098in}{3.175091in}}%
\pgfpathlineto{\pgfqpoint{3.549102in}{3.045545in}}%
\pgfpathlineto{\pgfqpoint{3.550104in}{3.255545in}}%
\pgfpathlineto{\pgfqpoint{3.552109in}{3.040091in}}%
\pgfpathlineto{\pgfqpoint{3.554113in}{2.843727in}}%
\pgfpathlineto{\pgfqpoint{3.556118in}{2.970545in}}%
\pgfpathlineto{\pgfqpoint{3.557120in}{2.959636in}}%
\pgfpathlineto{\pgfqpoint{3.558122in}{2.827364in}}%
\pgfpathlineto{\pgfqpoint{3.562131in}{2.903727in}}%
\pgfpathlineto{\pgfqpoint{3.564136in}{2.802818in}}%
\pgfpathlineto{\pgfqpoint{3.565138in}{2.864182in}}%
\pgfpathlineto{\pgfqpoint{3.566140in}{2.839636in}}%
\pgfpathlineto{\pgfqpoint{3.567143in}{2.941909in}}%
\pgfpathlineto{\pgfqpoint{3.569147in}{2.828727in}}%
\pgfpathlineto{\pgfqpoint{3.570149in}{2.910545in}}%
\pgfpathlineto{\pgfqpoint{3.571152in}{2.811000in}}%
\pgfpathlineto{\pgfqpoint{3.572154in}{2.963727in}}%
\pgfpathlineto{\pgfqpoint{3.573156in}{2.876455in}}%
\pgfpathlineto{\pgfqpoint{3.575161in}{2.921455in}}%
\pgfpathlineto{\pgfqpoint{3.576163in}{2.892818in}}%
\pgfpathlineto{\pgfqpoint{3.577165in}{3.034636in}}%
\pgfpathlineto{\pgfqpoint{3.578167in}{2.995091in}}%
\pgfpathlineto{\pgfqpoint{3.579170in}{3.123273in}}%
\pgfpathlineto{\pgfqpoint{3.580172in}{3.119182in}}%
\pgfpathlineto{\pgfqpoint{3.581174in}{3.056455in}}%
\pgfpathlineto{\pgfqpoint{3.583178in}{3.364636in}}%
\pgfpathlineto{\pgfqpoint{3.584181in}{3.777818in}}%
\pgfpathlineto{\pgfqpoint{3.585183in}{3.700091in}}%
\pgfpathlineto{\pgfqpoint{3.586185in}{3.453273in}}%
\pgfpathlineto{\pgfqpoint{3.589192in}{3.931909in}}%
\pgfpathlineto{\pgfqpoint{3.590194in}{3.839182in}}%
\pgfpathlineto{\pgfqpoint{3.591196in}{3.606000in}}%
\pgfpathlineto{\pgfqpoint{3.592199in}{3.859636in}}%
\pgfpathlineto{\pgfqpoint{3.593201in}{3.757364in}}%
\pgfpathlineto{\pgfqpoint{3.594203in}{3.783273in}}%
\pgfpathlineto{\pgfqpoint{3.595205in}{3.732818in}}%
\pgfpathlineto{\pgfqpoint{3.596208in}{3.121909in}}%
\pgfpathlineto{\pgfqpoint{3.597210in}{3.679636in}}%
\pgfpathlineto{\pgfqpoint{3.598212in}{3.217364in}}%
\pgfpathlineto{\pgfqpoint{3.599214in}{3.736909in}}%
\pgfpathlineto{\pgfqpoint{3.601219in}{3.064636in}}%
\pgfpathlineto{\pgfqpoint{3.603223in}{2.933727in}}%
\pgfpathlineto{\pgfqpoint{3.604226in}{3.014182in}}%
\pgfpathlineto{\pgfqpoint{3.606230in}{2.995091in}}%
\pgfpathlineto{\pgfqpoint{3.608235in}{2.832818in}}%
\pgfpathlineto{\pgfqpoint{3.611241in}{2.951455in}}%
\pgfpathlineto{\pgfqpoint{3.612244in}{2.894182in}}%
\pgfpathlineto{\pgfqpoint{3.613246in}{2.771455in}}%
\pgfpathlineto{\pgfqpoint{3.615250in}{2.832818in}}%
\pgfpathlineto{\pgfqpoint{3.617255in}{2.984182in}}%
\pgfpathlineto{\pgfqpoint{3.618257in}{2.768727in}}%
\pgfpathlineto{\pgfqpoint{3.619259in}{2.832818in}}%
\pgfpathlineto{\pgfqpoint{3.620261in}{2.783727in}}%
\pgfpathlineto{\pgfqpoint{3.622266in}{2.989636in}}%
\pgfpathlineto{\pgfqpoint{3.623268in}{2.899636in}}%
\pgfpathlineto{\pgfqpoint{3.624270in}{2.952818in}}%
\pgfpathlineto{\pgfqpoint{3.625273in}{2.824636in}}%
\pgfpathlineto{\pgfqpoint{3.627277in}{2.974636in}}%
\pgfpathlineto{\pgfqpoint{3.628279in}{3.310091in}}%
\pgfpathlineto{\pgfqpoint{3.629282in}{3.225545in}}%
\pgfpathlineto{\pgfqpoint{3.630284in}{3.012818in}}%
\pgfpathlineto{\pgfqpoint{3.632288in}{3.130091in}}%
\pgfpathlineto{\pgfqpoint{3.634293in}{3.866455in}}%
\pgfpathlineto{\pgfqpoint{3.635295in}{3.569182in}}%
\pgfpathlineto{\pgfqpoint{3.636297in}{3.614182in}}%
\pgfpathlineto{\pgfqpoint{3.637300in}{3.150545in}}%
\pgfpathlineto{\pgfqpoint{3.639304in}{3.889636in}}%
\pgfpathlineto{\pgfqpoint{3.642311in}{3.213273in}}%
\pgfpathlineto{\pgfqpoint{3.644315in}{3.810545in}}%
\pgfpathlineto{\pgfqpoint{3.645318in}{3.634636in}}%
\pgfpathlineto{\pgfqpoint{3.646320in}{3.696000in}}%
\pgfpathlineto{\pgfqpoint{3.648324in}{3.160091in}}%
\pgfpathlineto{\pgfqpoint{3.649327in}{3.488727in}}%
\pgfpathlineto{\pgfqpoint{3.650329in}{3.045545in}}%
\pgfpathlineto{\pgfqpoint{3.651331in}{3.134182in}}%
\pgfpathlineto{\pgfqpoint{3.653335in}{2.988273in}}%
\pgfpathlineto{\pgfqpoint{3.654338in}{2.937818in}}%
\pgfpathlineto{\pgfqpoint{3.655340in}{2.951455in}}%
\pgfpathlineto{\pgfqpoint{3.656342in}{3.016909in}}%
\pgfpathlineto{\pgfqpoint{3.658347in}{2.892818in}}%
\pgfpathlineto{\pgfqpoint{3.660351in}{2.811000in}}%
\pgfpathlineto{\pgfqpoint{3.661353in}{2.995091in}}%
\pgfpathlineto{\pgfqpoint{3.663358in}{2.817818in}}%
\pgfpathlineto{\pgfqpoint{3.664360in}{2.826000in}}%
\pgfpathlineto{\pgfqpoint{3.665362in}{2.816455in}}%
\pgfpathlineto{\pgfqpoint{3.666365in}{2.958273in}}%
\pgfpathlineto{\pgfqpoint{3.667367in}{2.811000in}}%
\pgfpathlineto{\pgfqpoint{3.669371in}{2.836909in}}%
\pgfpathlineto{\pgfqpoint{3.670374in}{2.834182in}}%
\pgfpathlineto{\pgfqpoint{3.671376in}{2.895545in}}%
\pgfpathlineto{\pgfqpoint{3.672378in}{2.843727in}}%
\pgfpathlineto{\pgfqpoint{3.674383in}{2.902364in}}%
\pgfpathlineto{\pgfqpoint{3.675385in}{2.868273in}}%
\pgfpathlineto{\pgfqpoint{3.676387in}{2.952818in}}%
\pgfpathlineto{\pgfqpoint{3.677389in}{2.916000in}}%
\pgfpathlineto{\pgfqpoint{3.678392in}{3.068727in}}%
\pgfpathlineto{\pgfqpoint{3.679394in}{3.061909in}}%
\pgfpathlineto{\pgfqpoint{3.680396in}{3.019636in}}%
\pgfpathlineto{\pgfqpoint{3.681398in}{3.094636in}}%
\pgfpathlineto{\pgfqpoint{3.682401in}{3.063273in}}%
\pgfpathlineto{\pgfqpoint{3.683403in}{3.706909in}}%
\pgfpathlineto{\pgfqpoint{3.684405in}{3.698727in}}%
\pgfpathlineto{\pgfqpoint{3.685407in}{3.593727in}}%
\pgfpathlineto{\pgfqpoint{3.686410in}{3.682364in}}%
\pgfpathlineto{\pgfqpoint{3.687412in}{3.344182in}}%
\pgfpathlineto{\pgfqpoint{3.688414in}{3.904636in}}%
\pgfpathlineto{\pgfqpoint{3.691421in}{3.630545in}}%
\pgfpathlineto{\pgfqpoint{3.692423in}{3.087818in}}%
\pgfpathlineto{\pgfqpoint{3.694427in}{3.832364in}}%
\pgfpathlineto{\pgfqpoint{3.695430in}{3.798273in}}%
\pgfpathlineto{\pgfqpoint{3.697434in}{3.419182in}}%
\pgfpathlineto{\pgfqpoint{3.699439in}{3.691909in}}%
\pgfpathlineto{\pgfqpoint{3.702445in}{3.030545in}}%
\pgfpathlineto{\pgfqpoint{3.703448in}{3.124636in}}%
\pgfpathlineto{\pgfqpoint{3.704450in}{3.042818in}}%
\pgfpathlineto{\pgfqpoint{3.705452in}{3.055091in}}%
\pgfpathlineto{\pgfqpoint{3.707457in}{2.884636in}}%
\pgfpathlineto{\pgfqpoint{3.708459in}{2.933727in}}%
\pgfpathlineto{\pgfqpoint{3.709461in}{2.875091in}}%
\pgfpathlineto{\pgfqpoint{3.710463in}{2.944636in}}%
\pgfpathlineto{\pgfqpoint{3.711466in}{2.920091in}}%
\pgfpathlineto{\pgfqpoint{3.712468in}{2.808273in}}%
\pgfpathlineto{\pgfqpoint{3.713470in}{2.873727in}}%
\pgfpathlineto{\pgfqpoint{3.714472in}{2.809636in}}%
\pgfpathlineto{\pgfqpoint{3.716477in}{2.922818in}}%
\pgfpathlineto{\pgfqpoint{3.717479in}{2.786455in}}%
\pgfpathlineto{\pgfqpoint{3.718481in}{2.794636in}}%
\pgfpathlineto{\pgfqpoint{3.719484in}{2.806909in}}%
\pgfpathlineto{\pgfqpoint{3.721488in}{2.944636in}}%
\pgfpathlineto{\pgfqpoint{3.723492in}{2.819182in}}%
\pgfpathlineto{\pgfqpoint{3.724495in}{2.797364in}}%
\pgfpathlineto{\pgfqpoint{3.728504in}{2.971909in}}%
\pgfpathlineto{\pgfqpoint{3.729506in}{2.838273in}}%
\pgfpathlineto{\pgfqpoint{3.731510in}{2.978727in}}%
\pgfpathlineto{\pgfqpoint{3.733515in}{3.595091in}}%
\pgfpathlineto{\pgfqpoint{3.734517in}{3.124636in}}%
\pgfpathlineto{\pgfqpoint{3.735519in}{3.229636in}}%
\pgfpathlineto{\pgfqpoint{3.736522in}{3.027818in}}%
\pgfpathlineto{\pgfqpoint{3.738526in}{3.859636in}}%
\pgfpathlineto{\pgfqpoint{3.739528in}{3.727364in}}%
\pgfpathlineto{\pgfqpoint{3.740531in}{3.760091in}}%
\pgfpathlineto{\pgfqpoint{3.741533in}{3.634636in}}%
\pgfpathlineto{\pgfqpoint{3.742535in}{3.713727in}}%
\pgfpathlineto{\pgfqpoint{3.743537in}{3.931909in}}%
\pgfpathlineto{\pgfqpoint{3.744540in}{3.784636in}}%
\pgfpathlineto{\pgfqpoint{3.745542in}{3.814636in}}%
\pgfpathlineto{\pgfqpoint{3.747546in}{3.607364in}}%
\pgfpathlineto{\pgfqpoint{3.748549in}{3.742364in}}%
\pgfpathlineto{\pgfqpoint{3.750553in}{3.619636in}}%
\pgfpathlineto{\pgfqpoint{3.751555in}{3.486000in}}%
\pgfpathlineto{\pgfqpoint{3.752558in}{3.128727in}}%
\pgfpathlineto{\pgfqpoint{3.753560in}{3.130091in}}%
\pgfpathlineto{\pgfqpoint{3.754562in}{3.034636in}}%
\pgfpathlineto{\pgfqpoint{3.755564in}{3.169636in}}%
\pgfpathlineto{\pgfqpoint{3.756567in}{2.955545in}}%
\pgfpathlineto{\pgfqpoint{3.757569in}{3.025091in}}%
\pgfpathlineto{\pgfqpoint{3.759573in}{2.909182in}}%
\pgfpathlineto{\pgfqpoint{3.760575in}{2.981455in}}%
\pgfpathlineto{\pgfqpoint{3.761578in}{2.903727in}}%
\pgfpathlineto{\pgfqpoint{3.762580in}{2.926909in}}%
\pgfpathlineto{\pgfqpoint{3.764584in}{2.816455in}}%
\pgfpathlineto{\pgfqpoint{3.765587in}{2.914636in}}%
\pgfpathlineto{\pgfqpoint{3.767591in}{2.798727in}}%
\pgfpathlineto{\pgfqpoint{3.768593in}{2.800091in}}%
\pgfpathlineto{\pgfqpoint{3.769596in}{2.749636in}}%
\pgfpathlineto{\pgfqpoint{3.770598in}{2.909182in}}%
\pgfpathlineto{\pgfqpoint{3.772602in}{2.819182in}}%
\pgfpathlineto{\pgfqpoint{3.773605in}{2.816455in}}%
\pgfpathlineto{\pgfqpoint{3.774607in}{2.727818in}}%
\pgfpathlineto{\pgfqpoint{3.775609in}{2.910545in}}%
\pgfpathlineto{\pgfqpoint{3.776611in}{2.905091in}}%
\pgfpathlineto{\pgfqpoint{3.777614in}{2.871000in}}%
\pgfpathlineto{\pgfqpoint{3.778616in}{2.881909in}}%
\pgfpathlineto{\pgfqpoint{3.779618in}{2.875091in}}%
\pgfpathlineto{\pgfqpoint{3.782625in}{3.247364in}}%
\pgfpathlineto{\pgfqpoint{3.783627in}{3.091909in}}%
\pgfpathlineto{\pgfqpoint{3.785632in}{3.243273in}}%
\pgfpathlineto{\pgfqpoint{3.786634in}{3.101455in}}%
\pgfpathlineto{\pgfqpoint{3.788638in}{3.904636in}}%
\pgfpathlineto{\pgfqpoint{3.790643in}{3.772364in}}%
\pgfpathlineto{\pgfqpoint{3.791645in}{3.266455in}}%
\pgfpathlineto{\pgfqpoint{3.793649in}{3.922364in}}%
\pgfpathlineto{\pgfqpoint{3.794652in}{3.761455in}}%
\pgfpathlineto{\pgfqpoint{3.795654in}{3.768273in}}%
\pgfpathlineto{\pgfqpoint{3.796656in}{3.175091in}}%
\pgfpathlineto{\pgfqpoint{3.797658in}{3.705545in}}%
\pgfpathlineto{\pgfqpoint{3.798661in}{3.682364in}}%
\pgfpathlineto{\pgfqpoint{3.799663in}{3.685091in}}%
\pgfpathlineto{\pgfqpoint{3.800665in}{3.701455in}}%
\pgfpathlineto{\pgfqpoint{3.801667in}{3.247364in}}%
\pgfpathlineto{\pgfqpoint{3.802670in}{3.289636in}}%
\pgfpathlineto{\pgfqpoint{3.803672in}{3.113727in}}%
\pgfpathlineto{\pgfqpoint{3.805676in}{3.232364in}}%
\pgfpathlineto{\pgfqpoint{3.807681in}{3.027818in}}%
\pgfpathlineto{\pgfqpoint{3.808683in}{2.926909in}}%
\pgfpathlineto{\pgfqpoint{3.810688in}{2.985545in}}%
\pgfpathlineto{\pgfqpoint{3.812692in}{2.948727in}}%
\pgfpathlineto{\pgfqpoint{3.814697in}{2.856000in}}%
\pgfpathlineto{\pgfqpoint{3.815699in}{2.879182in}}%
\pgfpathlineto{\pgfqpoint{3.816701in}{2.801455in}}%
\pgfpathlineto{\pgfqpoint{3.817703in}{2.933727in}}%
\pgfpathlineto{\pgfqpoint{3.818706in}{2.785091in}}%
\pgfpathlineto{\pgfqpoint{3.819708in}{2.806909in}}%
\pgfpathlineto{\pgfqpoint{3.820710in}{2.879182in}}%
\pgfpathlineto{\pgfqpoint{3.821712in}{2.782364in}}%
\pgfpathlineto{\pgfqpoint{3.822715in}{2.862818in}}%
\pgfpathlineto{\pgfqpoint{3.823717in}{2.736000in}}%
\pgfpathlineto{\pgfqpoint{3.825721in}{2.869636in}}%
\pgfpathlineto{\pgfqpoint{3.826724in}{2.856000in}}%
\pgfpathlineto{\pgfqpoint{3.827726in}{2.879182in}}%
\pgfpathlineto{\pgfqpoint{3.828728in}{2.869636in}}%
\pgfpathlineto{\pgfqpoint{3.829730in}{2.925545in}}%
\pgfpathlineto{\pgfqpoint{3.830732in}{2.917364in}}%
\pgfpathlineto{\pgfqpoint{3.831735in}{2.955545in}}%
\pgfpathlineto{\pgfqpoint{3.832737in}{3.083727in}}%
\pgfpathlineto{\pgfqpoint{3.833739in}{3.033273in}}%
\pgfpathlineto{\pgfqpoint{3.834741in}{3.340091in}}%
\pgfpathlineto{\pgfqpoint{3.835744in}{3.150545in}}%
\pgfpathlineto{\pgfqpoint{3.836746in}{3.259636in}}%
\pgfpathlineto{\pgfqpoint{3.837748in}{3.713727in}}%
\pgfpathlineto{\pgfqpoint{3.838750in}{3.636000in}}%
\pgfpathlineto{\pgfqpoint{3.839753in}{3.754636in}}%
\pgfpathlineto{\pgfqpoint{3.840755in}{3.671455in}}%
\pgfpathlineto{\pgfqpoint{3.841757in}{3.409636in}}%
\pgfpathlineto{\pgfqpoint{3.842759in}{3.855545in}}%
\pgfpathlineto{\pgfqpoint{3.843762in}{3.787364in}}%
\pgfpathlineto{\pgfqpoint{3.844764in}{3.880091in}}%
\pgfpathlineto{\pgfqpoint{3.846768in}{3.360545in}}%
\pgfpathlineto{\pgfqpoint{3.847771in}{3.784636in}}%
\pgfpathlineto{\pgfqpoint{3.848773in}{3.728727in}}%
\pgfpathlineto{\pgfqpoint{3.849775in}{3.783273in}}%
\pgfpathlineto{\pgfqpoint{3.851780in}{3.338727in}}%
\pgfpathlineto{\pgfqpoint{3.852782in}{3.540545in}}%
\pgfpathlineto{\pgfqpoint{3.853784in}{3.491455in}}%
\pgfpathlineto{\pgfqpoint{3.858795in}{2.946000in}}%
\pgfpathlineto{\pgfqpoint{3.859798in}{3.036000in}}%
\pgfpathlineto{\pgfqpoint{3.860800in}{2.956909in}}%
\pgfpathlineto{\pgfqpoint{3.861802in}{2.971909in}}%
\pgfpathlineto{\pgfqpoint{3.863807in}{2.843727in}}%
\pgfpathlineto{\pgfqpoint{3.864809in}{2.903727in}}%
\pgfpathlineto{\pgfqpoint{3.866813in}{2.835545in}}%
\pgfpathlineto{\pgfqpoint{3.867815in}{2.812364in}}%
\pgfpathlineto{\pgfqpoint{3.868818in}{2.731909in}}%
\pgfpathlineto{\pgfqpoint{3.869820in}{2.846455in}}%
\pgfpathlineto{\pgfqpoint{3.870822in}{2.806909in}}%
\pgfpathlineto{\pgfqpoint{3.871824in}{2.845091in}}%
\pgfpathlineto{\pgfqpoint{3.872827in}{2.828727in}}%
\pgfpathlineto{\pgfqpoint{3.873829in}{2.703273in}}%
\pgfpathlineto{\pgfqpoint{3.874831in}{2.787818in}}%
\pgfpathlineto{\pgfqpoint{3.875833in}{2.763273in}}%
\pgfpathlineto{\pgfqpoint{3.876836in}{2.868273in}}%
\pgfpathlineto{\pgfqpoint{3.877838in}{2.841000in}}%
\pgfpathlineto{\pgfqpoint{3.878840in}{2.763273in}}%
\pgfpathlineto{\pgfqpoint{3.882849in}{2.965091in}}%
\pgfpathlineto{\pgfqpoint{3.883851in}{2.886000in}}%
\pgfpathlineto{\pgfqpoint{3.884854in}{3.019636in}}%
\pgfpathlineto{\pgfqpoint{3.885856in}{2.922818in}}%
\pgfpathlineto{\pgfqpoint{3.889865in}{3.652364in}}%
\pgfpathlineto{\pgfqpoint{3.890867in}{3.022364in}}%
\pgfpathlineto{\pgfqpoint{3.892872in}{3.754636in}}%
\pgfpathlineto{\pgfqpoint{3.893874in}{3.837818in}}%
\pgfpathlineto{\pgfqpoint{3.894876in}{3.825545in}}%
\pgfpathlineto{\pgfqpoint{3.895878in}{3.244636in}}%
\pgfpathlineto{\pgfqpoint{3.897883in}{3.754636in}}%
\pgfpathlineto{\pgfqpoint{3.898885in}{3.771000in}}%
\pgfpathlineto{\pgfqpoint{3.899887in}{3.856909in}}%
\pgfpathlineto{\pgfqpoint{3.901892in}{3.416455in}}%
\pgfpathlineto{\pgfqpoint{3.903896in}{3.694636in}}%
\pgfpathlineto{\pgfqpoint{3.904898in}{3.720545in}}%
\pgfpathlineto{\pgfqpoint{3.906903in}{3.117818in}}%
\pgfpathlineto{\pgfqpoint{3.907905in}{3.126000in}}%
\pgfpathlineto{\pgfqpoint{3.908907in}{3.111000in}}%
\pgfpathlineto{\pgfqpoint{3.909910in}{3.307364in}}%
\pgfpathlineto{\pgfqpoint{3.910912in}{3.003273in}}%
\pgfpathlineto{\pgfqpoint{3.911914in}{3.057818in}}%
\pgfpathlineto{\pgfqpoint{3.912916in}{2.914636in}}%
\pgfpathlineto{\pgfqpoint{3.914921in}{2.948727in}}%
\pgfpathlineto{\pgfqpoint{3.915923in}{2.888727in}}%
\pgfpathlineto{\pgfqpoint{3.916925in}{2.906455in}}%
\pgfpathlineto{\pgfqpoint{3.918930in}{2.796000in}}%
\pgfpathlineto{\pgfqpoint{3.919932in}{2.865545in}}%
\pgfpathlineto{\pgfqpoint{3.920934in}{2.838273in}}%
\pgfpathlineto{\pgfqpoint{3.921937in}{2.864182in}}%
\pgfpathlineto{\pgfqpoint{3.923941in}{2.761909in}}%
\pgfpathlineto{\pgfqpoint{3.924943in}{2.834182in}}%
\pgfpathlineto{\pgfqpoint{3.925946in}{2.763273in}}%
\pgfpathlineto{\pgfqpoint{3.926948in}{2.858727in}}%
\pgfpathlineto{\pgfqpoint{3.928952in}{2.768727in}}%
\pgfpathlineto{\pgfqpoint{3.929955in}{2.861455in}}%
\pgfpathlineto{\pgfqpoint{3.930957in}{2.801455in}}%
\pgfpathlineto{\pgfqpoint{3.931959in}{2.920091in}}%
\pgfpathlineto{\pgfqpoint{3.932961in}{2.898273in}}%
\pgfpathlineto{\pgfqpoint{3.934966in}{2.932364in}}%
\pgfpathlineto{\pgfqpoint{3.935968in}{2.921455in}}%
\pgfpathlineto{\pgfqpoint{3.936970in}{3.091909in}}%
\pgfpathlineto{\pgfqpoint{3.937972in}{3.063273in}}%
\pgfpathlineto{\pgfqpoint{3.938975in}{3.416455in}}%
\pgfpathlineto{\pgfqpoint{3.940979in}{3.055091in}}%
\pgfpathlineto{\pgfqpoint{3.942984in}{3.660545in}}%
\pgfpathlineto{\pgfqpoint{3.944988in}{3.777818in}}%
\pgfpathlineto{\pgfqpoint{3.945990in}{3.521455in}}%
\pgfpathlineto{\pgfqpoint{3.947995in}{3.773727in}}%
\pgfpathlineto{\pgfqpoint{3.948997in}{3.843273in}}%
\pgfpathlineto{\pgfqpoint{3.949999in}{3.792818in}}%
\pgfpathlineto{\pgfqpoint{3.951002in}{3.220091in}}%
\pgfpathlineto{\pgfqpoint{3.952004in}{3.700091in}}%
\pgfpathlineto{\pgfqpoint{3.953006in}{3.670091in}}%
\pgfpathlineto{\pgfqpoint{3.954008in}{3.779182in}}%
\pgfpathlineto{\pgfqpoint{3.955011in}{3.607364in}}%
\pgfpathlineto{\pgfqpoint{3.957015in}{3.126000in}}%
\pgfpathlineto{\pgfqpoint{3.958017in}{3.135545in}}%
\pgfpathlineto{\pgfqpoint{3.959020in}{3.535091in}}%
\pgfpathlineto{\pgfqpoint{3.961024in}{3.031909in}}%
\pgfpathlineto{\pgfqpoint{3.962026in}{3.008727in}}%
\pgfpathlineto{\pgfqpoint{3.963029in}{2.901000in}}%
\pgfpathlineto{\pgfqpoint{3.964031in}{2.982818in}}%
\pgfpathlineto{\pgfqpoint{3.967038in}{2.865545in}}%
\pgfpathlineto{\pgfqpoint{3.968040in}{2.763273in}}%
\pgfpathlineto{\pgfqpoint{3.970044in}{2.849182in}}%
\pgfpathlineto{\pgfqpoint{3.971046in}{2.783727in}}%
\pgfpathlineto{\pgfqpoint{3.972049in}{2.816455in}}%
\pgfpathlineto{\pgfqpoint{3.973051in}{2.682818in}}%
\pgfpathlineto{\pgfqpoint{3.975055in}{2.791909in}}%
\pgfpathlineto{\pgfqpoint{3.976058in}{2.811000in}}%
\pgfpathlineto{\pgfqpoint{3.977060in}{2.875091in}}%
\pgfpathlineto{\pgfqpoint{3.978062in}{2.745545in}}%
\pgfpathlineto{\pgfqpoint{3.980067in}{2.816455in}}%
\pgfpathlineto{\pgfqpoint{3.982071in}{2.910545in}}%
\pgfpathlineto{\pgfqpoint{3.983073in}{2.918727in}}%
\pgfpathlineto{\pgfqpoint{3.984076in}{2.943273in}}%
\pgfpathlineto{\pgfqpoint{3.985078in}{2.894182in}}%
\pgfpathlineto{\pgfqpoint{3.986080in}{2.935091in}}%
\pgfpathlineto{\pgfqpoint{3.988085in}{3.106909in}}%
\pgfpathlineto{\pgfqpoint{3.989087in}{3.581455in}}%
\pgfpathlineto{\pgfqpoint{3.990089in}{3.076909in}}%
\pgfpathlineto{\pgfqpoint{3.991091in}{3.097364in}}%
\pgfpathlineto{\pgfqpoint{3.994098in}{3.753273in}}%
\pgfpathlineto{\pgfqpoint{3.995100in}{3.547364in}}%
\pgfpathlineto{\pgfqpoint{3.997105in}{3.741000in}}%
\pgfpathlineto{\pgfqpoint{3.998107in}{3.723273in}}%
\pgfpathlineto{\pgfqpoint{3.999109in}{3.825545in}}%
\pgfpathlineto{\pgfqpoint{4.001114in}{3.558273in}}%
\pgfpathlineto{\pgfqpoint{4.004121in}{3.736909in}}%
\pgfpathlineto{\pgfqpoint{4.007127in}{3.161455in}}%
\pgfpathlineto{\pgfqpoint{4.009132in}{3.405545in}}%
\pgfpathlineto{\pgfqpoint{4.011136in}{3.060545in}}%
\pgfpathlineto{\pgfqpoint{4.013141in}{2.944636in}}%
\pgfpathlineto{\pgfqpoint{4.014143in}{3.025091in}}%
\pgfpathlineto{\pgfqpoint{4.015145in}{2.928273in}}%
\pgfpathlineto{\pgfqpoint{4.016147in}{2.969182in}}%
\pgfpathlineto{\pgfqpoint{4.018152in}{2.772818in}}%
\pgfpathlineto{\pgfqpoint{4.019154in}{2.875091in}}%
\pgfpathlineto{\pgfqpoint{4.020156in}{2.841000in}}%
\pgfpathlineto{\pgfqpoint{4.021159in}{2.922818in}}%
\pgfpathlineto{\pgfqpoint{4.023163in}{2.725091in}}%
\pgfpathlineto{\pgfqpoint{4.024165in}{2.789182in}}%
\pgfpathlineto{\pgfqpoint{4.025168in}{2.771455in}}%
\pgfpathlineto{\pgfqpoint{4.026170in}{2.888727in}}%
\pgfpathlineto{\pgfqpoint{4.028174in}{2.719636in}}%
\pgfpathlineto{\pgfqpoint{4.030179in}{2.767364in}}%
\pgfpathlineto{\pgfqpoint{4.031181in}{2.881909in}}%
\pgfpathlineto{\pgfqpoint{4.032183in}{2.877818in}}%
\pgfpathlineto{\pgfqpoint{4.033186in}{2.835545in}}%
\pgfpathlineto{\pgfqpoint{4.034188in}{2.838273in}}%
\pgfpathlineto{\pgfqpoint{4.035190in}{2.860091in}}%
\pgfpathlineto{\pgfqpoint{4.039199in}{3.127364in}}%
\pgfpathlineto{\pgfqpoint{4.040201in}{2.969182in}}%
\pgfpathlineto{\pgfqpoint{4.041203in}{3.250091in}}%
\pgfpathlineto{\pgfqpoint{4.042206in}{3.083727in}}%
\pgfpathlineto{\pgfqpoint{4.044210in}{3.734182in}}%
\pgfpathlineto{\pgfqpoint{4.045212in}{3.540545in}}%
\pgfpathlineto{\pgfqpoint{4.046215in}{3.661909in}}%
\pgfpathlineto{\pgfqpoint{4.047217in}{3.592364in}}%
\pgfpathlineto{\pgfqpoint{4.049221in}{3.841909in}}%
\pgfpathlineto{\pgfqpoint{4.052228in}{3.619636in}}%
\pgfpathlineto{\pgfqpoint{4.054233in}{3.817364in}}%
\pgfpathlineto{\pgfqpoint{4.057239in}{3.316909in}}%
\pgfpathlineto{\pgfqpoint{4.059244in}{3.649636in}}%
\pgfpathlineto{\pgfqpoint{4.062251in}{2.974636in}}%
\pgfpathlineto{\pgfqpoint{4.064255in}{3.087818in}}%
\pgfpathlineto{\pgfqpoint{4.066260in}{3.022364in}}%
\pgfpathlineto{\pgfqpoint{4.067262in}{2.828727in}}%
\pgfpathlineto{\pgfqpoint{4.069266in}{2.935091in}}%
\pgfpathlineto{\pgfqpoint{4.070269in}{2.935091in}}%
\pgfpathlineto{\pgfqpoint{4.072273in}{2.800091in}}%
\pgfpathlineto{\pgfqpoint{4.074278in}{2.890091in}}%
\pgfpathlineto{\pgfqpoint{4.075280in}{2.857364in}}%
\pgfpathlineto{\pgfqpoint{4.076282in}{2.881909in}}%
\pgfpathlineto{\pgfqpoint{4.077284in}{2.781000in}}%
\pgfpathlineto{\pgfqpoint{4.078286in}{2.798727in}}%
\pgfpathlineto{\pgfqpoint{4.079289in}{2.794636in}}%
\pgfpathlineto{\pgfqpoint{4.080291in}{2.819182in}}%
\pgfpathlineto{\pgfqpoint{4.081293in}{2.946000in}}%
\pgfpathlineto{\pgfqpoint{4.082295in}{2.789182in}}%
\pgfpathlineto{\pgfqpoint{4.084300in}{2.838273in}}%
\pgfpathlineto{\pgfqpoint{4.085302in}{3.081000in}}%
\pgfpathlineto{\pgfqpoint{4.086304in}{2.861455in}}%
\pgfpathlineto{\pgfqpoint{4.087307in}{2.944636in}}%
\pgfpathlineto{\pgfqpoint{4.088309in}{2.823273in}}%
\pgfpathlineto{\pgfqpoint{4.089311in}{3.102818in}}%
\pgfpathlineto{\pgfqpoint{4.090313in}{3.750545in}}%
\pgfpathlineto{\pgfqpoint{4.092318in}{2.967818in}}%
\pgfpathlineto{\pgfqpoint{4.093320in}{3.046909in}}%
\pgfpathlineto{\pgfqpoint{4.095325in}{3.700091in}}%
\pgfpathlineto{\pgfqpoint{4.096327in}{3.664636in}}%
\pgfpathlineto{\pgfqpoint{4.097329in}{3.682364in}}%
\pgfpathlineto{\pgfqpoint{4.098331in}{3.730091in}}%
\pgfpathlineto{\pgfqpoint{4.099334in}{3.604636in}}%
\pgfpathlineto{\pgfqpoint{4.100336in}{3.289636in}}%
\pgfpathlineto{\pgfqpoint{4.101338in}{3.731455in}}%
\pgfpathlineto{\pgfqpoint{4.102340in}{3.698727in}}%
\pgfpathlineto{\pgfqpoint{4.103343in}{3.168273in}}%
\pgfpathlineto{\pgfqpoint{4.104345in}{3.175091in}}%
\pgfpathlineto{\pgfqpoint{4.105347in}{3.150545in}}%
\pgfpathlineto{\pgfqpoint{4.106349in}{3.359182in}}%
\pgfpathlineto{\pgfqpoint{4.107352in}{3.236455in}}%
\pgfpathlineto{\pgfqpoint{4.108354in}{3.281455in}}%
\pgfpathlineto{\pgfqpoint{4.109356in}{3.030545in}}%
\pgfpathlineto{\pgfqpoint{4.110358in}{3.101455in}}%
\pgfpathlineto{\pgfqpoint{4.111361in}{2.969182in}}%
\pgfpathlineto{\pgfqpoint{4.112363in}{2.997818in}}%
\pgfpathlineto{\pgfqpoint{4.114367in}{3.081000in}}%
\pgfpathlineto{\pgfqpoint{4.115369in}{3.081000in}}%
\pgfpathlineto{\pgfqpoint{4.117374in}{2.805545in}}%
\pgfpathlineto{\pgfqpoint{4.120381in}{3.014182in}}%
\pgfpathlineto{\pgfqpoint{4.121383in}{2.928273in}}%
\pgfpathlineto{\pgfqpoint{4.122385in}{2.723727in}}%
\pgfpathlineto{\pgfqpoint{4.124390in}{2.800091in}}%
\pgfpathlineto{\pgfqpoint{4.126394in}{2.971909in}}%
\pgfpathlineto{\pgfqpoint{4.128399in}{2.729182in}}%
\pgfpathlineto{\pgfqpoint{4.131405in}{2.955545in}}%
\pgfpathlineto{\pgfqpoint{4.134412in}{2.768727in}}%
\pgfpathlineto{\pgfqpoint{4.135414in}{3.036000in}}%
\pgfpathlineto{\pgfqpoint{4.136417in}{2.973273in}}%
\pgfpathlineto{\pgfqpoint{4.137419in}{3.031909in}}%
\pgfpathlineto{\pgfqpoint{4.139423in}{2.913273in}}%
\pgfpathlineto{\pgfqpoint{4.140426in}{3.100091in}}%
\pgfpathlineto{\pgfqpoint{4.141428in}{3.034636in}}%
\pgfpathlineto{\pgfqpoint{4.142430in}{3.162818in}}%
\pgfpathlineto{\pgfqpoint{4.144435in}{3.618273in}}%
\pgfpathlineto{\pgfqpoint{4.145437in}{3.663273in}}%
\pgfpathlineto{\pgfqpoint{4.146439in}{3.176455in}}%
\pgfpathlineto{\pgfqpoint{4.148443in}{3.753273in}}%
\pgfpathlineto{\pgfqpoint{4.149446in}{3.743727in}}%
\pgfpathlineto{\pgfqpoint{4.150448in}{3.757364in}}%
\pgfpathlineto{\pgfqpoint{4.151450in}{3.486000in}}%
\pgfpathlineto{\pgfqpoint{4.153455in}{3.799636in}}%
\pgfpathlineto{\pgfqpoint{4.154457in}{3.712364in}}%
\pgfpathlineto{\pgfqpoint{4.155459in}{3.764182in}}%
\pgfpathlineto{\pgfqpoint{4.157464in}{3.368727in}}%
\pgfpathlineto{\pgfqpoint{4.158466in}{3.438273in}}%
\pgfpathlineto{\pgfqpoint{4.159468in}{3.327818in}}%
\pgfpathlineto{\pgfqpoint{4.160470in}{3.599182in}}%
\pgfpathlineto{\pgfqpoint{4.162475in}{3.064636in}}%
\pgfpathlineto{\pgfqpoint{4.163477in}{2.984182in}}%
\pgfpathlineto{\pgfqpoint{4.165482in}{3.106909in}}%
\pgfpathlineto{\pgfqpoint{4.169491in}{2.869636in}}%
\pgfpathlineto{\pgfqpoint{4.170493in}{2.989636in}}%
\pgfpathlineto{\pgfqpoint{4.171495in}{2.909182in}}%
\pgfpathlineto{\pgfqpoint{4.172497in}{2.940545in}}%
\pgfpathlineto{\pgfqpoint{4.174502in}{2.805545in}}%
\pgfpathlineto{\pgfqpoint{4.175504in}{2.898273in}}%
\pgfpathlineto{\pgfqpoint{4.176506in}{2.821909in}}%
\pgfpathlineto{\pgfqpoint{4.177509in}{2.827364in}}%
\pgfpathlineto{\pgfqpoint{4.179513in}{2.764636in}}%
\pgfpathlineto{\pgfqpoint{4.180515in}{2.860091in}}%
\pgfpathlineto{\pgfqpoint{4.181518in}{2.816455in}}%
\pgfpathlineto{\pgfqpoint{4.182520in}{2.841000in}}%
\pgfpathlineto{\pgfqpoint{4.184524in}{2.772818in}}%
\pgfpathlineto{\pgfqpoint{4.186529in}{2.881909in}}%
\pgfpathlineto{\pgfqpoint{4.187531in}{2.866909in}}%
\pgfpathlineto{\pgfqpoint{4.188533in}{2.816455in}}%
\pgfpathlineto{\pgfqpoint{4.189535in}{2.835545in}}%
\pgfpathlineto{\pgfqpoint{4.191540in}{2.965091in}}%
\pgfpathlineto{\pgfqpoint{4.192542in}{3.003273in}}%
\pgfpathlineto{\pgfqpoint{4.194547in}{2.958273in}}%
\pgfpathlineto{\pgfqpoint{4.195549in}{3.090545in}}%
\pgfpathlineto{\pgfqpoint{4.196551in}{3.014182in}}%
\pgfpathlineto{\pgfqpoint{4.197553in}{3.547364in}}%
\pgfpathlineto{\pgfqpoint{4.198556in}{3.378273in}}%
\pgfpathlineto{\pgfqpoint{4.199558in}{3.507818in}}%
\pgfpathlineto{\pgfqpoint{4.200560in}{3.473727in}}%
\pgfpathlineto{\pgfqpoint{4.201562in}{3.124636in}}%
\pgfpathlineto{\pgfqpoint{4.203567in}{3.747818in}}%
\pgfpathlineto{\pgfqpoint{4.205571in}{3.641455in}}%
\pgfpathlineto{\pgfqpoint{4.206574in}{3.203727in}}%
\pgfpathlineto{\pgfqpoint{4.208578in}{3.769636in}}%
\pgfpathlineto{\pgfqpoint{4.210583in}{3.705545in}}%
\pgfpathlineto{\pgfqpoint{4.211585in}{3.454636in}}%
\pgfpathlineto{\pgfqpoint{4.212587in}{3.629182in}}%
\pgfpathlineto{\pgfqpoint{4.213589in}{3.571909in}}%
\pgfpathlineto{\pgfqpoint{4.214592in}{3.608727in}}%
\pgfpathlineto{\pgfqpoint{4.215594in}{3.569182in}}%
\pgfpathlineto{\pgfqpoint{4.217598in}{3.117818in}}%
\pgfpathlineto{\pgfqpoint{4.218600in}{3.086455in}}%
\pgfpathlineto{\pgfqpoint{4.219603in}{3.120545in}}%
\pgfpathlineto{\pgfqpoint{4.220605in}{3.207818in}}%
\pgfpathlineto{\pgfqpoint{4.222609in}{3.001909in}}%
\pgfpathlineto{\pgfqpoint{4.223612in}{2.914636in}}%
\pgfpathlineto{\pgfqpoint{4.224614in}{2.986909in}}%
\pgfpathlineto{\pgfqpoint{4.225616in}{2.974636in}}%
\pgfpathlineto{\pgfqpoint{4.229625in}{2.806909in}}%
\pgfpathlineto{\pgfqpoint{4.230627in}{2.887364in}}%
\pgfpathlineto{\pgfqpoint{4.231630in}{2.881909in}}%
\pgfpathlineto{\pgfqpoint{4.232632in}{2.909182in}}%
\pgfpathlineto{\pgfqpoint{4.234636in}{2.775545in}}%
\pgfpathlineto{\pgfqpoint{4.235639in}{2.836909in}}%
\pgfpathlineto{\pgfqpoint{4.236641in}{2.782364in}}%
\pgfpathlineto{\pgfqpoint{4.237643in}{2.826000in}}%
\pgfpathlineto{\pgfqpoint{4.239648in}{2.782364in}}%
\pgfpathlineto{\pgfqpoint{4.240650in}{2.887364in}}%
\pgfpathlineto{\pgfqpoint{4.241652in}{2.872364in}}%
\pgfpathlineto{\pgfqpoint{4.242654in}{2.877818in}}%
\pgfpathlineto{\pgfqpoint{4.243657in}{2.830091in}}%
\pgfpathlineto{\pgfqpoint{4.245661in}{2.895545in}}%
\pgfpathlineto{\pgfqpoint{4.246663in}{2.933727in}}%
\pgfpathlineto{\pgfqpoint{4.247666in}{3.025091in}}%
\pgfpathlineto{\pgfqpoint{4.248668in}{2.969182in}}%
\pgfpathlineto{\pgfqpoint{4.251675in}{3.057818in}}%
\pgfpathlineto{\pgfqpoint{4.252677in}{3.499636in}}%
\pgfpathlineto{\pgfqpoint{4.253679in}{3.446455in}}%
\pgfpathlineto{\pgfqpoint{4.255683in}{3.222818in}}%
\pgfpathlineto{\pgfqpoint{4.256686in}{3.081000in}}%
\pgfpathlineto{\pgfqpoint{4.258690in}{3.690545in}}%
\pgfpathlineto{\pgfqpoint{4.259692in}{3.664636in}}%
\pgfpathlineto{\pgfqpoint{4.260695in}{3.701455in}}%
\pgfpathlineto{\pgfqpoint{4.261697in}{3.368727in}}%
\pgfpathlineto{\pgfqpoint{4.262699in}{3.735545in}}%
\pgfpathlineto{\pgfqpoint{4.263701in}{3.702818in}}%
\pgfpathlineto{\pgfqpoint{4.265706in}{3.614182in}}%
\pgfpathlineto{\pgfqpoint{4.266708in}{3.396000in}}%
\pgfpathlineto{\pgfqpoint{4.267710in}{3.715091in}}%
\pgfpathlineto{\pgfqpoint{4.268713in}{3.712364in}}%
\pgfpathlineto{\pgfqpoint{4.269715in}{3.675545in}}%
\pgfpathlineto{\pgfqpoint{4.270717in}{3.490091in}}%
\pgfpathlineto{\pgfqpoint{4.271719in}{3.079636in}}%
\pgfpathlineto{\pgfqpoint{4.272722in}{3.169636in}}%
\pgfpathlineto{\pgfqpoint{4.273724in}{3.134182in}}%
\pgfpathlineto{\pgfqpoint{4.274726in}{3.301909in}}%
\pgfpathlineto{\pgfqpoint{4.276731in}{3.019636in}}%
\pgfpathlineto{\pgfqpoint{4.277733in}{3.011455in}}%
\pgfpathlineto{\pgfqpoint{4.278735in}{2.970545in}}%
\pgfpathlineto{\pgfqpoint{4.280740in}{3.006000in}}%
\pgfpathlineto{\pgfqpoint{4.281742in}{2.877818in}}%
\pgfpathlineto{\pgfqpoint{4.282744in}{2.905091in}}%
\pgfpathlineto{\pgfqpoint{4.283746in}{2.819182in}}%
\pgfpathlineto{\pgfqpoint{4.285751in}{2.924182in}}%
\pgfpathlineto{\pgfqpoint{4.286753in}{2.831455in}}%
\pgfpathlineto{\pgfqpoint{4.287755in}{2.877818in}}%
\pgfpathlineto{\pgfqpoint{4.288758in}{2.771455in}}%
\pgfpathlineto{\pgfqpoint{4.290762in}{2.851909in}}%
\pgfpathlineto{\pgfqpoint{4.291764in}{2.862818in}}%
\pgfpathlineto{\pgfqpoint{4.292766in}{2.860091in}}%
\pgfpathlineto{\pgfqpoint{4.293769in}{2.760545in}}%
\pgfpathlineto{\pgfqpoint{4.294771in}{2.782364in}}%
\pgfpathlineto{\pgfqpoint{4.296775in}{2.872364in}}%
\pgfpathlineto{\pgfqpoint{4.297778in}{2.877818in}}%
\pgfpathlineto{\pgfqpoint{4.298780in}{2.805545in}}%
\pgfpathlineto{\pgfqpoint{4.299782in}{2.817818in}}%
\pgfpathlineto{\pgfqpoint{4.302789in}{2.999182in}}%
\pgfpathlineto{\pgfqpoint{4.303791in}{2.920091in}}%
\pgfpathlineto{\pgfqpoint{4.305796in}{2.950091in}}%
\pgfpathlineto{\pgfqpoint{4.306798in}{3.001909in}}%
\pgfpathlineto{\pgfqpoint{4.307800in}{3.498273in}}%
\pgfpathlineto{\pgfqpoint{4.308802in}{3.181909in}}%
\pgfpathlineto{\pgfqpoint{4.309805in}{3.301909in}}%
\pgfpathlineto{\pgfqpoint{4.310807in}{3.075545in}}%
\pgfpathlineto{\pgfqpoint{4.311809in}{3.120545in}}%
\pgfpathlineto{\pgfqpoint{4.312811in}{3.668727in}}%
\pgfpathlineto{\pgfqpoint{4.313814in}{3.615545in}}%
\pgfpathlineto{\pgfqpoint{4.314816in}{3.645545in}}%
\pgfpathlineto{\pgfqpoint{4.316820in}{3.346909in}}%
\pgfpathlineto{\pgfqpoint{4.318825in}{3.711000in}}%
\pgfpathlineto{\pgfqpoint{4.319827in}{3.701455in}}%
\pgfpathlineto{\pgfqpoint{4.321832in}{3.432818in}}%
\pgfpathlineto{\pgfqpoint{4.322834in}{3.697364in}}%
\pgfpathlineto{\pgfqpoint{4.323836in}{3.685091in}}%
\pgfpathlineto{\pgfqpoint{4.324838in}{3.681000in}}%
\pgfpathlineto{\pgfqpoint{4.325840in}{3.521455in}}%
\pgfpathlineto{\pgfqpoint{4.326843in}{3.150545in}}%
\pgfpathlineto{\pgfqpoint{4.327845in}{3.301909in}}%
\pgfpathlineto{\pgfqpoint{4.328847in}{3.246000in}}%
\pgfpathlineto{\pgfqpoint{4.329849in}{3.366000in}}%
\pgfpathlineto{\pgfqpoint{4.331854in}{3.064636in}}%
\pgfpathlineto{\pgfqpoint{4.332856in}{3.056455in}}%
\pgfpathlineto{\pgfqpoint{4.333858in}{2.989636in}}%
\pgfpathlineto{\pgfqpoint{4.334861in}{3.048273in}}%
\pgfpathlineto{\pgfqpoint{4.338870in}{2.866909in}}%
\pgfpathlineto{\pgfqpoint{4.339872in}{2.917364in}}%
\pgfpathlineto{\pgfqpoint{4.340874in}{2.899636in}}%
\pgfpathlineto{\pgfqpoint{4.341876in}{2.907818in}}%
\pgfpathlineto{\pgfqpoint{4.343881in}{2.815091in}}%
\pgfpathlineto{\pgfqpoint{4.344883in}{2.826000in}}%
\pgfpathlineto{\pgfqpoint{4.345885in}{2.864182in}}%
\pgfpathlineto{\pgfqpoint{4.347890in}{2.838273in}}%
\pgfpathlineto{\pgfqpoint{4.348892in}{2.787818in}}%
\pgfpathlineto{\pgfqpoint{4.350897in}{2.890091in}}%
\pgfpathlineto{\pgfqpoint{4.351899in}{2.838273in}}%
\pgfpathlineto{\pgfqpoint{4.352901in}{2.854636in}}%
\pgfpathlineto{\pgfqpoint{4.353903in}{2.789182in}}%
\pgfpathlineto{\pgfqpoint{4.356910in}{2.901000in}}%
\pgfpathlineto{\pgfqpoint{4.357912in}{2.903727in}}%
\pgfpathlineto{\pgfqpoint{4.358915in}{2.841000in}}%
\pgfpathlineto{\pgfqpoint{4.361921in}{3.052364in}}%
\pgfpathlineto{\pgfqpoint{4.362923in}{3.033273in}}%
\pgfpathlineto{\pgfqpoint{4.363926in}{2.980091in}}%
\pgfpathlineto{\pgfqpoint{4.365930in}{3.064636in}}%
\pgfpathlineto{\pgfqpoint{4.366932in}{3.192818in}}%
\pgfpathlineto{\pgfqpoint{4.367935in}{3.589636in}}%
\pgfpathlineto{\pgfqpoint{4.370941in}{3.076909in}}%
\pgfpathlineto{\pgfqpoint{4.373948in}{3.776455in}}%
\pgfpathlineto{\pgfqpoint{4.374950in}{3.762818in}}%
\pgfpathlineto{\pgfqpoint{4.376955in}{3.199636in}}%
\pgfpathlineto{\pgfqpoint{4.378959in}{3.694636in}}%
\pgfpathlineto{\pgfqpoint{4.379962in}{3.720545in}}%
\pgfpathlineto{\pgfqpoint{4.382968in}{3.248727in}}%
\pgfpathlineto{\pgfqpoint{4.383971in}{3.211909in}}%
\pgfpathlineto{\pgfqpoint{4.384973in}{3.312818in}}%
\pgfpathlineto{\pgfqpoint{4.386977in}{3.064636in}}%
\pgfpathlineto{\pgfqpoint{4.388982in}{2.981455in}}%
\pgfpathlineto{\pgfqpoint{4.389984in}{3.045545in}}%
\pgfpathlineto{\pgfqpoint{4.391989in}{2.969182in}}%
\pgfpathlineto{\pgfqpoint{4.392991in}{2.954182in}}%
\pgfpathlineto{\pgfqpoint{4.393993in}{2.903727in}}%
\pgfpathlineto{\pgfqpoint{4.394995in}{2.951455in}}%
\pgfpathlineto{\pgfqpoint{4.395997in}{2.884636in}}%
\pgfpathlineto{\pgfqpoint{4.397000in}{2.898273in}}%
\pgfpathlineto{\pgfqpoint{4.398002in}{2.881909in}}%
\pgfpathlineto{\pgfqpoint{4.399004in}{2.832818in}}%
\pgfpathlineto{\pgfqpoint{4.400006in}{2.913273in}}%
\pgfpathlineto{\pgfqpoint{4.401009in}{2.850545in}}%
\pgfpathlineto{\pgfqpoint{4.402011in}{2.856000in}}%
\pgfpathlineto{\pgfqpoint{4.404015in}{2.791909in}}%
\pgfpathlineto{\pgfqpoint{4.406020in}{2.883273in}}%
\pgfpathlineto{\pgfqpoint{4.407022in}{2.888727in}}%
\pgfpathlineto{\pgfqpoint{4.409027in}{2.823273in}}%
\pgfpathlineto{\pgfqpoint{4.411031in}{2.931000in}}%
\pgfpathlineto{\pgfqpoint{4.412033in}{2.955545in}}%
\pgfpathlineto{\pgfqpoint{4.414038in}{2.910545in}}%
\pgfpathlineto{\pgfqpoint{4.417045in}{3.156000in}}%
\pgfpathlineto{\pgfqpoint{4.418047in}{3.183273in}}%
\pgfpathlineto{\pgfqpoint{4.419049in}{3.090545in}}%
\pgfpathlineto{\pgfqpoint{4.420051in}{3.205091in}}%
\pgfpathlineto{\pgfqpoint{4.421054in}{3.045545in}}%
\pgfpathlineto{\pgfqpoint{4.423058in}{3.634636in}}%
\pgfpathlineto{\pgfqpoint{4.424060in}{3.646909in}}%
\pgfpathlineto{\pgfqpoint{4.425063in}{3.641455in}}%
\pgfpathlineto{\pgfqpoint{4.426065in}{3.158727in}}%
\pgfpathlineto{\pgfqpoint{4.428069in}{3.670091in}}%
\pgfpathlineto{\pgfqpoint{4.429072in}{3.690545in}}%
\pgfpathlineto{\pgfqpoint{4.430074in}{3.668727in}}%
\pgfpathlineto{\pgfqpoint{4.431076in}{3.372818in}}%
\pgfpathlineto{\pgfqpoint{4.432078in}{3.409636in}}%
\pgfpathlineto{\pgfqpoint{4.434083in}{3.648273in}}%
\pgfpathlineto{\pgfqpoint{4.435085in}{3.693273in}}%
\pgfpathlineto{\pgfqpoint{4.437089in}{3.141000in}}%
\pgfpathlineto{\pgfqpoint{4.438092in}{3.136909in}}%
\pgfpathlineto{\pgfqpoint{4.440096in}{3.192818in}}%
\pgfpathlineto{\pgfqpoint{4.441098in}{3.015545in}}%
\pgfpathlineto{\pgfqpoint{4.442101in}{3.036000in}}%
\pgfpathlineto{\pgfqpoint{4.444105in}{2.977364in}}%
\pgfpathlineto{\pgfqpoint{4.445107in}{3.008727in}}%
\pgfpathlineto{\pgfqpoint{4.446110in}{2.948727in}}%
\pgfpathlineto{\pgfqpoint{4.447112in}{2.985545in}}%
\pgfpathlineto{\pgfqpoint{4.449116in}{2.872364in}}%
\pgfpathlineto{\pgfqpoint{4.451121in}{2.909182in}}%
\pgfpathlineto{\pgfqpoint{4.452123in}{2.911909in}}%
\pgfpathlineto{\pgfqpoint{4.454128in}{2.802818in}}%
\pgfpathlineto{\pgfqpoint{4.455130in}{2.886000in}}%
\pgfpathlineto{\pgfqpoint{4.456132in}{2.868273in}}%
\pgfpathlineto{\pgfqpoint{4.457134in}{2.884636in}}%
\pgfpathlineto{\pgfqpoint{4.459139in}{2.802818in}}%
\pgfpathlineto{\pgfqpoint{4.462146in}{2.931000in}}%
\pgfpathlineto{\pgfqpoint{4.464150in}{2.858727in}}%
\pgfpathlineto{\pgfqpoint{4.466154in}{2.921455in}}%
\pgfpathlineto{\pgfqpoint{4.467157in}{3.044182in}}%
\pgfpathlineto{\pgfqpoint{4.469161in}{3.003273in}}%
\pgfpathlineto{\pgfqpoint{4.470163in}{3.011455in}}%
\pgfpathlineto{\pgfqpoint{4.471166in}{3.000545in}}%
\pgfpathlineto{\pgfqpoint{4.474172in}{3.438273in}}%
\pgfpathlineto{\pgfqpoint{4.476177in}{3.071455in}}%
\pgfpathlineto{\pgfqpoint{4.480186in}{3.691909in}}%
\pgfpathlineto{\pgfqpoint{4.481188in}{3.267818in}}%
\pgfpathlineto{\pgfqpoint{4.483193in}{3.611455in}}%
\pgfpathlineto{\pgfqpoint{4.485197in}{3.660545in}}%
\pgfpathlineto{\pgfqpoint{4.486199in}{3.297818in}}%
\pgfpathlineto{\pgfqpoint{4.487202in}{3.442364in}}%
\pgfpathlineto{\pgfqpoint{4.488204in}{3.150545in}}%
\pgfpathlineto{\pgfqpoint{4.489206in}{3.338727in}}%
\pgfpathlineto{\pgfqpoint{4.490208in}{3.277364in}}%
\pgfpathlineto{\pgfqpoint{4.492213in}{3.097364in}}%
\pgfpathlineto{\pgfqpoint{4.493215in}{3.016909in}}%
\pgfpathlineto{\pgfqpoint{4.494217in}{3.042818in}}%
\pgfpathlineto{\pgfqpoint{4.497224in}{2.989636in}}%
\pgfpathlineto{\pgfqpoint{4.498226in}{2.916000in}}%
\pgfpathlineto{\pgfqpoint{4.499229in}{2.932364in}}%
\pgfpathlineto{\pgfqpoint{4.500231in}{2.925545in}}%
\pgfpathlineto{\pgfqpoint{4.501233in}{2.928273in}}%
\pgfpathlineto{\pgfqpoint{4.502235in}{2.913273in}}%
\pgfpathlineto{\pgfqpoint{4.503237in}{2.843727in}}%
\pgfpathlineto{\pgfqpoint{4.504240in}{2.849182in}}%
\pgfpathlineto{\pgfqpoint{4.505242in}{2.892818in}}%
\pgfpathlineto{\pgfqpoint{4.506244in}{2.872364in}}%
\pgfpathlineto{\pgfqpoint{4.507246in}{2.920091in}}%
\pgfpathlineto{\pgfqpoint{4.508249in}{2.812364in}}%
\pgfpathlineto{\pgfqpoint{4.512258in}{2.901000in}}%
\pgfpathlineto{\pgfqpoint{4.513260in}{2.839636in}}%
\pgfpathlineto{\pgfqpoint{4.515264in}{2.879182in}}%
\pgfpathlineto{\pgfqpoint{4.517269in}{2.976000in}}%
\pgfpathlineto{\pgfqpoint{4.518271in}{2.978727in}}%
\pgfpathlineto{\pgfqpoint{4.519273in}{2.985545in}}%
\pgfpathlineto{\pgfqpoint{4.520276in}{2.970545in}}%
\pgfpathlineto{\pgfqpoint{4.522280in}{3.070091in}}%
\pgfpathlineto{\pgfqpoint{4.523282in}{3.053727in}}%
\pgfpathlineto{\pgfqpoint{4.524285in}{3.209182in}}%
\pgfpathlineto{\pgfqpoint{4.525287in}{3.121909in}}%
\pgfpathlineto{\pgfqpoint{4.526289in}{3.132818in}}%
\pgfpathlineto{\pgfqpoint{4.529296in}{3.615545in}}%
\pgfpathlineto{\pgfqpoint{4.531300in}{3.231000in}}%
\pgfpathlineto{\pgfqpoint{4.533305in}{3.551455in}}%
\pgfpathlineto{\pgfqpoint{4.534307in}{3.637364in}}%
\pgfpathlineto{\pgfqpoint{4.536312in}{3.255545in}}%
\pgfpathlineto{\pgfqpoint{4.537314in}{3.295091in}}%
\pgfpathlineto{\pgfqpoint{4.538316in}{3.254182in}}%
\pgfpathlineto{\pgfqpoint{4.539318in}{3.360545in}}%
\pgfpathlineto{\pgfqpoint{4.541323in}{3.094636in}}%
\pgfpathlineto{\pgfqpoint{4.543327in}{3.026455in}}%
\pgfpathlineto{\pgfqpoint{4.544329in}{3.071455in}}%
\pgfpathlineto{\pgfqpoint{4.548338in}{2.909182in}}%
\pgfpathlineto{\pgfqpoint{4.549341in}{2.940545in}}%
\pgfpathlineto{\pgfqpoint{4.550343in}{2.926909in}}%
\pgfpathlineto{\pgfqpoint{4.551345in}{2.956909in}}%
\pgfpathlineto{\pgfqpoint{4.553350in}{2.823273in}}%
\pgfpathlineto{\pgfqpoint{4.554352in}{2.862818in}}%
\pgfpathlineto{\pgfqpoint{4.555354in}{2.853273in}}%
\pgfpathlineto{\pgfqpoint{4.556356in}{2.932364in}}%
\pgfpathlineto{\pgfqpoint{4.558361in}{2.808273in}}%
\pgfpathlineto{\pgfqpoint{4.561368in}{2.914636in}}%
\pgfpathlineto{\pgfqpoint{4.563372in}{2.842364in}}%
\pgfpathlineto{\pgfqpoint{4.567381in}{2.977364in}}%
\pgfpathlineto{\pgfqpoint{4.568383in}{2.963727in}}%
\pgfpathlineto{\pgfqpoint{4.572392in}{3.078273in}}%
\pgfpathlineto{\pgfqpoint{4.574397in}{3.372818in}}%
\pgfpathlineto{\pgfqpoint{4.575399in}{3.075545in}}%
\pgfpathlineto{\pgfqpoint{4.576401in}{3.214636in}}%
\pgfpathlineto{\pgfqpoint{4.577403in}{3.151909in}}%
\pgfpathlineto{\pgfqpoint{4.579408in}{3.599182in}}%
\pgfpathlineto{\pgfqpoint{4.580410in}{3.232364in}}%
\pgfpathlineto{\pgfqpoint{4.581412in}{3.346909in}}%
\pgfpathlineto{\pgfqpoint{4.582415in}{3.232364in}}%
\pgfpathlineto{\pgfqpoint{4.584419in}{3.586909in}}%
\pgfpathlineto{\pgfqpoint{4.587426in}{3.100091in}}%
\pgfpathlineto{\pgfqpoint{4.589430in}{3.292364in}}%
\pgfpathlineto{\pgfqpoint{4.591435in}{3.078273in}}%
\pgfpathlineto{\pgfqpoint{4.592437in}{2.992364in}}%
\pgfpathlineto{\pgfqpoint{4.594442in}{3.006000in}}%
\pgfpathlineto{\pgfqpoint{4.595444in}{2.974636in}}%
\pgfpathlineto{\pgfqpoint{4.596446in}{2.991000in}}%
\pgfpathlineto{\pgfqpoint{4.597448in}{2.921455in}}%
\pgfpathlineto{\pgfqpoint{4.598451in}{2.926909in}}%
\pgfpathlineto{\pgfqpoint{4.600455in}{2.888727in}}%
\pgfpathlineto{\pgfqpoint{4.601457in}{2.941909in}}%
\pgfpathlineto{\pgfqpoint{4.602460in}{2.864182in}}%
\pgfpathlineto{\pgfqpoint{4.604464in}{2.880545in}}%
\pgfpathlineto{\pgfqpoint{4.605466in}{2.846455in}}%
\pgfpathlineto{\pgfqpoint{4.606469in}{2.901000in}}%
\pgfpathlineto{\pgfqpoint{4.607471in}{2.853273in}}%
\pgfpathlineto{\pgfqpoint{4.608473in}{2.864182in}}%
\pgfpathlineto{\pgfqpoint{4.609475in}{2.862818in}}%
\pgfpathlineto{\pgfqpoint{4.610477in}{2.872364in}}%
\pgfpathlineto{\pgfqpoint{4.611480in}{2.910545in}}%
\pgfpathlineto{\pgfqpoint{4.612482in}{2.888727in}}%
\pgfpathlineto{\pgfqpoint{4.613484in}{2.910545in}}%
\pgfpathlineto{\pgfqpoint{4.614486in}{2.891455in}}%
\pgfpathlineto{\pgfqpoint{4.615489in}{2.896909in}}%
\pgfpathlineto{\pgfqpoint{4.618495in}{3.000545in}}%
\pgfpathlineto{\pgfqpoint{4.619498in}{3.000545in}}%
\pgfpathlineto{\pgfqpoint{4.621502in}{3.044182in}}%
\pgfpathlineto{\pgfqpoint{4.622504in}{3.030545in}}%
\pgfpathlineto{\pgfqpoint{4.623507in}{3.175091in}}%
\pgfpathlineto{\pgfqpoint{4.626513in}{3.105545in}}%
\pgfpathlineto{\pgfqpoint{4.627516in}{3.142364in}}%
\pgfpathlineto{\pgfqpoint{4.628518in}{3.555545in}}%
\pgfpathlineto{\pgfqpoint{4.629520in}{3.501000in}}%
\pgfpathlineto{\pgfqpoint{4.631525in}{3.265091in}}%
\pgfpathlineto{\pgfqpoint{4.632527in}{3.263727in}}%
\pgfpathlineto{\pgfqpoint{4.633529in}{3.570545in}}%
\pgfpathlineto{\pgfqpoint{4.634531in}{3.479182in}}%
\pgfpathlineto{\pgfqpoint{4.636536in}{3.169636in}}%
\pgfpathlineto{\pgfqpoint{4.637538in}{3.126000in}}%
\pgfpathlineto{\pgfqpoint{4.638540in}{3.217364in}}%
\pgfpathlineto{\pgfqpoint{4.640545in}{3.066000in}}%
\pgfpathlineto{\pgfqpoint{4.642549in}{2.991000in}}%
\pgfpathlineto{\pgfqpoint{4.643551in}{3.060545in}}%
\pgfpathlineto{\pgfqpoint{4.645556in}{3.001909in}}%
\pgfpathlineto{\pgfqpoint{4.647560in}{2.913273in}}%
\pgfpathlineto{\pgfqpoint{4.648563in}{2.922818in}}%
\pgfpathlineto{\pgfqpoint{4.649565in}{2.903727in}}%
\pgfpathlineto{\pgfqpoint{4.650567in}{2.939182in}}%
\pgfpathlineto{\pgfqpoint{4.652572in}{2.875091in}}%
\pgfpathlineto{\pgfqpoint{4.653574in}{2.875091in}}%
\pgfpathlineto{\pgfqpoint{4.654576in}{2.843727in}}%
\pgfpathlineto{\pgfqpoint{4.656581in}{2.902364in}}%
\pgfpathlineto{\pgfqpoint{4.659587in}{2.854636in}}%
\pgfpathlineto{\pgfqpoint{4.661592in}{2.925545in}}%
\pgfpathlineto{\pgfqpoint{4.662594in}{2.899636in}}%
\pgfpathlineto{\pgfqpoint{4.663596in}{2.921455in}}%
\pgfpathlineto{\pgfqpoint{4.664599in}{2.901000in}}%
\pgfpathlineto{\pgfqpoint{4.666603in}{2.952818in}}%
\pgfpathlineto{\pgfqpoint{4.667605in}{2.954182in}}%
\pgfpathlineto{\pgfqpoint{4.668608in}{2.988273in}}%
\pgfpathlineto{\pgfqpoint{4.669610in}{2.956909in}}%
\pgfpathlineto{\pgfqpoint{4.673619in}{3.158727in}}%
\pgfpathlineto{\pgfqpoint{4.674621in}{3.078273in}}%
\pgfpathlineto{\pgfqpoint{4.675623in}{3.097364in}}%
\pgfpathlineto{\pgfqpoint{4.676626in}{3.090545in}}%
\pgfpathlineto{\pgfqpoint{4.677628in}{3.187364in}}%
\pgfpathlineto{\pgfqpoint{4.678630in}{3.503727in}}%
\pgfpathlineto{\pgfqpoint{4.680634in}{3.191455in}}%
\pgfpathlineto{\pgfqpoint{4.681637in}{3.184636in}}%
\pgfpathlineto{\pgfqpoint{4.682639in}{3.236455in}}%
\pgfpathlineto{\pgfqpoint{4.683641in}{3.533727in}}%
\pgfpathlineto{\pgfqpoint{4.686648in}{3.105545in}}%
\pgfpathlineto{\pgfqpoint{4.687650in}{3.135545in}}%
\pgfpathlineto{\pgfqpoint{4.688652in}{3.211909in}}%
\pgfpathlineto{\pgfqpoint{4.691659in}{3.038727in}}%
\pgfpathlineto{\pgfqpoint{4.692661in}{3.021000in}}%
\pgfpathlineto{\pgfqpoint{4.693664in}{3.059182in}}%
\pgfpathlineto{\pgfqpoint{4.694666in}{3.016909in}}%
\pgfpathlineto{\pgfqpoint{4.695668in}{3.026455in}}%
\pgfpathlineto{\pgfqpoint{4.697673in}{2.950091in}}%
\pgfpathlineto{\pgfqpoint{4.699677in}{2.936455in}}%
\pgfpathlineto{\pgfqpoint{4.700679in}{2.958273in}}%
\pgfpathlineto{\pgfqpoint{4.702684in}{2.894182in}}%
\pgfpathlineto{\pgfqpoint{4.703686in}{2.892818in}}%
\pgfpathlineto{\pgfqpoint{4.704688in}{2.851909in}}%
\pgfpathlineto{\pgfqpoint{4.705691in}{2.920091in}}%
\pgfpathlineto{\pgfqpoint{4.706693in}{2.917364in}}%
\pgfpathlineto{\pgfqpoint{4.709700in}{2.865545in}}%
\pgfpathlineto{\pgfqpoint{4.711704in}{2.933727in}}%
\pgfpathlineto{\pgfqpoint{4.713709in}{2.894182in}}%
\pgfpathlineto{\pgfqpoint{4.714711in}{2.894182in}}%
\pgfpathlineto{\pgfqpoint{4.717717in}{2.982818in}}%
\pgfpathlineto{\pgfqpoint{4.718720in}{2.947364in}}%
\pgfpathlineto{\pgfqpoint{4.719722in}{2.954182in}}%
\pgfpathlineto{\pgfqpoint{4.723731in}{3.078273in}}%
\pgfpathlineto{\pgfqpoint{4.724733in}{3.053727in}}%
\pgfpathlineto{\pgfqpoint{4.725735in}{3.083727in}}%
\pgfpathlineto{\pgfqpoint{4.726738in}{3.051000in}}%
\pgfpathlineto{\pgfqpoint{4.727740in}{3.138273in}}%
\pgfpathlineto{\pgfqpoint{4.728742in}{3.329182in}}%
\pgfpathlineto{\pgfqpoint{4.729744in}{3.218727in}}%
\pgfpathlineto{\pgfqpoint{4.730747in}{3.285545in}}%
\pgfpathlineto{\pgfqpoint{4.731749in}{3.096000in}}%
\pgfpathlineto{\pgfqpoint{4.732751in}{3.179182in}}%
\pgfpathlineto{\pgfqpoint{4.733753in}{3.368727in}}%
\pgfpathlineto{\pgfqpoint{4.736760in}{3.085091in}}%
\pgfpathlineto{\pgfqpoint{4.737762in}{3.116455in}}%
\pgfpathlineto{\pgfqpoint{4.738765in}{3.102818in}}%
\pgfpathlineto{\pgfqpoint{4.740769in}{3.119182in}}%
\pgfpathlineto{\pgfqpoint{4.742774in}{3.027818in}}%
\pgfpathlineto{\pgfqpoint{4.743776in}{2.992364in}}%
\pgfpathlineto{\pgfqpoint{4.745780in}{3.038727in}}%
\pgfpathlineto{\pgfqpoint{4.748787in}{2.925545in}}%
\pgfpathlineto{\pgfqpoint{4.749789in}{2.921455in}}%
\pgfpathlineto{\pgfqpoint{4.750791in}{2.950091in}}%
\pgfpathlineto{\pgfqpoint{4.751794in}{2.940545in}}%
\pgfpathlineto{\pgfqpoint{4.752796in}{2.941909in}}%
\pgfpathlineto{\pgfqpoint{4.754800in}{2.890091in}}%
\pgfpathlineto{\pgfqpoint{4.755803in}{2.931000in}}%
\pgfpathlineto{\pgfqpoint{4.756805in}{2.924182in}}%
\pgfpathlineto{\pgfqpoint{4.757807in}{2.928273in}}%
\pgfpathlineto{\pgfqpoint{4.758809in}{2.884636in}}%
\pgfpathlineto{\pgfqpoint{4.759812in}{2.887364in}}%
\pgfpathlineto{\pgfqpoint{4.760814in}{2.931000in}}%
\pgfpathlineto{\pgfqpoint{4.761816in}{2.918727in}}%
\pgfpathlineto{\pgfqpoint{4.762818in}{2.936455in}}%
\pgfpathlineto{\pgfqpoint{4.763821in}{2.909182in}}%
\pgfpathlineto{\pgfqpoint{4.767830in}{2.973273in}}%
\pgfpathlineto{\pgfqpoint{4.768832in}{2.959636in}}%
\pgfpathlineto{\pgfqpoint{4.772841in}{3.070091in}}%
\pgfpathlineto{\pgfqpoint{4.773843in}{3.026455in}}%
\pgfpathlineto{\pgfqpoint{4.775848in}{3.102818in}}%
\pgfpathlineto{\pgfqpoint{4.776850in}{3.034636in}}%
\pgfpathlineto{\pgfqpoint{4.778854in}{3.168273in}}%
\pgfpathlineto{\pgfqpoint{4.781861in}{3.091909in}}%
\pgfpathlineto{\pgfqpoint{4.783866in}{3.251455in}}%
\pgfpathlineto{\pgfqpoint{4.784868in}{3.207818in}}%
\pgfpathlineto{\pgfqpoint{4.786872in}{3.086455in}}%
\pgfpathlineto{\pgfqpoint{4.787874in}{3.147818in}}%
\pgfpathlineto{\pgfqpoint{4.788877in}{3.091909in}}%
\pgfpathlineto{\pgfqpoint{4.789879in}{3.126000in}}%
\pgfpathlineto{\pgfqpoint{4.791883in}{3.044182in}}%
\pgfpathlineto{\pgfqpoint{4.796895in}{2.981455in}}%
\pgfpathlineto{\pgfqpoint{4.797897in}{2.992364in}}%
\pgfpathlineto{\pgfqpoint{4.798899in}{2.946000in}}%
\pgfpathlineto{\pgfqpoint{4.799901in}{2.988273in}}%
\pgfpathlineto{\pgfqpoint{4.802908in}{2.937818in}}%
\pgfpathlineto{\pgfqpoint{4.803910in}{2.905091in}}%
\pgfpathlineto{\pgfqpoint{4.805915in}{2.932364in}}%
\pgfpathlineto{\pgfqpoint{4.808922in}{2.880545in}}%
\pgfpathlineto{\pgfqpoint{4.809924in}{2.891455in}}%
\pgfpathlineto{\pgfqpoint{4.811928in}{2.950091in}}%
\pgfpathlineto{\pgfqpoint{4.812931in}{2.946000in}}%
\pgfpathlineto{\pgfqpoint{4.813933in}{2.909182in}}%
\pgfpathlineto{\pgfqpoint{4.815937in}{2.946000in}}%
\pgfpathlineto{\pgfqpoint{4.816940in}{2.984182in}}%
\pgfpathlineto{\pgfqpoint{4.818944in}{2.958273in}}%
\pgfpathlineto{\pgfqpoint{4.821951in}{3.014182in}}%
\pgfpathlineto{\pgfqpoint{4.822953in}{3.052364in}}%
\pgfpathlineto{\pgfqpoint{4.823955in}{3.034636in}}%
\pgfpathlineto{\pgfqpoint{4.824957in}{3.051000in}}%
\pgfpathlineto{\pgfqpoint{4.825960in}{3.004636in}}%
\pgfpathlineto{\pgfqpoint{4.826962in}{3.038727in}}%
\pgfpathlineto{\pgfqpoint{4.827964in}{3.134182in}}%
\pgfpathlineto{\pgfqpoint{4.828966in}{3.102818in}}%
\pgfpathlineto{\pgfqpoint{4.829969in}{3.136909in}}%
\pgfpathlineto{\pgfqpoint{4.830971in}{3.037364in}}%
\pgfpathlineto{\pgfqpoint{4.831973in}{3.056455in}}%
\pgfpathlineto{\pgfqpoint{4.834980in}{3.225545in}}%
\pgfpathlineto{\pgfqpoint{4.836984in}{3.078273in}}%
\pgfpathlineto{\pgfqpoint{4.838989in}{3.116455in}}%
\pgfpathlineto{\pgfqpoint{4.839991in}{3.131455in}}%
\pgfpathlineto{\pgfqpoint{4.841996in}{3.037364in}}%
\pgfpathlineto{\pgfqpoint{4.842998in}{3.036000in}}%
\pgfpathlineto{\pgfqpoint{4.844000in}{3.030545in}}%
\pgfpathlineto{\pgfqpoint{4.845002in}{3.041455in}}%
\pgfpathlineto{\pgfqpoint{4.847007in}{3.004636in}}%
\pgfpathlineto{\pgfqpoint{4.848009in}{2.992364in}}%
\pgfpathlineto{\pgfqpoint{4.849011in}{2.962364in}}%
\pgfpathlineto{\pgfqpoint{4.850014in}{2.988273in}}%
\pgfpathlineto{\pgfqpoint{4.851016in}{2.970545in}}%
\pgfpathlineto{\pgfqpoint{4.852018in}{2.992364in}}%
\pgfpathlineto{\pgfqpoint{4.854023in}{2.913273in}}%
\pgfpathlineto{\pgfqpoint{4.857029in}{2.971909in}}%
\pgfpathlineto{\pgfqpoint{4.859034in}{2.901000in}}%
\pgfpathlineto{\pgfqpoint{4.862040in}{2.941909in}}%
\pgfpathlineto{\pgfqpoint{4.863043in}{2.935091in}}%
\pgfpathlineto{\pgfqpoint{4.864045in}{2.907818in}}%
\pgfpathlineto{\pgfqpoint{4.867052in}{2.967818in}}%
\pgfpathlineto{\pgfqpoint{4.869056in}{2.962364in}}%
\pgfpathlineto{\pgfqpoint{4.871061in}{2.982818in}}%
\pgfpathlineto{\pgfqpoint{4.872063in}{3.029182in}}%
\pgfpathlineto{\pgfqpoint{4.873065in}{3.016909in}}%
\pgfpathlineto{\pgfqpoint{4.875070in}{3.045545in}}%
\pgfpathlineto{\pgfqpoint{4.876072in}{3.003273in}}%
\pgfpathlineto{\pgfqpoint{4.878076in}{3.089182in}}%
\pgfpathlineto{\pgfqpoint{4.879079in}{3.089182in}}%
\pgfpathlineto{\pgfqpoint{4.880081in}{3.098727in}}%
\pgfpathlineto{\pgfqpoint{4.881083in}{3.041455in}}%
\pgfpathlineto{\pgfqpoint{4.882085in}{3.059182in}}%
\pgfpathlineto{\pgfqpoint{4.884090in}{3.177818in}}%
\pgfpathlineto{\pgfqpoint{4.886094in}{3.071455in}}%
\pgfpathlineto{\pgfqpoint{4.887097in}{3.074182in}}%
\pgfpathlineto{\pgfqpoint{4.890103in}{3.128727in}}%
\pgfpathlineto{\pgfqpoint{4.892108in}{3.027818in}}%
\pgfpathlineto{\pgfqpoint{4.895114in}{3.056455in}}%
\pgfpathlineto{\pgfqpoint{4.898121in}{2.978727in}}%
\pgfpathlineto{\pgfqpoint{4.899123in}{2.997818in}}%
\pgfpathlineto{\pgfqpoint{4.900126in}{2.996455in}}%
\pgfpathlineto{\pgfqpoint{4.901128in}{2.982818in}}%
\pgfpathlineto{\pgfqpoint{4.902130in}{2.988273in}}%
\pgfpathlineto{\pgfqpoint{4.903132in}{2.947364in}}%
\pgfpathlineto{\pgfqpoint{4.905137in}{2.966455in}}%
\pgfpathlineto{\pgfqpoint{4.906139in}{2.952818in}}%
\pgfpathlineto{\pgfqpoint{4.907141in}{2.969182in}}%
\pgfpathlineto{\pgfqpoint{4.909146in}{2.913273in}}%
\pgfpathlineto{\pgfqpoint{4.912153in}{2.970545in}}%
\pgfpathlineto{\pgfqpoint{4.914157in}{2.932364in}}%
\pgfpathlineto{\pgfqpoint{4.917164in}{2.989636in}}%
\pgfpathlineto{\pgfqpoint{4.918166in}{2.965091in}}%
\pgfpathlineto{\pgfqpoint{4.919168in}{2.988273in}}%
\pgfpathlineto{\pgfqpoint{4.920171in}{2.977364in}}%
\pgfpathlineto{\pgfqpoint{4.921173in}{2.988273in}}%
\pgfpathlineto{\pgfqpoint{4.924180in}{3.060545in}}%
\pgfpathlineto{\pgfqpoint{4.926184in}{3.014182in}}%
\pgfpathlineto{\pgfqpoint{4.929191in}{3.097364in}}%
\pgfpathlineto{\pgfqpoint{4.930193in}{3.082364in}}%
\pgfpathlineto{\pgfqpoint{4.931195in}{3.042818in}}%
\pgfpathlineto{\pgfqpoint{4.934202in}{3.124636in}}%
\pgfpathlineto{\pgfqpoint{4.937209in}{3.056455in}}%
\pgfpathlineto{\pgfqpoint{4.938211in}{3.068727in}}%
\pgfpathlineto{\pgfqpoint{4.939213in}{3.102818in}}%
\pgfpathlineto{\pgfqpoint{4.941218in}{3.033273in}}%
\pgfpathlineto{\pgfqpoint{4.942220in}{3.038727in}}%
\pgfpathlineto{\pgfqpoint{4.943222in}{3.029182in}}%
\pgfpathlineto{\pgfqpoint{4.944224in}{3.057818in}}%
\pgfpathlineto{\pgfqpoint{4.946229in}{3.014182in}}%
\pgfpathlineto{\pgfqpoint{4.948233in}{2.984182in}}%
\pgfpathlineto{\pgfqpoint{4.949236in}{2.995091in}}%
\pgfpathlineto{\pgfqpoint{4.950238in}{2.993727in}}%
\pgfpathlineto{\pgfqpoint{4.952242in}{2.971909in}}%
\pgfpathlineto{\pgfqpoint{4.953245in}{2.939182in}}%
\pgfpathlineto{\pgfqpoint{4.954247in}{2.954182in}}%
\pgfpathlineto{\pgfqpoint{4.955249in}{2.947364in}}%
\pgfpathlineto{\pgfqpoint{4.956251in}{2.969182in}}%
\pgfpathlineto{\pgfqpoint{4.957254in}{2.962364in}}%
\pgfpathlineto{\pgfqpoint{4.958256in}{2.928273in}}%
\pgfpathlineto{\pgfqpoint{4.960260in}{2.937818in}}%
\pgfpathlineto{\pgfqpoint{4.962265in}{2.974636in}}%
\pgfpathlineto{\pgfqpoint{4.964269in}{2.944636in}}%
\pgfpathlineto{\pgfqpoint{4.965271in}{2.962364in}}%
\pgfpathlineto{\pgfqpoint{4.966274in}{3.001909in}}%
\pgfpathlineto{\pgfqpoint{4.968278in}{2.977364in}}%
\pgfpathlineto{\pgfqpoint{4.969280in}{2.989636in}}%
\pgfpathlineto{\pgfqpoint{4.970283in}{2.980091in}}%
\pgfpathlineto{\pgfqpoint{4.973289in}{3.045545in}}%
\pgfpathlineto{\pgfqpoint{4.974292in}{3.041455in}}%
\pgfpathlineto{\pgfqpoint{4.975294in}{3.004636in}}%
\pgfpathlineto{\pgfqpoint{4.976296in}{3.010091in}}%
\pgfpathlineto{\pgfqpoint{4.977298in}{3.026455in}}%
\pgfpathlineto{\pgfqpoint{4.978301in}{3.082364in}}%
\pgfpathlineto{\pgfqpoint{4.979303in}{3.079636in}}%
\pgfpathlineto{\pgfqpoint{4.981307in}{3.012818in}}%
\pgfpathlineto{\pgfqpoint{4.984314in}{3.109636in}}%
\pgfpathlineto{\pgfqpoint{4.986319in}{3.036000in}}%
\pgfpathlineto{\pgfqpoint{4.987321in}{3.036000in}}%
\pgfpathlineto{\pgfqpoint{4.988323in}{3.057818in}}%
\pgfpathlineto{\pgfqpoint{4.989325in}{3.116455in}}%
\pgfpathlineto{\pgfqpoint{4.992332in}{3.006000in}}%
\pgfpathlineto{\pgfqpoint{4.994337in}{3.056455in}}%
\pgfpathlineto{\pgfqpoint{4.997343in}{2.970545in}}%
\pgfpathlineto{\pgfqpoint{4.999348in}{2.980091in}}%
\pgfpathlineto{\pgfqpoint{5.001352in}{3.001909in}}%
\pgfpathlineto{\pgfqpoint{5.003357in}{2.943273in}}%
\pgfpathlineto{\pgfqpoint{5.005361in}{2.970545in}}%
\pgfpathlineto{\pgfqpoint{5.006363in}{3.010091in}}%
\pgfpathlineto{\pgfqpoint{5.008368in}{2.932364in}}%
\pgfpathlineto{\pgfqpoint{5.009370in}{2.939182in}}%
\pgfpathlineto{\pgfqpoint{5.010372in}{2.954182in}}%
\pgfpathlineto{\pgfqpoint{5.011375in}{2.986909in}}%
\pgfpathlineto{\pgfqpoint{5.012377in}{2.977364in}}%
\pgfpathlineto{\pgfqpoint{5.013379in}{2.952818in}}%
\pgfpathlineto{\pgfqpoint{5.014381in}{2.958273in}}%
\pgfpathlineto{\pgfqpoint{5.015384in}{2.954182in}}%
\pgfpathlineto{\pgfqpoint{5.017388in}{2.982818in}}%
\pgfpathlineto{\pgfqpoint{5.018390in}{2.971909in}}%
\pgfpathlineto{\pgfqpoint{5.020395in}{2.986909in}}%
\pgfpathlineto{\pgfqpoint{5.021397in}{2.997818in}}%
\pgfpathlineto{\pgfqpoint{5.022399in}{2.993727in}}%
\pgfpathlineto{\pgfqpoint{5.025406in}{3.022364in}}%
\pgfpathlineto{\pgfqpoint{5.027411in}{3.006000in}}%
\pgfpathlineto{\pgfqpoint{5.030417in}{3.064636in}}%
\pgfpathlineto{\pgfqpoint{5.031420in}{3.040091in}}%
\pgfpathlineto{\pgfqpoint{5.032422in}{2.988273in}}%
\pgfpathlineto{\pgfqpoint{5.034426in}{3.059182in}}%
\pgfpathlineto{\pgfqpoint{5.035428in}{3.066000in}}%
\pgfpathlineto{\pgfqpoint{5.037433in}{3.023727in}}%
\pgfpathlineto{\pgfqpoint{5.039437in}{3.079636in}}%
\pgfpathlineto{\pgfqpoint{5.041442in}{3.046909in}}%
\pgfpathlineto{\pgfqpoint{5.042444in}{3.006000in}}%
\pgfpathlineto{\pgfqpoint{5.043446in}{3.015545in}}%
\pgfpathlineto{\pgfqpoint{5.044449in}{3.042818in}}%
\pgfpathlineto{\pgfqpoint{5.045451in}{3.040091in}}%
\pgfpathlineto{\pgfqpoint{5.046453in}{3.026455in}}%
\pgfpathlineto{\pgfqpoint{5.047455in}{2.982818in}}%
\pgfpathlineto{\pgfqpoint{5.048458in}{2.993727in}}%
\pgfpathlineto{\pgfqpoint{5.049460in}{2.985545in}}%
\pgfpathlineto{\pgfqpoint{5.050462in}{3.022364in}}%
\pgfpathlineto{\pgfqpoint{5.051464in}{3.019636in}}%
\pgfpathlineto{\pgfqpoint{5.053469in}{2.955545in}}%
\pgfpathlineto{\pgfqpoint{5.054471in}{2.961000in}}%
\pgfpathlineto{\pgfqpoint{5.056476in}{3.006000in}}%
\pgfpathlineto{\pgfqpoint{5.058480in}{2.950091in}}%
\pgfpathlineto{\pgfqpoint{5.059482in}{2.955545in}}%
\pgfpathlineto{\pgfqpoint{5.061487in}{3.006000in}}%
\pgfpathlineto{\pgfqpoint{5.063491in}{2.944636in}}%
\pgfpathlineto{\pgfqpoint{5.064494in}{2.952818in}}%
\pgfpathlineto{\pgfqpoint{5.065496in}{2.963727in}}%
\pgfpathlineto{\pgfqpoint{5.066498in}{2.996455in}}%
\pgfpathlineto{\pgfqpoint{5.068502in}{2.973273in}}%
\pgfpathlineto{\pgfqpoint{5.069505in}{2.977364in}}%
\pgfpathlineto{\pgfqpoint{5.070507in}{3.000545in}}%
\pgfpathlineto{\pgfqpoint{5.071509in}{2.980091in}}%
\pgfpathlineto{\pgfqpoint{5.073514in}{3.012818in}}%
\pgfpathlineto{\pgfqpoint{5.074516in}{3.001909in}}%
\pgfpathlineto{\pgfqpoint{5.075518in}{3.036000in}}%
\pgfpathlineto{\pgfqpoint{5.077523in}{3.014182in}}%
\pgfpathlineto{\pgfqpoint{5.078525in}{3.037364in}}%
\pgfpathlineto{\pgfqpoint{5.079527in}{3.033273in}}%
\pgfpathlineto{\pgfqpoint{5.080529in}{3.055091in}}%
\pgfpathlineto{\pgfqpoint{5.082534in}{3.019636in}}%
\pgfpathlineto{\pgfqpoint{5.084538in}{3.067364in}}%
\pgfpathlineto{\pgfqpoint{5.085541in}{3.074182in}}%
\pgfpathlineto{\pgfqpoint{5.087545in}{3.015545in}}%
\pgfpathlineto{\pgfqpoint{5.088547in}{3.053727in}}%
\pgfpathlineto{\pgfqpoint{5.089550in}{3.046909in}}%
\pgfpathlineto{\pgfqpoint{5.090552in}{3.057818in}}%
\pgfpathlineto{\pgfqpoint{5.092556in}{3.011455in}}%
\pgfpathlineto{\pgfqpoint{5.095563in}{3.040091in}}%
\pgfpathlineto{\pgfqpoint{5.098570in}{2.976000in}}%
\pgfpathlineto{\pgfqpoint{5.100574in}{3.012818in}}%
\pgfpathlineto{\pgfqpoint{5.103581in}{2.959636in}}%
\pgfpathlineto{\pgfqpoint{5.104583in}{2.962364in}}%
\pgfpathlineto{\pgfqpoint{5.105585in}{2.992364in}}%
\pgfpathlineto{\pgfqpoint{5.106588in}{2.985545in}}%
\pgfpathlineto{\pgfqpoint{5.109594in}{2.941909in}}%
\pgfpathlineto{\pgfqpoint{5.111599in}{2.993727in}}%
\pgfpathlineto{\pgfqpoint{5.114606in}{2.952818in}}%
\pgfpathlineto{\pgfqpoint{5.116610in}{2.989636in}}%
\pgfpathlineto{\pgfqpoint{5.118615in}{2.978727in}}%
\pgfpathlineto{\pgfqpoint{5.119617in}{2.981455in}}%
\pgfpathlineto{\pgfqpoint{5.121621in}{2.999182in}}%
\pgfpathlineto{\pgfqpoint{5.122624in}{3.010091in}}%
\pgfpathlineto{\pgfqpoint{5.124628in}{3.003273in}}%
\pgfpathlineto{\pgfqpoint{5.125630in}{3.022364in}}%
\pgfpathlineto{\pgfqpoint{5.126633in}{3.012818in}}%
\pgfpathlineto{\pgfqpoint{5.128637in}{3.046909in}}%
\pgfpathlineto{\pgfqpoint{5.130642in}{3.038727in}}%
\pgfpathlineto{\pgfqpoint{5.131644in}{3.004636in}}%
\pgfpathlineto{\pgfqpoint{5.133648in}{3.056455in}}%
\pgfpathlineto{\pgfqpoint{5.134651in}{3.051000in}}%
\pgfpathlineto{\pgfqpoint{5.135653in}{3.053727in}}%
\pgfpathlineto{\pgfqpoint{5.136655in}{3.021000in}}%
\pgfpathlineto{\pgfqpoint{5.137657in}{3.025091in}}%
\pgfpathlineto{\pgfqpoint{5.138660in}{3.053727in}}%
\pgfpathlineto{\pgfqpoint{5.139662in}{3.051000in}}%
\pgfpathlineto{\pgfqpoint{5.140664in}{3.051000in}}%
\pgfpathlineto{\pgfqpoint{5.141666in}{3.006000in}}%
\pgfpathlineto{\pgfqpoint{5.142668in}{3.026455in}}%
\pgfpathlineto{\pgfqpoint{5.143671in}{3.010091in}}%
\pgfpathlineto{\pgfqpoint{5.144673in}{3.042818in}}%
\pgfpathlineto{\pgfqpoint{5.145675in}{3.031909in}}%
\pgfpathlineto{\pgfqpoint{5.147680in}{2.988273in}}%
\pgfpathlineto{\pgfqpoint{5.149684in}{3.007364in}}%
\pgfpathlineto{\pgfqpoint{5.150686in}{3.014182in}}%
\pgfpathlineto{\pgfqpoint{5.151689in}{2.985545in}}%
\pgfpathlineto{\pgfqpoint{5.152691in}{3.001909in}}%
\pgfpathlineto{\pgfqpoint{5.153693in}{2.973273in}}%
\pgfpathlineto{\pgfqpoint{5.154695in}{2.976000in}}%
\pgfpathlineto{\pgfqpoint{5.155698in}{2.993727in}}%
\pgfpathlineto{\pgfqpoint{5.157702in}{2.986909in}}%
\pgfpathlineto{\pgfqpoint{5.159707in}{2.958273in}}%
\pgfpathlineto{\pgfqpoint{5.161711in}{2.993727in}}%
\pgfpathlineto{\pgfqpoint{5.162713in}{2.995091in}}%
\pgfpathlineto{\pgfqpoint{5.164718in}{2.966455in}}%
\pgfpathlineto{\pgfqpoint{5.166722in}{2.988273in}}%
\pgfpathlineto{\pgfqpoint{5.167725in}{2.995091in}}%
\pgfpathlineto{\pgfqpoint{5.169729in}{2.981455in}}%
\pgfpathlineto{\pgfqpoint{5.170731in}{2.982818in}}%
\pgfpathlineto{\pgfqpoint{5.174740in}{3.029182in}}%
\pgfpathlineto{\pgfqpoint{5.175742in}{3.006000in}}%
\pgfpathlineto{\pgfqpoint{5.176745in}{3.007364in}}%
\pgfpathlineto{\pgfqpoint{5.177747in}{3.008727in}}%
\pgfpathlineto{\pgfqpoint{5.178749in}{3.014182in}}%
\pgfpathlineto{\pgfqpoint{5.179751in}{3.036000in}}%
\pgfpathlineto{\pgfqpoint{5.180754in}{3.027818in}}%
\pgfpathlineto{\pgfqpoint{5.181756in}{2.992364in}}%
\pgfpathlineto{\pgfqpoint{5.182758in}{3.026455in}}%
\pgfpathlineto{\pgfqpoint{5.183760in}{3.025091in}}%
\pgfpathlineto{\pgfqpoint{5.185765in}{3.059182in}}%
\pgfpathlineto{\pgfqpoint{5.186767in}{3.015545in}}%
\pgfpathlineto{\pgfqpoint{5.188772in}{3.057818in}}%
\pgfpathlineto{\pgfqpoint{5.189774in}{3.040091in}}%
\pgfpathlineto{\pgfqpoint{5.190776in}{3.057818in}}%
\pgfpathlineto{\pgfqpoint{5.192781in}{3.033273in}}%
\pgfpathlineto{\pgfqpoint{5.193783in}{3.029182in}}%
\pgfpathlineto{\pgfqpoint{5.194785in}{3.049636in}}%
\pgfpathlineto{\pgfqpoint{5.195787in}{3.045545in}}%
\pgfpathlineto{\pgfqpoint{5.196790in}{3.018273in}}%
\pgfpathlineto{\pgfqpoint{5.197792in}{3.027818in}}%
\pgfpathlineto{\pgfqpoint{5.198794in}{3.021000in}}%
\pgfpathlineto{\pgfqpoint{5.199796in}{3.038727in}}%
\pgfpathlineto{\pgfqpoint{5.201801in}{3.010091in}}%
\pgfpathlineto{\pgfqpoint{5.203805in}{2.997818in}}%
\pgfpathlineto{\pgfqpoint{5.204808in}{3.008727in}}%
\pgfpathlineto{\pgfqpoint{5.205810in}{3.007364in}}%
\pgfpathlineto{\pgfqpoint{5.206812in}{2.995091in}}%
\pgfpathlineto{\pgfqpoint{5.207814in}{3.011455in}}%
\pgfpathlineto{\pgfqpoint{5.208817in}{2.986909in}}%
\pgfpathlineto{\pgfqpoint{5.210821in}{2.997818in}}%
\pgfpathlineto{\pgfqpoint{5.211823in}{2.986909in}}%
\pgfpathlineto{\pgfqpoint{5.212825in}{2.997818in}}%
\pgfpathlineto{\pgfqpoint{5.213828in}{2.970545in}}%
\pgfpathlineto{\pgfqpoint{5.215832in}{2.984182in}}%
\pgfpathlineto{\pgfqpoint{5.216834in}{2.981455in}}%
\pgfpathlineto{\pgfqpoint{5.217837in}{2.995091in}}%
\pgfpathlineto{\pgfqpoint{5.218839in}{2.986909in}}%
\pgfpathlineto{\pgfqpoint{5.219841in}{2.988273in}}%
\pgfpathlineto{\pgfqpoint{5.222848in}{2.996455in}}%
\pgfpathlineto{\pgfqpoint{5.223850in}{2.999182in}}%
\pgfpathlineto{\pgfqpoint{5.224852in}{3.015545in}}%
\pgfpathlineto{\pgfqpoint{5.225855in}{2.999182in}}%
\pgfpathlineto{\pgfqpoint{5.227859in}{3.021000in}}%
\pgfpathlineto{\pgfqpoint{5.228861in}{3.018273in}}%
\pgfpathlineto{\pgfqpoint{5.229864in}{3.027818in}}%
\pgfpathlineto{\pgfqpoint{5.230866in}{3.012818in}}%
\pgfpathlineto{\pgfqpoint{5.233873in}{3.037364in}}%
\pgfpathlineto{\pgfqpoint{5.234875in}{3.072818in}}%
\pgfpathlineto{\pgfqpoint{5.235877in}{3.021000in}}%
\pgfpathlineto{\pgfqpoint{5.239886in}{3.052364in}}%
\pgfpathlineto{\pgfqpoint{5.241891in}{3.027818in}}%
\pgfpathlineto{\pgfqpoint{5.243895in}{3.037364in}}%
\pgfpathlineto{\pgfqpoint{5.244897in}{3.038727in}}%
\pgfpathlineto{\pgfqpoint{5.246902in}{3.018273in}}%
\pgfpathlineto{\pgfqpoint{5.247904in}{3.008727in}}%
\pgfpathlineto{\pgfqpoint{5.248906in}{3.012818in}}%
\pgfpathlineto{\pgfqpoint{5.249908in}{3.007364in}}%
\pgfpathlineto{\pgfqpoint{5.250911in}{3.012818in}}%
\pgfpathlineto{\pgfqpoint{5.251913in}{3.011455in}}%
\pgfpathlineto{\pgfqpoint{5.253917in}{2.989636in}}%
\pgfpathlineto{\pgfqpoint{5.254920in}{3.003273in}}%
\pgfpathlineto{\pgfqpoint{5.255922in}{2.996455in}}%
\pgfpathlineto{\pgfqpoint{5.256924in}{3.012818in}}%
\pgfpathlineto{\pgfqpoint{5.258929in}{2.976000in}}%
\pgfpathlineto{\pgfqpoint{5.259931in}{2.993727in}}%
\pgfpathlineto{\pgfqpoint{5.262938in}{2.981455in}}%
\pgfpathlineto{\pgfqpoint{5.263940in}{2.982818in}}%
\pgfpathlineto{\pgfqpoint{5.264942in}{2.981455in}}%
\pgfpathlineto{\pgfqpoint{5.265944in}{2.973273in}}%
\pgfpathlineto{\pgfqpoint{5.267949in}{2.995091in}}%
\pgfpathlineto{\pgfqpoint{5.268951in}{2.989636in}}%
\pgfpathlineto{\pgfqpoint{5.269953in}{3.003273in}}%
\pgfpathlineto{\pgfqpoint{5.270956in}{2.992364in}}%
\pgfpathlineto{\pgfqpoint{5.272960in}{3.006000in}}%
\pgfpathlineto{\pgfqpoint{5.273962in}{3.007364in}}%
\pgfpathlineto{\pgfqpoint{5.274965in}{3.019636in}}%
\pgfpathlineto{\pgfqpoint{5.275967in}{3.010091in}}%
\pgfpathlineto{\pgfqpoint{5.276969in}{3.025091in}}%
\pgfpathlineto{\pgfqpoint{5.277971in}{3.010091in}}%
\pgfpathlineto{\pgfqpoint{5.279976in}{3.034636in}}%
\pgfpathlineto{\pgfqpoint{5.280978in}{3.022364in}}%
\pgfpathlineto{\pgfqpoint{5.281980in}{3.026455in}}%
\pgfpathlineto{\pgfqpoint{5.283985in}{3.044182in}}%
\pgfpathlineto{\pgfqpoint{5.284987in}{3.040091in}}%
\pgfpathlineto{\pgfqpoint{5.285989in}{3.014182in}}%
\pgfpathlineto{\pgfqpoint{5.288996in}{3.036000in}}%
\pgfpathlineto{\pgfqpoint{5.289998in}{3.031909in}}%
\pgfpathlineto{\pgfqpoint{5.291000in}{3.008727in}}%
\pgfpathlineto{\pgfqpoint{5.292003in}{3.012818in}}%
\pgfpathlineto{\pgfqpoint{5.294007in}{3.036000in}}%
\pgfpathlineto{\pgfqpoint{5.297014in}{2.993727in}}%
\pgfpathlineto{\pgfqpoint{5.299018in}{3.019636in}}%
\pgfpathlineto{\pgfqpoint{5.300021in}{3.004636in}}%
\pgfpathlineto{\pgfqpoint{5.301023in}{3.012818in}}%
\pgfpathlineto{\pgfqpoint{5.303027in}{2.991000in}}%
\pgfpathlineto{\pgfqpoint{5.305032in}{3.003273in}}%
\pgfpathlineto{\pgfqpoint{5.306034in}{3.008727in}}%
\pgfpathlineto{\pgfqpoint{5.309041in}{2.982818in}}%
\pgfpathlineto{\pgfqpoint{5.310043in}{2.977364in}}%
\pgfpathlineto{\pgfqpoint{5.312048in}{3.001909in}}%
\pgfpathlineto{\pgfqpoint{5.314052in}{2.974636in}}%
\pgfpathlineto{\pgfqpoint{5.315054in}{2.974636in}}%
\pgfpathlineto{\pgfqpoint{5.317059in}{2.999182in}}%
\pgfpathlineto{\pgfqpoint{5.319063in}{2.986909in}}%
\pgfpathlineto{\pgfqpoint{5.322070in}{3.001909in}}%
\pgfpathlineto{\pgfqpoint{5.323072in}{2.991000in}}%
\pgfpathlineto{\pgfqpoint{5.325077in}{3.034636in}}%
\pgfpathlineto{\pgfqpoint{5.327081in}{2.986909in}}%
\pgfpathlineto{\pgfqpoint{5.328083in}{2.986909in}}%
\pgfpathlineto{\pgfqpoint{5.329086in}{3.036000in}}%
\pgfpathlineto{\pgfqpoint{5.331090in}{3.012818in}}%
\pgfpathlineto{\pgfqpoint{5.332092in}{2.992364in}}%
\pgfpathlineto{\pgfqpoint{5.335099in}{3.041455in}}%
\pgfpathlineto{\pgfqpoint{5.338106in}{3.007364in}}%
\pgfpathlineto{\pgfqpoint{5.341113in}{3.033273in}}%
\pgfpathlineto{\pgfqpoint{5.342115in}{3.004636in}}%
\pgfpathlineto{\pgfqpoint{5.345122in}{3.027818in}}%
\pgfpathlineto{\pgfqpoint{5.347126in}{3.012818in}}%
\pgfpathlineto{\pgfqpoint{5.348128in}{3.015545in}}%
\pgfpathlineto{\pgfqpoint{5.349131in}{3.012818in}}%
\pgfpathlineto{\pgfqpoint{5.350133in}{3.006000in}}%
\pgfpathlineto{\pgfqpoint{5.351135in}{3.010091in}}%
\pgfpathlineto{\pgfqpoint{5.354142in}{2.980091in}}%
\pgfpathlineto{\pgfqpoint{5.356146in}{3.011455in}}%
\pgfpathlineto{\pgfqpoint{5.357148in}{2.991000in}}%
\pgfpathlineto{\pgfqpoint{5.361157in}{3.016909in}}%
\pgfpathlineto{\pgfqpoint{5.362160in}{2.978727in}}%
\pgfpathlineto{\pgfqpoint{5.363162in}{3.022364in}}%
\pgfpathlineto{\pgfqpoint{5.366169in}{2.981455in}}%
\pgfpathlineto{\pgfqpoint{5.367171in}{3.014182in}}%
\pgfpathlineto{\pgfqpoint{5.368173in}{3.011455in}}%
\pgfpathlineto{\pgfqpoint{5.370178in}{2.982818in}}%
\pgfpathlineto{\pgfqpoint{5.371180in}{3.011455in}}%
\pgfpathlineto{\pgfqpoint{5.372182in}{3.000545in}}%
\pgfpathlineto{\pgfqpoint{5.373184in}{3.001909in}}%
\pgfpathlineto{\pgfqpoint{5.375189in}{3.027818in}}%
\pgfpathlineto{\pgfqpoint{5.377193in}{3.001909in}}%
\pgfpathlineto{\pgfqpoint{5.378196in}{3.004636in}}%
\pgfpathlineto{\pgfqpoint{5.379198in}{3.029182in}}%
\pgfpathlineto{\pgfqpoint{5.380200in}{3.027818in}}%
\pgfpathlineto{\pgfqpoint{5.382205in}{2.992364in}}%
\pgfpathlineto{\pgfqpoint{5.384209in}{3.029182in}}%
\pgfpathlineto{\pgfqpoint{5.385211in}{3.034636in}}%
\pgfpathlineto{\pgfqpoint{5.387216in}{3.011455in}}%
\pgfpathlineto{\pgfqpoint{5.388218in}{3.015545in}}%
\pgfpathlineto{\pgfqpoint{5.389220in}{3.034636in}}%
\pgfpathlineto{\pgfqpoint{5.390222in}{3.021000in}}%
\pgfpathlineto{\pgfqpoint{5.391225in}{3.023727in}}%
\pgfpathlineto{\pgfqpoint{5.392227in}{3.006000in}}%
\pgfpathlineto{\pgfqpoint{5.394231in}{3.023727in}}%
\pgfpathlineto{\pgfqpoint{5.395234in}{3.014182in}}%
\pgfpathlineto{\pgfqpoint{5.396236in}{3.016909in}}%
\pgfpathlineto{\pgfqpoint{5.397238in}{3.010091in}}%
\pgfpathlineto{\pgfqpoint{5.398240in}{3.019636in}}%
\pgfpathlineto{\pgfqpoint{5.399243in}{3.012818in}}%
\pgfpathlineto{\pgfqpoint{5.400245in}{3.018273in}}%
\pgfpathlineto{\pgfqpoint{5.401247in}{3.014182in}}%
\pgfpathlineto{\pgfqpoint{5.402249in}{2.999182in}}%
\pgfpathlineto{\pgfqpoint{5.403252in}{3.021000in}}%
\pgfpathlineto{\pgfqpoint{5.404254in}{2.995091in}}%
\pgfpathlineto{\pgfqpoint{5.405256in}{3.007364in}}%
\pgfpathlineto{\pgfqpoint{5.407261in}{3.003273in}}%
\pgfpathlineto{\pgfqpoint{5.408263in}{2.986909in}}%
\pgfpathlineto{\pgfqpoint{5.409265in}{2.993727in}}%
\pgfpathlineto{\pgfqpoint{5.410267in}{2.992364in}}%
\pgfpathlineto{\pgfqpoint{5.411270in}{2.995091in}}%
\pgfpathlineto{\pgfqpoint{5.413274in}{3.012818in}}%
\pgfpathlineto{\pgfqpoint{5.416281in}{2.970545in}}%
\pgfpathlineto{\pgfqpoint{5.417283in}{2.982818in}}%
\pgfpathlineto{\pgfqpoint{5.418285in}{3.018273in}}%
\pgfpathlineto{\pgfqpoint{5.420290in}{2.996455in}}%
\pgfpathlineto{\pgfqpoint{5.421292in}{3.001909in}}%
\pgfpathlineto{\pgfqpoint{5.422294in}{2.986909in}}%
\pgfpathlineto{\pgfqpoint{5.425301in}{3.025091in}}%
\pgfpathlineto{\pgfqpoint{5.426303in}{2.997818in}}%
\pgfpathlineto{\pgfqpoint{5.427305in}{3.001909in}}%
\pgfpathlineto{\pgfqpoint{5.428308in}{3.000545in}}%
\pgfpathlineto{\pgfqpoint{5.429310in}{3.022364in}}%
\pgfpathlineto{\pgfqpoint{5.431314in}{2.996455in}}%
\pgfpathlineto{\pgfqpoint{5.432317in}{2.997818in}}%
\pgfpathlineto{\pgfqpoint{5.433319in}{3.041455in}}%
\pgfpathlineto{\pgfqpoint{5.435323in}{3.004636in}}%
\pgfpathlineto{\pgfqpoint{5.436326in}{3.021000in}}%
\pgfpathlineto{\pgfqpoint{5.437328in}{3.012818in}}%
\pgfpathlineto{\pgfqpoint{5.438330in}{3.025091in}}%
\pgfpathlineto{\pgfqpoint{5.439332in}{3.023727in}}%
\pgfpathlineto{\pgfqpoint{5.440335in}{3.004636in}}%
\pgfpathlineto{\pgfqpoint{5.442339in}{3.023727in}}%
\pgfpathlineto{\pgfqpoint{5.443341in}{3.007364in}}%
\pgfpathlineto{\pgfqpoint{5.444344in}{3.029182in}}%
\pgfpathlineto{\pgfqpoint{5.445346in}{3.025091in}}%
\pgfpathlineto{\pgfqpoint{5.447350in}{3.004636in}}%
\pgfpathlineto{\pgfqpoint{5.449355in}{3.027818in}}%
\pgfpathlineto{\pgfqpoint{5.451359in}{2.999182in}}%
\pgfpathlineto{\pgfqpoint{5.454366in}{3.025091in}}%
\pgfpathlineto{\pgfqpoint{5.455368in}{3.026455in}}%
\pgfpathlineto{\pgfqpoint{5.456371in}{3.010091in}}%
\pgfpathlineto{\pgfqpoint{5.457373in}{3.012818in}}%
\pgfpathlineto{\pgfqpoint{5.458375in}{3.010091in}}%
\pgfpathlineto{\pgfqpoint{5.459377in}{2.997818in}}%
\pgfpathlineto{\pgfqpoint{5.461382in}{3.016909in}}%
\pgfpathlineto{\pgfqpoint{5.463386in}{2.985545in}}%
\pgfpathlineto{\pgfqpoint{5.464388in}{2.988273in}}%
\pgfpathlineto{\pgfqpoint{5.466393in}{3.006000in}}%
\pgfpathlineto{\pgfqpoint{5.467395in}{2.992364in}}%
\pgfpathlineto{\pgfqpoint{5.468397in}{2.993727in}}%
\pgfpathlineto{\pgfqpoint{5.469400in}{2.999182in}}%
\pgfpathlineto{\pgfqpoint{5.470402in}{2.996455in}}%
\pgfpathlineto{\pgfqpoint{5.471404in}{3.003273in}}%
\pgfpathlineto{\pgfqpoint{5.472406in}{2.999182in}}%
\pgfpathlineto{\pgfqpoint{5.473409in}{3.000545in}}%
\pgfpathlineto{\pgfqpoint{5.476415in}{2.985545in}}%
\pgfpathlineto{\pgfqpoint{5.477418in}{3.008727in}}%
\pgfpathlineto{\pgfqpoint{5.479422in}{3.001909in}}%
\pgfpathlineto{\pgfqpoint{5.481427in}{3.014182in}}%
\pgfpathlineto{\pgfqpoint{5.482429in}{2.993727in}}%
\pgfpathlineto{\pgfqpoint{5.485436in}{3.015545in}}%
\pgfpathlineto{\pgfqpoint{5.486438in}{2.991000in}}%
\pgfpathlineto{\pgfqpoint{5.488442in}{3.003273in}}%
\pgfpathlineto{\pgfqpoint{5.489445in}{3.000545in}}%
\pgfpathlineto{\pgfqpoint{5.490447in}{3.042818in}}%
\pgfpathlineto{\pgfqpoint{5.492451in}{3.007364in}}%
\pgfpathlineto{\pgfqpoint{5.493453in}{3.010091in}}%
\pgfpathlineto{\pgfqpoint{5.494456in}{3.008727in}}%
\pgfpathlineto{\pgfqpoint{5.495458in}{3.021000in}}%
\pgfpathlineto{\pgfqpoint{5.496460in}{3.016909in}}%
\pgfpathlineto{\pgfqpoint{5.498465in}{3.006000in}}%
\pgfpathlineto{\pgfqpoint{5.499467in}{3.018273in}}%
\pgfpathlineto{\pgfqpoint{5.500469in}{3.007364in}}%
\pgfpathlineto{\pgfqpoint{5.501471in}{3.021000in}}%
\pgfpathlineto{\pgfqpoint{5.502474in}{3.014182in}}%
\pgfpathlineto{\pgfqpoint{5.504478in}{2.988273in}}%
\pgfpathlineto{\pgfqpoint{5.506483in}{3.010091in}}%
\pgfpathlineto{\pgfqpoint{5.507485in}{3.010091in}}%
\pgfpathlineto{\pgfqpoint{5.508487in}{2.989636in}}%
\pgfpathlineto{\pgfqpoint{5.510492in}{3.006000in}}%
\pgfpathlineto{\pgfqpoint{5.512496in}{2.989636in}}%
\pgfpathlineto{\pgfqpoint{5.514501in}{3.001909in}}%
\pgfpathlineto{\pgfqpoint{5.515503in}{3.001909in}}%
\pgfpathlineto{\pgfqpoint{5.516505in}{2.981455in}}%
\pgfpathlineto{\pgfqpoint{5.517507in}{3.004636in}}%
\pgfpathlineto{\pgfqpoint{5.518510in}{2.997818in}}%
\pgfpathlineto{\pgfqpoint{5.519512in}{2.999182in}}%
\pgfpathlineto{\pgfqpoint{5.522519in}{3.010091in}}%
\pgfpathlineto{\pgfqpoint{5.523521in}{3.014182in}}%
\pgfpathlineto{\pgfqpoint{5.524523in}{3.006000in}}%
\pgfpathlineto{\pgfqpoint{5.525525in}{3.016909in}}%
\pgfpathlineto{\pgfqpoint{5.526528in}{3.000545in}}%
\pgfpathlineto{\pgfqpoint{5.527530in}{3.007364in}}%
\pgfpathlineto{\pgfqpoint{5.528532in}{2.999182in}}%
\pgfpathlineto{\pgfqpoint{5.530536in}{3.008727in}}%
\pgfpathlineto{\pgfqpoint{5.532541in}{3.023727in}}%
\pgfpathlineto{\pgfqpoint{5.533543in}{3.010091in}}%
\pgfpathlineto{\pgfqpoint{5.534545in}{3.016909in}}%
\pgfpathlineto{\pgfqpoint{5.534545in}{3.016909in}}%
\pgfusepath{stroke}%
\end{pgfscope}%
\begin{pgfscope}%
\pgfpathrectangle{\pgfqpoint{0.800000in}{0.528000in}}{\pgfqpoint{4.960000in}{3.696000in}}%
\pgfusepath{clip}%
\pgfsetrectcap%
\pgfsetroundjoin%
\pgfsetlinewidth{1.505625pt}%
\definecolor{currentstroke}{rgb}{0.564706,0.478431,0.662745}%
\pgfsetstrokecolor{currentstroke}%
\pgfsetdash{}{0pt}%
\pgfpathmoveto{\pgfqpoint{1.025455in}{3.023727in}}%
\pgfpathlineto{\pgfqpoint{1.027459in}{2.992364in}}%
\pgfpathlineto{\pgfqpoint{1.028461in}{2.988273in}}%
\pgfpathlineto{\pgfqpoint{1.029464in}{2.992364in}}%
\pgfpathlineto{\pgfqpoint{1.030466in}{2.984182in}}%
\pgfpathlineto{\pgfqpoint{1.031468in}{2.986909in}}%
\pgfpathlineto{\pgfqpoint{1.032470in}{2.996455in}}%
\pgfpathlineto{\pgfqpoint{1.033472in}{2.993727in}}%
\pgfpathlineto{\pgfqpoint{1.034475in}{2.996455in}}%
\pgfpathlineto{\pgfqpoint{1.035477in}{2.995091in}}%
\pgfpathlineto{\pgfqpoint{1.036479in}{2.984182in}}%
\pgfpathlineto{\pgfqpoint{1.037481in}{2.992364in}}%
\pgfpathlineto{\pgfqpoint{1.038484in}{2.978727in}}%
\pgfpathlineto{\pgfqpoint{1.039486in}{2.982818in}}%
\pgfpathlineto{\pgfqpoint{1.040488in}{2.978727in}}%
\pgfpathlineto{\pgfqpoint{1.041490in}{2.981455in}}%
\pgfpathlineto{\pgfqpoint{1.043495in}{2.967818in}}%
\pgfpathlineto{\pgfqpoint{1.045499in}{2.963727in}}%
\pgfpathlineto{\pgfqpoint{1.048506in}{2.973273in}}%
\pgfpathlineto{\pgfqpoint{1.049508in}{2.974636in}}%
\pgfpathlineto{\pgfqpoint{1.050511in}{2.959636in}}%
\pgfpathlineto{\pgfqpoint{1.051513in}{2.982818in}}%
\pgfpathlineto{\pgfqpoint{1.053517in}{2.966455in}}%
\pgfpathlineto{\pgfqpoint{1.055522in}{2.978727in}}%
\pgfpathlineto{\pgfqpoint{1.056524in}{2.959636in}}%
\pgfpathlineto{\pgfqpoint{1.058529in}{2.978727in}}%
\pgfpathlineto{\pgfqpoint{1.059531in}{2.980091in}}%
\pgfpathlineto{\pgfqpoint{1.063540in}{2.999182in}}%
\pgfpathlineto{\pgfqpoint{1.064542in}{2.978727in}}%
\pgfpathlineto{\pgfqpoint{1.065544in}{2.992364in}}%
\pgfpathlineto{\pgfqpoint{1.066547in}{2.991000in}}%
\pgfpathlineto{\pgfqpoint{1.067549in}{2.995091in}}%
\pgfpathlineto{\pgfqpoint{1.068551in}{2.985545in}}%
\pgfpathlineto{\pgfqpoint{1.069553in}{2.986909in}}%
\pgfpathlineto{\pgfqpoint{1.071558in}{2.999182in}}%
\pgfpathlineto{\pgfqpoint{1.075567in}{2.973273in}}%
\pgfpathlineto{\pgfqpoint{1.076569in}{2.974636in}}%
\pgfpathlineto{\pgfqpoint{1.078573in}{2.969182in}}%
\pgfpathlineto{\pgfqpoint{1.079576in}{2.970545in}}%
\pgfpathlineto{\pgfqpoint{1.081580in}{2.959636in}}%
\pgfpathlineto{\pgfqpoint{1.083585in}{2.969182in}}%
\pgfpathlineto{\pgfqpoint{1.084587in}{2.970545in}}%
\pgfpathlineto{\pgfqpoint{1.086591in}{2.952818in}}%
\pgfpathlineto{\pgfqpoint{1.087594in}{2.981455in}}%
\pgfpathlineto{\pgfqpoint{1.088596in}{2.971909in}}%
\pgfpathlineto{\pgfqpoint{1.091603in}{2.982818in}}%
\pgfpathlineto{\pgfqpoint{1.092605in}{2.980091in}}%
\pgfpathlineto{\pgfqpoint{1.093607in}{2.989636in}}%
\pgfpathlineto{\pgfqpoint{1.094609in}{2.970545in}}%
\pgfpathlineto{\pgfqpoint{1.095612in}{2.986909in}}%
\pgfpathlineto{\pgfqpoint{1.096614in}{2.981455in}}%
\pgfpathlineto{\pgfqpoint{1.099621in}{2.991000in}}%
\pgfpathlineto{\pgfqpoint{1.100623in}{2.977364in}}%
\pgfpathlineto{\pgfqpoint{1.101625in}{2.993727in}}%
\pgfpathlineto{\pgfqpoint{1.102627in}{2.978727in}}%
\pgfpathlineto{\pgfqpoint{1.104632in}{2.985545in}}%
\pgfpathlineto{\pgfqpoint{1.106636in}{2.988273in}}%
\pgfpathlineto{\pgfqpoint{1.108641in}{2.984182in}}%
\pgfpathlineto{\pgfqpoint{1.110645in}{2.973273in}}%
\pgfpathlineto{\pgfqpoint{1.111647in}{2.965091in}}%
\pgfpathlineto{\pgfqpoint{1.113652in}{2.986909in}}%
\pgfpathlineto{\pgfqpoint{1.116659in}{2.963727in}}%
\pgfpathlineto{\pgfqpoint{1.117661in}{2.965091in}}%
\pgfpathlineto{\pgfqpoint{1.119665in}{2.980091in}}%
\pgfpathlineto{\pgfqpoint{1.120668in}{2.965091in}}%
\pgfpathlineto{\pgfqpoint{1.121670in}{2.969182in}}%
\pgfpathlineto{\pgfqpoint{1.122672in}{2.961000in}}%
\pgfpathlineto{\pgfqpoint{1.124677in}{2.974636in}}%
\pgfpathlineto{\pgfqpoint{1.126681in}{2.967818in}}%
\pgfpathlineto{\pgfqpoint{1.129688in}{2.985545in}}%
\pgfpathlineto{\pgfqpoint{1.131692in}{2.982818in}}%
\pgfpathlineto{\pgfqpoint{1.133697in}{2.986909in}}%
\pgfpathlineto{\pgfqpoint{1.134699in}{2.973273in}}%
\pgfpathlineto{\pgfqpoint{1.137706in}{2.996455in}}%
\pgfpathlineto{\pgfqpoint{1.139710in}{2.973273in}}%
\pgfpathlineto{\pgfqpoint{1.142717in}{2.986909in}}%
\pgfpathlineto{\pgfqpoint{1.144721in}{2.977364in}}%
\pgfpathlineto{\pgfqpoint{1.145724in}{2.984182in}}%
\pgfpathlineto{\pgfqpoint{1.147728in}{2.977364in}}%
\pgfpathlineto{\pgfqpoint{1.148730in}{2.980091in}}%
\pgfpathlineto{\pgfqpoint{1.151737in}{2.963727in}}%
\pgfpathlineto{\pgfqpoint{1.152739in}{2.973273in}}%
\pgfpathlineto{\pgfqpoint{1.153742in}{2.961000in}}%
\pgfpathlineto{\pgfqpoint{1.155746in}{2.969182in}}%
\pgfpathlineto{\pgfqpoint{1.156748in}{2.969182in}}%
\pgfpathlineto{\pgfqpoint{1.157751in}{2.981455in}}%
\pgfpathlineto{\pgfqpoint{1.158753in}{2.976000in}}%
\pgfpathlineto{\pgfqpoint{1.160757in}{2.988273in}}%
\pgfpathlineto{\pgfqpoint{1.162762in}{2.980091in}}%
\pgfpathlineto{\pgfqpoint{1.163764in}{2.984182in}}%
\pgfpathlineto{\pgfqpoint{1.164766in}{2.977364in}}%
\pgfpathlineto{\pgfqpoint{1.165769in}{2.991000in}}%
\pgfpathlineto{\pgfqpoint{1.166771in}{2.986909in}}%
\pgfpathlineto{\pgfqpoint{1.167773in}{2.996455in}}%
\pgfpathlineto{\pgfqpoint{1.168775in}{2.988273in}}%
\pgfpathlineto{\pgfqpoint{1.169778in}{2.993727in}}%
\pgfpathlineto{\pgfqpoint{1.170780in}{2.984182in}}%
\pgfpathlineto{\pgfqpoint{1.173786in}{2.989636in}}%
\pgfpathlineto{\pgfqpoint{1.175791in}{2.984182in}}%
\pgfpathlineto{\pgfqpoint{1.176793in}{2.976000in}}%
\pgfpathlineto{\pgfqpoint{1.177795in}{2.995091in}}%
\pgfpathlineto{\pgfqpoint{1.179800in}{2.969182in}}%
\pgfpathlineto{\pgfqpoint{1.180802in}{2.977364in}}%
\pgfpathlineto{\pgfqpoint{1.181804in}{2.973273in}}%
\pgfpathlineto{\pgfqpoint{1.182807in}{2.963727in}}%
\pgfpathlineto{\pgfqpoint{1.183809in}{2.981455in}}%
\pgfpathlineto{\pgfqpoint{1.184811in}{2.969182in}}%
\pgfpathlineto{\pgfqpoint{1.185813in}{2.982818in}}%
\pgfpathlineto{\pgfqpoint{1.187818in}{2.966455in}}%
\pgfpathlineto{\pgfqpoint{1.190825in}{2.977364in}}%
\pgfpathlineto{\pgfqpoint{1.191827in}{2.970545in}}%
\pgfpathlineto{\pgfqpoint{1.193831in}{2.988273in}}%
\pgfpathlineto{\pgfqpoint{1.194834in}{2.985545in}}%
\pgfpathlineto{\pgfqpoint{1.195836in}{2.988273in}}%
\pgfpathlineto{\pgfqpoint{1.196838in}{2.996455in}}%
\pgfpathlineto{\pgfqpoint{1.197840in}{2.993727in}}%
\pgfpathlineto{\pgfqpoint{1.199845in}{2.986909in}}%
\pgfpathlineto{\pgfqpoint{1.200847in}{2.982818in}}%
\pgfpathlineto{\pgfqpoint{1.201849in}{2.989636in}}%
\pgfpathlineto{\pgfqpoint{1.202852in}{2.985545in}}%
\pgfpathlineto{\pgfqpoint{1.203854in}{2.988273in}}%
\pgfpathlineto{\pgfqpoint{1.204856in}{2.984182in}}%
\pgfpathlineto{\pgfqpoint{1.205858in}{2.974636in}}%
\pgfpathlineto{\pgfqpoint{1.206861in}{2.991000in}}%
\pgfpathlineto{\pgfqpoint{1.207863in}{2.989636in}}%
\pgfpathlineto{\pgfqpoint{1.209867in}{2.993727in}}%
\pgfpathlineto{\pgfqpoint{1.210869in}{2.966455in}}%
\pgfpathlineto{\pgfqpoint{1.211872in}{2.970545in}}%
\pgfpathlineto{\pgfqpoint{1.212874in}{2.985545in}}%
\pgfpathlineto{\pgfqpoint{1.214878in}{2.973273in}}%
\pgfpathlineto{\pgfqpoint{1.215881in}{2.977364in}}%
\pgfpathlineto{\pgfqpoint{1.217885in}{2.965091in}}%
\pgfpathlineto{\pgfqpoint{1.219890in}{2.971909in}}%
\pgfpathlineto{\pgfqpoint{1.220892in}{2.974636in}}%
\pgfpathlineto{\pgfqpoint{1.221894in}{2.982818in}}%
\pgfpathlineto{\pgfqpoint{1.222896in}{2.962364in}}%
\pgfpathlineto{\pgfqpoint{1.224901in}{2.982818in}}%
\pgfpathlineto{\pgfqpoint{1.226905in}{2.974636in}}%
\pgfpathlineto{\pgfqpoint{1.227908in}{2.999182in}}%
\pgfpathlineto{\pgfqpoint{1.228910in}{2.971909in}}%
\pgfpathlineto{\pgfqpoint{1.231917in}{2.993727in}}%
\pgfpathlineto{\pgfqpoint{1.232919in}{2.986909in}}%
\pgfpathlineto{\pgfqpoint{1.233921in}{2.997818in}}%
\pgfpathlineto{\pgfqpoint{1.234923in}{2.991000in}}%
\pgfpathlineto{\pgfqpoint{1.235926in}{3.000545in}}%
\pgfpathlineto{\pgfqpoint{1.237930in}{2.986909in}}%
\pgfpathlineto{\pgfqpoint{1.238932in}{2.986909in}}%
\pgfpathlineto{\pgfqpoint{1.239935in}{2.996455in}}%
\pgfpathlineto{\pgfqpoint{1.240937in}{2.984182in}}%
\pgfpathlineto{\pgfqpoint{1.241939in}{2.991000in}}%
\pgfpathlineto{\pgfqpoint{1.242941in}{2.971909in}}%
\pgfpathlineto{\pgfqpoint{1.244946in}{2.988273in}}%
\pgfpathlineto{\pgfqpoint{1.245948in}{2.986909in}}%
\pgfpathlineto{\pgfqpoint{1.247952in}{2.973273in}}%
\pgfpathlineto{\pgfqpoint{1.248955in}{2.976000in}}%
\pgfpathlineto{\pgfqpoint{1.249957in}{2.973273in}}%
\pgfpathlineto{\pgfqpoint{1.250959in}{2.977364in}}%
\pgfpathlineto{\pgfqpoint{1.252964in}{2.969182in}}%
\pgfpathlineto{\pgfqpoint{1.253966in}{2.985545in}}%
\pgfpathlineto{\pgfqpoint{1.254968in}{2.959636in}}%
\pgfpathlineto{\pgfqpoint{1.256973in}{2.974636in}}%
\pgfpathlineto{\pgfqpoint{1.257975in}{2.971909in}}%
\pgfpathlineto{\pgfqpoint{1.258977in}{2.967818in}}%
\pgfpathlineto{\pgfqpoint{1.260982in}{2.974636in}}%
\pgfpathlineto{\pgfqpoint{1.263988in}{2.991000in}}%
\pgfpathlineto{\pgfqpoint{1.264991in}{2.978727in}}%
\pgfpathlineto{\pgfqpoint{1.266995in}{2.997818in}}%
\pgfpathlineto{\pgfqpoint{1.267997in}{2.992364in}}%
\pgfpathlineto{\pgfqpoint{1.269000in}{2.996455in}}%
\pgfpathlineto{\pgfqpoint{1.271004in}{2.981455in}}%
\pgfpathlineto{\pgfqpoint{1.272006in}{2.993727in}}%
\pgfpathlineto{\pgfqpoint{1.274011in}{2.988273in}}%
\pgfpathlineto{\pgfqpoint{1.275013in}{2.993727in}}%
\pgfpathlineto{\pgfqpoint{1.277018in}{2.986909in}}%
\pgfpathlineto{\pgfqpoint{1.278020in}{2.992364in}}%
\pgfpathlineto{\pgfqpoint{1.279022in}{2.989636in}}%
\pgfpathlineto{\pgfqpoint{1.280024in}{2.993727in}}%
\pgfpathlineto{\pgfqpoint{1.281026in}{2.980091in}}%
\pgfpathlineto{\pgfqpoint{1.282029in}{2.981455in}}%
\pgfpathlineto{\pgfqpoint{1.283031in}{2.978727in}}%
\pgfpathlineto{\pgfqpoint{1.284033in}{2.988273in}}%
\pgfpathlineto{\pgfqpoint{1.285035in}{2.986909in}}%
\pgfpathlineto{\pgfqpoint{1.286038in}{2.988273in}}%
\pgfpathlineto{\pgfqpoint{1.287040in}{2.969182in}}%
\pgfpathlineto{\pgfqpoint{1.288042in}{2.989636in}}%
\pgfpathlineto{\pgfqpoint{1.289044in}{2.974636in}}%
\pgfpathlineto{\pgfqpoint{1.290047in}{2.984182in}}%
\pgfpathlineto{\pgfqpoint{1.291049in}{2.976000in}}%
\pgfpathlineto{\pgfqpoint{1.292051in}{2.977364in}}%
\pgfpathlineto{\pgfqpoint{1.293053in}{2.986909in}}%
\pgfpathlineto{\pgfqpoint{1.294056in}{2.974636in}}%
\pgfpathlineto{\pgfqpoint{1.296060in}{2.981455in}}%
\pgfpathlineto{\pgfqpoint{1.297062in}{2.976000in}}%
\pgfpathlineto{\pgfqpoint{1.298065in}{2.988273in}}%
\pgfpathlineto{\pgfqpoint{1.299067in}{2.978727in}}%
\pgfpathlineto{\pgfqpoint{1.302074in}{2.986909in}}%
\pgfpathlineto{\pgfqpoint{1.303076in}{2.988273in}}%
\pgfpathlineto{\pgfqpoint{1.304078in}{2.996455in}}%
\pgfpathlineto{\pgfqpoint{1.306083in}{2.982818in}}%
\pgfpathlineto{\pgfqpoint{1.308087in}{2.999182in}}%
\pgfpathlineto{\pgfqpoint{1.309089in}{2.992364in}}%
\pgfpathlineto{\pgfqpoint{1.310092in}{2.993727in}}%
\pgfpathlineto{\pgfqpoint{1.311094in}{2.982818in}}%
\pgfpathlineto{\pgfqpoint{1.312096in}{2.985545in}}%
\pgfpathlineto{\pgfqpoint{1.313098in}{2.995091in}}%
\pgfpathlineto{\pgfqpoint{1.314101in}{2.984182in}}%
\pgfpathlineto{\pgfqpoint{1.315103in}{2.985545in}}%
\pgfpathlineto{\pgfqpoint{1.316105in}{2.976000in}}%
\pgfpathlineto{\pgfqpoint{1.317107in}{2.978727in}}%
\pgfpathlineto{\pgfqpoint{1.319112in}{2.976000in}}%
\pgfpathlineto{\pgfqpoint{1.320114in}{2.985545in}}%
\pgfpathlineto{\pgfqpoint{1.322118in}{2.974636in}}%
\pgfpathlineto{\pgfqpoint{1.323121in}{2.980091in}}%
\pgfpathlineto{\pgfqpoint{1.324123in}{2.978727in}}%
\pgfpathlineto{\pgfqpoint{1.325125in}{2.967818in}}%
\pgfpathlineto{\pgfqpoint{1.326127in}{2.984182in}}%
\pgfpathlineto{\pgfqpoint{1.329134in}{2.966455in}}%
\pgfpathlineto{\pgfqpoint{1.332141in}{2.973273in}}%
\pgfpathlineto{\pgfqpoint{1.333143in}{2.982818in}}%
\pgfpathlineto{\pgfqpoint{1.334145in}{2.977364in}}%
\pgfpathlineto{\pgfqpoint{1.336150in}{2.989636in}}%
\pgfpathlineto{\pgfqpoint{1.337152in}{2.984182in}}%
\pgfpathlineto{\pgfqpoint{1.338154in}{2.988273in}}%
\pgfpathlineto{\pgfqpoint{1.339157in}{2.981455in}}%
\pgfpathlineto{\pgfqpoint{1.341161in}{2.991000in}}%
\pgfpathlineto{\pgfqpoint{1.342163in}{2.986909in}}%
\pgfpathlineto{\pgfqpoint{1.344168in}{3.003273in}}%
\pgfpathlineto{\pgfqpoint{1.345170in}{2.991000in}}%
\pgfpathlineto{\pgfqpoint{1.346172in}{2.996455in}}%
\pgfpathlineto{\pgfqpoint{1.347175in}{2.995091in}}%
\pgfpathlineto{\pgfqpoint{1.348177in}{3.001909in}}%
\pgfpathlineto{\pgfqpoint{1.350181in}{2.984182in}}%
\pgfpathlineto{\pgfqpoint{1.351183in}{2.992364in}}%
\pgfpathlineto{\pgfqpoint{1.352186in}{2.978727in}}%
\pgfpathlineto{\pgfqpoint{1.354190in}{2.982818in}}%
\pgfpathlineto{\pgfqpoint{1.356195in}{2.977364in}}%
\pgfpathlineto{\pgfqpoint{1.359201in}{2.971909in}}%
\pgfpathlineto{\pgfqpoint{1.360204in}{2.986909in}}%
\pgfpathlineto{\pgfqpoint{1.362208in}{2.974636in}}%
\pgfpathlineto{\pgfqpoint{1.364213in}{2.971909in}}%
\pgfpathlineto{\pgfqpoint{1.365215in}{2.974636in}}%
\pgfpathlineto{\pgfqpoint{1.366217in}{2.970545in}}%
\pgfpathlineto{\pgfqpoint{1.368222in}{2.980091in}}%
\pgfpathlineto{\pgfqpoint{1.369224in}{2.971909in}}%
\pgfpathlineto{\pgfqpoint{1.370226in}{2.977364in}}%
\pgfpathlineto{\pgfqpoint{1.372231in}{2.970545in}}%
\pgfpathlineto{\pgfqpoint{1.374235in}{2.988273in}}%
\pgfpathlineto{\pgfqpoint{1.375237in}{2.973273in}}%
\pgfpathlineto{\pgfqpoint{1.378244in}{2.995091in}}%
\pgfpathlineto{\pgfqpoint{1.380249in}{2.982818in}}%
\pgfpathlineto{\pgfqpoint{1.384258in}{2.996455in}}%
\pgfpathlineto{\pgfqpoint{1.386262in}{2.984182in}}%
\pgfpathlineto{\pgfqpoint{1.387264in}{2.985545in}}%
\pgfpathlineto{\pgfqpoint{1.389269in}{2.995091in}}%
\pgfpathlineto{\pgfqpoint{1.391273in}{2.980091in}}%
\pgfpathlineto{\pgfqpoint{1.392275in}{2.984182in}}%
\pgfpathlineto{\pgfqpoint{1.393278in}{2.965091in}}%
\pgfpathlineto{\pgfqpoint{1.395282in}{2.982818in}}%
\pgfpathlineto{\pgfqpoint{1.400293in}{2.977364in}}%
\pgfpathlineto{\pgfqpoint{1.401296in}{2.980091in}}%
\pgfpathlineto{\pgfqpoint{1.402298in}{2.973273in}}%
\pgfpathlineto{\pgfqpoint{1.403300in}{2.976000in}}%
\pgfpathlineto{\pgfqpoint{1.405305in}{2.969182in}}%
\pgfpathlineto{\pgfqpoint{1.408311in}{2.974636in}}%
\pgfpathlineto{\pgfqpoint{1.409314in}{2.997818in}}%
\pgfpathlineto{\pgfqpoint{1.410316in}{2.969182in}}%
\pgfpathlineto{\pgfqpoint{1.412320in}{2.988273in}}%
\pgfpathlineto{\pgfqpoint{1.413323in}{2.980091in}}%
\pgfpathlineto{\pgfqpoint{1.414325in}{2.988273in}}%
\pgfpathlineto{\pgfqpoint{1.415327in}{2.978727in}}%
\pgfpathlineto{\pgfqpoint{1.416329in}{2.991000in}}%
\pgfpathlineto{\pgfqpoint{1.417332in}{2.973273in}}%
\pgfpathlineto{\pgfqpoint{1.419336in}{2.989636in}}%
\pgfpathlineto{\pgfqpoint{1.420338in}{2.988273in}}%
\pgfpathlineto{\pgfqpoint{1.421340in}{2.999182in}}%
\pgfpathlineto{\pgfqpoint{1.422343in}{2.986909in}}%
\pgfpathlineto{\pgfqpoint{1.423345in}{2.996455in}}%
\pgfpathlineto{\pgfqpoint{1.425349in}{2.984182in}}%
\pgfpathlineto{\pgfqpoint{1.426352in}{2.991000in}}%
\pgfpathlineto{\pgfqpoint{1.427354in}{2.977364in}}%
\pgfpathlineto{\pgfqpoint{1.428356in}{2.991000in}}%
\pgfpathlineto{\pgfqpoint{1.430361in}{2.977364in}}%
\pgfpathlineto{\pgfqpoint{1.431363in}{2.991000in}}%
\pgfpathlineto{\pgfqpoint{1.433367in}{2.967818in}}%
\pgfpathlineto{\pgfqpoint{1.436374in}{2.982818in}}%
\pgfpathlineto{\pgfqpoint{1.437376in}{2.961000in}}%
\pgfpathlineto{\pgfqpoint{1.438379in}{2.974636in}}%
\pgfpathlineto{\pgfqpoint{1.439381in}{2.971909in}}%
\pgfpathlineto{\pgfqpoint{1.440383in}{2.984182in}}%
\pgfpathlineto{\pgfqpoint{1.442388in}{2.970545in}}%
\pgfpathlineto{\pgfqpoint{1.443390in}{2.966455in}}%
\pgfpathlineto{\pgfqpoint{1.444392in}{2.967818in}}%
\pgfpathlineto{\pgfqpoint{1.445394in}{2.980091in}}%
\pgfpathlineto{\pgfqpoint{1.446397in}{2.978727in}}%
\pgfpathlineto{\pgfqpoint{1.448401in}{2.986909in}}%
\pgfpathlineto{\pgfqpoint{1.449403in}{2.969182in}}%
\pgfpathlineto{\pgfqpoint{1.451408in}{2.984182in}}%
\pgfpathlineto{\pgfqpoint{1.452410in}{2.984182in}}%
\pgfpathlineto{\pgfqpoint{1.453412in}{2.999182in}}%
\pgfpathlineto{\pgfqpoint{1.454415in}{2.982818in}}%
\pgfpathlineto{\pgfqpoint{1.455417in}{2.986909in}}%
\pgfpathlineto{\pgfqpoint{1.456419in}{2.981455in}}%
\pgfpathlineto{\pgfqpoint{1.458423in}{3.000545in}}%
\pgfpathlineto{\pgfqpoint{1.459426in}{2.996455in}}%
\pgfpathlineto{\pgfqpoint{1.461430in}{2.967818in}}%
\pgfpathlineto{\pgfqpoint{1.462432in}{3.000545in}}%
\pgfpathlineto{\pgfqpoint{1.463435in}{2.984182in}}%
\pgfpathlineto{\pgfqpoint{1.464437in}{2.988273in}}%
\pgfpathlineto{\pgfqpoint{1.465439in}{2.986909in}}%
\pgfpathlineto{\pgfqpoint{1.466441in}{2.989636in}}%
\pgfpathlineto{\pgfqpoint{1.467444in}{2.988273in}}%
\pgfpathlineto{\pgfqpoint{1.468446in}{2.973273in}}%
\pgfpathlineto{\pgfqpoint{1.470450in}{2.999182in}}%
\pgfpathlineto{\pgfqpoint{1.471453in}{2.974636in}}%
\pgfpathlineto{\pgfqpoint{1.472455in}{2.988273in}}%
\pgfpathlineto{\pgfqpoint{1.473457in}{2.966455in}}%
\pgfpathlineto{\pgfqpoint{1.475462in}{2.981455in}}%
\pgfpathlineto{\pgfqpoint{1.476464in}{2.980091in}}%
\pgfpathlineto{\pgfqpoint{1.478468in}{2.965091in}}%
\pgfpathlineto{\pgfqpoint{1.480473in}{2.982818in}}%
\pgfpathlineto{\pgfqpoint{1.481475in}{2.974636in}}%
\pgfpathlineto{\pgfqpoint{1.482477in}{2.978727in}}%
\pgfpathlineto{\pgfqpoint{1.483480in}{2.959636in}}%
\pgfpathlineto{\pgfqpoint{1.484482in}{2.963727in}}%
\pgfpathlineto{\pgfqpoint{1.486486in}{2.986909in}}%
\pgfpathlineto{\pgfqpoint{1.488491in}{2.976000in}}%
\pgfpathlineto{\pgfqpoint{1.489493in}{2.986909in}}%
\pgfpathlineto{\pgfqpoint{1.490495in}{2.970545in}}%
\pgfpathlineto{\pgfqpoint{1.492500in}{2.992364in}}%
\pgfpathlineto{\pgfqpoint{1.494504in}{2.981455in}}%
\pgfpathlineto{\pgfqpoint{1.495506in}{2.980091in}}%
\pgfpathlineto{\pgfqpoint{1.497511in}{2.982818in}}%
\pgfpathlineto{\pgfqpoint{1.498513in}{2.996455in}}%
\pgfpathlineto{\pgfqpoint{1.499515in}{2.995091in}}%
\pgfpathlineto{\pgfqpoint{1.500518in}{2.980091in}}%
\pgfpathlineto{\pgfqpoint{1.502522in}{3.003273in}}%
\pgfpathlineto{\pgfqpoint{1.504527in}{2.996455in}}%
\pgfpathlineto{\pgfqpoint{1.505529in}{2.977364in}}%
\pgfpathlineto{\pgfqpoint{1.506531in}{3.001909in}}%
\pgfpathlineto{\pgfqpoint{1.507533in}{2.989636in}}%
\pgfpathlineto{\pgfqpoint{1.508536in}{2.991000in}}%
\pgfpathlineto{\pgfqpoint{1.510540in}{2.985545in}}%
\pgfpathlineto{\pgfqpoint{1.512545in}{2.973273in}}%
\pgfpathlineto{\pgfqpoint{1.513547in}{2.986909in}}%
\pgfpathlineto{\pgfqpoint{1.514549in}{2.985545in}}%
\pgfpathlineto{\pgfqpoint{1.515551in}{2.985545in}}%
\pgfpathlineto{\pgfqpoint{1.516554in}{2.984182in}}%
\pgfpathlineto{\pgfqpoint{1.518558in}{2.974636in}}%
\pgfpathlineto{\pgfqpoint{1.520563in}{2.986909in}}%
\pgfpathlineto{\pgfqpoint{1.521565in}{2.967818in}}%
\pgfpathlineto{\pgfqpoint{1.522567in}{2.973273in}}%
\pgfpathlineto{\pgfqpoint{1.523569in}{2.971909in}}%
\pgfpathlineto{\pgfqpoint{1.526576in}{2.976000in}}%
\pgfpathlineto{\pgfqpoint{1.527578in}{2.958273in}}%
\pgfpathlineto{\pgfqpoint{1.528580in}{2.977364in}}%
\pgfpathlineto{\pgfqpoint{1.529583in}{2.974636in}}%
\pgfpathlineto{\pgfqpoint{1.531587in}{2.986909in}}%
\pgfpathlineto{\pgfqpoint{1.532589in}{2.965091in}}%
\pgfpathlineto{\pgfqpoint{1.534594in}{2.980091in}}%
\pgfpathlineto{\pgfqpoint{1.537601in}{2.992364in}}%
\pgfpathlineto{\pgfqpoint{1.538603in}{3.006000in}}%
\pgfpathlineto{\pgfqpoint{1.540607in}{2.997818in}}%
\pgfpathlineto{\pgfqpoint{1.541610in}{2.996455in}}%
\pgfpathlineto{\pgfqpoint{1.543614in}{3.008727in}}%
\pgfpathlineto{\pgfqpoint{1.544616in}{2.995091in}}%
\pgfpathlineto{\pgfqpoint{1.545619in}{2.999182in}}%
\pgfpathlineto{\pgfqpoint{1.546621in}{2.997818in}}%
\pgfpathlineto{\pgfqpoint{1.548625in}{3.014182in}}%
\pgfpathlineto{\pgfqpoint{1.551632in}{2.977364in}}%
\pgfpathlineto{\pgfqpoint{1.552634in}{3.001909in}}%
\pgfpathlineto{\pgfqpoint{1.553637in}{3.000545in}}%
\pgfpathlineto{\pgfqpoint{1.554639in}{2.982818in}}%
\pgfpathlineto{\pgfqpoint{1.555641in}{2.989636in}}%
\pgfpathlineto{\pgfqpoint{1.557646in}{2.981455in}}%
\pgfpathlineto{\pgfqpoint{1.558648in}{2.985545in}}%
\pgfpathlineto{\pgfqpoint{1.559650in}{2.980091in}}%
\pgfpathlineto{\pgfqpoint{1.560652in}{2.992364in}}%
\pgfpathlineto{\pgfqpoint{1.561655in}{2.971909in}}%
\pgfpathlineto{\pgfqpoint{1.562657in}{2.978727in}}%
\pgfpathlineto{\pgfqpoint{1.564661in}{2.967818in}}%
\pgfpathlineto{\pgfqpoint{1.565663in}{2.982818in}}%
\pgfpathlineto{\pgfqpoint{1.567668in}{2.965091in}}%
\pgfpathlineto{\pgfqpoint{1.568670in}{2.973273in}}%
\pgfpathlineto{\pgfqpoint{1.569672in}{2.959636in}}%
\pgfpathlineto{\pgfqpoint{1.570675in}{2.980091in}}%
\pgfpathlineto{\pgfqpoint{1.571677in}{2.970545in}}%
\pgfpathlineto{\pgfqpoint{1.572679in}{2.981455in}}%
\pgfpathlineto{\pgfqpoint{1.573681in}{2.966455in}}%
\pgfpathlineto{\pgfqpoint{1.575686in}{2.982818in}}%
\pgfpathlineto{\pgfqpoint{1.576688in}{2.971909in}}%
\pgfpathlineto{\pgfqpoint{1.577690in}{2.988273in}}%
\pgfpathlineto{\pgfqpoint{1.579695in}{2.966455in}}%
\pgfpathlineto{\pgfqpoint{1.582702in}{2.999182in}}%
\pgfpathlineto{\pgfqpoint{1.584706in}{2.988273in}}%
\pgfpathlineto{\pgfqpoint{1.586711in}{2.986909in}}%
\pgfpathlineto{\pgfqpoint{1.588715in}{3.012818in}}%
\pgfpathlineto{\pgfqpoint{1.589717in}{3.004636in}}%
\pgfpathlineto{\pgfqpoint{1.591722in}{2.982818in}}%
\pgfpathlineto{\pgfqpoint{1.593726in}{3.012818in}}%
\pgfpathlineto{\pgfqpoint{1.594729in}{3.006000in}}%
\pgfpathlineto{\pgfqpoint{1.595731in}{2.978727in}}%
\pgfpathlineto{\pgfqpoint{1.597735in}{3.007364in}}%
\pgfpathlineto{\pgfqpoint{1.598737in}{3.021000in}}%
\pgfpathlineto{\pgfqpoint{1.601744in}{2.980091in}}%
\pgfpathlineto{\pgfqpoint{1.602746in}{2.991000in}}%
\pgfpathlineto{\pgfqpoint{1.603749in}{2.982818in}}%
\pgfpathlineto{\pgfqpoint{1.605753in}{2.988273in}}%
\pgfpathlineto{\pgfqpoint{1.606755in}{2.986909in}}%
\pgfpathlineto{\pgfqpoint{1.608760in}{2.962364in}}%
\pgfpathlineto{\pgfqpoint{1.609762in}{2.958273in}}%
\pgfpathlineto{\pgfqpoint{1.610764in}{2.981455in}}%
\pgfpathlineto{\pgfqpoint{1.611767in}{2.973273in}}%
\pgfpathlineto{\pgfqpoint{1.613771in}{2.950091in}}%
\pgfpathlineto{\pgfqpoint{1.614773in}{2.969182in}}%
\pgfpathlineto{\pgfqpoint{1.615776in}{2.965091in}}%
\pgfpathlineto{\pgfqpoint{1.616778in}{2.984182in}}%
\pgfpathlineto{\pgfqpoint{1.617780in}{2.950091in}}%
\pgfpathlineto{\pgfqpoint{1.620787in}{2.984182in}}%
\pgfpathlineto{\pgfqpoint{1.621789in}{2.978727in}}%
\pgfpathlineto{\pgfqpoint{1.622791in}{2.988273in}}%
\pgfpathlineto{\pgfqpoint{1.623794in}{2.980091in}}%
\pgfpathlineto{\pgfqpoint{1.624796in}{2.982818in}}%
\pgfpathlineto{\pgfqpoint{1.625798in}{2.980091in}}%
\pgfpathlineto{\pgfqpoint{1.628805in}{3.003273in}}%
\pgfpathlineto{\pgfqpoint{1.629807in}{3.001909in}}%
\pgfpathlineto{\pgfqpoint{1.630809in}{2.991000in}}%
\pgfpathlineto{\pgfqpoint{1.631812in}{2.995091in}}%
\pgfpathlineto{\pgfqpoint{1.632814in}{3.012818in}}%
\pgfpathlineto{\pgfqpoint{1.633816in}{3.010091in}}%
\pgfpathlineto{\pgfqpoint{1.635820in}{2.995091in}}%
\pgfpathlineto{\pgfqpoint{1.638827in}{3.018273in}}%
\pgfpathlineto{\pgfqpoint{1.640832in}{2.993727in}}%
\pgfpathlineto{\pgfqpoint{1.643838in}{3.010091in}}%
\pgfpathlineto{\pgfqpoint{1.644841in}{3.004636in}}%
\pgfpathlineto{\pgfqpoint{1.645843in}{2.981455in}}%
\pgfpathlineto{\pgfqpoint{1.646845in}{2.988273in}}%
\pgfpathlineto{\pgfqpoint{1.647847in}{2.977364in}}%
\pgfpathlineto{\pgfqpoint{1.648850in}{2.996455in}}%
\pgfpathlineto{\pgfqpoint{1.650854in}{2.984182in}}%
\pgfpathlineto{\pgfqpoint{1.651856in}{2.965091in}}%
\pgfpathlineto{\pgfqpoint{1.654863in}{2.984182in}}%
\pgfpathlineto{\pgfqpoint{1.655865in}{2.982818in}}%
\pgfpathlineto{\pgfqpoint{1.656868in}{2.980091in}}%
\pgfpathlineto{\pgfqpoint{1.657870in}{2.958273in}}%
\pgfpathlineto{\pgfqpoint{1.658872in}{2.974636in}}%
\pgfpathlineto{\pgfqpoint{1.659874in}{2.971909in}}%
\pgfpathlineto{\pgfqpoint{1.660877in}{2.977364in}}%
\pgfpathlineto{\pgfqpoint{1.661879in}{2.974636in}}%
\pgfpathlineto{\pgfqpoint{1.663883in}{2.948727in}}%
\pgfpathlineto{\pgfqpoint{1.665888in}{2.978727in}}%
\pgfpathlineto{\pgfqpoint{1.666890in}{2.976000in}}%
\pgfpathlineto{\pgfqpoint{1.667892in}{2.969182in}}%
\pgfpathlineto{\pgfqpoint{1.668895in}{2.978727in}}%
\pgfpathlineto{\pgfqpoint{1.669897in}{2.969182in}}%
\pgfpathlineto{\pgfqpoint{1.670899in}{2.986909in}}%
\pgfpathlineto{\pgfqpoint{1.671901in}{2.982818in}}%
\pgfpathlineto{\pgfqpoint{1.672903in}{2.985545in}}%
\pgfpathlineto{\pgfqpoint{1.673906in}{2.992364in}}%
\pgfpathlineto{\pgfqpoint{1.676912in}{2.977364in}}%
\pgfpathlineto{\pgfqpoint{1.678917in}{3.012818in}}%
\pgfpathlineto{\pgfqpoint{1.680921in}{2.985545in}}%
\pgfpathlineto{\pgfqpoint{1.681924in}{2.992364in}}%
\pgfpathlineto{\pgfqpoint{1.682926in}{3.016909in}}%
\pgfpathlineto{\pgfqpoint{1.683928in}{3.011455in}}%
\pgfpathlineto{\pgfqpoint{1.685933in}{2.991000in}}%
\pgfpathlineto{\pgfqpoint{1.687937in}{3.001909in}}%
\pgfpathlineto{\pgfqpoint{1.688939in}{3.018273in}}%
\pgfpathlineto{\pgfqpoint{1.689942in}{2.996455in}}%
\pgfpathlineto{\pgfqpoint{1.690944in}{2.999182in}}%
\pgfpathlineto{\pgfqpoint{1.691946in}{2.984182in}}%
\pgfpathlineto{\pgfqpoint{1.693951in}{3.006000in}}%
\pgfpathlineto{\pgfqpoint{1.694953in}{2.988273in}}%
\pgfpathlineto{\pgfqpoint{1.695955in}{2.996455in}}%
\pgfpathlineto{\pgfqpoint{1.697960in}{2.973273in}}%
\pgfpathlineto{\pgfqpoint{1.698962in}{2.991000in}}%
\pgfpathlineto{\pgfqpoint{1.699964in}{2.984182in}}%
\pgfpathlineto{\pgfqpoint{1.700966in}{2.986909in}}%
\pgfpathlineto{\pgfqpoint{1.702971in}{2.961000in}}%
\pgfpathlineto{\pgfqpoint{1.703973in}{2.954182in}}%
\pgfpathlineto{\pgfqpoint{1.705977in}{2.977364in}}%
\pgfpathlineto{\pgfqpoint{1.707982in}{2.941909in}}%
\pgfpathlineto{\pgfqpoint{1.710989in}{2.974636in}}%
\pgfpathlineto{\pgfqpoint{1.713995in}{2.946000in}}%
\pgfpathlineto{\pgfqpoint{1.716000in}{2.977364in}}%
\pgfpathlineto{\pgfqpoint{1.717002in}{2.973273in}}%
\pgfpathlineto{\pgfqpoint{1.718004in}{3.003273in}}%
\pgfpathlineto{\pgfqpoint{1.719007in}{2.974636in}}%
\pgfpathlineto{\pgfqpoint{1.720009in}{2.977364in}}%
\pgfpathlineto{\pgfqpoint{1.722013in}{2.991000in}}%
\pgfpathlineto{\pgfqpoint{1.724018in}{3.012818in}}%
\pgfpathlineto{\pgfqpoint{1.726022in}{2.991000in}}%
\pgfpathlineto{\pgfqpoint{1.727025in}{3.004636in}}%
\pgfpathlineto{\pgfqpoint{1.728027in}{3.038727in}}%
\pgfpathlineto{\pgfqpoint{1.729029in}{3.029182in}}%
\pgfpathlineto{\pgfqpoint{1.731034in}{3.000545in}}%
\pgfpathlineto{\pgfqpoint{1.733038in}{3.036000in}}%
\pgfpathlineto{\pgfqpoint{1.734040in}{3.031909in}}%
\pgfpathlineto{\pgfqpoint{1.736045in}{2.988273in}}%
\pgfpathlineto{\pgfqpoint{1.738049in}{3.018273in}}%
\pgfpathlineto{\pgfqpoint{1.741056in}{2.981455in}}%
\pgfpathlineto{\pgfqpoint{1.742058in}{2.981455in}}%
\pgfpathlineto{\pgfqpoint{1.743060in}{2.984182in}}%
\pgfpathlineto{\pgfqpoint{1.744063in}{2.956909in}}%
\pgfpathlineto{\pgfqpoint{1.745065in}{2.980091in}}%
\pgfpathlineto{\pgfqpoint{1.746067in}{2.974636in}}%
\pgfpathlineto{\pgfqpoint{1.749074in}{2.937818in}}%
\pgfpathlineto{\pgfqpoint{1.751078in}{2.981455in}}%
\pgfpathlineto{\pgfqpoint{1.754085in}{2.926909in}}%
\pgfpathlineto{\pgfqpoint{1.756090in}{2.966455in}}%
\pgfpathlineto{\pgfqpoint{1.757092in}{2.982818in}}%
\pgfpathlineto{\pgfqpoint{1.759096in}{2.947364in}}%
\pgfpathlineto{\pgfqpoint{1.761101in}{2.976000in}}%
\pgfpathlineto{\pgfqpoint{1.762103in}{2.991000in}}%
\pgfpathlineto{\pgfqpoint{1.763105in}{2.980091in}}%
\pgfpathlineto{\pgfqpoint{1.765110in}{2.996455in}}%
\pgfpathlineto{\pgfqpoint{1.766112in}{2.996455in}}%
\pgfpathlineto{\pgfqpoint{1.769119in}{3.019636in}}%
\pgfpathlineto{\pgfqpoint{1.771123in}{2.989636in}}%
\pgfpathlineto{\pgfqpoint{1.773128in}{3.049636in}}%
\pgfpathlineto{\pgfqpoint{1.776134in}{2.984182in}}%
\pgfpathlineto{\pgfqpoint{1.777137in}{2.997818in}}%
\pgfpathlineto{\pgfqpoint{1.779141in}{3.025091in}}%
\pgfpathlineto{\pgfqpoint{1.781146in}{2.973273in}}%
\pgfpathlineto{\pgfqpoint{1.784152in}{3.006000in}}%
\pgfpathlineto{\pgfqpoint{1.786157in}{2.970545in}}%
\pgfpathlineto{\pgfqpoint{1.787159in}{2.971909in}}%
\pgfpathlineto{\pgfqpoint{1.788161in}{2.951455in}}%
\pgfpathlineto{\pgfqpoint{1.790166in}{2.977364in}}%
\pgfpathlineto{\pgfqpoint{1.792170in}{2.959636in}}%
\pgfpathlineto{\pgfqpoint{1.794175in}{2.928273in}}%
\pgfpathlineto{\pgfqpoint{1.795177in}{2.970545in}}%
\pgfpathlineto{\pgfqpoint{1.796179in}{2.969182in}}%
\pgfpathlineto{\pgfqpoint{1.798184in}{2.931000in}}%
\pgfpathlineto{\pgfqpoint{1.801191in}{2.969182in}}%
\pgfpathlineto{\pgfqpoint{1.802193in}{2.951455in}}%
\pgfpathlineto{\pgfqpoint{1.804197in}{2.965091in}}%
\pgfpathlineto{\pgfqpoint{1.805200in}{2.961000in}}%
\pgfpathlineto{\pgfqpoint{1.806202in}{2.996455in}}%
\pgfpathlineto{\pgfqpoint{1.807204in}{2.978727in}}%
\pgfpathlineto{\pgfqpoint{1.808206in}{2.980091in}}%
\pgfpathlineto{\pgfqpoint{1.809209in}{2.982818in}}%
\pgfpathlineto{\pgfqpoint{1.810211in}{2.967818in}}%
\pgfpathlineto{\pgfqpoint{1.813217in}{3.010091in}}%
\pgfpathlineto{\pgfqpoint{1.814220in}{3.031909in}}%
\pgfpathlineto{\pgfqpoint{1.815222in}{2.993727in}}%
\pgfpathlineto{\pgfqpoint{1.816224in}{2.999182in}}%
\pgfpathlineto{\pgfqpoint{1.817226in}{2.997818in}}%
\pgfpathlineto{\pgfqpoint{1.819231in}{3.034636in}}%
\pgfpathlineto{\pgfqpoint{1.821235in}{2.992364in}}%
\pgfpathlineto{\pgfqpoint{1.822238in}{2.993727in}}%
\pgfpathlineto{\pgfqpoint{1.823240in}{3.023727in}}%
\pgfpathlineto{\pgfqpoint{1.824242in}{3.015545in}}%
\pgfpathlineto{\pgfqpoint{1.825244in}{2.995091in}}%
\pgfpathlineto{\pgfqpoint{1.826247in}{2.997818in}}%
\pgfpathlineto{\pgfqpoint{1.827249in}{2.961000in}}%
\pgfpathlineto{\pgfqpoint{1.828251in}{3.015545in}}%
\pgfpathlineto{\pgfqpoint{1.830256in}{2.959636in}}%
\pgfpathlineto{\pgfqpoint{1.831258in}{2.991000in}}%
\pgfpathlineto{\pgfqpoint{1.832260in}{2.965091in}}%
\pgfpathlineto{\pgfqpoint{1.833262in}{2.974636in}}%
\pgfpathlineto{\pgfqpoint{1.834265in}{2.943273in}}%
\pgfpathlineto{\pgfqpoint{1.836269in}{2.970545in}}%
\pgfpathlineto{\pgfqpoint{1.838274in}{2.924182in}}%
\pgfpathlineto{\pgfqpoint{1.839276in}{2.921455in}}%
\pgfpathlineto{\pgfqpoint{1.841280in}{2.969182in}}%
\pgfpathlineto{\pgfqpoint{1.843285in}{2.903727in}}%
\pgfpathlineto{\pgfqpoint{1.844287in}{2.913273in}}%
\pgfpathlineto{\pgfqpoint{1.846291in}{2.973273in}}%
\pgfpathlineto{\pgfqpoint{1.848296in}{2.941909in}}%
\pgfpathlineto{\pgfqpoint{1.849298in}{2.936455in}}%
\pgfpathlineto{\pgfqpoint{1.851303in}{2.989636in}}%
\pgfpathlineto{\pgfqpoint{1.852305in}{2.993727in}}%
\pgfpathlineto{\pgfqpoint{1.853307in}{2.991000in}}%
\pgfpathlineto{\pgfqpoint{1.854309in}{2.958273in}}%
\pgfpathlineto{\pgfqpoint{1.858318in}{3.038727in}}%
\pgfpathlineto{\pgfqpoint{1.861325in}{2.962364in}}%
\pgfpathlineto{\pgfqpoint{1.863330in}{3.036000in}}%
\pgfpathlineto{\pgfqpoint{1.864332in}{3.027818in}}%
\pgfpathlineto{\pgfqpoint{1.866336in}{2.985545in}}%
\pgfpathlineto{\pgfqpoint{1.868341in}{3.014182in}}%
\pgfpathlineto{\pgfqpoint{1.869343in}{3.026455in}}%
\pgfpathlineto{\pgfqpoint{1.871348in}{2.977364in}}%
\pgfpathlineto{\pgfqpoint{1.873352in}{3.007364in}}%
\pgfpathlineto{\pgfqpoint{1.874354in}{3.004636in}}%
\pgfpathlineto{\pgfqpoint{1.875357in}{2.995091in}}%
\pgfpathlineto{\pgfqpoint{1.876359in}{2.959636in}}%
\pgfpathlineto{\pgfqpoint{1.877361in}{2.991000in}}%
\pgfpathlineto{\pgfqpoint{1.879366in}{2.954182in}}%
\pgfpathlineto{\pgfqpoint{1.880368in}{2.980091in}}%
\pgfpathlineto{\pgfqpoint{1.881370in}{2.959636in}}%
\pgfpathlineto{\pgfqpoint{1.882372in}{2.967818in}}%
\pgfpathlineto{\pgfqpoint{1.884377in}{2.935091in}}%
\pgfpathlineto{\pgfqpoint{1.886381in}{2.974636in}}%
\pgfpathlineto{\pgfqpoint{1.888386in}{2.913273in}}%
\pgfpathlineto{\pgfqpoint{1.889388in}{2.913273in}}%
\pgfpathlineto{\pgfqpoint{1.891392in}{2.970545in}}%
\pgfpathlineto{\pgfqpoint{1.894399in}{2.905091in}}%
\pgfpathlineto{\pgfqpoint{1.896404in}{2.966455in}}%
\pgfpathlineto{\pgfqpoint{1.897406in}{2.966455in}}%
\pgfpathlineto{\pgfqpoint{1.898408in}{2.914636in}}%
\pgfpathlineto{\pgfqpoint{1.900413in}{2.963727in}}%
\pgfpathlineto{\pgfqpoint{1.901415in}{2.969182in}}%
\pgfpathlineto{\pgfqpoint{1.902417in}{2.992364in}}%
\pgfpathlineto{\pgfqpoint{1.903419in}{2.965091in}}%
\pgfpathlineto{\pgfqpoint{1.905424in}{3.000545in}}%
\pgfpathlineto{\pgfqpoint{1.906426in}{2.995091in}}%
\pgfpathlineto{\pgfqpoint{1.907428in}{3.042818in}}%
\pgfpathlineto{\pgfqpoint{1.908431in}{3.029182in}}%
\pgfpathlineto{\pgfqpoint{1.909433in}{3.034636in}}%
\pgfpathlineto{\pgfqpoint{1.911437in}{2.992364in}}%
\pgfpathlineto{\pgfqpoint{1.913442in}{3.056455in}}%
\pgfpathlineto{\pgfqpoint{1.914444in}{3.053727in}}%
\pgfpathlineto{\pgfqpoint{1.915446in}{3.040091in}}%
\pgfpathlineto{\pgfqpoint{1.916449in}{3.007364in}}%
\pgfpathlineto{\pgfqpoint{1.919455in}{3.061909in}}%
\pgfpathlineto{\pgfqpoint{1.921460in}{2.985545in}}%
\pgfpathlineto{\pgfqpoint{1.922462in}{2.993727in}}%
\pgfpathlineto{\pgfqpoint{1.924466in}{3.021000in}}%
\pgfpathlineto{\pgfqpoint{1.926471in}{2.963727in}}%
\pgfpathlineto{\pgfqpoint{1.927473in}{2.962364in}}%
\pgfpathlineto{\pgfqpoint{1.928475in}{2.937818in}}%
\pgfpathlineto{\pgfqpoint{1.929478in}{2.962364in}}%
\pgfpathlineto{\pgfqpoint{1.930480in}{2.961000in}}%
\pgfpathlineto{\pgfqpoint{1.931482in}{2.952818in}}%
\pgfpathlineto{\pgfqpoint{1.933487in}{2.898273in}}%
\pgfpathlineto{\pgfqpoint{1.934489in}{2.902364in}}%
\pgfpathlineto{\pgfqpoint{1.935491in}{2.951455in}}%
\pgfpathlineto{\pgfqpoint{1.938498in}{2.903727in}}%
\pgfpathlineto{\pgfqpoint{1.939500in}{2.921455in}}%
\pgfpathlineto{\pgfqpoint{1.940502in}{2.976000in}}%
\pgfpathlineto{\pgfqpoint{1.941505in}{2.969182in}}%
\pgfpathlineto{\pgfqpoint{1.943509in}{2.926909in}}%
\pgfpathlineto{\pgfqpoint{1.944511in}{2.932364in}}%
\pgfpathlineto{\pgfqpoint{1.945514in}{2.948727in}}%
\pgfpathlineto{\pgfqpoint{1.946516in}{2.986909in}}%
\pgfpathlineto{\pgfqpoint{1.947518in}{2.982818in}}%
\pgfpathlineto{\pgfqpoint{1.948520in}{3.006000in}}%
\pgfpathlineto{\pgfqpoint{1.949523in}{2.992364in}}%
\pgfpathlineto{\pgfqpoint{1.950525in}{3.003273in}}%
\pgfpathlineto{\pgfqpoint{1.951527in}{2.999182in}}%
\pgfpathlineto{\pgfqpoint{1.953531in}{2.955545in}}%
\pgfpathlineto{\pgfqpoint{1.954534in}{3.007364in}}%
\pgfpathlineto{\pgfqpoint{1.955536in}{3.004636in}}%
\pgfpathlineto{\pgfqpoint{1.956538in}{2.974636in}}%
\pgfpathlineto{\pgfqpoint{1.957540in}{3.025091in}}%
\pgfpathlineto{\pgfqpoint{1.958543in}{2.985545in}}%
\pgfpathlineto{\pgfqpoint{1.960547in}{3.010091in}}%
\pgfpathlineto{\pgfqpoint{1.962552in}{2.950091in}}%
\pgfpathlineto{\pgfqpoint{1.964556in}{3.063273in}}%
\pgfpathlineto{\pgfqpoint{1.966561in}{2.980091in}}%
\pgfpathlineto{\pgfqpoint{1.968565in}{3.026455in}}%
\pgfpathlineto{\pgfqpoint{1.970570in}{2.948727in}}%
\pgfpathlineto{\pgfqpoint{1.971572in}{2.947364in}}%
\pgfpathlineto{\pgfqpoint{1.972574in}{2.973273in}}%
\pgfpathlineto{\pgfqpoint{1.974579in}{2.895545in}}%
\pgfpathlineto{\pgfqpoint{1.976583in}{2.976000in}}%
\pgfpathlineto{\pgfqpoint{1.977585in}{2.955545in}}%
\pgfpathlineto{\pgfqpoint{1.978588in}{2.886000in}}%
\pgfpathlineto{\pgfqpoint{1.980592in}{2.932364in}}%
\pgfpathlineto{\pgfqpoint{1.981594in}{2.993727in}}%
\pgfpathlineto{\pgfqpoint{1.983599in}{2.911909in}}%
\pgfpathlineto{\pgfqpoint{1.984601in}{2.903727in}}%
\pgfpathlineto{\pgfqpoint{1.986606in}{2.961000in}}%
\pgfpathlineto{\pgfqpoint{1.987608in}{2.937818in}}%
\pgfpathlineto{\pgfqpoint{1.988610in}{2.965091in}}%
\pgfpathlineto{\pgfqpoint{1.989612in}{2.922818in}}%
\pgfpathlineto{\pgfqpoint{1.991617in}{2.952818in}}%
\pgfpathlineto{\pgfqpoint{1.992619in}{3.019636in}}%
\pgfpathlineto{\pgfqpoint{1.993621in}{3.006000in}}%
\pgfpathlineto{\pgfqpoint{1.994623in}{2.961000in}}%
\pgfpathlineto{\pgfqpoint{1.995626in}{2.962364in}}%
\pgfpathlineto{\pgfqpoint{1.998632in}{3.064636in}}%
\pgfpathlineto{\pgfqpoint{1.999635in}{2.996455in}}%
\pgfpathlineto{\pgfqpoint{2.000637in}{3.003273in}}%
\pgfpathlineto{\pgfqpoint{2.001639in}{2.997818in}}%
\pgfpathlineto{\pgfqpoint{2.003644in}{3.081000in}}%
\pgfpathlineto{\pgfqpoint{2.004646in}{3.071455in}}%
\pgfpathlineto{\pgfqpoint{2.005648in}{3.044182in}}%
\pgfpathlineto{\pgfqpoint{2.006650in}{2.974636in}}%
\pgfpathlineto{\pgfqpoint{2.008655in}{3.121909in}}%
\pgfpathlineto{\pgfqpoint{2.011662in}{2.963727in}}%
\pgfpathlineto{\pgfqpoint{2.013666in}{3.037364in}}%
\pgfpathlineto{\pgfqpoint{2.016673in}{2.961000in}}%
\pgfpathlineto{\pgfqpoint{2.017675in}{2.982818in}}%
\pgfpathlineto{\pgfqpoint{2.019680in}{2.917364in}}%
\pgfpathlineto{\pgfqpoint{2.020682in}{2.946000in}}%
\pgfpathlineto{\pgfqpoint{2.021684in}{2.943273in}}%
\pgfpathlineto{\pgfqpoint{2.024691in}{2.883273in}}%
\pgfpathlineto{\pgfqpoint{2.026695in}{2.943273in}}%
\pgfpathlineto{\pgfqpoint{2.028700in}{2.892818in}}%
\pgfpathlineto{\pgfqpoint{2.029702in}{2.899636in}}%
\pgfpathlineto{\pgfqpoint{2.031706in}{2.958273in}}%
\pgfpathlineto{\pgfqpoint{2.033711in}{2.906455in}}%
\pgfpathlineto{\pgfqpoint{2.036718in}{2.984182in}}%
\pgfpathlineto{\pgfqpoint{2.037720in}{2.981455in}}%
\pgfpathlineto{\pgfqpoint{2.038722in}{2.903727in}}%
\pgfpathlineto{\pgfqpoint{2.040727in}{2.944636in}}%
\pgfpathlineto{\pgfqpoint{2.041729in}{2.977364in}}%
\pgfpathlineto{\pgfqpoint{2.042731in}{3.057818in}}%
\pgfpathlineto{\pgfqpoint{2.043733in}{3.046909in}}%
\pgfpathlineto{\pgfqpoint{2.045738in}{3.007364in}}%
\pgfpathlineto{\pgfqpoint{2.046740in}{2.991000in}}%
\pgfpathlineto{\pgfqpoint{2.048745in}{3.102818in}}%
\pgfpathlineto{\pgfqpoint{2.051751in}{3.018273in}}%
\pgfpathlineto{\pgfqpoint{2.053756in}{3.078273in}}%
\pgfpathlineto{\pgfqpoint{2.054758in}{3.098727in}}%
\pgfpathlineto{\pgfqpoint{2.056763in}{2.943273in}}%
\pgfpathlineto{\pgfqpoint{2.057765in}{2.995091in}}%
\pgfpathlineto{\pgfqpoint{2.058767in}{3.117818in}}%
\pgfpathlineto{\pgfqpoint{2.059769in}{3.113727in}}%
\pgfpathlineto{\pgfqpoint{2.061774in}{2.997818in}}%
\pgfpathlineto{\pgfqpoint{2.062776in}{2.941909in}}%
\pgfpathlineto{\pgfqpoint{2.063778in}{2.950091in}}%
\pgfpathlineto{\pgfqpoint{2.064780in}{3.004636in}}%
\pgfpathlineto{\pgfqpoint{2.067787in}{2.941909in}}%
\pgfpathlineto{\pgfqpoint{2.068789in}{2.865545in}}%
\pgfpathlineto{\pgfqpoint{2.070794in}{2.936455in}}%
\pgfpathlineto{\pgfqpoint{2.071796in}{2.937818in}}%
\pgfpathlineto{\pgfqpoint{2.073801in}{2.854636in}}%
\pgfpathlineto{\pgfqpoint{2.074803in}{2.861455in}}%
\pgfpathlineto{\pgfqpoint{2.076807in}{2.939182in}}%
\pgfpathlineto{\pgfqpoint{2.078812in}{2.836909in}}%
\pgfpathlineto{\pgfqpoint{2.081819in}{2.936455in}}%
\pgfpathlineto{\pgfqpoint{2.083823in}{2.911909in}}%
\pgfpathlineto{\pgfqpoint{2.084825in}{2.913273in}}%
\pgfpathlineto{\pgfqpoint{2.086830in}{2.974636in}}%
\pgfpathlineto{\pgfqpoint{2.087832in}{2.944636in}}%
\pgfpathlineto{\pgfqpoint{2.091841in}{3.027818in}}%
\pgfpathlineto{\pgfqpoint{2.092843in}{3.018273in}}%
\pgfpathlineto{\pgfqpoint{2.093846in}{3.022364in}}%
\pgfpathlineto{\pgfqpoint{2.094848in}{3.090545in}}%
\pgfpathlineto{\pgfqpoint{2.096852in}{3.023727in}}%
\pgfpathlineto{\pgfqpoint{2.099859in}{3.181909in}}%
\pgfpathlineto{\pgfqpoint{2.100861in}{3.022364in}}%
\pgfpathlineto{\pgfqpoint{2.101863in}{3.023727in}}%
\pgfpathlineto{\pgfqpoint{2.103868in}{3.124636in}}%
\pgfpathlineto{\pgfqpoint{2.104870in}{3.115091in}}%
\pgfpathlineto{\pgfqpoint{2.106875in}{3.023727in}}%
\pgfpathlineto{\pgfqpoint{2.107877in}{3.037364in}}%
\pgfpathlineto{\pgfqpoint{2.109881in}{2.980091in}}%
\pgfpathlineto{\pgfqpoint{2.110884in}{2.988273in}}%
\pgfpathlineto{\pgfqpoint{2.111886in}{2.999182in}}%
\pgfpathlineto{\pgfqpoint{2.112888in}{2.939182in}}%
\pgfpathlineto{\pgfqpoint{2.113890in}{2.963727in}}%
\pgfpathlineto{\pgfqpoint{2.115895in}{2.947364in}}%
\pgfpathlineto{\pgfqpoint{2.116897in}{2.941909in}}%
\pgfpathlineto{\pgfqpoint{2.118902in}{2.864182in}}%
\pgfpathlineto{\pgfqpoint{2.120906in}{2.925545in}}%
\pgfpathlineto{\pgfqpoint{2.121908in}{2.925545in}}%
\pgfpathlineto{\pgfqpoint{2.123913in}{2.850545in}}%
\pgfpathlineto{\pgfqpoint{2.124915in}{2.862818in}}%
\pgfpathlineto{\pgfqpoint{2.126920in}{2.921455in}}%
\pgfpathlineto{\pgfqpoint{2.128924in}{2.868273in}}%
\pgfpathlineto{\pgfqpoint{2.130928in}{2.962364in}}%
\pgfpathlineto{\pgfqpoint{2.131931in}{2.959636in}}%
\pgfpathlineto{\pgfqpoint{2.132933in}{2.935091in}}%
\pgfpathlineto{\pgfqpoint{2.134937in}{2.961000in}}%
\pgfpathlineto{\pgfqpoint{2.136942in}{3.018273in}}%
\pgfpathlineto{\pgfqpoint{2.137944in}{3.014182in}}%
\pgfpathlineto{\pgfqpoint{2.138946in}{3.072818in}}%
\pgfpathlineto{\pgfqpoint{2.139949in}{3.063273in}}%
\pgfpathlineto{\pgfqpoint{2.140951in}{3.040091in}}%
\pgfpathlineto{\pgfqpoint{2.142955in}{3.094636in}}%
\pgfpathlineto{\pgfqpoint{2.143958in}{3.304636in}}%
\pgfpathlineto{\pgfqpoint{2.145962in}{3.052364in}}%
\pgfpathlineto{\pgfqpoint{2.146964in}{3.037364in}}%
\pgfpathlineto{\pgfqpoint{2.148969in}{3.258273in}}%
\pgfpathlineto{\pgfqpoint{2.150973in}{3.025091in}}%
\pgfpathlineto{\pgfqpoint{2.151976in}{3.014182in}}%
\pgfpathlineto{\pgfqpoint{2.153980in}{3.079636in}}%
\pgfpathlineto{\pgfqpoint{2.158991in}{2.946000in}}%
\pgfpathlineto{\pgfqpoint{2.159994in}{2.944636in}}%
\pgfpathlineto{\pgfqpoint{2.160996in}{2.958273in}}%
\pgfpathlineto{\pgfqpoint{2.161998in}{2.937818in}}%
\pgfpathlineto{\pgfqpoint{2.164003in}{2.872364in}}%
\pgfpathlineto{\pgfqpoint{2.165005in}{2.880545in}}%
\pgfpathlineto{\pgfqpoint{2.166007in}{2.931000in}}%
\pgfpathlineto{\pgfqpoint{2.167009in}{2.916000in}}%
\pgfpathlineto{\pgfqpoint{2.169014in}{2.830091in}}%
\pgfpathlineto{\pgfqpoint{2.170016in}{2.846455in}}%
\pgfpathlineto{\pgfqpoint{2.172020in}{2.921455in}}%
\pgfpathlineto{\pgfqpoint{2.174025in}{2.851909in}}%
\pgfpathlineto{\pgfqpoint{2.175027in}{2.858727in}}%
\pgfpathlineto{\pgfqpoint{2.177032in}{2.946000in}}%
\pgfpathlineto{\pgfqpoint{2.179036in}{2.896909in}}%
\pgfpathlineto{\pgfqpoint{2.181041in}{2.989636in}}%
\pgfpathlineto{\pgfqpoint{2.182043in}{2.991000in}}%
\pgfpathlineto{\pgfqpoint{2.183045in}{3.004636in}}%
\pgfpathlineto{\pgfqpoint{2.184047in}{2.982818in}}%
\pgfpathlineto{\pgfqpoint{2.186052in}{3.025091in}}%
\pgfpathlineto{\pgfqpoint{2.187054in}{3.019636in}}%
\pgfpathlineto{\pgfqpoint{2.189059in}{3.119182in}}%
\pgfpathlineto{\pgfqpoint{2.192065in}{3.052364in}}%
\pgfpathlineto{\pgfqpoint{2.194070in}{3.382364in}}%
\pgfpathlineto{\pgfqpoint{2.197077in}{3.063273in}}%
\pgfpathlineto{\pgfqpoint{2.199081in}{3.255545in}}%
\pgfpathlineto{\pgfqpoint{2.201085in}{3.044182in}}%
\pgfpathlineto{\pgfqpoint{2.202088in}{3.052364in}}%
\pgfpathlineto{\pgfqpoint{2.203090in}{3.085091in}}%
\pgfpathlineto{\pgfqpoint{2.207099in}{2.963727in}}%
\pgfpathlineto{\pgfqpoint{2.208101in}{2.993727in}}%
\pgfpathlineto{\pgfqpoint{2.209103in}{2.962364in}}%
\pgfpathlineto{\pgfqpoint{2.210106in}{2.976000in}}%
\pgfpathlineto{\pgfqpoint{2.211108in}{2.952818in}}%
\pgfpathlineto{\pgfqpoint{2.214115in}{2.832818in}}%
\pgfpathlineto{\pgfqpoint{2.216119in}{2.922818in}}%
\pgfpathlineto{\pgfqpoint{2.218124in}{2.847818in}}%
\pgfpathlineto{\pgfqpoint{2.219126in}{2.790545in}}%
\pgfpathlineto{\pgfqpoint{2.221130in}{2.907818in}}%
\pgfpathlineto{\pgfqpoint{2.222133in}{2.920091in}}%
\pgfpathlineto{\pgfqpoint{2.224137in}{2.845091in}}%
\pgfpathlineto{\pgfqpoint{2.227144in}{2.951455in}}%
\pgfpathlineto{\pgfqpoint{2.228146in}{2.890091in}}%
\pgfpathlineto{\pgfqpoint{2.229148in}{2.894182in}}%
\pgfpathlineto{\pgfqpoint{2.232155in}{3.003273in}}%
\pgfpathlineto{\pgfqpoint{2.233157in}{3.001909in}}%
\pgfpathlineto{\pgfqpoint{2.234160in}{3.006000in}}%
\pgfpathlineto{\pgfqpoint{2.235162in}{3.034636in}}%
\pgfpathlineto{\pgfqpoint{2.236164in}{3.010091in}}%
\pgfpathlineto{\pgfqpoint{2.237166in}{3.037364in}}%
\pgfpathlineto{\pgfqpoint{2.240173in}{3.251455in}}%
\pgfpathlineto{\pgfqpoint{2.241175in}{3.067364in}}%
\pgfpathlineto{\pgfqpoint{2.242177in}{3.098727in}}%
\pgfpathlineto{\pgfqpoint{2.244182in}{3.461455in}}%
\pgfpathlineto{\pgfqpoint{2.245184in}{3.420545in}}%
\pgfpathlineto{\pgfqpoint{2.247189in}{3.089182in}}%
\pgfpathlineto{\pgfqpoint{2.249193in}{3.524182in}}%
\pgfpathlineto{\pgfqpoint{2.251198in}{3.031909in}}%
\pgfpathlineto{\pgfqpoint{2.253202in}{3.025091in}}%
\pgfpathlineto{\pgfqpoint{2.254204in}{3.115091in}}%
\pgfpathlineto{\pgfqpoint{2.259216in}{2.907818in}}%
\pgfpathlineto{\pgfqpoint{2.261220in}{2.917364in}}%
\pgfpathlineto{\pgfqpoint{2.262222in}{2.947364in}}%
\pgfpathlineto{\pgfqpoint{2.264227in}{2.835545in}}%
\pgfpathlineto{\pgfqpoint{2.267234in}{2.895545in}}%
\pgfpathlineto{\pgfqpoint{2.269238in}{2.813727in}}%
\pgfpathlineto{\pgfqpoint{2.271242in}{2.901000in}}%
\pgfpathlineto{\pgfqpoint{2.272245in}{2.917364in}}%
\pgfpathlineto{\pgfqpoint{2.274249in}{2.836909in}}%
\pgfpathlineto{\pgfqpoint{2.276254in}{2.921455in}}%
\pgfpathlineto{\pgfqpoint{2.277256in}{2.922818in}}%
\pgfpathlineto{\pgfqpoint{2.279260in}{2.898273in}}%
\pgfpathlineto{\pgfqpoint{2.280263in}{2.913273in}}%
\pgfpathlineto{\pgfqpoint{2.282267in}{2.991000in}}%
\pgfpathlineto{\pgfqpoint{2.283269in}{2.966455in}}%
\pgfpathlineto{\pgfqpoint{2.284272in}{3.051000in}}%
\pgfpathlineto{\pgfqpoint{2.286276in}{3.012818in}}%
\pgfpathlineto{\pgfqpoint{2.287278in}{3.044182in}}%
\pgfpathlineto{\pgfqpoint{2.288281in}{3.034636in}}%
\pgfpathlineto{\pgfqpoint{2.289283in}{3.177818in}}%
\pgfpathlineto{\pgfqpoint{2.291287in}{3.070091in}}%
\pgfpathlineto{\pgfqpoint{2.292290in}{3.090545in}}%
\pgfpathlineto{\pgfqpoint{2.293292in}{3.192818in}}%
\pgfpathlineto{\pgfqpoint{2.294294in}{3.503727in}}%
\pgfpathlineto{\pgfqpoint{2.296299in}{3.086455in}}%
\pgfpathlineto{\pgfqpoint{2.297301in}{3.127364in}}%
\pgfpathlineto{\pgfqpoint{2.299305in}{3.529636in}}%
\pgfpathlineto{\pgfqpoint{2.300308in}{3.405545in}}%
\pgfpathlineto{\pgfqpoint{2.302312in}{3.078273in}}%
\pgfpathlineto{\pgfqpoint{2.304317in}{3.173727in}}%
\pgfpathlineto{\pgfqpoint{2.307323in}{2.978727in}}%
\pgfpathlineto{\pgfqpoint{2.308325in}{3.031909in}}%
\pgfpathlineto{\pgfqpoint{2.309328in}{3.021000in}}%
\pgfpathlineto{\pgfqpoint{2.312334in}{2.914636in}}%
\pgfpathlineto{\pgfqpoint{2.313337in}{2.898273in}}%
\pgfpathlineto{\pgfqpoint{2.314339in}{2.924182in}}%
\pgfpathlineto{\pgfqpoint{2.315341in}{2.888727in}}%
\pgfpathlineto{\pgfqpoint{2.316343in}{2.935091in}}%
\pgfpathlineto{\pgfqpoint{2.319350in}{2.816455in}}%
\pgfpathlineto{\pgfqpoint{2.321355in}{2.907818in}}%
\pgfpathlineto{\pgfqpoint{2.322357in}{2.888727in}}%
\pgfpathlineto{\pgfqpoint{2.324361in}{2.820545in}}%
\pgfpathlineto{\pgfqpoint{2.325364in}{2.857364in}}%
\pgfpathlineto{\pgfqpoint{2.326366in}{2.937818in}}%
\pgfpathlineto{\pgfqpoint{2.327368in}{2.914636in}}%
\pgfpathlineto{\pgfqpoint{2.329373in}{2.839636in}}%
\pgfpathlineto{\pgfqpoint{2.331377in}{2.991000in}}%
\pgfpathlineto{\pgfqpoint{2.334384in}{2.905091in}}%
\pgfpathlineto{\pgfqpoint{2.337391in}{3.021000in}}%
\pgfpathlineto{\pgfqpoint{2.338393in}{3.040091in}}%
\pgfpathlineto{\pgfqpoint{2.339395in}{3.022364in}}%
\pgfpathlineto{\pgfqpoint{2.340397in}{3.031909in}}%
\pgfpathlineto{\pgfqpoint{2.341400in}{3.064636in}}%
\pgfpathlineto{\pgfqpoint{2.342402in}{3.055091in}}%
\pgfpathlineto{\pgfqpoint{2.344406in}{3.472364in}}%
\pgfpathlineto{\pgfqpoint{2.346411in}{3.168273in}}%
\pgfpathlineto{\pgfqpoint{2.347413in}{3.184636in}}%
\pgfpathlineto{\pgfqpoint{2.349417in}{3.655091in}}%
\pgfpathlineto{\pgfqpoint{2.352424in}{3.154636in}}%
\pgfpathlineto{\pgfqpoint{2.354429in}{3.601909in}}%
\pgfpathlineto{\pgfqpoint{2.356433in}{3.049636in}}%
\pgfpathlineto{\pgfqpoint{2.357435in}{3.025091in}}%
\pgfpathlineto{\pgfqpoint{2.359440in}{3.104182in}}%
\pgfpathlineto{\pgfqpoint{2.361444in}{2.991000in}}%
\pgfpathlineto{\pgfqpoint{2.365453in}{2.940545in}}%
\pgfpathlineto{\pgfqpoint{2.366456in}{2.941909in}}%
\pgfpathlineto{\pgfqpoint{2.369462in}{2.830091in}}%
\pgfpathlineto{\pgfqpoint{2.371467in}{2.906455in}}%
\pgfpathlineto{\pgfqpoint{2.373471in}{2.800091in}}%
\pgfpathlineto{\pgfqpoint{2.374474in}{2.798727in}}%
\pgfpathlineto{\pgfqpoint{2.375476in}{2.813727in}}%
\pgfpathlineto{\pgfqpoint{2.376478in}{2.890091in}}%
\pgfpathlineto{\pgfqpoint{2.378482in}{2.791909in}}%
\pgfpathlineto{\pgfqpoint{2.379485in}{2.796000in}}%
\pgfpathlineto{\pgfqpoint{2.381489in}{2.921455in}}%
\pgfpathlineto{\pgfqpoint{2.382491in}{2.886000in}}%
\pgfpathlineto{\pgfqpoint{2.383494in}{2.820545in}}%
\pgfpathlineto{\pgfqpoint{2.384496in}{2.847818in}}%
\pgfpathlineto{\pgfqpoint{2.386500in}{2.959636in}}%
\pgfpathlineto{\pgfqpoint{2.387503in}{2.969182in}}%
\pgfpathlineto{\pgfqpoint{2.389507in}{2.916000in}}%
\pgfpathlineto{\pgfqpoint{2.391512in}{3.045545in}}%
\pgfpathlineto{\pgfqpoint{2.392514in}{3.025091in}}%
\pgfpathlineto{\pgfqpoint{2.393516in}{3.131455in}}%
\pgfpathlineto{\pgfqpoint{2.395521in}{3.089182in}}%
\pgfpathlineto{\pgfqpoint{2.396523in}{3.072818in}}%
\pgfpathlineto{\pgfqpoint{2.397525in}{3.134182in}}%
\pgfpathlineto{\pgfqpoint{2.398527in}{3.511909in}}%
\pgfpathlineto{\pgfqpoint{2.399530in}{3.465545in}}%
\pgfpathlineto{\pgfqpoint{2.401534in}{3.085091in}}%
\pgfpathlineto{\pgfqpoint{2.402536in}{3.139636in}}%
\pgfpathlineto{\pgfqpoint{2.404541in}{3.653727in}}%
\pgfpathlineto{\pgfqpoint{2.405543in}{3.510545in}}%
\pgfpathlineto{\pgfqpoint{2.406545in}{3.128727in}}%
\pgfpathlineto{\pgfqpoint{2.407548in}{3.156000in}}%
\pgfpathlineto{\pgfqpoint{2.409552in}{3.642818in}}%
\pgfpathlineto{\pgfqpoint{2.411557in}{3.068727in}}%
\pgfpathlineto{\pgfqpoint{2.412559in}{3.059182in}}%
\pgfpathlineto{\pgfqpoint{2.413561in}{3.016909in}}%
\pgfpathlineto{\pgfqpoint{2.414563in}{3.031909in}}%
\pgfpathlineto{\pgfqpoint{2.417570in}{2.971909in}}%
\pgfpathlineto{\pgfqpoint{2.419574in}{2.839636in}}%
\pgfpathlineto{\pgfqpoint{2.420577in}{2.901000in}}%
\pgfpathlineto{\pgfqpoint{2.421579in}{2.895545in}}%
\pgfpathlineto{\pgfqpoint{2.422581in}{2.890091in}}%
\pgfpathlineto{\pgfqpoint{2.424586in}{2.779636in}}%
\pgfpathlineto{\pgfqpoint{2.426590in}{2.879182in}}%
\pgfpathlineto{\pgfqpoint{2.427592in}{2.860091in}}%
\pgfpathlineto{\pgfqpoint{2.429597in}{2.731909in}}%
\pgfpathlineto{\pgfqpoint{2.431601in}{2.877818in}}%
\pgfpathlineto{\pgfqpoint{2.432604in}{2.857364in}}%
\pgfpathlineto{\pgfqpoint{2.434608in}{2.748273in}}%
\pgfpathlineto{\pgfqpoint{2.436613in}{2.899636in}}%
\pgfpathlineto{\pgfqpoint{2.437615in}{2.888727in}}%
\pgfpathlineto{\pgfqpoint{2.438617in}{2.839636in}}%
\pgfpathlineto{\pgfqpoint{2.441624in}{2.971909in}}%
\pgfpathlineto{\pgfqpoint{2.442626in}{2.970545in}}%
\pgfpathlineto{\pgfqpoint{2.443628in}{2.932364in}}%
\pgfpathlineto{\pgfqpoint{2.449642in}{3.091909in}}%
\pgfpathlineto{\pgfqpoint{2.450644in}{3.108273in}}%
\pgfpathlineto{\pgfqpoint{2.451646in}{3.056455in}}%
\pgfpathlineto{\pgfqpoint{2.454653in}{3.555545in}}%
\pgfpathlineto{\pgfqpoint{2.456657in}{3.106909in}}%
\pgfpathlineto{\pgfqpoint{2.457660in}{3.247364in}}%
\pgfpathlineto{\pgfqpoint{2.459664in}{3.653727in}}%
\pgfpathlineto{\pgfqpoint{2.460666in}{3.543273in}}%
\pgfpathlineto{\pgfqpoint{2.461669in}{3.091909in}}%
\pgfpathlineto{\pgfqpoint{2.462671in}{3.177818in}}%
\pgfpathlineto{\pgfqpoint{2.464675in}{3.595091in}}%
\pgfpathlineto{\pgfqpoint{2.466680in}{3.055091in}}%
\pgfpathlineto{\pgfqpoint{2.468684in}{3.270545in}}%
\pgfpathlineto{\pgfqpoint{2.469687in}{3.195545in}}%
\pgfpathlineto{\pgfqpoint{2.471691in}{2.950091in}}%
\pgfpathlineto{\pgfqpoint{2.473696in}{3.018273in}}%
\pgfpathlineto{\pgfqpoint{2.474698in}{3.004636in}}%
\pgfpathlineto{\pgfqpoint{2.476702in}{2.905091in}}%
\pgfpathlineto{\pgfqpoint{2.478707in}{2.856000in}}%
\pgfpathlineto{\pgfqpoint{2.479709in}{2.881909in}}%
\pgfpathlineto{\pgfqpoint{2.481714in}{2.865545in}}%
\pgfpathlineto{\pgfqpoint{2.482716in}{2.836909in}}%
\pgfpathlineto{\pgfqpoint{2.483718in}{2.748273in}}%
\pgfpathlineto{\pgfqpoint{2.484720in}{2.753727in}}%
\pgfpathlineto{\pgfqpoint{2.486725in}{2.857364in}}%
\pgfpathlineto{\pgfqpoint{2.487727in}{2.820545in}}%
\pgfpathlineto{\pgfqpoint{2.489731in}{2.718273in}}%
\pgfpathlineto{\pgfqpoint{2.491736in}{2.876455in}}%
\pgfpathlineto{\pgfqpoint{2.493740in}{2.752364in}}%
\pgfpathlineto{\pgfqpoint{2.494743in}{2.753727in}}%
\pgfpathlineto{\pgfqpoint{2.496747in}{2.937818in}}%
\pgfpathlineto{\pgfqpoint{2.498752in}{2.849182in}}%
\pgfpathlineto{\pgfqpoint{2.502761in}{3.051000in}}%
\pgfpathlineto{\pgfqpoint{2.503763in}{3.071455in}}%
\pgfpathlineto{\pgfqpoint{2.505767in}{2.969182in}}%
\pgfpathlineto{\pgfqpoint{2.506770in}{2.985545in}}%
\pgfpathlineto{\pgfqpoint{2.507772in}{3.094636in}}%
\pgfpathlineto{\pgfqpoint{2.509776in}{3.667364in}}%
\pgfpathlineto{\pgfqpoint{2.511781in}{3.121909in}}%
\pgfpathlineto{\pgfqpoint{2.513785in}{3.614182in}}%
\pgfpathlineto{\pgfqpoint{2.514788in}{3.656455in}}%
\pgfpathlineto{\pgfqpoint{2.516792in}{3.104182in}}%
\pgfpathlineto{\pgfqpoint{2.518797in}{3.682364in}}%
\pgfpathlineto{\pgfqpoint{2.519799in}{3.749182in}}%
\pgfpathlineto{\pgfqpoint{2.521803in}{3.156000in}}%
\pgfpathlineto{\pgfqpoint{2.522805in}{3.289636in}}%
\pgfpathlineto{\pgfqpoint{2.524810in}{3.636000in}}%
\pgfpathlineto{\pgfqpoint{2.526814in}{3.019636in}}%
\pgfpathlineto{\pgfqpoint{2.528819in}{3.325091in}}%
\pgfpathlineto{\pgfqpoint{2.531826in}{2.931000in}}%
\pgfpathlineto{\pgfqpoint{2.533830in}{3.026455in}}%
\pgfpathlineto{\pgfqpoint{2.534832in}{3.016909in}}%
\pgfpathlineto{\pgfqpoint{2.536837in}{2.896909in}}%
\pgfpathlineto{\pgfqpoint{2.538841in}{2.841000in}}%
\pgfpathlineto{\pgfqpoint{2.539844in}{2.876455in}}%
\pgfpathlineto{\pgfqpoint{2.540846in}{2.839636in}}%
\pgfpathlineto{\pgfqpoint{2.541848in}{2.886000in}}%
\pgfpathlineto{\pgfqpoint{2.542850in}{2.846455in}}%
\pgfpathlineto{\pgfqpoint{2.543853in}{2.712818in}}%
\pgfpathlineto{\pgfqpoint{2.546859in}{2.886000in}}%
\pgfpathlineto{\pgfqpoint{2.548864in}{2.730545in}}%
\pgfpathlineto{\pgfqpoint{2.549866in}{2.744182in}}%
\pgfpathlineto{\pgfqpoint{2.551871in}{2.888727in}}%
\pgfpathlineto{\pgfqpoint{2.552873in}{2.865545in}}%
\pgfpathlineto{\pgfqpoint{2.553875in}{2.736000in}}%
\pgfpathlineto{\pgfqpoint{2.554877in}{2.753727in}}%
\pgfpathlineto{\pgfqpoint{2.556882in}{2.940545in}}%
\pgfpathlineto{\pgfqpoint{2.559888in}{2.850545in}}%
\pgfpathlineto{\pgfqpoint{2.561893in}{2.969182in}}%
\pgfpathlineto{\pgfqpoint{2.562895in}{2.982818in}}%
\pgfpathlineto{\pgfqpoint{2.563897in}{2.973273in}}%
\pgfpathlineto{\pgfqpoint{2.566904in}{3.067364in}}%
\pgfpathlineto{\pgfqpoint{2.567906in}{3.044182in}}%
\pgfpathlineto{\pgfqpoint{2.568909in}{3.091909in}}%
\pgfpathlineto{\pgfqpoint{2.569911in}{3.514636in}}%
\pgfpathlineto{\pgfqpoint{2.571915in}{3.166909in}}%
\pgfpathlineto{\pgfqpoint{2.572918in}{3.258273in}}%
\pgfpathlineto{\pgfqpoint{2.574922in}{3.685091in}}%
\pgfpathlineto{\pgfqpoint{2.576927in}{3.266455in}}%
\pgfpathlineto{\pgfqpoint{2.578931in}{3.745091in}}%
\pgfpathlineto{\pgfqpoint{2.579933in}{3.768273in}}%
\pgfpathlineto{\pgfqpoint{2.580936in}{3.641455in}}%
\pgfpathlineto{\pgfqpoint{2.581938in}{3.344182in}}%
\pgfpathlineto{\pgfqpoint{2.583942in}{3.683727in}}%
\pgfpathlineto{\pgfqpoint{2.584945in}{3.734182in}}%
\pgfpathlineto{\pgfqpoint{2.586949in}{3.066000in}}%
\pgfpathlineto{\pgfqpoint{2.587951in}{3.162818in}}%
\pgfpathlineto{\pgfqpoint{2.588954in}{3.612818in}}%
\pgfpathlineto{\pgfqpoint{2.591960in}{2.986909in}}%
\pgfpathlineto{\pgfqpoint{2.592962in}{3.000545in}}%
\pgfpathlineto{\pgfqpoint{2.593965in}{3.102818in}}%
\pgfpathlineto{\pgfqpoint{2.594967in}{3.070091in}}%
\pgfpathlineto{\pgfqpoint{2.596971in}{2.902364in}}%
\pgfpathlineto{\pgfqpoint{2.597974in}{2.851909in}}%
\pgfpathlineto{\pgfqpoint{2.598976in}{2.890091in}}%
\pgfpathlineto{\pgfqpoint{2.600980in}{2.836909in}}%
\pgfpathlineto{\pgfqpoint{2.601983in}{2.871000in}}%
\pgfpathlineto{\pgfqpoint{2.603987in}{2.716909in}}%
\pgfpathlineto{\pgfqpoint{2.604989in}{2.745545in}}%
\pgfpathlineto{\pgfqpoint{2.606994in}{2.877818in}}%
\pgfpathlineto{\pgfqpoint{2.608998in}{2.708727in}}%
\pgfpathlineto{\pgfqpoint{2.610001in}{2.725091in}}%
\pgfpathlineto{\pgfqpoint{2.612005in}{2.876455in}}%
\pgfpathlineto{\pgfqpoint{2.614010in}{2.755091in}}%
\pgfpathlineto{\pgfqpoint{2.615012in}{2.783727in}}%
\pgfpathlineto{\pgfqpoint{2.617016in}{2.926909in}}%
\pgfpathlineto{\pgfqpoint{2.618019in}{2.913273in}}%
\pgfpathlineto{\pgfqpoint{2.619021in}{2.931000in}}%
\pgfpathlineto{\pgfqpoint{2.620023in}{2.884636in}}%
\pgfpathlineto{\pgfqpoint{2.625034in}{3.145091in}}%
\pgfpathlineto{\pgfqpoint{2.626036in}{3.098727in}}%
\pgfpathlineto{\pgfqpoint{2.628041in}{3.211909in}}%
\pgfpathlineto{\pgfqpoint{2.630045in}{3.711000in}}%
\pgfpathlineto{\pgfqpoint{2.632050in}{3.385091in}}%
\pgfpathlineto{\pgfqpoint{2.634054in}{3.773727in}}%
\pgfpathlineto{\pgfqpoint{2.635057in}{3.783273in}}%
\pgfpathlineto{\pgfqpoint{2.637061in}{3.359182in}}%
\pgfpathlineto{\pgfqpoint{2.639066in}{3.773727in}}%
\pgfpathlineto{\pgfqpoint{2.640068in}{3.719182in}}%
\pgfpathlineto{\pgfqpoint{2.642072in}{3.056455in}}%
\pgfpathlineto{\pgfqpoint{2.643075in}{3.190091in}}%
\pgfpathlineto{\pgfqpoint{2.644077in}{3.513273in}}%
\pgfpathlineto{\pgfqpoint{2.646081in}{3.036000in}}%
\pgfpathlineto{\pgfqpoint{2.648086in}{2.955545in}}%
\pgfpathlineto{\pgfqpoint{2.649088in}{3.033273in}}%
\pgfpathlineto{\pgfqpoint{2.651093in}{2.865545in}}%
\pgfpathlineto{\pgfqpoint{2.652095in}{2.892818in}}%
\pgfpathlineto{\pgfqpoint{2.654099in}{2.789182in}}%
\pgfpathlineto{\pgfqpoint{2.655102in}{2.768727in}}%
\pgfpathlineto{\pgfqpoint{2.656104in}{2.779636in}}%
\pgfpathlineto{\pgfqpoint{2.657106in}{2.871000in}}%
\pgfpathlineto{\pgfqpoint{2.659111in}{2.673273in}}%
\pgfpathlineto{\pgfqpoint{2.660113in}{2.703273in}}%
\pgfpathlineto{\pgfqpoint{2.662117in}{2.873727in}}%
\pgfpathlineto{\pgfqpoint{2.664122in}{2.701909in}}%
\pgfpathlineto{\pgfqpoint{2.665124in}{2.757818in}}%
\pgfpathlineto{\pgfqpoint{2.667128in}{2.950091in}}%
\pgfpathlineto{\pgfqpoint{2.668131in}{2.924182in}}%
\pgfpathlineto{\pgfqpoint{2.670135in}{2.781000in}}%
\pgfpathlineto{\pgfqpoint{2.673142in}{3.142364in}}%
\pgfpathlineto{\pgfqpoint{2.674144in}{3.723273in}}%
\pgfpathlineto{\pgfqpoint{2.675146in}{3.558273in}}%
\pgfpathlineto{\pgfqpoint{2.676149in}{3.059182in}}%
\pgfpathlineto{\pgfqpoint{2.677151in}{3.067364in}}%
\pgfpathlineto{\pgfqpoint{2.678153in}{3.003273in}}%
\pgfpathlineto{\pgfqpoint{2.679155in}{3.690545in}}%
\pgfpathlineto{\pgfqpoint{2.680158in}{3.681000in}}%
\pgfpathlineto{\pgfqpoint{2.682162in}{3.142364in}}%
\pgfpathlineto{\pgfqpoint{2.683164in}{3.132818in}}%
\pgfpathlineto{\pgfqpoint{2.684167in}{3.541909in}}%
\pgfpathlineto{\pgfqpoint{2.685169in}{3.516000in}}%
\pgfpathlineto{\pgfqpoint{2.687173in}{2.961000in}}%
\pgfpathlineto{\pgfqpoint{2.688176in}{3.116455in}}%
\pgfpathlineto{\pgfqpoint{2.689178in}{3.756000in}}%
\pgfpathlineto{\pgfqpoint{2.690180in}{3.719182in}}%
\pgfpathlineto{\pgfqpoint{2.692185in}{2.997818in}}%
\pgfpathlineto{\pgfqpoint{2.693187in}{2.999182in}}%
\pgfpathlineto{\pgfqpoint{2.694189in}{3.004636in}}%
\pgfpathlineto{\pgfqpoint{2.695191in}{3.000545in}}%
\pgfpathlineto{\pgfqpoint{2.697196in}{2.956909in}}%
\pgfpathlineto{\pgfqpoint{2.699200in}{3.036000in}}%
\pgfpathlineto{\pgfqpoint{2.700202in}{2.812364in}}%
\pgfpathlineto{\pgfqpoint{2.701205in}{2.873727in}}%
\pgfpathlineto{\pgfqpoint{2.702207in}{2.864182in}}%
\pgfpathlineto{\pgfqpoint{2.703209in}{2.886000in}}%
\pgfpathlineto{\pgfqpoint{2.704211in}{2.967818in}}%
\pgfpathlineto{\pgfqpoint{2.705214in}{2.858727in}}%
\pgfpathlineto{\pgfqpoint{2.706216in}{2.877818in}}%
\pgfpathlineto{\pgfqpoint{2.707218in}{2.944636in}}%
\pgfpathlineto{\pgfqpoint{2.708220in}{2.871000in}}%
\pgfpathlineto{\pgfqpoint{2.709223in}{2.708727in}}%
\pgfpathlineto{\pgfqpoint{2.710225in}{2.740091in}}%
\pgfpathlineto{\pgfqpoint{2.711227in}{2.913273in}}%
\pgfpathlineto{\pgfqpoint{2.712229in}{2.881909in}}%
\pgfpathlineto{\pgfqpoint{2.714234in}{3.067364in}}%
\pgfpathlineto{\pgfqpoint{2.715236in}{2.839636in}}%
\pgfpathlineto{\pgfqpoint{2.718243in}{2.940545in}}%
\pgfpathlineto{\pgfqpoint{2.719245in}{3.141000in}}%
\pgfpathlineto{\pgfqpoint{2.720247in}{3.580091in}}%
\pgfpathlineto{\pgfqpoint{2.722252in}{3.029182in}}%
\pgfpathlineto{\pgfqpoint{2.723254in}{3.016909in}}%
\pgfpathlineto{\pgfqpoint{2.725259in}{3.217364in}}%
\pgfpathlineto{\pgfqpoint{2.726261in}{3.371455in}}%
\pgfpathlineto{\pgfqpoint{2.727263in}{3.157364in}}%
\pgfpathlineto{\pgfqpoint{2.729268in}{3.730091in}}%
\pgfpathlineto{\pgfqpoint{2.730270in}{3.660545in}}%
\pgfpathlineto{\pgfqpoint{2.732274in}{3.307364in}}%
\pgfpathlineto{\pgfqpoint{2.734279in}{3.779182in}}%
\pgfpathlineto{\pgfqpoint{2.735281in}{3.732818in}}%
\pgfpathlineto{\pgfqpoint{2.737285in}{3.443727in}}%
\pgfpathlineto{\pgfqpoint{2.739290in}{3.726000in}}%
\pgfpathlineto{\pgfqpoint{2.741294in}{2.984182in}}%
\pgfpathlineto{\pgfqpoint{2.742297in}{2.984182in}}%
\pgfpathlineto{\pgfqpoint{2.744301in}{3.640091in}}%
\pgfpathlineto{\pgfqpoint{2.747308in}{2.873727in}}%
\pgfpathlineto{\pgfqpoint{2.750315in}{2.969182in}}%
\pgfpathlineto{\pgfqpoint{2.751317in}{2.958273in}}%
\pgfpathlineto{\pgfqpoint{2.754324in}{2.625545in}}%
\pgfpathlineto{\pgfqpoint{2.756328in}{2.823273in}}%
\pgfpathlineto{\pgfqpoint{2.757330in}{2.806909in}}%
\pgfpathlineto{\pgfqpoint{2.759335in}{2.682818in}}%
\pgfpathlineto{\pgfqpoint{2.760337in}{2.693727in}}%
\pgfpathlineto{\pgfqpoint{2.762342in}{2.817818in}}%
\pgfpathlineto{\pgfqpoint{2.763344in}{2.741455in}}%
\pgfpathlineto{\pgfqpoint{2.764346in}{2.742818in}}%
\pgfpathlineto{\pgfqpoint{2.765348in}{2.789182in}}%
\pgfpathlineto{\pgfqpoint{2.767353in}{2.966455in}}%
\pgfpathlineto{\pgfqpoint{2.769357in}{2.826000in}}%
\pgfpathlineto{\pgfqpoint{2.772364in}{3.022364in}}%
\pgfpathlineto{\pgfqpoint{2.773366in}{3.608727in}}%
\pgfpathlineto{\pgfqpoint{2.775371in}{3.042818in}}%
\pgfpathlineto{\pgfqpoint{2.776373in}{3.056455in}}%
\pgfpathlineto{\pgfqpoint{2.777375in}{3.027818in}}%
\pgfpathlineto{\pgfqpoint{2.779380in}{3.735545in}}%
\pgfpathlineto{\pgfqpoint{2.780382in}{3.772364in}}%
\pgfpathlineto{\pgfqpoint{2.781384in}{3.393273in}}%
\pgfpathlineto{\pgfqpoint{2.785393in}{3.831000in}}%
\pgfpathlineto{\pgfqpoint{2.787398in}{3.619636in}}%
\pgfpathlineto{\pgfqpoint{2.789402in}{3.810545in}}%
\pgfpathlineto{\pgfqpoint{2.790404in}{3.689182in}}%
\pgfpathlineto{\pgfqpoint{2.791407in}{3.121909in}}%
\pgfpathlineto{\pgfqpoint{2.792409in}{3.138273in}}%
\pgfpathlineto{\pgfqpoint{2.794413in}{3.584182in}}%
\pgfpathlineto{\pgfqpoint{2.795416in}{3.413727in}}%
\pgfpathlineto{\pgfqpoint{2.797420in}{2.977364in}}%
\pgfpathlineto{\pgfqpoint{2.798422in}{2.836909in}}%
\pgfpathlineto{\pgfqpoint{2.799425in}{2.881909in}}%
\pgfpathlineto{\pgfqpoint{2.800427in}{2.868273in}}%
\pgfpathlineto{\pgfqpoint{2.801429in}{2.835545in}}%
\pgfpathlineto{\pgfqpoint{2.802431in}{2.845091in}}%
\pgfpathlineto{\pgfqpoint{2.804436in}{2.663727in}}%
\pgfpathlineto{\pgfqpoint{2.807442in}{2.770091in}}%
\pgfpathlineto{\pgfqpoint{2.809447in}{2.614636in}}%
\pgfpathlineto{\pgfqpoint{2.812454in}{2.876455in}}%
\pgfpathlineto{\pgfqpoint{2.814458in}{2.673273in}}%
\pgfpathlineto{\pgfqpoint{2.817465in}{3.003273in}}%
\pgfpathlineto{\pgfqpoint{2.818467in}{2.939182in}}%
\pgfpathlineto{\pgfqpoint{2.820472in}{2.976000in}}%
\pgfpathlineto{\pgfqpoint{2.821474in}{2.992364in}}%
\pgfpathlineto{\pgfqpoint{2.822476in}{3.034636in}}%
\pgfpathlineto{\pgfqpoint{2.823478in}{3.016909in}}%
\pgfpathlineto{\pgfqpoint{2.824481in}{3.657818in}}%
\pgfpathlineto{\pgfqpoint{2.825483in}{3.640091in}}%
\pgfpathlineto{\pgfqpoint{2.826485in}{3.280091in}}%
\pgfpathlineto{\pgfqpoint{2.827487in}{3.349636in}}%
\pgfpathlineto{\pgfqpoint{2.830494in}{3.799636in}}%
\pgfpathlineto{\pgfqpoint{2.831496in}{3.528273in}}%
\pgfpathlineto{\pgfqpoint{2.832499in}{3.533727in}}%
\pgfpathlineto{\pgfqpoint{2.834503in}{3.839182in}}%
\pgfpathlineto{\pgfqpoint{2.835505in}{3.790091in}}%
\pgfpathlineto{\pgfqpoint{2.836508in}{3.619636in}}%
\pgfpathlineto{\pgfqpoint{2.837510in}{3.646909in}}%
\pgfpathlineto{\pgfqpoint{2.839514in}{3.907364in}}%
\pgfpathlineto{\pgfqpoint{2.840516in}{3.705545in}}%
\pgfpathlineto{\pgfqpoint{2.842521in}{3.014182in}}%
\pgfpathlineto{\pgfqpoint{2.844525in}{3.736909in}}%
\pgfpathlineto{\pgfqpoint{2.846530in}{2.991000in}}%
\pgfpathlineto{\pgfqpoint{2.848534in}{2.798727in}}%
\pgfpathlineto{\pgfqpoint{2.850539in}{2.932364in}}%
\pgfpathlineto{\pgfqpoint{2.851541in}{2.969182in}}%
\pgfpathlineto{\pgfqpoint{2.852543in}{2.881909in}}%
\pgfpathlineto{\pgfqpoint{2.854548in}{2.670545in}}%
\pgfpathlineto{\pgfqpoint{2.856552in}{2.869636in}}%
\pgfpathlineto{\pgfqpoint{2.857555in}{2.858727in}}%
\pgfpathlineto{\pgfqpoint{2.859559in}{2.655545in}}%
\pgfpathlineto{\pgfqpoint{2.860561in}{2.697818in}}%
\pgfpathlineto{\pgfqpoint{2.862566in}{2.809636in}}%
\pgfpathlineto{\pgfqpoint{2.864570in}{2.812364in}}%
\pgfpathlineto{\pgfqpoint{2.866575in}{2.916000in}}%
\pgfpathlineto{\pgfqpoint{2.868579in}{2.849182in}}%
\pgfpathlineto{\pgfqpoint{2.870584in}{2.954182in}}%
\pgfpathlineto{\pgfqpoint{2.871586in}{3.113727in}}%
\pgfpathlineto{\pgfqpoint{2.872588in}{3.089182in}}%
\pgfpathlineto{\pgfqpoint{2.873590in}{3.008727in}}%
\pgfpathlineto{\pgfqpoint{2.875595in}{3.278727in}}%
\pgfpathlineto{\pgfqpoint{2.876597in}{3.449182in}}%
\pgfpathlineto{\pgfqpoint{2.877599in}{3.325091in}}%
\pgfpathlineto{\pgfqpoint{2.879604in}{3.708273in}}%
\pgfpathlineto{\pgfqpoint{2.880606in}{3.732818in}}%
\pgfpathlineto{\pgfqpoint{2.882611in}{3.631909in}}%
\pgfpathlineto{\pgfqpoint{2.883613in}{3.691909in}}%
\pgfpathlineto{\pgfqpoint{2.884615in}{3.863727in}}%
\pgfpathlineto{\pgfqpoint{2.885617in}{3.719182in}}%
\pgfpathlineto{\pgfqpoint{2.887622in}{3.239182in}}%
\pgfpathlineto{\pgfqpoint{2.889626in}{3.821455in}}%
\pgfpathlineto{\pgfqpoint{2.891631in}{3.663273in}}%
\pgfpathlineto{\pgfqpoint{2.892633in}{3.082364in}}%
\pgfpathlineto{\pgfqpoint{2.894638in}{3.396000in}}%
\pgfpathlineto{\pgfqpoint{2.898647in}{2.854636in}}%
\pgfpathlineto{\pgfqpoint{2.900651in}{2.961000in}}%
\pgfpathlineto{\pgfqpoint{2.901653in}{2.955545in}}%
\pgfpathlineto{\pgfqpoint{2.903658in}{2.699182in}}%
\pgfpathlineto{\pgfqpoint{2.904660in}{2.669182in}}%
\pgfpathlineto{\pgfqpoint{2.906665in}{2.876455in}}%
\pgfpathlineto{\pgfqpoint{2.907667in}{2.809636in}}%
\pgfpathlineto{\pgfqpoint{2.908669in}{2.666455in}}%
\pgfpathlineto{\pgfqpoint{2.909671in}{2.686909in}}%
\pgfpathlineto{\pgfqpoint{2.910673in}{2.751000in}}%
\pgfpathlineto{\pgfqpoint{2.911676in}{2.913273in}}%
\pgfpathlineto{\pgfqpoint{2.913680in}{2.674636in}}%
\pgfpathlineto{\pgfqpoint{2.914682in}{2.708727in}}%
\pgfpathlineto{\pgfqpoint{2.916687in}{2.997818in}}%
\pgfpathlineto{\pgfqpoint{2.918691in}{2.853273in}}%
\pgfpathlineto{\pgfqpoint{2.919694in}{2.845091in}}%
\pgfpathlineto{\pgfqpoint{2.923703in}{3.126000in}}%
\pgfpathlineto{\pgfqpoint{2.925707in}{3.191455in}}%
\pgfpathlineto{\pgfqpoint{2.926709in}{3.288273in}}%
\pgfpathlineto{\pgfqpoint{2.928714in}{3.775091in}}%
\pgfpathlineto{\pgfqpoint{2.930718in}{3.640091in}}%
\pgfpathlineto{\pgfqpoint{2.931721in}{3.649636in}}%
\pgfpathlineto{\pgfqpoint{2.932723in}{3.711000in}}%
\pgfpathlineto{\pgfqpoint{2.933725in}{3.888273in}}%
\pgfpathlineto{\pgfqpoint{2.934727in}{3.870545in}}%
\pgfpathlineto{\pgfqpoint{2.936732in}{3.618273in}}%
\pgfpathlineto{\pgfqpoint{2.937734in}{3.588273in}}%
\pgfpathlineto{\pgfqpoint{2.939739in}{3.773727in}}%
\pgfpathlineto{\pgfqpoint{2.940741in}{3.716455in}}%
\pgfpathlineto{\pgfqpoint{2.942745in}{3.423273in}}%
\pgfpathlineto{\pgfqpoint{2.943747in}{3.475091in}}%
\pgfpathlineto{\pgfqpoint{2.945752in}{3.003273in}}%
\pgfpathlineto{\pgfqpoint{2.946754in}{2.907818in}}%
\pgfpathlineto{\pgfqpoint{2.948759in}{3.010091in}}%
\pgfpathlineto{\pgfqpoint{2.950763in}{2.763273in}}%
\pgfpathlineto{\pgfqpoint{2.951765in}{2.824636in}}%
\pgfpathlineto{\pgfqpoint{2.954772in}{2.658273in}}%
\pgfpathlineto{\pgfqpoint{2.956777in}{2.787818in}}%
\pgfpathlineto{\pgfqpoint{2.959783in}{2.636455in}}%
\pgfpathlineto{\pgfqpoint{2.962790in}{2.869636in}}%
\pgfpathlineto{\pgfqpoint{2.963792in}{2.775545in}}%
\pgfpathlineto{\pgfqpoint{2.965797in}{2.849182in}}%
\pgfpathlineto{\pgfqpoint{2.967801in}{2.943273in}}%
\pgfpathlineto{\pgfqpoint{2.968804in}{2.928273in}}%
\pgfpathlineto{\pgfqpoint{2.969806in}{2.872364in}}%
\pgfpathlineto{\pgfqpoint{2.971810in}{2.965091in}}%
\pgfpathlineto{\pgfqpoint{2.972813in}{3.149182in}}%
\pgfpathlineto{\pgfqpoint{2.973815in}{3.629182in}}%
\pgfpathlineto{\pgfqpoint{2.975819in}{3.321000in}}%
\pgfpathlineto{\pgfqpoint{2.976822in}{3.220091in}}%
\pgfpathlineto{\pgfqpoint{2.978826in}{3.833727in}}%
\pgfpathlineto{\pgfqpoint{2.981833in}{3.548727in}}%
\pgfpathlineto{\pgfqpoint{2.983837in}{3.953727in}}%
\pgfpathlineto{\pgfqpoint{2.984839in}{3.937364in}}%
\pgfpathlineto{\pgfqpoint{2.986844in}{3.623727in}}%
\pgfpathlineto{\pgfqpoint{2.989851in}{3.851455in}}%
\pgfpathlineto{\pgfqpoint{2.990853in}{3.573273in}}%
\pgfpathlineto{\pgfqpoint{2.991855in}{2.947364in}}%
\pgfpathlineto{\pgfqpoint{2.993860in}{3.277364in}}%
\pgfpathlineto{\pgfqpoint{2.996866in}{2.871000in}}%
\pgfpathlineto{\pgfqpoint{2.997869in}{2.880545in}}%
\pgfpathlineto{\pgfqpoint{2.999873in}{2.836909in}}%
\pgfpathlineto{\pgfqpoint{3.000875in}{2.841000in}}%
\pgfpathlineto{\pgfqpoint{3.004884in}{2.646000in}}%
\pgfpathlineto{\pgfqpoint{3.005887in}{2.854636in}}%
\pgfpathlineto{\pgfqpoint{3.006889in}{2.835545in}}%
\pgfpathlineto{\pgfqpoint{3.009896in}{2.606455in}}%
\pgfpathlineto{\pgfqpoint{3.011900in}{2.865545in}}%
\pgfpathlineto{\pgfqpoint{3.014907in}{2.722364in}}%
\pgfpathlineto{\pgfqpoint{3.017913in}{3.010091in}}%
\pgfpathlineto{\pgfqpoint{3.019918in}{2.876455in}}%
\pgfpathlineto{\pgfqpoint{3.020920in}{2.902364in}}%
\pgfpathlineto{\pgfqpoint{3.021922in}{3.007364in}}%
\pgfpathlineto{\pgfqpoint{3.022925in}{3.469636in}}%
\pgfpathlineto{\pgfqpoint{3.023927in}{3.364636in}}%
\pgfpathlineto{\pgfqpoint{3.024929in}{3.548727in}}%
\pgfpathlineto{\pgfqpoint{3.026934in}{3.060545in}}%
\pgfpathlineto{\pgfqpoint{3.028938in}{3.698727in}}%
\pgfpathlineto{\pgfqpoint{3.029940in}{3.795545in}}%
\pgfpathlineto{\pgfqpoint{3.031945in}{3.543273in}}%
\pgfpathlineto{\pgfqpoint{3.034952in}{4.013727in}}%
\pgfpathlineto{\pgfqpoint{3.036956in}{3.506455in}}%
\pgfpathlineto{\pgfqpoint{3.038961in}{3.869182in}}%
\pgfpathlineto{\pgfqpoint{3.039963in}{3.820091in}}%
\pgfpathlineto{\pgfqpoint{3.041967in}{3.111000in}}%
\pgfpathlineto{\pgfqpoint{3.042970in}{3.548727in}}%
\pgfpathlineto{\pgfqpoint{3.043972in}{3.529636in}}%
\pgfpathlineto{\pgfqpoint{3.044974in}{3.573273in}}%
\pgfpathlineto{\pgfqpoint{3.046979in}{2.901000in}}%
\pgfpathlineto{\pgfqpoint{3.047981in}{2.931000in}}%
\pgfpathlineto{\pgfqpoint{3.048983in}{2.760545in}}%
\pgfpathlineto{\pgfqpoint{3.050987in}{2.847818in}}%
\pgfpathlineto{\pgfqpoint{3.051990in}{2.853273in}}%
\pgfpathlineto{\pgfqpoint{3.053994in}{2.599636in}}%
\pgfpathlineto{\pgfqpoint{3.057001in}{2.781000in}}%
\pgfpathlineto{\pgfqpoint{3.059005in}{2.595545in}}%
\pgfpathlineto{\pgfqpoint{3.061010in}{2.757818in}}%
\pgfpathlineto{\pgfqpoint{3.062012in}{2.892818in}}%
\pgfpathlineto{\pgfqpoint{3.064017in}{2.731909in}}%
\pgfpathlineto{\pgfqpoint{3.065019in}{2.761909in}}%
\pgfpathlineto{\pgfqpoint{3.069028in}{2.939182in}}%
\pgfpathlineto{\pgfqpoint{3.070030in}{2.947364in}}%
\pgfpathlineto{\pgfqpoint{3.071032in}{2.926909in}}%
\pgfpathlineto{\pgfqpoint{3.072035in}{3.072818in}}%
\pgfpathlineto{\pgfqpoint{3.073037in}{3.042818in}}%
\pgfpathlineto{\pgfqpoint{3.074039in}{3.119182in}}%
\pgfpathlineto{\pgfqpoint{3.075041in}{3.622364in}}%
\pgfpathlineto{\pgfqpoint{3.076044in}{3.040091in}}%
\pgfpathlineto{\pgfqpoint{3.078048in}{3.657818in}}%
\pgfpathlineto{\pgfqpoint{3.079050in}{3.686455in}}%
\pgfpathlineto{\pgfqpoint{3.080053in}{3.822818in}}%
\pgfpathlineto{\pgfqpoint{3.082057in}{3.687818in}}%
\pgfpathlineto{\pgfqpoint{3.085064in}{4.056000in}}%
\pgfpathlineto{\pgfqpoint{3.087068in}{3.713727in}}%
\pgfpathlineto{\pgfqpoint{3.088070in}{3.779182in}}%
\pgfpathlineto{\pgfqpoint{3.089073in}{3.942818in}}%
\pgfpathlineto{\pgfqpoint{3.090075in}{3.931909in}}%
\pgfpathlineto{\pgfqpoint{3.093082in}{3.007364in}}%
\pgfpathlineto{\pgfqpoint{3.094084in}{3.551455in}}%
\pgfpathlineto{\pgfqpoint{3.095086in}{3.398727in}}%
\pgfpathlineto{\pgfqpoint{3.096088in}{2.969182in}}%
\pgfpathlineto{\pgfqpoint{3.097091in}{3.018273in}}%
\pgfpathlineto{\pgfqpoint{3.099095in}{2.819182in}}%
\pgfpathlineto{\pgfqpoint{3.101100in}{2.851909in}}%
\pgfpathlineto{\pgfqpoint{3.102102in}{2.980091in}}%
\pgfpathlineto{\pgfqpoint{3.104106in}{2.654182in}}%
\pgfpathlineto{\pgfqpoint{3.105109in}{2.678727in}}%
\pgfpathlineto{\pgfqpoint{3.107113in}{2.872364in}}%
\pgfpathlineto{\pgfqpoint{3.109118in}{2.659636in}}%
\pgfpathlineto{\pgfqpoint{3.110120in}{2.701909in}}%
\pgfpathlineto{\pgfqpoint{3.112124in}{2.890091in}}%
\pgfpathlineto{\pgfqpoint{3.114129in}{2.781000in}}%
\pgfpathlineto{\pgfqpoint{3.115131in}{2.826000in}}%
\pgfpathlineto{\pgfqpoint{3.117136in}{2.993727in}}%
\pgfpathlineto{\pgfqpoint{3.118138in}{2.830091in}}%
\pgfpathlineto{\pgfqpoint{3.119140in}{2.842364in}}%
\pgfpathlineto{\pgfqpoint{3.121144in}{2.969182in}}%
\pgfpathlineto{\pgfqpoint{3.122147in}{3.194182in}}%
\pgfpathlineto{\pgfqpoint{3.123149in}{3.011455in}}%
\pgfpathlineto{\pgfqpoint{3.124151in}{3.149182in}}%
\pgfpathlineto{\pgfqpoint{3.125153in}{3.045545in}}%
\pgfpathlineto{\pgfqpoint{3.126156in}{3.059182in}}%
\pgfpathlineto{\pgfqpoint{3.129162in}{3.713727in}}%
\pgfpathlineto{\pgfqpoint{3.131167in}{3.660545in}}%
\pgfpathlineto{\pgfqpoint{3.132169in}{3.476455in}}%
\pgfpathlineto{\pgfqpoint{3.135176in}{3.976909in}}%
\pgfpathlineto{\pgfqpoint{3.137180in}{3.691909in}}%
\pgfpathlineto{\pgfqpoint{3.139185in}{3.929182in}}%
\pgfpathlineto{\pgfqpoint{3.141189in}{3.784636in}}%
\pgfpathlineto{\pgfqpoint{3.142192in}{3.603273in}}%
\pgfpathlineto{\pgfqpoint{3.143194in}{3.083727in}}%
\pgfpathlineto{\pgfqpoint{3.144196in}{3.705545in}}%
\pgfpathlineto{\pgfqpoint{3.145198in}{3.168273in}}%
\pgfpathlineto{\pgfqpoint{3.146201in}{3.233727in}}%
\pgfpathlineto{\pgfqpoint{3.148205in}{2.830091in}}%
\pgfpathlineto{\pgfqpoint{3.149207in}{2.835545in}}%
\pgfpathlineto{\pgfqpoint{3.150210in}{2.800091in}}%
\pgfpathlineto{\pgfqpoint{3.151212in}{2.911909in}}%
\pgfpathlineto{\pgfqpoint{3.152214in}{2.910545in}}%
\pgfpathlineto{\pgfqpoint{3.154219in}{2.682818in}}%
\pgfpathlineto{\pgfqpoint{3.155221in}{2.663727in}}%
\pgfpathlineto{\pgfqpoint{3.157225in}{2.831455in}}%
\pgfpathlineto{\pgfqpoint{3.159230in}{2.663727in}}%
\pgfpathlineto{\pgfqpoint{3.162236in}{2.937818in}}%
\pgfpathlineto{\pgfqpoint{3.164241in}{2.723727in}}%
\pgfpathlineto{\pgfqpoint{3.165243in}{2.767364in}}%
\pgfpathlineto{\pgfqpoint{3.166245in}{2.895545in}}%
\pgfpathlineto{\pgfqpoint{3.167248in}{2.886000in}}%
\pgfpathlineto{\pgfqpoint{3.168250in}{2.931000in}}%
\pgfpathlineto{\pgfqpoint{3.170254in}{2.881909in}}%
\pgfpathlineto{\pgfqpoint{3.171257in}{3.014182in}}%
\pgfpathlineto{\pgfqpoint{3.172259in}{3.006000in}}%
\pgfpathlineto{\pgfqpoint{3.173261in}{3.076909in}}%
\pgfpathlineto{\pgfqpoint{3.174263in}{3.241909in}}%
\pgfpathlineto{\pgfqpoint{3.175266in}{3.097364in}}%
\pgfpathlineto{\pgfqpoint{3.177270in}{3.202364in}}%
\pgfpathlineto{\pgfqpoint{3.179275in}{3.749182in}}%
\pgfpathlineto{\pgfqpoint{3.180277in}{3.701455in}}%
\pgfpathlineto{\pgfqpoint{3.181279in}{3.539182in}}%
\pgfpathlineto{\pgfqpoint{3.182281in}{3.180545in}}%
\pgfpathlineto{\pgfqpoint{3.184286in}{3.841909in}}%
\pgfpathlineto{\pgfqpoint{3.186290in}{3.743727in}}%
\pgfpathlineto{\pgfqpoint{3.187293in}{3.194182in}}%
\pgfpathlineto{\pgfqpoint{3.189297in}{3.944182in}}%
\pgfpathlineto{\pgfqpoint{3.191302in}{3.843273in}}%
\pgfpathlineto{\pgfqpoint{3.192304in}{3.228273in}}%
\pgfpathlineto{\pgfqpoint{3.193306in}{3.569182in}}%
\pgfpathlineto{\pgfqpoint{3.195310in}{3.029182in}}%
\pgfpathlineto{\pgfqpoint{3.196313in}{3.104182in}}%
\pgfpathlineto{\pgfqpoint{3.197315in}{2.907818in}}%
\pgfpathlineto{\pgfqpoint{3.198317in}{2.961000in}}%
\pgfpathlineto{\pgfqpoint{3.200322in}{2.816455in}}%
\pgfpathlineto{\pgfqpoint{3.201324in}{2.898273in}}%
\pgfpathlineto{\pgfqpoint{3.204331in}{2.742818in}}%
\pgfpathlineto{\pgfqpoint{3.205333in}{2.729182in}}%
\pgfpathlineto{\pgfqpoint{3.206335in}{2.898273in}}%
\pgfpathlineto{\pgfqpoint{3.207337in}{2.860091in}}%
\pgfpathlineto{\pgfqpoint{3.209342in}{2.677364in}}%
\pgfpathlineto{\pgfqpoint{3.211346in}{2.888727in}}%
\pgfpathlineto{\pgfqpoint{3.212349in}{2.835545in}}%
\pgfpathlineto{\pgfqpoint{3.213351in}{2.831455in}}%
\pgfpathlineto{\pgfqpoint{3.214353in}{2.786455in}}%
\pgfpathlineto{\pgfqpoint{3.216358in}{3.006000in}}%
\pgfpathlineto{\pgfqpoint{3.218362in}{2.935091in}}%
\pgfpathlineto{\pgfqpoint{3.219364in}{2.865545in}}%
\pgfpathlineto{\pgfqpoint{3.222371in}{3.093273in}}%
\pgfpathlineto{\pgfqpoint{3.223373in}{3.348273in}}%
\pgfpathlineto{\pgfqpoint{3.225378in}{3.059182in}}%
\pgfpathlineto{\pgfqpoint{3.226380in}{3.078273in}}%
\pgfpathlineto{\pgfqpoint{3.227382in}{3.150545in}}%
\pgfpathlineto{\pgfqpoint{3.229387in}{3.867818in}}%
\pgfpathlineto{\pgfqpoint{3.232393in}{3.623727in}}%
\pgfpathlineto{\pgfqpoint{3.234398in}{3.933273in}}%
\pgfpathlineto{\pgfqpoint{3.235400in}{3.956455in}}%
\pgfpathlineto{\pgfqpoint{3.236402in}{3.670091in}}%
\pgfpathlineto{\pgfqpoint{3.237405in}{3.672818in}}%
\pgfpathlineto{\pgfqpoint{3.239409in}{3.916909in}}%
\pgfpathlineto{\pgfqpoint{3.240411in}{3.955091in}}%
\pgfpathlineto{\pgfqpoint{3.242416in}{3.131455in}}%
\pgfpathlineto{\pgfqpoint{3.243418in}{3.322364in}}%
\pgfpathlineto{\pgfqpoint{3.244420in}{3.102818in}}%
\pgfpathlineto{\pgfqpoint{3.245423in}{3.258273in}}%
\pgfpathlineto{\pgfqpoint{3.248429in}{2.898273in}}%
\pgfpathlineto{\pgfqpoint{3.249432in}{2.808273in}}%
\pgfpathlineto{\pgfqpoint{3.250434in}{2.943273in}}%
\pgfpathlineto{\pgfqpoint{3.253441in}{2.772818in}}%
\pgfpathlineto{\pgfqpoint{3.254443in}{2.650091in}}%
\pgfpathlineto{\pgfqpoint{3.257450in}{2.894182in}}%
\pgfpathlineto{\pgfqpoint{3.259454in}{2.714182in}}%
\pgfpathlineto{\pgfqpoint{3.262461in}{2.941909in}}%
\pgfpathlineto{\pgfqpoint{3.264465in}{2.800091in}}%
\pgfpathlineto{\pgfqpoint{3.267472in}{2.971909in}}%
\pgfpathlineto{\pgfqpoint{3.268474in}{2.931000in}}%
\pgfpathlineto{\pgfqpoint{3.269476in}{2.974636in}}%
\pgfpathlineto{\pgfqpoint{3.271481in}{2.937818in}}%
\pgfpathlineto{\pgfqpoint{3.272483in}{3.051000in}}%
\pgfpathlineto{\pgfqpoint{3.273485in}{3.329182in}}%
\pgfpathlineto{\pgfqpoint{3.274488in}{3.146455in}}%
\pgfpathlineto{\pgfqpoint{3.275490in}{3.236455in}}%
\pgfpathlineto{\pgfqpoint{3.276492in}{3.033273in}}%
\pgfpathlineto{\pgfqpoint{3.278497in}{3.780545in}}%
\pgfpathlineto{\pgfqpoint{3.279499in}{3.871909in}}%
\pgfpathlineto{\pgfqpoint{3.280501in}{3.798273in}}%
\pgfpathlineto{\pgfqpoint{3.281503in}{3.390545in}}%
\pgfpathlineto{\pgfqpoint{3.283508in}{3.809182in}}%
\pgfpathlineto{\pgfqpoint{3.284510in}{3.981000in}}%
\pgfpathlineto{\pgfqpoint{3.286515in}{3.438273in}}%
\pgfpathlineto{\pgfqpoint{3.288519in}{3.938727in}}%
\pgfpathlineto{\pgfqpoint{3.289521in}{4.046455in}}%
\pgfpathlineto{\pgfqpoint{3.290524in}{3.851455in}}%
\pgfpathlineto{\pgfqpoint{3.292528in}{3.100091in}}%
\pgfpathlineto{\pgfqpoint{3.293530in}{3.021000in}}%
\pgfpathlineto{\pgfqpoint{3.294533in}{3.631909in}}%
\pgfpathlineto{\pgfqpoint{3.295535in}{3.471000in}}%
\pgfpathlineto{\pgfqpoint{3.296537in}{3.018273in}}%
\pgfpathlineto{\pgfqpoint{3.297539in}{3.066000in}}%
\pgfpathlineto{\pgfqpoint{3.299544in}{2.806909in}}%
\pgfpathlineto{\pgfqpoint{3.300546in}{2.896909in}}%
\pgfpathlineto{\pgfqpoint{3.301548in}{2.868273in}}%
\pgfpathlineto{\pgfqpoint{3.302550in}{2.941909in}}%
\pgfpathlineto{\pgfqpoint{3.304555in}{2.730545in}}%
\pgfpathlineto{\pgfqpoint{3.306559in}{2.895545in}}%
\pgfpathlineto{\pgfqpoint{3.307562in}{2.861455in}}%
\pgfpathlineto{\pgfqpoint{3.309566in}{2.706000in}}%
\pgfpathlineto{\pgfqpoint{3.312573in}{2.902364in}}%
\pgfpathlineto{\pgfqpoint{3.314577in}{2.808273in}}%
\pgfpathlineto{\pgfqpoint{3.316582in}{2.860091in}}%
\pgfpathlineto{\pgfqpoint{3.319589in}{3.056455in}}%
\pgfpathlineto{\pgfqpoint{3.320591in}{2.939182in}}%
\pgfpathlineto{\pgfqpoint{3.321593in}{2.951455in}}%
\pgfpathlineto{\pgfqpoint{3.322595in}{3.034636in}}%
\pgfpathlineto{\pgfqpoint{3.323598in}{2.978727in}}%
\pgfpathlineto{\pgfqpoint{3.324600in}{3.311455in}}%
\pgfpathlineto{\pgfqpoint{3.326604in}{3.066000in}}%
\pgfpathlineto{\pgfqpoint{3.327607in}{3.360545in}}%
\pgfpathlineto{\pgfqpoint{3.328609in}{3.248727in}}%
\pgfpathlineto{\pgfqpoint{3.329611in}{3.736909in}}%
\pgfpathlineto{\pgfqpoint{3.331616in}{3.559636in}}%
\pgfpathlineto{\pgfqpoint{3.332618in}{3.659182in}}%
\pgfpathlineto{\pgfqpoint{3.334622in}{4.016455in}}%
\pgfpathlineto{\pgfqpoint{3.336627in}{3.610091in}}%
\pgfpathlineto{\pgfqpoint{3.337629in}{3.656455in}}%
\pgfpathlineto{\pgfqpoint{3.339633in}{3.892364in}}%
\pgfpathlineto{\pgfqpoint{3.341638in}{3.104182in}}%
\pgfpathlineto{\pgfqpoint{3.343642in}{3.741000in}}%
\pgfpathlineto{\pgfqpoint{3.344645in}{3.798273in}}%
\pgfpathlineto{\pgfqpoint{3.345647in}{3.012818in}}%
\pgfpathlineto{\pgfqpoint{3.346649in}{3.056455in}}%
\pgfpathlineto{\pgfqpoint{3.347651in}{2.903727in}}%
\pgfpathlineto{\pgfqpoint{3.348654in}{2.905091in}}%
\pgfpathlineto{\pgfqpoint{3.349656in}{3.411000in}}%
\pgfpathlineto{\pgfqpoint{3.350658in}{2.963727in}}%
\pgfpathlineto{\pgfqpoint{3.351660in}{2.985545in}}%
\pgfpathlineto{\pgfqpoint{3.353665in}{2.748273in}}%
\pgfpathlineto{\pgfqpoint{3.356672in}{2.922818in}}%
\pgfpathlineto{\pgfqpoint{3.358676in}{2.782364in}}%
\pgfpathlineto{\pgfqpoint{3.359678in}{2.766000in}}%
\pgfpathlineto{\pgfqpoint{3.361683in}{2.944636in}}%
\pgfpathlineto{\pgfqpoint{3.364690in}{2.786455in}}%
\pgfpathlineto{\pgfqpoint{3.366694in}{2.976000in}}%
\pgfpathlineto{\pgfqpoint{3.367696in}{2.898273in}}%
\pgfpathlineto{\pgfqpoint{3.368698in}{2.902364in}}%
\pgfpathlineto{\pgfqpoint{3.369701in}{2.909182in}}%
\pgfpathlineto{\pgfqpoint{3.370703in}{2.843727in}}%
\pgfpathlineto{\pgfqpoint{3.371705in}{3.018273in}}%
\pgfpathlineto{\pgfqpoint{3.372707in}{2.895545in}}%
\pgfpathlineto{\pgfqpoint{3.374712in}{3.211909in}}%
\pgfpathlineto{\pgfqpoint{3.375714in}{3.019636in}}%
\pgfpathlineto{\pgfqpoint{3.376716in}{3.121909in}}%
\pgfpathlineto{\pgfqpoint{3.377719in}{3.037364in}}%
\pgfpathlineto{\pgfqpoint{3.379723in}{3.749182in}}%
\pgfpathlineto{\pgfqpoint{3.381728in}{3.259636in}}%
\pgfpathlineto{\pgfqpoint{3.384734in}{3.982364in}}%
\pgfpathlineto{\pgfqpoint{3.387741in}{3.686455in}}%
\pgfpathlineto{\pgfqpoint{3.388743in}{3.931909in}}%
\pgfpathlineto{\pgfqpoint{3.389746in}{3.895091in}}%
\pgfpathlineto{\pgfqpoint{3.391750in}{3.736909in}}%
\pgfpathlineto{\pgfqpoint{3.392752in}{3.195545in}}%
\pgfpathlineto{\pgfqpoint{3.393755in}{3.747818in}}%
\pgfpathlineto{\pgfqpoint{3.394757in}{3.682364in}}%
\pgfpathlineto{\pgfqpoint{3.395759in}{3.428727in}}%
\pgfpathlineto{\pgfqpoint{3.396761in}{3.457364in}}%
\pgfpathlineto{\pgfqpoint{3.397764in}{2.950091in}}%
\pgfpathlineto{\pgfqpoint{3.398766in}{3.250091in}}%
\pgfpathlineto{\pgfqpoint{3.399768in}{2.991000in}}%
\pgfpathlineto{\pgfqpoint{3.400770in}{3.011455in}}%
\pgfpathlineto{\pgfqpoint{3.401773in}{3.037364in}}%
\pgfpathlineto{\pgfqpoint{3.403777in}{2.813727in}}%
\pgfpathlineto{\pgfqpoint{3.406784in}{2.969182in}}%
\pgfpathlineto{\pgfqpoint{3.408788in}{2.821909in}}%
\pgfpathlineto{\pgfqpoint{3.409790in}{2.808273in}}%
\pgfpathlineto{\pgfqpoint{3.411795in}{2.917364in}}%
\pgfpathlineto{\pgfqpoint{3.412797in}{2.839636in}}%
\pgfpathlineto{\pgfqpoint{3.413799in}{2.849182in}}%
\pgfpathlineto{\pgfqpoint{3.414802in}{2.790545in}}%
\pgfpathlineto{\pgfqpoint{3.416806in}{2.944636in}}%
\pgfpathlineto{\pgfqpoint{3.419813in}{2.793273in}}%
\pgfpathlineto{\pgfqpoint{3.421817in}{2.967818in}}%
\pgfpathlineto{\pgfqpoint{3.422820in}{2.914636in}}%
\pgfpathlineto{\pgfqpoint{3.423822in}{2.999182in}}%
\pgfpathlineto{\pgfqpoint{3.424824in}{2.887364in}}%
\pgfpathlineto{\pgfqpoint{3.427831in}{3.051000in}}%
\pgfpathlineto{\pgfqpoint{3.428833in}{3.559636in}}%
\pgfpathlineto{\pgfqpoint{3.430838in}{3.104182in}}%
\pgfpathlineto{\pgfqpoint{3.431840in}{3.091909in}}%
\pgfpathlineto{\pgfqpoint{3.433844in}{3.897818in}}%
\pgfpathlineto{\pgfqpoint{3.436851in}{3.701455in}}%
\pgfpathlineto{\pgfqpoint{3.438856in}{4.012364in}}%
\pgfpathlineto{\pgfqpoint{3.441862in}{3.537818in}}%
\pgfpathlineto{\pgfqpoint{3.442864in}{3.816000in}}%
\pgfpathlineto{\pgfqpoint{3.443867in}{3.753273in}}%
\pgfpathlineto{\pgfqpoint{3.444869in}{3.588273in}}%
\pgfpathlineto{\pgfqpoint{3.445871in}{3.611455in}}%
\pgfpathlineto{\pgfqpoint{3.447876in}{2.971909in}}%
\pgfpathlineto{\pgfqpoint{3.448878in}{3.254182in}}%
\pgfpathlineto{\pgfqpoint{3.452887in}{2.888727in}}%
\pgfpathlineto{\pgfqpoint{3.453889in}{2.876455in}}%
\pgfpathlineto{\pgfqpoint{3.454891in}{2.821909in}}%
\pgfpathlineto{\pgfqpoint{3.455894in}{2.914636in}}%
\pgfpathlineto{\pgfqpoint{3.457898in}{2.875091in}}%
\pgfpathlineto{\pgfqpoint{3.459903in}{2.763273in}}%
\pgfpathlineto{\pgfqpoint{3.460905in}{2.909182in}}%
\pgfpathlineto{\pgfqpoint{3.461907in}{2.861455in}}%
\pgfpathlineto{\pgfqpoint{3.462909in}{2.892818in}}%
\pgfpathlineto{\pgfqpoint{3.463912in}{2.805545in}}%
\pgfpathlineto{\pgfqpoint{3.464914in}{2.830091in}}%
\pgfpathlineto{\pgfqpoint{3.465916in}{3.016909in}}%
\pgfpathlineto{\pgfqpoint{3.466918in}{2.905091in}}%
\pgfpathlineto{\pgfqpoint{3.467921in}{2.962364in}}%
\pgfpathlineto{\pgfqpoint{3.469925in}{2.861455in}}%
\pgfpathlineto{\pgfqpoint{3.472932in}{3.027818in}}%
\pgfpathlineto{\pgfqpoint{3.473934in}{3.010091in}}%
\pgfpathlineto{\pgfqpoint{3.474936in}{2.932364in}}%
\pgfpathlineto{\pgfqpoint{3.476941in}{3.011455in}}%
\pgfpathlineto{\pgfqpoint{3.478945in}{3.698727in}}%
\pgfpathlineto{\pgfqpoint{3.481952in}{3.274636in}}%
\pgfpathlineto{\pgfqpoint{3.483956in}{3.833727in}}%
\pgfpathlineto{\pgfqpoint{3.484959in}{3.877364in}}%
\pgfpathlineto{\pgfqpoint{3.486963in}{3.442364in}}%
\pgfpathlineto{\pgfqpoint{3.487965in}{3.847364in}}%
\pgfpathlineto{\pgfqpoint{3.488968in}{3.817364in}}%
\pgfpathlineto{\pgfqpoint{3.489970in}{3.956455in}}%
\pgfpathlineto{\pgfqpoint{3.490972in}{3.719182in}}%
\pgfpathlineto{\pgfqpoint{3.491974in}{3.199636in}}%
\pgfpathlineto{\pgfqpoint{3.492977in}{3.769636in}}%
\pgfpathlineto{\pgfqpoint{3.493979in}{3.510545in}}%
\pgfpathlineto{\pgfqpoint{3.494981in}{3.700091in}}%
\pgfpathlineto{\pgfqpoint{3.495983in}{3.600545in}}%
\pgfpathlineto{\pgfqpoint{3.497988in}{3.045545in}}%
\pgfpathlineto{\pgfqpoint{3.499992in}{2.978727in}}%
\pgfpathlineto{\pgfqpoint{3.500995in}{3.061909in}}%
\pgfpathlineto{\pgfqpoint{3.504001in}{2.830091in}}%
\pgfpathlineto{\pgfqpoint{3.507008in}{2.913273in}}%
\pgfpathlineto{\pgfqpoint{3.509013in}{2.771455in}}%
\pgfpathlineto{\pgfqpoint{3.511017in}{2.876455in}}%
\pgfpathlineto{\pgfqpoint{3.512019in}{2.931000in}}%
\pgfpathlineto{\pgfqpoint{3.514024in}{2.768727in}}%
\pgfpathlineto{\pgfqpoint{3.515026in}{2.902364in}}%
\pgfpathlineto{\pgfqpoint{3.516028in}{2.860091in}}%
\pgfpathlineto{\pgfqpoint{3.518033in}{2.905091in}}%
\pgfpathlineto{\pgfqpoint{3.519035in}{2.790545in}}%
\pgfpathlineto{\pgfqpoint{3.520037in}{2.903727in}}%
\pgfpathlineto{\pgfqpoint{3.521039in}{2.873727in}}%
\pgfpathlineto{\pgfqpoint{3.522042in}{2.898273in}}%
\pgfpathlineto{\pgfqpoint{3.523044in}{3.007364in}}%
\pgfpathlineto{\pgfqpoint{3.524046in}{2.984182in}}%
\pgfpathlineto{\pgfqpoint{3.525048in}{3.010091in}}%
\pgfpathlineto{\pgfqpoint{3.526051in}{2.980091in}}%
\pgfpathlineto{\pgfqpoint{3.527053in}{3.104182in}}%
\pgfpathlineto{\pgfqpoint{3.528055in}{3.524182in}}%
\pgfpathlineto{\pgfqpoint{3.529057in}{3.381000in}}%
\pgfpathlineto{\pgfqpoint{3.530060in}{3.708273in}}%
\pgfpathlineto{\pgfqpoint{3.531062in}{3.156000in}}%
\pgfpathlineto{\pgfqpoint{3.532064in}{3.629182in}}%
\pgfpathlineto{\pgfqpoint{3.533066in}{3.622364in}}%
\pgfpathlineto{\pgfqpoint{3.535071in}{3.948273in}}%
\pgfpathlineto{\pgfqpoint{3.536073in}{3.623727in}}%
\pgfpathlineto{\pgfqpoint{3.540082in}{4.028727in}}%
\pgfpathlineto{\pgfqpoint{3.542087in}{3.522818in}}%
\pgfpathlineto{\pgfqpoint{3.543089in}{3.741000in}}%
\pgfpathlineto{\pgfqpoint{3.544091in}{3.713727in}}%
\pgfpathlineto{\pgfqpoint{3.545093in}{3.859636in}}%
\pgfpathlineto{\pgfqpoint{3.547098in}{3.175091in}}%
\pgfpathlineto{\pgfqpoint{3.549102in}{3.045545in}}%
\pgfpathlineto{\pgfqpoint{3.550104in}{3.255545in}}%
\pgfpathlineto{\pgfqpoint{3.552109in}{3.040091in}}%
\pgfpathlineto{\pgfqpoint{3.554113in}{2.843727in}}%
\pgfpathlineto{\pgfqpoint{3.556118in}{2.970545in}}%
\pgfpathlineto{\pgfqpoint{3.557120in}{2.959636in}}%
\pgfpathlineto{\pgfqpoint{3.558122in}{2.827364in}}%
\pgfpathlineto{\pgfqpoint{3.562131in}{2.903727in}}%
\pgfpathlineto{\pgfqpoint{3.564136in}{2.802818in}}%
\pgfpathlineto{\pgfqpoint{3.565138in}{2.864182in}}%
\pgfpathlineto{\pgfqpoint{3.566140in}{2.839636in}}%
\pgfpathlineto{\pgfqpoint{3.567143in}{2.941909in}}%
\pgfpathlineto{\pgfqpoint{3.569147in}{2.828727in}}%
\pgfpathlineto{\pgfqpoint{3.570149in}{2.910545in}}%
\pgfpathlineto{\pgfqpoint{3.571152in}{2.811000in}}%
\pgfpathlineto{\pgfqpoint{3.572154in}{2.963727in}}%
\pgfpathlineto{\pgfqpoint{3.573156in}{2.876455in}}%
\pgfpathlineto{\pgfqpoint{3.575161in}{2.921455in}}%
\pgfpathlineto{\pgfqpoint{3.576163in}{2.892818in}}%
\pgfpathlineto{\pgfqpoint{3.577165in}{3.034636in}}%
\pgfpathlineto{\pgfqpoint{3.578167in}{2.995091in}}%
\pgfpathlineto{\pgfqpoint{3.579170in}{3.123273in}}%
\pgfpathlineto{\pgfqpoint{3.580172in}{3.119182in}}%
\pgfpathlineto{\pgfqpoint{3.581174in}{3.056455in}}%
\pgfpathlineto{\pgfqpoint{3.583178in}{3.364636in}}%
\pgfpathlineto{\pgfqpoint{3.584181in}{3.777818in}}%
\pgfpathlineto{\pgfqpoint{3.585183in}{3.700091in}}%
\pgfpathlineto{\pgfqpoint{3.586185in}{3.453273in}}%
\pgfpathlineto{\pgfqpoint{3.589192in}{3.931909in}}%
\pgfpathlineto{\pgfqpoint{3.590194in}{3.839182in}}%
\pgfpathlineto{\pgfqpoint{3.591196in}{3.606000in}}%
\pgfpathlineto{\pgfqpoint{3.592199in}{3.859636in}}%
\pgfpathlineto{\pgfqpoint{3.593201in}{3.757364in}}%
\pgfpathlineto{\pgfqpoint{3.594203in}{3.783273in}}%
\pgfpathlineto{\pgfqpoint{3.595205in}{3.732818in}}%
\pgfpathlineto{\pgfqpoint{3.596208in}{3.121909in}}%
\pgfpathlineto{\pgfqpoint{3.597210in}{3.679636in}}%
\pgfpathlineto{\pgfqpoint{3.598212in}{3.217364in}}%
\pgfpathlineto{\pgfqpoint{3.599214in}{3.736909in}}%
\pgfpathlineto{\pgfqpoint{3.601219in}{3.064636in}}%
\pgfpathlineto{\pgfqpoint{3.603223in}{2.933727in}}%
\pgfpathlineto{\pgfqpoint{3.604226in}{3.014182in}}%
\pgfpathlineto{\pgfqpoint{3.606230in}{2.995091in}}%
\pgfpathlineto{\pgfqpoint{3.608235in}{2.832818in}}%
\pgfpathlineto{\pgfqpoint{3.611241in}{2.951455in}}%
\pgfpathlineto{\pgfqpoint{3.612244in}{2.894182in}}%
\pgfpathlineto{\pgfqpoint{3.613246in}{2.771455in}}%
\pgfpathlineto{\pgfqpoint{3.615250in}{2.832818in}}%
\pgfpathlineto{\pgfqpoint{3.617255in}{2.984182in}}%
\pgfpathlineto{\pgfqpoint{3.618257in}{2.768727in}}%
\pgfpathlineto{\pgfqpoint{3.619259in}{2.832818in}}%
\pgfpathlineto{\pgfqpoint{3.620261in}{2.783727in}}%
\pgfpathlineto{\pgfqpoint{3.622266in}{2.989636in}}%
\pgfpathlineto{\pgfqpoint{3.623268in}{2.899636in}}%
\pgfpathlineto{\pgfqpoint{3.624270in}{2.952818in}}%
\pgfpathlineto{\pgfqpoint{3.625273in}{2.824636in}}%
\pgfpathlineto{\pgfqpoint{3.627277in}{2.974636in}}%
\pgfpathlineto{\pgfqpoint{3.628279in}{3.310091in}}%
\pgfpathlineto{\pgfqpoint{3.629282in}{3.225545in}}%
\pgfpathlineto{\pgfqpoint{3.630284in}{3.012818in}}%
\pgfpathlineto{\pgfqpoint{3.632288in}{3.130091in}}%
\pgfpathlineto{\pgfqpoint{3.634293in}{3.866455in}}%
\pgfpathlineto{\pgfqpoint{3.635295in}{3.569182in}}%
\pgfpathlineto{\pgfqpoint{3.636297in}{3.614182in}}%
\pgfpathlineto{\pgfqpoint{3.637300in}{3.150545in}}%
\pgfpathlineto{\pgfqpoint{3.639304in}{3.889636in}}%
\pgfpathlineto{\pgfqpoint{3.642311in}{3.213273in}}%
\pgfpathlineto{\pgfqpoint{3.644315in}{3.810545in}}%
\pgfpathlineto{\pgfqpoint{3.645318in}{3.634636in}}%
\pgfpathlineto{\pgfqpoint{3.646320in}{3.696000in}}%
\pgfpathlineto{\pgfqpoint{3.648324in}{3.160091in}}%
\pgfpathlineto{\pgfqpoint{3.649327in}{3.488727in}}%
\pgfpathlineto{\pgfqpoint{3.650329in}{3.045545in}}%
\pgfpathlineto{\pgfqpoint{3.651331in}{3.134182in}}%
\pgfpathlineto{\pgfqpoint{3.653335in}{2.988273in}}%
\pgfpathlineto{\pgfqpoint{3.654338in}{2.937818in}}%
\pgfpathlineto{\pgfqpoint{3.655340in}{2.951455in}}%
\pgfpathlineto{\pgfqpoint{3.656342in}{3.016909in}}%
\pgfpathlineto{\pgfqpoint{3.658347in}{2.892818in}}%
\pgfpathlineto{\pgfqpoint{3.660351in}{2.811000in}}%
\pgfpathlineto{\pgfqpoint{3.661353in}{2.995091in}}%
\pgfpathlineto{\pgfqpoint{3.663358in}{2.817818in}}%
\pgfpathlineto{\pgfqpoint{3.664360in}{2.826000in}}%
\pgfpathlineto{\pgfqpoint{3.665362in}{2.816455in}}%
\pgfpathlineto{\pgfqpoint{3.666365in}{2.958273in}}%
\pgfpathlineto{\pgfqpoint{3.667367in}{2.811000in}}%
\pgfpathlineto{\pgfqpoint{3.669371in}{2.836909in}}%
\pgfpathlineto{\pgfqpoint{3.670374in}{2.834182in}}%
\pgfpathlineto{\pgfqpoint{3.671376in}{2.895545in}}%
\pgfpathlineto{\pgfqpoint{3.672378in}{2.843727in}}%
\pgfpathlineto{\pgfqpoint{3.674383in}{2.902364in}}%
\pgfpathlineto{\pgfqpoint{3.675385in}{2.868273in}}%
\pgfpathlineto{\pgfqpoint{3.676387in}{2.952818in}}%
\pgfpathlineto{\pgfqpoint{3.677389in}{2.916000in}}%
\pgfpathlineto{\pgfqpoint{3.678392in}{3.068727in}}%
\pgfpathlineto{\pgfqpoint{3.679394in}{3.061909in}}%
\pgfpathlineto{\pgfqpoint{3.680396in}{3.019636in}}%
\pgfpathlineto{\pgfqpoint{3.681398in}{3.094636in}}%
\pgfpathlineto{\pgfqpoint{3.682401in}{3.063273in}}%
\pgfpathlineto{\pgfqpoint{3.683403in}{3.706909in}}%
\pgfpathlineto{\pgfqpoint{3.684405in}{3.698727in}}%
\pgfpathlineto{\pgfqpoint{3.685407in}{3.593727in}}%
\pgfpathlineto{\pgfqpoint{3.686410in}{3.682364in}}%
\pgfpathlineto{\pgfqpoint{3.687412in}{3.344182in}}%
\pgfpathlineto{\pgfqpoint{3.688414in}{3.904636in}}%
\pgfpathlineto{\pgfqpoint{3.691421in}{3.630545in}}%
\pgfpathlineto{\pgfqpoint{3.692423in}{3.087818in}}%
\pgfpathlineto{\pgfqpoint{3.694427in}{3.832364in}}%
\pgfpathlineto{\pgfqpoint{3.695430in}{3.798273in}}%
\pgfpathlineto{\pgfqpoint{3.697434in}{3.419182in}}%
\pgfpathlineto{\pgfqpoint{3.699439in}{3.691909in}}%
\pgfpathlineto{\pgfqpoint{3.702445in}{3.030545in}}%
\pgfpathlineto{\pgfqpoint{3.703448in}{3.124636in}}%
\pgfpathlineto{\pgfqpoint{3.704450in}{3.042818in}}%
\pgfpathlineto{\pgfqpoint{3.705452in}{3.055091in}}%
\pgfpathlineto{\pgfqpoint{3.707457in}{2.884636in}}%
\pgfpathlineto{\pgfqpoint{3.708459in}{2.933727in}}%
\pgfpathlineto{\pgfqpoint{3.709461in}{2.875091in}}%
\pgfpathlineto{\pgfqpoint{3.710463in}{2.944636in}}%
\pgfpathlineto{\pgfqpoint{3.711466in}{2.920091in}}%
\pgfpathlineto{\pgfqpoint{3.712468in}{2.808273in}}%
\pgfpathlineto{\pgfqpoint{3.713470in}{2.873727in}}%
\pgfpathlineto{\pgfqpoint{3.714472in}{2.809636in}}%
\pgfpathlineto{\pgfqpoint{3.716477in}{2.922818in}}%
\pgfpathlineto{\pgfqpoint{3.717479in}{2.786455in}}%
\pgfpathlineto{\pgfqpoint{3.718481in}{2.794636in}}%
\pgfpathlineto{\pgfqpoint{3.719484in}{2.806909in}}%
\pgfpathlineto{\pgfqpoint{3.721488in}{2.944636in}}%
\pgfpathlineto{\pgfqpoint{3.723492in}{2.819182in}}%
\pgfpathlineto{\pgfqpoint{3.724495in}{2.797364in}}%
\pgfpathlineto{\pgfqpoint{3.728504in}{2.971909in}}%
\pgfpathlineto{\pgfqpoint{3.729506in}{2.838273in}}%
\pgfpathlineto{\pgfqpoint{3.731510in}{2.978727in}}%
\pgfpathlineto{\pgfqpoint{3.733515in}{3.595091in}}%
\pgfpathlineto{\pgfqpoint{3.734517in}{3.124636in}}%
\pgfpathlineto{\pgfqpoint{3.735519in}{3.229636in}}%
\pgfpathlineto{\pgfqpoint{3.736522in}{3.027818in}}%
\pgfpathlineto{\pgfqpoint{3.738526in}{3.859636in}}%
\pgfpathlineto{\pgfqpoint{3.739528in}{3.727364in}}%
\pgfpathlineto{\pgfqpoint{3.740531in}{3.760091in}}%
\pgfpathlineto{\pgfqpoint{3.741533in}{3.634636in}}%
\pgfpathlineto{\pgfqpoint{3.742535in}{3.713727in}}%
\pgfpathlineto{\pgfqpoint{3.743537in}{3.931909in}}%
\pgfpathlineto{\pgfqpoint{3.744540in}{3.784636in}}%
\pgfpathlineto{\pgfqpoint{3.745542in}{3.814636in}}%
\pgfpathlineto{\pgfqpoint{3.747546in}{3.607364in}}%
\pgfpathlineto{\pgfqpoint{3.748549in}{3.742364in}}%
\pgfpathlineto{\pgfqpoint{3.750553in}{3.619636in}}%
\pgfpathlineto{\pgfqpoint{3.751555in}{3.486000in}}%
\pgfpathlineto{\pgfqpoint{3.752558in}{3.128727in}}%
\pgfpathlineto{\pgfqpoint{3.753560in}{3.130091in}}%
\pgfpathlineto{\pgfqpoint{3.754562in}{3.034636in}}%
\pgfpathlineto{\pgfqpoint{3.755564in}{3.169636in}}%
\pgfpathlineto{\pgfqpoint{3.756567in}{2.955545in}}%
\pgfpathlineto{\pgfqpoint{3.757569in}{3.025091in}}%
\pgfpathlineto{\pgfqpoint{3.759573in}{2.909182in}}%
\pgfpathlineto{\pgfqpoint{3.760575in}{2.981455in}}%
\pgfpathlineto{\pgfqpoint{3.761578in}{2.903727in}}%
\pgfpathlineto{\pgfqpoint{3.762580in}{2.926909in}}%
\pgfpathlineto{\pgfqpoint{3.764584in}{2.816455in}}%
\pgfpathlineto{\pgfqpoint{3.765587in}{2.914636in}}%
\pgfpathlineto{\pgfqpoint{3.767591in}{2.798727in}}%
\pgfpathlineto{\pgfqpoint{3.768593in}{2.800091in}}%
\pgfpathlineto{\pgfqpoint{3.769596in}{2.749636in}}%
\pgfpathlineto{\pgfqpoint{3.770598in}{2.909182in}}%
\pgfpathlineto{\pgfqpoint{3.772602in}{2.819182in}}%
\pgfpathlineto{\pgfqpoint{3.773605in}{2.816455in}}%
\pgfpathlineto{\pgfqpoint{3.774607in}{2.727818in}}%
\pgfpathlineto{\pgfqpoint{3.775609in}{2.910545in}}%
\pgfpathlineto{\pgfqpoint{3.776611in}{2.905091in}}%
\pgfpathlineto{\pgfqpoint{3.777614in}{2.871000in}}%
\pgfpathlineto{\pgfqpoint{3.778616in}{2.881909in}}%
\pgfpathlineto{\pgfqpoint{3.779618in}{2.875091in}}%
\pgfpathlineto{\pgfqpoint{3.782625in}{3.247364in}}%
\pgfpathlineto{\pgfqpoint{3.783627in}{3.091909in}}%
\pgfpathlineto{\pgfqpoint{3.785632in}{3.243273in}}%
\pgfpathlineto{\pgfqpoint{3.786634in}{3.101455in}}%
\pgfpathlineto{\pgfqpoint{3.788638in}{3.904636in}}%
\pgfpathlineto{\pgfqpoint{3.790643in}{3.772364in}}%
\pgfpathlineto{\pgfqpoint{3.791645in}{3.266455in}}%
\pgfpathlineto{\pgfqpoint{3.793649in}{3.922364in}}%
\pgfpathlineto{\pgfqpoint{3.794652in}{3.761455in}}%
\pgfpathlineto{\pgfqpoint{3.795654in}{3.768273in}}%
\pgfpathlineto{\pgfqpoint{3.796656in}{3.175091in}}%
\pgfpathlineto{\pgfqpoint{3.797658in}{3.705545in}}%
\pgfpathlineto{\pgfqpoint{3.798661in}{3.682364in}}%
\pgfpathlineto{\pgfqpoint{3.799663in}{3.685091in}}%
\pgfpathlineto{\pgfqpoint{3.800665in}{3.701455in}}%
\pgfpathlineto{\pgfqpoint{3.801667in}{3.247364in}}%
\pgfpathlineto{\pgfqpoint{3.802670in}{3.289636in}}%
\pgfpathlineto{\pgfqpoint{3.803672in}{3.113727in}}%
\pgfpathlineto{\pgfqpoint{3.805676in}{3.232364in}}%
\pgfpathlineto{\pgfqpoint{3.807681in}{3.027818in}}%
\pgfpathlineto{\pgfqpoint{3.808683in}{2.926909in}}%
\pgfpathlineto{\pgfqpoint{3.810688in}{2.985545in}}%
\pgfpathlineto{\pgfqpoint{3.812692in}{2.948727in}}%
\pgfpathlineto{\pgfqpoint{3.814697in}{2.856000in}}%
\pgfpathlineto{\pgfqpoint{3.815699in}{2.879182in}}%
\pgfpathlineto{\pgfqpoint{3.816701in}{2.801455in}}%
\pgfpathlineto{\pgfqpoint{3.817703in}{2.933727in}}%
\pgfpathlineto{\pgfqpoint{3.818706in}{2.785091in}}%
\pgfpathlineto{\pgfqpoint{3.819708in}{2.806909in}}%
\pgfpathlineto{\pgfqpoint{3.820710in}{2.879182in}}%
\pgfpathlineto{\pgfqpoint{3.821712in}{2.782364in}}%
\pgfpathlineto{\pgfqpoint{3.822715in}{2.862818in}}%
\pgfpathlineto{\pgfqpoint{3.823717in}{2.736000in}}%
\pgfpathlineto{\pgfqpoint{3.825721in}{2.869636in}}%
\pgfpathlineto{\pgfqpoint{3.826724in}{2.856000in}}%
\pgfpathlineto{\pgfqpoint{3.827726in}{2.879182in}}%
\pgfpathlineto{\pgfqpoint{3.828728in}{2.869636in}}%
\pgfpathlineto{\pgfqpoint{3.829730in}{2.925545in}}%
\pgfpathlineto{\pgfqpoint{3.830732in}{2.917364in}}%
\pgfpathlineto{\pgfqpoint{3.831735in}{2.955545in}}%
\pgfpathlineto{\pgfqpoint{3.832737in}{3.083727in}}%
\pgfpathlineto{\pgfqpoint{3.833739in}{3.033273in}}%
\pgfpathlineto{\pgfqpoint{3.834741in}{3.340091in}}%
\pgfpathlineto{\pgfqpoint{3.835744in}{3.150545in}}%
\pgfpathlineto{\pgfqpoint{3.836746in}{3.259636in}}%
\pgfpathlineto{\pgfqpoint{3.837748in}{3.713727in}}%
\pgfpathlineto{\pgfqpoint{3.838750in}{3.636000in}}%
\pgfpathlineto{\pgfqpoint{3.839753in}{3.754636in}}%
\pgfpathlineto{\pgfqpoint{3.840755in}{3.671455in}}%
\pgfpathlineto{\pgfqpoint{3.841757in}{3.409636in}}%
\pgfpathlineto{\pgfqpoint{3.842759in}{3.855545in}}%
\pgfpathlineto{\pgfqpoint{3.843762in}{3.787364in}}%
\pgfpathlineto{\pgfqpoint{3.844764in}{3.880091in}}%
\pgfpathlineto{\pgfqpoint{3.846768in}{3.360545in}}%
\pgfpathlineto{\pgfqpoint{3.847771in}{3.784636in}}%
\pgfpathlineto{\pgfqpoint{3.848773in}{3.728727in}}%
\pgfpathlineto{\pgfqpoint{3.849775in}{3.783273in}}%
\pgfpathlineto{\pgfqpoint{3.851780in}{3.338727in}}%
\pgfpathlineto{\pgfqpoint{3.852782in}{3.540545in}}%
\pgfpathlineto{\pgfqpoint{3.853784in}{3.491455in}}%
\pgfpathlineto{\pgfqpoint{3.858795in}{2.946000in}}%
\pgfpathlineto{\pgfqpoint{3.859798in}{3.036000in}}%
\pgfpathlineto{\pgfqpoint{3.860800in}{2.956909in}}%
\pgfpathlineto{\pgfqpoint{3.861802in}{2.971909in}}%
\pgfpathlineto{\pgfqpoint{3.863807in}{2.843727in}}%
\pgfpathlineto{\pgfqpoint{3.864809in}{2.903727in}}%
\pgfpathlineto{\pgfqpoint{3.866813in}{2.835545in}}%
\pgfpathlineto{\pgfqpoint{3.867815in}{2.812364in}}%
\pgfpathlineto{\pgfqpoint{3.868818in}{2.731909in}}%
\pgfpathlineto{\pgfqpoint{3.869820in}{2.846455in}}%
\pgfpathlineto{\pgfqpoint{3.870822in}{2.806909in}}%
\pgfpathlineto{\pgfqpoint{3.871824in}{2.845091in}}%
\pgfpathlineto{\pgfqpoint{3.872827in}{2.828727in}}%
\pgfpathlineto{\pgfqpoint{3.873829in}{2.703273in}}%
\pgfpathlineto{\pgfqpoint{3.874831in}{2.787818in}}%
\pgfpathlineto{\pgfqpoint{3.875833in}{2.763273in}}%
\pgfpathlineto{\pgfqpoint{3.876836in}{2.868273in}}%
\pgfpathlineto{\pgfqpoint{3.877838in}{2.841000in}}%
\pgfpathlineto{\pgfqpoint{3.878840in}{2.763273in}}%
\pgfpathlineto{\pgfqpoint{3.882849in}{2.965091in}}%
\pgfpathlineto{\pgfqpoint{3.883851in}{2.886000in}}%
\pgfpathlineto{\pgfqpoint{3.884854in}{3.019636in}}%
\pgfpathlineto{\pgfqpoint{3.885856in}{2.922818in}}%
\pgfpathlineto{\pgfqpoint{3.889865in}{3.652364in}}%
\pgfpathlineto{\pgfqpoint{3.890867in}{3.022364in}}%
\pgfpathlineto{\pgfqpoint{3.892872in}{3.754636in}}%
\pgfpathlineto{\pgfqpoint{3.893874in}{3.837818in}}%
\pgfpathlineto{\pgfqpoint{3.894876in}{3.825545in}}%
\pgfpathlineto{\pgfqpoint{3.895878in}{3.244636in}}%
\pgfpathlineto{\pgfqpoint{3.897883in}{3.754636in}}%
\pgfpathlineto{\pgfqpoint{3.898885in}{3.771000in}}%
\pgfpathlineto{\pgfqpoint{3.899887in}{3.856909in}}%
\pgfpathlineto{\pgfqpoint{3.901892in}{3.416455in}}%
\pgfpathlineto{\pgfqpoint{3.903896in}{3.694636in}}%
\pgfpathlineto{\pgfqpoint{3.904898in}{3.720545in}}%
\pgfpathlineto{\pgfqpoint{3.906903in}{3.117818in}}%
\pgfpathlineto{\pgfqpoint{3.907905in}{3.126000in}}%
\pgfpathlineto{\pgfqpoint{3.908907in}{3.111000in}}%
\pgfpathlineto{\pgfqpoint{3.909910in}{3.307364in}}%
\pgfpathlineto{\pgfqpoint{3.910912in}{3.003273in}}%
\pgfpathlineto{\pgfqpoint{3.911914in}{3.057818in}}%
\pgfpathlineto{\pgfqpoint{3.912916in}{2.914636in}}%
\pgfpathlineto{\pgfqpoint{3.914921in}{2.948727in}}%
\pgfpathlineto{\pgfqpoint{3.915923in}{2.888727in}}%
\pgfpathlineto{\pgfqpoint{3.916925in}{2.906455in}}%
\pgfpathlineto{\pgfqpoint{3.918930in}{2.796000in}}%
\pgfpathlineto{\pgfqpoint{3.919932in}{2.865545in}}%
\pgfpathlineto{\pgfqpoint{3.920934in}{2.838273in}}%
\pgfpathlineto{\pgfqpoint{3.921937in}{2.864182in}}%
\pgfpathlineto{\pgfqpoint{3.923941in}{2.761909in}}%
\pgfpathlineto{\pgfqpoint{3.924943in}{2.834182in}}%
\pgfpathlineto{\pgfqpoint{3.925946in}{2.763273in}}%
\pgfpathlineto{\pgfqpoint{3.926948in}{2.858727in}}%
\pgfpathlineto{\pgfqpoint{3.928952in}{2.768727in}}%
\pgfpathlineto{\pgfqpoint{3.929955in}{2.861455in}}%
\pgfpathlineto{\pgfqpoint{3.930957in}{2.801455in}}%
\pgfpathlineto{\pgfqpoint{3.931959in}{2.920091in}}%
\pgfpathlineto{\pgfqpoint{3.932961in}{2.898273in}}%
\pgfpathlineto{\pgfqpoint{3.934966in}{2.932364in}}%
\pgfpathlineto{\pgfqpoint{3.935968in}{2.921455in}}%
\pgfpathlineto{\pgfqpoint{3.936970in}{3.091909in}}%
\pgfpathlineto{\pgfqpoint{3.937972in}{3.063273in}}%
\pgfpathlineto{\pgfqpoint{3.938975in}{3.416455in}}%
\pgfpathlineto{\pgfqpoint{3.940979in}{3.055091in}}%
\pgfpathlineto{\pgfqpoint{3.942984in}{3.660545in}}%
\pgfpathlineto{\pgfqpoint{3.944988in}{3.777818in}}%
\pgfpathlineto{\pgfqpoint{3.945990in}{3.521455in}}%
\pgfpathlineto{\pgfqpoint{3.947995in}{3.773727in}}%
\pgfpathlineto{\pgfqpoint{3.948997in}{3.843273in}}%
\pgfpathlineto{\pgfqpoint{3.949999in}{3.792818in}}%
\pgfpathlineto{\pgfqpoint{3.951002in}{3.220091in}}%
\pgfpathlineto{\pgfqpoint{3.952004in}{3.700091in}}%
\pgfpathlineto{\pgfqpoint{3.953006in}{3.670091in}}%
\pgfpathlineto{\pgfqpoint{3.954008in}{3.779182in}}%
\pgfpathlineto{\pgfqpoint{3.955011in}{3.607364in}}%
\pgfpathlineto{\pgfqpoint{3.957015in}{3.126000in}}%
\pgfpathlineto{\pgfqpoint{3.958017in}{3.135545in}}%
\pgfpathlineto{\pgfqpoint{3.959020in}{3.535091in}}%
\pgfpathlineto{\pgfqpoint{3.961024in}{3.031909in}}%
\pgfpathlineto{\pgfqpoint{3.962026in}{3.008727in}}%
\pgfpathlineto{\pgfqpoint{3.963029in}{2.901000in}}%
\pgfpathlineto{\pgfqpoint{3.964031in}{2.982818in}}%
\pgfpathlineto{\pgfqpoint{3.967038in}{2.865545in}}%
\pgfpathlineto{\pgfqpoint{3.968040in}{2.763273in}}%
\pgfpathlineto{\pgfqpoint{3.970044in}{2.849182in}}%
\pgfpathlineto{\pgfqpoint{3.971046in}{2.783727in}}%
\pgfpathlineto{\pgfqpoint{3.972049in}{2.816455in}}%
\pgfpathlineto{\pgfqpoint{3.973051in}{2.682818in}}%
\pgfpathlineto{\pgfqpoint{3.975055in}{2.791909in}}%
\pgfpathlineto{\pgfqpoint{3.976058in}{2.811000in}}%
\pgfpathlineto{\pgfqpoint{3.977060in}{2.875091in}}%
\pgfpathlineto{\pgfqpoint{3.978062in}{2.745545in}}%
\pgfpathlineto{\pgfqpoint{3.980067in}{2.816455in}}%
\pgfpathlineto{\pgfqpoint{3.982071in}{2.910545in}}%
\pgfpathlineto{\pgfqpoint{3.983073in}{2.918727in}}%
\pgfpathlineto{\pgfqpoint{3.984076in}{2.943273in}}%
\pgfpathlineto{\pgfqpoint{3.985078in}{2.894182in}}%
\pgfpathlineto{\pgfqpoint{3.986080in}{2.935091in}}%
\pgfpathlineto{\pgfqpoint{3.988085in}{3.106909in}}%
\pgfpathlineto{\pgfqpoint{3.989087in}{3.581455in}}%
\pgfpathlineto{\pgfqpoint{3.990089in}{3.076909in}}%
\pgfpathlineto{\pgfqpoint{3.991091in}{3.097364in}}%
\pgfpathlineto{\pgfqpoint{3.994098in}{3.753273in}}%
\pgfpathlineto{\pgfqpoint{3.995100in}{3.547364in}}%
\pgfpathlineto{\pgfqpoint{3.997105in}{3.741000in}}%
\pgfpathlineto{\pgfqpoint{3.998107in}{3.723273in}}%
\pgfpathlineto{\pgfqpoint{3.999109in}{3.825545in}}%
\pgfpathlineto{\pgfqpoint{4.001114in}{3.558273in}}%
\pgfpathlineto{\pgfqpoint{4.004121in}{3.736909in}}%
\pgfpathlineto{\pgfqpoint{4.007127in}{3.161455in}}%
\pgfpathlineto{\pgfqpoint{4.009132in}{3.405545in}}%
\pgfpathlineto{\pgfqpoint{4.011136in}{3.060545in}}%
\pgfpathlineto{\pgfqpoint{4.013141in}{2.944636in}}%
\pgfpathlineto{\pgfqpoint{4.014143in}{3.025091in}}%
\pgfpathlineto{\pgfqpoint{4.015145in}{2.928273in}}%
\pgfpathlineto{\pgfqpoint{4.016147in}{2.969182in}}%
\pgfpathlineto{\pgfqpoint{4.018152in}{2.772818in}}%
\pgfpathlineto{\pgfqpoint{4.019154in}{2.875091in}}%
\pgfpathlineto{\pgfqpoint{4.020156in}{2.841000in}}%
\pgfpathlineto{\pgfqpoint{4.021159in}{2.922818in}}%
\pgfpathlineto{\pgfqpoint{4.023163in}{2.725091in}}%
\pgfpathlineto{\pgfqpoint{4.024165in}{2.789182in}}%
\pgfpathlineto{\pgfqpoint{4.025168in}{2.771455in}}%
\pgfpathlineto{\pgfqpoint{4.026170in}{2.888727in}}%
\pgfpathlineto{\pgfqpoint{4.028174in}{2.719636in}}%
\pgfpathlineto{\pgfqpoint{4.030179in}{2.767364in}}%
\pgfpathlineto{\pgfqpoint{4.031181in}{2.881909in}}%
\pgfpathlineto{\pgfqpoint{4.032183in}{2.877818in}}%
\pgfpathlineto{\pgfqpoint{4.033186in}{2.835545in}}%
\pgfpathlineto{\pgfqpoint{4.034188in}{2.838273in}}%
\pgfpathlineto{\pgfqpoint{4.035190in}{2.860091in}}%
\pgfpathlineto{\pgfqpoint{4.039199in}{3.127364in}}%
\pgfpathlineto{\pgfqpoint{4.040201in}{2.969182in}}%
\pgfpathlineto{\pgfqpoint{4.041203in}{3.250091in}}%
\pgfpathlineto{\pgfqpoint{4.042206in}{3.083727in}}%
\pgfpathlineto{\pgfqpoint{4.044210in}{3.734182in}}%
\pgfpathlineto{\pgfqpoint{4.045212in}{3.540545in}}%
\pgfpathlineto{\pgfqpoint{4.046215in}{3.661909in}}%
\pgfpathlineto{\pgfqpoint{4.047217in}{3.592364in}}%
\pgfpathlineto{\pgfqpoint{4.049221in}{3.841909in}}%
\pgfpathlineto{\pgfqpoint{4.052228in}{3.619636in}}%
\pgfpathlineto{\pgfqpoint{4.054233in}{3.817364in}}%
\pgfpathlineto{\pgfqpoint{4.057239in}{3.316909in}}%
\pgfpathlineto{\pgfqpoint{4.059244in}{3.649636in}}%
\pgfpathlineto{\pgfqpoint{4.062251in}{2.974636in}}%
\pgfpathlineto{\pgfqpoint{4.064255in}{3.087818in}}%
\pgfpathlineto{\pgfqpoint{4.066260in}{3.022364in}}%
\pgfpathlineto{\pgfqpoint{4.067262in}{2.828727in}}%
\pgfpathlineto{\pgfqpoint{4.069266in}{2.935091in}}%
\pgfpathlineto{\pgfqpoint{4.070269in}{2.935091in}}%
\pgfpathlineto{\pgfqpoint{4.072273in}{2.800091in}}%
\pgfpathlineto{\pgfqpoint{4.074278in}{2.890091in}}%
\pgfpathlineto{\pgfqpoint{4.075280in}{2.857364in}}%
\pgfpathlineto{\pgfqpoint{4.076282in}{2.881909in}}%
\pgfpathlineto{\pgfqpoint{4.077284in}{2.781000in}}%
\pgfpathlineto{\pgfqpoint{4.078286in}{2.798727in}}%
\pgfpathlineto{\pgfqpoint{4.079289in}{2.794636in}}%
\pgfpathlineto{\pgfqpoint{4.080291in}{2.819182in}}%
\pgfpathlineto{\pgfqpoint{4.081293in}{2.946000in}}%
\pgfpathlineto{\pgfqpoint{4.082295in}{2.789182in}}%
\pgfpathlineto{\pgfqpoint{4.084300in}{2.838273in}}%
\pgfpathlineto{\pgfqpoint{4.085302in}{3.081000in}}%
\pgfpathlineto{\pgfqpoint{4.086304in}{2.861455in}}%
\pgfpathlineto{\pgfqpoint{4.087307in}{2.944636in}}%
\pgfpathlineto{\pgfqpoint{4.088309in}{2.823273in}}%
\pgfpathlineto{\pgfqpoint{4.089311in}{3.102818in}}%
\pgfpathlineto{\pgfqpoint{4.090313in}{3.750545in}}%
\pgfpathlineto{\pgfqpoint{4.092318in}{2.967818in}}%
\pgfpathlineto{\pgfqpoint{4.093320in}{3.046909in}}%
\pgfpathlineto{\pgfqpoint{4.095325in}{3.700091in}}%
\pgfpathlineto{\pgfqpoint{4.096327in}{3.664636in}}%
\pgfpathlineto{\pgfqpoint{4.097329in}{3.682364in}}%
\pgfpathlineto{\pgfqpoint{4.098331in}{3.730091in}}%
\pgfpathlineto{\pgfqpoint{4.099334in}{3.604636in}}%
\pgfpathlineto{\pgfqpoint{4.100336in}{3.289636in}}%
\pgfpathlineto{\pgfqpoint{4.101338in}{3.731455in}}%
\pgfpathlineto{\pgfqpoint{4.102340in}{3.698727in}}%
\pgfpathlineto{\pgfqpoint{4.103343in}{3.168273in}}%
\pgfpathlineto{\pgfqpoint{4.104345in}{3.175091in}}%
\pgfpathlineto{\pgfqpoint{4.105347in}{3.150545in}}%
\pgfpathlineto{\pgfqpoint{4.106349in}{3.359182in}}%
\pgfpathlineto{\pgfqpoint{4.107352in}{3.236455in}}%
\pgfpathlineto{\pgfqpoint{4.108354in}{3.281455in}}%
\pgfpathlineto{\pgfqpoint{4.109356in}{3.030545in}}%
\pgfpathlineto{\pgfqpoint{4.110358in}{3.101455in}}%
\pgfpathlineto{\pgfqpoint{4.111361in}{2.969182in}}%
\pgfpathlineto{\pgfqpoint{4.112363in}{2.997818in}}%
\pgfpathlineto{\pgfqpoint{4.114367in}{3.081000in}}%
\pgfpathlineto{\pgfqpoint{4.115369in}{3.081000in}}%
\pgfpathlineto{\pgfqpoint{4.117374in}{2.805545in}}%
\pgfpathlineto{\pgfqpoint{4.120381in}{3.014182in}}%
\pgfpathlineto{\pgfqpoint{4.121383in}{2.928273in}}%
\pgfpathlineto{\pgfqpoint{4.122385in}{2.723727in}}%
\pgfpathlineto{\pgfqpoint{4.124390in}{2.800091in}}%
\pgfpathlineto{\pgfqpoint{4.126394in}{2.971909in}}%
\pgfpathlineto{\pgfqpoint{4.128399in}{2.729182in}}%
\pgfpathlineto{\pgfqpoint{4.131405in}{2.955545in}}%
\pgfpathlineto{\pgfqpoint{4.134412in}{2.768727in}}%
\pgfpathlineto{\pgfqpoint{4.135414in}{3.036000in}}%
\pgfpathlineto{\pgfqpoint{4.136417in}{2.973273in}}%
\pgfpathlineto{\pgfqpoint{4.137419in}{3.031909in}}%
\pgfpathlineto{\pgfqpoint{4.139423in}{2.913273in}}%
\pgfpathlineto{\pgfqpoint{4.140426in}{3.100091in}}%
\pgfpathlineto{\pgfqpoint{4.141428in}{3.034636in}}%
\pgfpathlineto{\pgfqpoint{4.142430in}{3.162818in}}%
\pgfpathlineto{\pgfqpoint{4.144435in}{3.618273in}}%
\pgfpathlineto{\pgfqpoint{4.145437in}{3.663273in}}%
\pgfpathlineto{\pgfqpoint{4.146439in}{3.176455in}}%
\pgfpathlineto{\pgfqpoint{4.148443in}{3.753273in}}%
\pgfpathlineto{\pgfqpoint{4.149446in}{3.743727in}}%
\pgfpathlineto{\pgfqpoint{4.150448in}{3.757364in}}%
\pgfpathlineto{\pgfqpoint{4.151450in}{3.486000in}}%
\pgfpathlineto{\pgfqpoint{4.153455in}{3.799636in}}%
\pgfpathlineto{\pgfqpoint{4.154457in}{3.712364in}}%
\pgfpathlineto{\pgfqpoint{4.155459in}{3.764182in}}%
\pgfpathlineto{\pgfqpoint{4.157464in}{3.368727in}}%
\pgfpathlineto{\pgfqpoint{4.158466in}{3.438273in}}%
\pgfpathlineto{\pgfqpoint{4.159468in}{3.327818in}}%
\pgfpathlineto{\pgfqpoint{4.160470in}{3.599182in}}%
\pgfpathlineto{\pgfqpoint{4.162475in}{3.064636in}}%
\pgfpathlineto{\pgfqpoint{4.163477in}{2.984182in}}%
\pgfpathlineto{\pgfqpoint{4.165482in}{3.106909in}}%
\pgfpathlineto{\pgfqpoint{4.169491in}{2.869636in}}%
\pgfpathlineto{\pgfqpoint{4.170493in}{2.989636in}}%
\pgfpathlineto{\pgfqpoint{4.171495in}{2.909182in}}%
\pgfpathlineto{\pgfqpoint{4.172497in}{2.940545in}}%
\pgfpathlineto{\pgfqpoint{4.174502in}{2.805545in}}%
\pgfpathlineto{\pgfqpoint{4.175504in}{2.898273in}}%
\pgfpathlineto{\pgfqpoint{4.176506in}{2.821909in}}%
\pgfpathlineto{\pgfqpoint{4.177509in}{2.827364in}}%
\pgfpathlineto{\pgfqpoint{4.179513in}{2.764636in}}%
\pgfpathlineto{\pgfqpoint{4.180515in}{2.860091in}}%
\pgfpathlineto{\pgfqpoint{4.181518in}{2.816455in}}%
\pgfpathlineto{\pgfqpoint{4.182520in}{2.841000in}}%
\pgfpathlineto{\pgfqpoint{4.184524in}{2.772818in}}%
\pgfpathlineto{\pgfqpoint{4.186529in}{2.881909in}}%
\pgfpathlineto{\pgfqpoint{4.187531in}{2.866909in}}%
\pgfpathlineto{\pgfqpoint{4.188533in}{2.816455in}}%
\pgfpathlineto{\pgfqpoint{4.189535in}{2.835545in}}%
\pgfpathlineto{\pgfqpoint{4.191540in}{2.965091in}}%
\pgfpathlineto{\pgfqpoint{4.192542in}{3.003273in}}%
\pgfpathlineto{\pgfqpoint{4.194547in}{2.958273in}}%
\pgfpathlineto{\pgfqpoint{4.195549in}{3.090545in}}%
\pgfpathlineto{\pgfqpoint{4.196551in}{3.014182in}}%
\pgfpathlineto{\pgfqpoint{4.197553in}{3.547364in}}%
\pgfpathlineto{\pgfqpoint{4.198556in}{3.378273in}}%
\pgfpathlineto{\pgfqpoint{4.199558in}{3.507818in}}%
\pgfpathlineto{\pgfqpoint{4.200560in}{3.473727in}}%
\pgfpathlineto{\pgfqpoint{4.201562in}{3.124636in}}%
\pgfpathlineto{\pgfqpoint{4.203567in}{3.747818in}}%
\pgfpathlineto{\pgfqpoint{4.205571in}{3.641455in}}%
\pgfpathlineto{\pgfqpoint{4.206574in}{3.203727in}}%
\pgfpathlineto{\pgfqpoint{4.208578in}{3.769636in}}%
\pgfpathlineto{\pgfqpoint{4.210583in}{3.705545in}}%
\pgfpathlineto{\pgfqpoint{4.211585in}{3.454636in}}%
\pgfpathlineto{\pgfqpoint{4.212587in}{3.629182in}}%
\pgfpathlineto{\pgfqpoint{4.213589in}{3.571909in}}%
\pgfpathlineto{\pgfqpoint{4.214592in}{3.608727in}}%
\pgfpathlineto{\pgfqpoint{4.215594in}{3.569182in}}%
\pgfpathlineto{\pgfqpoint{4.217598in}{3.117818in}}%
\pgfpathlineto{\pgfqpoint{4.218600in}{3.086455in}}%
\pgfpathlineto{\pgfqpoint{4.219603in}{3.120545in}}%
\pgfpathlineto{\pgfqpoint{4.220605in}{3.207818in}}%
\pgfpathlineto{\pgfqpoint{4.222609in}{3.001909in}}%
\pgfpathlineto{\pgfqpoint{4.223612in}{2.914636in}}%
\pgfpathlineto{\pgfqpoint{4.224614in}{2.986909in}}%
\pgfpathlineto{\pgfqpoint{4.225616in}{2.974636in}}%
\pgfpathlineto{\pgfqpoint{4.229625in}{2.806909in}}%
\pgfpathlineto{\pgfqpoint{4.230627in}{2.887364in}}%
\pgfpathlineto{\pgfqpoint{4.231630in}{2.881909in}}%
\pgfpathlineto{\pgfqpoint{4.232632in}{2.909182in}}%
\pgfpathlineto{\pgfqpoint{4.234636in}{2.775545in}}%
\pgfpathlineto{\pgfqpoint{4.235639in}{2.836909in}}%
\pgfpathlineto{\pgfqpoint{4.236641in}{2.782364in}}%
\pgfpathlineto{\pgfqpoint{4.237643in}{2.826000in}}%
\pgfpathlineto{\pgfqpoint{4.239648in}{2.782364in}}%
\pgfpathlineto{\pgfqpoint{4.240650in}{2.887364in}}%
\pgfpathlineto{\pgfqpoint{4.241652in}{2.872364in}}%
\pgfpathlineto{\pgfqpoint{4.242654in}{2.877818in}}%
\pgfpathlineto{\pgfqpoint{4.243657in}{2.830091in}}%
\pgfpathlineto{\pgfqpoint{4.245661in}{2.895545in}}%
\pgfpathlineto{\pgfqpoint{4.246663in}{2.933727in}}%
\pgfpathlineto{\pgfqpoint{4.247666in}{3.025091in}}%
\pgfpathlineto{\pgfqpoint{4.248668in}{2.969182in}}%
\pgfpathlineto{\pgfqpoint{4.251675in}{3.057818in}}%
\pgfpathlineto{\pgfqpoint{4.252677in}{3.499636in}}%
\pgfpathlineto{\pgfqpoint{4.253679in}{3.446455in}}%
\pgfpathlineto{\pgfqpoint{4.255683in}{3.222818in}}%
\pgfpathlineto{\pgfqpoint{4.256686in}{3.081000in}}%
\pgfpathlineto{\pgfqpoint{4.258690in}{3.690545in}}%
\pgfpathlineto{\pgfqpoint{4.259692in}{3.664636in}}%
\pgfpathlineto{\pgfqpoint{4.260695in}{3.701455in}}%
\pgfpathlineto{\pgfqpoint{4.261697in}{3.368727in}}%
\pgfpathlineto{\pgfqpoint{4.262699in}{3.735545in}}%
\pgfpathlineto{\pgfqpoint{4.263701in}{3.702818in}}%
\pgfpathlineto{\pgfqpoint{4.265706in}{3.614182in}}%
\pgfpathlineto{\pgfqpoint{4.266708in}{3.396000in}}%
\pgfpathlineto{\pgfqpoint{4.267710in}{3.715091in}}%
\pgfpathlineto{\pgfqpoint{4.268713in}{3.712364in}}%
\pgfpathlineto{\pgfqpoint{4.269715in}{3.675545in}}%
\pgfpathlineto{\pgfqpoint{4.270717in}{3.490091in}}%
\pgfpathlineto{\pgfqpoint{4.271719in}{3.079636in}}%
\pgfpathlineto{\pgfqpoint{4.272722in}{3.169636in}}%
\pgfpathlineto{\pgfqpoint{4.273724in}{3.134182in}}%
\pgfpathlineto{\pgfqpoint{4.274726in}{3.301909in}}%
\pgfpathlineto{\pgfqpoint{4.276731in}{3.019636in}}%
\pgfpathlineto{\pgfqpoint{4.277733in}{3.011455in}}%
\pgfpathlineto{\pgfqpoint{4.278735in}{2.970545in}}%
\pgfpathlineto{\pgfqpoint{4.280740in}{3.006000in}}%
\pgfpathlineto{\pgfqpoint{4.281742in}{2.877818in}}%
\pgfpathlineto{\pgfqpoint{4.282744in}{2.905091in}}%
\pgfpathlineto{\pgfqpoint{4.283746in}{2.819182in}}%
\pgfpathlineto{\pgfqpoint{4.285751in}{2.924182in}}%
\pgfpathlineto{\pgfqpoint{4.286753in}{2.831455in}}%
\pgfpathlineto{\pgfqpoint{4.287755in}{2.877818in}}%
\pgfpathlineto{\pgfqpoint{4.288758in}{2.771455in}}%
\pgfpathlineto{\pgfqpoint{4.290762in}{2.851909in}}%
\pgfpathlineto{\pgfqpoint{4.291764in}{2.862818in}}%
\pgfpathlineto{\pgfqpoint{4.292766in}{2.860091in}}%
\pgfpathlineto{\pgfqpoint{4.293769in}{2.760545in}}%
\pgfpathlineto{\pgfqpoint{4.294771in}{2.782364in}}%
\pgfpathlineto{\pgfqpoint{4.296775in}{2.872364in}}%
\pgfpathlineto{\pgfqpoint{4.297778in}{2.877818in}}%
\pgfpathlineto{\pgfqpoint{4.298780in}{2.805545in}}%
\pgfpathlineto{\pgfqpoint{4.299782in}{2.817818in}}%
\pgfpathlineto{\pgfqpoint{4.302789in}{2.999182in}}%
\pgfpathlineto{\pgfqpoint{4.303791in}{2.920091in}}%
\pgfpathlineto{\pgfqpoint{4.305796in}{2.950091in}}%
\pgfpathlineto{\pgfqpoint{4.306798in}{3.001909in}}%
\pgfpathlineto{\pgfqpoint{4.307800in}{3.498273in}}%
\pgfpathlineto{\pgfqpoint{4.308802in}{3.181909in}}%
\pgfpathlineto{\pgfqpoint{4.309805in}{3.301909in}}%
\pgfpathlineto{\pgfqpoint{4.310807in}{3.075545in}}%
\pgfpathlineto{\pgfqpoint{4.311809in}{3.120545in}}%
\pgfpathlineto{\pgfqpoint{4.312811in}{3.668727in}}%
\pgfpathlineto{\pgfqpoint{4.313814in}{3.615545in}}%
\pgfpathlineto{\pgfqpoint{4.314816in}{3.645545in}}%
\pgfpathlineto{\pgfqpoint{4.316820in}{3.346909in}}%
\pgfpathlineto{\pgfqpoint{4.318825in}{3.711000in}}%
\pgfpathlineto{\pgfqpoint{4.319827in}{3.701455in}}%
\pgfpathlineto{\pgfqpoint{4.321832in}{3.432818in}}%
\pgfpathlineto{\pgfqpoint{4.322834in}{3.697364in}}%
\pgfpathlineto{\pgfqpoint{4.323836in}{3.685091in}}%
\pgfpathlineto{\pgfqpoint{4.324838in}{3.681000in}}%
\pgfpathlineto{\pgfqpoint{4.325840in}{3.521455in}}%
\pgfpathlineto{\pgfqpoint{4.326843in}{3.150545in}}%
\pgfpathlineto{\pgfqpoint{4.327845in}{3.301909in}}%
\pgfpathlineto{\pgfqpoint{4.328847in}{3.246000in}}%
\pgfpathlineto{\pgfqpoint{4.329849in}{3.366000in}}%
\pgfpathlineto{\pgfqpoint{4.331854in}{3.064636in}}%
\pgfpathlineto{\pgfqpoint{4.332856in}{3.056455in}}%
\pgfpathlineto{\pgfqpoint{4.333858in}{2.989636in}}%
\pgfpathlineto{\pgfqpoint{4.334861in}{3.048273in}}%
\pgfpathlineto{\pgfqpoint{4.338870in}{2.866909in}}%
\pgfpathlineto{\pgfqpoint{4.339872in}{2.917364in}}%
\pgfpathlineto{\pgfqpoint{4.340874in}{2.899636in}}%
\pgfpathlineto{\pgfqpoint{4.341876in}{2.907818in}}%
\pgfpathlineto{\pgfqpoint{4.343881in}{2.815091in}}%
\pgfpathlineto{\pgfqpoint{4.344883in}{2.826000in}}%
\pgfpathlineto{\pgfqpoint{4.345885in}{2.864182in}}%
\pgfpathlineto{\pgfqpoint{4.347890in}{2.838273in}}%
\pgfpathlineto{\pgfqpoint{4.348892in}{2.787818in}}%
\pgfpathlineto{\pgfqpoint{4.350897in}{2.890091in}}%
\pgfpathlineto{\pgfqpoint{4.351899in}{2.838273in}}%
\pgfpathlineto{\pgfqpoint{4.352901in}{2.854636in}}%
\pgfpathlineto{\pgfqpoint{4.353903in}{2.789182in}}%
\pgfpathlineto{\pgfqpoint{4.356910in}{2.901000in}}%
\pgfpathlineto{\pgfqpoint{4.357912in}{2.903727in}}%
\pgfpathlineto{\pgfqpoint{4.358915in}{2.841000in}}%
\pgfpathlineto{\pgfqpoint{4.361921in}{3.052364in}}%
\pgfpathlineto{\pgfqpoint{4.362923in}{3.033273in}}%
\pgfpathlineto{\pgfqpoint{4.363926in}{2.980091in}}%
\pgfpathlineto{\pgfqpoint{4.365930in}{3.064636in}}%
\pgfpathlineto{\pgfqpoint{4.366932in}{3.192818in}}%
\pgfpathlineto{\pgfqpoint{4.367935in}{3.589636in}}%
\pgfpathlineto{\pgfqpoint{4.370941in}{3.076909in}}%
\pgfpathlineto{\pgfqpoint{4.373948in}{3.776455in}}%
\pgfpathlineto{\pgfqpoint{4.374950in}{3.762818in}}%
\pgfpathlineto{\pgfqpoint{4.376955in}{3.199636in}}%
\pgfpathlineto{\pgfqpoint{4.378959in}{3.694636in}}%
\pgfpathlineto{\pgfqpoint{4.379962in}{3.720545in}}%
\pgfpathlineto{\pgfqpoint{4.382968in}{3.248727in}}%
\pgfpathlineto{\pgfqpoint{4.383971in}{3.211909in}}%
\pgfpathlineto{\pgfqpoint{4.384973in}{3.312818in}}%
\pgfpathlineto{\pgfqpoint{4.386977in}{3.064636in}}%
\pgfpathlineto{\pgfqpoint{4.388982in}{2.981455in}}%
\pgfpathlineto{\pgfqpoint{4.389984in}{3.045545in}}%
\pgfpathlineto{\pgfqpoint{4.391989in}{2.969182in}}%
\pgfpathlineto{\pgfqpoint{4.392991in}{2.954182in}}%
\pgfpathlineto{\pgfqpoint{4.393993in}{2.903727in}}%
\pgfpathlineto{\pgfqpoint{4.394995in}{2.951455in}}%
\pgfpathlineto{\pgfqpoint{4.395997in}{2.884636in}}%
\pgfpathlineto{\pgfqpoint{4.397000in}{2.898273in}}%
\pgfpathlineto{\pgfqpoint{4.398002in}{2.881909in}}%
\pgfpathlineto{\pgfqpoint{4.399004in}{2.832818in}}%
\pgfpathlineto{\pgfqpoint{4.400006in}{2.913273in}}%
\pgfpathlineto{\pgfqpoint{4.401009in}{2.850545in}}%
\pgfpathlineto{\pgfqpoint{4.402011in}{2.856000in}}%
\pgfpathlineto{\pgfqpoint{4.404015in}{2.791909in}}%
\pgfpathlineto{\pgfqpoint{4.406020in}{2.883273in}}%
\pgfpathlineto{\pgfqpoint{4.407022in}{2.888727in}}%
\pgfpathlineto{\pgfqpoint{4.409027in}{2.823273in}}%
\pgfpathlineto{\pgfqpoint{4.411031in}{2.931000in}}%
\pgfpathlineto{\pgfqpoint{4.412033in}{2.955545in}}%
\pgfpathlineto{\pgfqpoint{4.414038in}{2.910545in}}%
\pgfpathlineto{\pgfqpoint{4.417045in}{3.156000in}}%
\pgfpathlineto{\pgfqpoint{4.418047in}{3.183273in}}%
\pgfpathlineto{\pgfqpoint{4.419049in}{3.090545in}}%
\pgfpathlineto{\pgfqpoint{4.420051in}{3.205091in}}%
\pgfpathlineto{\pgfqpoint{4.421054in}{3.045545in}}%
\pgfpathlineto{\pgfqpoint{4.423058in}{3.634636in}}%
\pgfpathlineto{\pgfqpoint{4.424060in}{3.646909in}}%
\pgfpathlineto{\pgfqpoint{4.425063in}{3.641455in}}%
\pgfpathlineto{\pgfqpoint{4.426065in}{3.158727in}}%
\pgfpathlineto{\pgfqpoint{4.428069in}{3.670091in}}%
\pgfpathlineto{\pgfqpoint{4.429072in}{3.690545in}}%
\pgfpathlineto{\pgfqpoint{4.430074in}{3.668727in}}%
\pgfpathlineto{\pgfqpoint{4.431076in}{3.372818in}}%
\pgfpathlineto{\pgfqpoint{4.432078in}{3.409636in}}%
\pgfpathlineto{\pgfqpoint{4.434083in}{3.648273in}}%
\pgfpathlineto{\pgfqpoint{4.435085in}{3.693273in}}%
\pgfpathlineto{\pgfqpoint{4.437089in}{3.141000in}}%
\pgfpathlineto{\pgfqpoint{4.438092in}{3.136909in}}%
\pgfpathlineto{\pgfqpoint{4.440096in}{3.192818in}}%
\pgfpathlineto{\pgfqpoint{4.441098in}{3.015545in}}%
\pgfpathlineto{\pgfqpoint{4.442101in}{3.036000in}}%
\pgfpathlineto{\pgfqpoint{4.444105in}{2.977364in}}%
\pgfpathlineto{\pgfqpoint{4.445107in}{3.008727in}}%
\pgfpathlineto{\pgfqpoint{4.446110in}{2.948727in}}%
\pgfpathlineto{\pgfqpoint{4.447112in}{2.985545in}}%
\pgfpathlineto{\pgfqpoint{4.449116in}{2.872364in}}%
\pgfpathlineto{\pgfqpoint{4.451121in}{2.909182in}}%
\pgfpathlineto{\pgfqpoint{4.452123in}{2.911909in}}%
\pgfpathlineto{\pgfqpoint{4.454128in}{2.802818in}}%
\pgfpathlineto{\pgfqpoint{4.455130in}{2.886000in}}%
\pgfpathlineto{\pgfqpoint{4.456132in}{2.868273in}}%
\pgfpathlineto{\pgfqpoint{4.457134in}{2.884636in}}%
\pgfpathlineto{\pgfqpoint{4.459139in}{2.802818in}}%
\pgfpathlineto{\pgfqpoint{4.462146in}{2.931000in}}%
\pgfpathlineto{\pgfqpoint{4.464150in}{2.858727in}}%
\pgfpathlineto{\pgfqpoint{4.466154in}{2.921455in}}%
\pgfpathlineto{\pgfqpoint{4.467157in}{3.044182in}}%
\pgfpathlineto{\pgfqpoint{4.469161in}{3.003273in}}%
\pgfpathlineto{\pgfqpoint{4.470163in}{3.011455in}}%
\pgfpathlineto{\pgfqpoint{4.471166in}{3.000545in}}%
\pgfpathlineto{\pgfqpoint{4.474172in}{3.438273in}}%
\pgfpathlineto{\pgfqpoint{4.476177in}{3.071455in}}%
\pgfpathlineto{\pgfqpoint{4.480186in}{3.691909in}}%
\pgfpathlineto{\pgfqpoint{4.481188in}{3.267818in}}%
\pgfpathlineto{\pgfqpoint{4.483193in}{3.611455in}}%
\pgfpathlineto{\pgfqpoint{4.485197in}{3.660545in}}%
\pgfpathlineto{\pgfqpoint{4.486199in}{3.297818in}}%
\pgfpathlineto{\pgfqpoint{4.487202in}{3.442364in}}%
\pgfpathlineto{\pgfqpoint{4.488204in}{3.150545in}}%
\pgfpathlineto{\pgfqpoint{4.489206in}{3.338727in}}%
\pgfpathlineto{\pgfqpoint{4.490208in}{3.277364in}}%
\pgfpathlineto{\pgfqpoint{4.492213in}{3.097364in}}%
\pgfpathlineto{\pgfqpoint{4.493215in}{3.016909in}}%
\pgfpathlineto{\pgfqpoint{4.494217in}{3.042818in}}%
\pgfpathlineto{\pgfqpoint{4.497224in}{2.989636in}}%
\pgfpathlineto{\pgfqpoint{4.498226in}{2.916000in}}%
\pgfpathlineto{\pgfqpoint{4.499229in}{2.932364in}}%
\pgfpathlineto{\pgfqpoint{4.500231in}{2.925545in}}%
\pgfpathlineto{\pgfqpoint{4.501233in}{2.928273in}}%
\pgfpathlineto{\pgfqpoint{4.502235in}{2.913273in}}%
\pgfpathlineto{\pgfqpoint{4.503237in}{2.843727in}}%
\pgfpathlineto{\pgfqpoint{4.504240in}{2.849182in}}%
\pgfpathlineto{\pgfqpoint{4.505242in}{2.892818in}}%
\pgfpathlineto{\pgfqpoint{4.506244in}{2.872364in}}%
\pgfpathlineto{\pgfqpoint{4.507246in}{2.920091in}}%
\pgfpathlineto{\pgfqpoint{4.508249in}{2.812364in}}%
\pgfpathlineto{\pgfqpoint{4.512258in}{2.901000in}}%
\pgfpathlineto{\pgfqpoint{4.513260in}{2.839636in}}%
\pgfpathlineto{\pgfqpoint{4.515264in}{2.879182in}}%
\pgfpathlineto{\pgfqpoint{4.517269in}{2.976000in}}%
\pgfpathlineto{\pgfqpoint{4.518271in}{2.978727in}}%
\pgfpathlineto{\pgfqpoint{4.519273in}{2.985545in}}%
\pgfpathlineto{\pgfqpoint{4.520276in}{2.970545in}}%
\pgfpathlineto{\pgfqpoint{4.522280in}{3.070091in}}%
\pgfpathlineto{\pgfqpoint{4.523282in}{3.053727in}}%
\pgfpathlineto{\pgfqpoint{4.524285in}{3.209182in}}%
\pgfpathlineto{\pgfqpoint{4.525287in}{3.121909in}}%
\pgfpathlineto{\pgfqpoint{4.526289in}{3.132818in}}%
\pgfpathlineto{\pgfqpoint{4.529296in}{3.615545in}}%
\pgfpathlineto{\pgfqpoint{4.531300in}{3.231000in}}%
\pgfpathlineto{\pgfqpoint{4.533305in}{3.551455in}}%
\pgfpathlineto{\pgfqpoint{4.534307in}{3.637364in}}%
\pgfpathlineto{\pgfqpoint{4.536312in}{3.255545in}}%
\pgfpathlineto{\pgfqpoint{4.537314in}{3.295091in}}%
\pgfpathlineto{\pgfqpoint{4.538316in}{3.254182in}}%
\pgfpathlineto{\pgfqpoint{4.539318in}{3.360545in}}%
\pgfpathlineto{\pgfqpoint{4.541323in}{3.094636in}}%
\pgfpathlineto{\pgfqpoint{4.543327in}{3.026455in}}%
\pgfpathlineto{\pgfqpoint{4.544329in}{3.071455in}}%
\pgfpathlineto{\pgfqpoint{4.548338in}{2.909182in}}%
\pgfpathlineto{\pgfqpoint{4.549341in}{2.940545in}}%
\pgfpathlineto{\pgfqpoint{4.550343in}{2.926909in}}%
\pgfpathlineto{\pgfqpoint{4.551345in}{2.956909in}}%
\pgfpathlineto{\pgfqpoint{4.553350in}{2.823273in}}%
\pgfpathlineto{\pgfqpoint{4.554352in}{2.862818in}}%
\pgfpathlineto{\pgfqpoint{4.555354in}{2.853273in}}%
\pgfpathlineto{\pgfqpoint{4.556356in}{2.932364in}}%
\pgfpathlineto{\pgfqpoint{4.558361in}{2.808273in}}%
\pgfpathlineto{\pgfqpoint{4.561368in}{2.914636in}}%
\pgfpathlineto{\pgfqpoint{4.563372in}{2.842364in}}%
\pgfpathlineto{\pgfqpoint{4.567381in}{2.977364in}}%
\pgfpathlineto{\pgfqpoint{4.568383in}{2.963727in}}%
\pgfpathlineto{\pgfqpoint{4.572392in}{3.078273in}}%
\pgfpathlineto{\pgfqpoint{4.574397in}{3.372818in}}%
\pgfpathlineto{\pgfqpoint{4.575399in}{3.075545in}}%
\pgfpathlineto{\pgfqpoint{4.576401in}{3.214636in}}%
\pgfpathlineto{\pgfqpoint{4.577403in}{3.151909in}}%
\pgfpathlineto{\pgfqpoint{4.579408in}{3.599182in}}%
\pgfpathlineto{\pgfqpoint{4.580410in}{3.232364in}}%
\pgfpathlineto{\pgfqpoint{4.581412in}{3.346909in}}%
\pgfpathlineto{\pgfqpoint{4.582415in}{3.232364in}}%
\pgfpathlineto{\pgfqpoint{4.584419in}{3.586909in}}%
\pgfpathlineto{\pgfqpoint{4.587426in}{3.100091in}}%
\pgfpathlineto{\pgfqpoint{4.589430in}{3.292364in}}%
\pgfpathlineto{\pgfqpoint{4.591435in}{3.078273in}}%
\pgfpathlineto{\pgfqpoint{4.592437in}{2.992364in}}%
\pgfpathlineto{\pgfqpoint{4.594442in}{3.006000in}}%
\pgfpathlineto{\pgfqpoint{4.595444in}{2.974636in}}%
\pgfpathlineto{\pgfqpoint{4.596446in}{2.991000in}}%
\pgfpathlineto{\pgfqpoint{4.597448in}{2.921455in}}%
\pgfpathlineto{\pgfqpoint{4.598451in}{2.926909in}}%
\pgfpathlineto{\pgfqpoint{4.600455in}{2.888727in}}%
\pgfpathlineto{\pgfqpoint{4.601457in}{2.941909in}}%
\pgfpathlineto{\pgfqpoint{4.602460in}{2.864182in}}%
\pgfpathlineto{\pgfqpoint{4.604464in}{2.880545in}}%
\pgfpathlineto{\pgfqpoint{4.605466in}{2.846455in}}%
\pgfpathlineto{\pgfqpoint{4.606469in}{2.901000in}}%
\pgfpathlineto{\pgfqpoint{4.607471in}{2.853273in}}%
\pgfpathlineto{\pgfqpoint{4.608473in}{2.864182in}}%
\pgfpathlineto{\pgfqpoint{4.609475in}{2.862818in}}%
\pgfpathlineto{\pgfqpoint{4.610477in}{2.872364in}}%
\pgfpathlineto{\pgfqpoint{4.611480in}{2.910545in}}%
\pgfpathlineto{\pgfqpoint{4.612482in}{2.888727in}}%
\pgfpathlineto{\pgfqpoint{4.613484in}{2.910545in}}%
\pgfpathlineto{\pgfqpoint{4.614486in}{2.891455in}}%
\pgfpathlineto{\pgfqpoint{4.615489in}{2.896909in}}%
\pgfpathlineto{\pgfqpoint{4.618495in}{3.000545in}}%
\pgfpathlineto{\pgfqpoint{4.619498in}{3.000545in}}%
\pgfpathlineto{\pgfqpoint{4.621502in}{3.044182in}}%
\pgfpathlineto{\pgfqpoint{4.622504in}{3.030545in}}%
\pgfpathlineto{\pgfqpoint{4.623507in}{3.175091in}}%
\pgfpathlineto{\pgfqpoint{4.626513in}{3.105545in}}%
\pgfpathlineto{\pgfqpoint{4.627516in}{3.142364in}}%
\pgfpathlineto{\pgfqpoint{4.628518in}{3.555545in}}%
\pgfpathlineto{\pgfqpoint{4.629520in}{3.501000in}}%
\pgfpathlineto{\pgfqpoint{4.631525in}{3.265091in}}%
\pgfpathlineto{\pgfqpoint{4.632527in}{3.263727in}}%
\pgfpathlineto{\pgfqpoint{4.633529in}{3.570545in}}%
\pgfpathlineto{\pgfqpoint{4.634531in}{3.479182in}}%
\pgfpathlineto{\pgfqpoint{4.636536in}{3.169636in}}%
\pgfpathlineto{\pgfqpoint{4.637538in}{3.126000in}}%
\pgfpathlineto{\pgfqpoint{4.638540in}{3.217364in}}%
\pgfpathlineto{\pgfqpoint{4.640545in}{3.066000in}}%
\pgfpathlineto{\pgfqpoint{4.642549in}{2.991000in}}%
\pgfpathlineto{\pgfqpoint{4.643551in}{3.060545in}}%
\pgfpathlineto{\pgfqpoint{4.645556in}{3.001909in}}%
\pgfpathlineto{\pgfqpoint{4.647560in}{2.913273in}}%
\pgfpathlineto{\pgfqpoint{4.648563in}{2.922818in}}%
\pgfpathlineto{\pgfqpoint{4.649565in}{2.903727in}}%
\pgfpathlineto{\pgfqpoint{4.650567in}{2.939182in}}%
\pgfpathlineto{\pgfqpoint{4.652572in}{2.875091in}}%
\pgfpathlineto{\pgfqpoint{4.653574in}{2.875091in}}%
\pgfpathlineto{\pgfqpoint{4.654576in}{2.843727in}}%
\pgfpathlineto{\pgfqpoint{4.656581in}{2.902364in}}%
\pgfpathlineto{\pgfqpoint{4.659587in}{2.854636in}}%
\pgfpathlineto{\pgfqpoint{4.661592in}{2.925545in}}%
\pgfpathlineto{\pgfqpoint{4.662594in}{2.899636in}}%
\pgfpathlineto{\pgfqpoint{4.663596in}{2.921455in}}%
\pgfpathlineto{\pgfqpoint{4.664599in}{2.901000in}}%
\pgfpathlineto{\pgfqpoint{4.666603in}{2.952818in}}%
\pgfpathlineto{\pgfqpoint{4.667605in}{2.954182in}}%
\pgfpathlineto{\pgfqpoint{4.668608in}{2.988273in}}%
\pgfpathlineto{\pgfqpoint{4.669610in}{2.956909in}}%
\pgfpathlineto{\pgfqpoint{4.673619in}{3.158727in}}%
\pgfpathlineto{\pgfqpoint{4.674621in}{3.078273in}}%
\pgfpathlineto{\pgfqpoint{4.675623in}{3.097364in}}%
\pgfpathlineto{\pgfqpoint{4.676626in}{3.090545in}}%
\pgfpathlineto{\pgfqpoint{4.677628in}{3.187364in}}%
\pgfpathlineto{\pgfqpoint{4.678630in}{3.503727in}}%
\pgfpathlineto{\pgfqpoint{4.680634in}{3.191455in}}%
\pgfpathlineto{\pgfqpoint{4.681637in}{3.184636in}}%
\pgfpathlineto{\pgfqpoint{4.682639in}{3.236455in}}%
\pgfpathlineto{\pgfqpoint{4.683641in}{3.533727in}}%
\pgfpathlineto{\pgfqpoint{4.686648in}{3.105545in}}%
\pgfpathlineto{\pgfqpoint{4.687650in}{3.135545in}}%
\pgfpathlineto{\pgfqpoint{4.688652in}{3.211909in}}%
\pgfpathlineto{\pgfqpoint{4.691659in}{3.038727in}}%
\pgfpathlineto{\pgfqpoint{4.692661in}{3.021000in}}%
\pgfpathlineto{\pgfqpoint{4.693664in}{3.059182in}}%
\pgfpathlineto{\pgfqpoint{4.694666in}{3.016909in}}%
\pgfpathlineto{\pgfqpoint{4.695668in}{3.026455in}}%
\pgfpathlineto{\pgfqpoint{4.697673in}{2.950091in}}%
\pgfpathlineto{\pgfqpoint{4.699677in}{2.936455in}}%
\pgfpathlineto{\pgfqpoint{4.700679in}{2.958273in}}%
\pgfpathlineto{\pgfqpoint{4.702684in}{2.894182in}}%
\pgfpathlineto{\pgfqpoint{4.703686in}{2.892818in}}%
\pgfpathlineto{\pgfqpoint{4.704688in}{2.851909in}}%
\pgfpathlineto{\pgfqpoint{4.705691in}{2.920091in}}%
\pgfpathlineto{\pgfqpoint{4.706693in}{2.917364in}}%
\pgfpathlineto{\pgfqpoint{4.709700in}{2.865545in}}%
\pgfpathlineto{\pgfqpoint{4.711704in}{2.933727in}}%
\pgfpathlineto{\pgfqpoint{4.713709in}{2.894182in}}%
\pgfpathlineto{\pgfqpoint{4.714711in}{2.894182in}}%
\pgfpathlineto{\pgfqpoint{4.717717in}{2.982818in}}%
\pgfpathlineto{\pgfqpoint{4.718720in}{2.947364in}}%
\pgfpathlineto{\pgfqpoint{4.719722in}{2.954182in}}%
\pgfpathlineto{\pgfqpoint{4.723731in}{3.078273in}}%
\pgfpathlineto{\pgfqpoint{4.724733in}{3.053727in}}%
\pgfpathlineto{\pgfqpoint{4.725735in}{3.083727in}}%
\pgfpathlineto{\pgfqpoint{4.726738in}{3.051000in}}%
\pgfpathlineto{\pgfqpoint{4.727740in}{3.138273in}}%
\pgfpathlineto{\pgfqpoint{4.728742in}{3.329182in}}%
\pgfpathlineto{\pgfqpoint{4.729744in}{3.218727in}}%
\pgfpathlineto{\pgfqpoint{4.730747in}{3.285545in}}%
\pgfpathlineto{\pgfqpoint{4.731749in}{3.096000in}}%
\pgfpathlineto{\pgfqpoint{4.732751in}{3.179182in}}%
\pgfpathlineto{\pgfqpoint{4.733753in}{3.368727in}}%
\pgfpathlineto{\pgfqpoint{4.736760in}{3.085091in}}%
\pgfpathlineto{\pgfqpoint{4.737762in}{3.116455in}}%
\pgfpathlineto{\pgfqpoint{4.738765in}{3.102818in}}%
\pgfpathlineto{\pgfqpoint{4.740769in}{3.119182in}}%
\pgfpathlineto{\pgfqpoint{4.742774in}{3.027818in}}%
\pgfpathlineto{\pgfqpoint{4.743776in}{2.992364in}}%
\pgfpathlineto{\pgfqpoint{4.745780in}{3.038727in}}%
\pgfpathlineto{\pgfqpoint{4.748787in}{2.925545in}}%
\pgfpathlineto{\pgfqpoint{4.749789in}{2.921455in}}%
\pgfpathlineto{\pgfqpoint{4.750791in}{2.950091in}}%
\pgfpathlineto{\pgfqpoint{4.751794in}{2.940545in}}%
\pgfpathlineto{\pgfqpoint{4.752796in}{2.941909in}}%
\pgfpathlineto{\pgfqpoint{4.754800in}{2.890091in}}%
\pgfpathlineto{\pgfqpoint{4.755803in}{2.931000in}}%
\pgfpathlineto{\pgfqpoint{4.756805in}{2.924182in}}%
\pgfpathlineto{\pgfqpoint{4.757807in}{2.928273in}}%
\pgfpathlineto{\pgfqpoint{4.758809in}{2.884636in}}%
\pgfpathlineto{\pgfqpoint{4.759812in}{2.887364in}}%
\pgfpathlineto{\pgfqpoint{4.760814in}{2.931000in}}%
\pgfpathlineto{\pgfqpoint{4.761816in}{2.918727in}}%
\pgfpathlineto{\pgfqpoint{4.762818in}{2.936455in}}%
\pgfpathlineto{\pgfqpoint{4.763821in}{2.909182in}}%
\pgfpathlineto{\pgfqpoint{4.767830in}{2.973273in}}%
\pgfpathlineto{\pgfqpoint{4.768832in}{2.959636in}}%
\pgfpathlineto{\pgfqpoint{4.772841in}{3.070091in}}%
\pgfpathlineto{\pgfqpoint{4.773843in}{3.026455in}}%
\pgfpathlineto{\pgfqpoint{4.775848in}{3.102818in}}%
\pgfpathlineto{\pgfqpoint{4.776850in}{3.034636in}}%
\pgfpathlineto{\pgfqpoint{4.778854in}{3.168273in}}%
\pgfpathlineto{\pgfqpoint{4.781861in}{3.091909in}}%
\pgfpathlineto{\pgfqpoint{4.783866in}{3.251455in}}%
\pgfpathlineto{\pgfqpoint{4.784868in}{3.207818in}}%
\pgfpathlineto{\pgfqpoint{4.786872in}{3.086455in}}%
\pgfpathlineto{\pgfqpoint{4.787874in}{3.147818in}}%
\pgfpathlineto{\pgfqpoint{4.788877in}{3.091909in}}%
\pgfpathlineto{\pgfqpoint{4.789879in}{3.126000in}}%
\pgfpathlineto{\pgfqpoint{4.791883in}{3.044182in}}%
\pgfpathlineto{\pgfqpoint{4.796895in}{2.981455in}}%
\pgfpathlineto{\pgfqpoint{4.797897in}{2.992364in}}%
\pgfpathlineto{\pgfqpoint{4.798899in}{2.946000in}}%
\pgfpathlineto{\pgfqpoint{4.799901in}{2.988273in}}%
\pgfpathlineto{\pgfqpoint{4.802908in}{2.937818in}}%
\pgfpathlineto{\pgfqpoint{4.803910in}{2.905091in}}%
\pgfpathlineto{\pgfqpoint{4.805915in}{2.932364in}}%
\pgfpathlineto{\pgfqpoint{4.808922in}{2.880545in}}%
\pgfpathlineto{\pgfqpoint{4.809924in}{2.891455in}}%
\pgfpathlineto{\pgfqpoint{4.811928in}{2.950091in}}%
\pgfpathlineto{\pgfqpoint{4.812931in}{2.946000in}}%
\pgfpathlineto{\pgfqpoint{4.813933in}{2.909182in}}%
\pgfpathlineto{\pgfqpoint{4.815937in}{2.946000in}}%
\pgfpathlineto{\pgfqpoint{4.816940in}{2.984182in}}%
\pgfpathlineto{\pgfqpoint{4.818944in}{2.958273in}}%
\pgfpathlineto{\pgfqpoint{4.821951in}{3.014182in}}%
\pgfpathlineto{\pgfqpoint{4.822953in}{3.052364in}}%
\pgfpathlineto{\pgfqpoint{4.823955in}{3.034636in}}%
\pgfpathlineto{\pgfqpoint{4.824957in}{3.051000in}}%
\pgfpathlineto{\pgfqpoint{4.825960in}{3.004636in}}%
\pgfpathlineto{\pgfqpoint{4.826962in}{3.038727in}}%
\pgfpathlineto{\pgfqpoint{4.827964in}{3.134182in}}%
\pgfpathlineto{\pgfqpoint{4.828966in}{3.102818in}}%
\pgfpathlineto{\pgfqpoint{4.829969in}{3.136909in}}%
\pgfpathlineto{\pgfqpoint{4.830971in}{3.037364in}}%
\pgfpathlineto{\pgfqpoint{4.831973in}{3.056455in}}%
\pgfpathlineto{\pgfqpoint{4.834980in}{3.225545in}}%
\pgfpathlineto{\pgfqpoint{4.836984in}{3.078273in}}%
\pgfpathlineto{\pgfqpoint{4.838989in}{3.116455in}}%
\pgfpathlineto{\pgfqpoint{4.839991in}{3.131455in}}%
\pgfpathlineto{\pgfqpoint{4.841996in}{3.037364in}}%
\pgfpathlineto{\pgfqpoint{4.842998in}{3.036000in}}%
\pgfpathlineto{\pgfqpoint{4.844000in}{3.030545in}}%
\pgfpathlineto{\pgfqpoint{4.845002in}{3.041455in}}%
\pgfpathlineto{\pgfqpoint{4.847007in}{3.004636in}}%
\pgfpathlineto{\pgfqpoint{4.848009in}{2.992364in}}%
\pgfpathlineto{\pgfqpoint{4.849011in}{2.962364in}}%
\pgfpathlineto{\pgfqpoint{4.850014in}{2.988273in}}%
\pgfpathlineto{\pgfqpoint{4.851016in}{2.970545in}}%
\pgfpathlineto{\pgfqpoint{4.852018in}{2.992364in}}%
\pgfpathlineto{\pgfqpoint{4.854023in}{2.913273in}}%
\pgfpathlineto{\pgfqpoint{4.857029in}{2.971909in}}%
\pgfpathlineto{\pgfqpoint{4.859034in}{2.901000in}}%
\pgfpathlineto{\pgfqpoint{4.862040in}{2.941909in}}%
\pgfpathlineto{\pgfqpoint{4.863043in}{2.935091in}}%
\pgfpathlineto{\pgfqpoint{4.864045in}{2.907818in}}%
\pgfpathlineto{\pgfqpoint{4.867052in}{2.967818in}}%
\pgfpathlineto{\pgfqpoint{4.869056in}{2.962364in}}%
\pgfpathlineto{\pgfqpoint{4.871061in}{2.982818in}}%
\pgfpathlineto{\pgfqpoint{4.872063in}{3.029182in}}%
\pgfpathlineto{\pgfqpoint{4.873065in}{3.016909in}}%
\pgfpathlineto{\pgfqpoint{4.875070in}{3.045545in}}%
\pgfpathlineto{\pgfqpoint{4.876072in}{3.003273in}}%
\pgfpathlineto{\pgfqpoint{4.878076in}{3.089182in}}%
\pgfpathlineto{\pgfqpoint{4.879079in}{3.089182in}}%
\pgfpathlineto{\pgfqpoint{4.880081in}{3.098727in}}%
\pgfpathlineto{\pgfqpoint{4.881083in}{3.041455in}}%
\pgfpathlineto{\pgfqpoint{4.882085in}{3.059182in}}%
\pgfpathlineto{\pgfqpoint{4.884090in}{3.177818in}}%
\pgfpathlineto{\pgfqpoint{4.886094in}{3.071455in}}%
\pgfpathlineto{\pgfqpoint{4.887097in}{3.074182in}}%
\pgfpathlineto{\pgfqpoint{4.890103in}{3.128727in}}%
\pgfpathlineto{\pgfqpoint{4.892108in}{3.027818in}}%
\pgfpathlineto{\pgfqpoint{4.895114in}{3.056455in}}%
\pgfpathlineto{\pgfqpoint{4.898121in}{2.978727in}}%
\pgfpathlineto{\pgfqpoint{4.899123in}{2.997818in}}%
\pgfpathlineto{\pgfqpoint{4.900126in}{2.996455in}}%
\pgfpathlineto{\pgfqpoint{4.901128in}{2.982818in}}%
\pgfpathlineto{\pgfqpoint{4.902130in}{2.988273in}}%
\pgfpathlineto{\pgfqpoint{4.903132in}{2.947364in}}%
\pgfpathlineto{\pgfqpoint{4.905137in}{2.966455in}}%
\pgfpathlineto{\pgfqpoint{4.906139in}{2.952818in}}%
\pgfpathlineto{\pgfqpoint{4.907141in}{2.969182in}}%
\pgfpathlineto{\pgfqpoint{4.909146in}{2.913273in}}%
\pgfpathlineto{\pgfqpoint{4.912153in}{2.970545in}}%
\pgfpathlineto{\pgfqpoint{4.914157in}{2.932364in}}%
\pgfpathlineto{\pgfqpoint{4.917164in}{2.989636in}}%
\pgfpathlineto{\pgfqpoint{4.918166in}{2.965091in}}%
\pgfpathlineto{\pgfqpoint{4.919168in}{2.988273in}}%
\pgfpathlineto{\pgfqpoint{4.920171in}{2.977364in}}%
\pgfpathlineto{\pgfqpoint{4.921173in}{2.988273in}}%
\pgfpathlineto{\pgfqpoint{4.924180in}{3.060545in}}%
\pgfpathlineto{\pgfqpoint{4.926184in}{3.014182in}}%
\pgfpathlineto{\pgfqpoint{4.929191in}{3.097364in}}%
\pgfpathlineto{\pgfqpoint{4.930193in}{3.082364in}}%
\pgfpathlineto{\pgfqpoint{4.931195in}{3.042818in}}%
\pgfpathlineto{\pgfqpoint{4.934202in}{3.124636in}}%
\pgfpathlineto{\pgfqpoint{4.937209in}{3.056455in}}%
\pgfpathlineto{\pgfqpoint{4.938211in}{3.068727in}}%
\pgfpathlineto{\pgfqpoint{4.939213in}{3.102818in}}%
\pgfpathlineto{\pgfqpoint{4.941218in}{3.033273in}}%
\pgfpathlineto{\pgfqpoint{4.942220in}{3.038727in}}%
\pgfpathlineto{\pgfqpoint{4.943222in}{3.029182in}}%
\pgfpathlineto{\pgfqpoint{4.944224in}{3.057818in}}%
\pgfpathlineto{\pgfqpoint{4.946229in}{3.014182in}}%
\pgfpathlineto{\pgfqpoint{4.948233in}{2.984182in}}%
\pgfpathlineto{\pgfqpoint{4.949236in}{2.995091in}}%
\pgfpathlineto{\pgfqpoint{4.950238in}{2.993727in}}%
\pgfpathlineto{\pgfqpoint{4.952242in}{2.971909in}}%
\pgfpathlineto{\pgfqpoint{4.953245in}{2.939182in}}%
\pgfpathlineto{\pgfqpoint{4.954247in}{2.954182in}}%
\pgfpathlineto{\pgfqpoint{4.955249in}{2.947364in}}%
\pgfpathlineto{\pgfqpoint{4.956251in}{2.969182in}}%
\pgfpathlineto{\pgfqpoint{4.957254in}{2.962364in}}%
\pgfpathlineto{\pgfqpoint{4.958256in}{2.928273in}}%
\pgfpathlineto{\pgfqpoint{4.960260in}{2.937818in}}%
\pgfpathlineto{\pgfqpoint{4.962265in}{2.974636in}}%
\pgfpathlineto{\pgfqpoint{4.964269in}{2.944636in}}%
\pgfpathlineto{\pgfqpoint{4.965271in}{2.962364in}}%
\pgfpathlineto{\pgfqpoint{4.966274in}{3.001909in}}%
\pgfpathlineto{\pgfqpoint{4.968278in}{2.977364in}}%
\pgfpathlineto{\pgfqpoint{4.969280in}{2.989636in}}%
\pgfpathlineto{\pgfqpoint{4.970283in}{2.980091in}}%
\pgfpathlineto{\pgfqpoint{4.973289in}{3.045545in}}%
\pgfpathlineto{\pgfqpoint{4.974292in}{3.041455in}}%
\pgfpathlineto{\pgfqpoint{4.975294in}{3.004636in}}%
\pgfpathlineto{\pgfqpoint{4.976296in}{3.010091in}}%
\pgfpathlineto{\pgfqpoint{4.977298in}{3.026455in}}%
\pgfpathlineto{\pgfqpoint{4.978301in}{3.082364in}}%
\pgfpathlineto{\pgfqpoint{4.979303in}{3.079636in}}%
\pgfpathlineto{\pgfqpoint{4.981307in}{3.012818in}}%
\pgfpathlineto{\pgfqpoint{4.984314in}{3.109636in}}%
\pgfpathlineto{\pgfqpoint{4.986319in}{3.036000in}}%
\pgfpathlineto{\pgfqpoint{4.987321in}{3.036000in}}%
\pgfpathlineto{\pgfqpoint{4.988323in}{3.057818in}}%
\pgfpathlineto{\pgfqpoint{4.989325in}{3.116455in}}%
\pgfpathlineto{\pgfqpoint{4.992332in}{3.006000in}}%
\pgfpathlineto{\pgfqpoint{4.994337in}{3.056455in}}%
\pgfpathlineto{\pgfqpoint{4.997343in}{2.970545in}}%
\pgfpathlineto{\pgfqpoint{4.999348in}{2.980091in}}%
\pgfpathlineto{\pgfqpoint{5.001352in}{3.001909in}}%
\pgfpathlineto{\pgfqpoint{5.003357in}{2.943273in}}%
\pgfpathlineto{\pgfqpoint{5.005361in}{2.970545in}}%
\pgfpathlineto{\pgfqpoint{5.006363in}{3.010091in}}%
\pgfpathlineto{\pgfqpoint{5.008368in}{2.932364in}}%
\pgfpathlineto{\pgfqpoint{5.009370in}{2.939182in}}%
\pgfpathlineto{\pgfqpoint{5.010372in}{2.954182in}}%
\pgfpathlineto{\pgfqpoint{5.011375in}{2.986909in}}%
\pgfpathlineto{\pgfqpoint{5.012377in}{2.977364in}}%
\pgfpathlineto{\pgfqpoint{5.013379in}{2.952818in}}%
\pgfpathlineto{\pgfqpoint{5.014381in}{2.958273in}}%
\pgfpathlineto{\pgfqpoint{5.015384in}{2.954182in}}%
\pgfpathlineto{\pgfqpoint{5.017388in}{2.982818in}}%
\pgfpathlineto{\pgfqpoint{5.018390in}{2.971909in}}%
\pgfpathlineto{\pgfqpoint{5.020395in}{2.986909in}}%
\pgfpathlineto{\pgfqpoint{5.021397in}{2.997818in}}%
\pgfpathlineto{\pgfqpoint{5.022399in}{2.993727in}}%
\pgfpathlineto{\pgfqpoint{5.025406in}{3.022364in}}%
\pgfpathlineto{\pgfqpoint{5.027411in}{3.006000in}}%
\pgfpathlineto{\pgfqpoint{5.030417in}{3.064636in}}%
\pgfpathlineto{\pgfqpoint{5.031420in}{3.040091in}}%
\pgfpathlineto{\pgfqpoint{5.032422in}{2.988273in}}%
\pgfpathlineto{\pgfqpoint{5.034426in}{3.059182in}}%
\pgfpathlineto{\pgfqpoint{5.035428in}{3.066000in}}%
\pgfpathlineto{\pgfqpoint{5.037433in}{3.023727in}}%
\pgfpathlineto{\pgfqpoint{5.039437in}{3.079636in}}%
\pgfpathlineto{\pgfqpoint{5.041442in}{3.046909in}}%
\pgfpathlineto{\pgfqpoint{5.042444in}{3.006000in}}%
\pgfpathlineto{\pgfqpoint{5.043446in}{3.015545in}}%
\pgfpathlineto{\pgfqpoint{5.044449in}{3.042818in}}%
\pgfpathlineto{\pgfqpoint{5.045451in}{3.040091in}}%
\pgfpathlineto{\pgfqpoint{5.046453in}{3.026455in}}%
\pgfpathlineto{\pgfqpoint{5.047455in}{2.982818in}}%
\pgfpathlineto{\pgfqpoint{5.048458in}{2.993727in}}%
\pgfpathlineto{\pgfqpoint{5.049460in}{2.985545in}}%
\pgfpathlineto{\pgfqpoint{5.050462in}{3.022364in}}%
\pgfpathlineto{\pgfqpoint{5.051464in}{3.019636in}}%
\pgfpathlineto{\pgfqpoint{5.053469in}{2.955545in}}%
\pgfpathlineto{\pgfqpoint{5.054471in}{2.961000in}}%
\pgfpathlineto{\pgfqpoint{5.056476in}{3.006000in}}%
\pgfpathlineto{\pgfqpoint{5.058480in}{2.950091in}}%
\pgfpathlineto{\pgfqpoint{5.059482in}{2.955545in}}%
\pgfpathlineto{\pgfqpoint{5.061487in}{3.006000in}}%
\pgfpathlineto{\pgfqpoint{5.063491in}{2.944636in}}%
\pgfpathlineto{\pgfqpoint{5.064494in}{2.952818in}}%
\pgfpathlineto{\pgfqpoint{5.065496in}{2.963727in}}%
\pgfpathlineto{\pgfqpoint{5.066498in}{2.996455in}}%
\pgfpathlineto{\pgfqpoint{5.068502in}{2.973273in}}%
\pgfpathlineto{\pgfqpoint{5.069505in}{2.977364in}}%
\pgfpathlineto{\pgfqpoint{5.070507in}{3.000545in}}%
\pgfpathlineto{\pgfqpoint{5.071509in}{2.980091in}}%
\pgfpathlineto{\pgfqpoint{5.073514in}{3.012818in}}%
\pgfpathlineto{\pgfqpoint{5.074516in}{3.001909in}}%
\pgfpathlineto{\pgfqpoint{5.075518in}{3.036000in}}%
\pgfpathlineto{\pgfqpoint{5.077523in}{3.014182in}}%
\pgfpathlineto{\pgfqpoint{5.078525in}{3.037364in}}%
\pgfpathlineto{\pgfqpoint{5.079527in}{3.033273in}}%
\pgfpathlineto{\pgfqpoint{5.080529in}{3.055091in}}%
\pgfpathlineto{\pgfqpoint{5.082534in}{3.019636in}}%
\pgfpathlineto{\pgfqpoint{5.084538in}{3.067364in}}%
\pgfpathlineto{\pgfqpoint{5.085541in}{3.074182in}}%
\pgfpathlineto{\pgfqpoint{5.087545in}{3.015545in}}%
\pgfpathlineto{\pgfqpoint{5.088547in}{3.053727in}}%
\pgfpathlineto{\pgfqpoint{5.089550in}{3.046909in}}%
\pgfpathlineto{\pgfqpoint{5.090552in}{3.057818in}}%
\pgfpathlineto{\pgfqpoint{5.092556in}{3.011455in}}%
\pgfpathlineto{\pgfqpoint{5.095563in}{3.040091in}}%
\pgfpathlineto{\pgfqpoint{5.098570in}{2.976000in}}%
\pgfpathlineto{\pgfqpoint{5.100574in}{3.012818in}}%
\pgfpathlineto{\pgfqpoint{5.103581in}{2.959636in}}%
\pgfpathlineto{\pgfqpoint{5.104583in}{2.962364in}}%
\pgfpathlineto{\pgfqpoint{5.105585in}{2.992364in}}%
\pgfpathlineto{\pgfqpoint{5.106588in}{2.985545in}}%
\pgfpathlineto{\pgfqpoint{5.109594in}{2.941909in}}%
\pgfpathlineto{\pgfqpoint{5.111599in}{2.993727in}}%
\pgfpathlineto{\pgfqpoint{5.114606in}{2.952818in}}%
\pgfpathlineto{\pgfqpoint{5.116610in}{2.989636in}}%
\pgfpathlineto{\pgfqpoint{5.118615in}{2.978727in}}%
\pgfpathlineto{\pgfqpoint{5.119617in}{2.981455in}}%
\pgfpathlineto{\pgfqpoint{5.121621in}{2.999182in}}%
\pgfpathlineto{\pgfqpoint{5.122624in}{3.010091in}}%
\pgfpathlineto{\pgfqpoint{5.124628in}{3.003273in}}%
\pgfpathlineto{\pgfqpoint{5.125630in}{3.022364in}}%
\pgfpathlineto{\pgfqpoint{5.126633in}{3.012818in}}%
\pgfpathlineto{\pgfqpoint{5.128637in}{3.046909in}}%
\pgfpathlineto{\pgfqpoint{5.130642in}{3.038727in}}%
\pgfpathlineto{\pgfqpoint{5.131644in}{3.004636in}}%
\pgfpathlineto{\pgfqpoint{5.133648in}{3.056455in}}%
\pgfpathlineto{\pgfqpoint{5.134651in}{3.051000in}}%
\pgfpathlineto{\pgfqpoint{5.135653in}{3.053727in}}%
\pgfpathlineto{\pgfqpoint{5.136655in}{3.021000in}}%
\pgfpathlineto{\pgfqpoint{5.137657in}{3.025091in}}%
\pgfpathlineto{\pgfqpoint{5.138660in}{3.053727in}}%
\pgfpathlineto{\pgfqpoint{5.139662in}{3.051000in}}%
\pgfpathlineto{\pgfqpoint{5.140664in}{3.051000in}}%
\pgfpathlineto{\pgfqpoint{5.141666in}{3.006000in}}%
\pgfpathlineto{\pgfqpoint{5.142668in}{3.026455in}}%
\pgfpathlineto{\pgfqpoint{5.143671in}{3.010091in}}%
\pgfpathlineto{\pgfqpoint{5.144673in}{3.042818in}}%
\pgfpathlineto{\pgfqpoint{5.145675in}{3.031909in}}%
\pgfpathlineto{\pgfqpoint{5.147680in}{2.988273in}}%
\pgfpathlineto{\pgfqpoint{5.149684in}{3.007364in}}%
\pgfpathlineto{\pgfqpoint{5.150686in}{3.014182in}}%
\pgfpathlineto{\pgfqpoint{5.151689in}{2.985545in}}%
\pgfpathlineto{\pgfqpoint{5.152691in}{3.001909in}}%
\pgfpathlineto{\pgfqpoint{5.153693in}{2.973273in}}%
\pgfpathlineto{\pgfqpoint{5.154695in}{2.976000in}}%
\pgfpathlineto{\pgfqpoint{5.155698in}{2.993727in}}%
\pgfpathlineto{\pgfqpoint{5.157702in}{2.986909in}}%
\pgfpathlineto{\pgfqpoint{5.159707in}{2.958273in}}%
\pgfpathlineto{\pgfqpoint{5.161711in}{2.993727in}}%
\pgfpathlineto{\pgfqpoint{5.162713in}{2.995091in}}%
\pgfpathlineto{\pgfqpoint{5.164718in}{2.966455in}}%
\pgfpathlineto{\pgfqpoint{5.166722in}{2.988273in}}%
\pgfpathlineto{\pgfqpoint{5.167725in}{2.995091in}}%
\pgfpathlineto{\pgfqpoint{5.169729in}{2.981455in}}%
\pgfpathlineto{\pgfqpoint{5.170731in}{2.982818in}}%
\pgfpathlineto{\pgfqpoint{5.174740in}{3.029182in}}%
\pgfpathlineto{\pgfqpoint{5.175742in}{3.006000in}}%
\pgfpathlineto{\pgfqpoint{5.176745in}{3.007364in}}%
\pgfpathlineto{\pgfqpoint{5.177747in}{3.008727in}}%
\pgfpathlineto{\pgfqpoint{5.178749in}{3.014182in}}%
\pgfpathlineto{\pgfqpoint{5.179751in}{3.036000in}}%
\pgfpathlineto{\pgfqpoint{5.180754in}{3.027818in}}%
\pgfpathlineto{\pgfqpoint{5.181756in}{2.992364in}}%
\pgfpathlineto{\pgfqpoint{5.182758in}{3.026455in}}%
\pgfpathlineto{\pgfqpoint{5.183760in}{3.025091in}}%
\pgfpathlineto{\pgfqpoint{5.185765in}{3.059182in}}%
\pgfpathlineto{\pgfqpoint{5.186767in}{3.015545in}}%
\pgfpathlineto{\pgfqpoint{5.188772in}{3.057818in}}%
\pgfpathlineto{\pgfqpoint{5.189774in}{3.040091in}}%
\pgfpathlineto{\pgfqpoint{5.190776in}{3.057818in}}%
\pgfpathlineto{\pgfqpoint{5.192781in}{3.033273in}}%
\pgfpathlineto{\pgfqpoint{5.193783in}{3.029182in}}%
\pgfpathlineto{\pgfqpoint{5.194785in}{3.049636in}}%
\pgfpathlineto{\pgfqpoint{5.195787in}{3.045545in}}%
\pgfpathlineto{\pgfqpoint{5.196790in}{3.018273in}}%
\pgfpathlineto{\pgfqpoint{5.197792in}{3.027818in}}%
\pgfpathlineto{\pgfqpoint{5.198794in}{3.021000in}}%
\pgfpathlineto{\pgfqpoint{5.199796in}{3.038727in}}%
\pgfpathlineto{\pgfqpoint{5.201801in}{3.010091in}}%
\pgfpathlineto{\pgfqpoint{5.203805in}{2.997818in}}%
\pgfpathlineto{\pgfqpoint{5.204808in}{3.008727in}}%
\pgfpathlineto{\pgfqpoint{5.205810in}{3.007364in}}%
\pgfpathlineto{\pgfqpoint{5.206812in}{2.995091in}}%
\pgfpathlineto{\pgfqpoint{5.207814in}{3.011455in}}%
\pgfpathlineto{\pgfqpoint{5.208817in}{2.986909in}}%
\pgfpathlineto{\pgfqpoint{5.210821in}{2.997818in}}%
\pgfpathlineto{\pgfqpoint{5.211823in}{2.986909in}}%
\pgfpathlineto{\pgfqpoint{5.212825in}{2.997818in}}%
\pgfpathlineto{\pgfqpoint{5.213828in}{2.970545in}}%
\pgfpathlineto{\pgfqpoint{5.215832in}{2.984182in}}%
\pgfpathlineto{\pgfqpoint{5.216834in}{2.981455in}}%
\pgfpathlineto{\pgfqpoint{5.217837in}{2.995091in}}%
\pgfpathlineto{\pgfqpoint{5.218839in}{2.986909in}}%
\pgfpathlineto{\pgfqpoint{5.219841in}{2.988273in}}%
\pgfpathlineto{\pgfqpoint{5.222848in}{2.996455in}}%
\pgfpathlineto{\pgfqpoint{5.223850in}{2.999182in}}%
\pgfpathlineto{\pgfqpoint{5.224852in}{3.015545in}}%
\pgfpathlineto{\pgfqpoint{5.225855in}{2.999182in}}%
\pgfpathlineto{\pgfqpoint{5.227859in}{3.021000in}}%
\pgfpathlineto{\pgfqpoint{5.228861in}{3.018273in}}%
\pgfpathlineto{\pgfqpoint{5.229864in}{3.027818in}}%
\pgfpathlineto{\pgfqpoint{5.230866in}{3.012818in}}%
\pgfpathlineto{\pgfqpoint{5.233873in}{3.037364in}}%
\pgfpathlineto{\pgfqpoint{5.234875in}{3.072818in}}%
\pgfpathlineto{\pgfqpoint{5.235877in}{3.021000in}}%
\pgfpathlineto{\pgfqpoint{5.239886in}{3.052364in}}%
\pgfpathlineto{\pgfqpoint{5.241891in}{3.027818in}}%
\pgfpathlineto{\pgfqpoint{5.243895in}{3.037364in}}%
\pgfpathlineto{\pgfqpoint{5.244897in}{3.038727in}}%
\pgfpathlineto{\pgfqpoint{5.246902in}{3.018273in}}%
\pgfpathlineto{\pgfqpoint{5.247904in}{3.008727in}}%
\pgfpathlineto{\pgfqpoint{5.248906in}{3.012818in}}%
\pgfpathlineto{\pgfqpoint{5.249908in}{3.007364in}}%
\pgfpathlineto{\pgfqpoint{5.250911in}{3.012818in}}%
\pgfpathlineto{\pgfqpoint{5.251913in}{3.011455in}}%
\pgfpathlineto{\pgfqpoint{5.253917in}{2.989636in}}%
\pgfpathlineto{\pgfqpoint{5.254920in}{3.003273in}}%
\pgfpathlineto{\pgfqpoint{5.255922in}{2.996455in}}%
\pgfpathlineto{\pgfqpoint{5.256924in}{3.012818in}}%
\pgfpathlineto{\pgfqpoint{5.258929in}{2.976000in}}%
\pgfpathlineto{\pgfqpoint{5.259931in}{2.993727in}}%
\pgfpathlineto{\pgfqpoint{5.262938in}{2.981455in}}%
\pgfpathlineto{\pgfqpoint{5.263940in}{2.982818in}}%
\pgfpathlineto{\pgfqpoint{5.264942in}{2.981455in}}%
\pgfpathlineto{\pgfqpoint{5.265944in}{2.973273in}}%
\pgfpathlineto{\pgfqpoint{5.267949in}{2.995091in}}%
\pgfpathlineto{\pgfqpoint{5.268951in}{2.989636in}}%
\pgfpathlineto{\pgfqpoint{5.269953in}{3.003273in}}%
\pgfpathlineto{\pgfqpoint{5.270956in}{2.992364in}}%
\pgfpathlineto{\pgfqpoint{5.272960in}{3.006000in}}%
\pgfpathlineto{\pgfqpoint{5.273962in}{3.007364in}}%
\pgfpathlineto{\pgfqpoint{5.274965in}{3.019636in}}%
\pgfpathlineto{\pgfqpoint{5.275967in}{3.010091in}}%
\pgfpathlineto{\pgfqpoint{5.276969in}{3.025091in}}%
\pgfpathlineto{\pgfqpoint{5.277971in}{3.010091in}}%
\pgfpathlineto{\pgfqpoint{5.279976in}{3.034636in}}%
\pgfpathlineto{\pgfqpoint{5.280978in}{3.022364in}}%
\pgfpathlineto{\pgfqpoint{5.281980in}{3.026455in}}%
\pgfpathlineto{\pgfqpoint{5.283985in}{3.044182in}}%
\pgfpathlineto{\pgfqpoint{5.284987in}{3.040091in}}%
\pgfpathlineto{\pgfqpoint{5.285989in}{3.014182in}}%
\pgfpathlineto{\pgfqpoint{5.288996in}{3.036000in}}%
\pgfpathlineto{\pgfqpoint{5.289998in}{3.031909in}}%
\pgfpathlineto{\pgfqpoint{5.291000in}{3.008727in}}%
\pgfpathlineto{\pgfqpoint{5.292003in}{3.012818in}}%
\pgfpathlineto{\pgfqpoint{5.294007in}{3.036000in}}%
\pgfpathlineto{\pgfqpoint{5.297014in}{2.993727in}}%
\pgfpathlineto{\pgfqpoint{5.299018in}{3.019636in}}%
\pgfpathlineto{\pgfqpoint{5.300021in}{3.004636in}}%
\pgfpathlineto{\pgfqpoint{5.301023in}{3.012818in}}%
\pgfpathlineto{\pgfqpoint{5.303027in}{2.991000in}}%
\pgfpathlineto{\pgfqpoint{5.305032in}{3.003273in}}%
\pgfpathlineto{\pgfqpoint{5.306034in}{3.008727in}}%
\pgfpathlineto{\pgfqpoint{5.309041in}{2.982818in}}%
\pgfpathlineto{\pgfqpoint{5.310043in}{2.977364in}}%
\pgfpathlineto{\pgfqpoint{5.312048in}{3.001909in}}%
\pgfpathlineto{\pgfqpoint{5.314052in}{2.974636in}}%
\pgfpathlineto{\pgfqpoint{5.315054in}{2.974636in}}%
\pgfpathlineto{\pgfqpoint{5.317059in}{2.999182in}}%
\pgfpathlineto{\pgfqpoint{5.319063in}{2.986909in}}%
\pgfpathlineto{\pgfqpoint{5.322070in}{3.001909in}}%
\pgfpathlineto{\pgfqpoint{5.323072in}{2.991000in}}%
\pgfpathlineto{\pgfqpoint{5.325077in}{3.034636in}}%
\pgfpathlineto{\pgfqpoint{5.327081in}{2.986909in}}%
\pgfpathlineto{\pgfqpoint{5.328083in}{2.986909in}}%
\pgfpathlineto{\pgfqpoint{5.329086in}{3.036000in}}%
\pgfpathlineto{\pgfqpoint{5.331090in}{3.012818in}}%
\pgfpathlineto{\pgfqpoint{5.332092in}{2.992364in}}%
\pgfpathlineto{\pgfqpoint{5.335099in}{3.041455in}}%
\pgfpathlineto{\pgfqpoint{5.338106in}{3.007364in}}%
\pgfpathlineto{\pgfqpoint{5.341113in}{3.033273in}}%
\pgfpathlineto{\pgfqpoint{5.342115in}{3.004636in}}%
\pgfpathlineto{\pgfqpoint{5.345122in}{3.027818in}}%
\pgfpathlineto{\pgfqpoint{5.347126in}{3.012818in}}%
\pgfpathlineto{\pgfqpoint{5.348128in}{3.015545in}}%
\pgfpathlineto{\pgfqpoint{5.349131in}{3.012818in}}%
\pgfpathlineto{\pgfqpoint{5.350133in}{3.006000in}}%
\pgfpathlineto{\pgfqpoint{5.351135in}{3.010091in}}%
\pgfpathlineto{\pgfqpoint{5.354142in}{2.980091in}}%
\pgfpathlineto{\pgfqpoint{5.356146in}{3.011455in}}%
\pgfpathlineto{\pgfqpoint{5.357148in}{2.991000in}}%
\pgfpathlineto{\pgfqpoint{5.361157in}{3.016909in}}%
\pgfpathlineto{\pgfqpoint{5.362160in}{2.978727in}}%
\pgfpathlineto{\pgfqpoint{5.363162in}{3.022364in}}%
\pgfpathlineto{\pgfqpoint{5.366169in}{2.981455in}}%
\pgfpathlineto{\pgfqpoint{5.367171in}{3.014182in}}%
\pgfpathlineto{\pgfqpoint{5.368173in}{3.011455in}}%
\pgfpathlineto{\pgfqpoint{5.370178in}{2.982818in}}%
\pgfpathlineto{\pgfqpoint{5.371180in}{3.011455in}}%
\pgfpathlineto{\pgfqpoint{5.372182in}{3.000545in}}%
\pgfpathlineto{\pgfqpoint{5.373184in}{3.001909in}}%
\pgfpathlineto{\pgfqpoint{5.375189in}{3.027818in}}%
\pgfpathlineto{\pgfqpoint{5.377193in}{3.001909in}}%
\pgfpathlineto{\pgfqpoint{5.378196in}{3.004636in}}%
\pgfpathlineto{\pgfqpoint{5.379198in}{3.029182in}}%
\pgfpathlineto{\pgfqpoint{5.380200in}{3.027818in}}%
\pgfpathlineto{\pgfqpoint{5.382205in}{2.992364in}}%
\pgfpathlineto{\pgfqpoint{5.384209in}{3.029182in}}%
\pgfpathlineto{\pgfqpoint{5.385211in}{3.034636in}}%
\pgfpathlineto{\pgfqpoint{5.387216in}{3.011455in}}%
\pgfpathlineto{\pgfqpoint{5.388218in}{3.015545in}}%
\pgfpathlineto{\pgfqpoint{5.389220in}{3.034636in}}%
\pgfpathlineto{\pgfqpoint{5.390222in}{3.021000in}}%
\pgfpathlineto{\pgfqpoint{5.391225in}{3.023727in}}%
\pgfpathlineto{\pgfqpoint{5.392227in}{3.006000in}}%
\pgfpathlineto{\pgfqpoint{5.394231in}{3.023727in}}%
\pgfpathlineto{\pgfqpoint{5.395234in}{3.014182in}}%
\pgfpathlineto{\pgfqpoint{5.396236in}{3.016909in}}%
\pgfpathlineto{\pgfqpoint{5.397238in}{3.010091in}}%
\pgfpathlineto{\pgfqpoint{5.398240in}{3.019636in}}%
\pgfpathlineto{\pgfqpoint{5.399243in}{3.012818in}}%
\pgfpathlineto{\pgfqpoint{5.400245in}{3.018273in}}%
\pgfpathlineto{\pgfqpoint{5.401247in}{3.014182in}}%
\pgfpathlineto{\pgfqpoint{5.402249in}{2.999182in}}%
\pgfpathlineto{\pgfqpoint{5.403252in}{3.021000in}}%
\pgfpathlineto{\pgfqpoint{5.404254in}{2.995091in}}%
\pgfpathlineto{\pgfqpoint{5.405256in}{3.007364in}}%
\pgfpathlineto{\pgfqpoint{5.407261in}{3.003273in}}%
\pgfpathlineto{\pgfqpoint{5.408263in}{2.986909in}}%
\pgfpathlineto{\pgfqpoint{5.409265in}{2.993727in}}%
\pgfpathlineto{\pgfqpoint{5.410267in}{2.992364in}}%
\pgfpathlineto{\pgfqpoint{5.411270in}{2.995091in}}%
\pgfpathlineto{\pgfqpoint{5.413274in}{3.012818in}}%
\pgfpathlineto{\pgfqpoint{5.416281in}{2.970545in}}%
\pgfpathlineto{\pgfqpoint{5.417283in}{2.982818in}}%
\pgfpathlineto{\pgfqpoint{5.418285in}{3.018273in}}%
\pgfpathlineto{\pgfqpoint{5.420290in}{2.996455in}}%
\pgfpathlineto{\pgfqpoint{5.421292in}{3.001909in}}%
\pgfpathlineto{\pgfqpoint{5.422294in}{2.986909in}}%
\pgfpathlineto{\pgfqpoint{5.425301in}{3.025091in}}%
\pgfpathlineto{\pgfqpoint{5.426303in}{2.997818in}}%
\pgfpathlineto{\pgfqpoint{5.427305in}{3.001909in}}%
\pgfpathlineto{\pgfqpoint{5.428308in}{3.000545in}}%
\pgfpathlineto{\pgfqpoint{5.429310in}{3.022364in}}%
\pgfpathlineto{\pgfqpoint{5.431314in}{2.996455in}}%
\pgfpathlineto{\pgfqpoint{5.432317in}{2.997818in}}%
\pgfpathlineto{\pgfqpoint{5.433319in}{3.041455in}}%
\pgfpathlineto{\pgfqpoint{5.435323in}{3.004636in}}%
\pgfpathlineto{\pgfqpoint{5.436326in}{3.021000in}}%
\pgfpathlineto{\pgfqpoint{5.437328in}{3.012818in}}%
\pgfpathlineto{\pgfqpoint{5.438330in}{3.025091in}}%
\pgfpathlineto{\pgfqpoint{5.439332in}{3.023727in}}%
\pgfpathlineto{\pgfqpoint{5.440335in}{3.004636in}}%
\pgfpathlineto{\pgfqpoint{5.442339in}{3.023727in}}%
\pgfpathlineto{\pgfqpoint{5.443341in}{3.007364in}}%
\pgfpathlineto{\pgfqpoint{5.444344in}{3.029182in}}%
\pgfpathlineto{\pgfqpoint{5.445346in}{3.025091in}}%
\pgfpathlineto{\pgfqpoint{5.447350in}{3.004636in}}%
\pgfpathlineto{\pgfqpoint{5.449355in}{3.027818in}}%
\pgfpathlineto{\pgfqpoint{5.451359in}{2.999182in}}%
\pgfpathlineto{\pgfqpoint{5.454366in}{3.025091in}}%
\pgfpathlineto{\pgfqpoint{5.455368in}{3.026455in}}%
\pgfpathlineto{\pgfqpoint{5.456371in}{3.010091in}}%
\pgfpathlineto{\pgfqpoint{5.457373in}{3.012818in}}%
\pgfpathlineto{\pgfqpoint{5.458375in}{3.010091in}}%
\pgfpathlineto{\pgfqpoint{5.459377in}{2.997818in}}%
\pgfpathlineto{\pgfqpoint{5.461382in}{3.016909in}}%
\pgfpathlineto{\pgfqpoint{5.463386in}{2.985545in}}%
\pgfpathlineto{\pgfqpoint{5.464388in}{2.988273in}}%
\pgfpathlineto{\pgfqpoint{5.466393in}{3.006000in}}%
\pgfpathlineto{\pgfqpoint{5.467395in}{2.992364in}}%
\pgfpathlineto{\pgfqpoint{5.468397in}{2.993727in}}%
\pgfpathlineto{\pgfqpoint{5.469400in}{2.999182in}}%
\pgfpathlineto{\pgfqpoint{5.470402in}{2.996455in}}%
\pgfpathlineto{\pgfqpoint{5.471404in}{3.003273in}}%
\pgfpathlineto{\pgfqpoint{5.472406in}{2.999182in}}%
\pgfpathlineto{\pgfqpoint{5.473409in}{3.000545in}}%
\pgfpathlineto{\pgfqpoint{5.476415in}{2.985545in}}%
\pgfpathlineto{\pgfqpoint{5.477418in}{3.008727in}}%
\pgfpathlineto{\pgfqpoint{5.479422in}{3.001909in}}%
\pgfpathlineto{\pgfqpoint{5.481427in}{3.014182in}}%
\pgfpathlineto{\pgfqpoint{5.482429in}{2.993727in}}%
\pgfpathlineto{\pgfqpoint{5.485436in}{3.015545in}}%
\pgfpathlineto{\pgfqpoint{5.486438in}{2.991000in}}%
\pgfpathlineto{\pgfqpoint{5.488442in}{3.003273in}}%
\pgfpathlineto{\pgfqpoint{5.489445in}{3.000545in}}%
\pgfpathlineto{\pgfqpoint{5.490447in}{3.042818in}}%
\pgfpathlineto{\pgfqpoint{5.492451in}{3.007364in}}%
\pgfpathlineto{\pgfqpoint{5.493453in}{3.010091in}}%
\pgfpathlineto{\pgfqpoint{5.494456in}{3.008727in}}%
\pgfpathlineto{\pgfqpoint{5.495458in}{3.021000in}}%
\pgfpathlineto{\pgfqpoint{5.496460in}{3.016909in}}%
\pgfpathlineto{\pgfqpoint{5.498465in}{3.006000in}}%
\pgfpathlineto{\pgfqpoint{5.499467in}{3.018273in}}%
\pgfpathlineto{\pgfqpoint{5.500469in}{3.007364in}}%
\pgfpathlineto{\pgfqpoint{5.501471in}{3.021000in}}%
\pgfpathlineto{\pgfqpoint{5.502474in}{3.014182in}}%
\pgfpathlineto{\pgfqpoint{5.504478in}{2.988273in}}%
\pgfpathlineto{\pgfqpoint{5.506483in}{3.010091in}}%
\pgfpathlineto{\pgfqpoint{5.507485in}{3.010091in}}%
\pgfpathlineto{\pgfqpoint{5.508487in}{2.989636in}}%
\pgfpathlineto{\pgfqpoint{5.510492in}{3.006000in}}%
\pgfpathlineto{\pgfqpoint{5.512496in}{2.989636in}}%
\pgfpathlineto{\pgfqpoint{5.514501in}{3.001909in}}%
\pgfpathlineto{\pgfqpoint{5.515503in}{3.001909in}}%
\pgfpathlineto{\pgfqpoint{5.516505in}{2.981455in}}%
\pgfpathlineto{\pgfqpoint{5.517507in}{3.004636in}}%
\pgfpathlineto{\pgfqpoint{5.518510in}{2.997818in}}%
\pgfpathlineto{\pgfqpoint{5.519512in}{2.999182in}}%
\pgfpathlineto{\pgfqpoint{5.522519in}{3.010091in}}%
\pgfpathlineto{\pgfqpoint{5.523521in}{3.014182in}}%
\pgfpathlineto{\pgfqpoint{5.524523in}{3.006000in}}%
\pgfpathlineto{\pgfqpoint{5.525525in}{3.016909in}}%
\pgfpathlineto{\pgfqpoint{5.526528in}{3.000545in}}%
\pgfpathlineto{\pgfqpoint{5.527530in}{3.007364in}}%
\pgfpathlineto{\pgfqpoint{5.528532in}{2.999182in}}%
\pgfpathlineto{\pgfqpoint{5.530536in}{3.008727in}}%
\pgfpathlineto{\pgfqpoint{5.532541in}{3.023727in}}%
\pgfpathlineto{\pgfqpoint{5.533543in}{3.010091in}}%
\pgfpathlineto{\pgfqpoint{5.534545in}{3.016909in}}%
\pgfpathlineto{\pgfqpoint{5.534545in}{3.016909in}}%
\pgfusepath{stroke}%
\end{pgfscope}%
\begin{pgfscope}%
\pgfpathrectangle{\pgfqpoint{0.800000in}{0.528000in}}{\pgfqpoint{4.960000in}{3.696000in}}%
\pgfusepath{clip}%
\pgfsetrectcap%
\pgfsetroundjoin%
\pgfsetlinewidth{1.505625pt}%
\definecolor{currentstroke}{rgb}{0.843137,0.509804,0.494118}%
\pgfsetstrokecolor{currentstroke}%
\pgfsetdash{}{0pt}%
\pgfpathmoveto{\pgfqpoint{1.025455in}{2.970953in}}%
\pgfpathlineto{\pgfqpoint{1.119665in}{2.981093in}}%
\pgfpathlineto{\pgfqpoint{1.206861in}{2.992705in}}%
\pgfpathlineto{\pgfqpoint{1.288042in}{3.005745in}}%
\pgfpathlineto{\pgfqpoint{1.364213in}{3.020194in}}%
\pgfpathlineto{\pgfqpoint{1.437376in}{3.036316in}}%
\pgfpathlineto{\pgfqpoint{1.507533in}{3.054034in}}%
\pgfpathlineto{\pgfqpoint{1.575686in}{3.073528in}}%
\pgfpathlineto{\pgfqpoint{1.641834in}{3.094739in}}%
\pgfpathlineto{\pgfqpoint{1.706980in}{3.117946in}}%
\pgfpathlineto{\pgfqpoint{1.771123in}{3.143131in}}%
\pgfpathlineto{\pgfqpoint{1.835267in}{3.170693in}}%
\pgfpathlineto{\pgfqpoint{1.899410in}{3.200664in}}%
\pgfpathlineto{\pgfqpoint{1.964556in}{3.233566in}}%
\pgfpathlineto{\pgfqpoint{2.030704in}{3.269472in}}%
\pgfpathlineto{\pgfqpoint{2.098857in}{3.309003in}}%
\pgfpathlineto{\pgfqpoint{2.170016in}{3.352856in}}%
\pgfpathlineto{\pgfqpoint{2.246186in}{3.402440in}}%
\pgfpathlineto{\pgfqpoint{2.330375in}{3.459947in}}%
\pgfpathlineto{\pgfqpoint{2.431601in}{3.531918in}}%
\pgfpathlineto{\pgfqpoint{2.773366in}{3.777370in}}%
\pgfpathlineto{\pgfqpoint{2.846530in}{3.825685in}}%
\pgfpathlineto{\pgfqpoint{2.910673in}{3.865449in}}%
\pgfpathlineto{\pgfqpoint{2.968804in}{3.898943in}}%
\pgfpathlineto{\pgfqpoint{3.022925in}{3.927627in}}%
\pgfpathlineto{\pgfqpoint{3.074039in}{3.952248in}}%
\pgfpathlineto{\pgfqpoint{3.123149in}{3.973440in}}%
\pgfpathlineto{\pgfqpoint{3.170254in}{3.991336in}}%
\pgfpathlineto{\pgfqpoint{3.215355in}{4.006111in}}%
\pgfpathlineto{\pgfqpoint{3.259454in}{4.018224in}}%
\pgfpathlineto{\pgfqpoint{3.302550in}{4.027753in}}%
\pgfpathlineto{\pgfqpoint{3.344645in}{4.034797in}}%
\pgfpathlineto{\pgfqpoint{3.386739in}{4.039559in}}%
\pgfpathlineto{\pgfqpoint{3.427831in}{4.041977in}}%
\pgfpathlineto{\pgfqpoint{3.468923in}{4.042177in}}%
\pgfpathlineto{\pgfqpoint{3.510015in}{4.040159in}}%
\pgfpathlineto{\pgfqpoint{3.551107in}{4.035933in}}%
\pgfpathlineto{\pgfqpoint{3.593201in}{4.029343in}}%
\pgfpathlineto{\pgfqpoint{3.635295in}{4.020505in}}%
\pgfpathlineto{\pgfqpoint{3.678392in}{4.009186in}}%
\pgfpathlineto{\pgfqpoint{3.722490in}{3.995301in}}%
\pgfpathlineto{\pgfqpoint{3.767591in}{3.978791in}}%
\pgfpathlineto{\pgfqpoint{3.814697in}{3.959177in}}%
\pgfpathlineto{\pgfqpoint{3.863807in}{3.936308in}}%
\pgfpathlineto{\pgfqpoint{3.915923in}{3.909541in}}%
\pgfpathlineto{\pgfqpoint{3.971046in}{3.878692in}}%
\pgfpathlineto{\pgfqpoint{4.031181in}{3.842414in}}%
\pgfpathlineto{\pgfqpoint{4.097329in}{3.799822in}}%
\pgfpathlineto{\pgfqpoint{4.172497in}{3.748706in}}%
\pgfpathlineto{\pgfqpoint{4.266708in}{3.681796in}}%
\pgfpathlineto{\pgfqpoint{4.620500in}{3.427728in}}%
\pgfpathlineto{\pgfqpoint{4.702684in}{3.373031in}}%
\pgfpathlineto{\pgfqpoint{4.777852in}{3.325626in}}%
\pgfpathlineto{\pgfqpoint{4.849011in}{3.283367in}}%
\pgfpathlineto{\pgfqpoint{4.917164in}{3.245474in}}%
\pgfpathlineto{\pgfqpoint{4.983312in}{3.211223in}}%
\pgfpathlineto{\pgfqpoint{5.048458in}{3.179984in}}%
\pgfpathlineto{\pgfqpoint{5.112601in}{3.151656in}}%
\pgfpathlineto{\pgfqpoint{5.176745in}{3.125719in}}%
\pgfpathlineto{\pgfqpoint{5.241891in}{3.101770in}}%
\pgfpathlineto{\pgfqpoint{5.308039in}{3.079838in}}%
\pgfpathlineto{\pgfqpoint{5.375189in}{3.059919in}}%
\pgfpathlineto{\pgfqpoint{5.444344in}{3.041737in}}%
\pgfpathlineto{\pgfqpoint{5.515503in}{3.025328in}}%
\pgfpathlineto{\pgfqpoint{5.534545in}{3.021309in}}%
\pgfpathlineto{\pgfqpoint{5.534545in}{3.021309in}}%
\pgfusepath{stroke}%
\end{pgfscope}%
\begin{pgfscope}%
\pgfsetbuttcap%
\pgfsetmiterjoin%
\definecolor{currentfill}{rgb}{0.000000,0.000000,0.000000}%
\pgfsetfillcolor{currentfill}%
\pgfsetlinewidth{1.003750pt}%
\definecolor{currentstroke}{rgb}{0.000000,0.000000,0.000000}%
\pgfsetstrokecolor{currentstroke}%
\pgfsetdash{}{0pt}%
\pgfsys@defobject{currentmarker}{\pgfqpoint{-0.069444in}{-0.069444in}}{\pgfqpoint{0.069444in}{0.069444in}}{%
\pgfpathmoveto{\pgfqpoint{0.069444in}{-0.000000in}}%
\pgfpathlineto{\pgfqpoint{-0.069444in}{0.069444in}}%
\pgfpathlineto{\pgfqpoint{-0.069444in}{-0.069444in}}%
\pgfpathlineto{\pgfqpoint{0.069444in}{-0.000000in}}%
\pgfpathclose%
\pgfusepath{stroke,fill}%
}%
\begin{pgfscope}%
\pgfsys@transformshift{5.760000in}{0.696000in}%
\pgfsys@useobject{currentmarker}{}%
\end{pgfscope}%
\end{pgfscope}%
\begin{pgfscope}%
\pgfsetbuttcap%
\pgfsetmiterjoin%
\definecolor{currentfill}{rgb}{0.000000,0.000000,0.000000}%
\pgfsetfillcolor{currentfill}%
\pgfsetlinewidth{1.003750pt}%
\definecolor{currentstroke}{rgb}{0.000000,0.000000,0.000000}%
\pgfsetstrokecolor{currentstroke}%
\pgfsetdash{}{0pt}%
\pgfsys@defobject{currentmarker}{\pgfqpoint{-0.069444in}{-0.069444in}}{\pgfqpoint{0.069444in}{0.069444in}}{%
\pgfpathmoveto{\pgfqpoint{0.000000in}{0.069444in}}%
\pgfpathlineto{\pgfqpoint{-0.069444in}{-0.069444in}}%
\pgfpathlineto{\pgfqpoint{0.069444in}{-0.069444in}}%
\pgfpathlineto{\pgfqpoint{0.000000in}{0.069444in}}%
\pgfpathclose%
\pgfusepath{stroke,fill}%
}%
\begin{pgfscope}%
\pgfsys@transformshift{1.025455in}{4.224000in}%
\pgfsys@useobject{currentmarker}{}%
\end{pgfscope}%
\end{pgfscope}%
\begin{pgfscope}%
\pgfsetrectcap%
\pgfsetmiterjoin%
\pgfsetlinewidth{0.803000pt}%
\definecolor{currentstroke}{rgb}{0.000000,0.000000,0.000000}%
\pgfsetstrokecolor{currentstroke}%
\pgfsetdash{}{0pt}%
\pgfpathmoveto{\pgfqpoint{1.025455in}{0.528000in}}%
\pgfpathlineto{\pgfqpoint{1.025455in}{4.224000in}}%
\pgfusepath{stroke}%
\end{pgfscope}%
\begin{pgfscope}%
\pgfsetrectcap%
\pgfsetmiterjoin%
\pgfsetlinewidth{0.803000pt}%
\definecolor{currentstroke}{rgb}{0.000000,0.000000,0.000000}%
\pgfsetstrokecolor{currentstroke}%
\pgfsetdash{}{0pt}%
\pgfpathmoveto{\pgfqpoint{0.800000in}{0.696000in}}%
\pgfpathlineto{\pgfqpoint{5.760000in}{0.696000in}}%
\pgfusepath{stroke}%
\end{pgfscope}%
\begin{pgfscope}%
\pgfsetbuttcap%
\pgfsetmiterjoin%
\definecolor{currentfill}{rgb}{1.000000,1.000000,1.000000}%
\pgfsetfillcolor{currentfill}%
\pgfsetfillopacity{0.800000}%
\pgfsetlinewidth{1.003750pt}%
\definecolor{currentstroke}{rgb}{0.800000,0.800000,0.800000}%
\pgfsetstrokecolor{currentstroke}%
\pgfsetstrokeopacity{0.800000}%
\pgfsetdash{}{0pt}%
\pgfpathmoveto{\pgfqpoint{4.781218in}{3.487889in}}%
\pgfpathlineto{\pgfqpoint{5.662778in}{3.487889in}}%
\pgfpathquadraticcurveto{\pgfqpoint{5.690556in}{3.487889in}}{\pgfqpoint{5.690556in}{3.515667in}}%
\pgfpathlineto{\pgfqpoint{5.690556in}{4.126778in}}%
\pgfpathquadraticcurveto{\pgfqpoint{5.690556in}{4.154556in}}{\pgfqpoint{5.662778in}{4.154556in}}%
\pgfpathlineto{\pgfqpoint{4.781218in}{4.154556in}}%
\pgfpathquadraticcurveto{\pgfqpoint{4.753440in}{4.154556in}}{\pgfqpoint{4.753440in}{4.126778in}}%
\pgfpathlineto{\pgfqpoint{4.753440in}{3.515667in}}%
\pgfpathquadraticcurveto{\pgfqpoint{4.753440in}{3.487889in}}{\pgfqpoint{4.781218in}{3.487889in}}%
\pgfpathlineto{\pgfqpoint{4.781218in}{3.487889in}}%
\pgfpathclose%
\pgfusepath{stroke,fill}%
\end{pgfscope}%
\begin{pgfscope}%
\pgfsetrectcap%
\pgfsetroundjoin%
\pgfsetlinewidth{1.505625pt}%
\definecolor{currentstroke}{rgb}{0.564706,0.478431,0.662745}%
\pgfsetstrokecolor{currentstroke}%
\pgfsetdash{}{0pt}%
\pgfpathmoveto{\pgfqpoint{4.808996in}{4.043444in}}%
\pgfpathlineto{\pgfqpoint{4.947884in}{4.043444in}}%
\pgfpathlineto{\pgfqpoint{5.086773in}{4.043444in}}%
\pgfusepath{stroke}%
\end{pgfscope}%
\begin{pgfscope}%
\definecolor{textcolor}{rgb}{0.000000,0.000000,0.000000}%
\pgfsetstrokecolor{textcolor}%
\pgfsetfillcolor{textcolor}%
\pgftext[x=5.197884in,y=3.994833in,left,base]{\color{textcolor}\rmfamily\fontsize{10.000000}{12.000000}\selectfont \(\displaystyle I_{\mathrm{mes}}(t)\)}%
\end{pgfscope}%
\begin{pgfscope}%
\pgfsetrectcap%
\pgfsetroundjoin%
\pgfsetlinewidth{1.505625pt}%
\definecolor{currentstroke}{rgb}{0.564706,0.478431,0.662745}%
\pgfsetstrokecolor{currentstroke}%
\pgfsetdash{}{0pt}%
\pgfpathmoveto{\pgfqpoint{4.808996in}{3.835111in}}%
\pgfpathlineto{\pgfqpoint{4.947884in}{3.835111in}}%
\pgfpathlineto{\pgfqpoint{5.086773in}{3.835111in}}%
\pgfusepath{stroke}%
\end{pgfscope}%
\begin{pgfscope}%
\definecolor{textcolor}{rgb}{0.000000,0.000000,0.000000}%
\pgfsetstrokecolor{textcolor}%
\pgfsetfillcolor{textcolor}%
\pgftext[x=5.197884in,y=3.786500in,left,base]{\color{textcolor}\rmfamily\fontsize{10.000000}{12.000000}\selectfont \(\displaystyle I_{\mathrm{mes}}(t)\)}%
\end{pgfscope}%
\begin{pgfscope}%
\pgfsetrectcap%
\pgfsetroundjoin%
\pgfsetlinewidth{1.505625pt}%
\definecolor{currentstroke}{rgb}{0.843137,0.509804,0.494118}%
\pgfsetstrokecolor{currentstroke}%
\pgfsetdash{}{0pt}%
\pgfpathmoveto{\pgfqpoint{4.808996in}{3.626778in}}%
\pgfpathlineto{\pgfqpoint{4.947884in}{3.626778in}}%
\pgfpathlineto{\pgfqpoint{5.086773in}{3.626778in}}%
\pgfusepath{stroke}%
\end{pgfscope}%
\begin{pgfscope}%
\definecolor{textcolor}{rgb}{0.000000,0.000000,0.000000}%
\pgfsetstrokecolor{textcolor}%
\pgfsetfillcolor{textcolor}%
\pgftext[x=5.197884in,y=3.578167in,left,base]{\color{textcolor}\rmfamily\fontsize{10.000000}{12.000000}\selectfont \(\displaystyle I_{\mathrm{mod}}(t)\)}%
\end{pgfscope}%
\end{pgfpicture}%
\makeatother%
\endgroup%
}
		\caption{Modélisation par un filtre à profil spectral gaussien}
	\end{figure}

	Pour la modélisation filtre à profil spectral rectangulaire, l'enveloppe haute est de la forme \[
		I_\mathrm{haut} = M + B \cdot \operatorname{sinc}(2\pi\,v\,(t-t_0)\,\mathrm{\Delta}\sigma)
	.\] Après régression, on trouve les valeurs $M \simeq 1\,647$, $B\simeq 799$, $\mathrm{\Delta}\sigma \simeq 49\,273\:\mathrm{m}^{-1}$ et $t_0 \simeq 23{,}598$.
	On représente la modélisation sur la figure ci-dessous.

	\begin{figure}[H]
		\centering
		\resizebox{\linewidth}{!}{%% Creator: Matplotlib, PGF backend
%%
%% To include the figure in your LaTeX document, write
%%   \input{<filename>.pgf}
%%
%% Make sure the required packages are loaded in your preamble
%%   \usepackage{pgf}
%%
%% Also ensure that all the required font packages are loaded; for instance,
%% the lmodern package is sometimes necessary when using math font.
%%   \usepackage{lmodern}
%%
%% Figures using additional raster images can only be included by \input if
%% they are in the same directory as the main LaTeX file. For loading figures
%% from other directories you can use the `import` package
%%   \usepackage{import}
%%
%% and then include the figures with
%%   \import{<path to file>}{<filename>.pgf}
%%
%% Matplotlib used the following preamble
%%   
%%   \makeatletter\@ifpackageloaded{underscore}{}{\usepackage[strings]{underscore}}\makeatother
%%
\begingroup%
\makeatletter%
\begin{pgfpicture}%
\pgfpathrectangle{\pgfpointorigin}{\pgfqpoint{6.400000in}{4.800000in}}%
\pgfusepath{use as bounding box, clip}%
\begin{pgfscope}%
\pgfsetbuttcap%
\pgfsetmiterjoin%
\definecolor{currentfill}{rgb}{1.000000,1.000000,1.000000}%
\pgfsetfillcolor{currentfill}%
\pgfsetlinewidth{0.000000pt}%
\definecolor{currentstroke}{rgb}{1.000000,1.000000,1.000000}%
\pgfsetstrokecolor{currentstroke}%
\pgfsetdash{}{0pt}%
\pgfpathmoveto{\pgfqpoint{0.000000in}{0.000000in}}%
\pgfpathlineto{\pgfqpoint{6.400000in}{0.000000in}}%
\pgfpathlineto{\pgfqpoint{6.400000in}{4.800000in}}%
\pgfpathlineto{\pgfqpoint{0.000000in}{4.800000in}}%
\pgfpathlineto{\pgfqpoint{0.000000in}{0.000000in}}%
\pgfpathclose%
\pgfusepath{fill}%
\end{pgfscope}%
\begin{pgfscope}%
\pgfsetbuttcap%
\pgfsetmiterjoin%
\definecolor{currentfill}{rgb}{1.000000,1.000000,1.000000}%
\pgfsetfillcolor{currentfill}%
\pgfsetlinewidth{0.000000pt}%
\definecolor{currentstroke}{rgb}{0.000000,0.000000,0.000000}%
\pgfsetstrokecolor{currentstroke}%
\pgfsetstrokeopacity{0.000000}%
\pgfsetdash{}{0pt}%
\pgfpathmoveto{\pgfqpoint{0.800000in}{0.528000in}}%
\pgfpathlineto{\pgfqpoint{5.760000in}{0.528000in}}%
\pgfpathlineto{\pgfqpoint{5.760000in}{4.224000in}}%
\pgfpathlineto{\pgfqpoint{0.800000in}{4.224000in}}%
\pgfpathlineto{\pgfqpoint{0.800000in}{0.528000in}}%
\pgfpathclose%
\pgfusepath{fill}%
\end{pgfscope}%
\begin{pgfscope}%
\pgfsetbuttcap%
\pgfsetroundjoin%
\definecolor{currentfill}{rgb}{0.000000,0.000000,0.000000}%
\pgfsetfillcolor{currentfill}%
\pgfsetlinewidth{0.803000pt}%
\definecolor{currentstroke}{rgb}{0.000000,0.000000,0.000000}%
\pgfsetstrokecolor{currentstroke}%
\pgfsetdash{}{0pt}%
\pgfsys@defobject{currentmarker}{\pgfqpoint{0.000000in}{-0.048611in}}{\pgfqpoint{0.000000in}{0.000000in}}{%
\pgfpathmoveto{\pgfqpoint{0.000000in}{0.000000in}}%
\pgfpathlineto{\pgfqpoint{0.000000in}{-0.048611in}}%
\pgfusepath{stroke,fill}%
}%
\begin{pgfscope}%
\pgfsys@transformshift{1.025455in}{0.696000in}%
\pgfsys@useobject{currentmarker}{}%
\end{pgfscope}%
\end{pgfscope}%
\begin{pgfscope}%
\definecolor{textcolor}{rgb}{0.000000,0.000000,0.000000}%
\pgfsetstrokecolor{textcolor}%
\pgfsetfillcolor{textcolor}%
\pgftext[x=1.025455in,y=0.598778in,,top]{\color{textcolor}\rmfamily\fontsize{10.000000}{12.000000}\selectfont \(\displaystyle {0}\)}%
\end{pgfscope}%
\begin{pgfscope}%
\pgfsetbuttcap%
\pgfsetroundjoin%
\definecolor{currentfill}{rgb}{0.000000,0.000000,0.000000}%
\pgfsetfillcolor{currentfill}%
\pgfsetlinewidth{0.803000pt}%
\definecolor{currentstroke}{rgb}{0.000000,0.000000,0.000000}%
\pgfsetstrokecolor{currentstroke}%
\pgfsetdash{}{0pt}%
\pgfsys@defobject{currentmarker}{\pgfqpoint{0.000000in}{-0.048611in}}{\pgfqpoint{0.000000in}{0.000000in}}{%
\pgfpathmoveto{\pgfqpoint{0.000000in}{0.000000in}}%
\pgfpathlineto{\pgfqpoint{0.000000in}{-0.048611in}}%
\pgfusepath{stroke,fill}%
}%
\begin{pgfscope}%
\pgfsys@transformshift{2.027697in}{0.696000in}%
\pgfsys@useobject{currentmarker}{}%
\end{pgfscope}%
\end{pgfscope}%
\begin{pgfscope}%
\definecolor{textcolor}{rgb}{0.000000,0.000000,0.000000}%
\pgfsetstrokecolor{textcolor}%
\pgfsetfillcolor{textcolor}%
\pgftext[x=2.027697in,y=0.598778in,,top]{\color{textcolor}\rmfamily\fontsize{10.000000}{12.000000}\selectfont \(\displaystyle {10}\)}%
\end{pgfscope}%
\begin{pgfscope}%
\pgfsetbuttcap%
\pgfsetroundjoin%
\definecolor{currentfill}{rgb}{0.000000,0.000000,0.000000}%
\pgfsetfillcolor{currentfill}%
\pgfsetlinewidth{0.803000pt}%
\definecolor{currentstroke}{rgb}{0.000000,0.000000,0.000000}%
\pgfsetstrokecolor{currentstroke}%
\pgfsetdash{}{0pt}%
\pgfsys@defobject{currentmarker}{\pgfqpoint{0.000000in}{-0.048611in}}{\pgfqpoint{0.000000in}{0.000000in}}{%
\pgfpathmoveto{\pgfqpoint{0.000000in}{0.000000in}}%
\pgfpathlineto{\pgfqpoint{0.000000in}{-0.048611in}}%
\pgfusepath{stroke,fill}%
}%
\begin{pgfscope}%
\pgfsys@transformshift{3.029940in}{0.696000in}%
\pgfsys@useobject{currentmarker}{}%
\end{pgfscope}%
\end{pgfscope}%
\begin{pgfscope}%
\definecolor{textcolor}{rgb}{0.000000,0.000000,0.000000}%
\pgfsetstrokecolor{textcolor}%
\pgfsetfillcolor{textcolor}%
\pgftext[x=3.029940in,y=0.598778in,,top]{\color{textcolor}\rmfamily\fontsize{10.000000}{12.000000}\selectfont \(\displaystyle {20}\)}%
\end{pgfscope}%
\begin{pgfscope}%
\pgfsetbuttcap%
\pgfsetroundjoin%
\definecolor{currentfill}{rgb}{0.000000,0.000000,0.000000}%
\pgfsetfillcolor{currentfill}%
\pgfsetlinewidth{0.803000pt}%
\definecolor{currentstroke}{rgb}{0.000000,0.000000,0.000000}%
\pgfsetstrokecolor{currentstroke}%
\pgfsetdash{}{0pt}%
\pgfsys@defobject{currentmarker}{\pgfqpoint{0.000000in}{-0.048611in}}{\pgfqpoint{0.000000in}{0.000000in}}{%
\pgfpathmoveto{\pgfqpoint{0.000000in}{0.000000in}}%
\pgfpathlineto{\pgfqpoint{0.000000in}{-0.048611in}}%
\pgfusepath{stroke,fill}%
}%
\begin{pgfscope}%
\pgfsys@transformshift{4.032183in}{0.696000in}%
\pgfsys@useobject{currentmarker}{}%
\end{pgfscope}%
\end{pgfscope}%
\begin{pgfscope}%
\definecolor{textcolor}{rgb}{0.000000,0.000000,0.000000}%
\pgfsetstrokecolor{textcolor}%
\pgfsetfillcolor{textcolor}%
\pgftext[x=4.032183in,y=0.598778in,,top]{\color{textcolor}\rmfamily\fontsize{10.000000}{12.000000}\selectfont \(\displaystyle {30}\)}%
\end{pgfscope}%
\begin{pgfscope}%
\pgfsetbuttcap%
\pgfsetroundjoin%
\definecolor{currentfill}{rgb}{0.000000,0.000000,0.000000}%
\pgfsetfillcolor{currentfill}%
\pgfsetlinewidth{0.803000pt}%
\definecolor{currentstroke}{rgb}{0.000000,0.000000,0.000000}%
\pgfsetstrokecolor{currentstroke}%
\pgfsetdash{}{0pt}%
\pgfsys@defobject{currentmarker}{\pgfqpoint{0.000000in}{-0.048611in}}{\pgfqpoint{0.000000in}{0.000000in}}{%
\pgfpathmoveto{\pgfqpoint{0.000000in}{0.000000in}}%
\pgfpathlineto{\pgfqpoint{0.000000in}{-0.048611in}}%
\pgfusepath{stroke,fill}%
}%
\begin{pgfscope}%
\pgfsys@transformshift{5.034426in}{0.696000in}%
\pgfsys@useobject{currentmarker}{}%
\end{pgfscope}%
\end{pgfscope}%
\begin{pgfscope}%
\definecolor{textcolor}{rgb}{0.000000,0.000000,0.000000}%
\pgfsetstrokecolor{textcolor}%
\pgfsetfillcolor{textcolor}%
\pgftext[x=5.034426in,y=0.598778in,,top]{\color{textcolor}\rmfamily\fontsize{10.000000}{12.000000}\selectfont \(\displaystyle {40}\)}%
\end{pgfscope}%
\begin{pgfscope}%
\pgfsetbuttcap%
\pgfsetroundjoin%
\definecolor{currentfill}{rgb}{0.000000,0.000000,0.000000}%
\pgfsetfillcolor{currentfill}%
\pgfsetlinewidth{0.803000pt}%
\definecolor{currentstroke}{rgb}{0.000000,0.000000,0.000000}%
\pgfsetstrokecolor{currentstroke}%
\pgfsetdash{}{0pt}%
\pgfsys@defobject{currentmarker}{\pgfqpoint{-0.048611in}{0.000000in}}{\pgfqpoint{-0.000000in}{0.000000in}}{%
\pgfpathmoveto{\pgfqpoint{-0.000000in}{0.000000in}}%
\pgfpathlineto{\pgfqpoint{-0.048611in}{0.000000in}}%
\pgfusepath{stroke,fill}%
}%
\begin{pgfscope}%
\pgfsys@transformshift{1.025455in}{0.696000in}%
\pgfsys@useobject{currentmarker}{}%
\end{pgfscope}%
\end{pgfscope}%
\begin{pgfscope}%
\definecolor{textcolor}{rgb}{0.000000,0.000000,0.000000}%
\pgfsetstrokecolor{textcolor}%
\pgfsetfillcolor{textcolor}%
\pgftext[x=0.858788in, y=0.647775in, left, base]{\color{textcolor}\rmfamily\fontsize{10.000000}{12.000000}\selectfont \(\displaystyle {0}\)}%
\end{pgfscope}%
\begin{pgfscope}%
\pgfsetbuttcap%
\pgfsetroundjoin%
\definecolor{currentfill}{rgb}{0.000000,0.000000,0.000000}%
\pgfsetfillcolor{currentfill}%
\pgfsetlinewidth{0.803000pt}%
\definecolor{currentstroke}{rgb}{0.000000,0.000000,0.000000}%
\pgfsetstrokecolor{currentstroke}%
\pgfsetdash{}{0pt}%
\pgfsys@defobject{currentmarker}{\pgfqpoint{-0.048611in}{0.000000in}}{\pgfqpoint{-0.000000in}{0.000000in}}{%
\pgfpathmoveto{\pgfqpoint{-0.000000in}{0.000000in}}%
\pgfpathlineto{\pgfqpoint{-0.048611in}{0.000000in}}%
\pgfusepath{stroke,fill}%
}%
\begin{pgfscope}%
\pgfsys@transformshift{1.025455in}{1.377818in}%
\pgfsys@useobject{currentmarker}{}%
\end{pgfscope}%
\end{pgfscope}%
\begin{pgfscope}%
\definecolor{textcolor}{rgb}{0.000000,0.000000,0.000000}%
\pgfsetstrokecolor{textcolor}%
\pgfsetfillcolor{textcolor}%
\pgftext[x=0.719898in, y=1.329593in, left, base]{\color{textcolor}\rmfamily\fontsize{10.000000}{12.000000}\selectfont \(\displaystyle {500}\)}%
\end{pgfscope}%
\begin{pgfscope}%
\pgfsetbuttcap%
\pgfsetroundjoin%
\definecolor{currentfill}{rgb}{0.000000,0.000000,0.000000}%
\pgfsetfillcolor{currentfill}%
\pgfsetlinewidth{0.803000pt}%
\definecolor{currentstroke}{rgb}{0.000000,0.000000,0.000000}%
\pgfsetstrokecolor{currentstroke}%
\pgfsetdash{}{0pt}%
\pgfsys@defobject{currentmarker}{\pgfqpoint{-0.048611in}{0.000000in}}{\pgfqpoint{-0.000000in}{0.000000in}}{%
\pgfpathmoveto{\pgfqpoint{-0.000000in}{0.000000in}}%
\pgfpathlineto{\pgfqpoint{-0.048611in}{0.000000in}}%
\pgfusepath{stroke,fill}%
}%
\begin{pgfscope}%
\pgfsys@transformshift{1.025455in}{2.059636in}%
\pgfsys@useobject{currentmarker}{}%
\end{pgfscope}%
\end{pgfscope}%
\begin{pgfscope}%
\definecolor{textcolor}{rgb}{0.000000,0.000000,0.000000}%
\pgfsetstrokecolor{textcolor}%
\pgfsetfillcolor{textcolor}%
\pgftext[x=0.650454in, y=2.011411in, left, base]{\color{textcolor}\rmfamily\fontsize{10.000000}{12.000000}\selectfont \(\displaystyle {1000}\)}%
\end{pgfscope}%
\begin{pgfscope}%
\pgfsetbuttcap%
\pgfsetroundjoin%
\definecolor{currentfill}{rgb}{0.000000,0.000000,0.000000}%
\pgfsetfillcolor{currentfill}%
\pgfsetlinewidth{0.803000pt}%
\definecolor{currentstroke}{rgb}{0.000000,0.000000,0.000000}%
\pgfsetstrokecolor{currentstroke}%
\pgfsetdash{}{0pt}%
\pgfsys@defobject{currentmarker}{\pgfqpoint{-0.048611in}{0.000000in}}{\pgfqpoint{-0.000000in}{0.000000in}}{%
\pgfpathmoveto{\pgfqpoint{-0.000000in}{0.000000in}}%
\pgfpathlineto{\pgfqpoint{-0.048611in}{0.000000in}}%
\pgfusepath{stroke,fill}%
}%
\begin{pgfscope}%
\pgfsys@transformshift{1.025455in}{2.741455in}%
\pgfsys@useobject{currentmarker}{}%
\end{pgfscope}%
\end{pgfscope}%
\begin{pgfscope}%
\definecolor{textcolor}{rgb}{0.000000,0.000000,0.000000}%
\pgfsetstrokecolor{textcolor}%
\pgfsetfillcolor{textcolor}%
\pgftext[x=0.650454in, y=2.693229in, left, base]{\color{textcolor}\rmfamily\fontsize{10.000000}{12.000000}\selectfont \(\displaystyle {1500}\)}%
\end{pgfscope}%
\begin{pgfscope}%
\pgfsetbuttcap%
\pgfsetroundjoin%
\definecolor{currentfill}{rgb}{0.000000,0.000000,0.000000}%
\pgfsetfillcolor{currentfill}%
\pgfsetlinewidth{0.803000pt}%
\definecolor{currentstroke}{rgb}{0.000000,0.000000,0.000000}%
\pgfsetstrokecolor{currentstroke}%
\pgfsetdash{}{0pt}%
\pgfsys@defobject{currentmarker}{\pgfqpoint{-0.048611in}{0.000000in}}{\pgfqpoint{-0.000000in}{0.000000in}}{%
\pgfpathmoveto{\pgfqpoint{-0.000000in}{0.000000in}}%
\pgfpathlineto{\pgfqpoint{-0.048611in}{0.000000in}}%
\pgfusepath{stroke,fill}%
}%
\begin{pgfscope}%
\pgfsys@transformshift{1.025455in}{3.423273in}%
\pgfsys@useobject{currentmarker}{}%
\end{pgfscope}%
\end{pgfscope}%
\begin{pgfscope}%
\definecolor{textcolor}{rgb}{0.000000,0.000000,0.000000}%
\pgfsetstrokecolor{textcolor}%
\pgfsetfillcolor{textcolor}%
\pgftext[x=0.650454in, y=3.375047in, left, base]{\color{textcolor}\rmfamily\fontsize{10.000000}{12.000000}\selectfont \(\displaystyle {2000}\)}%
\end{pgfscope}%
\begin{pgfscope}%
\pgfsetbuttcap%
\pgfsetroundjoin%
\definecolor{currentfill}{rgb}{0.000000,0.000000,0.000000}%
\pgfsetfillcolor{currentfill}%
\pgfsetlinewidth{0.803000pt}%
\definecolor{currentstroke}{rgb}{0.000000,0.000000,0.000000}%
\pgfsetstrokecolor{currentstroke}%
\pgfsetdash{}{0pt}%
\pgfsys@defobject{currentmarker}{\pgfqpoint{-0.048611in}{0.000000in}}{\pgfqpoint{-0.000000in}{0.000000in}}{%
\pgfpathmoveto{\pgfqpoint{-0.000000in}{0.000000in}}%
\pgfpathlineto{\pgfqpoint{-0.048611in}{0.000000in}}%
\pgfusepath{stroke,fill}%
}%
\begin{pgfscope}%
\pgfsys@transformshift{1.025455in}{4.105091in}%
\pgfsys@useobject{currentmarker}{}%
\end{pgfscope}%
\end{pgfscope}%
\begin{pgfscope}%
\definecolor{textcolor}{rgb}{0.000000,0.000000,0.000000}%
\pgfsetstrokecolor{textcolor}%
\pgfsetfillcolor{textcolor}%
\pgftext[x=0.650454in, y=4.056866in, left, base]{\color{textcolor}\rmfamily\fontsize{10.000000}{12.000000}\selectfont \(\displaystyle {2500}\)}%
\end{pgfscope}%
\begin{pgfscope}%
\pgfpathrectangle{\pgfqpoint{0.800000in}{0.528000in}}{\pgfqpoint{4.960000in}{3.696000in}}%
\pgfusepath{clip}%
\pgfsetrectcap%
\pgfsetroundjoin%
\pgfsetlinewidth{1.505625pt}%
\definecolor{currentstroke}{rgb}{0.564706,0.478431,0.662745}%
\pgfsetstrokecolor{currentstroke}%
\pgfsetdash{}{0pt}%
\pgfpathmoveto{\pgfqpoint{1.025455in}{3.023727in}}%
\pgfpathlineto{\pgfqpoint{1.027459in}{2.992364in}}%
\pgfpathlineto{\pgfqpoint{1.028461in}{2.988273in}}%
\pgfpathlineto{\pgfqpoint{1.029464in}{2.992364in}}%
\pgfpathlineto{\pgfqpoint{1.030466in}{2.984182in}}%
\pgfpathlineto{\pgfqpoint{1.031468in}{2.986909in}}%
\pgfpathlineto{\pgfqpoint{1.032470in}{2.996455in}}%
\pgfpathlineto{\pgfqpoint{1.033472in}{2.993727in}}%
\pgfpathlineto{\pgfqpoint{1.034475in}{2.996455in}}%
\pgfpathlineto{\pgfqpoint{1.035477in}{2.995091in}}%
\pgfpathlineto{\pgfqpoint{1.036479in}{2.984182in}}%
\pgfpathlineto{\pgfqpoint{1.037481in}{2.992364in}}%
\pgfpathlineto{\pgfqpoint{1.038484in}{2.978727in}}%
\pgfpathlineto{\pgfqpoint{1.039486in}{2.982818in}}%
\pgfpathlineto{\pgfqpoint{1.040488in}{2.978727in}}%
\pgfpathlineto{\pgfqpoint{1.041490in}{2.981455in}}%
\pgfpathlineto{\pgfqpoint{1.043495in}{2.967818in}}%
\pgfpathlineto{\pgfqpoint{1.045499in}{2.963727in}}%
\pgfpathlineto{\pgfqpoint{1.048506in}{2.973273in}}%
\pgfpathlineto{\pgfqpoint{1.049508in}{2.974636in}}%
\pgfpathlineto{\pgfqpoint{1.050511in}{2.959636in}}%
\pgfpathlineto{\pgfqpoint{1.051513in}{2.982818in}}%
\pgfpathlineto{\pgfqpoint{1.053517in}{2.966455in}}%
\pgfpathlineto{\pgfqpoint{1.055522in}{2.978727in}}%
\pgfpathlineto{\pgfqpoint{1.056524in}{2.959636in}}%
\pgfpathlineto{\pgfqpoint{1.058529in}{2.978727in}}%
\pgfpathlineto{\pgfqpoint{1.059531in}{2.980091in}}%
\pgfpathlineto{\pgfqpoint{1.063540in}{2.999182in}}%
\pgfpathlineto{\pgfqpoint{1.064542in}{2.978727in}}%
\pgfpathlineto{\pgfqpoint{1.065544in}{2.992364in}}%
\pgfpathlineto{\pgfqpoint{1.066547in}{2.991000in}}%
\pgfpathlineto{\pgfqpoint{1.067549in}{2.995091in}}%
\pgfpathlineto{\pgfqpoint{1.068551in}{2.985545in}}%
\pgfpathlineto{\pgfqpoint{1.069553in}{2.986909in}}%
\pgfpathlineto{\pgfqpoint{1.071558in}{2.999182in}}%
\pgfpathlineto{\pgfqpoint{1.075567in}{2.973273in}}%
\pgfpathlineto{\pgfqpoint{1.076569in}{2.974636in}}%
\pgfpathlineto{\pgfqpoint{1.078573in}{2.969182in}}%
\pgfpathlineto{\pgfqpoint{1.079576in}{2.970545in}}%
\pgfpathlineto{\pgfqpoint{1.081580in}{2.959636in}}%
\pgfpathlineto{\pgfqpoint{1.083585in}{2.969182in}}%
\pgfpathlineto{\pgfqpoint{1.084587in}{2.970545in}}%
\pgfpathlineto{\pgfqpoint{1.086591in}{2.952818in}}%
\pgfpathlineto{\pgfqpoint{1.087594in}{2.981455in}}%
\pgfpathlineto{\pgfqpoint{1.088596in}{2.971909in}}%
\pgfpathlineto{\pgfqpoint{1.091603in}{2.982818in}}%
\pgfpathlineto{\pgfqpoint{1.092605in}{2.980091in}}%
\pgfpathlineto{\pgfqpoint{1.093607in}{2.989636in}}%
\pgfpathlineto{\pgfqpoint{1.094609in}{2.970545in}}%
\pgfpathlineto{\pgfqpoint{1.095612in}{2.986909in}}%
\pgfpathlineto{\pgfqpoint{1.096614in}{2.981455in}}%
\pgfpathlineto{\pgfqpoint{1.099621in}{2.991000in}}%
\pgfpathlineto{\pgfqpoint{1.100623in}{2.977364in}}%
\pgfpathlineto{\pgfqpoint{1.101625in}{2.993727in}}%
\pgfpathlineto{\pgfqpoint{1.102627in}{2.978727in}}%
\pgfpathlineto{\pgfqpoint{1.104632in}{2.985545in}}%
\pgfpathlineto{\pgfqpoint{1.106636in}{2.988273in}}%
\pgfpathlineto{\pgfqpoint{1.108641in}{2.984182in}}%
\pgfpathlineto{\pgfqpoint{1.110645in}{2.973273in}}%
\pgfpathlineto{\pgfqpoint{1.111647in}{2.965091in}}%
\pgfpathlineto{\pgfqpoint{1.113652in}{2.986909in}}%
\pgfpathlineto{\pgfqpoint{1.116659in}{2.963727in}}%
\pgfpathlineto{\pgfqpoint{1.117661in}{2.965091in}}%
\pgfpathlineto{\pgfqpoint{1.119665in}{2.980091in}}%
\pgfpathlineto{\pgfqpoint{1.120668in}{2.965091in}}%
\pgfpathlineto{\pgfqpoint{1.121670in}{2.969182in}}%
\pgfpathlineto{\pgfqpoint{1.122672in}{2.961000in}}%
\pgfpathlineto{\pgfqpoint{1.124677in}{2.974636in}}%
\pgfpathlineto{\pgfqpoint{1.126681in}{2.967818in}}%
\pgfpathlineto{\pgfqpoint{1.129688in}{2.985545in}}%
\pgfpathlineto{\pgfqpoint{1.131692in}{2.982818in}}%
\pgfpathlineto{\pgfqpoint{1.133697in}{2.986909in}}%
\pgfpathlineto{\pgfqpoint{1.134699in}{2.973273in}}%
\pgfpathlineto{\pgfqpoint{1.137706in}{2.996455in}}%
\pgfpathlineto{\pgfqpoint{1.139710in}{2.973273in}}%
\pgfpathlineto{\pgfqpoint{1.142717in}{2.986909in}}%
\pgfpathlineto{\pgfqpoint{1.144721in}{2.977364in}}%
\pgfpathlineto{\pgfqpoint{1.145724in}{2.984182in}}%
\pgfpathlineto{\pgfqpoint{1.147728in}{2.977364in}}%
\pgfpathlineto{\pgfqpoint{1.148730in}{2.980091in}}%
\pgfpathlineto{\pgfqpoint{1.151737in}{2.963727in}}%
\pgfpathlineto{\pgfqpoint{1.152739in}{2.973273in}}%
\pgfpathlineto{\pgfqpoint{1.153742in}{2.961000in}}%
\pgfpathlineto{\pgfqpoint{1.155746in}{2.969182in}}%
\pgfpathlineto{\pgfqpoint{1.156748in}{2.969182in}}%
\pgfpathlineto{\pgfqpoint{1.157751in}{2.981455in}}%
\pgfpathlineto{\pgfqpoint{1.158753in}{2.976000in}}%
\pgfpathlineto{\pgfqpoint{1.160757in}{2.988273in}}%
\pgfpathlineto{\pgfqpoint{1.162762in}{2.980091in}}%
\pgfpathlineto{\pgfqpoint{1.163764in}{2.984182in}}%
\pgfpathlineto{\pgfqpoint{1.164766in}{2.977364in}}%
\pgfpathlineto{\pgfqpoint{1.165769in}{2.991000in}}%
\pgfpathlineto{\pgfqpoint{1.166771in}{2.986909in}}%
\pgfpathlineto{\pgfqpoint{1.167773in}{2.996455in}}%
\pgfpathlineto{\pgfqpoint{1.168775in}{2.988273in}}%
\pgfpathlineto{\pgfqpoint{1.169778in}{2.993727in}}%
\pgfpathlineto{\pgfqpoint{1.170780in}{2.984182in}}%
\pgfpathlineto{\pgfqpoint{1.173786in}{2.989636in}}%
\pgfpathlineto{\pgfqpoint{1.175791in}{2.984182in}}%
\pgfpathlineto{\pgfqpoint{1.176793in}{2.976000in}}%
\pgfpathlineto{\pgfqpoint{1.177795in}{2.995091in}}%
\pgfpathlineto{\pgfqpoint{1.179800in}{2.969182in}}%
\pgfpathlineto{\pgfqpoint{1.180802in}{2.977364in}}%
\pgfpathlineto{\pgfqpoint{1.181804in}{2.973273in}}%
\pgfpathlineto{\pgfqpoint{1.182807in}{2.963727in}}%
\pgfpathlineto{\pgfqpoint{1.183809in}{2.981455in}}%
\pgfpathlineto{\pgfqpoint{1.184811in}{2.969182in}}%
\pgfpathlineto{\pgfqpoint{1.185813in}{2.982818in}}%
\pgfpathlineto{\pgfqpoint{1.187818in}{2.966455in}}%
\pgfpathlineto{\pgfqpoint{1.190825in}{2.977364in}}%
\pgfpathlineto{\pgfqpoint{1.191827in}{2.970545in}}%
\pgfpathlineto{\pgfqpoint{1.193831in}{2.988273in}}%
\pgfpathlineto{\pgfqpoint{1.194834in}{2.985545in}}%
\pgfpathlineto{\pgfqpoint{1.195836in}{2.988273in}}%
\pgfpathlineto{\pgfqpoint{1.196838in}{2.996455in}}%
\pgfpathlineto{\pgfqpoint{1.197840in}{2.993727in}}%
\pgfpathlineto{\pgfqpoint{1.199845in}{2.986909in}}%
\pgfpathlineto{\pgfqpoint{1.200847in}{2.982818in}}%
\pgfpathlineto{\pgfqpoint{1.201849in}{2.989636in}}%
\pgfpathlineto{\pgfqpoint{1.202852in}{2.985545in}}%
\pgfpathlineto{\pgfqpoint{1.203854in}{2.988273in}}%
\pgfpathlineto{\pgfqpoint{1.204856in}{2.984182in}}%
\pgfpathlineto{\pgfqpoint{1.205858in}{2.974636in}}%
\pgfpathlineto{\pgfqpoint{1.206861in}{2.991000in}}%
\pgfpathlineto{\pgfqpoint{1.207863in}{2.989636in}}%
\pgfpathlineto{\pgfqpoint{1.209867in}{2.993727in}}%
\pgfpathlineto{\pgfqpoint{1.210869in}{2.966455in}}%
\pgfpathlineto{\pgfqpoint{1.211872in}{2.970545in}}%
\pgfpathlineto{\pgfqpoint{1.212874in}{2.985545in}}%
\pgfpathlineto{\pgfqpoint{1.214878in}{2.973273in}}%
\pgfpathlineto{\pgfqpoint{1.215881in}{2.977364in}}%
\pgfpathlineto{\pgfqpoint{1.217885in}{2.965091in}}%
\pgfpathlineto{\pgfqpoint{1.219890in}{2.971909in}}%
\pgfpathlineto{\pgfqpoint{1.220892in}{2.974636in}}%
\pgfpathlineto{\pgfqpoint{1.221894in}{2.982818in}}%
\pgfpathlineto{\pgfqpoint{1.222896in}{2.962364in}}%
\pgfpathlineto{\pgfqpoint{1.224901in}{2.982818in}}%
\pgfpathlineto{\pgfqpoint{1.226905in}{2.974636in}}%
\pgfpathlineto{\pgfqpoint{1.227908in}{2.999182in}}%
\pgfpathlineto{\pgfqpoint{1.228910in}{2.971909in}}%
\pgfpathlineto{\pgfqpoint{1.231917in}{2.993727in}}%
\pgfpathlineto{\pgfqpoint{1.232919in}{2.986909in}}%
\pgfpathlineto{\pgfqpoint{1.233921in}{2.997818in}}%
\pgfpathlineto{\pgfqpoint{1.234923in}{2.991000in}}%
\pgfpathlineto{\pgfqpoint{1.235926in}{3.000545in}}%
\pgfpathlineto{\pgfqpoint{1.237930in}{2.986909in}}%
\pgfpathlineto{\pgfqpoint{1.238932in}{2.986909in}}%
\pgfpathlineto{\pgfqpoint{1.239935in}{2.996455in}}%
\pgfpathlineto{\pgfqpoint{1.240937in}{2.984182in}}%
\pgfpathlineto{\pgfqpoint{1.241939in}{2.991000in}}%
\pgfpathlineto{\pgfqpoint{1.242941in}{2.971909in}}%
\pgfpathlineto{\pgfqpoint{1.244946in}{2.988273in}}%
\pgfpathlineto{\pgfqpoint{1.245948in}{2.986909in}}%
\pgfpathlineto{\pgfqpoint{1.247952in}{2.973273in}}%
\pgfpathlineto{\pgfqpoint{1.248955in}{2.976000in}}%
\pgfpathlineto{\pgfqpoint{1.249957in}{2.973273in}}%
\pgfpathlineto{\pgfqpoint{1.250959in}{2.977364in}}%
\pgfpathlineto{\pgfqpoint{1.252964in}{2.969182in}}%
\pgfpathlineto{\pgfqpoint{1.253966in}{2.985545in}}%
\pgfpathlineto{\pgfqpoint{1.254968in}{2.959636in}}%
\pgfpathlineto{\pgfqpoint{1.256973in}{2.974636in}}%
\pgfpathlineto{\pgfqpoint{1.257975in}{2.971909in}}%
\pgfpathlineto{\pgfqpoint{1.258977in}{2.967818in}}%
\pgfpathlineto{\pgfqpoint{1.260982in}{2.974636in}}%
\pgfpathlineto{\pgfqpoint{1.263988in}{2.991000in}}%
\pgfpathlineto{\pgfqpoint{1.264991in}{2.978727in}}%
\pgfpathlineto{\pgfqpoint{1.266995in}{2.997818in}}%
\pgfpathlineto{\pgfqpoint{1.267997in}{2.992364in}}%
\pgfpathlineto{\pgfqpoint{1.269000in}{2.996455in}}%
\pgfpathlineto{\pgfqpoint{1.271004in}{2.981455in}}%
\pgfpathlineto{\pgfqpoint{1.272006in}{2.993727in}}%
\pgfpathlineto{\pgfqpoint{1.274011in}{2.988273in}}%
\pgfpathlineto{\pgfqpoint{1.275013in}{2.993727in}}%
\pgfpathlineto{\pgfqpoint{1.277018in}{2.986909in}}%
\pgfpathlineto{\pgfqpoint{1.278020in}{2.992364in}}%
\pgfpathlineto{\pgfqpoint{1.279022in}{2.989636in}}%
\pgfpathlineto{\pgfqpoint{1.280024in}{2.993727in}}%
\pgfpathlineto{\pgfqpoint{1.281026in}{2.980091in}}%
\pgfpathlineto{\pgfqpoint{1.282029in}{2.981455in}}%
\pgfpathlineto{\pgfqpoint{1.283031in}{2.978727in}}%
\pgfpathlineto{\pgfqpoint{1.284033in}{2.988273in}}%
\pgfpathlineto{\pgfqpoint{1.285035in}{2.986909in}}%
\pgfpathlineto{\pgfqpoint{1.286038in}{2.988273in}}%
\pgfpathlineto{\pgfqpoint{1.287040in}{2.969182in}}%
\pgfpathlineto{\pgfqpoint{1.288042in}{2.989636in}}%
\pgfpathlineto{\pgfqpoint{1.289044in}{2.974636in}}%
\pgfpathlineto{\pgfqpoint{1.290047in}{2.984182in}}%
\pgfpathlineto{\pgfqpoint{1.291049in}{2.976000in}}%
\pgfpathlineto{\pgfqpoint{1.292051in}{2.977364in}}%
\pgfpathlineto{\pgfqpoint{1.293053in}{2.986909in}}%
\pgfpathlineto{\pgfqpoint{1.294056in}{2.974636in}}%
\pgfpathlineto{\pgfqpoint{1.296060in}{2.981455in}}%
\pgfpathlineto{\pgfqpoint{1.297062in}{2.976000in}}%
\pgfpathlineto{\pgfqpoint{1.298065in}{2.988273in}}%
\pgfpathlineto{\pgfqpoint{1.299067in}{2.978727in}}%
\pgfpathlineto{\pgfqpoint{1.302074in}{2.986909in}}%
\pgfpathlineto{\pgfqpoint{1.303076in}{2.988273in}}%
\pgfpathlineto{\pgfqpoint{1.304078in}{2.996455in}}%
\pgfpathlineto{\pgfqpoint{1.306083in}{2.982818in}}%
\pgfpathlineto{\pgfqpoint{1.308087in}{2.999182in}}%
\pgfpathlineto{\pgfqpoint{1.309089in}{2.992364in}}%
\pgfpathlineto{\pgfqpoint{1.310092in}{2.993727in}}%
\pgfpathlineto{\pgfqpoint{1.311094in}{2.982818in}}%
\pgfpathlineto{\pgfqpoint{1.312096in}{2.985545in}}%
\pgfpathlineto{\pgfqpoint{1.313098in}{2.995091in}}%
\pgfpathlineto{\pgfqpoint{1.314101in}{2.984182in}}%
\pgfpathlineto{\pgfqpoint{1.315103in}{2.985545in}}%
\pgfpathlineto{\pgfqpoint{1.316105in}{2.976000in}}%
\pgfpathlineto{\pgfqpoint{1.317107in}{2.978727in}}%
\pgfpathlineto{\pgfqpoint{1.319112in}{2.976000in}}%
\pgfpathlineto{\pgfqpoint{1.320114in}{2.985545in}}%
\pgfpathlineto{\pgfqpoint{1.322118in}{2.974636in}}%
\pgfpathlineto{\pgfqpoint{1.323121in}{2.980091in}}%
\pgfpathlineto{\pgfqpoint{1.324123in}{2.978727in}}%
\pgfpathlineto{\pgfqpoint{1.325125in}{2.967818in}}%
\pgfpathlineto{\pgfqpoint{1.326127in}{2.984182in}}%
\pgfpathlineto{\pgfqpoint{1.329134in}{2.966455in}}%
\pgfpathlineto{\pgfqpoint{1.332141in}{2.973273in}}%
\pgfpathlineto{\pgfqpoint{1.333143in}{2.982818in}}%
\pgfpathlineto{\pgfqpoint{1.334145in}{2.977364in}}%
\pgfpathlineto{\pgfqpoint{1.336150in}{2.989636in}}%
\pgfpathlineto{\pgfqpoint{1.337152in}{2.984182in}}%
\pgfpathlineto{\pgfqpoint{1.338154in}{2.988273in}}%
\pgfpathlineto{\pgfqpoint{1.339157in}{2.981455in}}%
\pgfpathlineto{\pgfqpoint{1.341161in}{2.991000in}}%
\pgfpathlineto{\pgfqpoint{1.342163in}{2.986909in}}%
\pgfpathlineto{\pgfqpoint{1.344168in}{3.003273in}}%
\pgfpathlineto{\pgfqpoint{1.345170in}{2.991000in}}%
\pgfpathlineto{\pgfqpoint{1.346172in}{2.996455in}}%
\pgfpathlineto{\pgfqpoint{1.347175in}{2.995091in}}%
\pgfpathlineto{\pgfqpoint{1.348177in}{3.001909in}}%
\pgfpathlineto{\pgfqpoint{1.350181in}{2.984182in}}%
\pgfpathlineto{\pgfqpoint{1.351183in}{2.992364in}}%
\pgfpathlineto{\pgfqpoint{1.352186in}{2.978727in}}%
\pgfpathlineto{\pgfqpoint{1.354190in}{2.982818in}}%
\pgfpathlineto{\pgfqpoint{1.356195in}{2.977364in}}%
\pgfpathlineto{\pgfqpoint{1.359201in}{2.971909in}}%
\pgfpathlineto{\pgfqpoint{1.360204in}{2.986909in}}%
\pgfpathlineto{\pgfqpoint{1.362208in}{2.974636in}}%
\pgfpathlineto{\pgfqpoint{1.364213in}{2.971909in}}%
\pgfpathlineto{\pgfqpoint{1.365215in}{2.974636in}}%
\pgfpathlineto{\pgfqpoint{1.366217in}{2.970545in}}%
\pgfpathlineto{\pgfqpoint{1.368222in}{2.980091in}}%
\pgfpathlineto{\pgfqpoint{1.369224in}{2.971909in}}%
\pgfpathlineto{\pgfqpoint{1.370226in}{2.977364in}}%
\pgfpathlineto{\pgfqpoint{1.372231in}{2.970545in}}%
\pgfpathlineto{\pgfqpoint{1.374235in}{2.988273in}}%
\pgfpathlineto{\pgfqpoint{1.375237in}{2.973273in}}%
\pgfpathlineto{\pgfqpoint{1.378244in}{2.995091in}}%
\pgfpathlineto{\pgfqpoint{1.380249in}{2.982818in}}%
\pgfpathlineto{\pgfqpoint{1.384258in}{2.996455in}}%
\pgfpathlineto{\pgfqpoint{1.386262in}{2.984182in}}%
\pgfpathlineto{\pgfqpoint{1.387264in}{2.985545in}}%
\pgfpathlineto{\pgfqpoint{1.389269in}{2.995091in}}%
\pgfpathlineto{\pgfqpoint{1.391273in}{2.980091in}}%
\pgfpathlineto{\pgfqpoint{1.392275in}{2.984182in}}%
\pgfpathlineto{\pgfqpoint{1.393278in}{2.965091in}}%
\pgfpathlineto{\pgfqpoint{1.395282in}{2.982818in}}%
\pgfpathlineto{\pgfqpoint{1.400293in}{2.977364in}}%
\pgfpathlineto{\pgfqpoint{1.401296in}{2.980091in}}%
\pgfpathlineto{\pgfqpoint{1.402298in}{2.973273in}}%
\pgfpathlineto{\pgfqpoint{1.403300in}{2.976000in}}%
\pgfpathlineto{\pgfqpoint{1.405305in}{2.969182in}}%
\pgfpathlineto{\pgfqpoint{1.408311in}{2.974636in}}%
\pgfpathlineto{\pgfqpoint{1.409314in}{2.997818in}}%
\pgfpathlineto{\pgfqpoint{1.410316in}{2.969182in}}%
\pgfpathlineto{\pgfqpoint{1.412320in}{2.988273in}}%
\pgfpathlineto{\pgfqpoint{1.413323in}{2.980091in}}%
\pgfpathlineto{\pgfqpoint{1.414325in}{2.988273in}}%
\pgfpathlineto{\pgfqpoint{1.415327in}{2.978727in}}%
\pgfpathlineto{\pgfqpoint{1.416329in}{2.991000in}}%
\pgfpathlineto{\pgfqpoint{1.417332in}{2.973273in}}%
\pgfpathlineto{\pgfqpoint{1.419336in}{2.989636in}}%
\pgfpathlineto{\pgfqpoint{1.420338in}{2.988273in}}%
\pgfpathlineto{\pgfqpoint{1.421340in}{2.999182in}}%
\pgfpathlineto{\pgfqpoint{1.422343in}{2.986909in}}%
\pgfpathlineto{\pgfqpoint{1.423345in}{2.996455in}}%
\pgfpathlineto{\pgfqpoint{1.425349in}{2.984182in}}%
\pgfpathlineto{\pgfqpoint{1.426352in}{2.991000in}}%
\pgfpathlineto{\pgfqpoint{1.427354in}{2.977364in}}%
\pgfpathlineto{\pgfqpoint{1.428356in}{2.991000in}}%
\pgfpathlineto{\pgfqpoint{1.430361in}{2.977364in}}%
\pgfpathlineto{\pgfqpoint{1.431363in}{2.991000in}}%
\pgfpathlineto{\pgfqpoint{1.433367in}{2.967818in}}%
\pgfpathlineto{\pgfqpoint{1.436374in}{2.982818in}}%
\pgfpathlineto{\pgfqpoint{1.437376in}{2.961000in}}%
\pgfpathlineto{\pgfqpoint{1.438379in}{2.974636in}}%
\pgfpathlineto{\pgfqpoint{1.439381in}{2.971909in}}%
\pgfpathlineto{\pgfqpoint{1.440383in}{2.984182in}}%
\pgfpathlineto{\pgfqpoint{1.442388in}{2.970545in}}%
\pgfpathlineto{\pgfqpoint{1.443390in}{2.966455in}}%
\pgfpathlineto{\pgfqpoint{1.444392in}{2.967818in}}%
\pgfpathlineto{\pgfqpoint{1.445394in}{2.980091in}}%
\pgfpathlineto{\pgfqpoint{1.446397in}{2.978727in}}%
\pgfpathlineto{\pgfqpoint{1.448401in}{2.986909in}}%
\pgfpathlineto{\pgfqpoint{1.449403in}{2.969182in}}%
\pgfpathlineto{\pgfqpoint{1.451408in}{2.984182in}}%
\pgfpathlineto{\pgfqpoint{1.452410in}{2.984182in}}%
\pgfpathlineto{\pgfqpoint{1.453412in}{2.999182in}}%
\pgfpathlineto{\pgfqpoint{1.454415in}{2.982818in}}%
\pgfpathlineto{\pgfqpoint{1.455417in}{2.986909in}}%
\pgfpathlineto{\pgfqpoint{1.456419in}{2.981455in}}%
\pgfpathlineto{\pgfqpoint{1.458423in}{3.000545in}}%
\pgfpathlineto{\pgfqpoint{1.459426in}{2.996455in}}%
\pgfpathlineto{\pgfqpoint{1.461430in}{2.967818in}}%
\pgfpathlineto{\pgfqpoint{1.462432in}{3.000545in}}%
\pgfpathlineto{\pgfqpoint{1.463435in}{2.984182in}}%
\pgfpathlineto{\pgfqpoint{1.464437in}{2.988273in}}%
\pgfpathlineto{\pgfqpoint{1.465439in}{2.986909in}}%
\pgfpathlineto{\pgfqpoint{1.466441in}{2.989636in}}%
\pgfpathlineto{\pgfqpoint{1.467444in}{2.988273in}}%
\pgfpathlineto{\pgfqpoint{1.468446in}{2.973273in}}%
\pgfpathlineto{\pgfqpoint{1.470450in}{2.999182in}}%
\pgfpathlineto{\pgfqpoint{1.471453in}{2.974636in}}%
\pgfpathlineto{\pgfqpoint{1.472455in}{2.988273in}}%
\pgfpathlineto{\pgfqpoint{1.473457in}{2.966455in}}%
\pgfpathlineto{\pgfqpoint{1.475462in}{2.981455in}}%
\pgfpathlineto{\pgfqpoint{1.476464in}{2.980091in}}%
\pgfpathlineto{\pgfqpoint{1.478468in}{2.965091in}}%
\pgfpathlineto{\pgfqpoint{1.480473in}{2.982818in}}%
\pgfpathlineto{\pgfqpoint{1.481475in}{2.974636in}}%
\pgfpathlineto{\pgfqpoint{1.482477in}{2.978727in}}%
\pgfpathlineto{\pgfqpoint{1.483480in}{2.959636in}}%
\pgfpathlineto{\pgfqpoint{1.484482in}{2.963727in}}%
\pgfpathlineto{\pgfqpoint{1.486486in}{2.986909in}}%
\pgfpathlineto{\pgfqpoint{1.488491in}{2.976000in}}%
\pgfpathlineto{\pgfqpoint{1.489493in}{2.986909in}}%
\pgfpathlineto{\pgfqpoint{1.490495in}{2.970545in}}%
\pgfpathlineto{\pgfqpoint{1.492500in}{2.992364in}}%
\pgfpathlineto{\pgfqpoint{1.494504in}{2.981455in}}%
\pgfpathlineto{\pgfqpoint{1.495506in}{2.980091in}}%
\pgfpathlineto{\pgfqpoint{1.497511in}{2.982818in}}%
\pgfpathlineto{\pgfqpoint{1.498513in}{2.996455in}}%
\pgfpathlineto{\pgfqpoint{1.499515in}{2.995091in}}%
\pgfpathlineto{\pgfqpoint{1.500518in}{2.980091in}}%
\pgfpathlineto{\pgfqpoint{1.502522in}{3.003273in}}%
\pgfpathlineto{\pgfqpoint{1.504527in}{2.996455in}}%
\pgfpathlineto{\pgfqpoint{1.505529in}{2.977364in}}%
\pgfpathlineto{\pgfqpoint{1.506531in}{3.001909in}}%
\pgfpathlineto{\pgfqpoint{1.507533in}{2.989636in}}%
\pgfpathlineto{\pgfqpoint{1.508536in}{2.991000in}}%
\pgfpathlineto{\pgfqpoint{1.510540in}{2.985545in}}%
\pgfpathlineto{\pgfqpoint{1.512545in}{2.973273in}}%
\pgfpathlineto{\pgfqpoint{1.513547in}{2.986909in}}%
\pgfpathlineto{\pgfqpoint{1.514549in}{2.985545in}}%
\pgfpathlineto{\pgfqpoint{1.515551in}{2.985545in}}%
\pgfpathlineto{\pgfqpoint{1.516554in}{2.984182in}}%
\pgfpathlineto{\pgfqpoint{1.518558in}{2.974636in}}%
\pgfpathlineto{\pgfqpoint{1.520563in}{2.986909in}}%
\pgfpathlineto{\pgfqpoint{1.521565in}{2.967818in}}%
\pgfpathlineto{\pgfqpoint{1.522567in}{2.973273in}}%
\pgfpathlineto{\pgfqpoint{1.523569in}{2.971909in}}%
\pgfpathlineto{\pgfqpoint{1.526576in}{2.976000in}}%
\pgfpathlineto{\pgfqpoint{1.527578in}{2.958273in}}%
\pgfpathlineto{\pgfqpoint{1.528580in}{2.977364in}}%
\pgfpathlineto{\pgfqpoint{1.529583in}{2.974636in}}%
\pgfpathlineto{\pgfqpoint{1.531587in}{2.986909in}}%
\pgfpathlineto{\pgfqpoint{1.532589in}{2.965091in}}%
\pgfpathlineto{\pgfqpoint{1.534594in}{2.980091in}}%
\pgfpathlineto{\pgfqpoint{1.537601in}{2.992364in}}%
\pgfpathlineto{\pgfqpoint{1.538603in}{3.006000in}}%
\pgfpathlineto{\pgfqpoint{1.540607in}{2.997818in}}%
\pgfpathlineto{\pgfqpoint{1.541610in}{2.996455in}}%
\pgfpathlineto{\pgfqpoint{1.543614in}{3.008727in}}%
\pgfpathlineto{\pgfqpoint{1.544616in}{2.995091in}}%
\pgfpathlineto{\pgfqpoint{1.545619in}{2.999182in}}%
\pgfpathlineto{\pgfqpoint{1.546621in}{2.997818in}}%
\pgfpathlineto{\pgfqpoint{1.548625in}{3.014182in}}%
\pgfpathlineto{\pgfqpoint{1.551632in}{2.977364in}}%
\pgfpathlineto{\pgfqpoint{1.552634in}{3.001909in}}%
\pgfpathlineto{\pgfqpoint{1.553637in}{3.000545in}}%
\pgfpathlineto{\pgfqpoint{1.554639in}{2.982818in}}%
\pgfpathlineto{\pgfqpoint{1.555641in}{2.989636in}}%
\pgfpathlineto{\pgfqpoint{1.557646in}{2.981455in}}%
\pgfpathlineto{\pgfqpoint{1.558648in}{2.985545in}}%
\pgfpathlineto{\pgfqpoint{1.559650in}{2.980091in}}%
\pgfpathlineto{\pgfqpoint{1.560652in}{2.992364in}}%
\pgfpathlineto{\pgfqpoint{1.561655in}{2.971909in}}%
\pgfpathlineto{\pgfqpoint{1.562657in}{2.978727in}}%
\pgfpathlineto{\pgfqpoint{1.564661in}{2.967818in}}%
\pgfpathlineto{\pgfqpoint{1.565663in}{2.982818in}}%
\pgfpathlineto{\pgfqpoint{1.567668in}{2.965091in}}%
\pgfpathlineto{\pgfqpoint{1.568670in}{2.973273in}}%
\pgfpathlineto{\pgfqpoint{1.569672in}{2.959636in}}%
\pgfpathlineto{\pgfqpoint{1.570675in}{2.980091in}}%
\pgfpathlineto{\pgfqpoint{1.571677in}{2.970545in}}%
\pgfpathlineto{\pgfqpoint{1.572679in}{2.981455in}}%
\pgfpathlineto{\pgfqpoint{1.573681in}{2.966455in}}%
\pgfpathlineto{\pgfqpoint{1.575686in}{2.982818in}}%
\pgfpathlineto{\pgfqpoint{1.576688in}{2.971909in}}%
\pgfpathlineto{\pgfqpoint{1.577690in}{2.988273in}}%
\pgfpathlineto{\pgfqpoint{1.579695in}{2.966455in}}%
\pgfpathlineto{\pgfqpoint{1.582702in}{2.999182in}}%
\pgfpathlineto{\pgfqpoint{1.584706in}{2.988273in}}%
\pgfpathlineto{\pgfqpoint{1.586711in}{2.986909in}}%
\pgfpathlineto{\pgfqpoint{1.588715in}{3.012818in}}%
\pgfpathlineto{\pgfqpoint{1.589717in}{3.004636in}}%
\pgfpathlineto{\pgfqpoint{1.591722in}{2.982818in}}%
\pgfpathlineto{\pgfqpoint{1.593726in}{3.012818in}}%
\pgfpathlineto{\pgfqpoint{1.594729in}{3.006000in}}%
\pgfpathlineto{\pgfqpoint{1.595731in}{2.978727in}}%
\pgfpathlineto{\pgfqpoint{1.597735in}{3.007364in}}%
\pgfpathlineto{\pgfqpoint{1.598737in}{3.021000in}}%
\pgfpathlineto{\pgfqpoint{1.601744in}{2.980091in}}%
\pgfpathlineto{\pgfqpoint{1.602746in}{2.991000in}}%
\pgfpathlineto{\pgfqpoint{1.603749in}{2.982818in}}%
\pgfpathlineto{\pgfqpoint{1.605753in}{2.988273in}}%
\pgfpathlineto{\pgfqpoint{1.606755in}{2.986909in}}%
\pgfpathlineto{\pgfqpoint{1.608760in}{2.962364in}}%
\pgfpathlineto{\pgfqpoint{1.609762in}{2.958273in}}%
\pgfpathlineto{\pgfqpoint{1.610764in}{2.981455in}}%
\pgfpathlineto{\pgfqpoint{1.611767in}{2.973273in}}%
\pgfpathlineto{\pgfqpoint{1.613771in}{2.950091in}}%
\pgfpathlineto{\pgfqpoint{1.614773in}{2.969182in}}%
\pgfpathlineto{\pgfqpoint{1.615776in}{2.965091in}}%
\pgfpathlineto{\pgfqpoint{1.616778in}{2.984182in}}%
\pgfpathlineto{\pgfqpoint{1.617780in}{2.950091in}}%
\pgfpathlineto{\pgfqpoint{1.620787in}{2.984182in}}%
\pgfpathlineto{\pgfqpoint{1.621789in}{2.978727in}}%
\pgfpathlineto{\pgfqpoint{1.622791in}{2.988273in}}%
\pgfpathlineto{\pgfqpoint{1.623794in}{2.980091in}}%
\pgfpathlineto{\pgfqpoint{1.624796in}{2.982818in}}%
\pgfpathlineto{\pgfqpoint{1.625798in}{2.980091in}}%
\pgfpathlineto{\pgfqpoint{1.628805in}{3.003273in}}%
\pgfpathlineto{\pgfqpoint{1.629807in}{3.001909in}}%
\pgfpathlineto{\pgfqpoint{1.630809in}{2.991000in}}%
\pgfpathlineto{\pgfqpoint{1.631812in}{2.995091in}}%
\pgfpathlineto{\pgfqpoint{1.632814in}{3.012818in}}%
\pgfpathlineto{\pgfqpoint{1.633816in}{3.010091in}}%
\pgfpathlineto{\pgfqpoint{1.635820in}{2.995091in}}%
\pgfpathlineto{\pgfqpoint{1.638827in}{3.018273in}}%
\pgfpathlineto{\pgfqpoint{1.640832in}{2.993727in}}%
\pgfpathlineto{\pgfqpoint{1.643838in}{3.010091in}}%
\pgfpathlineto{\pgfqpoint{1.644841in}{3.004636in}}%
\pgfpathlineto{\pgfqpoint{1.645843in}{2.981455in}}%
\pgfpathlineto{\pgfqpoint{1.646845in}{2.988273in}}%
\pgfpathlineto{\pgfqpoint{1.647847in}{2.977364in}}%
\pgfpathlineto{\pgfqpoint{1.648850in}{2.996455in}}%
\pgfpathlineto{\pgfqpoint{1.650854in}{2.984182in}}%
\pgfpathlineto{\pgfqpoint{1.651856in}{2.965091in}}%
\pgfpathlineto{\pgfqpoint{1.654863in}{2.984182in}}%
\pgfpathlineto{\pgfqpoint{1.655865in}{2.982818in}}%
\pgfpathlineto{\pgfqpoint{1.656868in}{2.980091in}}%
\pgfpathlineto{\pgfqpoint{1.657870in}{2.958273in}}%
\pgfpathlineto{\pgfqpoint{1.658872in}{2.974636in}}%
\pgfpathlineto{\pgfqpoint{1.659874in}{2.971909in}}%
\pgfpathlineto{\pgfqpoint{1.660877in}{2.977364in}}%
\pgfpathlineto{\pgfqpoint{1.661879in}{2.974636in}}%
\pgfpathlineto{\pgfqpoint{1.663883in}{2.948727in}}%
\pgfpathlineto{\pgfqpoint{1.665888in}{2.978727in}}%
\pgfpathlineto{\pgfqpoint{1.666890in}{2.976000in}}%
\pgfpathlineto{\pgfqpoint{1.667892in}{2.969182in}}%
\pgfpathlineto{\pgfqpoint{1.668895in}{2.978727in}}%
\pgfpathlineto{\pgfqpoint{1.669897in}{2.969182in}}%
\pgfpathlineto{\pgfqpoint{1.670899in}{2.986909in}}%
\pgfpathlineto{\pgfqpoint{1.671901in}{2.982818in}}%
\pgfpathlineto{\pgfqpoint{1.672903in}{2.985545in}}%
\pgfpathlineto{\pgfqpoint{1.673906in}{2.992364in}}%
\pgfpathlineto{\pgfqpoint{1.676912in}{2.977364in}}%
\pgfpathlineto{\pgfqpoint{1.678917in}{3.012818in}}%
\pgfpathlineto{\pgfqpoint{1.680921in}{2.985545in}}%
\pgfpathlineto{\pgfqpoint{1.681924in}{2.992364in}}%
\pgfpathlineto{\pgfqpoint{1.682926in}{3.016909in}}%
\pgfpathlineto{\pgfqpoint{1.683928in}{3.011455in}}%
\pgfpathlineto{\pgfqpoint{1.685933in}{2.991000in}}%
\pgfpathlineto{\pgfqpoint{1.687937in}{3.001909in}}%
\pgfpathlineto{\pgfqpoint{1.688939in}{3.018273in}}%
\pgfpathlineto{\pgfqpoint{1.689942in}{2.996455in}}%
\pgfpathlineto{\pgfqpoint{1.690944in}{2.999182in}}%
\pgfpathlineto{\pgfqpoint{1.691946in}{2.984182in}}%
\pgfpathlineto{\pgfqpoint{1.693951in}{3.006000in}}%
\pgfpathlineto{\pgfqpoint{1.694953in}{2.988273in}}%
\pgfpathlineto{\pgfqpoint{1.695955in}{2.996455in}}%
\pgfpathlineto{\pgfqpoint{1.697960in}{2.973273in}}%
\pgfpathlineto{\pgfqpoint{1.698962in}{2.991000in}}%
\pgfpathlineto{\pgfqpoint{1.699964in}{2.984182in}}%
\pgfpathlineto{\pgfqpoint{1.700966in}{2.986909in}}%
\pgfpathlineto{\pgfqpoint{1.702971in}{2.961000in}}%
\pgfpathlineto{\pgfqpoint{1.703973in}{2.954182in}}%
\pgfpathlineto{\pgfqpoint{1.705977in}{2.977364in}}%
\pgfpathlineto{\pgfqpoint{1.707982in}{2.941909in}}%
\pgfpathlineto{\pgfqpoint{1.710989in}{2.974636in}}%
\pgfpathlineto{\pgfqpoint{1.713995in}{2.946000in}}%
\pgfpathlineto{\pgfqpoint{1.716000in}{2.977364in}}%
\pgfpathlineto{\pgfqpoint{1.717002in}{2.973273in}}%
\pgfpathlineto{\pgfqpoint{1.718004in}{3.003273in}}%
\pgfpathlineto{\pgfqpoint{1.719007in}{2.974636in}}%
\pgfpathlineto{\pgfqpoint{1.720009in}{2.977364in}}%
\pgfpathlineto{\pgfqpoint{1.722013in}{2.991000in}}%
\pgfpathlineto{\pgfqpoint{1.724018in}{3.012818in}}%
\pgfpathlineto{\pgfqpoint{1.726022in}{2.991000in}}%
\pgfpathlineto{\pgfqpoint{1.727025in}{3.004636in}}%
\pgfpathlineto{\pgfqpoint{1.728027in}{3.038727in}}%
\pgfpathlineto{\pgfqpoint{1.729029in}{3.029182in}}%
\pgfpathlineto{\pgfqpoint{1.731034in}{3.000545in}}%
\pgfpathlineto{\pgfqpoint{1.733038in}{3.036000in}}%
\pgfpathlineto{\pgfqpoint{1.734040in}{3.031909in}}%
\pgfpathlineto{\pgfqpoint{1.736045in}{2.988273in}}%
\pgfpathlineto{\pgfqpoint{1.738049in}{3.018273in}}%
\pgfpathlineto{\pgfqpoint{1.741056in}{2.981455in}}%
\pgfpathlineto{\pgfqpoint{1.742058in}{2.981455in}}%
\pgfpathlineto{\pgfqpoint{1.743060in}{2.984182in}}%
\pgfpathlineto{\pgfqpoint{1.744063in}{2.956909in}}%
\pgfpathlineto{\pgfqpoint{1.745065in}{2.980091in}}%
\pgfpathlineto{\pgfqpoint{1.746067in}{2.974636in}}%
\pgfpathlineto{\pgfqpoint{1.749074in}{2.937818in}}%
\pgfpathlineto{\pgfqpoint{1.751078in}{2.981455in}}%
\pgfpathlineto{\pgfqpoint{1.754085in}{2.926909in}}%
\pgfpathlineto{\pgfqpoint{1.756090in}{2.966455in}}%
\pgfpathlineto{\pgfqpoint{1.757092in}{2.982818in}}%
\pgfpathlineto{\pgfqpoint{1.759096in}{2.947364in}}%
\pgfpathlineto{\pgfqpoint{1.761101in}{2.976000in}}%
\pgfpathlineto{\pgfqpoint{1.762103in}{2.991000in}}%
\pgfpathlineto{\pgfqpoint{1.763105in}{2.980091in}}%
\pgfpathlineto{\pgfqpoint{1.765110in}{2.996455in}}%
\pgfpathlineto{\pgfqpoint{1.766112in}{2.996455in}}%
\pgfpathlineto{\pgfqpoint{1.769119in}{3.019636in}}%
\pgfpathlineto{\pgfqpoint{1.771123in}{2.989636in}}%
\pgfpathlineto{\pgfqpoint{1.773128in}{3.049636in}}%
\pgfpathlineto{\pgfqpoint{1.776134in}{2.984182in}}%
\pgfpathlineto{\pgfqpoint{1.777137in}{2.997818in}}%
\pgfpathlineto{\pgfqpoint{1.779141in}{3.025091in}}%
\pgfpathlineto{\pgfqpoint{1.781146in}{2.973273in}}%
\pgfpathlineto{\pgfqpoint{1.784152in}{3.006000in}}%
\pgfpathlineto{\pgfqpoint{1.786157in}{2.970545in}}%
\pgfpathlineto{\pgfqpoint{1.787159in}{2.971909in}}%
\pgfpathlineto{\pgfqpoint{1.788161in}{2.951455in}}%
\pgfpathlineto{\pgfqpoint{1.790166in}{2.977364in}}%
\pgfpathlineto{\pgfqpoint{1.792170in}{2.959636in}}%
\pgfpathlineto{\pgfqpoint{1.794175in}{2.928273in}}%
\pgfpathlineto{\pgfqpoint{1.795177in}{2.970545in}}%
\pgfpathlineto{\pgfqpoint{1.796179in}{2.969182in}}%
\pgfpathlineto{\pgfqpoint{1.798184in}{2.931000in}}%
\pgfpathlineto{\pgfqpoint{1.801191in}{2.969182in}}%
\pgfpathlineto{\pgfqpoint{1.802193in}{2.951455in}}%
\pgfpathlineto{\pgfqpoint{1.804197in}{2.965091in}}%
\pgfpathlineto{\pgfqpoint{1.805200in}{2.961000in}}%
\pgfpathlineto{\pgfqpoint{1.806202in}{2.996455in}}%
\pgfpathlineto{\pgfqpoint{1.807204in}{2.978727in}}%
\pgfpathlineto{\pgfqpoint{1.808206in}{2.980091in}}%
\pgfpathlineto{\pgfqpoint{1.809209in}{2.982818in}}%
\pgfpathlineto{\pgfqpoint{1.810211in}{2.967818in}}%
\pgfpathlineto{\pgfqpoint{1.813217in}{3.010091in}}%
\pgfpathlineto{\pgfqpoint{1.814220in}{3.031909in}}%
\pgfpathlineto{\pgfqpoint{1.815222in}{2.993727in}}%
\pgfpathlineto{\pgfqpoint{1.816224in}{2.999182in}}%
\pgfpathlineto{\pgfqpoint{1.817226in}{2.997818in}}%
\pgfpathlineto{\pgfqpoint{1.819231in}{3.034636in}}%
\pgfpathlineto{\pgfqpoint{1.821235in}{2.992364in}}%
\pgfpathlineto{\pgfqpoint{1.822238in}{2.993727in}}%
\pgfpathlineto{\pgfqpoint{1.823240in}{3.023727in}}%
\pgfpathlineto{\pgfqpoint{1.824242in}{3.015545in}}%
\pgfpathlineto{\pgfqpoint{1.825244in}{2.995091in}}%
\pgfpathlineto{\pgfqpoint{1.826247in}{2.997818in}}%
\pgfpathlineto{\pgfqpoint{1.827249in}{2.961000in}}%
\pgfpathlineto{\pgfqpoint{1.828251in}{3.015545in}}%
\pgfpathlineto{\pgfqpoint{1.830256in}{2.959636in}}%
\pgfpathlineto{\pgfqpoint{1.831258in}{2.991000in}}%
\pgfpathlineto{\pgfqpoint{1.832260in}{2.965091in}}%
\pgfpathlineto{\pgfqpoint{1.833262in}{2.974636in}}%
\pgfpathlineto{\pgfqpoint{1.834265in}{2.943273in}}%
\pgfpathlineto{\pgfqpoint{1.836269in}{2.970545in}}%
\pgfpathlineto{\pgfqpoint{1.838274in}{2.924182in}}%
\pgfpathlineto{\pgfqpoint{1.839276in}{2.921455in}}%
\pgfpathlineto{\pgfqpoint{1.841280in}{2.969182in}}%
\pgfpathlineto{\pgfqpoint{1.843285in}{2.903727in}}%
\pgfpathlineto{\pgfqpoint{1.844287in}{2.913273in}}%
\pgfpathlineto{\pgfqpoint{1.846291in}{2.973273in}}%
\pgfpathlineto{\pgfqpoint{1.848296in}{2.941909in}}%
\pgfpathlineto{\pgfqpoint{1.849298in}{2.936455in}}%
\pgfpathlineto{\pgfqpoint{1.851303in}{2.989636in}}%
\pgfpathlineto{\pgfqpoint{1.852305in}{2.993727in}}%
\pgfpathlineto{\pgfqpoint{1.853307in}{2.991000in}}%
\pgfpathlineto{\pgfqpoint{1.854309in}{2.958273in}}%
\pgfpathlineto{\pgfqpoint{1.858318in}{3.038727in}}%
\pgfpathlineto{\pgfqpoint{1.861325in}{2.962364in}}%
\pgfpathlineto{\pgfqpoint{1.863330in}{3.036000in}}%
\pgfpathlineto{\pgfqpoint{1.864332in}{3.027818in}}%
\pgfpathlineto{\pgfqpoint{1.866336in}{2.985545in}}%
\pgfpathlineto{\pgfqpoint{1.868341in}{3.014182in}}%
\pgfpathlineto{\pgfqpoint{1.869343in}{3.026455in}}%
\pgfpathlineto{\pgfqpoint{1.871348in}{2.977364in}}%
\pgfpathlineto{\pgfqpoint{1.873352in}{3.007364in}}%
\pgfpathlineto{\pgfqpoint{1.874354in}{3.004636in}}%
\pgfpathlineto{\pgfqpoint{1.875357in}{2.995091in}}%
\pgfpathlineto{\pgfqpoint{1.876359in}{2.959636in}}%
\pgfpathlineto{\pgfqpoint{1.877361in}{2.991000in}}%
\pgfpathlineto{\pgfqpoint{1.879366in}{2.954182in}}%
\pgfpathlineto{\pgfqpoint{1.880368in}{2.980091in}}%
\pgfpathlineto{\pgfqpoint{1.881370in}{2.959636in}}%
\pgfpathlineto{\pgfqpoint{1.882372in}{2.967818in}}%
\pgfpathlineto{\pgfqpoint{1.884377in}{2.935091in}}%
\pgfpathlineto{\pgfqpoint{1.886381in}{2.974636in}}%
\pgfpathlineto{\pgfqpoint{1.888386in}{2.913273in}}%
\pgfpathlineto{\pgfqpoint{1.889388in}{2.913273in}}%
\pgfpathlineto{\pgfqpoint{1.891392in}{2.970545in}}%
\pgfpathlineto{\pgfqpoint{1.894399in}{2.905091in}}%
\pgfpathlineto{\pgfqpoint{1.896404in}{2.966455in}}%
\pgfpathlineto{\pgfqpoint{1.897406in}{2.966455in}}%
\pgfpathlineto{\pgfqpoint{1.898408in}{2.914636in}}%
\pgfpathlineto{\pgfqpoint{1.900413in}{2.963727in}}%
\pgfpathlineto{\pgfqpoint{1.901415in}{2.969182in}}%
\pgfpathlineto{\pgfqpoint{1.902417in}{2.992364in}}%
\pgfpathlineto{\pgfqpoint{1.903419in}{2.965091in}}%
\pgfpathlineto{\pgfqpoint{1.905424in}{3.000545in}}%
\pgfpathlineto{\pgfqpoint{1.906426in}{2.995091in}}%
\pgfpathlineto{\pgfqpoint{1.907428in}{3.042818in}}%
\pgfpathlineto{\pgfqpoint{1.908431in}{3.029182in}}%
\pgfpathlineto{\pgfqpoint{1.909433in}{3.034636in}}%
\pgfpathlineto{\pgfqpoint{1.911437in}{2.992364in}}%
\pgfpathlineto{\pgfqpoint{1.913442in}{3.056455in}}%
\pgfpathlineto{\pgfqpoint{1.914444in}{3.053727in}}%
\pgfpathlineto{\pgfqpoint{1.915446in}{3.040091in}}%
\pgfpathlineto{\pgfqpoint{1.916449in}{3.007364in}}%
\pgfpathlineto{\pgfqpoint{1.919455in}{3.061909in}}%
\pgfpathlineto{\pgfqpoint{1.921460in}{2.985545in}}%
\pgfpathlineto{\pgfqpoint{1.922462in}{2.993727in}}%
\pgfpathlineto{\pgfqpoint{1.924466in}{3.021000in}}%
\pgfpathlineto{\pgfqpoint{1.926471in}{2.963727in}}%
\pgfpathlineto{\pgfqpoint{1.927473in}{2.962364in}}%
\pgfpathlineto{\pgfqpoint{1.928475in}{2.937818in}}%
\pgfpathlineto{\pgfqpoint{1.929478in}{2.962364in}}%
\pgfpathlineto{\pgfqpoint{1.930480in}{2.961000in}}%
\pgfpathlineto{\pgfqpoint{1.931482in}{2.952818in}}%
\pgfpathlineto{\pgfqpoint{1.933487in}{2.898273in}}%
\pgfpathlineto{\pgfqpoint{1.934489in}{2.902364in}}%
\pgfpathlineto{\pgfqpoint{1.935491in}{2.951455in}}%
\pgfpathlineto{\pgfqpoint{1.938498in}{2.903727in}}%
\pgfpathlineto{\pgfqpoint{1.939500in}{2.921455in}}%
\pgfpathlineto{\pgfqpoint{1.940502in}{2.976000in}}%
\pgfpathlineto{\pgfqpoint{1.941505in}{2.969182in}}%
\pgfpathlineto{\pgfqpoint{1.943509in}{2.926909in}}%
\pgfpathlineto{\pgfqpoint{1.944511in}{2.932364in}}%
\pgfpathlineto{\pgfqpoint{1.945514in}{2.948727in}}%
\pgfpathlineto{\pgfqpoint{1.946516in}{2.986909in}}%
\pgfpathlineto{\pgfqpoint{1.947518in}{2.982818in}}%
\pgfpathlineto{\pgfqpoint{1.948520in}{3.006000in}}%
\pgfpathlineto{\pgfqpoint{1.949523in}{2.992364in}}%
\pgfpathlineto{\pgfqpoint{1.950525in}{3.003273in}}%
\pgfpathlineto{\pgfqpoint{1.951527in}{2.999182in}}%
\pgfpathlineto{\pgfqpoint{1.953531in}{2.955545in}}%
\pgfpathlineto{\pgfqpoint{1.954534in}{3.007364in}}%
\pgfpathlineto{\pgfqpoint{1.955536in}{3.004636in}}%
\pgfpathlineto{\pgfqpoint{1.956538in}{2.974636in}}%
\pgfpathlineto{\pgfqpoint{1.957540in}{3.025091in}}%
\pgfpathlineto{\pgfqpoint{1.958543in}{2.985545in}}%
\pgfpathlineto{\pgfqpoint{1.960547in}{3.010091in}}%
\pgfpathlineto{\pgfqpoint{1.962552in}{2.950091in}}%
\pgfpathlineto{\pgfqpoint{1.964556in}{3.063273in}}%
\pgfpathlineto{\pgfqpoint{1.966561in}{2.980091in}}%
\pgfpathlineto{\pgfqpoint{1.968565in}{3.026455in}}%
\pgfpathlineto{\pgfqpoint{1.970570in}{2.948727in}}%
\pgfpathlineto{\pgfqpoint{1.971572in}{2.947364in}}%
\pgfpathlineto{\pgfqpoint{1.972574in}{2.973273in}}%
\pgfpathlineto{\pgfqpoint{1.974579in}{2.895545in}}%
\pgfpathlineto{\pgfqpoint{1.976583in}{2.976000in}}%
\pgfpathlineto{\pgfqpoint{1.977585in}{2.955545in}}%
\pgfpathlineto{\pgfqpoint{1.978588in}{2.886000in}}%
\pgfpathlineto{\pgfqpoint{1.980592in}{2.932364in}}%
\pgfpathlineto{\pgfqpoint{1.981594in}{2.993727in}}%
\pgfpathlineto{\pgfqpoint{1.983599in}{2.911909in}}%
\pgfpathlineto{\pgfqpoint{1.984601in}{2.903727in}}%
\pgfpathlineto{\pgfqpoint{1.986606in}{2.961000in}}%
\pgfpathlineto{\pgfqpoint{1.987608in}{2.937818in}}%
\pgfpathlineto{\pgfqpoint{1.988610in}{2.965091in}}%
\pgfpathlineto{\pgfqpoint{1.989612in}{2.922818in}}%
\pgfpathlineto{\pgfqpoint{1.991617in}{2.952818in}}%
\pgfpathlineto{\pgfqpoint{1.992619in}{3.019636in}}%
\pgfpathlineto{\pgfqpoint{1.993621in}{3.006000in}}%
\pgfpathlineto{\pgfqpoint{1.994623in}{2.961000in}}%
\pgfpathlineto{\pgfqpoint{1.995626in}{2.962364in}}%
\pgfpathlineto{\pgfqpoint{1.998632in}{3.064636in}}%
\pgfpathlineto{\pgfqpoint{1.999635in}{2.996455in}}%
\pgfpathlineto{\pgfqpoint{2.000637in}{3.003273in}}%
\pgfpathlineto{\pgfqpoint{2.001639in}{2.997818in}}%
\pgfpathlineto{\pgfqpoint{2.003644in}{3.081000in}}%
\pgfpathlineto{\pgfqpoint{2.004646in}{3.071455in}}%
\pgfpathlineto{\pgfqpoint{2.005648in}{3.044182in}}%
\pgfpathlineto{\pgfqpoint{2.006650in}{2.974636in}}%
\pgfpathlineto{\pgfqpoint{2.008655in}{3.121909in}}%
\pgfpathlineto{\pgfqpoint{2.011662in}{2.963727in}}%
\pgfpathlineto{\pgfqpoint{2.013666in}{3.037364in}}%
\pgfpathlineto{\pgfqpoint{2.016673in}{2.961000in}}%
\pgfpathlineto{\pgfqpoint{2.017675in}{2.982818in}}%
\pgfpathlineto{\pgfqpoint{2.019680in}{2.917364in}}%
\pgfpathlineto{\pgfqpoint{2.020682in}{2.946000in}}%
\pgfpathlineto{\pgfqpoint{2.021684in}{2.943273in}}%
\pgfpathlineto{\pgfqpoint{2.024691in}{2.883273in}}%
\pgfpathlineto{\pgfqpoint{2.026695in}{2.943273in}}%
\pgfpathlineto{\pgfqpoint{2.028700in}{2.892818in}}%
\pgfpathlineto{\pgfqpoint{2.029702in}{2.899636in}}%
\pgfpathlineto{\pgfqpoint{2.031706in}{2.958273in}}%
\pgfpathlineto{\pgfqpoint{2.033711in}{2.906455in}}%
\pgfpathlineto{\pgfqpoint{2.036718in}{2.984182in}}%
\pgfpathlineto{\pgfqpoint{2.037720in}{2.981455in}}%
\pgfpathlineto{\pgfqpoint{2.038722in}{2.903727in}}%
\pgfpathlineto{\pgfqpoint{2.040727in}{2.944636in}}%
\pgfpathlineto{\pgfqpoint{2.041729in}{2.977364in}}%
\pgfpathlineto{\pgfqpoint{2.042731in}{3.057818in}}%
\pgfpathlineto{\pgfqpoint{2.043733in}{3.046909in}}%
\pgfpathlineto{\pgfqpoint{2.045738in}{3.007364in}}%
\pgfpathlineto{\pgfqpoint{2.046740in}{2.991000in}}%
\pgfpathlineto{\pgfqpoint{2.048745in}{3.102818in}}%
\pgfpathlineto{\pgfqpoint{2.051751in}{3.018273in}}%
\pgfpathlineto{\pgfqpoint{2.053756in}{3.078273in}}%
\pgfpathlineto{\pgfqpoint{2.054758in}{3.098727in}}%
\pgfpathlineto{\pgfqpoint{2.056763in}{2.943273in}}%
\pgfpathlineto{\pgfqpoint{2.057765in}{2.995091in}}%
\pgfpathlineto{\pgfqpoint{2.058767in}{3.117818in}}%
\pgfpathlineto{\pgfqpoint{2.059769in}{3.113727in}}%
\pgfpathlineto{\pgfqpoint{2.061774in}{2.997818in}}%
\pgfpathlineto{\pgfqpoint{2.062776in}{2.941909in}}%
\pgfpathlineto{\pgfqpoint{2.063778in}{2.950091in}}%
\pgfpathlineto{\pgfqpoint{2.064780in}{3.004636in}}%
\pgfpathlineto{\pgfqpoint{2.067787in}{2.941909in}}%
\pgfpathlineto{\pgfqpoint{2.068789in}{2.865545in}}%
\pgfpathlineto{\pgfqpoint{2.070794in}{2.936455in}}%
\pgfpathlineto{\pgfqpoint{2.071796in}{2.937818in}}%
\pgfpathlineto{\pgfqpoint{2.073801in}{2.854636in}}%
\pgfpathlineto{\pgfqpoint{2.074803in}{2.861455in}}%
\pgfpathlineto{\pgfqpoint{2.076807in}{2.939182in}}%
\pgfpathlineto{\pgfqpoint{2.078812in}{2.836909in}}%
\pgfpathlineto{\pgfqpoint{2.081819in}{2.936455in}}%
\pgfpathlineto{\pgfqpoint{2.083823in}{2.911909in}}%
\pgfpathlineto{\pgfqpoint{2.084825in}{2.913273in}}%
\pgfpathlineto{\pgfqpoint{2.086830in}{2.974636in}}%
\pgfpathlineto{\pgfqpoint{2.087832in}{2.944636in}}%
\pgfpathlineto{\pgfqpoint{2.091841in}{3.027818in}}%
\pgfpathlineto{\pgfqpoint{2.092843in}{3.018273in}}%
\pgfpathlineto{\pgfqpoint{2.093846in}{3.022364in}}%
\pgfpathlineto{\pgfqpoint{2.094848in}{3.090545in}}%
\pgfpathlineto{\pgfqpoint{2.096852in}{3.023727in}}%
\pgfpathlineto{\pgfqpoint{2.099859in}{3.181909in}}%
\pgfpathlineto{\pgfqpoint{2.100861in}{3.022364in}}%
\pgfpathlineto{\pgfqpoint{2.101863in}{3.023727in}}%
\pgfpathlineto{\pgfqpoint{2.103868in}{3.124636in}}%
\pgfpathlineto{\pgfqpoint{2.104870in}{3.115091in}}%
\pgfpathlineto{\pgfqpoint{2.106875in}{3.023727in}}%
\pgfpathlineto{\pgfqpoint{2.107877in}{3.037364in}}%
\pgfpathlineto{\pgfqpoint{2.109881in}{2.980091in}}%
\pgfpathlineto{\pgfqpoint{2.110884in}{2.988273in}}%
\pgfpathlineto{\pgfqpoint{2.111886in}{2.999182in}}%
\pgfpathlineto{\pgfqpoint{2.112888in}{2.939182in}}%
\pgfpathlineto{\pgfqpoint{2.113890in}{2.963727in}}%
\pgfpathlineto{\pgfqpoint{2.115895in}{2.947364in}}%
\pgfpathlineto{\pgfqpoint{2.116897in}{2.941909in}}%
\pgfpathlineto{\pgfqpoint{2.118902in}{2.864182in}}%
\pgfpathlineto{\pgfqpoint{2.120906in}{2.925545in}}%
\pgfpathlineto{\pgfqpoint{2.121908in}{2.925545in}}%
\pgfpathlineto{\pgfqpoint{2.123913in}{2.850545in}}%
\pgfpathlineto{\pgfqpoint{2.124915in}{2.862818in}}%
\pgfpathlineto{\pgfqpoint{2.126920in}{2.921455in}}%
\pgfpathlineto{\pgfqpoint{2.128924in}{2.868273in}}%
\pgfpathlineto{\pgfqpoint{2.130928in}{2.962364in}}%
\pgfpathlineto{\pgfqpoint{2.131931in}{2.959636in}}%
\pgfpathlineto{\pgfqpoint{2.132933in}{2.935091in}}%
\pgfpathlineto{\pgfqpoint{2.134937in}{2.961000in}}%
\pgfpathlineto{\pgfqpoint{2.136942in}{3.018273in}}%
\pgfpathlineto{\pgfqpoint{2.137944in}{3.014182in}}%
\pgfpathlineto{\pgfqpoint{2.138946in}{3.072818in}}%
\pgfpathlineto{\pgfqpoint{2.139949in}{3.063273in}}%
\pgfpathlineto{\pgfqpoint{2.140951in}{3.040091in}}%
\pgfpathlineto{\pgfqpoint{2.142955in}{3.094636in}}%
\pgfpathlineto{\pgfqpoint{2.143958in}{3.304636in}}%
\pgfpathlineto{\pgfqpoint{2.145962in}{3.052364in}}%
\pgfpathlineto{\pgfqpoint{2.146964in}{3.037364in}}%
\pgfpathlineto{\pgfqpoint{2.148969in}{3.258273in}}%
\pgfpathlineto{\pgfqpoint{2.150973in}{3.025091in}}%
\pgfpathlineto{\pgfqpoint{2.151976in}{3.014182in}}%
\pgfpathlineto{\pgfqpoint{2.153980in}{3.079636in}}%
\pgfpathlineto{\pgfqpoint{2.158991in}{2.946000in}}%
\pgfpathlineto{\pgfqpoint{2.159994in}{2.944636in}}%
\pgfpathlineto{\pgfqpoint{2.160996in}{2.958273in}}%
\pgfpathlineto{\pgfqpoint{2.161998in}{2.937818in}}%
\pgfpathlineto{\pgfqpoint{2.164003in}{2.872364in}}%
\pgfpathlineto{\pgfqpoint{2.165005in}{2.880545in}}%
\pgfpathlineto{\pgfqpoint{2.166007in}{2.931000in}}%
\pgfpathlineto{\pgfqpoint{2.167009in}{2.916000in}}%
\pgfpathlineto{\pgfqpoint{2.169014in}{2.830091in}}%
\pgfpathlineto{\pgfqpoint{2.170016in}{2.846455in}}%
\pgfpathlineto{\pgfqpoint{2.172020in}{2.921455in}}%
\pgfpathlineto{\pgfqpoint{2.174025in}{2.851909in}}%
\pgfpathlineto{\pgfqpoint{2.175027in}{2.858727in}}%
\pgfpathlineto{\pgfqpoint{2.177032in}{2.946000in}}%
\pgfpathlineto{\pgfqpoint{2.179036in}{2.896909in}}%
\pgfpathlineto{\pgfqpoint{2.181041in}{2.989636in}}%
\pgfpathlineto{\pgfqpoint{2.182043in}{2.991000in}}%
\pgfpathlineto{\pgfqpoint{2.183045in}{3.004636in}}%
\pgfpathlineto{\pgfqpoint{2.184047in}{2.982818in}}%
\pgfpathlineto{\pgfqpoint{2.186052in}{3.025091in}}%
\pgfpathlineto{\pgfqpoint{2.187054in}{3.019636in}}%
\pgfpathlineto{\pgfqpoint{2.189059in}{3.119182in}}%
\pgfpathlineto{\pgfqpoint{2.192065in}{3.052364in}}%
\pgfpathlineto{\pgfqpoint{2.194070in}{3.382364in}}%
\pgfpathlineto{\pgfqpoint{2.197077in}{3.063273in}}%
\pgfpathlineto{\pgfqpoint{2.199081in}{3.255545in}}%
\pgfpathlineto{\pgfqpoint{2.201085in}{3.044182in}}%
\pgfpathlineto{\pgfqpoint{2.202088in}{3.052364in}}%
\pgfpathlineto{\pgfqpoint{2.203090in}{3.085091in}}%
\pgfpathlineto{\pgfqpoint{2.207099in}{2.963727in}}%
\pgfpathlineto{\pgfqpoint{2.208101in}{2.993727in}}%
\pgfpathlineto{\pgfqpoint{2.209103in}{2.962364in}}%
\pgfpathlineto{\pgfqpoint{2.210106in}{2.976000in}}%
\pgfpathlineto{\pgfqpoint{2.211108in}{2.952818in}}%
\pgfpathlineto{\pgfqpoint{2.214115in}{2.832818in}}%
\pgfpathlineto{\pgfqpoint{2.216119in}{2.922818in}}%
\pgfpathlineto{\pgfqpoint{2.218124in}{2.847818in}}%
\pgfpathlineto{\pgfqpoint{2.219126in}{2.790545in}}%
\pgfpathlineto{\pgfqpoint{2.221130in}{2.907818in}}%
\pgfpathlineto{\pgfqpoint{2.222133in}{2.920091in}}%
\pgfpathlineto{\pgfqpoint{2.224137in}{2.845091in}}%
\pgfpathlineto{\pgfqpoint{2.227144in}{2.951455in}}%
\pgfpathlineto{\pgfqpoint{2.228146in}{2.890091in}}%
\pgfpathlineto{\pgfqpoint{2.229148in}{2.894182in}}%
\pgfpathlineto{\pgfqpoint{2.232155in}{3.003273in}}%
\pgfpathlineto{\pgfqpoint{2.233157in}{3.001909in}}%
\pgfpathlineto{\pgfqpoint{2.234160in}{3.006000in}}%
\pgfpathlineto{\pgfqpoint{2.235162in}{3.034636in}}%
\pgfpathlineto{\pgfqpoint{2.236164in}{3.010091in}}%
\pgfpathlineto{\pgfqpoint{2.237166in}{3.037364in}}%
\pgfpathlineto{\pgfqpoint{2.240173in}{3.251455in}}%
\pgfpathlineto{\pgfqpoint{2.241175in}{3.067364in}}%
\pgfpathlineto{\pgfqpoint{2.242177in}{3.098727in}}%
\pgfpathlineto{\pgfqpoint{2.244182in}{3.461455in}}%
\pgfpathlineto{\pgfqpoint{2.245184in}{3.420545in}}%
\pgfpathlineto{\pgfqpoint{2.247189in}{3.089182in}}%
\pgfpathlineto{\pgfqpoint{2.249193in}{3.524182in}}%
\pgfpathlineto{\pgfqpoint{2.251198in}{3.031909in}}%
\pgfpathlineto{\pgfqpoint{2.253202in}{3.025091in}}%
\pgfpathlineto{\pgfqpoint{2.254204in}{3.115091in}}%
\pgfpathlineto{\pgfqpoint{2.259216in}{2.907818in}}%
\pgfpathlineto{\pgfqpoint{2.261220in}{2.917364in}}%
\pgfpathlineto{\pgfqpoint{2.262222in}{2.947364in}}%
\pgfpathlineto{\pgfqpoint{2.264227in}{2.835545in}}%
\pgfpathlineto{\pgfqpoint{2.267234in}{2.895545in}}%
\pgfpathlineto{\pgfqpoint{2.269238in}{2.813727in}}%
\pgfpathlineto{\pgfqpoint{2.271242in}{2.901000in}}%
\pgfpathlineto{\pgfqpoint{2.272245in}{2.917364in}}%
\pgfpathlineto{\pgfqpoint{2.274249in}{2.836909in}}%
\pgfpathlineto{\pgfqpoint{2.276254in}{2.921455in}}%
\pgfpathlineto{\pgfqpoint{2.277256in}{2.922818in}}%
\pgfpathlineto{\pgfqpoint{2.279260in}{2.898273in}}%
\pgfpathlineto{\pgfqpoint{2.280263in}{2.913273in}}%
\pgfpathlineto{\pgfqpoint{2.282267in}{2.991000in}}%
\pgfpathlineto{\pgfqpoint{2.283269in}{2.966455in}}%
\pgfpathlineto{\pgfqpoint{2.284272in}{3.051000in}}%
\pgfpathlineto{\pgfqpoint{2.286276in}{3.012818in}}%
\pgfpathlineto{\pgfqpoint{2.287278in}{3.044182in}}%
\pgfpathlineto{\pgfqpoint{2.288281in}{3.034636in}}%
\pgfpathlineto{\pgfqpoint{2.289283in}{3.177818in}}%
\pgfpathlineto{\pgfqpoint{2.291287in}{3.070091in}}%
\pgfpathlineto{\pgfqpoint{2.292290in}{3.090545in}}%
\pgfpathlineto{\pgfqpoint{2.293292in}{3.192818in}}%
\pgfpathlineto{\pgfqpoint{2.294294in}{3.503727in}}%
\pgfpathlineto{\pgfqpoint{2.296299in}{3.086455in}}%
\pgfpathlineto{\pgfqpoint{2.297301in}{3.127364in}}%
\pgfpathlineto{\pgfqpoint{2.299305in}{3.529636in}}%
\pgfpathlineto{\pgfqpoint{2.300308in}{3.405545in}}%
\pgfpathlineto{\pgfqpoint{2.302312in}{3.078273in}}%
\pgfpathlineto{\pgfqpoint{2.304317in}{3.173727in}}%
\pgfpathlineto{\pgfqpoint{2.307323in}{2.978727in}}%
\pgfpathlineto{\pgfqpoint{2.308325in}{3.031909in}}%
\pgfpathlineto{\pgfqpoint{2.309328in}{3.021000in}}%
\pgfpathlineto{\pgfqpoint{2.312334in}{2.914636in}}%
\pgfpathlineto{\pgfqpoint{2.313337in}{2.898273in}}%
\pgfpathlineto{\pgfqpoint{2.314339in}{2.924182in}}%
\pgfpathlineto{\pgfqpoint{2.315341in}{2.888727in}}%
\pgfpathlineto{\pgfqpoint{2.316343in}{2.935091in}}%
\pgfpathlineto{\pgfqpoint{2.319350in}{2.816455in}}%
\pgfpathlineto{\pgfqpoint{2.321355in}{2.907818in}}%
\pgfpathlineto{\pgfqpoint{2.322357in}{2.888727in}}%
\pgfpathlineto{\pgfqpoint{2.324361in}{2.820545in}}%
\pgfpathlineto{\pgfqpoint{2.325364in}{2.857364in}}%
\pgfpathlineto{\pgfqpoint{2.326366in}{2.937818in}}%
\pgfpathlineto{\pgfqpoint{2.327368in}{2.914636in}}%
\pgfpathlineto{\pgfqpoint{2.329373in}{2.839636in}}%
\pgfpathlineto{\pgfqpoint{2.331377in}{2.991000in}}%
\pgfpathlineto{\pgfqpoint{2.334384in}{2.905091in}}%
\pgfpathlineto{\pgfqpoint{2.337391in}{3.021000in}}%
\pgfpathlineto{\pgfqpoint{2.338393in}{3.040091in}}%
\pgfpathlineto{\pgfqpoint{2.339395in}{3.022364in}}%
\pgfpathlineto{\pgfqpoint{2.340397in}{3.031909in}}%
\pgfpathlineto{\pgfqpoint{2.341400in}{3.064636in}}%
\pgfpathlineto{\pgfqpoint{2.342402in}{3.055091in}}%
\pgfpathlineto{\pgfqpoint{2.344406in}{3.472364in}}%
\pgfpathlineto{\pgfqpoint{2.346411in}{3.168273in}}%
\pgfpathlineto{\pgfqpoint{2.347413in}{3.184636in}}%
\pgfpathlineto{\pgfqpoint{2.349417in}{3.655091in}}%
\pgfpathlineto{\pgfqpoint{2.352424in}{3.154636in}}%
\pgfpathlineto{\pgfqpoint{2.354429in}{3.601909in}}%
\pgfpathlineto{\pgfqpoint{2.356433in}{3.049636in}}%
\pgfpathlineto{\pgfqpoint{2.357435in}{3.025091in}}%
\pgfpathlineto{\pgfqpoint{2.359440in}{3.104182in}}%
\pgfpathlineto{\pgfqpoint{2.361444in}{2.991000in}}%
\pgfpathlineto{\pgfqpoint{2.365453in}{2.940545in}}%
\pgfpathlineto{\pgfqpoint{2.366456in}{2.941909in}}%
\pgfpathlineto{\pgfqpoint{2.369462in}{2.830091in}}%
\pgfpathlineto{\pgfqpoint{2.371467in}{2.906455in}}%
\pgfpathlineto{\pgfqpoint{2.373471in}{2.800091in}}%
\pgfpathlineto{\pgfqpoint{2.374474in}{2.798727in}}%
\pgfpathlineto{\pgfqpoint{2.375476in}{2.813727in}}%
\pgfpathlineto{\pgfqpoint{2.376478in}{2.890091in}}%
\pgfpathlineto{\pgfqpoint{2.378482in}{2.791909in}}%
\pgfpathlineto{\pgfqpoint{2.379485in}{2.796000in}}%
\pgfpathlineto{\pgfqpoint{2.381489in}{2.921455in}}%
\pgfpathlineto{\pgfqpoint{2.382491in}{2.886000in}}%
\pgfpathlineto{\pgfqpoint{2.383494in}{2.820545in}}%
\pgfpathlineto{\pgfqpoint{2.384496in}{2.847818in}}%
\pgfpathlineto{\pgfqpoint{2.386500in}{2.959636in}}%
\pgfpathlineto{\pgfqpoint{2.387503in}{2.969182in}}%
\pgfpathlineto{\pgfqpoint{2.389507in}{2.916000in}}%
\pgfpathlineto{\pgfqpoint{2.391512in}{3.045545in}}%
\pgfpathlineto{\pgfqpoint{2.392514in}{3.025091in}}%
\pgfpathlineto{\pgfqpoint{2.393516in}{3.131455in}}%
\pgfpathlineto{\pgfqpoint{2.395521in}{3.089182in}}%
\pgfpathlineto{\pgfqpoint{2.396523in}{3.072818in}}%
\pgfpathlineto{\pgfqpoint{2.397525in}{3.134182in}}%
\pgfpathlineto{\pgfqpoint{2.398527in}{3.511909in}}%
\pgfpathlineto{\pgfqpoint{2.399530in}{3.465545in}}%
\pgfpathlineto{\pgfqpoint{2.401534in}{3.085091in}}%
\pgfpathlineto{\pgfqpoint{2.402536in}{3.139636in}}%
\pgfpathlineto{\pgfqpoint{2.404541in}{3.653727in}}%
\pgfpathlineto{\pgfqpoint{2.405543in}{3.510545in}}%
\pgfpathlineto{\pgfqpoint{2.406545in}{3.128727in}}%
\pgfpathlineto{\pgfqpoint{2.407548in}{3.156000in}}%
\pgfpathlineto{\pgfqpoint{2.409552in}{3.642818in}}%
\pgfpathlineto{\pgfqpoint{2.411557in}{3.068727in}}%
\pgfpathlineto{\pgfqpoint{2.412559in}{3.059182in}}%
\pgfpathlineto{\pgfqpoint{2.413561in}{3.016909in}}%
\pgfpathlineto{\pgfqpoint{2.414563in}{3.031909in}}%
\pgfpathlineto{\pgfqpoint{2.417570in}{2.971909in}}%
\pgfpathlineto{\pgfqpoint{2.419574in}{2.839636in}}%
\pgfpathlineto{\pgfqpoint{2.420577in}{2.901000in}}%
\pgfpathlineto{\pgfqpoint{2.421579in}{2.895545in}}%
\pgfpathlineto{\pgfqpoint{2.422581in}{2.890091in}}%
\pgfpathlineto{\pgfqpoint{2.424586in}{2.779636in}}%
\pgfpathlineto{\pgfqpoint{2.426590in}{2.879182in}}%
\pgfpathlineto{\pgfqpoint{2.427592in}{2.860091in}}%
\pgfpathlineto{\pgfqpoint{2.429597in}{2.731909in}}%
\pgfpathlineto{\pgfqpoint{2.431601in}{2.877818in}}%
\pgfpathlineto{\pgfqpoint{2.432604in}{2.857364in}}%
\pgfpathlineto{\pgfqpoint{2.434608in}{2.748273in}}%
\pgfpathlineto{\pgfqpoint{2.436613in}{2.899636in}}%
\pgfpathlineto{\pgfqpoint{2.437615in}{2.888727in}}%
\pgfpathlineto{\pgfqpoint{2.438617in}{2.839636in}}%
\pgfpathlineto{\pgfqpoint{2.441624in}{2.971909in}}%
\pgfpathlineto{\pgfqpoint{2.442626in}{2.970545in}}%
\pgfpathlineto{\pgfqpoint{2.443628in}{2.932364in}}%
\pgfpathlineto{\pgfqpoint{2.449642in}{3.091909in}}%
\pgfpathlineto{\pgfqpoint{2.450644in}{3.108273in}}%
\pgfpathlineto{\pgfqpoint{2.451646in}{3.056455in}}%
\pgfpathlineto{\pgfqpoint{2.454653in}{3.555545in}}%
\pgfpathlineto{\pgfqpoint{2.456657in}{3.106909in}}%
\pgfpathlineto{\pgfqpoint{2.457660in}{3.247364in}}%
\pgfpathlineto{\pgfqpoint{2.459664in}{3.653727in}}%
\pgfpathlineto{\pgfqpoint{2.460666in}{3.543273in}}%
\pgfpathlineto{\pgfqpoint{2.461669in}{3.091909in}}%
\pgfpathlineto{\pgfqpoint{2.462671in}{3.177818in}}%
\pgfpathlineto{\pgfqpoint{2.464675in}{3.595091in}}%
\pgfpathlineto{\pgfqpoint{2.466680in}{3.055091in}}%
\pgfpathlineto{\pgfqpoint{2.468684in}{3.270545in}}%
\pgfpathlineto{\pgfqpoint{2.469687in}{3.195545in}}%
\pgfpathlineto{\pgfqpoint{2.471691in}{2.950091in}}%
\pgfpathlineto{\pgfqpoint{2.473696in}{3.018273in}}%
\pgfpathlineto{\pgfqpoint{2.474698in}{3.004636in}}%
\pgfpathlineto{\pgfqpoint{2.476702in}{2.905091in}}%
\pgfpathlineto{\pgfqpoint{2.478707in}{2.856000in}}%
\pgfpathlineto{\pgfqpoint{2.479709in}{2.881909in}}%
\pgfpathlineto{\pgfqpoint{2.481714in}{2.865545in}}%
\pgfpathlineto{\pgfqpoint{2.482716in}{2.836909in}}%
\pgfpathlineto{\pgfqpoint{2.483718in}{2.748273in}}%
\pgfpathlineto{\pgfqpoint{2.484720in}{2.753727in}}%
\pgfpathlineto{\pgfqpoint{2.486725in}{2.857364in}}%
\pgfpathlineto{\pgfqpoint{2.487727in}{2.820545in}}%
\pgfpathlineto{\pgfqpoint{2.489731in}{2.718273in}}%
\pgfpathlineto{\pgfqpoint{2.491736in}{2.876455in}}%
\pgfpathlineto{\pgfqpoint{2.493740in}{2.752364in}}%
\pgfpathlineto{\pgfqpoint{2.494743in}{2.753727in}}%
\pgfpathlineto{\pgfqpoint{2.496747in}{2.937818in}}%
\pgfpathlineto{\pgfqpoint{2.498752in}{2.849182in}}%
\pgfpathlineto{\pgfqpoint{2.502761in}{3.051000in}}%
\pgfpathlineto{\pgfqpoint{2.503763in}{3.071455in}}%
\pgfpathlineto{\pgfqpoint{2.505767in}{2.969182in}}%
\pgfpathlineto{\pgfqpoint{2.506770in}{2.985545in}}%
\pgfpathlineto{\pgfqpoint{2.507772in}{3.094636in}}%
\pgfpathlineto{\pgfqpoint{2.509776in}{3.667364in}}%
\pgfpathlineto{\pgfqpoint{2.511781in}{3.121909in}}%
\pgfpathlineto{\pgfqpoint{2.513785in}{3.614182in}}%
\pgfpathlineto{\pgfqpoint{2.514788in}{3.656455in}}%
\pgfpathlineto{\pgfqpoint{2.516792in}{3.104182in}}%
\pgfpathlineto{\pgfqpoint{2.518797in}{3.682364in}}%
\pgfpathlineto{\pgfqpoint{2.519799in}{3.749182in}}%
\pgfpathlineto{\pgfqpoint{2.521803in}{3.156000in}}%
\pgfpathlineto{\pgfqpoint{2.522805in}{3.289636in}}%
\pgfpathlineto{\pgfqpoint{2.524810in}{3.636000in}}%
\pgfpathlineto{\pgfqpoint{2.526814in}{3.019636in}}%
\pgfpathlineto{\pgfqpoint{2.528819in}{3.325091in}}%
\pgfpathlineto{\pgfqpoint{2.531826in}{2.931000in}}%
\pgfpathlineto{\pgfqpoint{2.533830in}{3.026455in}}%
\pgfpathlineto{\pgfqpoint{2.534832in}{3.016909in}}%
\pgfpathlineto{\pgfqpoint{2.536837in}{2.896909in}}%
\pgfpathlineto{\pgfqpoint{2.538841in}{2.841000in}}%
\pgfpathlineto{\pgfqpoint{2.539844in}{2.876455in}}%
\pgfpathlineto{\pgfqpoint{2.540846in}{2.839636in}}%
\pgfpathlineto{\pgfqpoint{2.541848in}{2.886000in}}%
\pgfpathlineto{\pgfqpoint{2.542850in}{2.846455in}}%
\pgfpathlineto{\pgfqpoint{2.543853in}{2.712818in}}%
\pgfpathlineto{\pgfqpoint{2.546859in}{2.886000in}}%
\pgfpathlineto{\pgfqpoint{2.548864in}{2.730545in}}%
\pgfpathlineto{\pgfqpoint{2.549866in}{2.744182in}}%
\pgfpathlineto{\pgfqpoint{2.551871in}{2.888727in}}%
\pgfpathlineto{\pgfqpoint{2.552873in}{2.865545in}}%
\pgfpathlineto{\pgfqpoint{2.553875in}{2.736000in}}%
\pgfpathlineto{\pgfqpoint{2.554877in}{2.753727in}}%
\pgfpathlineto{\pgfqpoint{2.556882in}{2.940545in}}%
\pgfpathlineto{\pgfqpoint{2.559888in}{2.850545in}}%
\pgfpathlineto{\pgfqpoint{2.561893in}{2.969182in}}%
\pgfpathlineto{\pgfqpoint{2.562895in}{2.982818in}}%
\pgfpathlineto{\pgfqpoint{2.563897in}{2.973273in}}%
\pgfpathlineto{\pgfqpoint{2.566904in}{3.067364in}}%
\pgfpathlineto{\pgfqpoint{2.567906in}{3.044182in}}%
\pgfpathlineto{\pgfqpoint{2.568909in}{3.091909in}}%
\pgfpathlineto{\pgfqpoint{2.569911in}{3.514636in}}%
\pgfpathlineto{\pgfqpoint{2.571915in}{3.166909in}}%
\pgfpathlineto{\pgfqpoint{2.572918in}{3.258273in}}%
\pgfpathlineto{\pgfqpoint{2.574922in}{3.685091in}}%
\pgfpathlineto{\pgfqpoint{2.576927in}{3.266455in}}%
\pgfpathlineto{\pgfqpoint{2.578931in}{3.745091in}}%
\pgfpathlineto{\pgfqpoint{2.579933in}{3.768273in}}%
\pgfpathlineto{\pgfqpoint{2.580936in}{3.641455in}}%
\pgfpathlineto{\pgfqpoint{2.581938in}{3.344182in}}%
\pgfpathlineto{\pgfqpoint{2.583942in}{3.683727in}}%
\pgfpathlineto{\pgfqpoint{2.584945in}{3.734182in}}%
\pgfpathlineto{\pgfqpoint{2.586949in}{3.066000in}}%
\pgfpathlineto{\pgfqpoint{2.587951in}{3.162818in}}%
\pgfpathlineto{\pgfqpoint{2.588954in}{3.612818in}}%
\pgfpathlineto{\pgfqpoint{2.591960in}{2.986909in}}%
\pgfpathlineto{\pgfqpoint{2.592962in}{3.000545in}}%
\pgfpathlineto{\pgfqpoint{2.593965in}{3.102818in}}%
\pgfpathlineto{\pgfqpoint{2.594967in}{3.070091in}}%
\pgfpathlineto{\pgfqpoint{2.596971in}{2.902364in}}%
\pgfpathlineto{\pgfqpoint{2.597974in}{2.851909in}}%
\pgfpathlineto{\pgfqpoint{2.598976in}{2.890091in}}%
\pgfpathlineto{\pgfqpoint{2.600980in}{2.836909in}}%
\pgfpathlineto{\pgfqpoint{2.601983in}{2.871000in}}%
\pgfpathlineto{\pgfqpoint{2.603987in}{2.716909in}}%
\pgfpathlineto{\pgfqpoint{2.604989in}{2.745545in}}%
\pgfpathlineto{\pgfqpoint{2.606994in}{2.877818in}}%
\pgfpathlineto{\pgfqpoint{2.608998in}{2.708727in}}%
\pgfpathlineto{\pgfqpoint{2.610001in}{2.725091in}}%
\pgfpathlineto{\pgfqpoint{2.612005in}{2.876455in}}%
\pgfpathlineto{\pgfqpoint{2.614010in}{2.755091in}}%
\pgfpathlineto{\pgfqpoint{2.615012in}{2.783727in}}%
\pgfpathlineto{\pgfqpoint{2.617016in}{2.926909in}}%
\pgfpathlineto{\pgfqpoint{2.618019in}{2.913273in}}%
\pgfpathlineto{\pgfqpoint{2.619021in}{2.931000in}}%
\pgfpathlineto{\pgfqpoint{2.620023in}{2.884636in}}%
\pgfpathlineto{\pgfqpoint{2.625034in}{3.145091in}}%
\pgfpathlineto{\pgfqpoint{2.626036in}{3.098727in}}%
\pgfpathlineto{\pgfqpoint{2.628041in}{3.211909in}}%
\pgfpathlineto{\pgfqpoint{2.630045in}{3.711000in}}%
\pgfpathlineto{\pgfqpoint{2.632050in}{3.385091in}}%
\pgfpathlineto{\pgfqpoint{2.634054in}{3.773727in}}%
\pgfpathlineto{\pgfqpoint{2.635057in}{3.783273in}}%
\pgfpathlineto{\pgfqpoint{2.637061in}{3.359182in}}%
\pgfpathlineto{\pgfqpoint{2.639066in}{3.773727in}}%
\pgfpathlineto{\pgfqpoint{2.640068in}{3.719182in}}%
\pgfpathlineto{\pgfqpoint{2.642072in}{3.056455in}}%
\pgfpathlineto{\pgfqpoint{2.643075in}{3.190091in}}%
\pgfpathlineto{\pgfqpoint{2.644077in}{3.513273in}}%
\pgfpathlineto{\pgfqpoint{2.646081in}{3.036000in}}%
\pgfpathlineto{\pgfqpoint{2.648086in}{2.955545in}}%
\pgfpathlineto{\pgfqpoint{2.649088in}{3.033273in}}%
\pgfpathlineto{\pgfqpoint{2.651093in}{2.865545in}}%
\pgfpathlineto{\pgfqpoint{2.652095in}{2.892818in}}%
\pgfpathlineto{\pgfqpoint{2.654099in}{2.789182in}}%
\pgfpathlineto{\pgfqpoint{2.655102in}{2.768727in}}%
\pgfpathlineto{\pgfqpoint{2.656104in}{2.779636in}}%
\pgfpathlineto{\pgfqpoint{2.657106in}{2.871000in}}%
\pgfpathlineto{\pgfqpoint{2.659111in}{2.673273in}}%
\pgfpathlineto{\pgfqpoint{2.660113in}{2.703273in}}%
\pgfpathlineto{\pgfqpoint{2.662117in}{2.873727in}}%
\pgfpathlineto{\pgfqpoint{2.664122in}{2.701909in}}%
\pgfpathlineto{\pgfqpoint{2.665124in}{2.757818in}}%
\pgfpathlineto{\pgfqpoint{2.667128in}{2.950091in}}%
\pgfpathlineto{\pgfqpoint{2.668131in}{2.924182in}}%
\pgfpathlineto{\pgfqpoint{2.670135in}{2.781000in}}%
\pgfpathlineto{\pgfqpoint{2.673142in}{3.142364in}}%
\pgfpathlineto{\pgfqpoint{2.674144in}{3.723273in}}%
\pgfpathlineto{\pgfqpoint{2.675146in}{3.558273in}}%
\pgfpathlineto{\pgfqpoint{2.676149in}{3.059182in}}%
\pgfpathlineto{\pgfqpoint{2.677151in}{3.067364in}}%
\pgfpathlineto{\pgfqpoint{2.678153in}{3.003273in}}%
\pgfpathlineto{\pgfqpoint{2.679155in}{3.690545in}}%
\pgfpathlineto{\pgfqpoint{2.680158in}{3.681000in}}%
\pgfpathlineto{\pgfqpoint{2.682162in}{3.142364in}}%
\pgfpathlineto{\pgfqpoint{2.683164in}{3.132818in}}%
\pgfpathlineto{\pgfqpoint{2.684167in}{3.541909in}}%
\pgfpathlineto{\pgfqpoint{2.685169in}{3.516000in}}%
\pgfpathlineto{\pgfqpoint{2.687173in}{2.961000in}}%
\pgfpathlineto{\pgfqpoint{2.688176in}{3.116455in}}%
\pgfpathlineto{\pgfqpoint{2.689178in}{3.756000in}}%
\pgfpathlineto{\pgfqpoint{2.690180in}{3.719182in}}%
\pgfpathlineto{\pgfqpoint{2.692185in}{2.997818in}}%
\pgfpathlineto{\pgfqpoint{2.693187in}{2.999182in}}%
\pgfpathlineto{\pgfqpoint{2.694189in}{3.004636in}}%
\pgfpathlineto{\pgfqpoint{2.695191in}{3.000545in}}%
\pgfpathlineto{\pgfqpoint{2.697196in}{2.956909in}}%
\pgfpathlineto{\pgfqpoint{2.699200in}{3.036000in}}%
\pgfpathlineto{\pgfqpoint{2.700202in}{2.812364in}}%
\pgfpathlineto{\pgfqpoint{2.701205in}{2.873727in}}%
\pgfpathlineto{\pgfqpoint{2.702207in}{2.864182in}}%
\pgfpathlineto{\pgfqpoint{2.703209in}{2.886000in}}%
\pgfpathlineto{\pgfqpoint{2.704211in}{2.967818in}}%
\pgfpathlineto{\pgfqpoint{2.705214in}{2.858727in}}%
\pgfpathlineto{\pgfqpoint{2.706216in}{2.877818in}}%
\pgfpathlineto{\pgfqpoint{2.707218in}{2.944636in}}%
\pgfpathlineto{\pgfqpoint{2.708220in}{2.871000in}}%
\pgfpathlineto{\pgfqpoint{2.709223in}{2.708727in}}%
\pgfpathlineto{\pgfqpoint{2.710225in}{2.740091in}}%
\pgfpathlineto{\pgfqpoint{2.711227in}{2.913273in}}%
\pgfpathlineto{\pgfqpoint{2.712229in}{2.881909in}}%
\pgfpathlineto{\pgfqpoint{2.714234in}{3.067364in}}%
\pgfpathlineto{\pgfqpoint{2.715236in}{2.839636in}}%
\pgfpathlineto{\pgfqpoint{2.718243in}{2.940545in}}%
\pgfpathlineto{\pgfqpoint{2.719245in}{3.141000in}}%
\pgfpathlineto{\pgfqpoint{2.720247in}{3.580091in}}%
\pgfpathlineto{\pgfqpoint{2.722252in}{3.029182in}}%
\pgfpathlineto{\pgfqpoint{2.723254in}{3.016909in}}%
\pgfpathlineto{\pgfqpoint{2.725259in}{3.217364in}}%
\pgfpathlineto{\pgfqpoint{2.726261in}{3.371455in}}%
\pgfpathlineto{\pgfqpoint{2.727263in}{3.157364in}}%
\pgfpathlineto{\pgfqpoint{2.729268in}{3.730091in}}%
\pgfpathlineto{\pgfqpoint{2.730270in}{3.660545in}}%
\pgfpathlineto{\pgfqpoint{2.732274in}{3.307364in}}%
\pgfpathlineto{\pgfqpoint{2.734279in}{3.779182in}}%
\pgfpathlineto{\pgfqpoint{2.735281in}{3.732818in}}%
\pgfpathlineto{\pgfqpoint{2.737285in}{3.443727in}}%
\pgfpathlineto{\pgfqpoint{2.739290in}{3.726000in}}%
\pgfpathlineto{\pgfqpoint{2.741294in}{2.984182in}}%
\pgfpathlineto{\pgfqpoint{2.742297in}{2.984182in}}%
\pgfpathlineto{\pgfqpoint{2.744301in}{3.640091in}}%
\pgfpathlineto{\pgfqpoint{2.747308in}{2.873727in}}%
\pgfpathlineto{\pgfqpoint{2.750315in}{2.969182in}}%
\pgfpathlineto{\pgfqpoint{2.751317in}{2.958273in}}%
\pgfpathlineto{\pgfqpoint{2.754324in}{2.625545in}}%
\pgfpathlineto{\pgfqpoint{2.756328in}{2.823273in}}%
\pgfpathlineto{\pgfqpoint{2.757330in}{2.806909in}}%
\pgfpathlineto{\pgfqpoint{2.759335in}{2.682818in}}%
\pgfpathlineto{\pgfqpoint{2.760337in}{2.693727in}}%
\pgfpathlineto{\pgfqpoint{2.762342in}{2.817818in}}%
\pgfpathlineto{\pgfqpoint{2.763344in}{2.741455in}}%
\pgfpathlineto{\pgfqpoint{2.764346in}{2.742818in}}%
\pgfpathlineto{\pgfqpoint{2.765348in}{2.789182in}}%
\pgfpathlineto{\pgfqpoint{2.767353in}{2.966455in}}%
\pgfpathlineto{\pgfqpoint{2.769357in}{2.826000in}}%
\pgfpathlineto{\pgfqpoint{2.772364in}{3.022364in}}%
\pgfpathlineto{\pgfqpoint{2.773366in}{3.608727in}}%
\pgfpathlineto{\pgfqpoint{2.775371in}{3.042818in}}%
\pgfpathlineto{\pgfqpoint{2.776373in}{3.056455in}}%
\pgfpathlineto{\pgfqpoint{2.777375in}{3.027818in}}%
\pgfpathlineto{\pgfqpoint{2.779380in}{3.735545in}}%
\pgfpathlineto{\pgfqpoint{2.780382in}{3.772364in}}%
\pgfpathlineto{\pgfqpoint{2.781384in}{3.393273in}}%
\pgfpathlineto{\pgfqpoint{2.785393in}{3.831000in}}%
\pgfpathlineto{\pgfqpoint{2.787398in}{3.619636in}}%
\pgfpathlineto{\pgfqpoint{2.789402in}{3.810545in}}%
\pgfpathlineto{\pgfqpoint{2.790404in}{3.689182in}}%
\pgfpathlineto{\pgfqpoint{2.791407in}{3.121909in}}%
\pgfpathlineto{\pgfqpoint{2.792409in}{3.138273in}}%
\pgfpathlineto{\pgfqpoint{2.794413in}{3.584182in}}%
\pgfpathlineto{\pgfqpoint{2.795416in}{3.413727in}}%
\pgfpathlineto{\pgfqpoint{2.797420in}{2.977364in}}%
\pgfpathlineto{\pgfqpoint{2.798422in}{2.836909in}}%
\pgfpathlineto{\pgfqpoint{2.799425in}{2.881909in}}%
\pgfpathlineto{\pgfqpoint{2.800427in}{2.868273in}}%
\pgfpathlineto{\pgfqpoint{2.801429in}{2.835545in}}%
\pgfpathlineto{\pgfqpoint{2.802431in}{2.845091in}}%
\pgfpathlineto{\pgfqpoint{2.804436in}{2.663727in}}%
\pgfpathlineto{\pgfqpoint{2.807442in}{2.770091in}}%
\pgfpathlineto{\pgfqpoint{2.809447in}{2.614636in}}%
\pgfpathlineto{\pgfqpoint{2.812454in}{2.876455in}}%
\pgfpathlineto{\pgfqpoint{2.814458in}{2.673273in}}%
\pgfpathlineto{\pgfqpoint{2.817465in}{3.003273in}}%
\pgfpathlineto{\pgfqpoint{2.818467in}{2.939182in}}%
\pgfpathlineto{\pgfqpoint{2.820472in}{2.976000in}}%
\pgfpathlineto{\pgfqpoint{2.821474in}{2.992364in}}%
\pgfpathlineto{\pgfqpoint{2.822476in}{3.034636in}}%
\pgfpathlineto{\pgfqpoint{2.823478in}{3.016909in}}%
\pgfpathlineto{\pgfqpoint{2.824481in}{3.657818in}}%
\pgfpathlineto{\pgfqpoint{2.825483in}{3.640091in}}%
\pgfpathlineto{\pgfqpoint{2.826485in}{3.280091in}}%
\pgfpathlineto{\pgfqpoint{2.827487in}{3.349636in}}%
\pgfpathlineto{\pgfqpoint{2.830494in}{3.799636in}}%
\pgfpathlineto{\pgfqpoint{2.831496in}{3.528273in}}%
\pgfpathlineto{\pgfqpoint{2.832499in}{3.533727in}}%
\pgfpathlineto{\pgfqpoint{2.834503in}{3.839182in}}%
\pgfpathlineto{\pgfqpoint{2.835505in}{3.790091in}}%
\pgfpathlineto{\pgfqpoint{2.836508in}{3.619636in}}%
\pgfpathlineto{\pgfqpoint{2.837510in}{3.646909in}}%
\pgfpathlineto{\pgfqpoint{2.839514in}{3.907364in}}%
\pgfpathlineto{\pgfqpoint{2.840516in}{3.705545in}}%
\pgfpathlineto{\pgfqpoint{2.842521in}{3.014182in}}%
\pgfpathlineto{\pgfqpoint{2.844525in}{3.736909in}}%
\pgfpathlineto{\pgfqpoint{2.846530in}{2.991000in}}%
\pgfpathlineto{\pgfqpoint{2.848534in}{2.798727in}}%
\pgfpathlineto{\pgfqpoint{2.850539in}{2.932364in}}%
\pgfpathlineto{\pgfqpoint{2.851541in}{2.969182in}}%
\pgfpathlineto{\pgfqpoint{2.852543in}{2.881909in}}%
\pgfpathlineto{\pgfqpoint{2.854548in}{2.670545in}}%
\pgfpathlineto{\pgfqpoint{2.856552in}{2.869636in}}%
\pgfpathlineto{\pgfqpoint{2.857555in}{2.858727in}}%
\pgfpathlineto{\pgfqpoint{2.859559in}{2.655545in}}%
\pgfpathlineto{\pgfqpoint{2.860561in}{2.697818in}}%
\pgfpathlineto{\pgfqpoint{2.862566in}{2.809636in}}%
\pgfpathlineto{\pgfqpoint{2.864570in}{2.812364in}}%
\pgfpathlineto{\pgfqpoint{2.866575in}{2.916000in}}%
\pgfpathlineto{\pgfqpoint{2.868579in}{2.849182in}}%
\pgfpathlineto{\pgfqpoint{2.870584in}{2.954182in}}%
\pgfpathlineto{\pgfqpoint{2.871586in}{3.113727in}}%
\pgfpathlineto{\pgfqpoint{2.872588in}{3.089182in}}%
\pgfpathlineto{\pgfqpoint{2.873590in}{3.008727in}}%
\pgfpathlineto{\pgfqpoint{2.875595in}{3.278727in}}%
\pgfpathlineto{\pgfqpoint{2.876597in}{3.449182in}}%
\pgfpathlineto{\pgfqpoint{2.877599in}{3.325091in}}%
\pgfpathlineto{\pgfqpoint{2.879604in}{3.708273in}}%
\pgfpathlineto{\pgfqpoint{2.880606in}{3.732818in}}%
\pgfpathlineto{\pgfqpoint{2.882611in}{3.631909in}}%
\pgfpathlineto{\pgfqpoint{2.883613in}{3.691909in}}%
\pgfpathlineto{\pgfqpoint{2.884615in}{3.863727in}}%
\pgfpathlineto{\pgfqpoint{2.885617in}{3.719182in}}%
\pgfpathlineto{\pgfqpoint{2.887622in}{3.239182in}}%
\pgfpathlineto{\pgfqpoint{2.889626in}{3.821455in}}%
\pgfpathlineto{\pgfqpoint{2.891631in}{3.663273in}}%
\pgfpathlineto{\pgfqpoint{2.892633in}{3.082364in}}%
\pgfpathlineto{\pgfqpoint{2.894638in}{3.396000in}}%
\pgfpathlineto{\pgfqpoint{2.898647in}{2.854636in}}%
\pgfpathlineto{\pgfqpoint{2.900651in}{2.961000in}}%
\pgfpathlineto{\pgfqpoint{2.901653in}{2.955545in}}%
\pgfpathlineto{\pgfqpoint{2.903658in}{2.699182in}}%
\pgfpathlineto{\pgfqpoint{2.904660in}{2.669182in}}%
\pgfpathlineto{\pgfqpoint{2.906665in}{2.876455in}}%
\pgfpathlineto{\pgfqpoint{2.907667in}{2.809636in}}%
\pgfpathlineto{\pgfqpoint{2.908669in}{2.666455in}}%
\pgfpathlineto{\pgfqpoint{2.909671in}{2.686909in}}%
\pgfpathlineto{\pgfqpoint{2.910673in}{2.751000in}}%
\pgfpathlineto{\pgfqpoint{2.911676in}{2.913273in}}%
\pgfpathlineto{\pgfqpoint{2.913680in}{2.674636in}}%
\pgfpathlineto{\pgfqpoint{2.914682in}{2.708727in}}%
\pgfpathlineto{\pgfqpoint{2.916687in}{2.997818in}}%
\pgfpathlineto{\pgfqpoint{2.918691in}{2.853273in}}%
\pgfpathlineto{\pgfqpoint{2.919694in}{2.845091in}}%
\pgfpathlineto{\pgfqpoint{2.923703in}{3.126000in}}%
\pgfpathlineto{\pgfqpoint{2.925707in}{3.191455in}}%
\pgfpathlineto{\pgfqpoint{2.926709in}{3.288273in}}%
\pgfpathlineto{\pgfqpoint{2.928714in}{3.775091in}}%
\pgfpathlineto{\pgfqpoint{2.930718in}{3.640091in}}%
\pgfpathlineto{\pgfqpoint{2.931721in}{3.649636in}}%
\pgfpathlineto{\pgfqpoint{2.932723in}{3.711000in}}%
\pgfpathlineto{\pgfqpoint{2.933725in}{3.888273in}}%
\pgfpathlineto{\pgfqpoint{2.934727in}{3.870545in}}%
\pgfpathlineto{\pgfqpoint{2.936732in}{3.618273in}}%
\pgfpathlineto{\pgfqpoint{2.937734in}{3.588273in}}%
\pgfpathlineto{\pgfqpoint{2.939739in}{3.773727in}}%
\pgfpathlineto{\pgfqpoint{2.940741in}{3.716455in}}%
\pgfpathlineto{\pgfqpoint{2.942745in}{3.423273in}}%
\pgfpathlineto{\pgfqpoint{2.943747in}{3.475091in}}%
\pgfpathlineto{\pgfqpoint{2.945752in}{3.003273in}}%
\pgfpathlineto{\pgfqpoint{2.946754in}{2.907818in}}%
\pgfpathlineto{\pgfqpoint{2.948759in}{3.010091in}}%
\pgfpathlineto{\pgfqpoint{2.950763in}{2.763273in}}%
\pgfpathlineto{\pgfqpoint{2.951765in}{2.824636in}}%
\pgfpathlineto{\pgfqpoint{2.954772in}{2.658273in}}%
\pgfpathlineto{\pgfqpoint{2.956777in}{2.787818in}}%
\pgfpathlineto{\pgfqpoint{2.959783in}{2.636455in}}%
\pgfpathlineto{\pgfqpoint{2.962790in}{2.869636in}}%
\pgfpathlineto{\pgfqpoint{2.963792in}{2.775545in}}%
\pgfpathlineto{\pgfqpoint{2.965797in}{2.849182in}}%
\pgfpathlineto{\pgfqpoint{2.967801in}{2.943273in}}%
\pgfpathlineto{\pgfqpoint{2.968804in}{2.928273in}}%
\pgfpathlineto{\pgfqpoint{2.969806in}{2.872364in}}%
\pgfpathlineto{\pgfqpoint{2.971810in}{2.965091in}}%
\pgfpathlineto{\pgfqpoint{2.972813in}{3.149182in}}%
\pgfpathlineto{\pgfqpoint{2.973815in}{3.629182in}}%
\pgfpathlineto{\pgfqpoint{2.975819in}{3.321000in}}%
\pgfpathlineto{\pgfqpoint{2.976822in}{3.220091in}}%
\pgfpathlineto{\pgfqpoint{2.978826in}{3.833727in}}%
\pgfpathlineto{\pgfqpoint{2.981833in}{3.548727in}}%
\pgfpathlineto{\pgfqpoint{2.983837in}{3.953727in}}%
\pgfpathlineto{\pgfqpoint{2.984839in}{3.937364in}}%
\pgfpathlineto{\pgfqpoint{2.986844in}{3.623727in}}%
\pgfpathlineto{\pgfqpoint{2.989851in}{3.851455in}}%
\pgfpathlineto{\pgfqpoint{2.990853in}{3.573273in}}%
\pgfpathlineto{\pgfqpoint{2.991855in}{2.947364in}}%
\pgfpathlineto{\pgfqpoint{2.993860in}{3.277364in}}%
\pgfpathlineto{\pgfqpoint{2.996866in}{2.871000in}}%
\pgfpathlineto{\pgfqpoint{2.997869in}{2.880545in}}%
\pgfpathlineto{\pgfqpoint{2.999873in}{2.836909in}}%
\pgfpathlineto{\pgfqpoint{3.000875in}{2.841000in}}%
\pgfpathlineto{\pgfqpoint{3.004884in}{2.646000in}}%
\pgfpathlineto{\pgfqpoint{3.005887in}{2.854636in}}%
\pgfpathlineto{\pgfqpoint{3.006889in}{2.835545in}}%
\pgfpathlineto{\pgfqpoint{3.009896in}{2.606455in}}%
\pgfpathlineto{\pgfqpoint{3.011900in}{2.865545in}}%
\pgfpathlineto{\pgfqpoint{3.014907in}{2.722364in}}%
\pgfpathlineto{\pgfqpoint{3.017913in}{3.010091in}}%
\pgfpathlineto{\pgfqpoint{3.019918in}{2.876455in}}%
\pgfpathlineto{\pgfqpoint{3.020920in}{2.902364in}}%
\pgfpathlineto{\pgfqpoint{3.021922in}{3.007364in}}%
\pgfpathlineto{\pgfqpoint{3.022925in}{3.469636in}}%
\pgfpathlineto{\pgfqpoint{3.023927in}{3.364636in}}%
\pgfpathlineto{\pgfqpoint{3.024929in}{3.548727in}}%
\pgfpathlineto{\pgfqpoint{3.026934in}{3.060545in}}%
\pgfpathlineto{\pgfqpoint{3.028938in}{3.698727in}}%
\pgfpathlineto{\pgfqpoint{3.029940in}{3.795545in}}%
\pgfpathlineto{\pgfqpoint{3.031945in}{3.543273in}}%
\pgfpathlineto{\pgfqpoint{3.034952in}{4.013727in}}%
\pgfpathlineto{\pgfqpoint{3.036956in}{3.506455in}}%
\pgfpathlineto{\pgfqpoint{3.038961in}{3.869182in}}%
\pgfpathlineto{\pgfqpoint{3.039963in}{3.820091in}}%
\pgfpathlineto{\pgfqpoint{3.041967in}{3.111000in}}%
\pgfpathlineto{\pgfqpoint{3.042970in}{3.548727in}}%
\pgfpathlineto{\pgfqpoint{3.043972in}{3.529636in}}%
\pgfpathlineto{\pgfqpoint{3.044974in}{3.573273in}}%
\pgfpathlineto{\pgfqpoint{3.046979in}{2.901000in}}%
\pgfpathlineto{\pgfqpoint{3.047981in}{2.931000in}}%
\pgfpathlineto{\pgfqpoint{3.048983in}{2.760545in}}%
\pgfpathlineto{\pgfqpoint{3.050987in}{2.847818in}}%
\pgfpathlineto{\pgfqpoint{3.051990in}{2.853273in}}%
\pgfpathlineto{\pgfqpoint{3.053994in}{2.599636in}}%
\pgfpathlineto{\pgfqpoint{3.057001in}{2.781000in}}%
\pgfpathlineto{\pgfqpoint{3.059005in}{2.595545in}}%
\pgfpathlineto{\pgfqpoint{3.061010in}{2.757818in}}%
\pgfpathlineto{\pgfqpoint{3.062012in}{2.892818in}}%
\pgfpathlineto{\pgfqpoint{3.064017in}{2.731909in}}%
\pgfpathlineto{\pgfqpoint{3.065019in}{2.761909in}}%
\pgfpathlineto{\pgfqpoint{3.069028in}{2.939182in}}%
\pgfpathlineto{\pgfqpoint{3.070030in}{2.947364in}}%
\pgfpathlineto{\pgfqpoint{3.071032in}{2.926909in}}%
\pgfpathlineto{\pgfqpoint{3.072035in}{3.072818in}}%
\pgfpathlineto{\pgfqpoint{3.073037in}{3.042818in}}%
\pgfpathlineto{\pgfqpoint{3.074039in}{3.119182in}}%
\pgfpathlineto{\pgfqpoint{3.075041in}{3.622364in}}%
\pgfpathlineto{\pgfqpoint{3.076044in}{3.040091in}}%
\pgfpathlineto{\pgfqpoint{3.078048in}{3.657818in}}%
\pgfpathlineto{\pgfqpoint{3.079050in}{3.686455in}}%
\pgfpathlineto{\pgfqpoint{3.080053in}{3.822818in}}%
\pgfpathlineto{\pgfqpoint{3.082057in}{3.687818in}}%
\pgfpathlineto{\pgfqpoint{3.085064in}{4.056000in}}%
\pgfpathlineto{\pgfqpoint{3.087068in}{3.713727in}}%
\pgfpathlineto{\pgfqpoint{3.088070in}{3.779182in}}%
\pgfpathlineto{\pgfqpoint{3.089073in}{3.942818in}}%
\pgfpathlineto{\pgfqpoint{3.090075in}{3.931909in}}%
\pgfpathlineto{\pgfqpoint{3.093082in}{3.007364in}}%
\pgfpathlineto{\pgfqpoint{3.094084in}{3.551455in}}%
\pgfpathlineto{\pgfqpoint{3.095086in}{3.398727in}}%
\pgfpathlineto{\pgfqpoint{3.096088in}{2.969182in}}%
\pgfpathlineto{\pgfqpoint{3.097091in}{3.018273in}}%
\pgfpathlineto{\pgfqpoint{3.099095in}{2.819182in}}%
\pgfpathlineto{\pgfqpoint{3.101100in}{2.851909in}}%
\pgfpathlineto{\pgfqpoint{3.102102in}{2.980091in}}%
\pgfpathlineto{\pgfqpoint{3.104106in}{2.654182in}}%
\pgfpathlineto{\pgfqpoint{3.105109in}{2.678727in}}%
\pgfpathlineto{\pgfqpoint{3.107113in}{2.872364in}}%
\pgfpathlineto{\pgfqpoint{3.109118in}{2.659636in}}%
\pgfpathlineto{\pgfqpoint{3.110120in}{2.701909in}}%
\pgfpathlineto{\pgfqpoint{3.112124in}{2.890091in}}%
\pgfpathlineto{\pgfqpoint{3.114129in}{2.781000in}}%
\pgfpathlineto{\pgfqpoint{3.115131in}{2.826000in}}%
\pgfpathlineto{\pgfqpoint{3.117136in}{2.993727in}}%
\pgfpathlineto{\pgfqpoint{3.118138in}{2.830091in}}%
\pgfpathlineto{\pgfqpoint{3.119140in}{2.842364in}}%
\pgfpathlineto{\pgfqpoint{3.121144in}{2.969182in}}%
\pgfpathlineto{\pgfqpoint{3.122147in}{3.194182in}}%
\pgfpathlineto{\pgfqpoint{3.123149in}{3.011455in}}%
\pgfpathlineto{\pgfqpoint{3.124151in}{3.149182in}}%
\pgfpathlineto{\pgfqpoint{3.125153in}{3.045545in}}%
\pgfpathlineto{\pgfqpoint{3.126156in}{3.059182in}}%
\pgfpathlineto{\pgfqpoint{3.129162in}{3.713727in}}%
\pgfpathlineto{\pgfqpoint{3.131167in}{3.660545in}}%
\pgfpathlineto{\pgfqpoint{3.132169in}{3.476455in}}%
\pgfpathlineto{\pgfqpoint{3.135176in}{3.976909in}}%
\pgfpathlineto{\pgfqpoint{3.137180in}{3.691909in}}%
\pgfpathlineto{\pgfqpoint{3.139185in}{3.929182in}}%
\pgfpathlineto{\pgfqpoint{3.141189in}{3.784636in}}%
\pgfpathlineto{\pgfqpoint{3.142192in}{3.603273in}}%
\pgfpathlineto{\pgfqpoint{3.143194in}{3.083727in}}%
\pgfpathlineto{\pgfqpoint{3.144196in}{3.705545in}}%
\pgfpathlineto{\pgfqpoint{3.145198in}{3.168273in}}%
\pgfpathlineto{\pgfqpoint{3.146201in}{3.233727in}}%
\pgfpathlineto{\pgfqpoint{3.148205in}{2.830091in}}%
\pgfpathlineto{\pgfqpoint{3.149207in}{2.835545in}}%
\pgfpathlineto{\pgfqpoint{3.150210in}{2.800091in}}%
\pgfpathlineto{\pgfqpoint{3.151212in}{2.911909in}}%
\pgfpathlineto{\pgfqpoint{3.152214in}{2.910545in}}%
\pgfpathlineto{\pgfqpoint{3.154219in}{2.682818in}}%
\pgfpathlineto{\pgfqpoint{3.155221in}{2.663727in}}%
\pgfpathlineto{\pgfqpoint{3.157225in}{2.831455in}}%
\pgfpathlineto{\pgfqpoint{3.159230in}{2.663727in}}%
\pgfpathlineto{\pgfqpoint{3.162236in}{2.937818in}}%
\pgfpathlineto{\pgfqpoint{3.164241in}{2.723727in}}%
\pgfpathlineto{\pgfqpoint{3.165243in}{2.767364in}}%
\pgfpathlineto{\pgfqpoint{3.166245in}{2.895545in}}%
\pgfpathlineto{\pgfqpoint{3.167248in}{2.886000in}}%
\pgfpathlineto{\pgfqpoint{3.168250in}{2.931000in}}%
\pgfpathlineto{\pgfqpoint{3.170254in}{2.881909in}}%
\pgfpathlineto{\pgfqpoint{3.171257in}{3.014182in}}%
\pgfpathlineto{\pgfqpoint{3.172259in}{3.006000in}}%
\pgfpathlineto{\pgfqpoint{3.173261in}{3.076909in}}%
\pgfpathlineto{\pgfqpoint{3.174263in}{3.241909in}}%
\pgfpathlineto{\pgfqpoint{3.175266in}{3.097364in}}%
\pgfpathlineto{\pgfqpoint{3.177270in}{3.202364in}}%
\pgfpathlineto{\pgfqpoint{3.179275in}{3.749182in}}%
\pgfpathlineto{\pgfqpoint{3.180277in}{3.701455in}}%
\pgfpathlineto{\pgfqpoint{3.181279in}{3.539182in}}%
\pgfpathlineto{\pgfqpoint{3.182281in}{3.180545in}}%
\pgfpathlineto{\pgfqpoint{3.184286in}{3.841909in}}%
\pgfpathlineto{\pgfqpoint{3.186290in}{3.743727in}}%
\pgfpathlineto{\pgfqpoint{3.187293in}{3.194182in}}%
\pgfpathlineto{\pgfqpoint{3.189297in}{3.944182in}}%
\pgfpathlineto{\pgfqpoint{3.191302in}{3.843273in}}%
\pgfpathlineto{\pgfqpoint{3.192304in}{3.228273in}}%
\pgfpathlineto{\pgfqpoint{3.193306in}{3.569182in}}%
\pgfpathlineto{\pgfqpoint{3.195310in}{3.029182in}}%
\pgfpathlineto{\pgfqpoint{3.196313in}{3.104182in}}%
\pgfpathlineto{\pgfqpoint{3.197315in}{2.907818in}}%
\pgfpathlineto{\pgfqpoint{3.198317in}{2.961000in}}%
\pgfpathlineto{\pgfqpoint{3.200322in}{2.816455in}}%
\pgfpathlineto{\pgfqpoint{3.201324in}{2.898273in}}%
\pgfpathlineto{\pgfqpoint{3.204331in}{2.742818in}}%
\pgfpathlineto{\pgfqpoint{3.205333in}{2.729182in}}%
\pgfpathlineto{\pgfqpoint{3.206335in}{2.898273in}}%
\pgfpathlineto{\pgfqpoint{3.207337in}{2.860091in}}%
\pgfpathlineto{\pgfqpoint{3.209342in}{2.677364in}}%
\pgfpathlineto{\pgfqpoint{3.211346in}{2.888727in}}%
\pgfpathlineto{\pgfqpoint{3.212349in}{2.835545in}}%
\pgfpathlineto{\pgfqpoint{3.213351in}{2.831455in}}%
\pgfpathlineto{\pgfqpoint{3.214353in}{2.786455in}}%
\pgfpathlineto{\pgfqpoint{3.216358in}{3.006000in}}%
\pgfpathlineto{\pgfqpoint{3.218362in}{2.935091in}}%
\pgfpathlineto{\pgfqpoint{3.219364in}{2.865545in}}%
\pgfpathlineto{\pgfqpoint{3.222371in}{3.093273in}}%
\pgfpathlineto{\pgfqpoint{3.223373in}{3.348273in}}%
\pgfpathlineto{\pgfqpoint{3.225378in}{3.059182in}}%
\pgfpathlineto{\pgfqpoint{3.226380in}{3.078273in}}%
\pgfpathlineto{\pgfqpoint{3.227382in}{3.150545in}}%
\pgfpathlineto{\pgfqpoint{3.229387in}{3.867818in}}%
\pgfpathlineto{\pgfqpoint{3.232393in}{3.623727in}}%
\pgfpathlineto{\pgfqpoint{3.234398in}{3.933273in}}%
\pgfpathlineto{\pgfqpoint{3.235400in}{3.956455in}}%
\pgfpathlineto{\pgfqpoint{3.236402in}{3.670091in}}%
\pgfpathlineto{\pgfqpoint{3.237405in}{3.672818in}}%
\pgfpathlineto{\pgfqpoint{3.239409in}{3.916909in}}%
\pgfpathlineto{\pgfqpoint{3.240411in}{3.955091in}}%
\pgfpathlineto{\pgfqpoint{3.242416in}{3.131455in}}%
\pgfpathlineto{\pgfqpoint{3.243418in}{3.322364in}}%
\pgfpathlineto{\pgfqpoint{3.244420in}{3.102818in}}%
\pgfpathlineto{\pgfqpoint{3.245423in}{3.258273in}}%
\pgfpathlineto{\pgfqpoint{3.248429in}{2.898273in}}%
\pgfpathlineto{\pgfqpoint{3.249432in}{2.808273in}}%
\pgfpathlineto{\pgfqpoint{3.250434in}{2.943273in}}%
\pgfpathlineto{\pgfqpoint{3.253441in}{2.772818in}}%
\pgfpathlineto{\pgfqpoint{3.254443in}{2.650091in}}%
\pgfpathlineto{\pgfqpoint{3.257450in}{2.894182in}}%
\pgfpathlineto{\pgfqpoint{3.259454in}{2.714182in}}%
\pgfpathlineto{\pgfqpoint{3.262461in}{2.941909in}}%
\pgfpathlineto{\pgfqpoint{3.264465in}{2.800091in}}%
\pgfpathlineto{\pgfqpoint{3.267472in}{2.971909in}}%
\pgfpathlineto{\pgfqpoint{3.268474in}{2.931000in}}%
\pgfpathlineto{\pgfqpoint{3.269476in}{2.974636in}}%
\pgfpathlineto{\pgfqpoint{3.271481in}{2.937818in}}%
\pgfpathlineto{\pgfqpoint{3.272483in}{3.051000in}}%
\pgfpathlineto{\pgfqpoint{3.273485in}{3.329182in}}%
\pgfpathlineto{\pgfqpoint{3.274488in}{3.146455in}}%
\pgfpathlineto{\pgfqpoint{3.275490in}{3.236455in}}%
\pgfpathlineto{\pgfqpoint{3.276492in}{3.033273in}}%
\pgfpathlineto{\pgfqpoint{3.278497in}{3.780545in}}%
\pgfpathlineto{\pgfqpoint{3.279499in}{3.871909in}}%
\pgfpathlineto{\pgfqpoint{3.280501in}{3.798273in}}%
\pgfpathlineto{\pgfqpoint{3.281503in}{3.390545in}}%
\pgfpathlineto{\pgfqpoint{3.283508in}{3.809182in}}%
\pgfpathlineto{\pgfqpoint{3.284510in}{3.981000in}}%
\pgfpathlineto{\pgfqpoint{3.286515in}{3.438273in}}%
\pgfpathlineto{\pgfqpoint{3.288519in}{3.938727in}}%
\pgfpathlineto{\pgfqpoint{3.289521in}{4.046455in}}%
\pgfpathlineto{\pgfqpoint{3.290524in}{3.851455in}}%
\pgfpathlineto{\pgfqpoint{3.292528in}{3.100091in}}%
\pgfpathlineto{\pgfqpoint{3.293530in}{3.021000in}}%
\pgfpathlineto{\pgfqpoint{3.294533in}{3.631909in}}%
\pgfpathlineto{\pgfqpoint{3.295535in}{3.471000in}}%
\pgfpathlineto{\pgfqpoint{3.296537in}{3.018273in}}%
\pgfpathlineto{\pgfqpoint{3.297539in}{3.066000in}}%
\pgfpathlineto{\pgfqpoint{3.299544in}{2.806909in}}%
\pgfpathlineto{\pgfqpoint{3.300546in}{2.896909in}}%
\pgfpathlineto{\pgfqpoint{3.301548in}{2.868273in}}%
\pgfpathlineto{\pgfqpoint{3.302550in}{2.941909in}}%
\pgfpathlineto{\pgfqpoint{3.304555in}{2.730545in}}%
\pgfpathlineto{\pgfqpoint{3.306559in}{2.895545in}}%
\pgfpathlineto{\pgfqpoint{3.307562in}{2.861455in}}%
\pgfpathlineto{\pgfqpoint{3.309566in}{2.706000in}}%
\pgfpathlineto{\pgfqpoint{3.312573in}{2.902364in}}%
\pgfpathlineto{\pgfqpoint{3.314577in}{2.808273in}}%
\pgfpathlineto{\pgfqpoint{3.316582in}{2.860091in}}%
\pgfpathlineto{\pgfqpoint{3.319589in}{3.056455in}}%
\pgfpathlineto{\pgfqpoint{3.320591in}{2.939182in}}%
\pgfpathlineto{\pgfqpoint{3.321593in}{2.951455in}}%
\pgfpathlineto{\pgfqpoint{3.322595in}{3.034636in}}%
\pgfpathlineto{\pgfqpoint{3.323598in}{2.978727in}}%
\pgfpathlineto{\pgfqpoint{3.324600in}{3.311455in}}%
\pgfpathlineto{\pgfqpoint{3.326604in}{3.066000in}}%
\pgfpathlineto{\pgfqpoint{3.327607in}{3.360545in}}%
\pgfpathlineto{\pgfqpoint{3.328609in}{3.248727in}}%
\pgfpathlineto{\pgfqpoint{3.329611in}{3.736909in}}%
\pgfpathlineto{\pgfqpoint{3.331616in}{3.559636in}}%
\pgfpathlineto{\pgfqpoint{3.332618in}{3.659182in}}%
\pgfpathlineto{\pgfqpoint{3.334622in}{4.016455in}}%
\pgfpathlineto{\pgfqpoint{3.336627in}{3.610091in}}%
\pgfpathlineto{\pgfqpoint{3.337629in}{3.656455in}}%
\pgfpathlineto{\pgfqpoint{3.339633in}{3.892364in}}%
\pgfpathlineto{\pgfqpoint{3.341638in}{3.104182in}}%
\pgfpathlineto{\pgfqpoint{3.343642in}{3.741000in}}%
\pgfpathlineto{\pgfqpoint{3.344645in}{3.798273in}}%
\pgfpathlineto{\pgfqpoint{3.345647in}{3.012818in}}%
\pgfpathlineto{\pgfqpoint{3.346649in}{3.056455in}}%
\pgfpathlineto{\pgfqpoint{3.347651in}{2.903727in}}%
\pgfpathlineto{\pgfqpoint{3.348654in}{2.905091in}}%
\pgfpathlineto{\pgfqpoint{3.349656in}{3.411000in}}%
\pgfpathlineto{\pgfqpoint{3.350658in}{2.963727in}}%
\pgfpathlineto{\pgfqpoint{3.351660in}{2.985545in}}%
\pgfpathlineto{\pgfqpoint{3.353665in}{2.748273in}}%
\pgfpathlineto{\pgfqpoint{3.356672in}{2.922818in}}%
\pgfpathlineto{\pgfqpoint{3.358676in}{2.782364in}}%
\pgfpathlineto{\pgfqpoint{3.359678in}{2.766000in}}%
\pgfpathlineto{\pgfqpoint{3.361683in}{2.944636in}}%
\pgfpathlineto{\pgfqpoint{3.364690in}{2.786455in}}%
\pgfpathlineto{\pgfqpoint{3.366694in}{2.976000in}}%
\pgfpathlineto{\pgfqpoint{3.367696in}{2.898273in}}%
\pgfpathlineto{\pgfqpoint{3.368698in}{2.902364in}}%
\pgfpathlineto{\pgfqpoint{3.369701in}{2.909182in}}%
\pgfpathlineto{\pgfqpoint{3.370703in}{2.843727in}}%
\pgfpathlineto{\pgfqpoint{3.371705in}{3.018273in}}%
\pgfpathlineto{\pgfqpoint{3.372707in}{2.895545in}}%
\pgfpathlineto{\pgfqpoint{3.374712in}{3.211909in}}%
\pgfpathlineto{\pgfqpoint{3.375714in}{3.019636in}}%
\pgfpathlineto{\pgfqpoint{3.376716in}{3.121909in}}%
\pgfpathlineto{\pgfqpoint{3.377719in}{3.037364in}}%
\pgfpathlineto{\pgfqpoint{3.379723in}{3.749182in}}%
\pgfpathlineto{\pgfqpoint{3.381728in}{3.259636in}}%
\pgfpathlineto{\pgfqpoint{3.384734in}{3.982364in}}%
\pgfpathlineto{\pgfqpoint{3.387741in}{3.686455in}}%
\pgfpathlineto{\pgfqpoint{3.388743in}{3.931909in}}%
\pgfpathlineto{\pgfqpoint{3.389746in}{3.895091in}}%
\pgfpathlineto{\pgfqpoint{3.391750in}{3.736909in}}%
\pgfpathlineto{\pgfqpoint{3.392752in}{3.195545in}}%
\pgfpathlineto{\pgfqpoint{3.393755in}{3.747818in}}%
\pgfpathlineto{\pgfqpoint{3.394757in}{3.682364in}}%
\pgfpathlineto{\pgfqpoint{3.395759in}{3.428727in}}%
\pgfpathlineto{\pgfqpoint{3.396761in}{3.457364in}}%
\pgfpathlineto{\pgfqpoint{3.397764in}{2.950091in}}%
\pgfpathlineto{\pgfqpoint{3.398766in}{3.250091in}}%
\pgfpathlineto{\pgfqpoint{3.399768in}{2.991000in}}%
\pgfpathlineto{\pgfqpoint{3.400770in}{3.011455in}}%
\pgfpathlineto{\pgfqpoint{3.401773in}{3.037364in}}%
\pgfpathlineto{\pgfqpoint{3.403777in}{2.813727in}}%
\pgfpathlineto{\pgfqpoint{3.406784in}{2.969182in}}%
\pgfpathlineto{\pgfqpoint{3.408788in}{2.821909in}}%
\pgfpathlineto{\pgfqpoint{3.409790in}{2.808273in}}%
\pgfpathlineto{\pgfqpoint{3.411795in}{2.917364in}}%
\pgfpathlineto{\pgfqpoint{3.412797in}{2.839636in}}%
\pgfpathlineto{\pgfqpoint{3.413799in}{2.849182in}}%
\pgfpathlineto{\pgfqpoint{3.414802in}{2.790545in}}%
\pgfpathlineto{\pgfqpoint{3.416806in}{2.944636in}}%
\pgfpathlineto{\pgfqpoint{3.419813in}{2.793273in}}%
\pgfpathlineto{\pgfqpoint{3.421817in}{2.967818in}}%
\pgfpathlineto{\pgfqpoint{3.422820in}{2.914636in}}%
\pgfpathlineto{\pgfqpoint{3.423822in}{2.999182in}}%
\pgfpathlineto{\pgfqpoint{3.424824in}{2.887364in}}%
\pgfpathlineto{\pgfqpoint{3.427831in}{3.051000in}}%
\pgfpathlineto{\pgfqpoint{3.428833in}{3.559636in}}%
\pgfpathlineto{\pgfqpoint{3.430838in}{3.104182in}}%
\pgfpathlineto{\pgfqpoint{3.431840in}{3.091909in}}%
\pgfpathlineto{\pgfqpoint{3.433844in}{3.897818in}}%
\pgfpathlineto{\pgfqpoint{3.436851in}{3.701455in}}%
\pgfpathlineto{\pgfqpoint{3.438856in}{4.012364in}}%
\pgfpathlineto{\pgfqpoint{3.441862in}{3.537818in}}%
\pgfpathlineto{\pgfqpoint{3.442864in}{3.816000in}}%
\pgfpathlineto{\pgfqpoint{3.443867in}{3.753273in}}%
\pgfpathlineto{\pgfqpoint{3.444869in}{3.588273in}}%
\pgfpathlineto{\pgfqpoint{3.445871in}{3.611455in}}%
\pgfpathlineto{\pgfqpoint{3.447876in}{2.971909in}}%
\pgfpathlineto{\pgfqpoint{3.448878in}{3.254182in}}%
\pgfpathlineto{\pgfqpoint{3.452887in}{2.888727in}}%
\pgfpathlineto{\pgfqpoint{3.453889in}{2.876455in}}%
\pgfpathlineto{\pgfqpoint{3.454891in}{2.821909in}}%
\pgfpathlineto{\pgfqpoint{3.455894in}{2.914636in}}%
\pgfpathlineto{\pgfqpoint{3.457898in}{2.875091in}}%
\pgfpathlineto{\pgfqpoint{3.459903in}{2.763273in}}%
\pgfpathlineto{\pgfqpoint{3.460905in}{2.909182in}}%
\pgfpathlineto{\pgfqpoint{3.461907in}{2.861455in}}%
\pgfpathlineto{\pgfqpoint{3.462909in}{2.892818in}}%
\pgfpathlineto{\pgfqpoint{3.463912in}{2.805545in}}%
\pgfpathlineto{\pgfqpoint{3.464914in}{2.830091in}}%
\pgfpathlineto{\pgfqpoint{3.465916in}{3.016909in}}%
\pgfpathlineto{\pgfqpoint{3.466918in}{2.905091in}}%
\pgfpathlineto{\pgfqpoint{3.467921in}{2.962364in}}%
\pgfpathlineto{\pgfqpoint{3.469925in}{2.861455in}}%
\pgfpathlineto{\pgfqpoint{3.472932in}{3.027818in}}%
\pgfpathlineto{\pgfqpoint{3.473934in}{3.010091in}}%
\pgfpathlineto{\pgfqpoint{3.474936in}{2.932364in}}%
\pgfpathlineto{\pgfqpoint{3.476941in}{3.011455in}}%
\pgfpathlineto{\pgfqpoint{3.478945in}{3.698727in}}%
\pgfpathlineto{\pgfqpoint{3.481952in}{3.274636in}}%
\pgfpathlineto{\pgfqpoint{3.483956in}{3.833727in}}%
\pgfpathlineto{\pgfqpoint{3.484959in}{3.877364in}}%
\pgfpathlineto{\pgfqpoint{3.486963in}{3.442364in}}%
\pgfpathlineto{\pgfqpoint{3.487965in}{3.847364in}}%
\pgfpathlineto{\pgfqpoint{3.488968in}{3.817364in}}%
\pgfpathlineto{\pgfqpoint{3.489970in}{3.956455in}}%
\pgfpathlineto{\pgfqpoint{3.490972in}{3.719182in}}%
\pgfpathlineto{\pgfqpoint{3.491974in}{3.199636in}}%
\pgfpathlineto{\pgfqpoint{3.492977in}{3.769636in}}%
\pgfpathlineto{\pgfqpoint{3.493979in}{3.510545in}}%
\pgfpathlineto{\pgfqpoint{3.494981in}{3.700091in}}%
\pgfpathlineto{\pgfqpoint{3.495983in}{3.600545in}}%
\pgfpathlineto{\pgfqpoint{3.497988in}{3.045545in}}%
\pgfpathlineto{\pgfqpoint{3.499992in}{2.978727in}}%
\pgfpathlineto{\pgfqpoint{3.500995in}{3.061909in}}%
\pgfpathlineto{\pgfqpoint{3.504001in}{2.830091in}}%
\pgfpathlineto{\pgfqpoint{3.507008in}{2.913273in}}%
\pgfpathlineto{\pgfqpoint{3.509013in}{2.771455in}}%
\pgfpathlineto{\pgfqpoint{3.511017in}{2.876455in}}%
\pgfpathlineto{\pgfqpoint{3.512019in}{2.931000in}}%
\pgfpathlineto{\pgfqpoint{3.514024in}{2.768727in}}%
\pgfpathlineto{\pgfqpoint{3.515026in}{2.902364in}}%
\pgfpathlineto{\pgfqpoint{3.516028in}{2.860091in}}%
\pgfpathlineto{\pgfqpoint{3.518033in}{2.905091in}}%
\pgfpathlineto{\pgfqpoint{3.519035in}{2.790545in}}%
\pgfpathlineto{\pgfqpoint{3.520037in}{2.903727in}}%
\pgfpathlineto{\pgfqpoint{3.521039in}{2.873727in}}%
\pgfpathlineto{\pgfqpoint{3.522042in}{2.898273in}}%
\pgfpathlineto{\pgfqpoint{3.523044in}{3.007364in}}%
\pgfpathlineto{\pgfqpoint{3.524046in}{2.984182in}}%
\pgfpathlineto{\pgfqpoint{3.525048in}{3.010091in}}%
\pgfpathlineto{\pgfqpoint{3.526051in}{2.980091in}}%
\pgfpathlineto{\pgfqpoint{3.527053in}{3.104182in}}%
\pgfpathlineto{\pgfqpoint{3.528055in}{3.524182in}}%
\pgfpathlineto{\pgfqpoint{3.529057in}{3.381000in}}%
\pgfpathlineto{\pgfqpoint{3.530060in}{3.708273in}}%
\pgfpathlineto{\pgfqpoint{3.531062in}{3.156000in}}%
\pgfpathlineto{\pgfqpoint{3.532064in}{3.629182in}}%
\pgfpathlineto{\pgfqpoint{3.533066in}{3.622364in}}%
\pgfpathlineto{\pgfqpoint{3.535071in}{3.948273in}}%
\pgfpathlineto{\pgfqpoint{3.536073in}{3.623727in}}%
\pgfpathlineto{\pgfqpoint{3.540082in}{4.028727in}}%
\pgfpathlineto{\pgfqpoint{3.542087in}{3.522818in}}%
\pgfpathlineto{\pgfqpoint{3.543089in}{3.741000in}}%
\pgfpathlineto{\pgfqpoint{3.544091in}{3.713727in}}%
\pgfpathlineto{\pgfqpoint{3.545093in}{3.859636in}}%
\pgfpathlineto{\pgfqpoint{3.547098in}{3.175091in}}%
\pgfpathlineto{\pgfqpoint{3.549102in}{3.045545in}}%
\pgfpathlineto{\pgfqpoint{3.550104in}{3.255545in}}%
\pgfpathlineto{\pgfqpoint{3.552109in}{3.040091in}}%
\pgfpathlineto{\pgfqpoint{3.554113in}{2.843727in}}%
\pgfpathlineto{\pgfqpoint{3.556118in}{2.970545in}}%
\pgfpathlineto{\pgfqpoint{3.557120in}{2.959636in}}%
\pgfpathlineto{\pgfqpoint{3.558122in}{2.827364in}}%
\pgfpathlineto{\pgfqpoint{3.562131in}{2.903727in}}%
\pgfpathlineto{\pgfqpoint{3.564136in}{2.802818in}}%
\pgfpathlineto{\pgfqpoint{3.565138in}{2.864182in}}%
\pgfpathlineto{\pgfqpoint{3.566140in}{2.839636in}}%
\pgfpathlineto{\pgfqpoint{3.567143in}{2.941909in}}%
\pgfpathlineto{\pgfqpoint{3.569147in}{2.828727in}}%
\pgfpathlineto{\pgfqpoint{3.570149in}{2.910545in}}%
\pgfpathlineto{\pgfqpoint{3.571152in}{2.811000in}}%
\pgfpathlineto{\pgfqpoint{3.572154in}{2.963727in}}%
\pgfpathlineto{\pgfqpoint{3.573156in}{2.876455in}}%
\pgfpathlineto{\pgfqpoint{3.575161in}{2.921455in}}%
\pgfpathlineto{\pgfqpoint{3.576163in}{2.892818in}}%
\pgfpathlineto{\pgfqpoint{3.577165in}{3.034636in}}%
\pgfpathlineto{\pgfqpoint{3.578167in}{2.995091in}}%
\pgfpathlineto{\pgfqpoint{3.579170in}{3.123273in}}%
\pgfpathlineto{\pgfqpoint{3.580172in}{3.119182in}}%
\pgfpathlineto{\pgfqpoint{3.581174in}{3.056455in}}%
\pgfpathlineto{\pgfqpoint{3.583178in}{3.364636in}}%
\pgfpathlineto{\pgfqpoint{3.584181in}{3.777818in}}%
\pgfpathlineto{\pgfqpoint{3.585183in}{3.700091in}}%
\pgfpathlineto{\pgfqpoint{3.586185in}{3.453273in}}%
\pgfpathlineto{\pgfqpoint{3.589192in}{3.931909in}}%
\pgfpathlineto{\pgfqpoint{3.590194in}{3.839182in}}%
\pgfpathlineto{\pgfqpoint{3.591196in}{3.606000in}}%
\pgfpathlineto{\pgfqpoint{3.592199in}{3.859636in}}%
\pgfpathlineto{\pgfqpoint{3.593201in}{3.757364in}}%
\pgfpathlineto{\pgfqpoint{3.594203in}{3.783273in}}%
\pgfpathlineto{\pgfqpoint{3.595205in}{3.732818in}}%
\pgfpathlineto{\pgfqpoint{3.596208in}{3.121909in}}%
\pgfpathlineto{\pgfqpoint{3.597210in}{3.679636in}}%
\pgfpathlineto{\pgfqpoint{3.598212in}{3.217364in}}%
\pgfpathlineto{\pgfqpoint{3.599214in}{3.736909in}}%
\pgfpathlineto{\pgfqpoint{3.601219in}{3.064636in}}%
\pgfpathlineto{\pgfqpoint{3.603223in}{2.933727in}}%
\pgfpathlineto{\pgfqpoint{3.604226in}{3.014182in}}%
\pgfpathlineto{\pgfqpoint{3.606230in}{2.995091in}}%
\pgfpathlineto{\pgfqpoint{3.608235in}{2.832818in}}%
\pgfpathlineto{\pgfqpoint{3.611241in}{2.951455in}}%
\pgfpathlineto{\pgfqpoint{3.612244in}{2.894182in}}%
\pgfpathlineto{\pgfqpoint{3.613246in}{2.771455in}}%
\pgfpathlineto{\pgfqpoint{3.615250in}{2.832818in}}%
\pgfpathlineto{\pgfqpoint{3.617255in}{2.984182in}}%
\pgfpathlineto{\pgfqpoint{3.618257in}{2.768727in}}%
\pgfpathlineto{\pgfqpoint{3.619259in}{2.832818in}}%
\pgfpathlineto{\pgfqpoint{3.620261in}{2.783727in}}%
\pgfpathlineto{\pgfqpoint{3.622266in}{2.989636in}}%
\pgfpathlineto{\pgfqpoint{3.623268in}{2.899636in}}%
\pgfpathlineto{\pgfqpoint{3.624270in}{2.952818in}}%
\pgfpathlineto{\pgfqpoint{3.625273in}{2.824636in}}%
\pgfpathlineto{\pgfqpoint{3.627277in}{2.974636in}}%
\pgfpathlineto{\pgfqpoint{3.628279in}{3.310091in}}%
\pgfpathlineto{\pgfqpoint{3.629282in}{3.225545in}}%
\pgfpathlineto{\pgfqpoint{3.630284in}{3.012818in}}%
\pgfpathlineto{\pgfqpoint{3.632288in}{3.130091in}}%
\pgfpathlineto{\pgfqpoint{3.634293in}{3.866455in}}%
\pgfpathlineto{\pgfqpoint{3.635295in}{3.569182in}}%
\pgfpathlineto{\pgfqpoint{3.636297in}{3.614182in}}%
\pgfpathlineto{\pgfqpoint{3.637300in}{3.150545in}}%
\pgfpathlineto{\pgfqpoint{3.639304in}{3.889636in}}%
\pgfpathlineto{\pgfqpoint{3.642311in}{3.213273in}}%
\pgfpathlineto{\pgfqpoint{3.644315in}{3.810545in}}%
\pgfpathlineto{\pgfqpoint{3.645318in}{3.634636in}}%
\pgfpathlineto{\pgfqpoint{3.646320in}{3.696000in}}%
\pgfpathlineto{\pgfqpoint{3.648324in}{3.160091in}}%
\pgfpathlineto{\pgfqpoint{3.649327in}{3.488727in}}%
\pgfpathlineto{\pgfqpoint{3.650329in}{3.045545in}}%
\pgfpathlineto{\pgfqpoint{3.651331in}{3.134182in}}%
\pgfpathlineto{\pgfqpoint{3.653335in}{2.988273in}}%
\pgfpathlineto{\pgfqpoint{3.654338in}{2.937818in}}%
\pgfpathlineto{\pgfqpoint{3.655340in}{2.951455in}}%
\pgfpathlineto{\pgfqpoint{3.656342in}{3.016909in}}%
\pgfpathlineto{\pgfqpoint{3.658347in}{2.892818in}}%
\pgfpathlineto{\pgfqpoint{3.660351in}{2.811000in}}%
\pgfpathlineto{\pgfqpoint{3.661353in}{2.995091in}}%
\pgfpathlineto{\pgfqpoint{3.663358in}{2.817818in}}%
\pgfpathlineto{\pgfqpoint{3.664360in}{2.826000in}}%
\pgfpathlineto{\pgfqpoint{3.665362in}{2.816455in}}%
\pgfpathlineto{\pgfqpoint{3.666365in}{2.958273in}}%
\pgfpathlineto{\pgfqpoint{3.667367in}{2.811000in}}%
\pgfpathlineto{\pgfqpoint{3.669371in}{2.836909in}}%
\pgfpathlineto{\pgfqpoint{3.670374in}{2.834182in}}%
\pgfpathlineto{\pgfqpoint{3.671376in}{2.895545in}}%
\pgfpathlineto{\pgfqpoint{3.672378in}{2.843727in}}%
\pgfpathlineto{\pgfqpoint{3.674383in}{2.902364in}}%
\pgfpathlineto{\pgfqpoint{3.675385in}{2.868273in}}%
\pgfpathlineto{\pgfqpoint{3.676387in}{2.952818in}}%
\pgfpathlineto{\pgfqpoint{3.677389in}{2.916000in}}%
\pgfpathlineto{\pgfqpoint{3.678392in}{3.068727in}}%
\pgfpathlineto{\pgfqpoint{3.679394in}{3.061909in}}%
\pgfpathlineto{\pgfqpoint{3.680396in}{3.019636in}}%
\pgfpathlineto{\pgfqpoint{3.681398in}{3.094636in}}%
\pgfpathlineto{\pgfqpoint{3.682401in}{3.063273in}}%
\pgfpathlineto{\pgfqpoint{3.683403in}{3.706909in}}%
\pgfpathlineto{\pgfqpoint{3.684405in}{3.698727in}}%
\pgfpathlineto{\pgfqpoint{3.685407in}{3.593727in}}%
\pgfpathlineto{\pgfqpoint{3.686410in}{3.682364in}}%
\pgfpathlineto{\pgfqpoint{3.687412in}{3.344182in}}%
\pgfpathlineto{\pgfqpoint{3.688414in}{3.904636in}}%
\pgfpathlineto{\pgfqpoint{3.691421in}{3.630545in}}%
\pgfpathlineto{\pgfqpoint{3.692423in}{3.087818in}}%
\pgfpathlineto{\pgfqpoint{3.694427in}{3.832364in}}%
\pgfpathlineto{\pgfqpoint{3.695430in}{3.798273in}}%
\pgfpathlineto{\pgfqpoint{3.697434in}{3.419182in}}%
\pgfpathlineto{\pgfqpoint{3.699439in}{3.691909in}}%
\pgfpathlineto{\pgfqpoint{3.702445in}{3.030545in}}%
\pgfpathlineto{\pgfqpoint{3.703448in}{3.124636in}}%
\pgfpathlineto{\pgfqpoint{3.704450in}{3.042818in}}%
\pgfpathlineto{\pgfqpoint{3.705452in}{3.055091in}}%
\pgfpathlineto{\pgfqpoint{3.707457in}{2.884636in}}%
\pgfpathlineto{\pgfqpoint{3.708459in}{2.933727in}}%
\pgfpathlineto{\pgfqpoint{3.709461in}{2.875091in}}%
\pgfpathlineto{\pgfqpoint{3.710463in}{2.944636in}}%
\pgfpathlineto{\pgfqpoint{3.711466in}{2.920091in}}%
\pgfpathlineto{\pgfqpoint{3.712468in}{2.808273in}}%
\pgfpathlineto{\pgfqpoint{3.713470in}{2.873727in}}%
\pgfpathlineto{\pgfqpoint{3.714472in}{2.809636in}}%
\pgfpathlineto{\pgfqpoint{3.716477in}{2.922818in}}%
\pgfpathlineto{\pgfqpoint{3.717479in}{2.786455in}}%
\pgfpathlineto{\pgfqpoint{3.718481in}{2.794636in}}%
\pgfpathlineto{\pgfqpoint{3.719484in}{2.806909in}}%
\pgfpathlineto{\pgfqpoint{3.721488in}{2.944636in}}%
\pgfpathlineto{\pgfqpoint{3.723492in}{2.819182in}}%
\pgfpathlineto{\pgfqpoint{3.724495in}{2.797364in}}%
\pgfpathlineto{\pgfqpoint{3.728504in}{2.971909in}}%
\pgfpathlineto{\pgfqpoint{3.729506in}{2.838273in}}%
\pgfpathlineto{\pgfqpoint{3.731510in}{2.978727in}}%
\pgfpathlineto{\pgfqpoint{3.733515in}{3.595091in}}%
\pgfpathlineto{\pgfqpoint{3.734517in}{3.124636in}}%
\pgfpathlineto{\pgfqpoint{3.735519in}{3.229636in}}%
\pgfpathlineto{\pgfqpoint{3.736522in}{3.027818in}}%
\pgfpathlineto{\pgfqpoint{3.738526in}{3.859636in}}%
\pgfpathlineto{\pgfqpoint{3.739528in}{3.727364in}}%
\pgfpathlineto{\pgfqpoint{3.740531in}{3.760091in}}%
\pgfpathlineto{\pgfqpoint{3.741533in}{3.634636in}}%
\pgfpathlineto{\pgfqpoint{3.742535in}{3.713727in}}%
\pgfpathlineto{\pgfqpoint{3.743537in}{3.931909in}}%
\pgfpathlineto{\pgfqpoint{3.744540in}{3.784636in}}%
\pgfpathlineto{\pgfqpoint{3.745542in}{3.814636in}}%
\pgfpathlineto{\pgfqpoint{3.747546in}{3.607364in}}%
\pgfpathlineto{\pgfqpoint{3.748549in}{3.742364in}}%
\pgfpathlineto{\pgfqpoint{3.750553in}{3.619636in}}%
\pgfpathlineto{\pgfqpoint{3.751555in}{3.486000in}}%
\pgfpathlineto{\pgfqpoint{3.752558in}{3.128727in}}%
\pgfpathlineto{\pgfqpoint{3.753560in}{3.130091in}}%
\pgfpathlineto{\pgfqpoint{3.754562in}{3.034636in}}%
\pgfpathlineto{\pgfqpoint{3.755564in}{3.169636in}}%
\pgfpathlineto{\pgfqpoint{3.756567in}{2.955545in}}%
\pgfpathlineto{\pgfqpoint{3.757569in}{3.025091in}}%
\pgfpathlineto{\pgfqpoint{3.759573in}{2.909182in}}%
\pgfpathlineto{\pgfqpoint{3.760575in}{2.981455in}}%
\pgfpathlineto{\pgfqpoint{3.761578in}{2.903727in}}%
\pgfpathlineto{\pgfqpoint{3.762580in}{2.926909in}}%
\pgfpathlineto{\pgfqpoint{3.764584in}{2.816455in}}%
\pgfpathlineto{\pgfqpoint{3.765587in}{2.914636in}}%
\pgfpathlineto{\pgfqpoint{3.767591in}{2.798727in}}%
\pgfpathlineto{\pgfqpoint{3.768593in}{2.800091in}}%
\pgfpathlineto{\pgfqpoint{3.769596in}{2.749636in}}%
\pgfpathlineto{\pgfqpoint{3.770598in}{2.909182in}}%
\pgfpathlineto{\pgfqpoint{3.772602in}{2.819182in}}%
\pgfpathlineto{\pgfqpoint{3.773605in}{2.816455in}}%
\pgfpathlineto{\pgfqpoint{3.774607in}{2.727818in}}%
\pgfpathlineto{\pgfqpoint{3.775609in}{2.910545in}}%
\pgfpathlineto{\pgfqpoint{3.776611in}{2.905091in}}%
\pgfpathlineto{\pgfqpoint{3.777614in}{2.871000in}}%
\pgfpathlineto{\pgfqpoint{3.778616in}{2.881909in}}%
\pgfpathlineto{\pgfqpoint{3.779618in}{2.875091in}}%
\pgfpathlineto{\pgfqpoint{3.782625in}{3.247364in}}%
\pgfpathlineto{\pgfqpoint{3.783627in}{3.091909in}}%
\pgfpathlineto{\pgfqpoint{3.785632in}{3.243273in}}%
\pgfpathlineto{\pgfqpoint{3.786634in}{3.101455in}}%
\pgfpathlineto{\pgfqpoint{3.788638in}{3.904636in}}%
\pgfpathlineto{\pgfqpoint{3.790643in}{3.772364in}}%
\pgfpathlineto{\pgfqpoint{3.791645in}{3.266455in}}%
\pgfpathlineto{\pgfqpoint{3.793649in}{3.922364in}}%
\pgfpathlineto{\pgfqpoint{3.794652in}{3.761455in}}%
\pgfpathlineto{\pgfqpoint{3.795654in}{3.768273in}}%
\pgfpathlineto{\pgfqpoint{3.796656in}{3.175091in}}%
\pgfpathlineto{\pgfqpoint{3.797658in}{3.705545in}}%
\pgfpathlineto{\pgfqpoint{3.798661in}{3.682364in}}%
\pgfpathlineto{\pgfqpoint{3.799663in}{3.685091in}}%
\pgfpathlineto{\pgfqpoint{3.800665in}{3.701455in}}%
\pgfpathlineto{\pgfqpoint{3.801667in}{3.247364in}}%
\pgfpathlineto{\pgfqpoint{3.802670in}{3.289636in}}%
\pgfpathlineto{\pgfqpoint{3.803672in}{3.113727in}}%
\pgfpathlineto{\pgfqpoint{3.805676in}{3.232364in}}%
\pgfpathlineto{\pgfqpoint{3.807681in}{3.027818in}}%
\pgfpathlineto{\pgfqpoint{3.808683in}{2.926909in}}%
\pgfpathlineto{\pgfqpoint{3.810688in}{2.985545in}}%
\pgfpathlineto{\pgfqpoint{3.812692in}{2.948727in}}%
\pgfpathlineto{\pgfqpoint{3.814697in}{2.856000in}}%
\pgfpathlineto{\pgfqpoint{3.815699in}{2.879182in}}%
\pgfpathlineto{\pgfqpoint{3.816701in}{2.801455in}}%
\pgfpathlineto{\pgfqpoint{3.817703in}{2.933727in}}%
\pgfpathlineto{\pgfqpoint{3.818706in}{2.785091in}}%
\pgfpathlineto{\pgfqpoint{3.819708in}{2.806909in}}%
\pgfpathlineto{\pgfqpoint{3.820710in}{2.879182in}}%
\pgfpathlineto{\pgfqpoint{3.821712in}{2.782364in}}%
\pgfpathlineto{\pgfqpoint{3.822715in}{2.862818in}}%
\pgfpathlineto{\pgfqpoint{3.823717in}{2.736000in}}%
\pgfpathlineto{\pgfqpoint{3.825721in}{2.869636in}}%
\pgfpathlineto{\pgfqpoint{3.826724in}{2.856000in}}%
\pgfpathlineto{\pgfqpoint{3.827726in}{2.879182in}}%
\pgfpathlineto{\pgfqpoint{3.828728in}{2.869636in}}%
\pgfpathlineto{\pgfqpoint{3.829730in}{2.925545in}}%
\pgfpathlineto{\pgfqpoint{3.830732in}{2.917364in}}%
\pgfpathlineto{\pgfqpoint{3.831735in}{2.955545in}}%
\pgfpathlineto{\pgfqpoint{3.832737in}{3.083727in}}%
\pgfpathlineto{\pgfqpoint{3.833739in}{3.033273in}}%
\pgfpathlineto{\pgfqpoint{3.834741in}{3.340091in}}%
\pgfpathlineto{\pgfqpoint{3.835744in}{3.150545in}}%
\pgfpathlineto{\pgfqpoint{3.836746in}{3.259636in}}%
\pgfpathlineto{\pgfqpoint{3.837748in}{3.713727in}}%
\pgfpathlineto{\pgfqpoint{3.838750in}{3.636000in}}%
\pgfpathlineto{\pgfqpoint{3.839753in}{3.754636in}}%
\pgfpathlineto{\pgfqpoint{3.840755in}{3.671455in}}%
\pgfpathlineto{\pgfqpoint{3.841757in}{3.409636in}}%
\pgfpathlineto{\pgfqpoint{3.842759in}{3.855545in}}%
\pgfpathlineto{\pgfqpoint{3.843762in}{3.787364in}}%
\pgfpathlineto{\pgfqpoint{3.844764in}{3.880091in}}%
\pgfpathlineto{\pgfqpoint{3.846768in}{3.360545in}}%
\pgfpathlineto{\pgfqpoint{3.847771in}{3.784636in}}%
\pgfpathlineto{\pgfqpoint{3.848773in}{3.728727in}}%
\pgfpathlineto{\pgfqpoint{3.849775in}{3.783273in}}%
\pgfpathlineto{\pgfqpoint{3.851780in}{3.338727in}}%
\pgfpathlineto{\pgfqpoint{3.852782in}{3.540545in}}%
\pgfpathlineto{\pgfqpoint{3.853784in}{3.491455in}}%
\pgfpathlineto{\pgfqpoint{3.858795in}{2.946000in}}%
\pgfpathlineto{\pgfqpoint{3.859798in}{3.036000in}}%
\pgfpathlineto{\pgfqpoint{3.860800in}{2.956909in}}%
\pgfpathlineto{\pgfqpoint{3.861802in}{2.971909in}}%
\pgfpathlineto{\pgfqpoint{3.863807in}{2.843727in}}%
\pgfpathlineto{\pgfqpoint{3.864809in}{2.903727in}}%
\pgfpathlineto{\pgfqpoint{3.866813in}{2.835545in}}%
\pgfpathlineto{\pgfqpoint{3.867815in}{2.812364in}}%
\pgfpathlineto{\pgfqpoint{3.868818in}{2.731909in}}%
\pgfpathlineto{\pgfqpoint{3.869820in}{2.846455in}}%
\pgfpathlineto{\pgfqpoint{3.870822in}{2.806909in}}%
\pgfpathlineto{\pgfqpoint{3.871824in}{2.845091in}}%
\pgfpathlineto{\pgfqpoint{3.872827in}{2.828727in}}%
\pgfpathlineto{\pgfqpoint{3.873829in}{2.703273in}}%
\pgfpathlineto{\pgfqpoint{3.874831in}{2.787818in}}%
\pgfpathlineto{\pgfqpoint{3.875833in}{2.763273in}}%
\pgfpathlineto{\pgfqpoint{3.876836in}{2.868273in}}%
\pgfpathlineto{\pgfqpoint{3.877838in}{2.841000in}}%
\pgfpathlineto{\pgfqpoint{3.878840in}{2.763273in}}%
\pgfpathlineto{\pgfqpoint{3.882849in}{2.965091in}}%
\pgfpathlineto{\pgfqpoint{3.883851in}{2.886000in}}%
\pgfpathlineto{\pgfqpoint{3.884854in}{3.019636in}}%
\pgfpathlineto{\pgfqpoint{3.885856in}{2.922818in}}%
\pgfpathlineto{\pgfqpoint{3.889865in}{3.652364in}}%
\pgfpathlineto{\pgfqpoint{3.890867in}{3.022364in}}%
\pgfpathlineto{\pgfqpoint{3.892872in}{3.754636in}}%
\pgfpathlineto{\pgfqpoint{3.893874in}{3.837818in}}%
\pgfpathlineto{\pgfqpoint{3.894876in}{3.825545in}}%
\pgfpathlineto{\pgfqpoint{3.895878in}{3.244636in}}%
\pgfpathlineto{\pgfqpoint{3.897883in}{3.754636in}}%
\pgfpathlineto{\pgfqpoint{3.898885in}{3.771000in}}%
\pgfpathlineto{\pgfqpoint{3.899887in}{3.856909in}}%
\pgfpathlineto{\pgfqpoint{3.901892in}{3.416455in}}%
\pgfpathlineto{\pgfqpoint{3.903896in}{3.694636in}}%
\pgfpathlineto{\pgfqpoint{3.904898in}{3.720545in}}%
\pgfpathlineto{\pgfqpoint{3.906903in}{3.117818in}}%
\pgfpathlineto{\pgfqpoint{3.907905in}{3.126000in}}%
\pgfpathlineto{\pgfqpoint{3.908907in}{3.111000in}}%
\pgfpathlineto{\pgfqpoint{3.909910in}{3.307364in}}%
\pgfpathlineto{\pgfqpoint{3.910912in}{3.003273in}}%
\pgfpathlineto{\pgfqpoint{3.911914in}{3.057818in}}%
\pgfpathlineto{\pgfqpoint{3.912916in}{2.914636in}}%
\pgfpathlineto{\pgfqpoint{3.914921in}{2.948727in}}%
\pgfpathlineto{\pgfqpoint{3.915923in}{2.888727in}}%
\pgfpathlineto{\pgfqpoint{3.916925in}{2.906455in}}%
\pgfpathlineto{\pgfqpoint{3.918930in}{2.796000in}}%
\pgfpathlineto{\pgfqpoint{3.919932in}{2.865545in}}%
\pgfpathlineto{\pgfqpoint{3.920934in}{2.838273in}}%
\pgfpathlineto{\pgfqpoint{3.921937in}{2.864182in}}%
\pgfpathlineto{\pgfqpoint{3.923941in}{2.761909in}}%
\pgfpathlineto{\pgfqpoint{3.924943in}{2.834182in}}%
\pgfpathlineto{\pgfqpoint{3.925946in}{2.763273in}}%
\pgfpathlineto{\pgfqpoint{3.926948in}{2.858727in}}%
\pgfpathlineto{\pgfqpoint{3.928952in}{2.768727in}}%
\pgfpathlineto{\pgfqpoint{3.929955in}{2.861455in}}%
\pgfpathlineto{\pgfqpoint{3.930957in}{2.801455in}}%
\pgfpathlineto{\pgfqpoint{3.931959in}{2.920091in}}%
\pgfpathlineto{\pgfqpoint{3.932961in}{2.898273in}}%
\pgfpathlineto{\pgfqpoint{3.934966in}{2.932364in}}%
\pgfpathlineto{\pgfqpoint{3.935968in}{2.921455in}}%
\pgfpathlineto{\pgfqpoint{3.936970in}{3.091909in}}%
\pgfpathlineto{\pgfqpoint{3.937972in}{3.063273in}}%
\pgfpathlineto{\pgfqpoint{3.938975in}{3.416455in}}%
\pgfpathlineto{\pgfqpoint{3.940979in}{3.055091in}}%
\pgfpathlineto{\pgfqpoint{3.942984in}{3.660545in}}%
\pgfpathlineto{\pgfqpoint{3.944988in}{3.777818in}}%
\pgfpathlineto{\pgfqpoint{3.945990in}{3.521455in}}%
\pgfpathlineto{\pgfqpoint{3.947995in}{3.773727in}}%
\pgfpathlineto{\pgfqpoint{3.948997in}{3.843273in}}%
\pgfpathlineto{\pgfqpoint{3.949999in}{3.792818in}}%
\pgfpathlineto{\pgfqpoint{3.951002in}{3.220091in}}%
\pgfpathlineto{\pgfqpoint{3.952004in}{3.700091in}}%
\pgfpathlineto{\pgfqpoint{3.953006in}{3.670091in}}%
\pgfpathlineto{\pgfqpoint{3.954008in}{3.779182in}}%
\pgfpathlineto{\pgfqpoint{3.955011in}{3.607364in}}%
\pgfpathlineto{\pgfqpoint{3.957015in}{3.126000in}}%
\pgfpathlineto{\pgfqpoint{3.958017in}{3.135545in}}%
\pgfpathlineto{\pgfqpoint{3.959020in}{3.535091in}}%
\pgfpathlineto{\pgfqpoint{3.961024in}{3.031909in}}%
\pgfpathlineto{\pgfqpoint{3.962026in}{3.008727in}}%
\pgfpathlineto{\pgfqpoint{3.963029in}{2.901000in}}%
\pgfpathlineto{\pgfqpoint{3.964031in}{2.982818in}}%
\pgfpathlineto{\pgfqpoint{3.967038in}{2.865545in}}%
\pgfpathlineto{\pgfqpoint{3.968040in}{2.763273in}}%
\pgfpathlineto{\pgfqpoint{3.970044in}{2.849182in}}%
\pgfpathlineto{\pgfqpoint{3.971046in}{2.783727in}}%
\pgfpathlineto{\pgfqpoint{3.972049in}{2.816455in}}%
\pgfpathlineto{\pgfqpoint{3.973051in}{2.682818in}}%
\pgfpathlineto{\pgfqpoint{3.975055in}{2.791909in}}%
\pgfpathlineto{\pgfqpoint{3.976058in}{2.811000in}}%
\pgfpathlineto{\pgfqpoint{3.977060in}{2.875091in}}%
\pgfpathlineto{\pgfqpoint{3.978062in}{2.745545in}}%
\pgfpathlineto{\pgfqpoint{3.980067in}{2.816455in}}%
\pgfpathlineto{\pgfqpoint{3.982071in}{2.910545in}}%
\pgfpathlineto{\pgfqpoint{3.983073in}{2.918727in}}%
\pgfpathlineto{\pgfqpoint{3.984076in}{2.943273in}}%
\pgfpathlineto{\pgfqpoint{3.985078in}{2.894182in}}%
\pgfpathlineto{\pgfqpoint{3.986080in}{2.935091in}}%
\pgfpathlineto{\pgfqpoint{3.988085in}{3.106909in}}%
\pgfpathlineto{\pgfqpoint{3.989087in}{3.581455in}}%
\pgfpathlineto{\pgfqpoint{3.990089in}{3.076909in}}%
\pgfpathlineto{\pgfqpoint{3.991091in}{3.097364in}}%
\pgfpathlineto{\pgfqpoint{3.994098in}{3.753273in}}%
\pgfpathlineto{\pgfqpoint{3.995100in}{3.547364in}}%
\pgfpathlineto{\pgfqpoint{3.997105in}{3.741000in}}%
\pgfpathlineto{\pgfqpoint{3.998107in}{3.723273in}}%
\pgfpathlineto{\pgfqpoint{3.999109in}{3.825545in}}%
\pgfpathlineto{\pgfqpoint{4.001114in}{3.558273in}}%
\pgfpathlineto{\pgfqpoint{4.004121in}{3.736909in}}%
\pgfpathlineto{\pgfqpoint{4.007127in}{3.161455in}}%
\pgfpathlineto{\pgfqpoint{4.009132in}{3.405545in}}%
\pgfpathlineto{\pgfqpoint{4.011136in}{3.060545in}}%
\pgfpathlineto{\pgfqpoint{4.013141in}{2.944636in}}%
\pgfpathlineto{\pgfqpoint{4.014143in}{3.025091in}}%
\pgfpathlineto{\pgfqpoint{4.015145in}{2.928273in}}%
\pgfpathlineto{\pgfqpoint{4.016147in}{2.969182in}}%
\pgfpathlineto{\pgfqpoint{4.018152in}{2.772818in}}%
\pgfpathlineto{\pgfqpoint{4.019154in}{2.875091in}}%
\pgfpathlineto{\pgfqpoint{4.020156in}{2.841000in}}%
\pgfpathlineto{\pgfqpoint{4.021159in}{2.922818in}}%
\pgfpathlineto{\pgfqpoint{4.023163in}{2.725091in}}%
\pgfpathlineto{\pgfqpoint{4.024165in}{2.789182in}}%
\pgfpathlineto{\pgfqpoint{4.025168in}{2.771455in}}%
\pgfpathlineto{\pgfqpoint{4.026170in}{2.888727in}}%
\pgfpathlineto{\pgfqpoint{4.028174in}{2.719636in}}%
\pgfpathlineto{\pgfqpoint{4.030179in}{2.767364in}}%
\pgfpathlineto{\pgfqpoint{4.031181in}{2.881909in}}%
\pgfpathlineto{\pgfqpoint{4.032183in}{2.877818in}}%
\pgfpathlineto{\pgfqpoint{4.033186in}{2.835545in}}%
\pgfpathlineto{\pgfqpoint{4.034188in}{2.838273in}}%
\pgfpathlineto{\pgfqpoint{4.035190in}{2.860091in}}%
\pgfpathlineto{\pgfqpoint{4.039199in}{3.127364in}}%
\pgfpathlineto{\pgfqpoint{4.040201in}{2.969182in}}%
\pgfpathlineto{\pgfqpoint{4.041203in}{3.250091in}}%
\pgfpathlineto{\pgfqpoint{4.042206in}{3.083727in}}%
\pgfpathlineto{\pgfqpoint{4.044210in}{3.734182in}}%
\pgfpathlineto{\pgfqpoint{4.045212in}{3.540545in}}%
\pgfpathlineto{\pgfqpoint{4.046215in}{3.661909in}}%
\pgfpathlineto{\pgfqpoint{4.047217in}{3.592364in}}%
\pgfpathlineto{\pgfqpoint{4.049221in}{3.841909in}}%
\pgfpathlineto{\pgfqpoint{4.052228in}{3.619636in}}%
\pgfpathlineto{\pgfqpoint{4.054233in}{3.817364in}}%
\pgfpathlineto{\pgfqpoint{4.057239in}{3.316909in}}%
\pgfpathlineto{\pgfqpoint{4.059244in}{3.649636in}}%
\pgfpathlineto{\pgfqpoint{4.062251in}{2.974636in}}%
\pgfpathlineto{\pgfqpoint{4.064255in}{3.087818in}}%
\pgfpathlineto{\pgfqpoint{4.066260in}{3.022364in}}%
\pgfpathlineto{\pgfqpoint{4.067262in}{2.828727in}}%
\pgfpathlineto{\pgfqpoint{4.069266in}{2.935091in}}%
\pgfpathlineto{\pgfqpoint{4.070269in}{2.935091in}}%
\pgfpathlineto{\pgfqpoint{4.072273in}{2.800091in}}%
\pgfpathlineto{\pgfqpoint{4.074278in}{2.890091in}}%
\pgfpathlineto{\pgfqpoint{4.075280in}{2.857364in}}%
\pgfpathlineto{\pgfqpoint{4.076282in}{2.881909in}}%
\pgfpathlineto{\pgfqpoint{4.077284in}{2.781000in}}%
\pgfpathlineto{\pgfqpoint{4.078286in}{2.798727in}}%
\pgfpathlineto{\pgfqpoint{4.079289in}{2.794636in}}%
\pgfpathlineto{\pgfqpoint{4.080291in}{2.819182in}}%
\pgfpathlineto{\pgfqpoint{4.081293in}{2.946000in}}%
\pgfpathlineto{\pgfqpoint{4.082295in}{2.789182in}}%
\pgfpathlineto{\pgfqpoint{4.084300in}{2.838273in}}%
\pgfpathlineto{\pgfqpoint{4.085302in}{3.081000in}}%
\pgfpathlineto{\pgfqpoint{4.086304in}{2.861455in}}%
\pgfpathlineto{\pgfqpoint{4.087307in}{2.944636in}}%
\pgfpathlineto{\pgfqpoint{4.088309in}{2.823273in}}%
\pgfpathlineto{\pgfqpoint{4.089311in}{3.102818in}}%
\pgfpathlineto{\pgfqpoint{4.090313in}{3.750545in}}%
\pgfpathlineto{\pgfqpoint{4.092318in}{2.967818in}}%
\pgfpathlineto{\pgfqpoint{4.093320in}{3.046909in}}%
\pgfpathlineto{\pgfqpoint{4.095325in}{3.700091in}}%
\pgfpathlineto{\pgfqpoint{4.096327in}{3.664636in}}%
\pgfpathlineto{\pgfqpoint{4.097329in}{3.682364in}}%
\pgfpathlineto{\pgfqpoint{4.098331in}{3.730091in}}%
\pgfpathlineto{\pgfqpoint{4.099334in}{3.604636in}}%
\pgfpathlineto{\pgfqpoint{4.100336in}{3.289636in}}%
\pgfpathlineto{\pgfqpoint{4.101338in}{3.731455in}}%
\pgfpathlineto{\pgfqpoint{4.102340in}{3.698727in}}%
\pgfpathlineto{\pgfqpoint{4.103343in}{3.168273in}}%
\pgfpathlineto{\pgfqpoint{4.104345in}{3.175091in}}%
\pgfpathlineto{\pgfqpoint{4.105347in}{3.150545in}}%
\pgfpathlineto{\pgfqpoint{4.106349in}{3.359182in}}%
\pgfpathlineto{\pgfqpoint{4.107352in}{3.236455in}}%
\pgfpathlineto{\pgfqpoint{4.108354in}{3.281455in}}%
\pgfpathlineto{\pgfqpoint{4.109356in}{3.030545in}}%
\pgfpathlineto{\pgfqpoint{4.110358in}{3.101455in}}%
\pgfpathlineto{\pgfqpoint{4.111361in}{2.969182in}}%
\pgfpathlineto{\pgfqpoint{4.112363in}{2.997818in}}%
\pgfpathlineto{\pgfqpoint{4.114367in}{3.081000in}}%
\pgfpathlineto{\pgfqpoint{4.115369in}{3.081000in}}%
\pgfpathlineto{\pgfqpoint{4.117374in}{2.805545in}}%
\pgfpathlineto{\pgfqpoint{4.120381in}{3.014182in}}%
\pgfpathlineto{\pgfqpoint{4.121383in}{2.928273in}}%
\pgfpathlineto{\pgfqpoint{4.122385in}{2.723727in}}%
\pgfpathlineto{\pgfqpoint{4.124390in}{2.800091in}}%
\pgfpathlineto{\pgfqpoint{4.126394in}{2.971909in}}%
\pgfpathlineto{\pgfqpoint{4.128399in}{2.729182in}}%
\pgfpathlineto{\pgfqpoint{4.131405in}{2.955545in}}%
\pgfpathlineto{\pgfqpoint{4.134412in}{2.768727in}}%
\pgfpathlineto{\pgfqpoint{4.135414in}{3.036000in}}%
\pgfpathlineto{\pgfqpoint{4.136417in}{2.973273in}}%
\pgfpathlineto{\pgfqpoint{4.137419in}{3.031909in}}%
\pgfpathlineto{\pgfqpoint{4.139423in}{2.913273in}}%
\pgfpathlineto{\pgfqpoint{4.140426in}{3.100091in}}%
\pgfpathlineto{\pgfqpoint{4.141428in}{3.034636in}}%
\pgfpathlineto{\pgfqpoint{4.142430in}{3.162818in}}%
\pgfpathlineto{\pgfqpoint{4.144435in}{3.618273in}}%
\pgfpathlineto{\pgfqpoint{4.145437in}{3.663273in}}%
\pgfpathlineto{\pgfqpoint{4.146439in}{3.176455in}}%
\pgfpathlineto{\pgfqpoint{4.148443in}{3.753273in}}%
\pgfpathlineto{\pgfqpoint{4.149446in}{3.743727in}}%
\pgfpathlineto{\pgfqpoint{4.150448in}{3.757364in}}%
\pgfpathlineto{\pgfqpoint{4.151450in}{3.486000in}}%
\pgfpathlineto{\pgfqpoint{4.153455in}{3.799636in}}%
\pgfpathlineto{\pgfqpoint{4.154457in}{3.712364in}}%
\pgfpathlineto{\pgfqpoint{4.155459in}{3.764182in}}%
\pgfpathlineto{\pgfqpoint{4.157464in}{3.368727in}}%
\pgfpathlineto{\pgfqpoint{4.158466in}{3.438273in}}%
\pgfpathlineto{\pgfqpoint{4.159468in}{3.327818in}}%
\pgfpathlineto{\pgfqpoint{4.160470in}{3.599182in}}%
\pgfpathlineto{\pgfqpoint{4.162475in}{3.064636in}}%
\pgfpathlineto{\pgfqpoint{4.163477in}{2.984182in}}%
\pgfpathlineto{\pgfqpoint{4.165482in}{3.106909in}}%
\pgfpathlineto{\pgfqpoint{4.169491in}{2.869636in}}%
\pgfpathlineto{\pgfqpoint{4.170493in}{2.989636in}}%
\pgfpathlineto{\pgfqpoint{4.171495in}{2.909182in}}%
\pgfpathlineto{\pgfqpoint{4.172497in}{2.940545in}}%
\pgfpathlineto{\pgfqpoint{4.174502in}{2.805545in}}%
\pgfpathlineto{\pgfqpoint{4.175504in}{2.898273in}}%
\pgfpathlineto{\pgfqpoint{4.176506in}{2.821909in}}%
\pgfpathlineto{\pgfqpoint{4.177509in}{2.827364in}}%
\pgfpathlineto{\pgfqpoint{4.179513in}{2.764636in}}%
\pgfpathlineto{\pgfqpoint{4.180515in}{2.860091in}}%
\pgfpathlineto{\pgfqpoint{4.181518in}{2.816455in}}%
\pgfpathlineto{\pgfqpoint{4.182520in}{2.841000in}}%
\pgfpathlineto{\pgfqpoint{4.184524in}{2.772818in}}%
\pgfpathlineto{\pgfqpoint{4.186529in}{2.881909in}}%
\pgfpathlineto{\pgfqpoint{4.187531in}{2.866909in}}%
\pgfpathlineto{\pgfqpoint{4.188533in}{2.816455in}}%
\pgfpathlineto{\pgfqpoint{4.189535in}{2.835545in}}%
\pgfpathlineto{\pgfqpoint{4.191540in}{2.965091in}}%
\pgfpathlineto{\pgfqpoint{4.192542in}{3.003273in}}%
\pgfpathlineto{\pgfqpoint{4.194547in}{2.958273in}}%
\pgfpathlineto{\pgfqpoint{4.195549in}{3.090545in}}%
\pgfpathlineto{\pgfqpoint{4.196551in}{3.014182in}}%
\pgfpathlineto{\pgfqpoint{4.197553in}{3.547364in}}%
\pgfpathlineto{\pgfqpoint{4.198556in}{3.378273in}}%
\pgfpathlineto{\pgfqpoint{4.199558in}{3.507818in}}%
\pgfpathlineto{\pgfqpoint{4.200560in}{3.473727in}}%
\pgfpathlineto{\pgfqpoint{4.201562in}{3.124636in}}%
\pgfpathlineto{\pgfqpoint{4.203567in}{3.747818in}}%
\pgfpathlineto{\pgfqpoint{4.205571in}{3.641455in}}%
\pgfpathlineto{\pgfqpoint{4.206574in}{3.203727in}}%
\pgfpathlineto{\pgfqpoint{4.208578in}{3.769636in}}%
\pgfpathlineto{\pgfqpoint{4.210583in}{3.705545in}}%
\pgfpathlineto{\pgfqpoint{4.211585in}{3.454636in}}%
\pgfpathlineto{\pgfqpoint{4.212587in}{3.629182in}}%
\pgfpathlineto{\pgfqpoint{4.213589in}{3.571909in}}%
\pgfpathlineto{\pgfqpoint{4.214592in}{3.608727in}}%
\pgfpathlineto{\pgfqpoint{4.215594in}{3.569182in}}%
\pgfpathlineto{\pgfqpoint{4.217598in}{3.117818in}}%
\pgfpathlineto{\pgfqpoint{4.218600in}{3.086455in}}%
\pgfpathlineto{\pgfqpoint{4.219603in}{3.120545in}}%
\pgfpathlineto{\pgfqpoint{4.220605in}{3.207818in}}%
\pgfpathlineto{\pgfqpoint{4.222609in}{3.001909in}}%
\pgfpathlineto{\pgfqpoint{4.223612in}{2.914636in}}%
\pgfpathlineto{\pgfqpoint{4.224614in}{2.986909in}}%
\pgfpathlineto{\pgfqpoint{4.225616in}{2.974636in}}%
\pgfpathlineto{\pgfqpoint{4.229625in}{2.806909in}}%
\pgfpathlineto{\pgfqpoint{4.230627in}{2.887364in}}%
\pgfpathlineto{\pgfqpoint{4.231630in}{2.881909in}}%
\pgfpathlineto{\pgfqpoint{4.232632in}{2.909182in}}%
\pgfpathlineto{\pgfqpoint{4.234636in}{2.775545in}}%
\pgfpathlineto{\pgfqpoint{4.235639in}{2.836909in}}%
\pgfpathlineto{\pgfqpoint{4.236641in}{2.782364in}}%
\pgfpathlineto{\pgfqpoint{4.237643in}{2.826000in}}%
\pgfpathlineto{\pgfqpoint{4.239648in}{2.782364in}}%
\pgfpathlineto{\pgfqpoint{4.240650in}{2.887364in}}%
\pgfpathlineto{\pgfqpoint{4.241652in}{2.872364in}}%
\pgfpathlineto{\pgfqpoint{4.242654in}{2.877818in}}%
\pgfpathlineto{\pgfqpoint{4.243657in}{2.830091in}}%
\pgfpathlineto{\pgfqpoint{4.245661in}{2.895545in}}%
\pgfpathlineto{\pgfqpoint{4.246663in}{2.933727in}}%
\pgfpathlineto{\pgfqpoint{4.247666in}{3.025091in}}%
\pgfpathlineto{\pgfqpoint{4.248668in}{2.969182in}}%
\pgfpathlineto{\pgfqpoint{4.251675in}{3.057818in}}%
\pgfpathlineto{\pgfqpoint{4.252677in}{3.499636in}}%
\pgfpathlineto{\pgfqpoint{4.253679in}{3.446455in}}%
\pgfpathlineto{\pgfqpoint{4.255683in}{3.222818in}}%
\pgfpathlineto{\pgfqpoint{4.256686in}{3.081000in}}%
\pgfpathlineto{\pgfqpoint{4.258690in}{3.690545in}}%
\pgfpathlineto{\pgfqpoint{4.259692in}{3.664636in}}%
\pgfpathlineto{\pgfqpoint{4.260695in}{3.701455in}}%
\pgfpathlineto{\pgfqpoint{4.261697in}{3.368727in}}%
\pgfpathlineto{\pgfqpoint{4.262699in}{3.735545in}}%
\pgfpathlineto{\pgfqpoint{4.263701in}{3.702818in}}%
\pgfpathlineto{\pgfqpoint{4.265706in}{3.614182in}}%
\pgfpathlineto{\pgfqpoint{4.266708in}{3.396000in}}%
\pgfpathlineto{\pgfqpoint{4.267710in}{3.715091in}}%
\pgfpathlineto{\pgfqpoint{4.268713in}{3.712364in}}%
\pgfpathlineto{\pgfqpoint{4.269715in}{3.675545in}}%
\pgfpathlineto{\pgfqpoint{4.270717in}{3.490091in}}%
\pgfpathlineto{\pgfqpoint{4.271719in}{3.079636in}}%
\pgfpathlineto{\pgfqpoint{4.272722in}{3.169636in}}%
\pgfpathlineto{\pgfqpoint{4.273724in}{3.134182in}}%
\pgfpathlineto{\pgfqpoint{4.274726in}{3.301909in}}%
\pgfpathlineto{\pgfqpoint{4.276731in}{3.019636in}}%
\pgfpathlineto{\pgfqpoint{4.277733in}{3.011455in}}%
\pgfpathlineto{\pgfqpoint{4.278735in}{2.970545in}}%
\pgfpathlineto{\pgfqpoint{4.280740in}{3.006000in}}%
\pgfpathlineto{\pgfqpoint{4.281742in}{2.877818in}}%
\pgfpathlineto{\pgfqpoint{4.282744in}{2.905091in}}%
\pgfpathlineto{\pgfqpoint{4.283746in}{2.819182in}}%
\pgfpathlineto{\pgfqpoint{4.285751in}{2.924182in}}%
\pgfpathlineto{\pgfqpoint{4.286753in}{2.831455in}}%
\pgfpathlineto{\pgfqpoint{4.287755in}{2.877818in}}%
\pgfpathlineto{\pgfqpoint{4.288758in}{2.771455in}}%
\pgfpathlineto{\pgfqpoint{4.290762in}{2.851909in}}%
\pgfpathlineto{\pgfqpoint{4.291764in}{2.862818in}}%
\pgfpathlineto{\pgfqpoint{4.292766in}{2.860091in}}%
\pgfpathlineto{\pgfqpoint{4.293769in}{2.760545in}}%
\pgfpathlineto{\pgfqpoint{4.294771in}{2.782364in}}%
\pgfpathlineto{\pgfqpoint{4.296775in}{2.872364in}}%
\pgfpathlineto{\pgfqpoint{4.297778in}{2.877818in}}%
\pgfpathlineto{\pgfqpoint{4.298780in}{2.805545in}}%
\pgfpathlineto{\pgfqpoint{4.299782in}{2.817818in}}%
\pgfpathlineto{\pgfqpoint{4.302789in}{2.999182in}}%
\pgfpathlineto{\pgfqpoint{4.303791in}{2.920091in}}%
\pgfpathlineto{\pgfqpoint{4.305796in}{2.950091in}}%
\pgfpathlineto{\pgfqpoint{4.306798in}{3.001909in}}%
\pgfpathlineto{\pgfqpoint{4.307800in}{3.498273in}}%
\pgfpathlineto{\pgfqpoint{4.308802in}{3.181909in}}%
\pgfpathlineto{\pgfqpoint{4.309805in}{3.301909in}}%
\pgfpathlineto{\pgfqpoint{4.310807in}{3.075545in}}%
\pgfpathlineto{\pgfqpoint{4.311809in}{3.120545in}}%
\pgfpathlineto{\pgfqpoint{4.312811in}{3.668727in}}%
\pgfpathlineto{\pgfqpoint{4.313814in}{3.615545in}}%
\pgfpathlineto{\pgfqpoint{4.314816in}{3.645545in}}%
\pgfpathlineto{\pgfqpoint{4.316820in}{3.346909in}}%
\pgfpathlineto{\pgfqpoint{4.318825in}{3.711000in}}%
\pgfpathlineto{\pgfqpoint{4.319827in}{3.701455in}}%
\pgfpathlineto{\pgfqpoint{4.321832in}{3.432818in}}%
\pgfpathlineto{\pgfqpoint{4.322834in}{3.697364in}}%
\pgfpathlineto{\pgfqpoint{4.323836in}{3.685091in}}%
\pgfpathlineto{\pgfqpoint{4.324838in}{3.681000in}}%
\pgfpathlineto{\pgfqpoint{4.325840in}{3.521455in}}%
\pgfpathlineto{\pgfqpoint{4.326843in}{3.150545in}}%
\pgfpathlineto{\pgfqpoint{4.327845in}{3.301909in}}%
\pgfpathlineto{\pgfqpoint{4.328847in}{3.246000in}}%
\pgfpathlineto{\pgfqpoint{4.329849in}{3.366000in}}%
\pgfpathlineto{\pgfqpoint{4.331854in}{3.064636in}}%
\pgfpathlineto{\pgfqpoint{4.332856in}{3.056455in}}%
\pgfpathlineto{\pgfqpoint{4.333858in}{2.989636in}}%
\pgfpathlineto{\pgfqpoint{4.334861in}{3.048273in}}%
\pgfpathlineto{\pgfqpoint{4.338870in}{2.866909in}}%
\pgfpathlineto{\pgfqpoint{4.339872in}{2.917364in}}%
\pgfpathlineto{\pgfqpoint{4.340874in}{2.899636in}}%
\pgfpathlineto{\pgfqpoint{4.341876in}{2.907818in}}%
\pgfpathlineto{\pgfqpoint{4.343881in}{2.815091in}}%
\pgfpathlineto{\pgfqpoint{4.344883in}{2.826000in}}%
\pgfpathlineto{\pgfqpoint{4.345885in}{2.864182in}}%
\pgfpathlineto{\pgfqpoint{4.347890in}{2.838273in}}%
\pgfpathlineto{\pgfqpoint{4.348892in}{2.787818in}}%
\pgfpathlineto{\pgfqpoint{4.350897in}{2.890091in}}%
\pgfpathlineto{\pgfqpoint{4.351899in}{2.838273in}}%
\pgfpathlineto{\pgfqpoint{4.352901in}{2.854636in}}%
\pgfpathlineto{\pgfqpoint{4.353903in}{2.789182in}}%
\pgfpathlineto{\pgfqpoint{4.356910in}{2.901000in}}%
\pgfpathlineto{\pgfqpoint{4.357912in}{2.903727in}}%
\pgfpathlineto{\pgfqpoint{4.358915in}{2.841000in}}%
\pgfpathlineto{\pgfqpoint{4.361921in}{3.052364in}}%
\pgfpathlineto{\pgfqpoint{4.362923in}{3.033273in}}%
\pgfpathlineto{\pgfqpoint{4.363926in}{2.980091in}}%
\pgfpathlineto{\pgfqpoint{4.365930in}{3.064636in}}%
\pgfpathlineto{\pgfqpoint{4.366932in}{3.192818in}}%
\pgfpathlineto{\pgfqpoint{4.367935in}{3.589636in}}%
\pgfpathlineto{\pgfqpoint{4.370941in}{3.076909in}}%
\pgfpathlineto{\pgfqpoint{4.373948in}{3.776455in}}%
\pgfpathlineto{\pgfqpoint{4.374950in}{3.762818in}}%
\pgfpathlineto{\pgfqpoint{4.376955in}{3.199636in}}%
\pgfpathlineto{\pgfqpoint{4.378959in}{3.694636in}}%
\pgfpathlineto{\pgfqpoint{4.379962in}{3.720545in}}%
\pgfpathlineto{\pgfqpoint{4.382968in}{3.248727in}}%
\pgfpathlineto{\pgfqpoint{4.383971in}{3.211909in}}%
\pgfpathlineto{\pgfqpoint{4.384973in}{3.312818in}}%
\pgfpathlineto{\pgfqpoint{4.386977in}{3.064636in}}%
\pgfpathlineto{\pgfqpoint{4.388982in}{2.981455in}}%
\pgfpathlineto{\pgfqpoint{4.389984in}{3.045545in}}%
\pgfpathlineto{\pgfqpoint{4.391989in}{2.969182in}}%
\pgfpathlineto{\pgfqpoint{4.392991in}{2.954182in}}%
\pgfpathlineto{\pgfqpoint{4.393993in}{2.903727in}}%
\pgfpathlineto{\pgfqpoint{4.394995in}{2.951455in}}%
\pgfpathlineto{\pgfqpoint{4.395997in}{2.884636in}}%
\pgfpathlineto{\pgfqpoint{4.397000in}{2.898273in}}%
\pgfpathlineto{\pgfqpoint{4.398002in}{2.881909in}}%
\pgfpathlineto{\pgfqpoint{4.399004in}{2.832818in}}%
\pgfpathlineto{\pgfqpoint{4.400006in}{2.913273in}}%
\pgfpathlineto{\pgfqpoint{4.401009in}{2.850545in}}%
\pgfpathlineto{\pgfqpoint{4.402011in}{2.856000in}}%
\pgfpathlineto{\pgfqpoint{4.404015in}{2.791909in}}%
\pgfpathlineto{\pgfqpoint{4.406020in}{2.883273in}}%
\pgfpathlineto{\pgfqpoint{4.407022in}{2.888727in}}%
\pgfpathlineto{\pgfqpoint{4.409027in}{2.823273in}}%
\pgfpathlineto{\pgfqpoint{4.411031in}{2.931000in}}%
\pgfpathlineto{\pgfqpoint{4.412033in}{2.955545in}}%
\pgfpathlineto{\pgfqpoint{4.414038in}{2.910545in}}%
\pgfpathlineto{\pgfqpoint{4.417045in}{3.156000in}}%
\pgfpathlineto{\pgfqpoint{4.418047in}{3.183273in}}%
\pgfpathlineto{\pgfqpoint{4.419049in}{3.090545in}}%
\pgfpathlineto{\pgfqpoint{4.420051in}{3.205091in}}%
\pgfpathlineto{\pgfqpoint{4.421054in}{3.045545in}}%
\pgfpathlineto{\pgfqpoint{4.423058in}{3.634636in}}%
\pgfpathlineto{\pgfqpoint{4.424060in}{3.646909in}}%
\pgfpathlineto{\pgfqpoint{4.425063in}{3.641455in}}%
\pgfpathlineto{\pgfqpoint{4.426065in}{3.158727in}}%
\pgfpathlineto{\pgfqpoint{4.428069in}{3.670091in}}%
\pgfpathlineto{\pgfqpoint{4.429072in}{3.690545in}}%
\pgfpathlineto{\pgfqpoint{4.430074in}{3.668727in}}%
\pgfpathlineto{\pgfqpoint{4.431076in}{3.372818in}}%
\pgfpathlineto{\pgfqpoint{4.432078in}{3.409636in}}%
\pgfpathlineto{\pgfqpoint{4.434083in}{3.648273in}}%
\pgfpathlineto{\pgfqpoint{4.435085in}{3.693273in}}%
\pgfpathlineto{\pgfqpoint{4.437089in}{3.141000in}}%
\pgfpathlineto{\pgfqpoint{4.438092in}{3.136909in}}%
\pgfpathlineto{\pgfqpoint{4.440096in}{3.192818in}}%
\pgfpathlineto{\pgfqpoint{4.441098in}{3.015545in}}%
\pgfpathlineto{\pgfqpoint{4.442101in}{3.036000in}}%
\pgfpathlineto{\pgfqpoint{4.444105in}{2.977364in}}%
\pgfpathlineto{\pgfqpoint{4.445107in}{3.008727in}}%
\pgfpathlineto{\pgfqpoint{4.446110in}{2.948727in}}%
\pgfpathlineto{\pgfqpoint{4.447112in}{2.985545in}}%
\pgfpathlineto{\pgfqpoint{4.449116in}{2.872364in}}%
\pgfpathlineto{\pgfqpoint{4.451121in}{2.909182in}}%
\pgfpathlineto{\pgfqpoint{4.452123in}{2.911909in}}%
\pgfpathlineto{\pgfqpoint{4.454128in}{2.802818in}}%
\pgfpathlineto{\pgfqpoint{4.455130in}{2.886000in}}%
\pgfpathlineto{\pgfqpoint{4.456132in}{2.868273in}}%
\pgfpathlineto{\pgfqpoint{4.457134in}{2.884636in}}%
\pgfpathlineto{\pgfqpoint{4.459139in}{2.802818in}}%
\pgfpathlineto{\pgfqpoint{4.462146in}{2.931000in}}%
\pgfpathlineto{\pgfqpoint{4.464150in}{2.858727in}}%
\pgfpathlineto{\pgfqpoint{4.466154in}{2.921455in}}%
\pgfpathlineto{\pgfqpoint{4.467157in}{3.044182in}}%
\pgfpathlineto{\pgfqpoint{4.469161in}{3.003273in}}%
\pgfpathlineto{\pgfqpoint{4.470163in}{3.011455in}}%
\pgfpathlineto{\pgfqpoint{4.471166in}{3.000545in}}%
\pgfpathlineto{\pgfqpoint{4.474172in}{3.438273in}}%
\pgfpathlineto{\pgfqpoint{4.476177in}{3.071455in}}%
\pgfpathlineto{\pgfqpoint{4.480186in}{3.691909in}}%
\pgfpathlineto{\pgfqpoint{4.481188in}{3.267818in}}%
\pgfpathlineto{\pgfqpoint{4.483193in}{3.611455in}}%
\pgfpathlineto{\pgfqpoint{4.485197in}{3.660545in}}%
\pgfpathlineto{\pgfqpoint{4.486199in}{3.297818in}}%
\pgfpathlineto{\pgfqpoint{4.487202in}{3.442364in}}%
\pgfpathlineto{\pgfqpoint{4.488204in}{3.150545in}}%
\pgfpathlineto{\pgfqpoint{4.489206in}{3.338727in}}%
\pgfpathlineto{\pgfqpoint{4.490208in}{3.277364in}}%
\pgfpathlineto{\pgfqpoint{4.492213in}{3.097364in}}%
\pgfpathlineto{\pgfqpoint{4.493215in}{3.016909in}}%
\pgfpathlineto{\pgfqpoint{4.494217in}{3.042818in}}%
\pgfpathlineto{\pgfqpoint{4.497224in}{2.989636in}}%
\pgfpathlineto{\pgfqpoint{4.498226in}{2.916000in}}%
\pgfpathlineto{\pgfqpoint{4.499229in}{2.932364in}}%
\pgfpathlineto{\pgfqpoint{4.500231in}{2.925545in}}%
\pgfpathlineto{\pgfqpoint{4.501233in}{2.928273in}}%
\pgfpathlineto{\pgfqpoint{4.502235in}{2.913273in}}%
\pgfpathlineto{\pgfqpoint{4.503237in}{2.843727in}}%
\pgfpathlineto{\pgfqpoint{4.504240in}{2.849182in}}%
\pgfpathlineto{\pgfqpoint{4.505242in}{2.892818in}}%
\pgfpathlineto{\pgfqpoint{4.506244in}{2.872364in}}%
\pgfpathlineto{\pgfqpoint{4.507246in}{2.920091in}}%
\pgfpathlineto{\pgfqpoint{4.508249in}{2.812364in}}%
\pgfpathlineto{\pgfqpoint{4.512258in}{2.901000in}}%
\pgfpathlineto{\pgfqpoint{4.513260in}{2.839636in}}%
\pgfpathlineto{\pgfqpoint{4.515264in}{2.879182in}}%
\pgfpathlineto{\pgfqpoint{4.517269in}{2.976000in}}%
\pgfpathlineto{\pgfqpoint{4.518271in}{2.978727in}}%
\pgfpathlineto{\pgfqpoint{4.519273in}{2.985545in}}%
\pgfpathlineto{\pgfqpoint{4.520276in}{2.970545in}}%
\pgfpathlineto{\pgfqpoint{4.522280in}{3.070091in}}%
\pgfpathlineto{\pgfqpoint{4.523282in}{3.053727in}}%
\pgfpathlineto{\pgfqpoint{4.524285in}{3.209182in}}%
\pgfpathlineto{\pgfqpoint{4.525287in}{3.121909in}}%
\pgfpathlineto{\pgfqpoint{4.526289in}{3.132818in}}%
\pgfpathlineto{\pgfqpoint{4.529296in}{3.615545in}}%
\pgfpathlineto{\pgfqpoint{4.531300in}{3.231000in}}%
\pgfpathlineto{\pgfqpoint{4.533305in}{3.551455in}}%
\pgfpathlineto{\pgfqpoint{4.534307in}{3.637364in}}%
\pgfpathlineto{\pgfqpoint{4.536312in}{3.255545in}}%
\pgfpathlineto{\pgfqpoint{4.537314in}{3.295091in}}%
\pgfpathlineto{\pgfqpoint{4.538316in}{3.254182in}}%
\pgfpathlineto{\pgfqpoint{4.539318in}{3.360545in}}%
\pgfpathlineto{\pgfqpoint{4.541323in}{3.094636in}}%
\pgfpathlineto{\pgfqpoint{4.543327in}{3.026455in}}%
\pgfpathlineto{\pgfqpoint{4.544329in}{3.071455in}}%
\pgfpathlineto{\pgfqpoint{4.548338in}{2.909182in}}%
\pgfpathlineto{\pgfqpoint{4.549341in}{2.940545in}}%
\pgfpathlineto{\pgfqpoint{4.550343in}{2.926909in}}%
\pgfpathlineto{\pgfqpoint{4.551345in}{2.956909in}}%
\pgfpathlineto{\pgfqpoint{4.553350in}{2.823273in}}%
\pgfpathlineto{\pgfqpoint{4.554352in}{2.862818in}}%
\pgfpathlineto{\pgfqpoint{4.555354in}{2.853273in}}%
\pgfpathlineto{\pgfqpoint{4.556356in}{2.932364in}}%
\pgfpathlineto{\pgfqpoint{4.558361in}{2.808273in}}%
\pgfpathlineto{\pgfqpoint{4.561368in}{2.914636in}}%
\pgfpathlineto{\pgfqpoint{4.563372in}{2.842364in}}%
\pgfpathlineto{\pgfqpoint{4.567381in}{2.977364in}}%
\pgfpathlineto{\pgfqpoint{4.568383in}{2.963727in}}%
\pgfpathlineto{\pgfqpoint{4.572392in}{3.078273in}}%
\pgfpathlineto{\pgfqpoint{4.574397in}{3.372818in}}%
\pgfpathlineto{\pgfqpoint{4.575399in}{3.075545in}}%
\pgfpathlineto{\pgfqpoint{4.576401in}{3.214636in}}%
\pgfpathlineto{\pgfqpoint{4.577403in}{3.151909in}}%
\pgfpathlineto{\pgfqpoint{4.579408in}{3.599182in}}%
\pgfpathlineto{\pgfqpoint{4.580410in}{3.232364in}}%
\pgfpathlineto{\pgfqpoint{4.581412in}{3.346909in}}%
\pgfpathlineto{\pgfqpoint{4.582415in}{3.232364in}}%
\pgfpathlineto{\pgfqpoint{4.584419in}{3.586909in}}%
\pgfpathlineto{\pgfqpoint{4.587426in}{3.100091in}}%
\pgfpathlineto{\pgfqpoint{4.589430in}{3.292364in}}%
\pgfpathlineto{\pgfqpoint{4.591435in}{3.078273in}}%
\pgfpathlineto{\pgfqpoint{4.592437in}{2.992364in}}%
\pgfpathlineto{\pgfqpoint{4.594442in}{3.006000in}}%
\pgfpathlineto{\pgfqpoint{4.595444in}{2.974636in}}%
\pgfpathlineto{\pgfqpoint{4.596446in}{2.991000in}}%
\pgfpathlineto{\pgfqpoint{4.597448in}{2.921455in}}%
\pgfpathlineto{\pgfqpoint{4.598451in}{2.926909in}}%
\pgfpathlineto{\pgfqpoint{4.600455in}{2.888727in}}%
\pgfpathlineto{\pgfqpoint{4.601457in}{2.941909in}}%
\pgfpathlineto{\pgfqpoint{4.602460in}{2.864182in}}%
\pgfpathlineto{\pgfqpoint{4.604464in}{2.880545in}}%
\pgfpathlineto{\pgfqpoint{4.605466in}{2.846455in}}%
\pgfpathlineto{\pgfqpoint{4.606469in}{2.901000in}}%
\pgfpathlineto{\pgfqpoint{4.607471in}{2.853273in}}%
\pgfpathlineto{\pgfqpoint{4.608473in}{2.864182in}}%
\pgfpathlineto{\pgfqpoint{4.609475in}{2.862818in}}%
\pgfpathlineto{\pgfqpoint{4.610477in}{2.872364in}}%
\pgfpathlineto{\pgfqpoint{4.611480in}{2.910545in}}%
\pgfpathlineto{\pgfqpoint{4.612482in}{2.888727in}}%
\pgfpathlineto{\pgfqpoint{4.613484in}{2.910545in}}%
\pgfpathlineto{\pgfqpoint{4.614486in}{2.891455in}}%
\pgfpathlineto{\pgfqpoint{4.615489in}{2.896909in}}%
\pgfpathlineto{\pgfqpoint{4.618495in}{3.000545in}}%
\pgfpathlineto{\pgfqpoint{4.619498in}{3.000545in}}%
\pgfpathlineto{\pgfqpoint{4.621502in}{3.044182in}}%
\pgfpathlineto{\pgfqpoint{4.622504in}{3.030545in}}%
\pgfpathlineto{\pgfqpoint{4.623507in}{3.175091in}}%
\pgfpathlineto{\pgfqpoint{4.626513in}{3.105545in}}%
\pgfpathlineto{\pgfqpoint{4.627516in}{3.142364in}}%
\pgfpathlineto{\pgfqpoint{4.628518in}{3.555545in}}%
\pgfpathlineto{\pgfqpoint{4.629520in}{3.501000in}}%
\pgfpathlineto{\pgfqpoint{4.631525in}{3.265091in}}%
\pgfpathlineto{\pgfqpoint{4.632527in}{3.263727in}}%
\pgfpathlineto{\pgfqpoint{4.633529in}{3.570545in}}%
\pgfpathlineto{\pgfqpoint{4.634531in}{3.479182in}}%
\pgfpathlineto{\pgfqpoint{4.636536in}{3.169636in}}%
\pgfpathlineto{\pgfqpoint{4.637538in}{3.126000in}}%
\pgfpathlineto{\pgfqpoint{4.638540in}{3.217364in}}%
\pgfpathlineto{\pgfqpoint{4.640545in}{3.066000in}}%
\pgfpathlineto{\pgfqpoint{4.642549in}{2.991000in}}%
\pgfpathlineto{\pgfqpoint{4.643551in}{3.060545in}}%
\pgfpathlineto{\pgfqpoint{4.645556in}{3.001909in}}%
\pgfpathlineto{\pgfqpoint{4.647560in}{2.913273in}}%
\pgfpathlineto{\pgfqpoint{4.648563in}{2.922818in}}%
\pgfpathlineto{\pgfqpoint{4.649565in}{2.903727in}}%
\pgfpathlineto{\pgfqpoint{4.650567in}{2.939182in}}%
\pgfpathlineto{\pgfqpoint{4.652572in}{2.875091in}}%
\pgfpathlineto{\pgfqpoint{4.653574in}{2.875091in}}%
\pgfpathlineto{\pgfqpoint{4.654576in}{2.843727in}}%
\pgfpathlineto{\pgfqpoint{4.656581in}{2.902364in}}%
\pgfpathlineto{\pgfqpoint{4.659587in}{2.854636in}}%
\pgfpathlineto{\pgfqpoint{4.661592in}{2.925545in}}%
\pgfpathlineto{\pgfqpoint{4.662594in}{2.899636in}}%
\pgfpathlineto{\pgfqpoint{4.663596in}{2.921455in}}%
\pgfpathlineto{\pgfqpoint{4.664599in}{2.901000in}}%
\pgfpathlineto{\pgfqpoint{4.666603in}{2.952818in}}%
\pgfpathlineto{\pgfqpoint{4.667605in}{2.954182in}}%
\pgfpathlineto{\pgfqpoint{4.668608in}{2.988273in}}%
\pgfpathlineto{\pgfqpoint{4.669610in}{2.956909in}}%
\pgfpathlineto{\pgfqpoint{4.673619in}{3.158727in}}%
\pgfpathlineto{\pgfqpoint{4.674621in}{3.078273in}}%
\pgfpathlineto{\pgfqpoint{4.675623in}{3.097364in}}%
\pgfpathlineto{\pgfqpoint{4.676626in}{3.090545in}}%
\pgfpathlineto{\pgfqpoint{4.677628in}{3.187364in}}%
\pgfpathlineto{\pgfqpoint{4.678630in}{3.503727in}}%
\pgfpathlineto{\pgfqpoint{4.680634in}{3.191455in}}%
\pgfpathlineto{\pgfqpoint{4.681637in}{3.184636in}}%
\pgfpathlineto{\pgfqpoint{4.682639in}{3.236455in}}%
\pgfpathlineto{\pgfqpoint{4.683641in}{3.533727in}}%
\pgfpathlineto{\pgfqpoint{4.686648in}{3.105545in}}%
\pgfpathlineto{\pgfqpoint{4.687650in}{3.135545in}}%
\pgfpathlineto{\pgfqpoint{4.688652in}{3.211909in}}%
\pgfpathlineto{\pgfqpoint{4.691659in}{3.038727in}}%
\pgfpathlineto{\pgfqpoint{4.692661in}{3.021000in}}%
\pgfpathlineto{\pgfqpoint{4.693664in}{3.059182in}}%
\pgfpathlineto{\pgfqpoint{4.694666in}{3.016909in}}%
\pgfpathlineto{\pgfqpoint{4.695668in}{3.026455in}}%
\pgfpathlineto{\pgfqpoint{4.697673in}{2.950091in}}%
\pgfpathlineto{\pgfqpoint{4.699677in}{2.936455in}}%
\pgfpathlineto{\pgfqpoint{4.700679in}{2.958273in}}%
\pgfpathlineto{\pgfqpoint{4.702684in}{2.894182in}}%
\pgfpathlineto{\pgfqpoint{4.703686in}{2.892818in}}%
\pgfpathlineto{\pgfqpoint{4.704688in}{2.851909in}}%
\pgfpathlineto{\pgfqpoint{4.705691in}{2.920091in}}%
\pgfpathlineto{\pgfqpoint{4.706693in}{2.917364in}}%
\pgfpathlineto{\pgfqpoint{4.709700in}{2.865545in}}%
\pgfpathlineto{\pgfqpoint{4.711704in}{2.933727in}}%
\pgfpathlineto{\pgfqpoint{4.713709in}{2.894182in}}%
\pgfpathlineto{\pgfqpoint{4.714711in}{2.894182in}}%
\pgfpathlineto{\pgfqpoint{4.717717in}{2.982818in}}%
\pgfpathlineto{\pgfqpoint{4.718720in}{2.947364in}}%
\pgfpathlineto{\pgfqpoint{4.719722in}{2.954182in}}%
\pgfpathlineto{\pgfqpoint{4.723731in}{3.078273in}}%
\pgfpathlineto{\pgfqpoint{4.724733in}{3.053727in}}%
\pgfpathlineto{\pgfqpoint{4.725735in}{3.083727in}}%
\pgfpathlineto{\pgfqpoint{4.726738in}{3.051000in}}%
\pgfpathlineto{\pgfqpoint{4.727740in}{3.138273in}}%
\pgfpathlineto{\pgfqpoint{4.728742in}{3.329182in}}%
\pgfpathlineto{\pgfqpoint{4.729744in}{3.218727in}}%
\pgfpathlineto{\pgfqpoint{4.730747in}{3.285545in}}%
\pgfpathlineto{\pgfqpoint{4.731749in}{3.096000in}}%
\pgfpathlineto{\pgfqpoint{4.732751in}{3.179182in}}%
\pgfpathlineto{\pgfqpoint{4.733753in}{3.368727in}}%
\pgfpathlineto{\pgfqpoint{4.736760in}{3.085091in}}%
\pgfpathlineto{\pgfqpoint{4.737762in}{3.116455in}}%
\pgfpathlineto{\pgfqpoint{4.738765in}{3.102818in}}%
\pgfpathlineto{\pgfqpoint{4.740769in}{3.119182in}}%
\pgfpathlineto{\pgfqpoint{4.742774in}{3.027818in}}%
\pgfpathlineto{\pgfqpoint{4.743776in}{2.992364in}}%
\pgfpathlineto{\pgfqpoint{4.745780in}{3.038727in}}%
\pgfpathlineto{\pgfqpoint{4.748787in}{2.925545in}}%
\pgfpathlineto{\pgfqpoint{4.749789in}{2.921455in}}%
\pgfpathlineto{\pgfqpoint{4.750791in}{2.950091in}}%
\pgfpathlineto{\pgfqpoint{4.751794in}{2.940545in}}%
\pgfpathlineto{\pgfqpoint{4.752796in}{2.941909in}}%
\pgfpathlineto{\pgfqpoint{4.754800in}{2.890091in}}%
\pgfpathlineto{\pgfqpoint{4.755803in}{2.931000in}}%
\pgfpathlineto{\pgfqpoint{4.756805in}{2.924182in}}%
\pgfpathlineto{\pgfqpoint{4.757807in}{2.928273in}}%
\pgfpathlineto{\pgfqpoint{4.758809in}{2.884636in}}%
\pgfpathlineto{\pgfqpoint{4.759812in}{2.887364in}}%
\pgfpathlineto{\pgfqpoint{4.760814in}{2.931000in}}%
\pgfpathlineto{\pgfqpoint{4.761816in}{2.918727in}}%
\pgfpathlineto{\pgfqpoint{4.762818in}{2.936455in}}%
\pgfpathlineto{\pgfqpoint{4.763821in}{2.909182in}}%
\pgfpathlineto{\pgfqpoint{4.767830in}{2.973273in}}%
\pgfpathlineto{\pgfqpoint{4.768832in}{2.959636in}}%
\pgfpathlineto{\pgfqpoint{4.772841in}{3.070091in}}%
\pgfpathlineto{\pgfqpoint{4.773843in}{3.026455in}}%
\pgfpathlineto{\pgfqpoint{4.775848in}{3.102818in}}%
\pgfpathlineto{\pgfqpoint{4.776850in}{3.034636in}}%
\pgfpathlineto{\pgfqpoint{4.778854in}{3.168273in}}%
\pgfpathlineto{\pgfqpoint{4.781861in}{3.091909in}}%
\pgfpathlineto{\pgfqpoint{4.783866in}{3.251455in}}%
\pgfpathlineto{\pgfqpoint{4.784868in}{3.207818in}}%
\pgfpathlineto{\pgfqpoint{4.786872in}{3.086455in}}%
\pgfpathlineto{\pgfqpoint{4.787874in}{3.147818in}}%
\pgfpathlineto{\pgfqpoint{4.788877in}{3.091909in}}%
\pgfpathlineto{\pgfqpoint{4.789879in}{3.126000in}}%
\pgfpathlineto{\pgfqpoint{4.791883in}{3.044182in}}%
\pgfpathlineto{\pgfqpoint{4.796895in}{2.981455in}}%
\pgfpathlineto{\pgfqpoint{4.797897in}{2.992364in}}%
\pgfpathlineto{\pgfqpoint{4.798899in}{2.946000in}}%
\pgfpathlineto{\pgfqpoint{4.799901in}{2.988273in}}%
\pgfpathlineto{\pgfqpoint{4.802908in}{2.937818in}}%
\pgfpathlineto{\pgfqpoint{4.803910in}{2.905091in}}%
\pgfpathlineto{\pgfqpoint{4.805915in}{2.932364in}}%
\pgfpathlineto{\pgfqpoint{4.808922in}{2.880545in}}%
\pgfpathlineto{\pgfqpoint{4.809924in}{2.891455in}}%
\pgfpathlineto{\pgfqpoint{4.811928in}{2.950091in}}%
\pgfpathlineto{\pgfqpoint{4.812931in}{2.946000in}}%
\pgfpathlineto{\pgfqpoint{4.813933in}{2.909182in}}%
\pgfpathlineto{\pgfqpoint{4.815937in}{2.946000in}}%
\pgfpathlineto{\pgfqpoint{4.816940in}{2.984182in}}%
\pgfpathlineto{\pgfqpoint{4.818944in}{2.958273in}}%
\pgfpathlineto{\pgfqpoint{4.821951in}{3.014182in}}%
\pgfpathlineto{\pgfqpoint{4.822953in}{3.052364in}}%
\pgfpathlineto{\pgfqpoint{4.823955in}{3.034636in}}%
\pgfpathlineto{\pgfqpoint{4.824957in}{3.051000in}}%
\pgfpathlineto{\pgfqpoint{4.825960in}{3.004636in}}%
\pgfpathlineto{\pgfqpoint{4.826962in}{3.038727in}}%
\pgfpathlineto{\pgfqpoint{4.827964in}{3.134182in}}%
\pgfpathlineto{\pgfqpoint{4.828966in}{3.102818in}}%
\pgfpathlineto{\pgfqpoint{4.829969in}{3.136909in}}%
\pgfpathlineto{\pgfqpoint{4.830971in}{3.037364in}}%
\pgfpathlineto{\pgfqpoint{4.831973in}{3.056455in}}%
\pgfpathlineto{\pgfqpoint{4.834980in}{3.225545in}}%
\pgfpathlineto{\pgfqpoint{4.836984in}{3.078273in}}%
\pgfpathlineto{\pgfqpoint{4.838989in}{3.116455in}}%
\pgfpathlineto{\pgfqpoint{4.839991in}{3.131455in}}%
\pgfpathlineto{\pgfqpoint{4.841996in}{3.037364in}}%
\pgfpathlineto{\pgfqpoint{4.842998in}{3.036000in}}%
\pgfpathlineto{\pgfqpoint{4.844000in}{3.030545in}}%
\pgfpathlineto{\pgfqpoint{4.845002in}{3.041455in}}%
\pgfpathlineto{\pgfqpoint{4.847007in}{3.004636in}}%
\pgfpathlineto{\pgfqpoint{4.848009in}{2.992364in}}%
\pgfpathlineto{\pgfqpoint{4.849011in}{2.962364in}}%
\pgfpathlineto{\pgfqpoint{4.850014in}{2.988273in}}%
\pgfpathlineto{\pgfqpoint{4.851016in}{2.970545in}}%
\pgfpathlineto{\pgfqpoint{4.852018in}{2.992364in}}%
\pgfpathlineto{\pgfqpoint{4.854023in}{2.913273in}}%
\pgfpathlineto{\pgfqpoint{4.857029in}{2.971909in}}%
\pgfpathlineto{\pgfqpoint{4.859034in}{2.901000in}}%
\pgfpathlineto{\pgfqpoint{4.862040in}{2.941909in}}%
\pgfpathlineto{\pgfqpoint{4.863043in}{2.935091in}}%
\pgfpathlineto{\pgfqpoint{4.864045in}{2.907818in}}%
\pgfpathlineto{\pgfqpoint{4.867052in}{2.967818in}}%
\pgfpathlineto{\pgfqpoint{4.869056in}{2.962364in}}%
\pgfpathlineto{\pgfqpoint{4.871061in}{2.982818in}}%
\pgfpathlineto{\pgfqpoint{4.872063in}{3.029182in}}%
\pgfpathlineto{\pgfqpoint{4.873065in}{3.016909in}}%
\pgfpathlineto{\pgfqpoint{4.875070in}{3.045545in}}%
\pgfpathlineto{\pgfqpoint{4.876072in}{3.003273in}}%
\pgfpathlineto{\pgfqpoint{4.878076in}{3.089182in}}%
\pgfpathlineto{\pgfqpoint{4.879079in}{3.089182in}}%
\pgfpathlineto{\pgfqpoint{4.880081in}{3.098727in}}%
\pgfpathlineto{\pgfqpoint{4.881083in}{3.041455in}}%
\pgfpathlineto{\pgfqpoint{4.882085in}{3.059182in}}%
\pgfpathlineto{\pgfqpoint{4.884090in}{3.177818in}}%
\pgfpathlineto{\pgfqpoint{4.886094in}{3.071455in}}%
\pgfpathlineto{\pgfqpoint{4.887097in}{3.074182in}}%
\pgfpathlineto{\pgfqpoint{4.890103in}{3.128727in}}%
\pgfpathlineto{\pgfqpoint{4.892108in}{3.027818in}}%
\pgfpathlineto{\pgfqpoint{4.895114in}{3.056455in}}%
\pgfpathlineto{\pgfqpoint{4.898121in}{2.978727in}}%
\pgfpathlineto{\pgfqpoint{4.899123in}{2.997818in}}%
\pgfpathlineto{\pgfqpoint{4.900126in}{2.996455in}}%
\pgfpathlineto{\pgfqpoint{4.901128in}{2.982818in}}%
\pgfpathlineto{\pgfqpoint{4.902130in}{2.988273in}}%
\pgfpathlineto{\pgfqpoint{4.903132in}{2.947364in}}%
\pgfpathlineto{\pgfqpoint{4.905137in}{2.966455in}}%
\pgfpathlineto{\pgfqpoint{4.906139in}{2.952818in}}%
\pgfpathlineto{\pgfqpoint{4.907141in}{2.969182in}}%
\pgfpathlineto{\pgfqpoint{4.909146in}{2.913273in}}%
\pgfpathlineto{\pgfqpoint{4.912153in}{2.970545in}}%
\pgfpathlineto{\pgfqpoint{4.914157in}{2.932364in}}%
\pgfpathlineto{\pgfqpoint{4.917164in}{2.989636in}}%
\pgfpathlineto{\pgfqpoint{4.918166in}{2.965091in}}%
\pgfpathlineto{\pgfqpoint{4.919168in}{2.988273in}}%
\pgfpathlineto{\pgfqpoint{4.920171in}{2.977364in}}%
\pgfpathlineto{\pgfqpoint{4.921173in}{2.988273in}}%
\pgfpathlineto{\pgfqpoint{4.924180in}{3.060545in}}%
\pgfpathlineto{\pgfqpoint{4.926184in}{3.014182in}}%
\pgfpathlineto{\pgfqpoint{4.929191in}{3.097364in}}%
\pgfpathlineto{\pgfqpoint{4.930193in}{3.082364in}}%
\pgfpathlineto{\pgfqpoint{4.931195in}{3.042818in}}%
\pgfpathlineto{\pgfqpoint{4.934202in}{3.124636in}}%
\pgfpathlineto{\pgfqpoint{4.937209in}{3.056455in}}%
\pgfpathlineto{\pgfqpoint{4.938211in}{3.068727in}}%
\pgfpathlineto{\pgfqpoint{4.939213in}{3.102818in}}%
\pgfpathlineto{\pgfqpoint{4.941218in}{3.033273in}}%
\pgfpathlineto{\pgfqpoint{4.942220in}{3.038727in}}%
\pgfpathlineto{\pgfqpoint{4.943222in}{3.029182in}}%
\pgfpathlineto{\pgfqpoint{4.944224in}{3.057818in}}%
\pgfpathlineto{\pgfqpoint{4.946229in}{3.014182in}}%
\pgfpathlineto{\pgfqpoint{4.948233in}{2.984182in}}%
\pgfpathlineto{\pgfqpoint{4.949236in}{2.995091in}}%
\pgfpathlineto{\pgfqpoint{4.950238in}{2.993727in}}%
\pgfpathlineto{\pgfqpoint{4.952242in}{2.971909in}}%
\pgfpathlineto{\pgfqpoint{4.953245in}{2.939182in}}%
\pgfpathlineto{\pgfqpoint{4.954247in}{2.954182in}}%
\pgfpathlineto{\pgfqpoint{4.955249in}{2.947364in}}%
\pgfpathlineto{\pgfqpoint{4.956251in}{2.969182in}}%
\pgfpathlineto{\pgfqpoint{4.957254in}{2.962364in}}%
\pgfpathlineto{\pgfqpoint{4.958256in}{2.928273in}}%
\pgfpathlineto{\pgfqpoint{4.960260in}{2.937818in}}%
\pgfpathlineto{\pgfqpoint{4.962265in}{2.974636in}}%
\pgfpathlineto{\pgfqpoint{4.964269in}{2.944636in}}%
\pgfpathlineto{\pgfqpoint{4.965271in}{2.962364in}}%
\pgfpathlineto{\pgfqpoint{4.966274in}{3.001909in}}%
\pgfpathlineto{\pgfqpoint{4.968278in}{2.977364in}}%
\pgfpathlineto{\pgfqpoint{4.969280in}{2.989636in}}%
\pgfpathlineto{\pgfqpoint{4.970283in}{2.980091in}}%
\pgfpathlineto{\pgfqpoint{4.973289in}{3.045545in}}%
\pgfpathlineto{\pgfqpoint{4.974292in}{3.041455in}}%
\pgfpathlineto{\pgfqpoint{4.975294in}{3.004636in}}%
\pgfpathlineto{\pgfqpoint{4.976296in}{3.010091in}}%
\pgfpathlineto{\pgfqpoint{4.977298in}{3.026455in}}%
\pgfpathlineto{\pgfqpoint{4.978301in}{3.082364in}}%
\pgfpathlineto{\pgfqpoint{4.979303in}{3.079636in}}%
\pgfpathlineto{\pgfqpoint{4.981307in}{3.012818in}}%
\pgfpathlineto{\pgfqpoint{4.984314in}{3.109636in}}%
\pgfpathlineto{\pgfqpoint{4.986319in}{3.036000in}}%
\pgfpathlineto{\pgfqpoint{4.987321in}{3.036000in}}%
\pgfpathlineto{\pgfqpoint{4.988323in}{3.057818in}}%
\pgfpathlineto{\pgfqpoint{4.989325in}{3.116455in}}%
\pgfpathlineto{\pgfqpoint{4.992332in}{3.006000in}}%
\pgfpathlineto{\pgfqpoint{4.994337in}{3.056455in}}%
\pgfpathlineto{\pgfqpoint{4.997343in}{2.970545in}}%
\pgfpathlineto{\pgfqpoint{4.999348in}{2.980091in}}%
\pgfpathlineto{\pgfqpoint{5.001352in}{3.001909in}}%
\pgfpathlineto{\pgfqpoint{5.003357in}{2.943273in}}%
\pgfpathlineto{\pgfqpoint{5.005361in}{2.970545in}}%
\pgfpathlineto{\pgfqpoint{5.006363in}{3.010091in}}%
\pgfpathlineto{\pgfqpoint{5.008368in}{2.932364in}}%
\pgfpathlineto{\pgfqpoint{5.009370in}{2.939182in}}%
\pgfpathlineto{\pgfqpoint{5.010372in}{2.954182in}}%
\pgfpathlineto{\pgfqpoint{5.011375in}{2.986909in}}%
\pgfpathlineto{\pgfqpoint{5.012377in}{2.977364in}}%
\pgfpathlineto{\pgfqpoint{5.013379in}{2.952818in}}%
\pgfpathlineto{\pgfqpoint{5.014381in}{2.958273in}}%
\pgfpathlineto{\pgfqpoint{5.015384in}{2.954182in}}%
\pgfpathlineto{\pgfqpoint{5.017388in}{2.982818in}}%
\pgfpathlineto{\pgfqpoint{5.018390in}{2.971909in}}%
\pgfpathlineto{\pgfqpoint{5.020395in}{2.986909in}}%
\pgfpathlineto{\pgfqpoint{5.021397in}{2.997818in}}%
\pgfpathlineto{\pgfqpoint{5.022399in}{2.993727in}}%
\pgfpathlineto{\pgfqpoint{5.025406in}{3.022364in}}%
\pgfpathlineto{\pgfqpoint{5.027411in}{3.006000in}}%
\pgfpathlineto{\pgfqpoint{5.030417in}{3.064636in}}%
\pgfpathlineto{\pgfqpoint{5.031420in}{3.040091in}}%
\pgfpathlineto{\pgfqpoint{5.032422in}{2.988273in}}%
\pgfpathlineto{\pgfqpoint{5.034426in}{3.059182in}}%
\pgfpathlineto{\pgfqpoint{5.035428in}{3.066000in}}%
\pgfpathlineto{\pgfqpoint{5.037433in}{3.023727in}}%
\pgfpathlineto{\pgfqpoint{5.039437in}{3.079636in}}%
\pgfpathlineto{\pgfqpoint{5.041442in}{3.046909in}}%
\pgfpathlineto{\pgfqpoint{5.042444in}{3.006000in}}%
\pgfpathlineto{\pgfqpoint{5.043446in}{3.015545in}}%
\pgfpathlineto{\pgfqpoint{5.044449in}{3.042818in}}%
\pgfpathlineto{\pgfqpoint{5.045451in}{3.040091in}}%
\pgfpathlineto{\pgfqpoint{5.046453in}{3.026455in}}%
\pgfpathlineto{\pgfqpoint{5.047455in}{2.982818in}}%
\pgfpathlineto{\pgfqpoint{5.048458in}{2.993727in}}%
\pgfpathlineto{\pgfqpoint{5.049460in}{2.985545in}}%
\pgfpathlineto{\pgfqpoint{5.050462in}{3.022364in}}%
\pgfpathlineto{\pgfqpoint{5.051464in}{3.019636in}}%
\pgfpathlineto{\pgfqpoint{5.053469in}{2.955545in}}%
\pgfpathlineto{\pgfqpoint{5.054471in}{2.961000in}}%
\pgfpathlineto{\pgfqpoint{5.056476in}{3.006000in}}%
\pgfpathlineto{\pgfqpoint{5.058480in}{2.950091in}}%
\pgfpathlineto{\pgfqpoint{5.059482in}{2.955545in}}%
\pgfpathlineto{\pgfqpoint{5.061487in}{3.006000in}}%
\pgfpathlineto{\pgfqpoint{5.063491in}{2.944636in}}%
\pgfpathlineto{\pgfqpoint{5.064494in}{2.952818in}}%
\pgfpathlineto{\pgfqpoint{5.065496in}{2.963727in}}%
\pgfpathlineto{\pgfqpoint{5.066498in}{2.996455in}}%
\pgfpathlineto{\pgfqpoint{5.068502in}{2.973273in}}%
\pgfpathlineto{\pgfqpoint{5.069505in}{2.977364in}}%
\pgfpathlineto{\pgfqpoint{5.070507in}{3.000545in}}%
\pgfpathlineto{\pgfqpoint{5.071509in}{2.980091in}}%
\pgfpathlineto{\pgfqpoint{5.073514in}{3.012818in}}%
\pgfpathlineto{\pgfqpoint{5.074516in}{3.001909in}}%
\pgfpathlineto{\pgfqpoint{5.075518in}{3.036000in}}%
\pgfpathlineto{\pgfqpoint{5.077523in}{3.014182in}}%
\pgfpathlineto{\pgfqpoint{5.078525in}{3.037364in}}%
\pgfpathlineto{\pgfqpoint{5.079527in}{3.033273in}}%
\pgfpathlineto{\pgfqpoint{5.080529in}{3.055091in}}%
\pgfpathlineto{\pgfqpoint{5.082534in}{3.019636in}}%
\pgfpathlineto{\pgfqpoint{5.084538in}{3.067364in}}%
\pgfpathlineto{\pgfqpoint{5.085541in}{3.074182in}}%
\pgfpathlineto{\pgfqpoint{5.087545in}{3.015545in}}%
\pgfpathlineto{\pgfqpoint{5.088547in}{3.053727in}}%
\pgfpathlineto{\pgfqpoint{5.089550in}{3.046909in}}%
\pgfpathlineto{\pgfqpoint{5.090552in}{3.057818in}}%
\pgfpathlineto{\pgfqpoint{5.092556in}{3.011455in}}%
\pgfpathlineto{\pgfqpoint{5.095563in}{3.040091in}}%
\pgfpathlineto{\pgfqpoint{5.098570in}{2.976000in}}%
\pgfpathlineto{\pgfqpoint{5.100574in}{3.012818in}}%
\pgfpathlineto{\pgfqpoint{5.103581in}{2.959636in}}%
\pgfpathlineto{\pgfqpoint{5.104583in}{2.962364in}}%
\pgfpathlineto{\pgfqpoint{5.105585in}{2.992364in}}%
\pgfpathlineto{\pgfqpoint{5.106588in}{2.985545in}}%
\pgfpathlineto{\pgfqpoint{5.109594in}{2.941909in}}%
\pgfpathlineto{\pgfqpoint{5.111599in}{2.993727in}}%
\pgfpathlineto{\pgfqpoint{5.114606in}{2.952818in}}%
\pgfpathlineto{\pgfqpoint{5.116610in}{2.989636in}}%
\pgfpathlineto{\pgfqpoint{5.118615in}{2.978727in}}%
\pgfpathlineto{\pgfqpoint{5.119617in}{2.981455in}}%
\pgfpathlineto{\pgfqpoint{5.121621in}{2.999182in}}%
\pgfpathlineto{\pgfqpoint{5.122624in}{3.010091in}}%
\pgfpathlineto{\pgfqpoint{5.124628in}{3.003273in}}%
\pgfpathlineto{\pgfqpoint{5.125630in}{3.022364in}}%
\pgfpathlineto{\pgfqpoint{5.126633in}{3.012818in}}%
\pgfpathlineto{\pgfqpoint{5.128637in}{3.046909in}}%
\pgfpathlineto{\pgfqpoint{5.130642in}{3.038727in}}%
\pgfpathlineto{\pgfqpoint{5.131644in}{3.004636in}}%
\pgfpathlineto{\pgfqpoint{5.133648in}{3.056455in}}%
\pgfpathlineto{\pgfqpoint{5.134651in}{3.051000in}}%
\pgfpathlineto{\pgfqpoint{5.135653in}{3.053727in}}%
\pgfpathlineto{\pgfqpoint{5.136655in}{3.021000in}}%
\pgfpathlineto{\pgfqpoint{5.137657in}{3.025091in}}%
\pgfpathlineto{\pgfqpoint{5.138660in}{3.053727in}}%
\pgfpathlineto{\pgfqpoint{5.139662in}{3.051000in}}%
\pgfpathlineto{\pgfqpoint{5.140664in}{3.051000in}}%
\pgfpathlineto{\pgfqpoint{5.141666in}{3.006000in}}%
\pgfpathlineto{\pgfqpoint{5.142668in}{3.026455in}}%
\pgfpathlineto{\pgfqpoint{5.143671in}{3.010091in}}%
\pgfpathlineto{\pgfqpoint{5.144673in}{3.042818in}}%
\pgfpathlineto{\pgfqpoint{5.145675in}{3.031909in}}%
\pgfpathlineto{\pgfqpoint{5.147680in}{2.988273in}}%
\pgfpathlineto{\pgfqpoint{5.149684in}{3.007364in}}%
\pgfpathlineto{\pgfqpoint{5.150686in}{3.014182in}}%
\pgfpathlineto{\pgfqpoint{5.151689in}{2.985545in}}%
\pgfpathlineto{\pgfqpoint{5.152691in}{3.001909in}}%
\pgfpathlineto{\pgfqpoint{5.153693in}{2.973273in}}%
\pgfpathlineto{\pgfqpoint{5.154695in}{2.976000in}}%
\pgfpathlineto{\pgfqpoint{5.155698in}{2.993727in}}%
\pgfpathlineto{\pgfqpoint{5.157702in}{2.986909in}}%
\pgfpathlineto{\pgfqpoint{5.159707in}{2.958273in}}%
\pgfpathlineto{\pgfqpoint{5.161711in}{2.993727in}}%
\pgfpathlineto{\pgfqpoint{5.162713in}{2.995091in}}%
\pgfpathlineto{\pgfqpoint{5.164718in}{2.966455in}}%
\pgfpathlineto{\pgfqpoint{5.166722in}{2.988273in}}%
\pgfpathlineto{\pgfqpoint{5.167725in}{2.995091in}}%
\pgfpathlineto{\pgfqpoint{5.169729in}{2.981455in}}%
\pgfpathlineto{\pgfqpoint{5.170731in}{2.982818in}}%
\pgfpathlineto{\pgfqpoint{5.174740in}{3.029182in}}%
\pgfpathlineto{\pgfqpoint{5.175742in}{3.006000in}}%
\pgfpathlineto{\pgfqpoint{5.176745in}{3.007364in}}%
\pgfpathlineto{\pgfqpoint{5.177747in}{3.008727in}}%
\pgfpathlineto{\pgfqpoint{5.178749in}{3.014182in}}%
\pgfpathlineto{\pgfqpoint{5.179751in}{3.036000in}}%
\pgfpathlineto{\pgfqpoint{5.180754in}{3.027818in}}%
\pgfpathlineto{\pgfqpoint{5.181756in}{2.992364in}}%
\pgfpathlineto{\pgfqpoint{5.182758in}{3.026455in}}%
\pgfpathlineto{\pgfqpoint{5.183760in}{3.025091in}}%
\pgfpathlineto{\pgfqpoint{5.185765in}{3.059182in}}%
\pgfpathlineto{\pgfqpoint{5.186767in}{3.015545in}}%
\pgfpathlineto{\pgfqpoint{5.188772in}{3.057818in}}%
\pgfpathlineto{\pgfqpoint{5.189774in}{3.040091in}}%
\pgfpathlineto{\pgfqpoint{5.190776in}{3.057818in}}%
\pgfpathlineto{\pgfqpoint{5.192781in}{3.033273in}}%
\pgfpathlineto{\pgfqpoint{5.193783in}{3.029182in}}%
\pgfpathlineto{\pgfqpoint{5.194785in}{3.049636in}}%
\pgfpathlineto{\pgfqpoint{5.195787in}{3.045545in}}%
\pgfpathlineto{\pgfqpoint{5.196790in}{3.018273in}}%
\pgfpathlineto{\pgfqpoint{5.197792in}{3.027818in}}%
\pgfpathlineto{\pgfqpoint{5.198794in}{3.021000in}}%
\pgfpathlineto{\pgfqpoint{5.199796in}{3.038727in}}%
\pgfpathlineto{\pgfqpoint{5.201801in}{3.010091in}}%
\pgfpathlineto{\pgfqpoint{5.203805in}{2.997818in}}%
\pgfpathlineto{\pgfqpoint{5.204808in}{3.008727in}}%
\pgfpathlineto{\pgfqpoint{5.205810in}{3.007364in}}%
\pgfpathlineto{\pgfqpoint{5.206812in}{2.995091in}}%
\pgfpathlineto{\pgfqpoint{5.207814in}{3.011455in}}%
\pgfpathlineto{\pgfqpoint{5.208817in}{2.986909in}}%
\pgfpathlineto{\pgfqpoint{5.210821in}{2.997818in}}%
\pgfpathlineto{\pgfqpoint{5.211823in}{2.986909in}}%
\pgfpathlineto{\pgfqpoint{5.212825in}{2.997818in}}%
\pgfpathlineto{\pgfqpoint{5.213828in}{2.970545in}}%
\pgfpathlineto{\pgfqpoint{5.215832in}{2.984182in}}%
\pgfpathlineto{\pgfqpoint{5.216834in}{2.981455in}}%
\pgfpathlineto{\pgfqpoint{5.217837in}{2.995091in}}%
\pgfpathlineto{\pgfqpoint{5.218839in}{2.986909in}}%
\pgfpathlineto{\pgfqpoint{5.219841in}{2.988273in}}%
\pgfpathlineto{\pgfqpoint{5.222848in}{2.996455in}}%
\pgfpathlineto{\pgfqpoint{5.223850in}{2.999182in}}%
\pgfpathlineto{\pgfqpoint{5.224852in}{3.015545in}}%
\pgfpathlineto{\pgfqpoint{5.225855in}{2.999182in}}%
\pgfpathlineto{\pgfqpoint{5.227859in}{3.021000in}}%
\pgfpathlineto{\pgfqpoint{5.228861in}{3.018273in}}%
\pgfpathlineto{\pgfqpoint{5.229864in}{3.027818in}}%
\pgfpathlineto{\pgfqpoint{5.230866in}{3.012818in}}%
\pgfpathlineto{\pgfqpoint{5.233873in}{3.037364in}}%
\pgfpathlineto{\pgfqpoint{5.234875in}{3.072818in}}%
\pgfpathlineto{\pgfqpoint{5.235877in}{3.021000in}}%
\pgfpathlineto{\pgfqpoint{5.239886in}{3.052364in}}%
\pgfpathlineto{\pgfqpoint{5.241891in}{3.027818in}}%
\pgfpathlineto{\pgfqpoint{5.243895in}{3.037364in}}%
\pgfpathlineto{\pgfqpoint{5.244897in}{3.038727in}}%
\pgfpathlineto{\pgfqpoint{5.246902in}{3.018273in}}%
\pgfpathlineto{\pgfqpoint{5.247904in}{3.008727in}}%
\pgfpathlineto{\pgfqpoint{5.248906in}{3.012818in}}%
\pgfpathlineto{\pgfqpoint{5.249908in}{3.007364in}}%
\pgfpathlineto{\pgfqpoint{5.250911in}{3.012818in}}%
\pgfpathlineto{\pgfqpoint{5.251913in}{3.011455in}}%
\pgfpathlineto{\pgfqpoint{5.253917in}{2.989636in}}%
\pgfpathlineto{\pgfqpoint{5.254920in}{3.003273in}}%
\pgfpathlineto{\pgfqpoint{5.255922in}{2.996455in}}%
\pgfpathlineto{\pgfqpoint{5.256924in}{3.012818in}}%
\pgfpathlineto{\pgfqpoint{5.258929in}{2.976000in}}%
\pgfpathlineto{\pgfqpoint{5.259931in}{2.993727in}}%
\pgfpathlineto{\pgfqpoint{5.262938in}{2.981455in}}%
\pgfpathlineto{\pgfqpoint{5.263940in}{2.982818in}}%
\pgfpathlineto{\pgfqpoint{5.264942in}{2.981455in}}%
\pgfpathlineto{\pgfqpoint{5.265944in}{2.973273in}}%
\pgfpathlineto{\pgfqpoint{5.267949in}{2.995091in}}%
\pgfpathlineto{\pgfqpoint{5.268951in}{2.989636in}}%
\pgfpathlineto{\pgfqpoint{5.269953in}{3.003273in}}%
\pgfpathlineto{\pgfqpoint{5.270956in}{2.992364in}}%
\pgfpathlineto{\pgfqpoint{5.272960in}{3.006000in}}%
\pgfpathlineto{\pgfqpoint{5.273962in}{3.007364in}}%
\pgfpathlineto{\pgfqpoint{5.274965in}{3.019636in}}%
\pgfpathlineto{\pgfqpoint{5.275967in}{3.010091in}}%
\pgfpathlineto{\pgfqpoint{5.276969in}{3.025091in}}%
\pgfpathlineto{\pgfqpoint{5.277971in}{3.010091in}}%
\pgfpathlineto{\pgfqpoint{5.279976in}{3.034636in}}%
\pgfpathlineto{\pgfqpoint{5.280978in}{3.022364in}}%
\pgfpathlineto{\pgfqpoint{5.281980in}{3.026455in}}%
\pgfpathlineto{\pgfqpoint{5.283985in}{3.044182in}}%
\pgfpathlineto{\pgfqpoint{5.284987in}{3.040091in}}%
\pgfpathlineto{\pgfqpoint{5.285989in}{3.014182in}}%
\pgfpathlineto{\pgfqpoint{5.288996in}{3.036000in}}%
\pgfpathlineto{\pgfqpoint{5.289998in}{3.031909in}}%
\pgfpathlineto{\pgfqpoint{5.291000in}{3.008727in}}%
\pgfpathlineto{\pgfqpoint{5.292003in}{3.012818in}}%
\pgfpathlineto{\pgfqpoint{5.294007in}{3.036000in}}%
\pgfpathlineto{\pgfqpoint{5.297014in}{2.993727in}}%
\pgfpathlineto{\pgfqpoint{5.299018in}{3.019636in}}%
\pgfpathlineto{\pgfqpoint{5.300021in}{3.004636in}}%
\pgfpathlineto{\pgfqpoint{5.301023in}{3.012818in}}%
\pgfpathlineto{\pgfqpoint{5.303027in}{2.991000in}}%
\pgfpathlineto{\pgfqpoint{5.305032in}{3.003273in}}%
\pgfpathlineto{\pgfqpoint{5.306034in}{3.008727in}}%
\pgfpathlineto{\pgfqpoint{5.309041in}{2.982818in}}%
\pgfpathlineto{\pgfqpoint{5.310043in}{2.977364in}}%
\pgfpathlineto{\pgfqpoint{5.312048in}{3.001909in}}%
\pgfpathlineto{\pgfqpoint{5.314052in}{2.974636in}}%
\pgfpathlineto{\pgfqpoint{5.315054in}{2.974636in}}%
\pgfpathlineto{\pgfqpoint{5.317059in}{2.999182in}}%
\pgfpathlineto{\pgfqpoint{5.319063in}{2.986909in}}%
\pgfpathlineto{\pgfqpoint{5.322070in}{3.001909in}}%
\pgfpathlineto{\pgfqpoint{5.323072in}{2.991000in}}%
\pgfpathlineto{\pgfqpoint{5.325077in}{3.034636in}}%
\pgfpathlineto{\pgfqpoint{5.327081in}{2.986909in}}%
\pgfpathlineto{\pgfqpoint{5.328083in}{2.986909in}}%
\pgfpathlineto{\pgfqpoint{5.329086in}{3.036000in}}%
\pgfpathlineto{\pgfqpoint{5.331090in}{3.012818in}}%
\pgfpathlineto{\pgfqpoint{5.332092in}{2.992364in}}%
\pgfpathlineto{\pgfqpoint{5.335099in}{3.041455in}}%
\pgfpathlineto{\pgfqpoint{5.338106in}{3.007364in}}%
\pgfpathlineto{\pgfqpoint{5.341113in}{3.033273in}}%
\pgfpathlineto{\pgfqpoint{5.342115in}{3.004636in}}%
\pgfpathlineto{\pgfqpoint{5.345122in}{3.027818in}}%
\pgfpathlineto{\pgfqpoint{5.347126in}{3.012818in}}%
\pgfpathlineto{\pgfqpoint{5.348128in}{3.015545in}}%
\pgfpathlineto{\pgfqpoint{5.349131in}{3.012818in}}%
\pgfpathlineto{\pgfqpoint{5.350133in}{3.006000in}}%
\pgfpathlineto{\pgfqpoint{5.351135in}{3.010091in}}%
\pgfpathlineto{\pgfqpoint{5.354142in}{2.980091in}}%
\pgfpathlineto{\pgfqpoint{5.356146in}{3.011455in}}%
\pgfpathlineto{\pgfqpoint{5.357148in}{2.991000in}}%
\pgfpathlineto{\pgfqpoint{5.361157in}{3.016909in}}%
\pgfpathlineto{\pgfqpoint{5.362160in}{2.978727in}}%
\pgfpathlineto{\pgfqpoint{5.363162in}{3.022364in}}%
\pgfpathlineto{\pgfqpoint{5.366169in}{2.981455in}}%
\pgfpathlineto{\pgfqpoint{5.367171in}{3.014182in}}%
\pgfpathlineto{\pgfqpoint{5.368173in}{3.011455in}}%
\pgfpathlineto{\pgfqpoint{5.370178in}{2.982818in}}%
\pgfpathlineto{\pgfqpoint{5.371180in}{3.011455in}}%
\pgfpathlineto{\pgfqpoint{5.372182in}{3.000545in}}%
\pgfpathlineto{\pgfqpoint{5.373184in}{3.001909in}}%
\pgfpathlineto{\pgfqpoint{5.375189in}{3.027818in}}%
\pgfpathlineto{\pgfqpoint{5.377193in}{3.001909in}}%
\pgfpathlineto{\pgfqpoint{5.378196in}{3.004636in}}%
\pgfpathlineto{\pgfqpoint{5.379198in}{3.029182in}}%
\pgfpathlineto{\pgfqpoint{5.380200in}{3.027818in}}%
\pgfpathlineto{\pgfqpoint{5.382205in}{2.992364in}}%
\pgfpathlineto{\pgfqpoint{5.384209in}{3.029182in}}%
\pgfpathlineto{\pgfqpoint{5.385211in}{3.034636in}}%
\pgfpathlineto{\pgfqpoint{5.387216in}{3.011455in}}%
\pgfpathlineto{\pgfqpoint{5.388218in}{3.015545in}}%
\pgfpathlineto{\pgfqpoint{5.389220in}{3.034636in}}%
\pgfpathlineto{\pgfqpoint{5.390222in}{3.021000in}}%
\pgfpathlineto{\pgfqpoint{5.391225in}{3.023727in}}%
\pgfpathlineto{\pgfqpoint{5.392227in}{3.006000in}}%
\pgfpathlineto{\pgfqpoint{5.394231in}{3.023727in}}%
\pgfpathlineto{\pgfqpoint{5.395234in}{3.014182in}}%
\pgfpathlineto{\pgfqpoint{5.396236in}{3.016909in}}%
\pgfpathlineto{\pgfqpoint{5.397238in}{3.010091in}}%
\pgfpathlineto{\pgfqpoint{5.398240in}{3.019636in}}%
\pgfpathlineto{\pgfqpoint{5.399243in}{3.012818in}}%
\pgfpathlineto{\pgfqpoint{5.400245in}{3.018273in}}%
\pgfpathlineto{\pgfqpoint{5.401247in}{3.014182in}}%
\pgfpathlineto{\pgfqpoint{5.402249in}{2.999182in}}%
\pgfpathlineto{\pgfqpoint{5.403252in}{3.021000in}}%
\pgfpathlineto{\pgfqpoint{5.404254in}{2.995091in}}%
\pgfpathlineto{\pgfqpoint{5.405256in}{3.007364in}}%
\pgfpathlineto{\pgfqpoint{5.407261in}{3.003273in}}%
\pgfpathlineto{\pgfqpoint{5.408263in}{2.986909in}}%
\pgfpathlineto{\pgfqpoint{5.409265in}{2.993727in}}%
\pgfpathlineto{\pgfqpoint{5.410267in}{2.992364in}}%
\pgfpathlineto{\pgfqpoint{5.411270in}{2.995091in}}%
\pgfpathlineto{\pgfqpoint{5.413274in}{3.012818in}}%
\pgfpathlineto{\pgfqpoint{5.416281in}{2.970545in}}%
\pgfpathlineto{\pgfqpoint{5.417283in}{2.982818in}}%
\pgfpathlineto{\pgfqpoint{5.418285in}{3.018273in}}%
\pgfpathlineto{\pgfqpoint{5.420290in}{2.996455in}}%
\pgfpathlineto{\pgfqpoint{5.421292in}{3.001909in}}%
\pgfpathlineto{\pgfqpoint{5.422294in}{2.986909in}}%
\pgfpathlineto{\pgfqpoint{5.425301in}{3.025091in}}%
\pgfpathlineto{\pgfqpoint{5.426303in}{2.997818in}}%
\pgfpathlineto{\pgfqpoint{5.427305in}{3.001909in}}%
\pgfpathlineto{\pgfqpoint{5.428308in}{3.000545in}}%
\pgfpathlineto{\pgfqpoint{5.429310in}{3.022364in}}%
\pgfpathlineto{\pgfqpoint{5.431314in}{2.996455in}}%
\pgfpathlineto{\pgfqpoint{5.432317in}{2.997818in}}%
\pgfpathlineto{\pgfqpoint{5.433319in}{3.041455in}}%
\pgfpathlineto{\pgfqpoint{5.435323in}{3.004636in}}%
\pgfpathlineto{\pgfqpoint{5.436326in}{3.021000in}}%
\pgfpathlineto{\pgfqpoint{5.437328in}{3.012818in}}%
\pgfpathlineto{\pgfqpoint{5.438330in}{3.025091in}}%
\pgfpathlineto{\pgfqpoint{5.439332in}{3.023727in}}%
\pgfpathlineto{\pgfqpoint{5.440335in}{3.004636in}}%
\pgfpathlineto{\pgfqpoint{5.442339in}{3.023727in}}%
\pgfpathlineto{\pgfqpoint{5.443341in}{3.007364in}}%
\pgfpathlineto{\pgfqpoint{5.444344in}{3.029182in}}%
\pgfpathlineto{\pgfqpoint{5.445346in}{3.025091in}}%
\pgfpathlineto{\pgfqpoint{5.447350in}{3.004636in}}%
\pgfpathlineto{\pgfqpoint{5.449355in}{3.027818in}}%
\pgfpathlineto{\pgfqpoint{5.451359in}{2.999182in}}%
\pgfpathlineto{\pgfqpoint{5.454366in}{3.025091in}}%
\pgfpathlineto{\pgfqpoint{5.455368in}{3.026455in}}%
\pgfpathlineto{\pgfqpoint{5.456371in}{3.010091in}}%
\pgfpathlineto{\pgfqpoint{5.457373in}{3.012818in}}%
\pgfpathlineto{\pgfqpoint{5.458375in}{3.010091in}}%
\pgfpathlineto{\pgfqpoint{5.459377in}{2.997818in}}%
\pgfpathlineto{\pgfqpoint{5.461382in}{3.016909in}}%
\pgfpathlineto{\pgfqpoint{5.463386in}{2.985545in}}%
\pgfpathlineto{\pgfqpoint{5.464388in}{2.988273in}}%
\pgfpathlineto{\pgfqpoint{5.466393in}{3.006000in}}%
\pgfpathlineto{\pgfqpoint{5.467395in}{2.992364in}}%
\pgfpathlineto{\pgfqpoint{5.468397in}{2.993727in}}%
\pgfpathlineto{\pgfqpoint{5.469400in}{2.999182in}}%
\pgfpathlineto{\pgfqpoint{5.470402in}{2.996455in}}%
\pgfpathlineto{\pgfqpoint{5.471404in}{3.003273in}}%
\pgfpathlineto{\pgfqpoint{5.472406in}{2.999182in}}%
\pgfpathlineto{\pgfqpoint{5.473409in}{3.000545in}}%
\pgfpathlineto{\pgfqpoint{5.476415in}{2.985545in}}%
\pgfpathlineto{\pgfqpoint{5.477418in}{3.008727in}}%
\pgfpathlineto{\pgfqpoint{5.479422in}{3.001909in}}%
\pgfpathlineto{\pgfqpoint{5.481427in}{3.014182in}}%
\pgfpathlineto{\pgfqpoint{5.482429in}{2.993727in}}%
\pgfpathlineto{\pgfqpoint{5.485436in}{3.015545in}}%
\pgfpathlineto{\pgfqpoint{5.486438in}{2.991000in}}%
\pgfpathlineto{\pgfqpoint{5.488442in}{3.003273in}}%
\pgfpathlineto{\pgfqpoint{5.489445in}{3.000545in}}%
\pgfpathlineto{\pgfqpoint{5.490447in}{3.042818in}}%
\pgfpathlineto{\pgfqpoint{5.492451in}{3.007364in}}%
\pgfpathlineto{\pgfqpoint{5.493453in}{3.010091in}}%
\pgfpathlineto{\pgfqpoint{5.494456in}{3.008727in}}%
\pgfpathlineto{\pgfqpoint{5.495458in}{3.021000in}}%
\pgfpathlineto{\pgfqpoint{5.496460in}{3.016909in}}%
\pgfpathlineto{\pgfqpoint{5.498465in}{3.006000in}}%
\pgfpathlineto{\pgfqpoint{5.499467in}{3.018273in}}%
\pgfpathlineto{\pgfqpoint{5.500469in}{3.007364in}}%
\pgfpathlineto{\pgfqpoint{5.501471in}{3.021000in}}%
\pgfpathlineto{\pgfqpoint{5.502474in}{3.014182in}}%
\pgfpathlineto{\pgfqpoint{5.504478in}{2.988273in}}%
\pgfpathlineto{\pgfqpoint{5.506483in}{3.010091in}}%
\pgfpathlineto{\pgfqpoint{5.507485in}{3.010091in}}%
\pgfpathlineto{\pgfqpoint{5.508487in}{2.989636in}}%
\pgfpathlineto{\pgfqpoint{5.510492in}{3.006000in}}%
\pgfpathlineto{\pgfqpoint{5.512496in}{2.989636in}}%
\pgfpathlineto{\pgfqpoint{5.514501in}{3.001909in}}%
\pgfpathlineto{\pgfqpoint{5.515503in}{3.001909in}}%
\pgfpathlineto{\pgfqpoint{5.516505in}{2.981455in}}%
\pgfpathlineto{\pgfqpoint{5.517507in}{3.004636in}}%
\pgfpathlineto{\pgfqpoint{5.518510in}{2.997818in}}%
\pgfpathlineto{\pgfqpoint{5.519512in}{2.999182in}}%
\pgfpathlineto{\pgfqpoint{5.522519in}{3.010091in}}%
\pgfpathlineto{\pgfqpoint{5.523521in}{3.014182in}}%
\pgfpathlineto{\pgfqpoint{5.524523in}{3.006000in}}%
\pgfpathlineto{\pgfqpoint{5.525525in}{3.016909in}}%
\pgfpathlineto{\pgfqpoint{5.526528in}{3.000545in}}%
\pgfpathlineto{\pgfqpoint{5.527530in}{3.007364in}}%
\pgfpathlineto{\pgfqpoint{5.528532in}{2.999182in}}%
\pgfpathlineto{\pgfqpoint{5.530536in}{3.008727in}}%
\pgfpathlineto{\pgfqpoint{5.532541in}{3.023727in}}%
\pgfpathlineto{\pgfqpoint{5.533543in}{3.010091in}}%
\pgfpathlineto{\pgfqpoint{5.534545in}{3.016909in}}%
\pgfpathlineto{\pgfqpoint{5.534545in}{3.016909in}}%
\pgfusepath{stroke}%
\end{pgfscope}%
\begin{pgfscope}%
\pgfpathrectangle{\pgfqpoint{0.800000in}{0.528000in}}{\pgfqpoint{4.960000in}{3.696000in}}%
\pgfusepath{clip}%
\pgfsetrectcap%
\pgfsetroundjoin%
\pgfsetlinewidth{1.505625pt}%
\definecolor{currentstroke}{rgb}{0.843137,0.509804,0.494118}%
\pgfsetstrokecolor{currentstroke}%
\pgfsetdash{}{0pt}%
\pgfpathmoveto{\pgfqpoint{1.025455in}{3.155377in}}%
\pgfpathlineto{\pgfqpoint{1.078573in}{3.144237in}}%
\pgfpathlineto{\pgfqpoint{1.131692in}{3.130851in}}%
\pgfpathlineto{\pgfqpoint{1.185813in}{3.114894in}}%
\pgfpathlineto{\pgfqpoint{1.239935in}{3.096602in}}%
\pgfpathlineto{\pgfqpoint{1.295058in}{3.075593in}}%
\pgfpathlineto{\pgfqpoint{1.351183in}{3.051773in}}%
\pgfpathlineto{\pgfqpoint{1.408311in}{3.025065in}}%
\pgfpathlineto{\pgfqpoint{1.466441in}{2.995418in}}%
\pgfpathlineto{\pgfqpoint{1.526576in}{2.962227in}}%
\pgfpathlineto{\pgfqpoint{1.562657in}{2.942685in}}%
\pgfpathlineto{\pgfqpoint{1.626800in}{2.982252in}}%
\pgfpathlineto{\pgfqpoint{1.693951in}{3.026327in}}%
\pgfpathlineto{\pgfqpoint{1.765110in}{3.075717in}}%
\pgfpathlineto{\pgfqpoint{1.842283in}{3.132021in}}%
\pgfpathlineto{\pgfqpoint{1.928475in}{3.197719in}}%
\pgfpathlineto{\pgfqpoint{2.030704in}{3.278541in}}%
\pgfpathlineto{\pgfqpoint{2.183045in}{3.402193in}}%
\pgfpathlineto{\pgfqpoint{2.365453in}{3.549656in}}%
\pgfpathlineto{\pgfqpoint{2.464675in}{3.626959in}}%
\pgfpathlineto{\pgfqpoint{2.547862in}{3.688991in}}%
\pgfpathlineto{\pgfqpoint{2.621025in}{3.740843in}}%
\pgfpathlineto{\pgfqpoint{2.688176in}{3.785781in}}%
\pgfpathlineto{\pgfqpoint{2.751317in}{3.825401in}}%
\pgfpathlineto{\pgfqpoint{2.810449in}{3.859941in}}%
\pgfpathlineto{\pgfqpoint{2.867577in}{3.890754in}}%
\pgfpathlineto{\pgfqpoint{2.921698in}{3.917469in}}%
\pgfpathlineto{\pgfqpoint{2.974817in}{3.941210in}}%
\pgfpathlineto{\pgfqpoint{3.025931in}{3.961627in}}%
\pgfpathlineto{\pgfqpoint{3.076044in}{3.979240in}}%
\pgfpathlineto{\pgfqpoint{3.125153in}{3.994117in}}%
\pgfpathlineto{\pgfqpoint{3.173261in}{4.006341in}}%
\pgfpathlineto{\pgfqpoint{3.220367in}{4.016008in}}%
\pgfpathlineto{\pgfqpoint{3.266470in}{4.023227in}}%
\pgfpathlineto{\pgfqpoint{3.312573in}{4.028201in}}%
\pgfpathlineto{\pgfqpoint{3.358676in}{4.030911in}}%
\pgfpathlineto{\pgfqpoint{3.403777in}{4.031361in}}%
\pgfpathlineto{\pgfqpoint{3.448878in}{4.029633in}}%
\pgfpathlineto{\pgfqpoint{3.494981in}{4.025622in}}%
\pgfpathlineto{\pgfqpoint{3.541084in}{4.019357in}}%
\pgfpathlineto{\pgfqpoint{3.587187in}{4.010860in}}%
\pgfpathlineto{\pgfqpoint{3.634293in}{3.999909in}}%
\pgfpathlineto{\pgfqpoint{3.682401in}{3.986399in}}%
\pgfpathlineto{\pgfqpoint{3.731510in}{3.970242in}}%
\pgfpathlineto{\pgfqpoint{3.781623in}{3.951362in}}%
\pgfpathlineto{\pgfqpoint{3.832737in}{3.929699in}}%
\pgfpathlineto{\pgfqpoint{3.885856in}{3.904719in}}%
\pgfpathlineto{\pgfqpoint{3.939977in}{3.876807in}}%
\pgfpathlineto{\pgfqpoint{3.997105in}{3.844806in}}%
\pgfpathlineto{\pgfqpoint{4.056237in}{3.809128in}}%
\pgfpathlineto{\pgfqpoint{4.119378in}{3.768401in}}%
\pgfpathlineto{\pgfqpoint{4.186529in}{3.722416in}}%
\pgfpathlineto{\pgfqpoint{4.259692in}{3.669585in}}%
\pgfpathlineto{\pgfqpoint{4.341876in}{3.607426in}}%
\pgfpathlineto{\pgfqpoint{4.439094in}{3.530985in}}%
\pgfpathlineto{\pgfqpoint{4.575399in}{3.420677in}}%
\pgfpathlineto{\pgfqpoint{4.794890in}{3.243044in}}%
\pgfpathlineto{\pgfqpoint{4.895114in}{3.164998in}}%
\pgfpathlineto{\pgfqpoint{4.980305in}{3.101420in}}%
\pgfpathlineto{\pgfqpoint{5.056476in}{3.047292in}}%
\pgfpathlineto{\pgfqpoint{5.127635in}{2.999439in}}%
\pgfpathlineto{\pgfqpoint{5.193783in}{2.957587in}}%
\pgfpathlineto{\pgfqpoint{5.219841in}{2.941970in}}%
\pgfpathlineto{\pgfqpoint{5.281980in}{2.977694in}}%
\pgfpathlineto{\pgfqpoint{5.342115in}{3.009729in}}%
\pgfpathlineto{\pgfqpoint{5.400245in}{3.038209in}}%
\pgfpathlineto{\pgfqpoint{5.457373in}{3.063731in}}%
\pgfpathlineto{\pgfqpoint{5.513498in}{3.086358in}}%
\pgfpathlineto{\pgfqpoint{5.534545in}{3.094207in}}%
\pgfpathlineto{\pgfqpoint{5.534545in}{3.094207in}}%
\pgfusepath{stroke}%
\end{pgfscope}%
\begin{pgfscope}%
\pgfsetbuttcap%
\pgfsetmiterjoin%
\definecolor{currentfill}{rgb}{0.000000,0.000000,0.000000}%
\pgfsetfillcolor{currentfill}%
\pgfsetlinewidth{1.003750pt}%
\definecolor{currentstroke}{rgb}{0.000000,0.000000,0.000000}%
\pgfsetstrokecolor{currentstroke}%
\pgfsetdash{}{0pt}%
\pgfsys@defobject{currentmarker}{\pgfqpoint{-0.069444in}{-0.069444in}}{\pgfqpoint{0.069444in}{0.069444in}}{%
\pgfpathmoveto{\pgfqpoint{0.069444in}{-0.000000in}}%
\pgfpathlineto{\pgfqpoint{-0.069444in}{0.069444in}}%
\pgfpathlineto{\pgfqpoint{-0.069444in}{-0.069444in}}%
\pgfpathlineto{\pgfqpoint{0.069444in}{-0.000000in}}%
\pgfpathclose%
\pgfusepath{stroke,fill}%
}%
\begin{pgfscope}%
\pgfsys@transformshift{5.760000in}{0.696000in}%
\pgfsys@useobject{currentmarker}{}%
\end{pgfscope}%
\end{pgfscope}%
\begin{pgfscope}%
\pgfsetbuttcap%
\pgfsetmiterjoin%
\definecolor{currentfill}{rgb}{0.000000,0.000000,0.000000}%
\pgfsetfillcolor{currentfill}%
\pgfsetlinewidth{1.003750pt}%
\definecolor{currentstroke}{rgb}{0.000000,0.000000,0.000000}%
\pgfsetstrokecolor{currentstroke}%
\pgfsetdash{}{0pt}%
\pgfsys@defobject{currentmarker}{\pgfqpoint{-0.069444in}{-0.069444in}}{\pgfqpoint{0.069444in}{0.069444in}}{%
\pgfpathmoveto{\pgfqpoint{0.000000in}{0.069444in}}%
\pgfpathlineto{\pgfqpoint{-0.069444in}{-0.069444in}}%
\pgfpathlineto{\pgfqpoint{0.069444in}{-0.069444in}}%
\pgfpathlineto{\pgfqpoint{0.000000in}{0.069444in}}%
\pgfpathclose%
\pgfusepath{stroke,fill}%
}%
\begin{pgfscope}%
\pgfsys@transformshift{1.025455in}{4.224000in}%
\pgfsys@useobject{currentmarker}{}%
\end{pgfscope}%
\end{pgfscope}%
\begin{pgfscope}%
\pgfsetrectcap%
\pgfsetmiterjoin%
\pgfsetlinewidth{0.803000pt}%
\definecolor{currentstroke}{rgb}{0.000000,0.000000,0.000000}%
\pgfsetstrokecolor{currentstroke}%
\pgfsetdash{}{0pt}%
\pgfpathmoveto{\pgfqpoint{1.025455in}{0.528000in}}%
\pgfpathlineto{\pgfqpoint{1.025455in}{4.224000in}}%
\pgfusepath{stroke}%
\end{pgfscope}%
\begin{pgfscope}%
\pgfsetrectcap%
\pgfsetmiterjoin%
\pgfsetlinewidth{0.803000pt}%
\definecolor{currentstroke}{rgb}{0.000000,0.000000,0.000000}%
\pgfsetstrokecolor{currentstroke}%
\pgfsetdash{}{0pt}%
\pgfpathmoveto{\pgfqpoint{0.800000in}{0.696000in}}%
\pgfpathlineto{\pgfqpoint{5.760000in}{0.696000in}}%
\pgfusepath{stroke}%
\end{pgfscope}%
\begin{pgfscope}%
\pgfsetbuttcap%
\pgfsetmiterjoin%
\definecolor{currentfill}{rgb}{1.000000,1.000000,1.000000}%
\pgfsetfillcolor{currentfill}%
\pgfsetfillopacity{0.800000}%
\pgfsetlinewidth{1.003750pt}%
\definecolor{currentstroke}{rgb}{0.800000,0.800000,0.800000}%
\pgfsetstrokecolor{currentstroke}%
\pgfsetstrokeopacity{0.800000}%
\pgfsetdash{}{0pt}%
\pgfpathmoveto{\pgfqpoint{4.781218in}{3.696222in}}%
\pgfpathlineto{\pgfqpoint{5.662778in}{3.696222in}}%
\pgfpathquadraticcurveto{\pgfqpoint{5.690556in}{3.696222in}}{\pgfqpoint{5.690556in}{3.724000in}}%
\pgfpathlineto{\pgfqpoint{5.690556in}{4.126778in}}%
\pgfpathquadraticcurveto{\pgfqpoint{5.690556in}{4.154556in}}{\pgfqpoint{5.662778in}{4.154556in}}%
\pgfpathlineto{\pgfqpoint{4.781218in}{4.154556in}}%
\pgfpathquadraticcurveto{\pgfqpoint{4.753440in}{4.154556in}}{\pgfqpoint{4.753440in}{4.126778in}}%
\pgfpathlineto{\pgfqpoint{4.753440in}{3.724000in}}%
\pgfpathquadraticcurveto{\pgfqpoint{4.753440in}{3.696222in}}{\pgfqpoint{4.781218in}{3.696222in}}%
\pgfpathlineto{\pgfqpoint{4.781218in}{3.696222in}}%
\pgfpathclose%
\pgfusepath{stroke,fill}%
\end{pgfscope}%
\begin{pgfscope}%
\pgfsetrectcap%
\pgfsetroundjoin%
\pgfsetlinewidth{1.505625pt}%
\definecolor{currentstroke}{rgb}{0.564706,0.478431,0.662745}%
\pgfsetstrokecolor{currentstroke}%
\pgfsetdash{}{0pt}%
\pgfpathmoveto{\pgfqpoint{4.808996in}{4.043444in}}%
\pgfpathlineto{\pgfqpoint{4.947884in}{4.043444in}}%
\pgfpathlineto{\pgfqpoint{5.086773in}{4.043444in}}%
\pgfusepath{stroke}%
\end{pgfscope}%
\begin{pgfscope}%
\definecolor{textcolor}{rgb}{0.000000,0.000000,0.000000}%
\pgfsetstrokecolor{textcolor}%
\pgfsetfillcolor{textcolor}%
\pgftext[x=5.197884in,y=3.994833in,left,base]{\color{textcolor}\rmfamily\fontsize{10.000000}{12.000000}\selectfont \(\displaystyle I_{\mathrm{mes}}(t)\)}%
\end{pgfscope}%
\begin{pgfscope}%
\pgfsetrectcap%
\pgfsetroundjoin%
\pgfsetlinewidth{1.505625pt}%
\definecolor{currentstroke}{rgb}{0.843137,0.509804,0.494118}%
\pgfsetstrokecolor{currentstroke}%
\pgfsetdash{}{0pt}%
\pgfpathmoveto{\pgfqpoint{4.808996in}{3.835111in}}%
\pgfpathlineto{\pgfqpoint{4.947884in}{3.835111in}}%
\pgfpathlineto{\pgfqpoint{5.086773in}{3.835111in}}%
\pgfusepath{stroke}%
\end{pgfscope}%
\begin{pgfscope}%
\definecolor{textcolor}{rgb}{0.000000,0.000000,0.000000}%
\pgfsetstrokecolor{textcolor}%
\pgfsetfillcolor{textcolor}%
\pgftext[x=5.197884in,y=3.786500in,left,base]{\color{textcolor}\rmfamily\fontsize{10.000000}{12.000000}\selectfont \(\displaystyle I_{\mathrm{mod}}(t)\)}%
\end{pgfscope}%
\end{pgfpicture}%
\makeatother%
\endgroup%
}
		\caption{Modélisation par un filtre à profil spectral rectangulaire}
	\end{figure}

	On peut en déduire que la modélisation par un filtre gaussien est plus en accord avec le modèle : la modélisation utilisant le sinus cardinal possède des \guillemotleft~rebonds~\guillemotright\ au delà de la zone d'interférences, mais les données expérimentales ne possèdent pas ces \guillemotleft~rebonds.~\guillemotright\@ Malgré tout, le filtre interférentiel choisi semble être plus complexe que les deux modèles étudiés durant ce \textsc{tp}.

	\sign

	~
\end{document}
