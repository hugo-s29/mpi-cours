\documentclass[a4paper,twocolumn,10pt]{report}

%\usepackage{concmath}
\usepackage{tgschola}
\usepackage[margin=1in]{geometry}
\usepackage[utf8]{inputenc}
\usepackage[T1]{fontenc}
\usepackage{mathrsfs}
\usepackage{textcomp}
\usepackage[french]{babel}
\usepackage{amsmath}
\usepackage{amssymb}
\usepackage{cancel}
\usepackage{frcursive}
\usepackage[inline]{asymptote}
\usepackage{tikz}
\usepackage[european,straightvoltages,europeanresistors]{circuitikz}
\usepackage{tikz-cd}
\usepackage{tkz-tab}
\usepackage[b]{esvect}
\usepackage[framemethod=TikZ]{mdframed}
\usepackage{centernot}
\usepackage{diagbox}
\usepackage{dsfont}
\usepackage{fancyhdr}
\usepackage{float}
\usepackage{graphicx}
\usepackage{listings}
\usepackage{multicol}
\usepackage{nicematrix}
\usepackage{pdflscape}
\usepackage{stmaryrd}
\usepackage{xfrac}
\usepackage{hep-math-font}
\usepackage{amsthm}
\usepackage{thmtools}
\usepackage{indentfirst}
\usepackage[framemethod=TikZ]{mdframed}
\usepackage{accents}
\usepackage{soulutf8}
\usepackage{mathtools}
\usepackage{bodegraph}
\usepackage{slashbox}
\usepackage{enumitem}
\usepackage{calligra}
\usepackage{cinzel}
\usepackage{BOONDOX-calo}

% Tikz
\usetikzlibrary{babel}
\usetikzlibrary{positioning}
\usetikzlibrary{calc}

% global settings
\frenchspacing
\reversemarginpar
\setuldepth{a}

%\everymath{\displaystyle}

\frenchbsetup{StandardLists=true}

\def\asydir{asy}

%\sisetup{exponent-product=\cdot,output-decimal-marker={,},separate-uncertainty,range-phrase=\;à\;,locale=FR}

\setlength{\parskip}{1em}

\theoremstyle{definition}

% Changing math
\let\emptyset\varnothing
\let\ge\geqslant
\let\le\leqslant
\let\preceq\preccurlyeq
\let\succeq\succcurlyeq
\let\ds\displaystyle
\let\ts\textstyle

\newcommand{\C}{\mathds{C}}
\newcommand{\R}{\mathds{R}}
\newcommand{\Z}{\mathds{Z}}
\newcommand{\N}{\mathds{N}}
\newcommand{\Q}{\mathds{Q}}

\renewcommand{\O}{\emptyset}

\newcommand\ubar[1]{\underaccent{\bar}{#1}}

\renewcommand\Re{\expandafter\mathfrak{Re}}
\renewcommand\Im{\expandafter\mathfrak{Im}}

\let\slantedpartial\partial
\DeclareRobustCommand{\partial}{\text{\rotatebox[origin=t]{20}{\scalebox{0.95}[1]{$\slantedpartial$}}}\hspace{-1pt}}

% merging two maths characters w/ \charfusion
\makeatletter
\def\moverlay{\mathpalette\mov@rlay}
\def\mov@rlay#1#2{\leavevmode\vtop{%
   \baselineskip\z@skip \lineskiplimit-\maxdimen
   \ialign{\hfil$\m@th#1##$\hfil\cr#2\crcr}}}
\newcommand{\charfusion}[3][\mathord]{
    #1{\ifx#1\mathop\vphantom{#2}\fi
        \mathpalette\mov@rlay{#2\cr#3}
      }
    \ifx#1\mathop\expandafter\displaylimits\fi}
\makeatother

% custom math commands
\newcommand{\T}{{\!\!\,\top}}
\newcommand{\avrt}[1]{\rotatebox{-90}{$#1$}}
\newcommand{\bigcupdot}{\charfusion[\mathop]{\bigcup}{\cdot}}
\newcommand{\cupdot}{\charfusion[\mathbin]{\cup}{\cdot}}
%\newcommand{\danger}{{\large\fontencoding{U}\fontfamily{futs}\selectfont\char 66\relax}\;}
\newcommand{\tendsto}[1]{\xrightarrow[#1]{}}
\newcommand{\vrt}[1]{\rotatebox{90}{$#1$}}
\newcommand{\tsup}[1]{\textsuperscript{\underline{#1}}}
\newcommand{\tsub}[1]{\textsubscript{#1}}

\renewcommand{\mod}[1]{~\left[ #1 \right]}
\renewcommand{\t}{{}^t\!}
\newcommand{\s}{\text{\calligra s}}

% custom units / constants
%\DeclareSIUnit{\litre}{\ell}
\let\hbar\hslash

% header / footer
\pagestyle{fancy}
\fancyhead{} \fancyfoot{}
\fancyfoot[C]{\thepage}

% fonts
\let\sc\scshape
\let\bf\bfseries
\let\it\itshape
\let\sl\slshape

% custom math operators
\let\th\relax
\let\det\relax
\DeclareMathOperator*{\codim}{codim}
\DeclareMathOperator*{\dom}{dom}
\DeclareMathOperator*{\gO}{O}
\DeclareMathOperator*{\po}{\text{\cursive o}}
\DeclareMathOperator*{\sgn}{sgn}
\DeclareMathOperator*{\simi}{\sim}
\DeclareMathOperator{\Arccos}{Arccos}
\DeclareMathOperator{\Arcsin}{Arcsin}
\DeclareMathOperator{\Arctan}{Arctan}
\DeclareMathOperator{\Argsh}{Argsh}
\DeclareMathOperator{\Arg}{Arg}
\DeclareMathOperator{\Aut}{Aut}
\DeclareMathOperator{\Card}{Card}
\DeclareMathOperator{\Cl}{\mathcal{C}\!\ell}
\DeclareMathOperator{\Cov}{Cov}
\DeclareMathOperator{\Ker}{Ker}
\DeclareMathOperator{\Mat}{Mat}
\DeclareMathOperator{\PGCD}{PGCD}
\DeclareMathOperator{\PPCM}{PPCM}
\DeclareMathOperator{\Supp}{Supp}
\DeclareMathOperator{\Vect}{Vect}
\DeclareMathOperator{\argmax}{argmax}
\DeclareMathOperator{\argmin}{argmin}
\DeclareMathOperator{\ch}{ch}
\DeclareMathOperator{\com}{com}
\DeclareMathOperator{\cotan}{cotan}
\DeclareMathOperator{\det}{det}
\DeclareMathOperator{\id}{id}
\DeclareMathOperator{\rg}{rg}
\DeclareMathOperator{\rk}{rk}
\DeclareMathOperator{\sh}{sh}
\DeclareMathOperator{\th}{th}
\DeclareMathOperator{\tr}{tr}

% colors and page style
\definecolor{truewhite}{HTML}{ffffff}
\definecolor{white}{HTML}{faf4ed}
\definecolor{trueblack}{HTML}{000000}
\definecolor{black}{HTML}{575279}
\definecolor{mauve}{HTML}{907aa9}
\definecolor{blue}{HTML}{286983}
\definecolor{red}{HTML}{d7827e}
\definecolor{yellow}{HTML}{ea9d34}
\definecolor{gray}{HTML}{9893a5}
\definecolor{grey}{HTML}{9893a5}
\definecolor{green}{HTML}{a0d971}

\pagecolor{white}
\color{black}

\begin{asydef}
	settings.prc = false;
	settings.render=0;

	white = rgb("faf4ed");
	black = rgb("575279");
	blue = rgb("286983");
	red = rgb("d7827e");
	yellow = rgb("f6c177");
	orange = rgb("ea9d34");
	gray = rgb("9893a5");
	grey = rgb("9893a5");
	deepcyan = rgb("56949f");
	pink = rgb("b4637a");
	magenta = rgb("eb6f92");
	green = rgb("a0d971");
	purple = rgb("907aa9");

	defaultpen(black + fontsize(8pt));

	import three;
	currentlight = nolight;
\end{asydef}

% theorems, proofs, ...

\mdfsetup{skipabove=1em,skipbelow=1em, innertopmargin=6pt, innerbottommargin=6pt,}

\declaretheoremstyle[
	headfont=\normalfont\itshape,
	numbered=no,
	postheadspace=\newline,
	headpunct={:},
	qed=\qedsymbol]{demstyle}

\declaretheorem[style=demstyle, name=Démonstration]{dem}

\newcommand\veczero{\kern-1.2pt\vec{\kern1.2pt 0}} % \vec{0} looks weird since the `0' isn't italicized

\makeatletter
\renewcommand{\title}[2]{
	\AtBeginDocument{
		\begin{titlepage}
			\begin{center}
				\vspace{10cm}
				{\Large \sc Chapitre #1}\\
				\vspace{1cm}
				{\Huge \calligra #2}\\
				\vfill
				Hugo {\sc Salou} MPI${}^{\star}$\\
				{\small Dernière mise à jour le \@date }
			\end{center}
		\end{titlepage}
	}
}

\newcommand{\titletp}[4]{
	\AtBeginDocument{
		\begin{titlepage}
			\begin{center}
				\vspace{10cm}
				{\Large \sc tp #1}\\
				\vspace{1cm}
				{\Huge \textsc{\textit{#2}}}\\
				\vfill
				{#3}\textit{MPI}${}^{\star}$\\
			\end{center}
		\end{titlepage}
	}
	\fancyfoot{}\fancyhead{}
	\fancyfoot[R]{#4 \textit{MPI}${}^{\star}$}
	\fancyhead[C]{{\sc tp #1} : #2}
	\fancyhead[R]{\thepage}
}

\newcommand{\titletd}[2]{
	\AtBeginDocument{
		\begin{titlepage}
			\begin{center}
				\vspace{10cm}
				{\Large \sc td #1}\\
				\vspace{1cm}
				{\Huge \calligra #2}\\
				\vfill
				Hugo {\sc Salou} MPI${}^{\star}$\\
				{\small Dernière mise à jour le \@date }
			\end{center}
		\end{titlepage}
	}
}
\makeatother

\newcommand{\sign}{
	\null
	\vfill
	\begin{center}
		{
			\fontfamily{ccr}\selectfont
			\textit{\textbf{\.{\"i}}}
		}
	\end{center}
	\vfill
	\null
}

\renewcommand{\thefootnote}{\emph{\alph{footnote}}}

% figure support
\usepackage{import}
\usepackage{xifthen}
\pdfminorversion=7
\usepackage{pdfpages}
\usepackage{transparent}
\newcommand{\incfig}[1]{%
	\def\svgwidth{\columnwidth}
	\import{./figures/}{#1.pdf_tex}
}

\pdfsuppresswarningpagegroup=1
\ctikzset{tripoles/european not symbol=circle}

\newcommand{\missingpart}{{\large\color{red} Il manque quelque chose ici\ldots}}

\usepackage{caption}
\usepackage{subcaption}
\usepackage{comment}
\usepackage{pgfornament}
\usepackage{graphicx}

\definecolor{black}{HTML}{000000}
\definecolor{white}{HTML}{ffffff}
\color{black}
\pagecolor{white}

\begin{asydef}
	white = rgb("ffffff");
	black = rgb("000000");
	defaultpen(black + fontsize(8pt));
\end{asydef}

\titletp{m\:1}{Mesure des coefficients de frottement }{\begin{tabular}{c}Hugo \textsc{Salou}\\Noémie \textsc{Combey}\end{tabular}}{Hugo \textsc{Salou} \& Noémie \textsc{Combey}}

\newcommand{\red}[1]{{\color{red}#1}}
\newcommand{\green}[1]{{\color{green}#1}}
\def\thesection{\Roman{section}.}

\begin{document}
	L'objectif de ce \textsc{tp} est la mesure, avec incertitudes, des deux coefficients de frottement : statique ($f_\mathrm{s}$) et dynamique ($f_\mathrm{d}$).

	\section{Mesure du coefficient de frottement statique $f_\mathrm{s}$}

	Pour mesurer la valeur du coefficient de frottement statique, on dispose une planche en bois à un angle $\alpha$\/ de la paillasse.
	Initialement, on choisit $\alpha = 0$ : la planche est sur la paillasse. Puis, on pose un bloc de bois sur cette planche, sans lui donner de vitesse initiale.
	On augment progressivement l'angle $\alpha$, jusqu'à ce que le bloc ne commence à glisser.
	On note la valeur de l'angle $\alpha$\/ dans cette condition, on notera cet angle $\alpha_{\lim}$.
	Le montage est représenté dans la figure 1.

	\begin{figure}[H]
		\centering
		\incfig{fig1}
		\caption{Protocole 1, mesure de $f_\mathrm{s}$\/}
	\end{figure}

	On ne mesure pas l'angle $\alpha$\/ directement, on mesure plutôt la hauteur $h$\/ de la planche en un point fixe, et on mesure la distance $d$\/ entre ce point et le bord en contact avec la paillasse.
	En choisissant le même point pour chacune des mesures, la distance $d$\/ est constante, on ne mesure que $h$. De plus, on a l'égalité \[
		\alpha = \Arcsin\! \left( \frac{h}{d} \right)\!
	.\]
	
	Ainsi, on suit le protocole décrit précédemment, et on réalise les mesures des données représentées dans la table 1.

	\begin{table}[H]
		\centering
		\begin{tabular}{|c|c|}
			\hline
			$h$\/ ($\mathrm{mm}$) & $\alpha_{\lim}$\/ ($^\circ$)\\ \hline \hline
			$160$&$20{,}35$ \\ \hline
			$178$&$22{,}77$\\ \hline
			$168$&$21{,}42$\\ \hline
			$168$&$21{,}42$\\ \hline
			$150$&$19{,}031$\\ \hline
			$164$&$20{,}89$\\ \hline
			$162$&$20{,}62$\\ \hline
			$143$&$18{,}11$\\ \hline
			$175$&$22{,}36$\\ \hline
			$158$&$20{,}09$\\ \hline
			$176$&$22{,}50$\\ \hline
			$188$&$24{,}12$\\ \hline
		\end{tabular}
		\caption{Mesure de l'angle $\alpha$}
	\end{table}

	Les mesures sont réalisés avec des blocs de différentes masses, et différentes surfaces.
	Ainsi, on remarque que l'angle $\alpha_{\lim}$\/ (et donc $f_\mathrm{s}$) dépend de ces deux facteurs, même si les lois de \textsc{Coulomb} affirment que $f_\mathrm{s}$\/ ne dépend que des matériaux en contact.

	On estime l'incertitude de mesure sur la distance $d$\/ à $u(d) = 10\:\mathrm{mm}/\!\sqrt{3} = 6\:\mathrm{mm}$. La mesure de la distance $d$\/ est de \[
		{\color{cyan} d = 830 \pm 6\: \mathrm{mm}}
	.\] De plus, par analyse statistique des données de la table précédente, on trouve que \[
		{\color{cyan} h = 166 \pm 10\: \mathrm{mm}}
	.\]
	Par composition des incertitudes, on trouve que
	\[
		u\!\left( \frac{h}{d} \right) = \frac{1}{d} \cdot \sqrt{u^2(h) + u^2(d) \cdot \left( \frac{h}{d} \right)^{\!\!2}} \mathrel{\overset{\text{(AN)}}=} 0{,}01
	.\] Ainsi, toujours par composition des incertitudes, on a
	\begin{align*}
		u(\alpha) &= \Arcsin'(h / d) \cdot u(h / d)\\
		&= \frac{u(h / d)}{\sqrt{1 - \left( \frac{h}{d} \right)^2}} \\
		{}_\text{(AN)}&= 0{,}01\: \mathrm{rad} \\
		&= 0{,}7\:^\circ. \\
	\end{align*}
	On en déduit donc la mesure de l'angle limite $\alpha$\/ : \[
		{\color{cyan} \alpha_{\lim} = 21{,}1 \pm 0{,}7\:^\circ}
	.\]
	Et, d'après le cours, on sait calculer $f_\mathrm{s}$\/ avec la valeur de cet angle : $f_\mathrm{s} = \tan(\alpha_{\lim}) \simeq 0{,}38$.
	Or, par composition des incertitudes, on a
	\begin{align*}
		u(f_\mathrm{s}) &= (1 + \tan^2 \alpha_{\lim}) \cdot u(\alpha_{\lim})\\
		&= (1 + {f_\mathrm{s}}^2) \cdot u(\alpha_{\lim})  \\
		_\text{(AN)} &= 0{,}01 \\
	\end{align*}
	On en déduit donc la mesure du coefficient de frottement statique : \[
		{\color{cyan}f_\mathrm{s} = 0{,}38 \pm 0{,}01}
	.\]

	\section{Mesure du coefficient de frottement dynamique $f_\mathrm{d}$}

	Pour mesurer la valeur de $f_\mathrm{d}$, on utilise le protocole suivant. On place deux points sur la planche, un point d'arrivé, et un point de départ. On positionne la planche à un angle $\alpha$\/ constant, suffisamment proche de $\alpha_{\lim}$\/ pour qu'un bloc ne tombe pas trop vite, mais suffisamment loin de $\alpha_{\lim}$\/ pour qu'un bloc ne s'arrête pas entre le départ et l'arrivée. On mesure le temps $\mathrm{\Delta}t$\/ nécessaire pour que le bloc passe du point de départ au point d'arrivée.
	Ce protocole est représenté dans la figure 2.

	\begin{figure}[H]
		\centering
		\incfig{fig2}
		\caption{Protocole 2, mesure de $f_\mathrm{d}$\/}
	\end{figure}

	On commence par chercher une relation mathématique entre la durée $\mathrm{\Delta}t$, et le coefficient de frottement $f_\mathrm{d}$. Les forces s'appliquant au système \{\:bloc\:\} sont, son poids $\vec{P}$, et la réaction du support $\vec{R}$. On se place dans le référentiel terrestre $\mathcal{R}_\mathrm{t}$\/ supposé galiléen, et on utilise des coordonnées cartésiennes, avec l'axe $(Ox)$\/ suivant la planche, pointant vers la droite, et l'axe $(Oy)$\/ pointant vers le haut, perpendiculairement à la planche.
	Ainsi, d'après le principe fondamental de la dynamique, \[
		m \cdot \frac{\mathrm{d}^2 \vv{OM}}{\mathrm{d}t^2} = \vec{R} + \vec{P}
	.\]
	Dans ces coordonnées, on a $\vec{P} = mg\sin \alpha \:\vec{e}_x - mg \cos \alpha \:\vec{e}_y$. On décompose $\vec{R}$\/ en $\vec{R} = N\: \vec{e}_y - T\: \vec{e}_x$.
	Ainsi, en projetant selon $\vec{e}_x$\/ puis $\vec{e}_y$, le \textsc{pfd} devient \[
		\begin{cases}
			-m\ddot{x} = T - mg\sin \alpha \quad\quad &(1) \\
			\mathord{\phantom{-}}m\ddot{y} = N -mg\cos \alpha.\quad\quad&(2)
		\end{cases}
	\]
	Or, le bloc reste sur la planche, donc $y = 0$, et donc $\ddot{y} = 0$. Ainsi, $N = mg \cos \alpha$, et donc $T = f_\mathrm{d}\,mg \cos \alpha$, d'après les lois de \textsc{Coulomb}.
	On obtient donc \[
		\ddot{x} = g (\sin \alpha - f_\mathrm{d} \cos \alpha)
	.\] D'où, \[
		\dot{x} = t \cdot g(\sin \alpha - f_\mathrm{d} \cos \alpha) + v_0,
	\] et donc \[
		\mathrm{\Delta}x = \frac{1}{2}(\mathrm{\Delta}t)^2 \cdot g(\sin \alpha - f_\mathrm{d} \cos \alpha) + v_0 (\mathrm{\Delta}t)
	.\]
	On suppose $v_0 = 0\:\mathrm{m}/\mathrm{s}$, et ainsi \[
		f_\mathrm{d} = -\frac{2 \mathrm{\Delta}x}{(\mathrm{\Delta}t)^2 \cdot g \cdot \cos \alpha} + \tan \alpha
	.\]

	On réalise à présent les mesures à l'aide du protocole décrit précédemment. On rassemble ces mesures dans la table 2.

	\begin{table}[H]
		\centering
		\begin{tabular}{|c|c|}
			\hline & $\mathrm{\Delta}t$ ($\mathrm{s}$)\\\hline
			1&$4{,}063$ \\\hline
			2&$3{,}011$ \\\hline
			3&$3{,}025$ \\\hline
			4&$3{,}062$ \\\hline
			5&$4{,}036$ \\\hline
		\end{tabular}
		\begin{tabular}{|c|c|}
			\hline & $\mathrm{\Delta}t$ ($\mathrm{s}$)\\\hline
			6&$2{,}044$ \\\hline
			7&$2{,}078$ \\\hline
			8&$2{,}075$ \\\hline
			9&$5{,}028$ \\\hline
			10&$7{,}056$ \\\hline
		\end{tabular}
		\caption{Mesures de $\mathrm{\Delta}t$, pour des valeurs de $\mathrm{\Delta}x$\/ et $\alpha$\/ fixés.}
	\end{table}

	Comme pour la mesure précédente, les mesures de la table 2 sont réalisées avec des blocs de masse et surface différente. Ainsi, ces facteurs influent sur la valeur de $\mathrm{\Delta}t$, et donc sur la valeur de $f_\mathrm{d}$. Mais, les lois de \textsc{Coulomb} affirment que $f_\mathrm{d}$\/ ne dépendent pas de la masse, ou de la surface de contact mais des matériaux en contact.
	Ainsi, on choisit la valeur moyenne de $\mathrm{\Delta}t$\/ : \[
		{\color{cyan} \mathrm{\Delta}t = 3{,}5478\:\mathrm{s}}
	.\]
	Également, $\mathrm{\Delta}x = 0{,}602\:\mathrm{m}$, et $\alpha = 0{,}2426\:\mathrm{rad}$. On en déduit que \[
		{\color{cyan} f_\mathrm{d} = 0{,}24}
	.\]

	\section{Conclusion}

	D'après le cours, les valeurs de $f_\mathrm{s}$\/ et $f_\mathrm{d}$\/ pour le frottement bois--bois sont \[
		f_\mathrm{s} \simeq 0{,}5 \quad\text{ et }\quad f_\mathrm{d} \simeq 0{,}3
	.\] Les valeurs trouvées précédemment ($f_\mathrm{s} = 0{,}38$\/ et $f_\mathrm{d} = 0{,}24$), sont proches.
	On a pu remarquer que les lois de \textsc{Coulomb} ne sont que des approximations, qui ne prennent pas en compte la masse ou la surface en contact, même si ces facteurs ont un effet visible.

	\sign

	~
\end{document}
