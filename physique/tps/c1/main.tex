\documentclass[a4paper,twocolumn,10pt,margin=0.5in]{extreport}

%\usepackage{concmath}
\usepackage{tgschola}
\usepackage{pstricks}
\usepackage[margin=1in]{geometry}
\usepackage[utf8]{inputenc}
\usepackage[T1]{fontenc}
\usepackage{mathrsfs}
\usepackage{textcomp}
\usepackage[french]{babel}
\usepackage{amsmath}
\usepackage{amssymb}
\usepackage{cancel}
\usepackage{frcursive}
\usepackage[inline]{asymptote}
\usepackage{tikz}
\usepackage[european,straightvoltages,europeanresistors]{circuitikz}
\usepackage{tikz-cd}
\usepackage{tkz-tab}
\usepackage[b]{esvect}
\usepackage[framemethod=TikZ]{mdframed}
\usepackage{centernot}
\usepackage{diagbox}
\usepackage{dsfont}
\usepackage{fancyhdr}
\usepackage{float}
\usepackage{graphicx}
\usepackage{listings}
\usepackage{multicol}
\usepackage{nicematrix}
\usepackage{pdflscape}
\usepackage{stmaryrd}
\usepackage{xfrac}
\usepackage{hep-math-font}
\usepackage{amsthm}
\usepackage{thmtools}
\usepackage{indentfirst}
\usepackage[framemethod=TikZ]{mdframed}
\usepackage{accents}
\usepackage{soulutf8}
\usepackage{mathtools}
\usepackage{bodegraph}
\usepackage{slashbox}
\usepackage{enumitem}
\usepackage{calligra}
\usepackage{cinzel}
\usepackage{BOONDOX-calo}

% Tikz
\usetikzlibrary{babel}
\usetikzlibrary{positioning}
\usetikzlibrary{calc}

% global settings
\frenchspacing
\reversemarginpar
\setuldepth{a}

%\everymath{\displaystyle}

\frenchbsetup{StandardLists=true}

\def\asydir{asy}

%\sisetup{exponent-product=\cdot,output-decimal-marker={,},separate-uncertainty,range-phrase=\;à\;,locale=FR}

\setlength{\parskip}{1em}

\theoremstyle{definition}

% Changing math
\let\emptyset\varnothing
\let\ge\geqslant
\let\le\leqslant
\let\preceq\preccurlyeq
\let\succeq\succcurlyeq
\let\ds\displaystyle
\let\ts\textstyle

\newcommand{\C}{\mathds{C}}
\newcommand{\R}{\mathds{R}}
\newcommand{\Z}{\mathds{Z}}
\newcommand{\N}{\mathds{N}}
\newcommand{\Q}{\mathds{Q}}

\renewcommand{\O}{\emptyset}

\newcommand\ubar[1]{\underaccent{\bar}{#1}}

\renewcommand\Re{\expandafter\mathfrak{Re}}
\renewcommand\Im{\expandafter\mathfrak{Im}}

\let\slantedpartial\partial
\DeclareRobustCommand{\partial}{\text{\rotatebox[origin=t]{20}{\scalebox{0.95}[1]{$\slantedpartial$}}}\hspace{-1pt}}

% merging two maths characters w/ \charfusion
\makeatletter
\def\moverlay{\mathpalette\mov@rlay}
\def\mov@rlay#1#2{\leavevmode\vtop{%
   \baselineskip\z@skip \lineskiplimit-\maxdimen
   \ialign{\hfil$\m@th#1##$\hfil\cr#2\crcr}}}
\newcommand{\charfusion}[3][\mathord]{
    #1{\ifx#1\mathop\vphantom{#2}\fi
        \mathpalette\mov@rlay{#2\cr#3}
      }
    \ifx#1\mathop\expandafter\displaylimits\fi}
\makeatother

% custom math commands
\newcommand{\T}{{\!\!\,\top}}
\newcommand{\avrt}[1]{\rotatebox{-90}{$#1$}}
\newcommand{\bigcupdot}{\charfusion[\mathop]{\bigcup}{\cdot}}
\newcommand{\cupdot}{\charfusion[\mathbin]{\cup}{\cdot}}
%\newcommand{\danger}{{\large\fontencoding{U}\fontfamily{futs}\selectfont\char 66\relax}\;}
\newcommand{\tendsto}[1]{\xrightarrow[#1]{}}
\newcommand{\vrt}[1]{\rotatebox{90}{$#1$}}
\newcommand{\tsup}[1]{\textsuperscript{\underline{#1}}}
\newcommand{\tsub}[1]{\textsubscript{#1}}

\renewcommand{\mod}[1]{~\left[ #1 \right]}
\renewcommand{\t}{{}^t\!}
\newcommand{\s}{\text{\calligra s}}

% custom units / constants
%\DeclareSIUnit{\litre}{\ell}
\let\hbar\hslash

% header / footer
\pagestyle{fancy}
\fancyhead{} \fancyfoot{}
\fancyfoot[C]{\thepage}

% fonts
\let\sc\scshape
\let\bf\bfseries
\let\it\itshape
\let\sl\slshape

% custom math operators
\let\th\relax
\let\det\relax
\DeclareMathOperator*{\codim}{codim}
\DeclareMathOperator*{\dom}{dom}
\DeclareMathOperator*{\gO}{O}
\DeclareMathOperator*{\po}{\text{\cursive o}}
\DeclareMathOperator*{\sgn}{sgn}
\DeclareMathOperator*{\simi}{\sim}
\DeclareMathOperator{\Arccos}{Arccos}
\DeclareMathOperator{\Arcsin}{Arcsin}
\DeclareMathOperator{\Arctan}{Arctan}
\DeclareMathOperator{\Argsh}{Argsh}
\DeclareMathOperator{\Arg}{Arg}
\DeclareMathOperator{\Aut}{Aut}
\DeclareMathOperator{\Card}{Card}
\DeclareMathOperator{\Cl}{\mathcal{C}\!\ell}
\DeclareMathOperator{\Cov}{Cov}
\DeclareMathOperator{\Ker}{Ker}
\DeclareMathOperator{\Mat}{Mat}
\DeclareMathOperator{\PGCD}{PGCD}
\DeclareMathOperator{\PPCM}{PPCM}
\DeclareMathOperator{\Supp}{Supp}
\DeclareMathOperator{\Vect}{Vect}
\DeclareMathOperator{\argmax}{argmax}
\DeclareMathOperator{\argmin}{argmin}
\DeclareMathOperator{\ch}{ch}
\DeclareMathOperator{\com}{com}
\DeclareMathOperator{\cotan}{cotan}
\DeclareMathOperator{\det}{det}
\DeclareMathOperator{\id}{id}
\DeclareMathOperator{\rg}{rg}
\DeclareMathOperator{\rk}{rk}
\DeclareMathOperator{\sh}{sh}
\DeclareMathOperator{\th}{th}
\DeclareMathOperator{\tr}{tr}

% colors and page style
\definecolor{truewhite}{HTML}{ffffff}
\definecolor{white}{HTML}{faf4ed}
\definecolor{trueblack}{HTML}{000000}
\definecolor{black}{HTML}{575279}
\definecolor{mauve}{HTML}{907aa9}
\definecolor{blue}{HTML}{286983}
\definecolor{red}{HTML}{d7827e}
\definecolor{yellow}{HTML}{ea9d34}
\definecolor{gray}{HTML}{9893a5}
\definecolor{grey}{HTML}{9893a5}
\definecolor{green}{HTML}{a0d971}

\pagecolor{white}
\color{black}

\begin{asydef}
	settings.prc = false;
	settings.render=0;

	white = rgb("faf4ed");
	black = rgb("575279");
	blue = rgb("286983");
	red = rgb("d7827e");
	yellow = rgb("f6c177");
	orange = rgb("ea9d34");
	gray = rgb("9893a5");
	grey = rgb("9893a5");
	deepcyan = rgb("56949f");
	pink = rgb("b4637a");
	magenta = rgb("eb6f92");
	green = rgb("a0d971");
	purple = rgb("907aa9");

	defaultpen(black + fontsize(8pt));

	import three;
	currentlight = nolight;
\end{asydef}

% theorems, proofs, ...

\mdfsetup{skipabove=1em,skipbelow=1em, innertopmargin=6pt, innerbottommargin=6pt,}

\declaretheoremstyle[
	headfont=\normalfont\itshape,
	numbered=no,
	postheadspace=\newline,
	headpunct={:},
	qed=\qedsymbol]{demstyle}

\declaretheorem[style=demstyle, name=Démonstration]{dem}

\newcommand\veczero{\kern-1.2pt\vec{\kern1.2pt 0}} % \vec{0} looks weird since the `0' isn't italicized

\makeatletter
\renewcommand{\title}[2]{
	\AtBeginDocument{
		\begin{titlepage}
			\begin{center}
				\vspace{10cm}
				{\Large \sc Chapitre #1}\\
				\vspace{1cm}
				{\Huge \calligra #2}\\
				\vfill
				Hugo {\sc Salou} MPI${}^{\star}$\\
				{\small Dernière mise à jour le \@date }
			\end{center}
		\end{titlepage}
	}
}

\newcommand{\titletp}[4]{
	\AtBeginDocument{
		\begin{titlepage}
			\begin{center}
				\vspace{10cm}
				{\Large \sc tp #1}\\
				\vspace{1cm}
				{\Huge \textsc{\textit{#2}}}\\
				\vfill
				{#3}\textit{MPI}${}^{\star}$\\
			\end{center}
		\end{titlepage}
	}
	\fancyfoot{}\fancyhead{}
	\fancyfoot[R]{#4 \textit{MPI}${}^{\star}$}
	\fancyhead[C]{{\sc tp #1} : #2}
	\fancyhead[R]{\thepage}
}

\newcommand{\titletd}[2]{
	\AtBeginDocument{
		\begin{titlepage}
			\begin{center}
				\vspace{10cm}
				{\Large \sc td #1}\\
				\vspace{1cm}
				{\Huge \calligra #2}\\
				\vfill
				Hugo {\sc Salou} MPI${}^{\star}$\\
				{\small Dernière mise à jour le \@date }
			\end{center}
		\end{titlepage}
	}
}
\makeatother

\newcommand{\sign}{
	\null
	\vfill
	\begin{center}
		{
			\fontfamily{ccr}\selectfont
			\textit{\textbf{\.{\"i}}}
		}
	\end{center}
	\vfill
	\null
}

\renewcommand{\thefootnote}{\emph{\alph{footnote}}}

% figure support
\usepackage{import}
\usepackage{xifthen}
\pdfminorversion=7
\usepackage{pdfpages}
\usepackage{transparent}
\newcommand{\incfig}[1]{%
	\def\svgwidth{\columnwidth}
	\import{./figures/}{#1.pdf_tex}
}

\pdfsuppresswarningpagegroup=1
\ctikzset{tripoles/european not symbol=circle}

\newcommand{\missingpart}{{\large\color{red} Il manque quelque chose ici\ldots}}

\usepackage{caption}
\usepackage{subcaption}
\usepackage{comment}
\usepackage{pgfornament}
\usepackage{graphicx}
\usepackage{pst-labo}
\usepackage{pgfplots}
\usepackage{multirow}
\usepackage{rotating}

%\begin{comment}

\definecolor{black}{HTML}{000000}
\definecolor{white}{HTML}{ffffff}
\color{black}
\pagecolor{white}

\begin{asydef}
	white = rgb("ffffff");
	black = rgb("000000");
	defaultpen(black + fontsize(8pt));
\end{asydef}
%\end{comment}

\definecolor{BleuClair}{HTML}{286983}
\definecolor{LightBlue}{cmyk}{0.22,0.05,0,0.1}
\definecolor{GrisClair}{HTML}{9893a5}
\def\res#1{{\color{black}#1}}

\makeatletter
\def\pst@dosage@pHmetre{%
% boitier
\psframe[framearc=0.25](-1,0)(1,3)%
\psframe[fillcolor=GrisClair,fillstyle=solid](-0.75,2)(0.75,2.75)%
\rput(0,2.375){\white\footnotesize$\mathrm{pH}$}%
\pscircle(0,1.25){0.3}%
\psline{->}(0,1.25)(0,1.5)%
%\rput(0.65,1.5){$^\circ$C}%
\pscircle(-0.5,0.5){0.2}%
\pscircle(0.5,0.5){0.2}%
% électrode
\psframe[framearc=0.5](2.25,2.5)(2.75,4.5)%
\psframe[fillstyle=solid,fillcolor=GrisClair](2.25,4)(2.75,4.5)%
\pscircle(2.5,2.25){0.25}
% fil
\psline[linearc=0.4,linewidth=0.1,linecolor=LightBlue]%
(2.5,4.5)(2.5,5)(1.25,4.75)(0,5)(0,3)}
\makeatother

\titletp{c\:1}{Dosages acido-basiques}{\begin{tabular}{c}Hugo \textsc{Salou}\\Noémie \textsc{Combey}\end{tabular}}{Hugo \textsc{Salou} \& Noémie \textsc{Combey}}

\def\thesection{\Roman{section}.}
\def\thesubsection{\Roman{section}.\Alph{subsection}.}

\begin{document}
	L'objectif de ce \textsc{tp} est de réaliser des titrages acido-basique, par colorimétrie et suivi $\mathrm{pH}$-métrique. Ces dosages permettent de déterminer la concentration d'acide acétique dans un vinaigre, et déterminer s'il a été frelaté.

	\section{Étude théorique}

	Lors de l'ajout de soude dans un vinaigre frelaté avec de l'acide chlorhydrique, deux réactions peuvent se produire :
	\begin{align}
		&\mathrm{CH_3COOH + HO^- = CH_3COO^- + H_2O},\\
		&\mathrm{H_3O^+ + HO^- = H_2O + H_2O}.
	\end{align}
	Mais, la réaction (2) se réalisera avant la réaction (1) : le couple acido-basique $\mathrm{H_3O^+/H_2O}$\/ possède un $\mathrm{p}K_\mathrm{a}$\/ plus important que celui du couple $\mathrm{CH_3COOH/CH_3COO^-}$, comme le montre la figure ci-dessous.
	\begin{figure}[H]
		\centering
		\begin{tikzpicture}
			\draw[->,thick] (0,0) -- (0, 4);
			\draw (-0.1,0)--(0.1,0);
			\node[left] at (0,0) {$\mathrm{H_2O}$};
			\node[right] at (0,0) {$\mathrm{HO^-}$};
			\node[right] at (2, 0) {$0$};
			\draw (-0.1,3.5)--(0.1,3.5);
			\node[left] at (0,3.5) {$\mathrm{H_3O^+}$};
			\node[right] at (0,3.5) {$\mathrm{H_2O}$};
			\node[right] at (2, 3.5) {$14$};
			\draw (-0.1,1.2)--(0.1,1.2);
			\node[left] at (0,1.2) {$\mathrm{CH_3COOH}$};
			\node[right] at (0,1.2) {$\mathrm{CH_3COO^-}$};
			\node[right] at (2, 1.2) {$4{,}8$};
			\node at (0, 4.2) {$\mathrm{p}K_\mathrm{a}$};
		\end{tikzpicture}
		\caption{$\mathrm{p}K_\mathrm{a}$\/ des différents couples acido-basiques $\mathrm{H_3O^+ / H_2O}$, $\mathrm{CH_3 COOH}$\/ et $\mathrm{H_2O / HO^-}$.}
	\end{figure}
	\noindent On dosera donc en premier l'acide chlorhydrique.
	Le suivi $\mathrm{pH}$-métrique est donc bien adapté : l'acide chlorhydrique réagissant avec l'eau fait diminuer le $\mathrm{pH}$.
	Mais, un titrage calorimétrique est également possible en choisissant un indicateur coloré adapté, afin de voir le changement du $\mathrm{pH}$ lors de l'ajout de soude.
	Dans cette situation, on cherche à détecter le moment où les réactions (1) et (2) ne se produisent plus. Autrement dit, on cherche l'instant où la solution titrée devient basique.
	Par exemple, la \textit{phénolphtaléine} a pour zone de virage $8{,}2$--$10$, ce qui en fait un bon indicateur pour cette réaction.
	Le \textit{jaune d’alizarine jaune} peut également fonctionner, mais il faudra changer davantage le $\mathrm{pH}$\/ avant qu'il s'active.
	Le titrage du vinaigre se réalisera selon le protocole suivant.
	\begin{itemize}
		\item On dilue d'un facteur 10 la solution de vinaigre. À l'aide d'une pipette jaugée, on mesure $5\:\mathrm{mL}$\/ de vinaigre, que l'on place dans une fiole jaugée de $50\:\mathrm{mL}$. On ajoute, jusqu'au trait de jauge, de l'eau distillée en secouant régulièrement.
		\item On réalise un titrage rapide, avec suivi colorimétrique. On place dans un bêcher $10\:\mathrm{mL}$ de la solution diluée, à laquelle on ajoute la phténolphtaléine. On remplit une burette graduée de $50\:\mathrm{mL}$\/ de solution de soude. Tout en agitant le contenu du bêcher, on verse progressivement la solution de soude dans le bêcher. Lors du changement de couleur, on note le volume versé.
		\item On réalise un titrage, avec suivi $\mathrm{pH}$-métrique.
			On prélève $10\:\mathrm{mL}$\/ de la solution diluée, que l'on place dans un bêcher.
			On ajoute $90\:\mathrm{mL}$ d'eau distillée afin que la sonde $\mathrm{pH}$-métrique soit bien immergée.
			On reremplit la burette avec la solution de soude.
			On place la sonde $\mathrm{pH}$-métrique (on étalonne préalablement le $\mathrm{pH}$-mètre). On verse progressivement la solution de soude, et on note le $\mathrm{pH}$\/ de la solution dans \textit{Regressi}. On augment le nombre de mesures près du volume trouvé par titrage rapide.
		\item On trace la courbe dérivée, et on determine l'abscisse du ``pic'' de la dérivée : le volume à l'équivalence $V_\mathrm{\acute{e}q}$.
	\end{itemize}


	\begin{figure}[H]
		\centering
			\begin{pspicture}(0,0)(1,4)
				\psset{unit=0.3cm,glassType=becher}
				\pstDosage[phmetre,couleurReactifBurette=red]
			\end{pspicture}
		\caption{Protocole de titrage $\mathrm{pH}$-métrique du vinaigre}
	\end{figure}

	\section{Étude expérimentale}
	\subsection{Titrage colorimétrique}

	On réalise le titrage rapide du vinaigre A, comme décrit dans le protocole ci-avant, et on trouve un volume à l'équivalence \[
		V_\mathrm{\acute{e}q}' \simeq 20\:\mathrm{mL}
	.\] À l'équivalence, l'égalité suivante est vérifiée : \[
		\underbrace{[\mathrm{HO^-}]}_{c_\mathrm{s}} \cdot V_\mathrm{\acute{e}q}'  = [\mathrm{acides}] \cdot V_0
	.\] On en déduit le nombre de moles d'acide $n$ pour $100\:\mathrm{g}$ de vinaigre : \[
		n = \frac{100\:\mathrm{g}}{m_0} \cdot V_0 \cdot [\mathrm{acides}] \cdot 10
	\] où $m_0$ est la masse de la solution. Le facteur $10$ apparaît due à la dilution au $1/10$\:\tsup{ème} du vinaigre.
	Or, $m_0 / V_0 = \rho_0$\/ la masse volumique de la solution. Et, la densité du vinaigre $\mathrm{A}$ est connu, d'où $\rho_0 = d_\mathrm{A} \cdot \rho_\mathrm{eau}$.
	Ainsi, on a \[
		n = \frac{100\:\mathrm{g}}{\rho_\mathrm{eau} \cdot d_\mathrm{A}} \cdot \frac{c_\mathrm{s} \cdot V_\mathrm{\acute{e}q}'}{V_0} \cdot 10
	.\] La masse de $n$ moles d'acides acétique est donnée par {\small\[
		m = n \cdot M_{\mathrm{CH_3COOH}} = n \cdot \big(2M_\mathrm{C} + 4M_\mathrm{H} + 2 M_\mathrm{O}\big)
	.\]}%
	On réalise l'application numérique, et on trouve $\res{m = 12\:\mathrm{g}}$.
	Le vinaigre {A} est donc à $12\:^\circ$.

	\subsection{Titrage par suivi $\mathrm{pH}$-métrique}

	On réalise le titrage $\mathrm{pH}$-métrique, comme décrit dans le protocole ci-avant.
	On représente sur la figure ci-dessous la courbe $\mathrm{pH}$ en fonction du volume $V_\mathrm{vers\acute{e}}$ versé.

	\begin{figure}[H]
		\centering
		\begin{tikzpicture}[scale=0.80]
			\begin{axis}[xlabel=$V_\mathrm{vers\acute{e}}$, axis lines = left, grid style=dashed, ymajorgrids=true, xmajorgrids=true,legend pos=north west]
				\addplot table [x=V, y=pH, col sep=comma] {data/vin_a.csv};
				\addlegendentry{$\mathrm{pH}$}
				\addplot[mark=none,color=red] table [x=V, y=d, col sep=comma] {data/vin_a.csv};
				\addlegendentry{$\frac{\mathrm{d}\,\mathrm{pH}}{\mathrm{d}V_\mathrm{vers\acute{e}}}$}
			\end{axis}
		\end{tikzpicture}
		\caption{Courbe de titrage du vinaigre A}
	\end{figure}

	\noindent Un autre groupe réalise un titrage sur le vinaigre B, la courbe obtenue est représentée sur la figure ci-dessous.

	\begin{figure}[H]
		\centering
		\begin{tikzpicture}[scale=0.80]
			\begin{axis}[xlabel=$V_\mathrm{vers\acute{e}}$, axis lines = left, grid style=dashed, ymajorgrids=true, xmajorgrids=true,legend pos=north west]
				\addplot table [x=V, y=pH, col sep=comma] {data/vin_b.csv};
				\addlegendentry{$\mathrm{pH}$}
				\addplot[mark=none,color=red] table [x=V, y=d, col sep=comma] {data/vin_b.csv};
				\addlegendentry{$\frac{\mathrm{d}\,\mathrm{pH}}{\mathrm{d}V_\mathrm{vers\acute{e}}}$}
			\end{axis}
		\end{tikzpicture}
		\caption{Courbe de titrage du vinaigre B}
	\end{figure}

	Comparons ces deux courbes. Pour le vinaigre A, on ne remarque qu'une seule équivalence. Cependant, pour le vinaigre B, il semble y en avoir deux : une pour l'acide acétique, vers $20{,}8\:\mathrm{mL}$, et une pour l'acide chlorhydrique, vers $9\:\mathrm{mL}$.
	On en conclut que le vinaigre B est frelaté.

	À l'équivalence $1$ (équivalence de l'acide acétique), l'égalité suivante est vérifiée : \[
		c_\mathrm{s} \cdot (V_\mathrm{\acute{e}q, 1} - V_\mathrm{\acute{e}q, 2}) = [\mathrm{CH_3 COOH}] \cdot V_0 \cdot 10
	,\]avec la dilution de facteur $10$. Après application numérique, on détermine la concentration en acide acétique
	\begin{align*}
		[\mathrm{CH_3COOH}] &= 10\cdot  c_\mathrm{s} \cdot \frac{V_\mathrm{\acute{e}q,1} - V_\mathrm{\acute{e}q,2}}{V_0}\\
		&= 1{,}18 \: \mathrm{mol}/\mathrm{L}. \\
	\end{align*}
	Et, à l'équivalence $2$ (équivalence de l'acide chlorhydrique), l'égalité suivante est vérifiée \[
		c_\mathrm{s} \cdot V_\mathrm{\acute{e}q,2} = [\mathrm{H_3O^+}] \cdot V_0 \cdot 10
	.\] On en déduit la concentration en acide chlorhydrique du vinaigre : \[
		[\mathrm{H_3O^+}] = 10c_\mathrm{s} \cdot \frac{V_\mathrm{\acute{e}q,2}}{V_0} = 0{,}90\:\mathrm{mol}/\mathrm{L}
	.\]
	De la figure \textsc{3}, on peut en déduire la valeur du $\mathrm{p}K_\mathrm{a}$\/ de l'acide acétique : c'est la valeur du $\mathrm{pH}$\/ pour $V_\mathrm{vers\acute{e}} = V_\mathrm{\acute{e}q,1} / 2$.
	Ainsi, on trouve \[
		\mathrm{p}K_\mathrm{a} = 4{,}7
	.\]

	\subsection{Incertitudes de mesure}

	On résume les grandeurs, et leurs sources d'erreurs et incertitudes associées, dans la table ci-après.
	On a \[
		c_\mathrm{a} = \frac{V_\mathrm{\acute{e}q}}{V_0} \cdot c_\mathrm{s} \cdot 10
	.\] 
	D'après la formule des incertitudes composées, on a
	\[
		\frac{u(c_\mathrm{a})}{c_\mathrm{a}} = \sqrt{\left( \frac{u(V_\mathrm{\acute{e}q})}{V_\mathrm{\acute{e}q}} \right)^{\!\!2} + \left( \frac{u(V_0)}{V_0} \right)^{\!\!2}}
	.\]
	Ainsi, pour le titrage colorimétrique, on trouve $u(c_\mathrm{a}) / c_\mathrm{a} = 0{,}03$.
	Et, pour le titrage $\mathrm{pH}$-métrique, on trouve $u(c_\mathrm{a}) / c_\mathrm{a} = 0{,}01$.
	On en déduit les résultats des mesurages :
	\begin{itemize}
		\item pour le titrage colorimétrique \[
				c_\mathrm{a} = 2{,}00 \pm 0{,}06 \: \mathrm{mol}/\mathrm{L}
			,\]
		\item pour le titrage $\mathrm{pH}$-métrique : \[
				c_\mathrm{a}  = 2{,}08 \pm 0{,}03\:\mathrm{mol}/\mathrm{L}
			.\]
	\end{itemize}
	Pour la suite, on utilise la mesure du titrage $\mathrm{pH}$-métrique pour déterminer le degré d'acidité.
	On a
	\begin{align*}
		m &= n \cdot M_\mathrm{CH_3 COOH} \\
		&= c_\mathrm{a} \cdot \underbrace{\frac{100\:\mathrm{g}}{\rho_\mathrm{eau} \cdot d_\mathrm{A}}}_{V_\mathrm{100\:g}} \cdot 60  \\
	\end{align*}
	Les incertitude relative associée à $m$ est $u(m) / m = u(c_\mathrm{a}) / c_\mathrm{a}$. Ainsi, on en déduit le degré d'acidité du vinaigre A : \[
		m = 12{,}2 \pm 0{,}1\: \mathrm{g}
	.\]

	\def\arraystretch{1.5}
	\begin{sidewaystable}
		\centering
		\begin{tabular}{r|c|c|l|c}
		\hfill\textsc{Grandeur}\hfill\null & \textsc{Valeur mesurée} & \textsc{Type} &\hfill \textsc{Source d'erreur} \hfill\null & \textsc{Incertitude} $u(x)$\\ \hline \hline
		Volume équivalent & \multirow{2}{*}{$V_\mathrm{\acute{e}q}'$} & \multirow{2}{*}B & $\bullet$\quad Jugement de l'expérimentateur & \multirow{2}{*}{$u(V_\mathrm{\acute{e}q}') = 0{,}6\:\mathrm{mL}$} \\[-2mm]
		colorimétrique & & & $\bullet$\quad Fréquence des mesures ($1\:\mathrm{mL}$)\hfill\null &  \\ \hline
		Volume équivalent & \multirow{2}{*}{$V_\mathrm{\acute{e}q,2}$} & \multirow{2}{*}A & $\bullet$\quad Précision du $\mathrm{pH}$-mètre & \multirow2*{$u(V_\mathrm{\acute{e}q,2}) = 0{,}2\:\mathrm{mL}$} \\[-2mm]
		avec suivi $\mathrm{pH}$-métrique & & & $\bullet$\quad Précision de la verrerie & \\ \hline
		Volume prélevé & $V_0$ & B & $\bullet$\quad Précision de la verrerie & $u(V_0) = 0{,}01\:\mathrm{mL}$ \\ \hline
		\end{tabular}
		\caption{Sources d'incertitudes sur les différentes grandeurs}
	\end{sidewaystable}
\end{document}
